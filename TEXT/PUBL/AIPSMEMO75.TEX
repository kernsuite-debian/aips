%-----------------------------------------------------------------------
%;  Copyright (C) 1995
%;  Associated Universities, Inc. Washington DC, USA.
%;
%;  This program is free software; you can redistribute it and/or
%;  modify it under the terms of the GNU General Public License as
%;  published by the Free Software Foundation; either version 2 of
%;  the License, or (at your option) any later version.
%;
%;  This program is distributed in the hope that it will be useful,
%;  but WITHOUT ANY WARRANTY; without even the implied warranty of
%;  MERCHANTABILITY or FITNESS FOR A PARTICULAR PURPOSE.  See the
%;  GNU General Public License for more details.
%;
%;  You should have received a copy of the GNU General Public
%;  License along with this program; if not, write to the Free
%;  Software Foundation, Inc., 675 Massachusetts Ave, Cambridge,
%;  MA 02139, USA.
%;
%;  Correspondence concerning AIPS should be addressed as follows:
%;          Internet email: aipsmail@nrao.edu.
%;          Postal address: AIPS Project Office
%;                          National Radio Astronomy Observatory
%;                          520 Edgemont Road
%;                          Charlottesville, VA 22903-2475 USA
%-----------------------------------------------------------------------
\documentstyle{article}
\title{15APR91 DDT results on a Sun IPC, a Sun Sparcstation 2, a
IBM RS/6000 Model 550, and a Convex C1.}
\author{Brian Glendenning \& Gareth Hunt}
\begin{document}
\maketitle

\section{Introduction}

The Large DDT has been run on a Sun IPC, a Sun Sparcstation 2, A
Convex C1 and an IBM RS/6000 model 550.  The same binaries were run on
both Suns.

In all cases the binaries were
compiled with no optimization, aside from the routines in
\$QPSAP (and below) which were compiled with the maximum optimization
(``-O2'' on the Convex, ``-O4'' on Suns and \mbox{``-O -P
-Wp,-ea278''} on IBMs).  The pseudo-AP memory size was 1.25MW for all
machines (i.e., sum of PKPWD and PKPWD2).

On the workstations, all I/O was to and from a single local disk. On
the Convex, the masters and test files were on a single disk, however
scratch files could go on all 6 disks.  The machines were lightly
loaded, but were not in single user mode, so background tasks such as
processing incoming mail and serving NFS request were occurring while
the test was in progress.

This run on the Convex provided the calibration for the (Large)
AIPSmark, i.e.  a scaling was chosen to make the AIPSmark for the
Convex C1 close to 1.0.

\section{Correctness}

EDGSKP was set to 8 during the runs.  The IPC and SS2 apparently
performed identically; the IBM performance was essentially identical.
The Convex had the greatest differences, which is not surprising
considering that some of it's Q routines are very different (vector)
from the ones used by the workstations. See Table~1.

\begin{table}[h]
\begin{center}
\begin{tabular}{lrrrrrrrr}
	& \multicolumn{2}{c}{Sun IPC} & \multicolumn{2}{c}{Sun SS2} &
\multicolumn{2}{c}{IBM 550} & \multicolumn{2}{c}{Convex C1} \\
Task	& Peak	& RMS	& Peak	& RMS	& Peak	& RMS & Peak & RMS \\
\hline
UVMAP &     12.4 &   15.6 &   12.4 &   15.6 &   12.3 &   17.3 & 11.9 & 15.6 \\
UVBEAM &    14.2 &   16.1 &   14.2 &   16.1 &   14.9 &   17.7 & 14.0 & 16.1 \\
APCLN &     10.9 &   16.8 &   10.9 &   16.8 &   10.8 &   16.8 & 13.0 & 17.3 \\
APRES &     14.2 &   20.7 &   14.2 &   20.7 &   14.2 &   20.6 & 14.2 & 20.7 \\
MXMAP &     13.5 &   18.0 &   13.5 &   18.0 &   13.2 &   18.6 & 13.3 & 18.0 \\
MXBEAM &    14.7 &   19.4 &   14.7 &   19.4 &   14.8 &   19.4 & 14.3 & 19.3 \\
MXCLN &      9.5 &   15.8 &    9.5 &   15.8 &    9.5 &   15.9 &  9.7 & 15.9 \\
VTESS &     18.6 &   27.2 &   18.6 &   27.2 &   18.6 &   27.2 & 21.2 & 29.2 \\
\end{tabular}
\end{center}
\caption{Correct Bits}
\end{table}

\section{Task Real and CPU times}

See Table 2.

\begin{table}[h]
\begin{center}
\begin{tabular}{lrrrrrrrr}
	& \multicolumn{2}{c}{Sun IPC} & \multicolumn{2}{c}{Sun SS2} & \multicolumn{2}{c}{IBM 550} & \multicolumn{2}{c}{Convex C1} \\
Task & Real & CPU & Real & CPU & Real & CPU & Real & CPU \\
\hline

UVSRT(1) &     168 &    66  &   114 &    41  &    43  &   16 & 87 & 77 \\
UVMAP    &     323 &   195  &   142 &   103  &    88  &   25 & 116 & 93 \\
APCLN    &    2564 &  2245  &  1319 &  1278  &   444  &  411 & 607 & 549 \\
APRES    &     236 &   163  &   109 &    85  &    33  &   20 & 106 & 87 \\
ASCAL    &   11687 & 11574  &  6452 &  5920  &  1426  & 1412 & 2672 & 2494 \\
UVSRT(2) &     155 &    67  &    95 &    40  &    44  &   16 & 89 & 78 \\
MXMAP    &     305 &   172  &   133 &    95  &    66  &   25 & 132 & 97 \\
MXCLN    &    3660 &  2947  &  1831 &  1725  &   598  &  495 & 1146 & 931 \\
VTESS    &    1118 &   776  &   490 &   418  &   153  &  111 & 406 & 343 \\
\end{tabular}
\end{center}
\caption{Real and CPU Times (seconds)}
\end{table}

%\begin{table}[h]
%\begin{center}
%\begin{tabular}{lrrrrrr}
%	& \multicolumn{2}{c}{IPC:SS2} & \multicolumn{2}{c}{IPC:IBM} & \multicolumn{2}{c}{SS2:IBM} \\
%Task	& Real	& CPU	& Real	& CPU	& Real	& CPU \\
%\hline
%UVSRT(1) &    1.47 &  1.61 &   3.91 &  4.13 &   2.65 &  2.56 \\
%UVMAP &       2.27 &  1.89 &   3.67 &  7.80 &   1.61 &  4.12 \\
%APCLN &       1.94 &  1.76 &   5.77 &  5.46 &   2.97 &  3.11 \\
%APRES &       2.17 &  1.92 &   7.15 &  8.15 &   3.30 &  4.25 \\
%ASCAL &       1.81 &  1.96 &   8.20 &  8.20 &   4.52 &  4.19 \\
%UVSRT(2) &    1.63 &  1.68 &   3.52 &  4.19 &   2.16 &  2.50 \\
%MXMAP &       2.29 &  1.81 &   4.62 &  6.88 &   2.02 &  3.80 \\
%MXCLN &       2.00 &  1.71 &   6.12 &  5.95 &   3.06 &  3.48 \\
%VTESS &       2.28 &  1.86 &   7.31 &  6.99 &   3.20 &  3.77 \\
%\end{tabular}
%\end{center}
%\caption{Time Ratios}
%\end{table}

\section{Total Run Time}

This is the time taken from subtracting the start time (``run ddtexec'')
from the end time (``PRINTING ANSWERS, ERRORS, OTHER IMPORTANT
MESSAGES''). These times were obtained from looking at the verbose
output of the runs - i.e., ``log'' messages.

\begin{verbatim}
IBM =  3261 seconds
C1  =  6545 seconds
SS2 = 11164 seconds
IPC = 21045 seconds
\end{verbatim}

\section{AIPSmark}

AIPSmarks (Large) are defined to be:

$$ 5000 \over {(total~run~time) - 0.6 \times (ASCAL~run~time)} $$

(all times in seconds).

\begin{verbatim}

IBM = 2.08
C1  = 1.01
SS2 = 0.69
IPC = 0.36

\end{verbatim}

\end{document}
