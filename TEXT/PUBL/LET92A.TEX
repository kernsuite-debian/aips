%-----------------------------------------------------------------------
%;  Copyright (C) 1995
%;  Associated Universities, Inc. Washington DC, USA.
%;
%;  This program is free software; you can redistribute it and/or
%;  modify it under the terms of the GNU General Public License as
%;  published by the Free Software Foundation; either version 2 of
%;  the License, or (at your option) any later version.
%;
%;  This program is distributed in the hope that it will be useful,
%;  but WITHOUT ANY WARRANTY; without even the implied warranty of
%;  MERCHANTABILITY or FITNESS FOR A PARTICULAR PURPOSE.  See the
%;  GNU General Public License for more details.
%;
%;  You should have received a copy of the GNU General Public
%;  License along with this program; if not, write to the Free
%;  Software Foundation, Inc., 675 Massachusetts Ave, Cambridge,
%;  MA 02139, USA.
%;
%;  Correspondence concerning AIPS should be addressed as follows:
%;          Internet email: aipsmail@nrao.edu.
%;          Postal address: AIPS Project Office
%;                          National Radio Astronomy Observatory
%;                          520 Edgemont Road
%;                          Charlottesville, VA 22903-2475 USA
%-----------------------------------------------------------------------
%Body of \AIPS\ Letter for 15 April 1992
%last edited by Glen Langston on 1992 May 6
%last edited by Gareth Hunt   on 1991 July ?

\documentstyle [twoside]{article}

\newcommand{\AIPRELEASE}{April 15, 1992}
\newcommand{\AIPVOLUME}{Volume X}
\newcommand{\AIPNUMBER}{Number 1}
\newcommand{\RELEASENAME}{{\tt 15APR92}}

%macros and title page format for the \AIPS\ letter.
\input LETMAC92A.TEX

\newcommand{\MYSpace}{-11pt}

\normalstyle
\section{15APR92 Available, but Support for VMS and FPS Questioned}
The \RELEASENAME\ version of \AIPS\ is now available.  However, the
last computer running the VMS operating system in the AOC in Socorro
was unplugged in April, while  NRAO Charlottesville's old VAX has been
limping along without a service contract for some time now, and
threatens to die at any moment.  In other words, we have been unable
to test the \RELEASENAME\ version either on a VMS operating system or
on an FPS array processor.  Given all the changes described later in
this \Aipsletter, it is almost certain that there will be significant
problems executing the \RELEASENAME\ version of \AIPS\ on a VMS
machine.  In addition, of course, we can no longer write a VMS-{\tt
BACKUP}-format release tape.

We know that this is an unfortunate situation.  But we do not know how
unfortunate.  We {\it must\/} hear from you as soon as possible if you
expect to require that \AIPS\ continue to run on FPS array processors
and/or under the VMS operating system.  If we have little or no
response, then we will continue as we are, paying unreliable lip
service to VMS and abandoning array processors.  If there is
overwhelming response, then NRAO will have to consider buying --- or
accepting a donation of --- a small, modern VMS computer.  If there is
a modest response, then we will consider cooperative arrangements with
some VMS-based institution to develop and maintain the necessary
VMSisms.

{\Large\it VMS fans, the ball is now in your court --- please send us
your thoughts on this matter using any of the addresses in the
masthead.}

Along similar lines, system managers should be aware that
\RELEASENAME\ \AIPS\ supports Sun computers as Berkeley operating
systems.  We are examining Sun's shift to Bell UNIX-based operating
systems, but do {\it not} support them in the current release.

\section{Personnel}
Dean Schlemmer has left the \AIPS\ group and NRAO to work at the
University of Virginia.

Chris Flatters has returned to Socorro to develop software for
orbiting VLBI observations.

Gareth Hunt will (mostly) be leaving the \AIPS\ group to become
project manager for the \AIPTOO\ project.

Brian Glendenning will (mostly) be leaving the \AIPS\ group to become
project computing scientist for the \AIPTOO\ project.

\section{A New Face for \AIPS}

When \AIPS\ was first designed, NRAO was considered to be advanced to
have three computers --- a ModComp, a VAX, and an IBM mainframe ---
in the same building.  Primitive file transfers were possible between
the first two of these, but ``sneaker-net'' was still the most common
means of file and data transfer.  Things have changed!  At this
writing, there are 57 computers running \AIPS\ in the AOC in Socorro
and an additional 28 in  Charlottesville.  For all the obvious
maintenance, reliability, disk conservation, and user convenience
reasons, these computers should share their resources in a relatively
transparent way without degrading their performance.  \RELEASENAME\
\AIPS\ is the first ``network-wise'' release and is based upon
principles pioneered by Mark Calabretta at the Australia Telescope.

Network ``wisdom'' required changes to the way \AIPS\ is installed,
to the directory structure, and to the way the {\tt AIPS} program is
started. The main points of interest are summarized here.  More
details will be found in the {\it \AIPS\ Unix Installation Guide\/}
and the {\it \AIPS\ Unix Porting Reference Manual\/} that will accompany
the \RELEASENAME\ release.  For users, the most visible changes are
\begin{description}
\item{$\bullet$} \RELEASENAME\ \AIPS\ includes a new server process
which provides graphics display capability for workstations.  It is
called {\tt TEKSERVER} and provides a fast, Tektronix-emulation window
for \AIPS\ graphics tasks such as {\tt TKPL} and {\tt XGAUS}.  It plots
lines and characters much faster than do the TV server programs.

\item{$\bullet$} Users may specify, when they start {\tt AIPS}, which
computers are to provide the data areas.  One simply adds
{\tt da={\it host1},{\it host2},$\ldots$} to the command line to have
{\it host1}, {\it host2}, $\ldots$ \AIPS\ data areas included along
with those of any default hosts.  This has the disadvantage that {\tt
INDISK}s and {\tt OUTDISK}s can change if you change your selection of
disk hosts.  However, it has the advantage that you do not waste time
accessing and --- usually accidentally --- writing data areas on
computers you do not desire.  \AIPS\ disk reservation and
disk-dependent {\tt TIMDEST} limits are supported through the full set
of available data areas.

\item{$\bullet$} Users may specify, when they start {\tt AIPS}, which
device or workstation is to provide the display services.  One adds
{\tt tv=[{\it tvdisp}][:][{\it tvhost}]} where {\it tvdisp} is the
device/workstation on which the displays appear and {\it tvhost} is
the computer that will run the server processes for these displays.
Both of these default to your workstation, even if you are {\tt
rlogin}'d to some other computer.  The full functionality of this
option is available only under X-Windows, but the script will run
the {\tt SSS} server when {\it tvhost} is running SunView.  There is
only one image catalog file for each potential display machine, and
the correct one is used no matter what machine is running your {\tt
AIPS} program.

\item{$\bullet$} \RELEASENAME\ \AIPS\ contains a full implementation
of remote tape handling (see section below).  Users may specify on the
{\tt AIPS} command line a list of computers which may be used to
provide remote tape services.  Adding {\tt tp={\it host1},{\it
host2},$\ldots$} to the command line insures that the remote tape
daemons have been started on {\it host1}, {\it host2}, $\ldots$
in case they had not been started earlier.

\item{$\bullet$} If more than one printer device is available, the
{\tt AIPS} startup script provides a menu for easy selection.  Adding
{\tt pr={\it n}}, where {\it n} is the \AIPS-assigned number of the
desired device avoids repeated looks at the full menu.
\end{description}
For \AIPS\ Managers, the significant changes are
\begin{description}
\item{$\bullet$} The \AIPS\ environment is configured via plain-text
files to list the disks with time-destroy limits and exclusive user
lists ({\tt \$NET0/NETSP}), to list the required (disk 1 or more) and
optional disk areas ({\tt \$NET0/DADEVS.LIST}), to list host names,
architectures, sites, and descriptions ({\tt \$AIPS\_ROOT/HOSTS.LIST}),
and to list the system-parameter files \hbox{({\tt \$NET0/SPLIST})}.
The program {\tt SETSP} sets parameters in the (numerous) {\tt SP}
files, while procedures {\tt SYSETUP} and {\tt DASETUP} set up the
basic control files ({\tt AC}, {\tt GR}, etc.) and data areas,
respectively.

\item{$\bullet$} The directory tree has been changed to support
multiple architectures and multiple hosts.  There are
architecture-specific areas for binaries, memory files, libraries, and
a new {\tt PREP} area for intermediate results of compilation and
linking.

\item{$\bullet$} The installation process (finally) contains an {\tt
INSTEP1} procedure.  It will automate --- for the commonly used \AIPS\
architectures --- most of the process of setting up and preparing the
\AIPS\ system for re-compilation.

\item{$\bullet$} The \RELEASENAME\ version of {\tt LIBR} and {\tt
LINK} in {\tt \$SYSSUN} contain support for SUN shared libraries.
Dynamic linking can reduce the size of the binary load modules from
300 to about 80 Megabytes.  Under this option, {\tt COMLNK} often
produces alarming messages about unresolved external references, but
it produces working executables anyway.  Programs running while their
shared libraries are rebuilt have had a nasty habit of dying abruptly,
however, and all tasks using the {\tt APLNOT} library require over 8
Mbytes of swap space whether they use the few memory-hungry
subroutines or not.

\end{description}

\subsection{Workstation TV Displays}

The \AIPS\ package can use the MIT X-Windows System for grey-scale
display on workstation computers through a program called the
television server.  {\tt XAS} is the most portable of the \AIPS\
television servers since it uses vanilla X11 services.  This means
that {\tt XAS} should function with all X-Window System managers,
including {\tt olwm} (OpenLook), {\tt olvwm}, {\tt twm}, {\tt tvtwm},
and others.  For \RELEASENAME, {\tt XAS} was rewritten to use the same
Y and Z routines as the {\tt XVSS} (OpenLook) and {\tt SSS} (SunView)
servers.  It was given 2 grey-scale memory planes and 4 graphics
overlay planes with a much wider dynamic range for the images.  All
color, zoom, scroll, cursor, and window resize/move functions are now
fully and correctly supported.  {\tt XAS} now allows (via the {\tt
.Xdefaults} file) user-defined cursor shape, graphics and cursor
colors, and initial geometries.  All three servers were given a new
function to tell the client their basic parameters.  This allows {\tt
TVINIT} (subroutine {\tt YINIT}) to test the parameters of the TV
servers and correct the TV parameter file.  Subroutine {\tt YWINDO}
will also do this on TV open if a window-size error occurs.  This
allows different users to use different servers on same the host (at
different times) without requiring an \AIPS\ Manager to reset the
TV parameter file.

\subsection{Remote Tapes}
An \AIPS-like solution to provide remote tape services was implemented
in \RELEASENAME.  A daemon program, {\tt TPMON}, was written to
receive, effectively, tape Z-routine calls over some network socket
and then execute them on the local host.  The new \AIPS\ procedures
automatically see that appropriate {\tt TPMON}s are running locally
and will also start them on remote machines as specified by the user.
{\tt TPMON1} handles pseudo-tape disk files and hence is needed even
on tape-less computers.  {\tt TPMON2}, {\tt TPMON3}, $\ldots$, {\tt
TPMON}{\it n} handle true tape devices 1, 2, $\ldots$, $(n-1)$,
respectively.  For real tape devices, the {\tt MOUNT} command is now
required and the user may specify a remote computer and remote tape
drive number at that time.  For pseudo-tape disk files, the user
specifies {\tt INFILE} or {\tt OUTFILE} with the grammar:
\vspace{\MYSpace}
\begin{center}
{\tt INFILE =
'$<${\it host\_name}$>$::$<${\it logical\_name}$>$:$<${\it file\_name}$>$' }
\end{center}
\vspace{\MYSpace}
where $<${\it logical\_name}$>$ must be an environment variable known to
the remote {\tt TPMON}, \ie\ {\tt FITS} or some other generic variable
such as {\tt HOME} (note that both of these are on the remote
machine).  User-defined variables will not be known to the remote {\tt
TPMON} and, hence, cannot be used.  Tapes must also be {\tt
DISMOUNT}ed from the {\tt AIPS} program either explicitly or automatically
in the \hbox{{\tt EXIT}}.  Even the abort handler will attempt
dismounts from {\tt AIPS}!


\subsection{Memory Files}

The structure of \AIPS\ {\tt ME}mory files was also revised in
\RELEASENAME.  Space for POPS procedures and for temporary and
permanent variables was increased:
\vspace{\MYSpace}
\begin{center}
\begin{tabular}{lll}
     950 & $\rightarrow$ & 1450 words of temporary literal storage \\
   14760 & $\rightarrow$ &21928 words of program storage \\
    4173 & $\rightarrow$ & 6221 words of variable storage \\
    4096 & $\rightarrow$ &10240 words of source (text) storage. \\
\end{tabular}
\end{center}
\vspace{\MYSpace}
The work area for text is now kept in core to save disk.  This has the
side effect of speeding \POPS\ compilations enormously.  The
``virgin'' \POPS\ vocabulary ({\tt RESTORE 0}) area is now in a
release-dependent, \POPS-number-independent file, while the user
temporary storage area ({\tt STORE 1} / {\tt RESTORE 1}) area is kept
in host-dependent, \POPS-number-dependent files.  Storage areas 2 and
3 have been eliminated since they are seldom used and the files are
now large (150 Kbytes per \POPS\ number per host per storage area).
\eject

\section{Additional Improvements in 15APR92}

Other corrections and improvements made to \AIPS\ for the
\RELEASENAME\ release include:
\begin{description}
\myitem{AFILE} Introduced new task to manipulate text A-files produced
   by the Haystack VLBI {\tt FRNGE} program.
\myitem{AHIST} Developed new task to produce an ``adaptive'' (rolling
   window) histogram-equalized image.
\myitem{AIPS} Corrected conversions between celestial and galactic
   coordinates to account correctly for epoch and created a new {\tt
   AIPS} help file which is also available as a UNIX {\tt man} page.
\myitem{BPASS} Corrected and improved the handling of multiple
   polarizations with large numbers of IFs/channels and the handling
   of changes in reference antenna (required for interpolating in
   time).
\myitem{CALIB} Raised buffer sizes which were limiting
   self-calibration models to the first 1000 clean components.
\myitem{CHNDAT} Corrected the handling of the frequency increment in a
   large number of calibration and other \uvdata\ routines.
\myitem{CLCOR} Changed opcode meanings to help users avoid a common
   input error for polarization-angle corrections.
\myitem{CLIP} Improved to work on compressed \uvdata.
\myitem{CVEL} Rewrote to handle multi-source files more simply and
   to clean up numerous other details.
\myitem{DDT} Modified for running the huge test and added a whole new
   spectral-line test called {\tt VLAL}.
\myitem{FILLM} Corrected $(u,v,w)$ to make them refer to the reference
   frequency rather than the frequency of the current scan, changed
   {\tt VLAOBS} default to mean all programs matching the other
   adverbs, corrected date in appended files, weights, default
   integration time, and first entry in CL tables.
\myitem{FITLD} Wrote new task to load any number of FITS images and
   \uvdata\ sets from tape.
\myitem{FITTP} Dropped the conversion of FQ to CH tables and the
   warning about binary extension files, corrected limits on the
   number of columns in tables, and added support for new VLBA binary
   tables.
\myitem{FRING} Corrected least-squares fitting, the use of flagged or
   uninitialised data, and the optional display of the results.
\myitem{HOLGR} Submitted from the AT a new task to process antenna
   holography data.
\myitem{IMLIN} Added new task to fit the continuum with a polynomial
   baseline and subtract it from spectral-line images.
\myitem{KNTR} Added the option to plot the half-power beam contour.
\myitem{LISTR} Improved the scan listing format and corrected the
   calculation of the matrix average and RMS of phase.
\myitem{LWPLA} Made the Postscript output more readable so that the
   files may be edited using public domain tools and corrected
   handling of long buffers of grey pixels and characters.
\myitem{MAPIT} Changed automatic imaging and self-calibration routines
   to allow more user input, an OLAF-like interactive mode, and an
   initial self-calibration model.
\myitem{MK3IN} Added the capability to select scans based on the
   A-file text information, corrected the sign of the phases of the
   phase cals, added tables to contain Haystack {\tt FRNGE} results,
   and made numerous minor corrections and documentation improvements.
\myitem{MX} Improved the minor cycle cleaning step by including a
   larger portion of the synthesized beam during the early stages of
   deconvolution; this mainly helps data with very poor phase
   calibration.
\myitem{PHSRF} Added new task to reference all channels of a \uvdata\
   set to the phase and, optionally, amplitude of the average of a set
   of channels.
\myitem{SBCOR} Installed temporary new task to correct VLBI data that
   contain mixed MKIII/VLBA baselines.
\myitem{SN2CL} Fixed bugs related to duplicated records and unselected
   calibrators.
\myitem{SNCOR} Added options to zero fringe rates, to multiply
   amplitudes, to flag solutions with delays or rates outside
   specified ranges, and to reference all phases to a single IF.
\myitem{SNPLT} Changed to allow plotting of all IF-dependent variables
   on the same page and to allow plotting of the SNR/weight column.
\myitem{SPLIT} Corrected error which could cause some data to be
   written without the bandpass correction.
\myitem{STARS} Added columns to allow 20 different types of star
   markers, rotations, and labels and changed plot tasks to support
   them.
\myitem{TVFLG} Rearranged and reworded the menu, speeded handling of
   graphics channels for workstations, corrected gridding and other
   errors.
\myitem{UVCOP} Corrected to scale $(u,v,w)$ to new reference frequency
   when selecting by FQ ID.
\myitem{UVLIN} Added new task to subtract continuum from spectral-line
   \uvdata\ using a linear fit to real and imaginary parts.
\myitem{UVLSF} Submitted new task to subtract a least-squares-fit
   linear continuum/bandpass from spectral-line \uvdata, with the
   option to write a continuum \uvdata\ file.
\myitem{UVSUB} Improved to work on compressed \uvdata.
\myitem{VBPLT} Corrected plotting of VLBI closure-phase models.
\myitem{XTRAN} Corrected handling of negative declinations in finding
   coordinates from optical images.
\myitem{ZABORS} Fixed a nagging problem in the \AIPS\ abort handlers
   which occasionally caused tasks to hang waiting on message or
   accounting files.
\end{description}

\section{Communications}

The \RELEASENAME\ version of \AIPS\ is now available via a variety of
tape formats including Exabyte, QIC-24 Cartridge, TK50, and 9-track
1600- and 6250-bpi tapes and also via Internet \ftp.  We urge you to
obtain the latest version.  The \AIPS\ group is unable to provide
useful support for releases which are now more than a year out of
date.  In addition, this release supports new capabilities for the VLA
and VLBI, has modern, networked support for tapes, displays, etc., and
is incompatible in significant ways with previous releases (thereby
inhibiting partial installations).

The next planned release of \AIPS\ will be as {\tt 15APR93}, but
a VLBI-oriented {\tt 15OCT92} release may also be made.  Significant
bug fixes will be made available via anonymous \ftp\ (see ``Patch
Distribution'' below).  This should reduce the total effort required
to maintain \AIPS\ at NRAO and the user's home institution.  The
\AIPS\ News Letter will be targeted for the $15^{\uth}$ of April and
October in future, omitting the January and July issues.  There are
two quick ways to distribute news to the \AIPS\ community.  The old
way --- which still works --- is to send mail to bananas@nrao.edu for
forwarding to all addresses in that exploder.  The new way is to post
a message with the USENET News group ``alt.sci.astro.aips''.  This
news group allows \AIPS\ users to discuss methods of
radio-astronomical data reduction and provides a forum for discussion
of \AIPS\ questions, bugs, and features.

System managers should be aware that \RELEASENAME\ \AIPS\ supports
Sun computers as Berkeley operating systems.  We are examining Sun's
shift to Bell UNIX-based operating systems, but do {\it not} support
them in the current release.

\clearpage

\section{VLBA/VLBI Post-processing Software}

\subsection{VLBI Polarization Calibration}

   The \RELEASENAME\ release of \AIPS\ contains the initial version of
routines that allow the calibration of polarization sensitive VLBI
data.  These routines are incorporated into the standard \AIPS\
calibration tasks, principally {\tt PCAL} and \hbox{{\tt SPLIT}}.
However, the details of the calibration procedure differs
substantially from those used for VLA data.  Optimal procedures have
not yet been established so documentation is not well developed.
Persons wishing to calibrate VLBI polarization data should contact
Bill Cotton (Internet: bcotton@nrao.edu). The software in the
\RELEASENAME\ release only allows imaging data which has both cross
polarized correlations (RL and LR) measured for each visibility.  The
next release will allow imaging with only RL or LR for sources with
weak circular polarization.

\subsection{VLBI Data Processing Workshop}

On March 19, 1992 a small workshop was held in the AOC to discuss
the current state of the VLBI software within \AIPS\ and the plans for
future development.  The workshop was an informal one, but lists of
suggestions and priorities were developed.

The Workshop started with a presentation by Bill Cotton of the
programmer's view of multi-IF VLBI calibration.  One of the principal
points that Bill made was that display of multi-IF VLBI data is not
satisfactory at all, and this area should be an area of active
research.  The second presentation was by Richard Porcas.  He gave a
user's view of the  current software and pointed out areas that need
improvement and development, and also areas where possible pitfalls
can occur.

We spent a considerable amount of time discussing ideas for improving
the software and the whole VLBI system.  A list of the suggestions,
arranged by general priority, follows:

Short-term priorities ($< 6$ months):
\begin{description}
\item{$\bullet$} Write a task to generate an HF table (the table that
contains the output of the Haystack {\tt FRNGE} program) from an
\AIPS\ CL table.  The HF table will then be passed to the NASA {\tt
SOLVE}  package to facilitate comparison between the \AIPS\ and
Haystack fringe-fitting tasks and, ultimately, the VLBA and MkIII
correlators.

\item{$\bullet$} Encode the integration time of a given visibility
spectrum as a random parameter.

\item{$\bullet$} Frequency shift MkIII data by a small amount so as
not to lose the information in the channel at 0 MHz (relative). As a
side benefit of this change, encode the lower sideband (LSB) and upper
sideband (USB) data as separate IFs (in \AIPS\ parlance).  One
advantage of this latter change is that it will obviate the need for
the 130 degree difference between USB and LSB for VLBA -- non-VLBA
baselines.

\item{$\bullet$} Develop a faster, more sophisticated version of
{\tt IBLED}, possibly incorporating table editing.

\item{$\bullet$} Write a task to provide a data summary; \ie\ a list
of the amount of data per IF per baseline.

\item{$\bullet$} Develop a baseline-by-baseline fringe-fitting task
followed by an antenna-based determination of residual rates and
delays.  This should lead to increased robustness of fringe-fitting,
but would not have the signal-to-noise-ratio advantages of full global
fringe fitting.

\item{$\bullet$} Encode multi-band delay explicitly inside the SN and
CL tables.

\item{$\bullet$} Revamp {\tt ANCAL} for multi-IF data

\item{$\bullet$} Establish user documentation for various aspects of
VLBI software.

\item{$\bullet$} Create an e-mail exploder limited to those interested
in VLBI software within the \AIPS\ context.  ({\it Eds.~note: the
\AIPS\ USENET News group ``alt.sci.astro.aips'' and the bananas e-mail
exploder offer less limited, but entirely suitable, mechanisms for
this.})
\end{description}

Longer-range ($\leq 2$ years) priorities:
\begin{description}
\item{$\bullet$} Develop fringe-rate mapping for spectral line data.

\item{$\bullet$} Generate multi-IF displays.

\item{$\bullet$} Improve phase-referencing and Astrometry --- John
Conway and Tony Beasley will begin this development by attempting to
provide geometrical and atmospheric corrections to a modified version
of {\tt CLCOR}.
\end{description}

Other suggestions for less essential, but nonetheless useful,
developments:
\begin{description}
\item{$\bullet$} Read the \#SK file in {\tt MK3IN} to determine
various parameters of the observation.

\item{$\bullet$} Keep the amplitude of the phase-cal tone. We may need
an explicit phase-cal table since the VLBA will have multiple tones
per BBC.

\item{$\bullet$} Combine spectral and IF averaging inside {\tt SPLIT}.

\item{$\bullet$} Add a task to estimate the coherence time of the
data.

\item{$\bullet$} Write an image editor, \ie\ have the ability to place
a box on an image and edit the clean components within that box from
the CC file.  ({\it Ed.~note: this is already available in {\tt TAFLG}.})

\item{$\bullet$} Allow greater flexibility in using starting models,
\eg\ allow the use of multi-component models without having to encode
them as CC files attached to an image.

\item{$\bullet$} Create the ability to plot a model map in one step.

\item{$\bullet$} Write a task to take the phase difference between two
sets of simultaneous VLBI data (\eg\ S/X), and to be able to use this
to correct for ionospheric refraction or to scale atmospheric
refractive effects from one frequency to another.

\end{description}

\section{Patch Distribution}
Since \AIPS\ is now released only annually, we have developed a method
of distributing important bug fixes and improvements via {\it
anonymous} \ftp\ on the NRAO Cpu {\tt baboon} (192.33.115.103).
Documentation about patches to a release is placed in the
anonymous-ftp area {\tt pub/aips/}{\it release-name} and the code is
placed in suitable subdirectories below this.

Reports of significant bugs in {\tt 15APR91} \AIPS\ have been
relatively few; however, the documentation file {\tt
pub/aips/15APR91/README.15APR91} mentions the following items:
\begin{description}
\myitem{CALIB} Corrected limit in self-calibration to allow $> 1000$
   clean components for the source model.
\myitem{CLIP} Increased buffer size to allow clipping $> 256$ channels
   of spectral line \uvdata.
\myitem{FILLM} Fixed calculation of $(u,v,w)$ coordinates to be
   relative to a fixed reference frequency and made other less
   significant changes.
\myitem{HORUS} Fixed errors for uniform-weighted spectral-line images
   $\geq$ 1024x1024.  Also fixed {\tt MX}.
\myitem{MAPIT} Improved logic and interactivity of the procedures for
   automatic imaging and self-calibration.
\myitem{VTESS} Patched to allow it to handle 4096x4096-pixel images.
\myitem{INSTEP4} Corrected procedure to compile and link all programs
   in the {\tt AIPNOT} directory.
\myitem{SunOS} Described a work-around to a common bug in SunOS which
   causes {\tt f77} to compile {\tt DATA} statements incorrectly.
\myitem{SunOS} Revised {\tt SYSSUN} procedures to support dynamic
   (shared) libraries for \AIPS\ executables.
\end{description}
Note that we do not revise the original {\tt 15APR91} tape for these
patches.  No matter when you received your {\tt 15APR91} tape, you
must fetch and install these patches if you require them.  There are
three documentation files, {\tt README.15APR91},
{\tt README\_TOO.IBMRS6000}, and \hbox{{\tt README\_TOO.SPARC}}.  A
sample \ftp\ session, with {\it italic font} for commands typed by the
user, is given below.

\tablestyle
{\tt
\begin{tabular}[h]{l}
\underline{\hskip 6.2in} \\
\% {\it ftp 192.33.115.103} \hfill (if baboon.cv.nrao.edu doesn't work)\\
Connected to 192.33.115.103. \\
220 baboon FTP server (SunOS 4.1) ready. \\
Name (192.33.115.103:glangsto): {\it anonymous} \hfill (any one can log in) \\
331 Guest login ok, send ident as password. \\
Password:~ {\it glangsto@nrao.edu}  \hfill (use your e-mail address) \\
230 Guest login ok, access restrictions apply. \\
ftp$>$ {\it cd pub/aips/15APR91} \hfill (go to the directory with patches) \\
250 CWD command successful. \\
ftp$>$ {\it ls}      \hfill (list the directory contents) \\
200 PORT command successful. \\
150 ASCII data connection for /bin/ls (192.33.115.103,3154) (0 bytes). \\
HELP \\
Q \\
QY \\
README.15APR91 \\
226 ASCII Transfer complete. \\
42 bytes received in 0.0064 seconds (6.4 Kbytes/s) \\
\end{tabular}

\begin{tabular}[h]{l}
ftp$>$ {\it get README.15APR91} \hfill (get the instruction file) \\
200 PORT command successful. \\
150 ASCII data connection for README.15APR91 (192.33.115.103,3155) (4741 bytes). \\
ftp$>$ {\it cd Q/PGM/NOTST}        \hfill (go to the directory with CALIB.FOR) \\
250 CWD command successful. \\
ftp$>$ {\it ls}    \hfill (list the contents) \\
200 PORT command successful. \\
150 ASCII data connection for /bin/ls (192.33.115.103,3164) (0 bytes). \\
CALIB.FOR \\
226 ASCII Transfer complete. \\
11 bytes received in 0.03 seconds (0.35 Kbytes/s) \\
ftp$>$ {\it get CALIB.FOR} \hfill (get the program) \\
200 PORT command successful. \\
150 ASCII data connection for CALIB.FOR (192.33.115.103,3165) (102399 bytes). \\
226 ASCII Transfer complete. \\
local: CALIB.FOR remote: CALIB.FOR \\
104866 bytes received in 1.1 seconds (90 Kbytes/s) \\
ftp$>$ {\it quit} \hfill (exit the program) \\
221 Goodbye. \\
\underline{\hskip 6.2in} \\
\end{tabular}
}

\normalstyle

\subsection{Installing Patches}
The \AIPS\ files must be placed in the correct \AIPS\ source areas,
compiled, and linked for the fixes to take effect.  These steps should
only be done by the local \AIPS\ Manager.
\begin{description}
\item{1.} Set the \AIPS\ environment variables ({\tt \$CDTST} or {\tt
   CDNEW}) before you can compile and link.
\item{2.} Move the old version of the software to be changed to a
   backup area.
\item{3.} Move the new version to the replacement position.
\item{4.} Compile and link the software ({\tt COMRPL} and {\tt
   COMLNK}).
\end{description}
These steps are summarized in the {\tt README.15APR91} file.

A sample session for {\tt CALIB} is listed below:
\begin{center}
\begin{tabular}{lr}
\hline
\% \$CDTST & (Create the \AIPS\ environment) \\
\% mv \$QPGNOT/CALIB.FOR \$QPGNOT/CALIB.15APR91 & (make the backup) \\
\% mv CALIB.FOR \$QPGNOT & (move the patch) \\
\% COMLNK \$QPGNOT/CALIB.FOR & (Compile and link the patch) \\
\hline
\end{tabular}
\end{center}
If no errors are logged, the patch is complete. Do the others.

As bugs to \RELEASENAME\ are found, the patches will be placed in the
\ftp\ area for \RELEASENAME.

\section{Latest \AIPS\ Memos}
Below is a list of the latest \AIPS\ Memos.
\begin{center}
\begin{tabular}{ccl}
\hline
MEMO  &        DATE   &             TITLE and AUTHOR  \\
\hline\hline
  69  &       91/03/28 &    The 1990 \AIPS\ Site Directory. \\
      &                &     Alan Bridle and Joanne Nance, NRAO \\

  70  &       91/04/24 &    The 1990 \AIPS\ Site Survey. \\
      &                &     Alan Bridle and Joanne Nance, NRAO \\

  71 &        91/04/08&     A Comparison of DDT results IBM RS/6000 \\
      &                &    and Convex C-1. \\
      &                &     Patrick P. Murphy, NRAO \\

 *72 &        91/05/07&     \AIPS\ Imaging and Self-Calibration: MAPIT \\
      &                &     Glen Langston, NRAO \\


 *73  &       91/05/16 &    \AIPS\ DDT History (supersedes memo 63) \\
      &                &     Glen Langston, Pat Murphy and Dean
Schlemmer, NRAO \\


  74  &       91/08/08 &    \AIPS\ at the Australia Telescope National
Facility \\
      &                &     Mark Calabretta, ATNF \\

  75  &       91/09/23 &    15APR91 DDT Results on a Sun IPC, Sun Sparc-station 2 \\
      &                &    Convex C1, and an IBM RS/6000-Model 550 \\
      &                &     Brian Glendenning \& Gareth Hunt, NRAO \\

 *76 &        91/11/27&     Summary of \AIPS\ Continuum UV-data
Calibration  \\
      &                &    from VLA Archive Tape to UV FITS Tape \\
      &                &               (supersedes memo 68) \\
      &                &     Glen Langston, NRAO \\
\hline
\end{tabular}
\end{center}

To order, use an \AIPS\ order form or e-mail your request to
aipsmail@nrao.edu.  Memos can also be gotten via anonymous \ftp,
except for figures which may be missing in those denoted by an asterisk.

To use \ftp\ to retrieve the memos:
\begin{description}
\item{ 1.} \ftp\ baboon.cv.nrao.edu  or  192.33.115.103
\item{ 2.} login anonymous, for password use your e-mail address
\item{ 3.} cd pub/aips/memos
\item{ 4.} (get/read  AAAREADME for more information)
\end{description}

For a complete listing of the \AIPS\ memos series, contact Ernie Allen
at any of the addresses in the masthead.

\clearpage
\section{Gripes Database}

The gripes of \AIPS\ users are currently recorded in an emacs-based
{\it Gripes Database}.  This database lists in chronological order the
gripes of \AIPS\ users, and, in many cases, the answers to these gripes.
This database may be viewed by the astronomical community via an
anonymous login on the cpu {\tt gripe}.  The viewing program supports
VT100 terminals as well as most, if not all, workstation windows.

Full instructions for using the Gripe database may be obtained via
anonymous \ftp\ to baboon (192.33.115.103).  The file is called
{\tt GRIPE.README} and is located in the area {\tt pub/aips/gripes}.
To use the gripe database from a computer running the unix operating
system type:

\centerline{\tt rlogin gripe.nrao.edu -l gripe}

The Gripes database will automatically begin execution.

Below is an example of the Gripes database form to view very recent
Gripes that have keywords beginning with Work.  Any mail sent by the
session illustrated will go to glangsto in Charlottesville.

\begin{center}
\begin{verbatim}
		    AIPS Gripes Selection/options

 Selection criteria
     User (Joe Blow)                    ______________________________
     Status (new, answer)               ___________
     Keyword (MX, tape, Work-Around)    Work___________
     Beginning date (dd-mmm-yyyy)       15-Dec-91__
     Beginning Gripe no. (3456)	        4700_
     Arbitrary string in a gripe        ___________________________________

 Options
     Display (index, full, one, exit)   index__
     E-Mail address (jblow@esu.edu)     glangsto@nrao.edu__________________

 Please send comments and complaints to glangsto@nrao.edu
     ^P moves up   the screen,   (^P= control-P)
     ^N moves down the screen,   <return> to see gripes
     Gripes are best viewed with a VT100 or an XTERM

\end{verbatim}
\end{center}

When the Gripe database user types a $<$return$>$, the database will
show all gripes with number greater than 4700 which have a
``work-around''.

\begin{center}
\begin{verbatim}
                     AIPS Gripe index

  No.    User              Date         Status      Keyword(s)
4756   Elias Brinks        16-FEB-1992  ANSWER NEW   Work-around
     Adding files to tape after AVFILE doesn't work.
4757   Elias Brinks        16-FEB-1992  ANSWER NEW   Work-around
     add GETO2NAME verb
4763   Alan Bridle         13-APR-1992  ANSWER NEW   WORK-ROUND
      Slow Convex AIPS startup Script
\end{verbatim}
\end{center}

Gripes 4756, 4757 and 4763 may be viewed in turn with still more
$<$return$>$'s.

\clearpage
Upon the Gripe database user typing another $<$return$>$,
the database will show the text for gripe 4756.

\begin{center}
\begin{verbatim}
                     AIPS Gripe report _GRIPE-NO: 4756

_GRIPE-ENTERED: 16-FEB-1992 18:50:00   _SYSTEM: taos.aoc.nrao.edu
_STATUS: ANSWER NEW  _KEYWORD:  Work-around
_ONE-LINE: Adding files to tape after AVFILE doesn't work.
_USER: Elias Brinks   _USER-NUMBER: 1276   _AIPS-RELEASE:  15APR92

_GRIPE: Tape handling in AIPS TST is not fail safe. I tried to "edit" an
  existing Exabyte tape, sitting on TAOS and using its drive. I used
  AVFILE to skip a number of files. When I attempted to write a file
  (in fact overwriting an existing one) with FITTP and DOEOT -1 an error
  message appeared: ZTAP: LUN 31 IO ERROR AT 1 OP OF BAKF ERROR CODE 0005
  What's worse, as I had my FITTP set up in a POPS loop, when it got to
  the second file, the tape REWOUND!!!!! (and there is no off-line button
  on an Exabyte!!!

_ANSWERED-BY: Eric Greisen/Glen Langston     _ANSWER-DATE:  1992-March-9
_ANSWER: The Exabyte tape problem is a SUN Exabyte driver bug that
will only allow files to be written at the beginning of tape
or the end of information (double end-of-file or EOT).  Exabyte tapes
can be edited on IBM computers.

Your procedure would have worked much better if you had specified
DOWAIT = TRUE.  This allows FITTP to return a completion code to AIPS
which will stop the FOR loop before any further damage is done.
WAITTASK should only be used interactively or when you are certain you
want to continue even if the task involved fails.

\end{verbatim}
\end{center}

Gripes 4757 and 4763 may be viewed with still more $<$return$>$'s.

In order to improve access to the Gripes database, we expect to move
it from a computer in Charlottesville to one in Socorro.  We will then
change the alias {\tt gripe} so that remote users should see no
change.  Suggestions and feedback from the community concerning the
Gripes and the database would be appreciated.

\end{document}

