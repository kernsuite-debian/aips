%-----------------------------------------------------------------------
%;  Copyright (C) 1996-1997
%;  Associated Universities, Inc. Washington DC, USA.
%;
%;  This program is free software; you can redistribute it and/or
%;  modify it under the terms of the GNU General Public License as
%;  published by the Free Software Foundation; either version 2 of
%;  the License, or (at your option) any later version.
%;
%;  This program is distributed in the hope that it will be useful,
%;  but WITHOUT ANY WARRANTY; without even the implied warranty of
%;  MERCHANTABILITY or FITNESS FOR A PARTICULAR PURPOSE.  See the
%;  GNU General Public License for more details.
%;
%;  You should have received a copy of the GNU General Public
%;  License along with this program; if not, write to the Free
%;  Software Foundation, Inc., 675 Massachusetts Ave, Cambridge,
%;  MA 02139, USA.
%;
%;  Correspondence concerning AIPS should be addressed as follows:
%;          Internet email: aipsmail@nrao.edu.
%;          Postal address: AIPS Project Office
%;                          National Radio Astronomy Observatory
%;                          520 Edgemont Road
%;                          Charlottesville, VA 22903-2475 USA
%-----------------------------------------------------------------------
%Body of \AIPS\ Letter for 15 October 1996

\documentstyle [twoside]{article}

\newcommand{\AMark}{AIPSMark$^{(93)}$}
\newcommand{\AMarks}{AIPSMarks$^{(93)}$}
\newcommand{\LMark}{AIPSLoopMark$^{(93)}$}
\newcommand{\LMarks}{AIPSLoopMarks$^{(93)}$}
\newcommand{\AM}{A_m^{(93)}}
\newcommand{\ALM}{AL_m^{(93)}}

\newcommand{\AIPRELEASE}{October 15, 1996}
\newcommand{\AIPVOLUME}{Volume XVI}
\newcommand{\AIPNUMBER}{Number 2}
\newcommand{\RELEASENAME}{{\tt 15OCT96}}
\newcommand{\OLDNAME}{{\tt 15JAN96}}

%macros and title page format for the \AIPS\ letter.
\input LET96.MAC
\input psfig

\newcommand{\MYSpace}{-11pt}

\normalstyle
\vfill
\section{General developments in \AIPS}

\subsection{Staff}

Ketan Desai has moved from Charlottesville to Socorro, where he will
continue working on developments in \AIPS\ related to Space
\hbox{VLBI}. Eric Greisen has ``elected'' to concentrate his efforts on
developing \AIPS\ applications outside of {\tt TST}; it is the intention
that some of these developments will find their way back into standard
\AIPS.

\subsection{Current release}

The \RELEASENAME\ release of Classic \AIPS\ is now available.  It may
be obtained via {\it anonymous} ftp or by contacting Ernie Allen at
the address given in the masthead.  \AIPS\ is now copyright \copyright
1995, 1996 by Associated Universities, Inc., NRAO's parent
corporation, but may be made freely available under the terms of the
Free Software Foundation's General Public License \hbox{(GPL)}.  This
means that User Agreements are no longer required, that \AIPS\ may be
obtained via anonymous ftp without contacting NRAO, and that the
software may be redistributed (and/or modified), under certain
conditions.  The full text of the GPL can be found in the {\tt
15JUL95} \Aipsletter. Details on how to obtain \AIPS\ under the new
licensing system appear later in this \Aipsletter.

A total of 225 copies of the \OLDNAME\ release were distributed, of
which 104 were in source code form and 121 were distributed as binary
executables.  The table below shows the breakdown of how these copies
were distributed. This includes both source code distributions and
binary distributions. The latter method is gaining popularity quickly:
54 \% of all distributions include binaries.

\begin{center}
\begin{tabular}{|r|r|r|r|r|} \hline\hline
{ftp} & {8mm} & {4mm} & {QIC} & {Floppy} \\ \hline
186   &   23  &   14  &    1  &       1  \\ \hline\hline
\end{tabular}
\end{center}

\eject
User feedback suggests that the distribution over operating system for
installed versions of \OLDNAME\ was as follows:

\parbox{6cm}{\begin{center}
\begin{tabular}{|l|r|r|} \hline\hline
{Operating System} & {No.} & {\%} \\ \hline
Solaris/SunOS 5 &    422       & 50.5 \\
SunOS 4         &    183       & 21.9 \\
Dec Alpha       &     70       &  8.4 \\
PC Linux        &     61       &  7.3 \\
SGI             &     42       &  5.0 \\
IBM /AIX        &     27       &  3.2 \\
HP-UX           &     26       &  3.1 \\
Ultrix          &      4       &  0.5 \\
Convex          &      1       &  0.1 \\ \hline
Total           &    836       &      \\ \hline\hline
\end{tabular}
\end{center}} \hfill \parbox{10cm}{\begin{minipage}{10cm}

Solaris/SunOS 5 appears to have made significant gains over SunOS 4
since the last release. These figures are affected by the percentage
of \AIPS\ users that register with NRAO. We remind serious \AIPS\
users that registration is required in order to receive user support.

The next release of \AIPS\ will be {\tt 15APR97}, and will resume the
usual bi-annual release schedule.
\end{minipage}}

\subsection{System Support}

The SunOS 4.1.{\it x} operating system is now almost phased out at the
NRAO in favor of Solaris v 2.4 or v. 2.5. We will keep one or more
computers on the old system as long as we can.  Although we do not
anticipate major problems any time soon, it is inevitable that the
quality of our support for the old operating system will diminish with
time.

Starting with the current release, we will only ship Linux-elf
executables.

Binaries available via ftp are now GNU-zipped for significantly faster
download time.  A copy of {\tt gunzip} is included in each {\tt
\$LOAD} area so these files can always be uncompressed.

\section{Improvements for users in 15OCT96}

\subsection{$UV$ data processing}

\subsubsection{Model-fitting --- 1 - {\tt SLIME}}

{\tt SLIME} (Slick Interactive Model Editor) is a new model-fitting
program for \AIPS, and was developed as part of a program to add
support for Space VLBI to \AIPS. It has roughly the same capabilities
as {\tt UVFIT} but allows you to mix components of various types and
allows you to use any number of components and data points (subject to
the amount of memory on your workstation and your degree of patience).
{\tt SLIME} differs significantly from {\tt UVFIT} however in terms of the
user interface.

{\tt SLIME} is the first \AIPS\ task that is a true X Window System
program.  It presents you with a graphical model editor that displays
a scale diagram of your source model.  You may also open additional
windows that plot the amplitude and phase data for specified baselines
together with the visibility function expected from the current model.
You can move components or change their sizes by dragging their
graphical representations with the mouse in the editing window and the
plotted model data will be updated to reflect your changes.  This
reduces the amount of guess-work or trial-and-error needed to arrive
at a good initial model.  When you are finally satisfied that you have
a model that is in fairly good agreement with the data, you can ask
{\tt SLIME} to perform a least-squares fit to optimize the model
parameters.  {\tt SLIME} models can be stored in \AIPS\
CLEAN-component files and used as inputs for other \AIPS\ tasks.

Although it is a true \AIPS\ task that runs in the \AIPS\ environment
and uses the \AIPS\ libraries, {\tt SLIME} is written in C rather than
FORTRAN which is not well suited to developing X Window System
programs.  C++ was considered as an implementation language but the
potential for portability problems arising from the various different
dialects of C++ in use was judged to outweigh the advantages of C++.
{\tt SLIME} is, however, based on an object-oriented design and is intended
to be easily extensible.

Because the \AIPS\ compilation procedures do not handle this type of
program, {\tt SLIME} is distributed separately from \AIPS.  The source
code is available for down-loading through the {\tt SLIME} home page
at {\tt
http://www.nrao.edu/\lower.7ex\hbox{\char'176}cflatter/slime.html} and
can be installed into an existing \AIPS\ installation by anyone with
\AIPS\ manager privileges.  The existence of Motif is a prerequisite
for running {\tt SLIME}.  Configuration is automatic for SPARCstations
running Solaris 2.4 or later or ALPHA/AXP-based DEC workstations
running OSF/1.  {\tt SLIME} should also work on other Unix-based
systems that have X11R5, Motif 1.2 and the Display PostScript
extension to X. The Display PostScript requirement will be removed in
the next major release (expected to be available in the first quarter
of 1997) which will make it easier to port to additional Unix systems.
This release should also have facilities for plotting and fitting
closure-phase data.

Problem reports or comments concerning {\tt SLIME} should be sent to
Chris Flatters ({\tt cflatter@nrao.edu}) directly rather than using
the {\tt GRIPE} system or the designated AIP program.

\subsubsection{Model-fitting --- 2 - {\tt OMFIT}}

{\tt OMFIT} is an advanced model-fitting task that works directly on
$UV$ data and is far more flexible than {\tt UVFIT}.  {\tt OMFIT} can
fit to an arbitrary number of model components, with each chosen from
a small, but growing, pool of available component types.  The
available component types include several that model polarized flux
and at least a few that allow for multi-spectral fitting.  {\tt OMFIT}
can, optionally, perform self-calibration of the data, which is
offered in two flavors.  Simple self-calibration allows a single
amplitude and/or phase to be determined per antenna for each solution
interval while multiple self-calibration allows a separate
amplitude/phase self-calibration solution for each model component.
Questions, comments, and suggestions should be sent directly to Ketan
Desai ({\tt kdesai@nrao.edu}).

\subsubsection{Elevation interpolation}

A new task {\tt ELINT} is available in this release. The task {\tt
CLCAL} allows the interpolation over time of calibrator antenna gains
to target sources. If the elevation difference between the calibrator
and target sources is large, (either due to a large physical or time
separation), significant errors can result, especially at high
frequencies where there is a strong dependence of antenna gain on
elevation.  The task {\tt ELINT} both solves for the gain dependence
on elevation, and interpolates the required corrections into the {\tt
SN} table.  In addition, {\tt ELINT} solves for the flux densities of
the calibrators used, assuming the flux density of the first
calibrator is known.

The input data is an {\tt SN} table resulting from a preliminary
calibration using a set of selected calibrators. The task fits a given
type of fitting function to these data.  This functional form is then
used to interpolate gain values for the target sources. The fitting is
done independently for each antenna, each IF, and each polarization
(Stokes).

The simplest mode of operation is to determine the gain dependence
using only a single calibrator whose elevation range matches or
exceeds that of the target sources.  In many situations however, the
elevation range covered by a single calibrator is not sufficient,
while that of all, or many calibrators, is.  In this case, {\tt ELINT}
can solve for {\it both} the elevation gain dependence and the flux
density ratios between the calibrators, assuming the first-named
calibrator is the flux density reference. It then determines the
corrected flux densities of the calibrators.

The result of fitting can be displayed on the TV or recorded in a plot
file.  The average (for all selected antennas, IFs, and Stokes) found
voltage factors, (MEANFACTOR), original flux densities of selected
calibrators (FLUXOLD) and corrected flux densities of the calibrators
(FLUXNEW) are displayed.  The determined correction of the antennas'
gain vs.\ elevation is written in an output {\tt SN} table for all
selected target sources.

Two types of fitting function are provided: a third degree polynomial
({\tt OPCODE = 'POLE', 'POLZ', 'VLBI'} and {\tt 'VLBN'}) and secant z
({\tt OPCODE = 'SECZ'}).  Only amplitude fitting ({\tt OPTYPE = 'AMP
'}) has been installed now, but others are planned in the future.


\subsubsection{Bandpasses}

Bandpass interpolation in time has be reinstated in the calibration
system and is available in all tasks that support the {\tt DOBAND}
adverb (e.g. {\tt IMAGR}, {\tt SPLIT}). The current options, {\tt
DOBAND=1} (mean bandpass) and {\tt DOBAND=2} (nearest BP in time)
remain unchanged but {\tt DOBAND=3} (linear interpolation) has been
re-activated. A new option {\tt DOBAND=4} which uses weighted
two-point interpolation is also available. The BP entries are now
stored in a cache buffer rather than intermediate scratch files and
all interpolation modes treat flagged BP entries correctly. {\tt
DOBAND} interpolation modes 2 through 4 consider BP record flagging on
a record by record basis. New experimental BP interpolation modes 5
through 7 mirror the lower numbered options (2-4) but interpolate BP
entries and check flagging for each IF and polarization separately.

\subsubsection {{\tt FILLM} and the {\tt TY} table}

Until April 30, 1996, the VLA online system only wrote so-called {\it
Nominal Sensitivities} to the VLA archive tape. {\tt FILLM} writes
these values into the {\tt TSYS} column of the {\tt TY} extension file
belonging to the \AIPS\ $UV$ data set. Starting May 1, 1996, the VLA
online system also writes front-end and back-end system temperatures
(in degrees Kelvin) to tape. {\tt FILLM} was modified to read these
system temperatures, and to write either the front-end or the back-end
temperature to the {\tt TANT} column of the {\tt TY} table, which was
previously not used for VLA data. Default is back-end $\rm T_{sys}$
for P- and 4-band, and front-end $\rm T_{sys}$ for all other
bands. This default can be overridden using {\tt CPARM(2)} in {\tt
FILLM}.

%\subsubsection{GPS ionospheric data}
%
%{\tt LDGPS} is a new task that reads GPS ionospheric data from an
%ASCII file and writes it to the newly created {\tt GP} table attached
%to a $UV$ data file. This table is used by a new task {\tt GPSDL}
%which fits a simple, local model of the ionosphere to GPS data loaded
%by {\tt LDGPS} and uses this model to generate phase corrections for
%the excess path in the ionosphere as well as ionospheric Faraday
%rotation. This data path can be used by compact arrays equipped with a
%nearby GPS receiver.

\subsubsection{{\tt FXVLA} - 1996 leap second at the VLA}

This is a special purpose task written to correct a UT1 interpolation
error affecting data between 30 December 1995 and 5 January 1996. This
error was caused in the on-line system by interpolating over the leap
second discontinuity. This task will be retained in the standard
\AIPS\ distribution in case the affected data are reduced at some
point in the future. All users affected by this error have been
notified individually. For specific questions regarding {\tt FXVLA},
users should contact Michael Rupen ({\tt mrupen@nrao.edu}) or Athol
Kemball ({\tt akemball@nrao.edu}).


\subsubsection{Miscellaneous \uv\ task improvements}

\begin{description}

\myitem{\tt LTESS} {\tt LTESS} had the cutoff for the primary beam
hard-coded at $7\%$ but was modified to allow the user to
specify the cutoff. It still defaults to $7\%$ however.

\myitem{\tt APCAL} Null characters were stripped from antenna names to
increase the robustness of the calls to the {\tt KEYIN} routines, and
handling of source selection of the type {\tt "-SOURCE"} was improved.

\myitem{\tt SETJY} New polynomial coefficients, derived by Perley \&
Taylor from VLA flux density monitoring in 1995, were added, and a
rounding error introduced in SETJY for the Perley 1990 coefficients
was corrected. This error, for 3C286, ranged from $0.5\%$ to $1.9\%$
from P-band to Q-band.

\myitem{\tt CALIB} An error in the way {\tt CALIB} dealt with dual
polarization data was corrected. It sometimes occurred when one
polarization was missing throughout a solution interval. In the fit,
polarization-independent weights were used, which at times led to very
high gains in the missing sense of polarization. This in turn biased
the mean gain modulus (MGM), if computed. Errors in the MGM were
cumulative over several iterations, and sometimes significant.

\myitem{\tt UVLSF} {\tt BCHAN} and {\tt ECHAN} selection were applied to the
input when phase shifting and fitting the baseline.  Since they only
apply to output, the full loop is now done on input.

\myitem{\tt FILLM} {\tt FILLM} allows for a certain difference in
frequency/velocity before it assigns different {\tt FQID}'s. For some
extreme cases (sources at opposite ends of the sky) the old limit was
not sufficient; it is increased now.

\myitem{\tt SNCOR} Two new features were added to {\tt SNCOR}: 1) {\tt
OPCODE='PCOP'} allows solutions to be copied from one polarization to
another with the copy direction controlled by {\tt SNCORPRM(1)}; 2)
{\tt OPCODE='PNEG'} will flip the sign of the gain phase for all
selected solutions.  Both features are sometimes required in line
polarization calibration. Also added the capability of {\tt SNCOR} to
work on single source files.

\myitem{\tt POSSM} When {\tt NCOUNT=0}, and the user requested
multiple IFs to be plotted together ({\tt APARM(9)=1}), only the first
generated plot had the proper range of IFs plotted. This has been
fixed now. The functionality of {\tt APARM(8)} was extended: if {\tt
APARM(8)=1} the total power is plotted for selected antennas; if {\tt
APARM(8)=5} both amplitude and phase are plotted for selected antennas
and polarization. This allows cross-polarized autocorrelation data to
be examined.

\myitem{\tt SPFLG} Added the possibility to load only total-power
spectra for a first pass at data flagging. To this end, the
interpretation of {\tt DPARM(2)} has been changed as follows: {\tt
DPARM(2)=0} gives only cross-power spectra, {\tt DPARM(2)=1} gives
cross- and total- power spectra, {\tt DPARM(2)=2} gives {\it only}
total-power spectra. The ability to customize the reasons that {\tt
SPFLG} attaches to flagging commands entered into the {\tt FG} table
was added. To each flagging command stored in the {\tt FG} table is
attached an optionally customized string that will be recorded as the
reason for that flagging command.

\myitem{\tt VLACALIB} In this procedure the user now can specify the
version of the output {\tt SN} table using the {\tt SNVER}
adverb.

\myitem{\tt CLCOR} The amplitude factor used for the opacity
correction for {\tt OPTYPE = 'OPAC'} was corrected. The amplitude
factor was the square of what it should have been.

\myitem{\tt SMOOTH} Treatment of the adverb {\tt SMOOTH} and the
variable {\tt DOSMTH} in many of the tasks supporting was
streamlined.

\end{description}
\vskip -10pt %%% formatting, get STALIN entry on this page
\subsection{Image analysis and display}

\subsubsection{Image fitting}

Two dimensional Gaussian fits are used in astronomy for accurate
measurements of source parameters such as central position, peak flux
density and angular size. In many cases the estimation of the errors
is as important as the values of the parameters. The main drawback in
the error handling in both {\tt JMFIT} and {\tt IMFIT} was that both
tasks only used one set of expressions for the errors for all ratios
of the sizes of the fitted component and the beam size. Instead,
different formulae, depending on the beam size relative to the fitted
component size, should have been used.  The theory is presented in
\AIPS\ Memo \#92, {\it Errors in two dimensional Gaussian fits}, by
L. Kogan. The ideas discussed in this memo were implemented into {\tt
JMFIT}. The old version of {\tt JMFIT} was confusing as it provided
two sets of errors based on the rms obtained from the residual map and
from the data itself. Only one set of errors based on the rms obtained
from the data itself has been retained. A new adverb has been added to
allow the user to force a pre-determined rms of samples to be used in
the error calculations.  These changes make {\tt JMFIT} the task of
choice; {\tt IMFIT} is not strongly recommended.  There are plans to
implement a similar treatment of errors in {\tt SAD}.
\vskip -10pt %%% get STALIN on this page
\subsubsection{Miscellaneous analysis and display improvements}

\begin{description}

\myitem{\tt TVRGB} For $\tt DOOUT \> 1.5$, fully blanked pixels are
converted from black to white (transparent).

\myitem{\tt LWPLA} Added $11\times 17$ paper and user-dimension paper
to the output options.

\myitem{\tt PRTTP} A formatting error, causing the reference pixel number to be multiplied by 10, was corrected.

\myitem{\tt SAD} Changed {\tt SAD} to use {\tt GAUSPS} to avoid
precision problems in converting VLBI-scale component sizes to angular
sizes and to consolidate code with the improvements in {\tt
IMFIT}/{\tt JMFIT}.  {\tt SAD} was suffering from the same problems
that were fixed in {\tt IMFIT}/{\tt JMFIT}: milli-arcsecond-scale
components tended to be set to zero width.

\myitem{\tt STALIN} This verb, useful for erasing parts of a data
set's history file, has now been modified to handle much larger {\tt
HI} files than before.

\end{description}

\subsection{VLBI-related developments}

Although most \AIPS\ tasks are dual-purpose and many of the
developments listed above are applicable to VLBI and VLA data, we list
some enhancements here that are more specific to VLBI users.

\subsubsection{Data Loading --- {\tt FITLD}}

{\tt FITLD} has been upgraded in this release to deal with new
correlator developments and to add new user functionality. This
includes the ability: i) to select on IF using new adverbs {\tt BIF}
and {\tt EIF}; ii) to deal with data from multiple subarrays; and iii) to
write data with baseline-dependent integration times.

The VLBA correlator can now process data in multiple subarrays but
cannot label the subarrays at correlation time. If {\tt FITLD} detects
a subarray condition the user is informed and the data will be written
without an {\tt NX} or {\tt CL} table. Further processing of the data
is discussed under the paragraph concerning {\tt USUBA} below. The
correlator can now handle baseline-dependent integration times, as
part of a recent upgrade in support of Space VLBI. These data will
have a sort order of {\tt '**'} due to the multiple dump rates from
the individual baselines. Further software development within \AIPS\
will address this problem in future releases. {\tt FITLD} also now
allows the user to control whether the large {\tt VT} table,
containing tape statistics, will be copied to the output file.

{\it As part of the new support for baseline-dependent integration
times, the amplitude scaling of data within {\tt FITLD} and the VLBA
correlator has changed. VLBA users whose data are correlated after
mid-November 1996 will need to obtain the \RELEASENAME\ \AIPS\
release, containing the new release of {\tt FITLD}, for full
compatibility. If data correlated after this date are loaded with
earlier versions of {\tt FITLD} the global scaling of the final maps
will need correction}. This will be fully described in a forthcoming
\AIPS\ memorandum, but further information may be obtained from Phil
Diamond ({\tt pdiamond@nrao.edu}) or Athol Kemball ({\tt
akemball@nrao.edu}).

\subsubsection{Support for VLBA subarrays - {\tt USUBA}}

As discussed above, the VLBA correlator now supports multiple
subarrays. Immediately after loading the data using {\tt FITLD},
subarray identification and labeling needs to be performed using task
{\tt USUBA}. This task has been substantially re-written to support
this development in the VLBA correlator and can be operated in three
modes: i) automatic subarray identification; in this case an algorithm
is used which minimizes the total number of subarrays and maximizes
subarray continuity; ii) multiple subarray definition in an external
text file under full user control; and iii) selection of an individual
subarray through the input adverbs. These options allow flexible
subarray assignment under user control. For further information please
contact Athol Kemball ({\tt akemball@nrao.edu}).

\subsubsection{Baseline-based fringe-fitting}

Several enhancements have been made to baseline-based fringe fitting
within \AIPS\, as currently implemented in tasks, {\tt BLING} and {\tt
BLAPP}. These tasks were primarily developed for Space VLBI but have
general VLBI application.

It is now possible in {\tt BLING} to divide a model into the uv-data
before fringe-fitting. In addition, the Cotton-Schwab
baseline-stacking algorithm has been implemented to add in data from
indirect baselines for greater sensitivity.  The handling of
off-center windows has been changed to allow greater padding in the
FFT's and removes the need for non-linear least square fitting to
refine the fringe solutions. The speed-up settings available to the
user now restrict the FFT padding in the delay and rate solutions
separately.

Interpolation of unevenly spaced data in {\tt BLAPP} has been
improved, and uses the fringe acceleration term, if available. The
interpolated solution can be passed back to {\tt BLING} for use in
subsequent fringe-fit iterations. Several enhancements to the
robustness of {\tt BLING} and {\tt BLAPP} have also been made.

%%% shortened 96.11.25 PPM to avoid really bad page break
A new task {\tt BSPRT} allows printing of the
baseline-based fringe solution {\tt BS} tables produced by {\tt
BLING}. For further information on baseline-based fringe-fitting
please contact Chris Flatters ({\tt cflatter@nrao.edu}).

\subsubsection{Ground-phasing for Space VLBI fringe-fitting}

A new task has been developed to separately phase up the ground radio
telescopes in an orbiting VLBI experiment to produce a single
synthesized baseline between the ground array and the orbiting
antenna. This task, {\tt GPHAS}, allows data selection, calibration
and model division and is expected to play an important role in
fringe-fitting Space VLBI data. The synthesized baseline can be
fringe-fit using conventional tasks such as {\tt FRING} and {\tt BLING},
but this technique is expected to lead to a significant improvement in
sensitivity.  The baseline is written as a normal \AIPS\ uv-file and
can be edited and examined using standard \AIPS\ tasks. For further
details please contact Ketan Desai ({\tt kdesai@nrao.edu}).

\subsubsection{Data display - {\tt VPLOT} and {\tt CLPLT}}

 Several library routines have been developed to compute $(u,v,w)$
coordinates for an orbiting antenna defined by a set of standard Keplerian
elements, and to calculate model visibilities accordingly. These
routines have been implemented within {\tt VPLOT} and {\tt CLPLT} to
plot continuous model information across intervals for which there is
no correlated data. These two tasks have also been upgraded to plot
combined models from multiple image fields, as is required in
gravitational lens imaging and have also been upgraded to improve
subarray selection, as required by recent developments in the VLBA
correlator.

\subsubsection{Pulse calibration - {\tt PCCOR}}

The algorithm in {\tt PCCOR} has been enhanced to better resolve the
$2 \pi n$ ambiguity for IF channels that are non-contiguous in
frequency. Support for multiple subarrays has also been added, and a
new feature allows the cable delay to be switched off, which is useful
in geodesy applications.

\subsubsection{Coherence time estimation - {\tt COHER}}

A new task {\tt COHER} has been implemented which measures the
coherence time of uv-data. The coherence time is estimated for each
interval by comparing the ratio of successive vector and scalar
averages as a function of increasing integration time. The loss factor
or cutoff can be defined by the user. The task allows uv-data
selection and provides various output formats for the measured
coherence properties. This task is useful in VLBI fringe-fitting in general
but was written primarily with Space VLBI data in mind.

\subsubsection{Data simulation and software testing}

Two new tasks, {\tt DTSIM} and {\tt DTCHK}, have been implemented in
this release, but are the subject of on-going development. The task
{\tt DTSIM} generates fake uv-data for the dual purpose of testing the
correctness of \AIPS\ software, and the investigation of the
performance of reduction algorithms for data over a range of signal to
noise ratios. The task generates pre-fringe-fitting errors in a
variety of forms, adds noise under user control and computes (u,v,w)
coordinates to allow imaging tests. It is currently used in the
testing of Space VLBI reduction software and as part of a
complementary test suite for general VLBI tasks commonly used within
\AIPS. For further information please contact A. Kemball ({\tt
akemball@nrao.edu}) or K. Desai ({\tt kdesai@nrao.edu}).

\subsubsection{Contributed polarization calibration software}

The feed calibration task {\tt LPCAL} has been enhanced by Kari
Lepp\"anen (JIVE) to allow the use of a linearly-unpolarized feed
calibrator. No input model need be specified using {\tt IN2NAME} in
this case. The run file {\tt CRSFRING} has also been implemented which
determines cross-polarized delay and phase offsets. This procedure
uses task {\tt BLAVG} and is thus distinct from {\tt CROSSPOL},
although their overall purpose is the same.

\subsubsection{Changes to {\tt MK3IN}}

{\tt MK3IN} has been updated to include changes in the fractional
bit-shift correction (FBS), as supplied by Kari Lepp\"anen (JIVE).
Amplitude FBS corrections are required for high dynamic range imaging
as revealed by recent imaging tests using the EVN. It was also found
that the band center for the existing FBS phase correction was in
error by one half of a frequency channel; this has also been
corrected. {\tt MK3IN} will also ignore LO offsets, under user
control, but no corrections are currently applied for this effect.


\subsection{Documentation, on-line help, and user support}

\subsubsection{Designated AIP program}

We reinstated a modified form of the designated AIP program. \AIPS\
user support can now be obtained by the following methods:
\begin{description}
\vspace{-10pt}

\item{ 1.} {E-mail to {\tt aipsmail@nrao.edu}. This account is checked
several times a day, and messages are forwarded within the AIPS group
as appropriate.}

\item{ 2.} {Submit a gripe. This is usually done from within
AIPS. Newer versions of \AIPS\ ({\tt 15JAN96} and later) will
automatically send an e-mail message to NRAO. The gripe system should
be used for less urgent matters, such as suggestions for improvement.}

\item{ 3.} {Contact the AIPS group member currently designated to
provide user support. This listing is available on the WWW via}

\end{description}
\begin{center}
\vskip -10pt
{\tt http://www.cv.nrao.edu/aips/d\_aip.html}
\vskip -10pt
\end{center}

The ``designated AIP'' program now covers general AIPS user support,
including VLBI. Users may wish to contact individual members of the
AIPS group directly if their question is of a specialized nature, and
they know who in the AIPS group is the specialist in that area.

\subsubsection{\AIPS\ access to HTML help}

The experimental pseudoverb {\tt XHELP} was added to \AIPS. {\tt
XHELP} is intended to provide direct access to HTML versions of
cookbook-level help.  {\tt XHELP FOO} will look for a file named {\tt
FOO.html} in area {\tt AIPSHTML} and automatically loads it into a
browser (only Netscape is supported at the moment) which will be
started if not already running.  If {\tt FOO.html} does not exist then
\AIPS\ will check for {\tt FOO.HLP} in {\tt HLPFIL} and load it into
the browser as plain text.  If the {\tt .HLP} file doesn't exist then
a default index page ({\tt INDEX.html} from {\tt AIPSHTML}) will be
loaded.



\section{Improvements Primarily for Programmers in \OLDNAME}

\subsection{Changes affecting \AIPS\ Managers}

\subsection{Programming considerations}

\subsubsection{Memory usage}

A number of arrays in \AIPS\ were still dimensioned as the product of
{\tt MAXCHA} (maximum number of channels) and {\tt MAXIF} (maximum
number of IFs), rather than the maximum product of these parameters as
defined by {\tt MAXCIF}. This was corrected throughout the bandpass
system, and in many other tasks. Important common blocks in {\tt
DSEL.INC} have also been affected by these changes, which are part of
a general initiative to reduce memory usage. The buffer requirements
for {\tt TABIO} have been reduced by reading long table records
directly. Note that this has led to changes in the calling sequences
of several subroutines including {\tt TABINI}, {\tt TABSRT}, {\tt
TABMRG}, {\tt GETCOL} and {\tt PUTCOL}, including others. There have
also been changes in the calling sequence to several Z-routines,
including {\tt ZDCHIN}, {\tt ZCREAT}, {\tt ZEXIST}, {\tt
ZCMPRS}. Other subroutines affected include {\tt UCMPRS}, {\tt
CQMAKE}, {\tt DSMEAR} and {\tt MAKTAB}, amongst others. Local tasks
developed with libraries from earlier releases will be to be updated
accordingly.

New Z-routines to allow dynamic memory allocation have been
implemented ({\tt ZMEMRY} and {\tt ZMEMR2}).

\subsubsection{OOP routines and compressed data}

The OOP package was modified to write compressed data.  If a $UV$ file
is created by calling {\tt OUCREA} then that file will be compressed
if {\tt ISCOMP} is true in the corresponding {\tt UVDESC} object ({\tt
OUCREA} is usually called indirectly through {\tt OUVOPN}).  {\tt
OUVOPN} now detects when a file to be written is compressed and
records the location of the {\tt WEIGHT} and {\tt SCALE} parameters
(in {\tt UVWSPT}) for use by {\tt UVWRIT}.  {\tt OUVCLO} will also
check on this information and make sure that the header is correct for
compressed or uncompressed data to protect against destructive changes
to the {\tt UVDESC} object.

{\tt OUVCLN} now checks for a virtual {\tt DOUVCOMP} keyword attached
to the input {\tt UVDATA} object and compresses the output file if
this exists and has the value {\tt .TRUE.}.  {\tt OUVSCR} (in {\tt
UVDATA.FOR}), {\tt UV2SCR} ({\tt UVUTIL.FOR}) and {\tt CP2SCR} ({\tt
UVUTIL.FOR}) now take an extra argument that indicates whether scratch
data should be compressed.  Calls to the scratch file routines have
been updated in {\tt QUVUTIL.FOR} and tasks {\tt MAPBM}, {\tt FRCAL}
and {\tt SCMAP}.

{\tt UVDSCP} (in {\tt UVDESC.FOR}) now regards random parameter names
as size related and refuses to copy them.  Note, however, that {\tt
UVDCOP} copies the input file's compression state to the output file.

A new UVDATA routine, {\tt OUVPAK}, is responsible for adjusting the
descriptive information for uv data to reflect compression (\ie\
changing {\tt NAXIS(1)} and adding or removing the {\tt WEIGHT} and
{\tt SCALE} parameters).  This is principally used internally by
UVDATA but may be called by client programs (an obvious use is to
patch up descriptive information for a compressed file to reflect
uncompressed data: see {\tt UV2MS} for an example).

Task {\tt FIXWT} will now write compressed scratch files if {\tt
DOUVCOMP} is true and task {\tt UV2MS} will now append data to
compressed multi-source files and will create new multi-source files
in compressed format if {\tt DOUVCOMP} is true.  Tasks {\tt MULIF} and
{\tt SPECR} will now write compressed data if their input data was
compressed: strictly speaking, this is incorrect behavior since it is
not conditioned on {\tt DOUVCOMP} but has not been fixed yet since
there are no adverse effects from this.


\subsubsection{Tape handling}

A problem that was preventing Solaris 2.5/SunOS 5.5 systems from using
magnetic tapes (of any kind: exabyte, DAT, or 9-track) was solved.  If
the tape unit is opened with the non-blocking I/O bit set, the SCSI
driver refuses to write EOF marks.  Everything seems to work if the
{\tt open()} statements are done without the {\tt O\_NONBLOCK}
attribute (though DATs under 5.4 have not been tested; exabytes seem
fine with this fix under 5.4 or 5.5, as do DATs and 9-tracks under
5.5).

Exabytes do not yet work under Linux, but a remote Exabyte on another
architecture can be used normally. Work to fix this problem is in
progress. DAT drives appear to work fine under Linux.

\subsubsection{\AIPS\ errors using NFS version 3}

Some \AIPS\ errors have recently been reported using NFS~v.3 on AIPS
NFS data servers running SunOS 5.5 (Solaris 2.5). The form of the
error is that tables attached to \AIPS\ files on NFS-mounted data
areas are not always sorted correctly. This affects tasks such as {\tt
CLCAL} and {\tt TASRT}, although no explicit \AIPS\ error is triggered
and no error message is reported. In this case {\tt CLCAL} will
produce an incomplete and missorted calibration table which will
result in missing data. This problem does not affect locally mounted
disks. The problem was investigated in consultation with the site
where it occurred and was found to be corrected by installing SunOS
5.5 (Solaris 2.5) patch 103226-07, which fixes several NFS errors. The
problem was introduced by an earlier version of the same patch, which
explains why sites that did not have the patch installed at all did
not see the problem.  Users are advised to be aware of this potential
problem, and to install the necessary Sun patch if this problem is
present on their system. The error is repeatable and can easily be
verified by testing {\tt CL} table sorting using task {\tt TASRT}.  In
conducting this test the {\tt CL} table needs to be located on the
NFS-mounted disk and adverb {\tt BADDISK} should be set such that all
scratch files are written to the NFS disk also. For further
information please contact A.~Kemball (NRAO-Socorro) at {\tt
akemball@nrao.edu}.


\section{AIPS Publications and the World-Wide Web}

     The {\it World-Wide Web\/} (WWW) is a method for sending and
receiving hypertext over the Internet network and has been made easy
to use by clients such as {\it NCSA Mosaic, Netscape, Arena,\/} and
{\it Lynx\/}.  NRAO is among the many institutions which now offer
informative Web pages and networks of additional information.  The
NRAO ``home'' page is at the Universal Resource Locator (URL) address
\begin{center}
\vskip -10pt
{\tt http://www.nrao.edu/}
\vskip -10pt
\end{center}
The \AIPS\ group home page may be found from the NRAO home page or
addressed directly at URL
\begin{center}
\vskip -10pt
{\tt http://www.cv.nrao.edu/aips/}
\vskip -10pt
\end{center}
This page points at basic information, news items about \AIPS, the
PostScript text of recent \AIPSLETTER s, patch information for all
releases after {\tt 15JAN91}, the latest \AIPS\ benchmark data from
various computer systems, copies of {\tt CHANGE.DOC} for every release
since {\tt 15JAN90}, {\it all} relevant \AIPS\ Memos, {\it every}
chapter of the \Cookbook, and all recent quarterly reports to the
\hbox{NSF}.  There is even a tool to let you browse the {\tt 15OCT96}
versions of all help/explain files.  We recommend that you check this
URL occasionally since it changes when new software patches, revised
\Cookbook\ chapters, and new \AIPS\ Memos are released.

There is one new \AIPS\ Memo with this release:
\begin{center}
\vspace{-6pt}
\begin{tabular}{ccl}
\hline
Memo  &        Date   & Title and author  \\
\hline\hline
  92 & 96/09/30 & Errors in two-dimensional Gaussian Fits \\
     &          & \qquad L. Kogan, NRAO \\
\hline
\end{tabular}
\end{center}
\vspace{-6pt}
These memos are available through the WWW pages.  Since some Memos
are not available electronically and others do not yet have computer
readable figures, you may wish to write for a paper copy of these.  To
do so, use an \AIPS\ order form or e-mail your request to {\tt
aipsmail@nrao.edu}.  If you cannot use the Web, you can still use
\ftp\ to retrieve the Memos, \Cookbook\ chapters, etc.:
\begin{description}
\vspace{-10pt}
\item{ 1.} {\tt ftp aips.nrao.edu}  (currently on {\tt 192.33.115.103})
\item{ 2.} Login under user name anonymous and use your e-mail address
           as a password ({\it yourname}{\tt @} will do; ftp will
           fill in the machine you are using).
\item{ 3.} {\tt cd pub/aips/TEXT/PUBL}
\item{ 4.} {\tt get AAAREADME} and read it for lots more information.
\item{ 5.} {\tt get AIPSMEMO.LIST} for a full list of \AIPS\ Memos.
\end{description}

\section{Patch Distribution}

As before, important bug fixes and selected improvements in
\RELEASENAME\ can be downloaded via the Web at:

\begin{center}
\vskip -10pt
{\tt http://www.cv.nrao.edu/aips/15OCT96/patches.html}
\vskip -10pt
\end{center}

Alternatively one can use {\it anonymous} \ftp\ on the NRAO cpu {\tt
aips.nrao.edu} (currently located on {\tt baboon} which is {\tt
192.33.115.103}).  Documentation about patches to a release is placed
in the anonymous-ftp area {\tt pub/aips/}{\it release-name} and the
code is placed in suitable subdirectories below this. Information on
patches and how to fetch and apply them is also available through the
World-Wide Web pages for \hbox{\AIPS}.  As bugs in \RELEASENAME\ are
found, the patches will be placed in the {\tt ftp}/Web area for
\hbox{{\RELEASENAME}}.  No matter when you receive your \RELEASENAME\
``tape,'' you must fetch and install these patches if you require
them.

\vfill\eject

\section{Obtaining \AIPS\ under the GNU General Public License}

We have decided to make \AIPS\ available via anonymous ftp under the
GNU General Public License, the meaning of which was spelled out in
the {\tt 15JUL95} \hbox{\Aipsletter}.  The installation of \AIPS\ will
now proceed something like the following example:

We assume that you have created an account for \AIPS\ with a root
directory called \hbox{{\tt /AIPS}}.  Then do
\vskip -10pt
\begin{verbatim}
home_prompt<601> cd /AIPS
home_prompt<602> ftp aips.nrao.edu
Connected to baboon.cv.nrao.edu.
220 baboon FTP server (Version wu-2.4(1) Fri Apr 15 12:08:14 EDT 1994) ready.
Name (aips.nrao.cv:johndoe): anonymous
331 Guest login ok, send your complete e-mail address as password.
Password: johndoe@nrao.edu
230- This is the National Radio Astronomy Observatory ftp server for the
230- AIPS, AIPS++, and FIRST projects.  Your access from primate.cv.nrao.edu
230- has been logged, and all file transfers will be recorded.  If you do not
230- like this, type "quit" now.  Counting you there are 1 (max 20) ftp users.
230-
230- Current time in Charlottesville, Virginia is Mon Jan 18 10:18:46 1996.
230-
230-
230-Please read the file README
230-  it was last modified on Wed Mar  8 14:01:24 1995 - 316 days ago
230 Guest login ok, access restrictions apply.
ftp> cd aips/15OCT96
250 CWD command successful.
ftp> get README
200 PORT command successful.
150 Opening ASCII mode data connection for README (nnnn bytes).
226 Transfer complete.
local: README remote: README
nnnn bytes received in T seconds (5 Kbytes/s)
ftp> get INSTALL.PS
200 PORT command successful.
150 Opening ASCII mode data connection for INSTALL.PS (mmmmm bytes).
226 Transfer complete.
local: INSTALL.PS remote: INSTALL.PS
mmmmm bytes received in TT seconds (5 Kbytes/s)
ftp> binary
200 Type set to I.
ftp> hash
Hash mark printing on (8192 bytes/hash mark).
ftp> get 15OCT96.tar.gz
200 PORT command successful.
150 Opening ASCII mode data connection for 15OCT96.tar.gz ( bytes).
226 Transfer complete.
local: 15OCT96.tar.gz remote: 15OCT96.tar.gz
mmmmm bytes received in TTTTT seconds (5 Kbytes/s)
ftp> quit
221 Goodbye.
\end{verbatim}
\vskip -10pt
You should type in your full e-mail address (not {\tt
johndoe@nrao.edu}) at the password prompt.  The {\tt hash} command is
optional and may be inappropriate in some versions of ftp; it does
give a useful indication of progress in the long {\tt get} in most
versions.  If you do not have the GNU file compression code ({\tt
gzip}), you should {\tt get 15OCT96.tar}.  Out ftp server will
uncompress the gzipped file automatically.  (It would be around 3
times faster if you had {\tt gzip}.)

At this point you should read the {\tt README} file to review the
latest changes, if any, affecting your installation of \hbox{\AIPS}.
You should print out the {\tt INSTALL.PS} PostScript document and
read at least its overview section.  To create the rest of the {\tt
/AIPS} directory tree, and fill it with the \AIPS\ source code
\vskip -10pt
\begin{center}
\begin{tabular}{l}
   {\tt cd /AIPS} \\
   {\tt zcat 15OCT96.tar.gz | tar xvf -} \\
\multicolumn{1}{c}{or} \\
   {\tt tar xvf 15OCT96.tar}
\end{tabular}
\end{center}
\vskip -10pt
depending on whether you fetched the source file with compression or
without.

If you want to get the binary version(s) of \AIPS, you should read the
{\tt README} file for further directions.  They will tell you about a
procedure to run from the {\tt INSTEP1} installation procedure and/or
at a later time which will initiate a second ftp session to fetch the
appropriate contents from the {\tt \$LOAD}, {\tt \$LIBR}, {\tt MEMORY},
{\tt BIN}, and {\tt DA00} areas.  You may run this procedure more than
once if you need to fetch binaries for more than one architecture.
You may also have to run portions of this procedure ``by hand'' if you
encounter reliability problems with the network.

You will then have to run the {\tt INSTEP1} procedure, as usual, to
tell your \AIPS\ about your computer environment.  A new part of {\tt
INSTEP1} is its offer to assist you in ``registering'' your copy of
\hbox{\AIPS}.  It will help you complete a registration form and will
even e-mail it to us if you want.  When we get a registration request,
we will enter your information in our user data base and reply with
instructions and registration numeric ``keys'' which you may use to
complete the registration process (using {\tt SETPAR} and \hbox{{\tt
SETSP}}).  This may seem cumbersome and onerous, but we have two
reasons for doing this.  The first reason is to provide us with
information about the use of \hbox{\AIPS}.  This information is useful
to us to justify, to management and funding agencies, our existence
and our need for more employees or computers or disk or whatever.  The
second reason is a concern about excessive demands on our employees'
limited time to provide assistance to sites in installing and running
the software.  If an excessive demand should arise, information from
the registration process will allow us to set priorities among the
different sites.  This registration is entirely optional.  We will use
transaction logging in ftp and, hence, know which sites have fetched
the code.  We will assume that sites which do not register are not
``serious'' in their use of \AIPS\ and we will be unable to provide
any assistance to unregistered sites (except, of course, to help them
register).  This means that unregistered sites will receive no
assistance in installing \AIPS\ and users at those sites will receive
no assistance in using \AIPS, including no printed literature.  All
serious sites are strongly encouraged to register since registration
statistics are used to determine the level of effort that NRAO can
provide for the Classic \AIPS\ project.  The statistics are also used
to obtain assistance from computer vendors.
%  All serious sites are strongly encouraged to register,
%even if they do not need assistance during installation, since
%registration statistics are used to determine the level of effort that
%NRAO can provide for the Classic \AIPS\ project.

As of the {\tt 15JUL95} release, \AIPS\ is available under the GNU
General Public License.  The short statement of this license is in
every \AIPS\ file, is available on-line via {\tt HELP GNU}, and was
given (once) in the {\tt 15JUL95} \hbox{\Aipsletter}.  You should have
received the GNU General Public License from several sources, most
notably GNU themselves with their {\tt emacs}, {\tt gcc}, and numerous
other software products.  Since \AIPS\ now applies that license to
itself --- and intends to import and use other GNU-licensed routines
--- we also include the full license text on-line via {\tt EXPLAIN
GNU} and, once, in the {\tt 15JUL95} \hbox{\Aipsletter}.

\section{\Cookbook\ Update Continues}

The previous release of \AIPS\ saw an extensive rewrite of the
\Cookbook. In the current release, a few further modifications were
made, including the addition of a short section on polarization VLBI
calibration.
\vfill\eject
\centerline{\hss\psfig{figure=FIG/AIPSORDER.PS,height=23.3cm}\hss}
\vfill
% The dang mailing label template goes here.  Where is it????
\end{document}
