% -*- latex -*-
%-----------------------------------------------------------------------
%;  Copyright (C) 1997
%;  Associated Universities, Inc. Washington DC, USA.
%;
%;  This program is free software; you can redistribute it and/or
%;  modify it under the terms of the GNU General Public License as
%;  published by the Free Software Foundation; either version 2 of
%;  the License, or (at your option) any later version.
%;
%;  This program is distributed in the hope that it will be useful,
%;  but WITHOUT ANY WARRANTY; without even the implied warranty of
%;  MERCHANTABILITY or FITNESS FOR A PARTICULAR PURPOSE.  See the
%;  GNU General Public License for more details.
%;
%;  You should have received a copy of the GNU General Public
%;  License along with this program; if not, write to the Free
%;  Software Foundation, Inc., 675 Massachusetts Ave, Cambridge,
%;  MA 02139, USA.
%;
%;  Correspondence concerning AIPS should be addressed as follows:
%;          Internet email: aipsmail@nrao.edu.
%;          Postal address: AIPS Project Office
%;                          National Radio Astronomy Observatory
%;                          520 Edgemont Road
%;                          Charlottesville, VA 22903-2475 USA
%-----------------------------------------------------------------------
%Body of AIPSletter for 15 October 1997

\documentclass[twoside]{article}

\usepackage{epsfig}
\usepackage{aipsletter}

\begin{document}

\begin{aipsletter}{October~15~1997}{15OCT97}{XVII}{2}{Chris Flatters}

\section{IMPORTANT}

The \AIPSLetter\ is available on the World Wide Web. If you are
currently receiving a paper copy, we encourage you to obtain future
letters from our home page at
\address{http://www.cv.nrao.edu/aips/}. We will announce new editions
when they are available. If you prefer to get your \AIPSLetter\
\emph{via} the Web, please send a notice to
\address{aipsmail@nrao.edu} requesting removal from the paper copy
mailing list.

\section{General Developments in \AIPS}

\subsection{Current Release}

The \thisrelease\ release of Classic \AIPS\ is now available.  It may
be obtained via \emph{anonymous} ftp or by contacting Ernie Allen at
the address given in the masthead.  \AIPS\ is now copyright \copyright
1995, 1996, 1997 by Associated Universities, Inc., NRAO's parent
corporation, but may be made freely available under the terms of the
Free Software Foundation's General Public License \hbox{(GPL)}.  This
means that User Agreements are no longer required, that \AIPS\ may be
obtained via anonymous ftp without contacting NRAO, and that the
software may be redistributed (and/or modified), under certain
conditions.  The full text of the GPL can be found in the
\texttt{15JUL95} \AIPSLetter. Details on how to obtain \AIPS\ under the new
licensing system appear later in this \AIPSLetter.

The next release of \AIPS\ will be \texttt{15APR98}.  It is possible
to get early access to this release by running a ``midnight job''; see
the \AIPS\ home page for further details.

\subsection{Space VLBI Support}

The \thisrelease\ release of \AIPS\ incorporates a number of revisions
and enhancements arising from our experiences in processing
observations made during the in-orbit checkout (IOC) of the HALCA
spacecraft.  Several Space VLBI enhancements are summarized in the
next section of this newsletter.  We recommend that \thisrelease\ be
installed at all \AIPS\ sites that will be used to reduce HALCA
observations.

\subsection{Staff Changes}

In September Gustaaf van Moorsel left the \AIPS\ group to concentrate on
AOC computing issues, and Tony Beasley took over as head of the \AIPS\
group. Eric Greisen returned to the \AIPS\ group after 18 months of work
on his experimental version, CVX. Eric's CVX version of \AIPS\ has been
successfully merged with \texttt{15OCT97} \AIPS\ and the combined version
will form the basis of \texttt{15APR98}.

\subsection{\AIPS\ Distribution}

A total of 148 copies of the \texttt{15APR97} release were
distributed, of which 79 were in source code form and 69 were
distributed as binary executables.  The table below shows the
breakdown of how these copies were distributed. This includes both
source code distributions and binary distributions.

\begin{center}
\begin{tabular}{|r|r|r|r|r|} \hline\hline
{ftp} & {8mm} & {4mm} & {ZIP} & {Floppy} \\ \hline
137   &    6  &    3  &    2  &       0  \\ \hline\hline
\end{tabular}
\end{center}

User feedback suggests that the distribution over operating systems
for installed versions of \texttt{15APR97} was as follows.

\begin{center}
\begin{tabular}{|l|r|r|r|} \hline\hline
{Operating System} & {No.} & \texttt{15APR97}  &  \texttt{15OCT96}  \\
                   &       & {\%}   &  {\%}  \\ \hline
Solaris/SunOS 5 &    255       & 66 & 46 \\
PC Linux        &     61       & 16 & 19 \\
HP-UX           &     23       &  6 &  4 \\
Dec Alpha       &     23       &  6 & 10 \\
SunOS 4         &     19       &  5 & 13 \\
SGI             &      2       &  1 &  5 \\
IBM /AIX        &      1       &  0 &  4  \\
Total           &    384       &    &     \\ \hline\hline
\end{tabular}
\end{center}

The distribution of \texttt{15APR97} has been rather lower than that for
\texttt{15OCT96}, perhaps reflecting the fact that \texttt{15OCT96}
had a longer lifetime than normal as \texttt{15APR97} was delayed by
several months.  These figures are affected by the percentage of
\AIPS\ users that register with NRAO.  We remind serious \AIPS\ users
that registration is required in order to receive user support.

\section{Improvements for Users in \thisrelease}

This release includes several improvements over \texttt{15APR97}.  The
most significant of these are described below.  See the
\texttt{CHANGE.DOC} file for more details.

\subsection{General Improvements}

\subsubsection{Support for Large Files}

Unix file systems have traditionally limited file sizes to a maximum
of 2 Gbytes ($2\,097\,152$ bytes).  Several of the more modern
varieties of Unix have removed this restriction but \AIPS\ has
continued to assume that addresses in files can be stored in a
standard 32-bit signed integer and has, therefore, been unable to take
advantage of this change.  This size restriction has proved awkward
for some types of data reduction that require large volumes of data
(eg. spectral-line mapping and space VLBI).

Starting with the \thisrelease\ release of \AIPS\ we have removed this
limit so that \AIPS\ can take advantage of large file facilities if
they are made available by the host operating system.  We have been
able to use files larger than 2 Gbytes under IRIX 6 (SGI) and Digital
UNIX 4 (DEC); this should also be possible under other operating
systems.  You should note, however, that filesystems may need to be
set up in a particular manner before they can be used to store these
large files: consult your system documentation to see what needs to be
done.

\subsubsection{Model Fitting}

The \task{JMFIT} task, which fits model components to \AIPS\ images
now has a stricter convergence criteria so that it is less likely to
position components at integer cell coordinates.  It has also been
modified to give better error estimates (see \AIPS\ Memo~97 for more
details).

\subsection{Van Vleck correction in {\tt FILLM}}

The correlation coefficient produced by a digital correlator such as
used by the VLA differs in a non-linear way from that produced by a
perfect analog correlator.  The VLA online system does not correct the
visibilities for this difference.  In most cases, measured correlation
coefficients are too small for the effect to produce noticeable
artifacts, but in cases of uncommonly high flux densities a correction
--- the so-called \emph{Van Vleck correction} --- may be needed.
Fig.~1 shows the correction factor as a function of the correlation
coefficient.  In order to apply this correction, several quantities
are required which are recorded on the archive tape but are not copied
to the AIPS headers.  Therefore, the latest stage at which this
correction still can be applied is when the data are being read from
tape to \AIPS\ using the \AIPS\ task {\tt FILLM}.

This has now been implemented in {\tt FILLM}; the Van Vleck correction
will be applied when the third bit in parameter {\tt CPARM(6)} is {\it
on}.  This is done by adding 8 to whatever value of {\tt CPARM(6)} the
user wants.  Fig.~2 shows an example of a strong continuum source
(Cygnus A) without (top panel) and with (bottom panel) the Van Vleck
correction applied.

The Van Vleck correction in {\tt FILLM} currently only works on
continuum data.  Results should always be inspected carefully. There
are cases (strong masers) where spectral line observations could
benefit from a Van Vleck correction as well, and we may implement this
at some future stage.

\begin{figure}
\begin{minipage}[b]{0.46\linewidth}
\epsfig{figure=LET97B1.EPS,width=\linewidth}
\caption{The Van Vleck correction factor applied to
the visibilities as a function of the correlation coefficient.  This
graph refers to the actual Cygnus A data shown in Fig.~2.  Note the
unusually high correlation coefficients present in these data.}
\end{minipage}\hfill
\begin{minipage}[b]{0.46\linewidth}
\centerline{\epsfig{figure=LET97B2.PS,height=6cm}}
\caption{Cygnus A observations without (top) and with
(bottom) the Van Vleck correction applied.  Observations with other
synthesis arrays confirm the spurious nature of the lobes at both ends
in the top panel.}
\end{minipage}
\end{figure}

\subsection{Improvements for VLBI (Including Space VLBI)}

\subsubsection{VLBA Correlator}

\task{FITLD} and \task{INDXR} will now transfer clock offsets and
atmospheric delay from VLBA \texttt{MC} tables to the initial
calibration table of a uv data set.  This improves the tracking of the
data reduction process for astrometry.

\subsubsection{Mk~3 Correlators}

New tasks, \task{M3TAR} and \task{TFILE} have been added to the system
that will load data from Mk~3 VLBI correlators which has been encoded
in tar archive files.  Thanks go to M.~Wunderlich for contributing
this code.

\subsubsection{Forced Scan Breaks in \task{INDXR}}

\task{INDXR} will now accept a list of times in a text file and will
ensure that scans do not cross these times.  This is primarily
intended for Space VLBI, where scans should not span times at which
clocks are reset at the ground stations since resetting the clock
introduces discontinuous changes in residual delay.

\subsubsection{Phase Calibration Data}

VLBA antennas as well as some other VLBI antennas are equiped with
pulse calibration system which provides the information necessary to
align the phases of different IFs and polarizations as a function of
time, removing the effects of any differences and fluctuations in the
instrumentation. The instrumental delay can can be as large as 1
microsecond and can therefore cause several turns across typical VLBA
band. With at least two tones in each band, it is possible to measure
and remove the instrumental delay. The main problem is the existence
of $2\pi$ ambiguities in the measured phase.

The \AIPS\ task \task{PCCOR} uses pulse calibration data stored in
an \AIPS\ \texttt{PC} table to remove instrumental delays from UV
data. Phase ambiguities are removed using the fact that phase of a
calibrator's visibilities as a function of frequency should coincide
with a straight line corresponding to a common delay for all IFs and
polarizations.

\task{PCCOR} uses only the two tones with the lowest and highest
freqquency within each IF.  Tests of \task{PCCOR} have shown that it
produces a good alignment of phases in IFs separated by as much as
$500\,\mathrm{MHz}$.

Pulse calibration (\texttt{PC}) tables may be loaded from text files
extracted from VLBA log files using \task{PCLOD} or directly from
FITS data files; the Calagary S2 correlator already incorporates pulse
calibration data in its data sets and the VLBA correlator will begin
adding them to data files in 1998.

\subsubsection{Coherence Times}

\task{COHER} will now produce estimates of the coherence time for data
that has not yet been fringe-fitted.  These estimates may be used to
set an upper bound for \texttt{SOLINT} in \task{FRING} or \task{BLING}.

\subsubsection{Astrometry}

Most tasks that support astrometry have been updated in this release,
as has the \texttt{ASTROMET.HLP} file which describes the recommended
process for reducing astrometic data in \AIPS.

A new POPS procedure \task{REFREQ} allows the reference frequency to
be changed within a band and may be used to change the reference
frequency for fringe-fitting to the centre of the band to improve the
isolation of delay and rate errors.

\subsubsection{Baseline-Oriented Fringe-Fitting}

The size of the delay-rate region search by \task{BLING} is no
longer limited by the size of the \AIPS\ pseudo-AP.  The full
ambiguity range may be now be searched for arbitrarily long solution
intervals and arbitrarily large numbers of channels --- subject to
machine limitations, of course.

\task{BLING} has also be tuned for greater speed and should now be
significantly faster than \task{FRING} in most circumstances (much of
this tuning was done prior to the release of \texttt{15APR97} but was
not mentioned in the \AIPSLetter).

\subsubsection{\task{UVFIX} and Space VLBI}

\task{UVFIX} has been revised to caluculate correct ($u, v, w$)
coordinates for baselines to an orbiting antenna.  This assumes that
the orbital elements are encoded in the antenna (\texttt{AN}) table
and are referenced to the J2000 equinox.

Corrections for differential aberration have also been added to
\task{UVFIX}.


\vfill
\section{Recent \AIPS\ Memoranda}

The following memoranda are available from the \AIPS\ home page.

\begin{tabular}{lp{6in}}
95 &	AIPS/AIPS++ Interoperability \\
   &	A.J.~Kemball \\
   &    August 29 1997 \\
   &    This note discusses the question of interoperability between
	AIPS and AIPS++ during the transition period between the two
        systems. The objectives of such an effort and the technical
        means by which it might be achieved are considered.  A
        distributed object design, which has already been investigated
        as a prototype, is presented as a solution. \\
   &    \\
96 &    AIPS on an Alpha/AXP Clone \\
   &    Robert~L.~Millner, Patrick~P.~Murphy and Jeffrey~A.~Uphoff \\
   &	October 14 1997 \\
   &	There has been some interest, both within NRAO and in the
	general Radio Astronomy Community, in the possibility of
	running AIPS on one of the many ``clone'' systems based on the
	Digital ALPHA AXP 21164 processor. Recently,
	NRAO/Charlottesville acquired such a system and proceeded to
	install the Linux operating system thereon.  We have also started the
	process of porting AIPS to this 64-bit system, though the
	results reported herein are based solely on the use of binaries
	generated on an OSF/1 (Digital Unix) ALPHA processor and
	copied to the Linux system.  We have successfully run the
	``DDT'' suite of programs on this Linux/Alpha system
	and have achieved an \AIPSmark\ of $9.0$. \\
    &	\\
97 &	Test of Errors of the Fitting Parameters at Gaussian Fitting
	Task JMFIT \\
   &	L.~Kogan \\
   &	October 14 1997 \\
   &	Two-dimensional elliptical Gaussian fits are used in astronomy
	for accurate measurements of source parameters such as central
	position, peak flux density and angular size. The revised
	error analysis based on \emph{con} and \emph{kog} is
	implemented at the AIPS task JMFIT. A test of the errors of
	six parameters of fitted Gaussian into an image provided by
	JMFIT has been carried out. The test demonstrates a good
	agreement with the error predicted by JMFIT. \\
   &	\\
98 &	AIPSTerminal for Linux PCs \\
   &	Robert~L.~Millner and Patrick~P.~Murphy \\
   &	November 19 1997 \\
   &	There has been some interest in accessing \AIPS\ on a home
	Linux PC. Many home machines lack the resources to run
	\AIPS\ well and people may not wish to move large amounts of
	data to and from home.  A small install package was created
	for Red Hat Linux which contains a minimal subset of \AIPS\
	to run the TV, Message and Tek servers.  With this package,
	a user may run the servers from home, dial into the network and
	access \AIPS\ on their workstation in an efficient manner. \\
\end{tabular}

\end{aipsletter}

\end{document}

