%-----------------------------------------------------------------------
%;  Copyright (C) 2017
%;  Associated Universities, Inc. Washington DC, USA.
%;
%;  This program is free software; you can redistribute it and/or
%;  modify it under the terms of the GNU General Public License as
%;  published by the Free Software Foundation; either version 2 of
%;  the License, or (at your option) any later version.
%;
%;  This program is distributed in the hope that it will be useful,
%;  but WITHOUT ANY WARRANTY; without even the implied warranty of
%;  MERCHANTABILITY or FITNESS FOR A PARTICULAR PURPOSE.  See the
%;  GNU General Public License for more details.
%;
%;  You should have received a copy of the GNU General Public
%;  License along with this program; if not, write to the Free
%;  Software Foundation, Inc., 675 Massachusetts Ave, Cambridge,
%;  MA 02139, USA.
%;
%;  Correspondence concerning AIPS should be addressed as follows:
%;          Internet email: aipsmail@nrao.edu.
%;          Postal address: AIPS Project Office
%;                          National Radio Astronomy Observatory
%;                          520 Edgemont Road
%;                          Charlottesville, VA 22903-2475 USA
%-----------------------------------------------------------------------
%Body of intermediate AIPSletter for 31 December 2017 version

\documentclass[twoside]{article}
\usepackage{graphics}

\newcommand{\AIPRELEASE}{June 30, 2017}
\newcommand{\AIPVOLUME}{Volume XXXVII}
\newcommand{\AIPNUMBER}{Number 1}
\newcommand{\RELEASENAME}{{\tt 31DEC17}}
\newcommand{\NEWNAME}{{\tt 31DEC17}}
\newcommand{\OLDNAME}{{\tt 31DEC16}}

%macros and title page format for the \AIPS\ letter.
\input LET98.MAC
%\input psfig

\newcommand{\MYSpace}{-11pt}

\normalstyle

\section{Happy 38$^{\rm th}$ birthday \AIPS}

\subsection{\Aipsletter\ publication}

We have discontinued paper copies of the \Aipsletter\ other than for
libraries and NRAO staff.  The \Aipsletter\ will be available in
PostScript and pdf forms as always from the web site listed above.
New issues will be announced in the NRAO eNews mailing and on the
bananas and mnj list server.

\subsection{Current and future releases}

We have formal \AIPS\ releases on an annual basis.  While all
architectures can do a full installation from the source files,
Linux (32- and 64-bit), Solaris, and MacIntosh OS/X (PPC and Intel)
systems may install binary versions of recent releases.  The last,
``frozen'' release is called \OLDNAME\ while \RELEASENAME\ remains
under active development.  You may fetch and install a copy of these
versions at any time using {\it anonymous} {\tt ftp} for source-only
copies and {\tt rsync} for binary copies.  This \Aipsletter\ is
intended to advise you of improvements to date in \RELEASENAME\@.
Having fetched \RELEASENAME, you may update your installation whenever
you want by running the so-called ``Midnight Job'' (MNJ) which copies
and compiles the code selectively based on the changes and
compilations we have done.  The MNJ will also update sites that have
done a binary installation.  There is a guide to the install script
and an \AIPS\ Manager FAQ page on the \AIPS\ web site.

The MNJ for binary versions of \AIPS\ now uses solely the tool {\tt
  rsync} as does the initial installation.  For locally compiled
(``text'') installations, the Unix tool {\tt cvs} running with
anonymous ftp is used for the MNJ\@. Linux sites will almost
certainly have {\tt cvs} installed; but other sites may have to
install it from the web.  Secondary MNJs will still be possible using
{\tt ssh} or {\tt rcp} or NFS as with previous releases.  We have
found that {\tt cvs} works very well, although it has one quirk. If a
site modifies a file locally, but in an \AIPS-standard directory,
{\tt cvs} will detect the modification and attempt to reconcile the
local version with the NRAO-supplied version.  This usually produces a
file that will not compile or run as intended.  Use a copy of the task
and its help file in a private disk area instead.

\AIPS\ is now copyright \copyright\ 1995 through 2017 by Associated
Universities, Inc., NRAO's parent corporation, but may be made freely
available under the terms of the Free Software Foundation's General
Public License (GPL)\@.  This means that User Agreements are no longer
required, that \AIPS\ may be obtained via anonymous ftp without
contacting NRAO, and that the software may be redistributed (and/or
modified), under certain conditions.  The full text of the GPL can be
found in the \texttt{15JUL95} \Aipsletter, in each copy of \AIPS\
releases, and on the web at {\tt http://www.aips.nrao.edu/COPYING}.


\section{Improvements of interest in \RELEASENAME}

We expect to continue publishing the \Aipsletter\ approximately every
six months, but the publication is now primarily electronic.  There
have been several significant changes in \RELEASENAME\ in the last six
months.  Some of these were in the nature of bug fixes which were
applied to \OLDNAME\ before and after it was frozen.  If you are
running \OLDNAME, be sure that it is up to date; pay attention to the
patches and run a MNJ any time a patch relevant to you appears.
New tasks in \RELEASENAME\ include {\tt ELFIT} to fit polynomials to
selected data from tables, {\tt SYSOL} to calibrate Solar data sets
using the SysPower table, {\tt DFTIM} to make an image of visibility
data at a celestial coordinate as a function of time and frequency,
{\tt AGAUS} to fit Gaussian to absorption-line image cubes, {\tt
  ZAMAN} to fit Zeeman splitting models to absorption line image
cubes, and {\tt MODAB} to prepare models for the Gaussian fitting
tasks.  The calibration of polarization, especially of
linearly-polarized feeds, is now being addressed.

{\tt 31DEC14} contains a change to the ``standard'' random parameters
in $uv$ data and adds columns to the {\tt SN} table.  Note, however,
that the random parameters written to FITS files have not been changed.
Older releases of \AIPS\ cannot handle the new {\it internal} $uv$
format and might be confused by the {\tt SN} table as well.  {\tt
  31DEC09} contains a significant change in the format of the antenna
files, which will cause older releases to do wrong things to data
touched by {\tt 31DEC09} and later releases.  You are encouraged to
use a relatively recent version of \AIPS, whilst those with recent VLA
data to reduce should get release \OLDNAME\ or, preferably, the latest
release.

\subsection{Analysis}

The noise in both emission and absorption spectra is approximately
constant with frequency.  This makes the fitting of emission Gaussians
straightforward.  However, in absorption, the parameter which is
likley to be a sum of Gaussians is the optical depth.  The noise in
optical depth depends very much on the value of the optical depth and
so is a function of frequency.  New tasks have been written to fit
optical depth Gaussians to absorption spectra in brightness rather
than optical depth.  The mathematics is just different enough that new
tasks are justified.  The Gaussian-fitting task {\tt AGAUS} allows the
interactive fitting of optical-depth Gaussians in much the same way
as {\tt XGAUS} is used for emission spectra.  The data are displayed
in absorption, but when the user provides an initial guess of the
Gaussians, the display is briefly converted into optical depth.  When
all desired pixels of the input image cube have been fit, {\tt AGAUS}
enters an interactive image mode allowing for the fits to be corrected
or improved.

Similarly, a new Zeeman splitting task was written to handle the
mathematics of absorption spectra.  {\tt ZAMAN}, like {\tt ZEMAN}, can
fit multiple Gaussian components found by {\tt AGAUS}, or it can do
traditional means of fitting the V polarization cube for a Zeeman
splitting plus a gain times the I polarization cube.  Like the other
tasks in this set, the user may edit the results in an imaging stage.
Like {\tt ZEMAN}, however, the fitting is actually linear, so it is
normally not possible to make much improvement on the results.

To test these new tasks, another new task called {\tt MODAB} was
written.  It can create I and V polarizations cubes in either emission
or absorption with rather simple inputs.  Tests of the
Gaussian-fitting tasks with such model cubes revealed a small, but
systematic bias in the answers returned by {\tt XGAUS} and company.
The bias is caused by fitting a Zeeman-widened spectral line rather
than the line in the absence of magnetic fields.  AIPS Memo 122 was
written to describe the new tasks and to illustrate the nature of this
small bias.

The new tasks required small changes to the format of {\tt XG} and
{\tt ZE} tables and minor changes to {\tt XGAUS} and {\tt ZEMAN}\@.
The latter now uses the fit baseline in plotting the {\tt XGAUS}
models.  The drawing of polygons was altered to function correctly
with pixel-replicated (blown-up) images.  {\tt XG2PL} was revised to
enable plots of models from all 4 tasks including optical depth
spectra as well as the model-fit plus data spectra.

\begin{description}
\myitem{DFTIM} is a new task to produce a ``waterfall'' image from
    $uv$ data of the intensity at a celestial coordinate as a function
    of frequency and time.  It is an extension of the plot task {\tt
      DFTPL} and can do averaging in frequency and time.
\myitem{TVSPC} was changed to offer the option of displaying an
    additional image cube with the frequency plane selected from the
    spectral display.
\myitem{UVIMG} was changed to allow empty output cells to be zero
    rather than magic blanks and to use the Hermitian property at all
    appropriate times.  Spectral averaging was changed to allow the
    averaging of IFs as well as spectral channels.
\myitem{EVAUV} was given all of the data calibration and selection
    adverbs as well as {\tt CELLSIZE} and {\tt SPECINDX} which are
    needed when {\tt SMODEL} is used.  The DFT modeling code was
    changed to enable spectral index.
\myitem{UVMOD} was changed to clarify the use of spectral index which
    is now entirely based on the header frequency.
\myitem{Holography} tasks were changed to pass along and display
    useful parameters at every stage of the process.  The $x$ axis in
    {\tt HOLOG} was reversed to make the display be from the front of
    the telescope.  Everything was confirmed by experiments with, and
    then correction, of the surface of one of the VLA antennas.
\end{description}

\subsection{UV-data}

\subsubsection{Solar observations}

The VLA is (finally) getting around to commissioning Solar
observations following the upgrade.  Such observations require the
insertion of pads to reduce the signal levels by 20 db and the use of
correspondingly stronger noise diodes to measure the system gain.  The
on-line system has been modified to write a CalDevice table containing
both normal Tcal values and Solar Tcals.  The {\tt OBIT} program used
by {\tt BDF2AIPS} was modified by Bill Cotton (distribution level 567)
to read all Tcals, writing them into a new format of {\tt CD} table
containing two more columns for {\tt SOLCAL1} and {\tt SOLCAL2}.  This
version of {\tt OBIT} also writes a new column in the {\tt SY}
(SysPower) table called {\tt CAL TYPE} to say whether the Solar or
regular Tcals apply to the values in that row.

\AIPS\ was revised on June 15, 2017 to read the new table formats,
supplying suitable default values if the new columns are not present.
Then all tasks that use the {\tt SY} table were modified to use the
new Tcal array and calibration type.  These tasks include {\tt EDITA},
{\tt LISTR},  {\tt TYAPL}, {\tt TYCOP}, {\tt TYSMO}, {\tt SNPLT}, and
several others.  Finally, a new task {\tt SYSOL} was written.  For
non-Solar data, it is simply a version of {\tt TYAPL}\@.  But not all
antennas have Solar Tcals at all frequencies.  So like the old VLA
task {\tt SOLCL}, {\tt SYSOL} determines the average gain and weight
factors at each time for those antennas that have Solar Tcals and
applies the average to those antennas that do not.  The presence of
the {\tt CAL TYPE} parameter saves {\tt SYSOL} and all the other tasks
from the old problem of telling a Solar visibility from a non-Solar
one.

\subsubsection{New flux scale}

Rick Perley and Brian Butler have released a new calibration scale for
all standard calibration sources and quite a number of other sources
as well.  The new scale was introduced into \AIPS\ on March 31, 2017.
The new scale is thought to be valid from 50 MHz to 50 GHz for the
standard sources while some of the other sources have a more
restricted frequency range.  {\tt SETJY} implements this new scale by
default and will compare it to the previous default scale
(Scaife-Heald for low frequencies, Perley-Butler 2013 for $\ge 1$
GHz).  The standard sources are 3C48, 3C123, 3C138, 3C147, 3C196,
3C286, and 3C295\@.  Extra sources which will be computed, but which
may not be good calibrator sources are (one name each) J0444-2809,
PictorA, 3C144 (Taurus A, Crab Nebula), 3C218 (Hydra A), 3C274 (Virgo
A), 3C348 (Hercules A), 3C353, 3C380, 3C405 (Cygnus A), 3C444, 3C461
(Cassiopeia A).  The 3C name is the only one used in {\tt SETJY} for
those extra sources with a 3C name --- the parenthetical remarks are
here for clarity only.  Most of the sources in the extra list have
limits on the frequency range over which the function is valid, {\tt
  SETJY} will tell you if the frequencies are out of range.

Values for the standard calibrators were also changed in {\tt BPASS},
{\tt CPASS}, {\tt BLCHN}, and {\tt PCAL} where corrections for
spectral index may be made.  Other tasks which make spectral index
corrections depend on the values being set in the {\tt SU} table by
{\tt SETJY}\@.  These tasks include {\tt CLIP}, {\tt UVFND}, {\tt
  FTFLG}, {\tt SPFLG}, and {\tt UVPLT}\@.

\begin{description}
\myitem{PCLOD} was changed to handle multiple concatenated files with
   multiple sets of header cards and to round out the scan boundaries
   to avoid just missing samples.
\myitem{PCFIT} was changed to allow time averaging under control of
   {\tt SOLINT}\@.  This should allow even more accurate solutions for
   delay and phase.
\myitem{DBCON} was corrected for a bad error; it re-scaled the $uvw$
   values by the ratio of the header frequencies of the two data sets.
   The $uvw$ values are correct for the two frequencies and should not
   be altered except for shifts in the reference pixel.  The handling
   of the two antenna files was improved in case the second data set
   had an antenna not in the first.
\myitem{VBGLU} was changed to write out all versions of input tables
   which it finds.  Previously, it would only write one version.  Of
   course, it is up to the user to make sure that such gluing is
   meaningful.
\myitem{SOUSP} was changed to use {\tt REFREQ} to select the reference
   frequency for the spectral fit and displays.
\myitem{REWAY} now prints out statistical information to assist in
   finding reasonable ranges of weights and to locate odd baselines,
   IFs, and/or Stokes.
\myitem{ELFIT} is a new task to fit polynomials to table data as a
   function of elevation, zenith angle, hour angle, parallactic angle,
   and azimuth.
\myitem{CLCAL} was changed to refuse to re-reference {\tt SN} tables
   initially written by {\tt RLDLY}\@.  That task is designed to write
   the same delay and phase into every left-hand polarization;
   re-referencing changes that delay and phase to zero.
\myitem{FITLD} was changed so that {\tt ANTNAME} is not entirely
   ignored when concatenating outputs and to have the special value
   {\tt VLITE} produce the list {\tt V0} to {\tt V27}\@.  The number
   parser was changed to die on bad values only for structurally
   significant keywords.
\end{description}

\subsection{Imaging}

There are users who wish to do calibration in \AIPS\ who have source
models (with or without spectral index) in the form of large images.
The task {\tt IM2CC} was written some time ago to break the model
image into suitable facets with Clean Component files containing
all pixels above some specified level.  Procedures called {\tt
  OOFRING} and {\tt IMFRING} were written as part of the {\tt OOCAL}
{\tt RUN} file.  The latter executes {\tt IM2CC}, followed by
{\tt OOFRING}, and finally followed by a clean-up step.  {\tt OOFRING}
uses {\tt OOSUB} to divide the calibrator data by the model including
spectral index and then to run {\tt FRING} on the divided data set,
finally copying the resulting {\tt SN} table to the input $uv$ file.

\begin{description}
\myitem{IMAGR} was revised to offer an inverse taper option in which
   data at the center of the $uv$ plane are down-weighted.  This may
   be of some use with arrays that are heavily concentrated toward
   their center.  The loading of images to the TV display was changed
   to use the full range of image intensities only whenever the
   displayed facet changes.
\myitem{SPIXR} was changed to use {\tt REFREQ} to select the reference
   frequency for the output brightness and curvature images.
\myitem{SETFC} was changed to work either with or without an output
   {\tt BOXFILE}\@.
\myitem{SUBIM} was given the additional {\tt OPCODE = 'SUM'} used when
   writing only every {\tt XINC}'th and {\tt YINC}'th pixel to write
   the sum of the image pixels in each {\tt XINC} by {\tt YINC} area.
\myitem{CCRES} was corrected to honor blanked pixels properly.
\myitem{SQASH} was corrected to handle non-zero values of {\tt BLC}
   and {\tt TRC}.
\end{description}

\subsection{Display}

\begin{description}
\myitem{UVPLT} was changed to prepare the plot in array in memory
   before loading it to the TV display or plot file.  For large data
   sets, in which many input visibilities are displayed at the same
   plot pixel, this is a great improvement in performance.  The
   labeling is done the old-fashioned way which means that the user
   can still edit the output of {\tt LWPLA} fairly easily.  The plot
   can still be done the old way entirely under control of {\tt
     BPARM(10)}\@.  The binning option was changed to use double
   precision to avoid loss of accuracy.
\myitem{PCNTR} was given the option {\tt DOFRACT} to control whether
   the polarization lines are proportional to the total or fractional
   polarization.
\myitem{IRING} was changed to allow logarithmic scales on the $y$ axis
   and to allow up to 10240 rings.
\myitem{POSSM} was corrected to allow correlation functions to be
   labeled in seconds.
\end{description}


\subsection{General}

The \Cookbook\ was updated to include the new VLA Solar calibration,
the new and updated Gaussian-fitting tasks, {\tt DFTIM}, changes to
{\tt REWAY}, {\tt PCNTR}, and {\tt RMSD}, the new image option in {\tt
  TVSPC}, and the new calibration source flux scale.
Task {\tt NANS} was altered to search for NaNs (Not-a-Number bad
values) in image files as well as $uv$ files.  \AIPS' start-up scripts
and low-level disk access routines were changed to allow up to 71
disks while running \AIPS\ rather than ``just'' 35.

\section{Recent \AIPS\ Memoranda}

All \AIPS\ Memoranda are available from the \AIPS\ home page.  \AIPS\
Memo 120 describing the new \AIPS\ task {\tt TVSPC} was revised.
\AIPS\ Memo 118, ``Modeling Spectral Cubes in \AIPS'', received minor
revisions in February and new \AIPS\ Memo 122 describing {\tt AGAUS}
and {\tt ZAMAN} appeared.  \AIPS\ Memo 117, ``\AIPS\ FITS File
Format,'' was revised in June to describe the changed formats of the
CalDevice ({\tt CD}) and SysPower ({\tt SY}) tables.

\begin{tabular}{lp{5.5in}}
{\bf 120 {\it revised}} & {\bf Exploring Image Cubes in \AIPS}\\
   &  Eric W. Greisen, NRAO\\
   &  December 27, 2016\\
   &  \AIPS\ has recently acquired powerful tasks to fit models to the
      spectral axis of image cubes.  These tasks are easier to run if
      the user is already familiar with the general structure of the
      data cube.  A new task {\tt TVSPC} has been written to assist in
      acquiring this familiarity.  This task provides an exploration
      tool within the \AIPS\ environment, rather than requiring users
      to export their cubes to one or more of the many excellent
      visualization tools now available.  In the {\tt 31DEC17} version
      an additional data cube may be displayed one plane at a time.\\
\\
{\bf 122 {\it revised}} & {\bf Modeling Absorption-line Cubes in \AIPS}\\
   &  Eric W. Greisen, NRAO\\
   &  March 7, 2017\\
   &  \AIPS\ does Gaussian fitting of spectral lines with
      recently-overhauled task {\tt XGAUS} and can fit V polarization
      image cubes for Zeeman-splitting with the relatively new task
      {\tt ZEMAN}\@.  Both of these tasks are designed for emission
      spectra in which the noise is not a function of spectral
      channel.  In absorption, however, the noise in optical depth
      becomes high when the optical depth is high.  Therefore, new
      tasks {\tt AGAUS} and {\tt ZAMAN} have been written to provide
      similar functions but with mathematics suitable for absorption
      lines.  This memo describes the new tasks in some detail and
      includes a description of a new, simplified modeling task {\tt
        MODAB} which may also be useful.  That task has shown that the
      results of these four tasks are biased by the presence of the
      Zeeman splitting and need modest correction if they are meant to
      describe the actual pre-splitting line widths and magnetic field.
\end{tabular}

\section{Patch Distribution for \OLDNAME}

Important bug fixes and selected improvements in \OLDNAME\ can be
downloaded via the MNJ or from the Web beginning at:
\hspace{3em}{\tt http://www.aoc.nrao.edu/aips/patch.html}\\
Alternatively one can use {\it anonymous} \ftp\ to the NRAO server
{\tt ftp.aoc.nrao.edu}.  Documentation about patches to a release is
placed on this site at {\tt pub/software/aips/}{\it release-name} and
the code is placed in suitable sub-directories below this.  As bugs in
\NEWNAME\ are found, they are simply corrected since \NEWNAME\ remains
under development.  Corrections and additions are made with a midnight
job rather than with manual patches.  Because of the many binary
installations, we now actually patch the master version of \OLDNAME,
meaning that a MNJ run on \OLDNAME\ after the patch will fetch the
corrected code and/or binaries rather than failing.  Also,
installations of \OLDNAME\ after the patch date will contain the
corrected code.  The \OLDNAME\ release has had a number of important
patches:
\begin{enumerate}
   \item\ {\tt POSSM} would not label correlation functions in
     seconds. {\it 2017-01-04}
   \item\ {\tt ATLOD} would not open disk files properly. {\it
       2017-01-06}
   \item\ {\tt CCRES} did not honor blanked pixels. {\it 2017-01-12}
   \item\ {\tt DFTPL} did not use the correct frequencies when {\tt
       BIF > 1}. {\it 2017-01-31}
   \item\ {\tt DBCON} scaled uvw wrongly when combining data sets of
     different frequencies. {\it 2017-02-24}
   \item\ {\tt DFTPL} did not address data correctly when the input
     had more than one Stokes. {\it 2017-02-28}
   \item\ {\tt BPASS} did not normalize linear polarization
     bandpasses correctly for normalization type 1. {\it 2017-03-08}
   \item\ Automatic spectral index finding from the {\tt SU} table had
     problems when one or more IFs had no flux value. {\it 2017-03-21}
   \item\ {\tt UVMOD} did not handle frequencies correctly with
     spectral index. {\it 2017-03-21}
   \item\ {\tt SPIXR} did not label the output images with the
     frequency for which they were determined (1 GHz). {\it
       2017-03-31}
   \item\ {\tt PCAL} miscomputed the flux at 1 GHz for known sources
     at low frequency. {\it 2017-06-06}
   \item\ {\tt SQASH} mishandled non-default values of {\tt BLC} and
     {\tt TRC}\@. {\it 2017-06-21}
   \item\ {\tt UVLSF} copied tables twice to the output continuum data
     set. {\it 2017-06-30}
\end{enumerate}

\section{\AIPS\ Distribution}

We are now able to log apparent MNJ accesses and downloads of the tar
balls.  We count these by unique IP address.  Since some systems
assign the same computer different IP addresses at different times,
this will be a bit of an over-estimate of actual sites/computers.
However, a single IP address is often used to provide \AIPS\ to a
number of computers, so these numbers are probably an under-estimate
of the number of computers running current versions of \AIPS\@. In
2017, there have been a total of 375 IP addresses so far that have
accessed the NRAO cvs master.  Each of these has at least installed
\AIPS\@.  During 2017 more than 265 IP addresses have downloaded the
frozen form of \OLDNAME, while more than 433 IP addresses have
downloaded \RELEASENAME\@.  The binary version was accessed for
installation or MNJs by 258 sites in \OLDNAME\ and 377 sites in
\RELEASENAME\@.  A total of 768 different IP addresses have appeared
in one of our transaction log files.  These numbers are lower than
last year, which was lower than the year before.

\vfill\eject

\section{From the archives}

\newcommand{\Item}{\item\hspace{0.7em}}
\newcommand{\bre}{\begin{quote}\begin{enumerate}}
\newcommand{\ere}{\end{enumerate}\end{quote}\vfill}
\newcommand{\dgg}[1]{$#1^{\circ}\;\hbox{F}$}


% chapter 2 *************************************************
\subsection{Banana daiquiri}

\bre
\Item {Combine in an electric blender: 2 ounce {\bf light rum}, 0.5
     ounce {\bf banana liqueur}, 0.5 ounce {\bf lime juice}, 1/2 small
     {\bf banana} peeled and coarsely chopped, and 1/2 cup crushed
     {\bf ice}.}
\Item {Blend at high speed until smooth.}
\Item {Pour into large saucer champagne (or similar) glass.
           Serves one.}
\ere

% chapter 4  *************************************************
\subsection{Banana storage}

      Bananas ripen after harvesting.  They do it best at room
temperature.  Because of this there are three stages to banana
storage.
\par\bre
\Item {{\bf On the counter:} When you buy a bunch of
   bananas that are not exactly at the ripeness you want, you
   can keep them at room temperature until they are just right for
   you.  Be sure to keep them out of any plastic bags or containers.}
\Item {{\bf In the refrigerator:} If there are any bananas left,
   and they are at the ripeness you like, you can put them in the
   refrigerator.  The peel will get dusty brown and speckled, but the
   fruit inside will stay clear and fresh and at that stage of
   ripeness for 3 to 6 days.}
\Item {{\bf In the freezer:} If you want to keep your bananas even
   longer, you can freeze them.  Mash the bananas with a little lemon
   juice, put them in an air tight freezer container and freeze.  Once
   they're defrosted, you'll go bananas baking bread, muffins and a
   world of other banana yummies.  Or, you can freeze a whole banana
   on a Popsicle stick.  When it is frozen, dip it in chocolate sauce,
   maybe even roll it in nuts, then wrap it in aluminum foil and put
   it back in the freezer.  Talk about a scrumptious snack.}
\ere

% chapter 12 *************************************************
\subsection{Banana-chocolate tea bread}

\bre
\Item {Cream 1/2 cup softened {\bf butter}, gradually add 1 cup {\bf
     sugar}, beating until light and fluffy.  Add 2 {\bf eggs}, one at
     a time, beating well after each addition.}
\Item {Combine 1 1/2 cups all-purpose {\bf flour}, 2 tablespoons {\bf
     cocoa}, 1 teaspoon {\bf baking soda}, 1 teaspoon {\bf salt}, and
     1/2 teaspoon {\tt cinnamon}; sift together.}
\Item {Stir flour mixture into egg mixture, blending well.}
\Item {Add 1 teaspoon {\bf vanilla extract}; stir in  1 cup mashed
     {\bf banana}, 1/2 cup {\bf sour cream}, 1/2 cup chopped {\bf
     walnuts}, and 1/3 cup miniature {\bf semi-sweet chocolate
     chips}.}
\Item {Spoon batter into two greased and floured 7-1/2 x 3 x 2-inch
     loaf pans. Bake at \dgg{350} for 55 minutes or until a wooden
     pick inserted in center comes out clean. Cool in pans 10 minutes,
     remove from pans and cool completely on a wire rack.}
\item[ ]{\hfill Thanks to Tim D. Culey, Baton Rouge, La. ({\tt
     tsculey@bigfoot.com}).}
\ere


% chapter C *************************************************
\subsection{Golden mousse}

\bre
\Item {Combine 1 cup mashed ripe {\bf bananas}, 2
     tablespoons {\bf orange juice}, 1/4 cup shredded {\bf coconut}, 3
     tablespoons {\bf brown sugar}, a few grains {\bf salt}, and 1/8
     teaspoon grated {\bf orange rind}.}
\Item {Whip until stiff 1 cup {\bf heavy cream}.}
\Item {Fold whipped cream into fruit mixture and turn into
     freezing tray.  Freeze rapidly without stirring until firm.}
\ere

\vfill\eject

% mailer page
% \cleardoublepage
\pagestyle{empty}
 \vbox to 4.4in{
  \vspace{12pt}
%  \vfill
\centerline{\resizebox{!}{3.2in}{\includegraphics{FIG/Mandrill.eps}}}
%  \centerline{\rotatebox{-90}{\resizebox{!}{3.5in}{%
%  \includegraphics{FIG/Mandrill.color.plt}}}}
  \vspace{12pt}
  \centerline{{\huge \tt \AIPRELEASE}}
  \vspace{12pt}
  \vfill}
\phantom{...}
\centerline{\resizebox{!}{!}{\includegraphics{FIG/AIPSLETS.PS}}}

\end{document}
