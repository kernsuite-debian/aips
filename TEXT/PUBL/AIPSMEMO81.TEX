%-----------------------------------------------------------------------
%;  Copyright (C) 1995
%;  Associated Universities, Inc. Washington DC, USA.
%;
%;  This program is free software; you can redistribute it and/or
%;  modify it under the terms of the GNU General Public License as
%;  published by the Free Software Foundation; either version 2 of
%;  the License, or (at your option) any later version.
%;
%;  This program is distributed in the hope that it will be useful,
%;  but WITHOUT ANY WARRANTY; without even the implied warranty of
%;  MERCHANTABILITY or FITNESS FOR A PARTICULAR PURPOSE.  See the
%;  GNU General Public License for more details.
%;
%;  You should have received a copy of the GNU General Public
%;  License along with this program; if not, write to the Free
%;  Software Foundation, Inc., 675 Massachusetts Ave, Cambridge,
%;  MA 02139, USA.
%;
%;  Correspondence concerning AIPS should be addressed as follows:
%;          Internet email: aipsmail@nrao.edu.
%;          Postal address: AIPS Project Office
%;                          National Radio Astronomy Observatory
%;                          520 Edgemont Road
%;                          Charlottesville, VA 22903-2475 USA
%-----------------------------------------------------------------------

\documentstyle [twoside]{article}
\newcommand{\memnum}{81}
\newcommand{\memtit}{Tape and TV Performance in AIPS}
\title{\memtit}
\author{Eric W. Greisen}
%
%
\newcommand{\AIPS}{{$\cal AIPS\/$}}
\newcommand{\POPS}{{$\cal POPS\/$}}
\newcommand{\eg}{{\it e.g.},}
\newcommand{\ie}{{\it i.e.},}
\newcommand{\daemon}{d\ae mon}
\newcommand{\boxit}[3]{\vbox{\hrule height#1\hbox{\vrule width#1\kern#2%
\vbox{\kern#2{#3}\kern#2}\kern#2\vrule width#1}\hrule height#1}}
%
\parskip 4mm
\linewidth 6.5in
\textwidth 6.5in                     % text width excluding margin
\textheight 8.81 in
\marginparsep 0in
\oddsidemargin .25in                 % EWG from -.25
\evensidemargin -.25in
\topmargin -.5in
\headsep 0.25in
\headheight 0.25in
\parindent 0in
\newcommand{\normalstyle}{\baselineskip 4mm \parskip 2mm \normalsize}
\newcommand{\tablestyle}{\baselineskip 2mm \parskip 1mm \small }
%
%
\begin{document}

\pagestyle{myheadings}
\thispagestyle{empty}

\newcommand{\Rheading}{\AIPS\ Memo \memnum \hfill \memtit \hfill Page~~}
\newcommand{\Lheading}{~~Page \hfill \memtit \hfill \AIPS\ Memo \memnum}
\markboth{\Lheading}{\Rheading}
%
%

\vskip -.5cm
\pretolerance 10000
\listparindent 0cm
\labelsep 0cm
%
%

%\title{{\fbox{\AIPS\ Memo \memnum}} \\ \memtit}
\title{\memtit}
\vskip -30pt
\maketitle
\vskip -30pt
\normalstyle

\section{Introduction}

The NRAO will be phasing out its Convex C-1 computers beginning in the
next calendar year.  It would be desirable, according to some people,
to turn them completely off on January 1, 1993.  The principal losses
in so doing would be the four high-speed, half-inch reel tape drives
on each C-1 and the IVAS and IIS Model 70 TV display devices supported
by the C-1's.  It is therefore necessary for us to evaluate the extent
of, and to attempt to minimize, these losses.

In \AIPS\ Memo 80, I addressed the subject of the implementation of
remote tape devices in \AIPS, including measurements of performance
for various configurations of the Berkeley sockets used to implement
them.  Additional questions that arise with tapes are whether \AIPS\
even works with particular tape devices and, if so, how well.  This
memo presents some measurements of performance on digital audio tapes,
Exabyte tapes, and 6250-bpi, half-inch reel tapes on a Sun IPX
(``primate''), an IBM RS 6000/530 (``lemur''), and a Convex C-1
(``nrao1'').  The Exabyte on lemur is an 8200 model, while the one on
primate is a dual-density model used in 8200 mode.

After a very useful discussion with Richard Gooch of the Australia
Telescope, I began a number of modifications to the \AIPS\
``television'' display driver for workstations ({\tt XAS}).  This
memo also addresses briefly the nature of those modifications and
presents some measurements of the changes in performance.  In
addition, the results are compared to performances on the IIS Model 70
and IVAS displays on nrao1.

\section{Tape performance}

A number of test programs were run on the various computers and tape
drives.  These were primarily {\tt FITTP} to write FITS-format data to
tape, {\tt PRTTP} to read each record of the tape including parsing
the headers and printing summaries, and {\tt AVTP} to advance to the
end of tape reading one record followed by an advance-file for each
file on the tape.

The results given in the Table 1 below are not surprising.  Real
half-inch tape drives are faster, but, of course, hold very little
data by modern standards.  Those on lemur are faster than those on the
older nrao1.  Exabytes are faster than DATs by a modest margin when
the data files are around 20 Mbytes or more, but DATs are much faster
at handling end-of-file marks.  Thus DATs are to be preferred for
files around 3 Mbytes.  It should be possible to quantify this by
assuming that
$$T_{real} - T_{cpu} = N_{files} X + M_{bytes} Y$$
where $N_{files}$ is the number of files processed and $M_{bytes}$ is
the number of Megabytes of data.  The fit to this model is good in
most cases and the results are presented in Table 2.  If the numbers
are to be believed, Exabytes have a heavy overhead per file for
writing, but run about 1.5 times faster per Megabyte than DATs.  They
have more similar speeds when reading.

\vfill\eject
\vfill
\begin{center}
\begin{tabular}{lrrlrrrrrr}
\multicolumn{10}{c}{Table 1. \qquad Tape Operation Times (seconds)}\\
\hline
Function&\multicolumn{2}{c}{Size}&Computer&\multicolumn{2}{c}{Exabyte}&
\multicolumn{2}{c}{DAT}&\multicolumn{2}{c}{1/2-inch reel}\\
 &$N_{files}$&$M_{bytes}$& &$T_{cpu}$&$T_{real}$&$T_{cpu}$&$T_{real}$&
$T_{cpu}$&$T_{real}$\\
\hline
\\
{\tt FITTP} &49 uv&  150M&primate& 337.7& 3147& 338.8& 2010&     -&    -\\
{\tt FITTP} &16 uv&  438M&primate& 671.4& 2613& 669.5& 2823&     -&    -\\
{\tt FITTP} &18 uv&  484M&primate& 742.7& 2917& 736.9& 3139&     -&    -\\
{\tt FITTP} &14 uv&  333M&lemur  & 341.4& 2136& 342.5& 2206&     -&    -\\
{\tt FITTP} & 7 uv&  148M&lemur  & 153.6& 1040& 153.4&  988& 157.1&  448\\
{\tt FITTP} &23 uv&  4.5M&lemur  &  50.0& 1214&  50.3&  401&  50.2&  156\\
{\tt FITTP} & 3 uv&  118M&lemur  & 118.7&  705& 118.6&  719& 114.6&  296\\
{\tt FITTP} & 1 uv&  0.3M&nrao1  &     -&    -&     -&    -&   5.3&   11\\
{\tt FITTP} &23 uv&  3.4M&nrao1  &     -&    -&     -&    -& 202.0&  395\\
{\tt FITTP} &48 uv&  151M&nrao1  &     -&    -&     -&    -& 920.9& 1854\\
{\tt FITTP} & 1 uv&   28M&nrao1  &     -&    -&     -&    -& 103.4&  193\\
\\
{\tt PRTTP} &97 uv& 1405M&primate& 244.9& 6615& 249.5& 8281&     -&    -\\
{\tt PRTTP} & 7 uv&  148M&primate&  19.4&  704&  20.4&  871&     -&    -\\
{\tt PRTTP} &97 uv& 1405M&lemur  & 174.8& 6932& 176.0& 8272&     -&    -\\
{\tt PRTTP} & 7 uv&  148M&lemur  &  13.4&  748&  14.2&  871&  15.3&  230\\
{\tt PRTTP} &23 uv&  4.5M&lemur  &  32.0&  192&  31.7&   51&  31.6&   50\\
{\tt PRTTP} & 3 uv&  118M&lemur  &   8.7&  587&   8.5&  697&   8.9&  182\\
{\tt PRTTP} &47 uv&  142M&nrao1  &     -&    -&     -&    -&   7.6&   66\\
{\tt PRTTP} & 1 uv&   28M&nrao1  &     -&    -&     -&    -&   7.6&   66\\
\\
{\tt TPHEAD}& 1 uv&      &primate&      &   33&      &   11&     -&    -\\
{\tt TPHEAD}& 1 uv&      &lemur  &      &   16&      &   14&      &    8\\
\\
{\tt AVTP}  &49 uv&  150M&primate&   0.3&  567&   0.3&  389&     -&    -\\
{\tt AVTP}  &65 uv&  588M&primate&   0.3&  941&   0.3&  650&     -&    -\\
{\tt AVTP}  &97 uv& 1405M&primate&   0.6& 1677&   0.5& 1133&     -&    -\\
{\tt AVTP}  & 7 uv&  148M&primate&   0.2&  174&   0.2&  100&     -&    -\\
{\tt AVTP}  &83 uv& 1027M&lemur  &   0.2& 1070&   0.3&  964&     -&    -\\
{\tt AVTP}  & 7 uv&  148M&lemur  &   0.1&  150&   0.1&   96&   0.1&  220\\
{\tt AVTP}  &23 uv&  4.5M&lemur  &   0.1&  183&   0.2&  129&   0.1&   12\\
{\tt AVTP}  & 3 uv&  118M&lemur  &   0.1&  110&   0.1&   44&   0.1&  178\\
\\
{\tt REWIND}&97 uv& 1405M&primate&      &  119&      &   57&     -&    -\\
{\tt REWIND}& 7 uv&  148M&primate&      &   40&      &   15&     -&    -\\
{\tt REWIND}&97 uv& 1405M&lemur  &      &  109&      &   78&     -&    -\\
{\tt REWIND}& 7 uv&  148M&lemur  &      &   36&      &   35&      &   90\\
{\tt REWIND}&23 uv&  4.5M&lemur  &      &   27&      &    6&      &    8\\
{\tt REWIND}& 3 uv&  118M&lemur  &      &   34&      &   29&      &   75\\
\\
{\tt DISMOUNT}&49 uv&  150M&primate&      &   56&      &   42&     -&    -\\
{\tt DISMOUNT}& 7 uv&  148M&primate&      &   59&      &   39&     -&    -\\
\hline
\end{tabular}
\end{center}
\vfill\eject

\begin{center}
\begin{tabular}{ll|rrr|rrr}
\multicolumn{8}{c}{Table 2. \qquad Apparent Tape Rates \hfill}\\
\hline
Computer &Tape&\multicolumn{3}{c|}{X (sec/file)}&\multicolumn{3}{c}{Y (sec/Mbyte)}\\
 & &{\tt FITTP}&{\tt AVTP }&{\tt PRTTP}&{\tt FITTP}&{\tt AVTP }&{\tt PRTTP}\\
\hline
nrao1  &reel   &  8&   -& 4.0& 3.50&    -& 2.0\\
primate&DAT    & 21& 6.9&-1.0& 4.15& 0.35& 5.8\\
primate&Exabyte& 49& 9.5&-7.0& 2.65& 0.60& 4.8\\
lemur  &reel   &4.2& 0.2& 0.5& 1.40& 1.50& 1.45\\
lemur  &DAT    & 14& 5.6&-0.3& 4.70& 0.20& 5.84\\
lemur  &Exabyte& 50& 7.8& 6.0& 3.80& 0.73& 4.75\\
\hline
\end{tabular}
\end{center}

\section{Changes to XAS and Performance}

A number of changes have been made in the {\tt 15OCT92} version of the
\AIPS\ television driver \hbox{{\tt XAS}}.  First, the {\tt DISPLAY}
variable was changed from {\it host}{\tt :0} to simply {\tt :0}.  This
should prompt the X server to use Unix sockets rather than Internet
sockets, with some improvement in performance.  Second, the ``blit''
of the image from {\tt XAS}'s memory to the display was changed to be
as large as possible on each display update.  Previously, only a row
at a time was blitted when the image was zoomed and/or contained
graphics overlays.  Third, the {\tt XAS} memory was changed to use,
optionally, the X extension called ``shared memory.''  This greatly
improves blit speed after an initial overhead to synchronize the
memories.  Fourth, the application code was provided with the option
to ask {\tt XAS} to delay updating the display until instructed to do
so.  This allows multiple graphics planes to be turned on with a
single screen update, a full image to be loaded with a single blit to
the display rather than one blit per row, multiple line segments of a
plot to be drawn with a single blit to the display rather than very
many small blits, and so forth.  This option, implemented with
subroutine {\tt YHOLD}, is dangerous in that it requires considerable
care on the part of the application programmer to make certain the the
display is brought up to date whenever required.  As some protection
against programmer error, subroutine {\tt TVCLOS} forces
synchronization.  Also the new {\tt XAS} allows the user to set (via
his or her {\tt .Xdefaults} file) a maximum number of commands to be
done asynchronously before {\tt XAS} itself forces an update of the
screen.

The two tables below list some times to complete and some frames rates
for various TV functions using nrao1 for the hardware TV devices (IIS
and IVAS) and primate for various versions of {\tt XAS}.

\begin{center}
\begin{tabular}{lrrrrrrr}
\multicolumn{8}{c}{Table 3. \qquad TV Operation Times (seconds)\hfill
(smaller numbers are better)}\\
\hline
Function&{\tt 15APR92}&{\tt 15OCT92}&{\tt 15OCT92}&{\tt 15OCT92}&
{\tt 15OCT92}& IIS& IVAS\\
Computer      &primate&primate&primate&primate&primate&nrao1&nrao1\\
Asynchronous? & No& No& Yes&  No& Yes& na& na\\
Shared memory?& No& No&  No& Yes& Yes& na& na\\
\hline
25 TVINITs     & 150& 125&  89&  73&  69&  30& 162\\
25 TVLODs (256)&  53&  58&  41&  71&  40&  92& 105\\
25 TVLABELs    & 323& 267& 150& 344& 162&  64&  94\\
CNTR (real)    &  70&  64&  33&  98&  36&  28&  36\\
CNTR (cpu)     &13.0&16.3&16.0&17.6&16.5&14.3&12.5\\
\hline
\end{tabular}
\end{center}

\begin{center}
\begin{tabular}{lrrrrrrr}
\multicolumn{8}{c}{Table 4. \qquad TV Maximum Frames / second \hfill
(larger numbers are better)}\\
\hline
Function&\multicolumn{2}{c}{Size}&{\tt 15APR92}&{\tt 15OCT92}&
{\tt 15OCT92}& IIS& IVAS\\
Computer      &    &    &primate&primate&primate&nrao1&nrao1\\
Shared memory?&    &    &No  &No   &Yes  &na & na\\
\hline
TVblink        & 518& 518&  3.26&  4.70& 13.9& 7.5& ---\\
TVblink        &1142& 800&  1.00&  1.54& 10.0& ---& 3.9\\
TVmovie no zoom& 258& 198& 13.50& 21.00& 47.0& ---& ---\\
TVmovie 2x zoom& 570& 396&  2.07&  3.36&  6.8& 7.5& ---\\
TVmovie 3x zoom& 855& 594&  0.98&  1.72&  3.8& ---& ---\\
TVmovie 4x zoom&1140& 792&  0.62&  1.05&  3.4& ---& 6.35\\
\hline
\end{tabular}
\end{center}

The values in the first table may be understood after some reflection.
The overhead of synchronizing shared memory to display memory is not
trivial.   Therefore, shared memory can be very slow when the displays
are done a small amount at a time, as is usually required in
\hbox{\AIPS}, unless the screen updates are combined via the
asynchronous option.  In fact, for the image drawing functions in the
first table, the use of the asynchronous option is very much more
important than the shared memory option and regular memory is
preferable to shared in two of the four tests.

The second table was prepared with special versions of {\tt TVBLNK}
and {\tt TVMOVI} which were altered to run at maximum rates (no calls
to {\tt ZDELAY}) and to report the frame rates on button pushes.
There is no way that the asynchronous option may be used in these
algorithms; they are simply a measure of how quickly can we blit
portions of the image memories to the display (with zoom computations
where needed).  Clearly shared memory is a big winner in these
algorithms, pushing the screen hardware update rates in the fastest
case.

I do not understand why the IIS frame rates are one-fourth of the
screen refresh rate.  There was a background {\tt MX} running during
all of the nrao1 tests.  However, numerous frame rate measurements
gave consistent results, suggesting that {\tt MX} was not to blame.
Ignoring this (small) uncertainty, it is clear that the new {\tt XAS}
is quite competitive with the old hardware TVs for these standard
functions.  Of course, {\tt XAS} cannot display true-color images, nor
can it do our hue-intensity algorithm.
\end{document}
