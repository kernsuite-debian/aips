%-----------------------------------------------------------------------
%;  Copyright (C) 1995
%;  Associated Universities, Inc. Washington DC, USA.
%;
%;  This program is free software; you can redistribute it and/or
%;  modify it under the terms of the GNU General Public License as
%;  published by the Free Software Foundation; either version 2 of
%;  the License, or (at your option) any later version.
%;
%;  This program is distributed in the hope that it will be useful,
%;  but WITHOUT ANY WARRANTY; without even the implied warranty of
%;  MERCHANTABILITY or FITNESS FOR A PARTICULAR PURPOSE.  See the
%;  GNU General Public License for more details.
%;
%;  You should have received a copy of the GNU General Public
%;  License along with this program; if not, write to the Free
%;  Software Foundation, Inc., 675 Massachusetts Ave, Cambridge,
%;  MA 02139, USA.
%;
%;  Correspondence concerning AIPS should be addressed as follows:
%;          Internet email: aipsmail@nrao.edu.
%;          Postal address: AIPS Project Office
%;                          National Radio Astronomy Observatory
%;                          520 Edgemont Road
%;                          Charlottesville, VA 22903-2475 USA
%-----------------------------------------------------------------------
\documentstyle [twoside]{article}
\newcommand{\memnum}{87}
\newcommand{\whatmem}{\AIPS\ Memo \memnum}
%\newcommand{\whatmem}{{\bf D R A F T}}
\newcommand{\memtit}{The NRAO AIPS Project --- A Summary}
%
%
\newcommand{\AIPS}{{$\cal AIPS\/$}}
\newcommand{\POPS}{{$\cal POPS\/$}}
\newcommand{\eg}{{\it e.g.},}
\newcommand{\ie}{{\it i.e.},}
\newcommand{\daemon}{d\ae mon}
\newcommand{\boxit}[3]{\vbox{\hrule height#1\hbox{\vrule width#1\kern#2%
\vbox{\kern#2{#3}\kern#2}\kern#2\vrule width#1}\hrule height#1}}
%
\title{
%   \hphantom{Hello World} \\
   \vskip -35pt
   \fbox{{\large\whatmem}} \\
   \vskip 28pt
   \memtit\thanks{Updated from Alan Bridle's May 1987 AIPS Memo 51 of
       the same title.} \\}
\author{Alan H. Bridle, Eric W. Greisen}
%
\parskip 4mm
\linewidth 6.5in
\textwidth 6.5in                     % text width excluding margin
\textheight 8.81 in
\marginparsep 0in
\oddsidemargin .25in                 % EWG from -.25
\evensidemargin -.25in
\topmargin -.5in
\headsep 0.25in
\headheight 0.25in
\parindent 0in
\newcommand{\normalstyle}{\baselineskip 4mm \parskip 2mm \normalsize}
\newcommand{\tablestyle}{\baselineskip 2mm \parskip 1mm \small }
\input psfig
%
%
\begin{document}

\pagestyle{myheadings}
\thispagestyle{empty}

\newcommand{\Rheading}{\whatmem \hfill \memtit \hfill Page~~}
\newcommand{\Lheading}{~~Page \hfill \memtit \hfill \whatmem}
\markboth{\Lheading}{\Rheading}
%
%

\vskip -.5cm
\pretolerance 10000
\listparindent 0cm
\labelsep 0cm
%
%

\vskip -30pt
\maketitle
%\vskip -30pt
\normalstyle

%\begin{abstract}
%\end{abstract}

    The NRAO Astronomical Image Processing System (AIPS) is a software
package for interactive (and, optionally, batch) calibration and
editing of radio interferometric data and for the calibration,
construction, display and analysis of astronomical images made from
those data using Fourier synthesis methods.  Design and development of
the package began in Charlottesville, Virginia in 1978.  It presently
consists of over 800,000 lines of code, 80,000 lines of on-line
documentation, and 400,000 lines of other documentation.  It contains
over 300 distinct applications ``tasks,'' representing approximately
50 man-years of effort since 1978.  The AIPS group in Charlottesville
and Socorro has five full-time scientist/programmers, and several
other computing and scientific staff with partial responsibility to
the AIPS effort.  The group is responsible for the code design and
maintenance, for documentation aimed at users and programmers, and for
exporting the code to about 200 non-NRAO sites that have requested
copies of AIPS.  It currently offers AIPS installation kits for a
variety of UNIX systems, with updates available semi-annually.

    In 1983, when AIPS was selected as the primary data reduction
package for the Very Long Baseline Array (VLBA), the scope of the AIPS
effort was expanded to embrace all stages of radio interferometric
calibration, both continuum and spectral line.  The AIPS package
contains a full suite of calibration and editing functions for both
VLA and VLBI data, including interactive and batch methods for editing
visibility data.  For VLBI, it reads data in MkII, MkIII and VLBA
formats, performs global fringe-fitting by two alternative methods,
offers special phase-referencing and polarization calibration, and
performs geometric corrections, in addition to the standard
calibrations done for connected-element interferometers.  The
calibration methods for both domains encourage the use of realistic
models for the calibration sources and iterated models using
self-calibration for the program sources.

    AIPS has been the principal tool for display and analysis of both
two- and three-dimensional radio images (\ie\ continuum ``maps'' and
spectral-line ``cubes'') from the NRAO's Very Large Array (VLA) since
early in 1981.  It has also provided the main route for
self-calibration and imaging of VLA continuum and spectral-line data.
It contains facilities for display and editing of data in the
aperture, or u-v, plane; for image construction by Fourier inversion;
for deconvolution of the point source response by Clean and by maximum
entropy methods; for image combination, filtering, and parameter
estimation; and for a wide variety of TV and graphical displays.  It
records all user-generated operations and parameters that affect the
quality of the derived images, as ``history'' files that are appended
to the data sets and can be exported with them from AIPS in the
IAU-standard FITS (Flexible Image Transport System) format.  AIPS
implements a simple command language which is used to run ``tasks''
(\ie\ separate programs) and to interact with text, graphics and image
displays.  A batch mode is also available.  The package contains nearly
3.8 Mbytes of ``help'' text that provides on-line documentation for
users.  There is also a suite of printed manuals for users and for
programmers wishing to code their own applications ``tasks'' within
AIPS.

    An important aspect of AIPS is its portability.  It has been
designed to run, with minimal modifications, in a wide variety of
computing environments.  This has been accomplished by the use of
generic FORTRAN wherever possible and by the isolation of
system-dependent code into well-defined groups of routines.  AIPS
tries to present as nearly the same interface to the user as possible
when implemented in different computer architectures and under
different operating systems.  The NRAO has sought this level of
hardware and operating system independence in AIPS for two main
reasons.  The first is to ensure a growth path by allowing AIPS to
exploit computer manufacturers' advances in hardware and in compiler
technology relatively quickly, without major recoding.  (AIPS was
developed in ModComp and Vax/VMS environments with Floating Point
Systems array processors, but was migrated to vector pipeline machines
in 1985.  Its portability allowed it to take prompt advantage of the
new generation of vector and vector/parallel optimizing compilers
offered in 1986 by manufacturers such as Convex and Alliant.  It was
extended in simple ways in 1992 to take full advantage of the current,
highly-networked workstation environment).  The second is to service
the needs of NRAO users in their home institutes, where available
hardware and operating systems may differ substantially from NRAO's.
By doing this, the NRAO supports data reduction at its users' own
locations, where they can work without the deadlines and other
constraints implicit in a brief visit to an NRAO telescope site.  The
exportability of AIPS is now well exploited in the astronomical
community; the package is known to have been installed at some time on
a large number of different computers, and is currently in active use
for astronomical research at more than 140\footnote{``The 1990 AIPS
Site Survey'', AIPS Memo No.~70, Alan Bridle and Joanne Nance, April
1991} sites worldwide.  AIPS has been run on Cray and Fujitsu
supercomputers, on Convex and Alliant ``mini-supercomputers,'' on the
full variety of Vaxen and MicroVaxen, and on a wide range of UNIX
workstations including Apollo, Data General, Hewlett Packard, IBM,
MassComp, Nord, Silicon Graphics, Stellar and SUN products.  It is
available for use on 80386 and 80486 personal computers under the
public-domain Linux operating system.  In late $1990^1$, the total
computer power used for AIPS was the equivalent of about 6.5 Cray X-MP
processors running full-time.

    Similarly, a wide range of digital TV devices and printer/plotters
has been supported through AIPS's ``virtual device interfaces''.
Support for such peripherals is contained in well-isolated subroutines
coded and distributed by the AIPS group or by AIPS users elsewhere.
Television-like interactive display in now provided directly on
workstations using an AIPS television emulator and X-Windows.  Hardware
TV devices are no longer common, but those used at AIPS sites have
included IIS Model 70 and 75, IVAS, AED, Apollo, Aydin, Comtal,
DeAnza, Graphica, Graphics Strategies, Grinnell, Image Analytics,
Jupiter, Lexidata, Ramtek, RCI Trapix, Sigma ARGS, Vaxstation/GPX and
Vicom.   Printer/plotters include Versatec, QMS/Talaris, Apple,
Benson, CalComp, Canon, Digital Equipment, Facom, Hewlett-Packard,
Imagen, C.Itoh, Printek, Printronix and Zeta products.  Generic and
color encapsulated PostScript is produced by AIPS for a wide variety
of printers and film recorders.  The standard interactive graphics
interface in AIPS is the Tektronix 4012, now normally emulated on
workstations using an AIPS program and X-Windows.

    The principal users of AIPS are VLA, VLBA, and VLBI Network
observers.  A survey of AIPS sites carried out in late $1990^1$
showed that 61\%\ of all AIPS data processing worldwide was devoted to
VLA data reduction.  Outside the NRAO, AIPS is extensively used for
other astronomical imaging applications, however.  56\%\ of all AIPS
processing done outside the U.S. involved data from instruments other
than the \hbox{VLA}.  The astronomical applications of AIPS that do
not involve radio interferometry include the display and analysis of
line and continuum data from large single-dish radio surveys, and the
processing of image data at infrared, visible, ultraviolet and X-ray
wavelengths.  About 7\%\ of all AIPS processing involved astronomical
data at these shorter wavelengths, with 7\%\ of the computers in the
survey using AIPS more for such work than for radio and {\it another}
7\%\ of the computers using AIPS exclusively for non-radio work.

    Some AIPS use occurs outside observational astronomy, \eg\ in
visualization of numerical simulations of fluid processes, and in
medical imaging.  The distinctive features of AIPS that have attracted
users from outside the community of radio interferometrists are its
ability to handle many relevant coordinate geometries precisely, its
emphasis on display and analysis of the data in complementary Fourier
domains, the NRAO's support for exporting the package to different
computer architectures, and its extensive documentation.

As well as producing user- and programmer-oriented manuals for AIPS,
the group publishes a newsletter that is sent to over 775 AIPS users
outside the NRAO soon after each semi-annual ``release'' of new AIPS
code.  There is also a mechanism whereby users can report software
bugs or suggestions to the AIPS programmers and receive written
responses to them; this provides a formal route for user feedback to
the AIPS programmers and for the programmers to document difficult
points directly to individual users.  Much of the AIPS documentation
is now available to the ``World-Wide Web'' so that it may be examined
over the Internet (start with ``URL'' {\tt
http://info.cv.nrao.edu/aips/aips-home.html}).  The NRAO knows of over
230 AIPS ``tasks,'' or programs, that have been coded within the
package outside, and not distributed by, the observatory.

    The AIPS group has developed a package of benchmarking and
certification tests that process standard data sets through the dozen
most critical stages of interferometric data reduction, and compare
the results with those obtained on the NRAO's own computers.  This
``DDT'' package is used to verify the correctness of the results
produced by AIPS installations at new user sites or on new types of
computer, as well as to obtain comparative timing information for
different computer architectures and configurations.  It has been
extensively used as a benchmarking package to guide computer
procurements at the NRAO and elsewhere.  Two other packages, ``VLAC''
and ``VLAL'', are less widely used to verify the continued correctness
of calibration and spectral-line reductions.

     In 1992, the NRAO joined a consortium of institutions seeking to
replace all of the functionality of AIPS using modern coding
techniques and languages.  The ``{\tt aips++}'' project is expected to
provide the main software platform supporting radio-astronomical data
processing in the latter half of the 1990's.  Future development of
the original (``Classic'') AIPS will therefore be limited mostly to
calibration of VLBI data, general code maintenance with minor
enhancements, and improvements in the user documentation.

    Further information on AIPS can be obtained by writing by
electronic mail to {\tt aipsmail@nrao.edu} or by paper mail to the
AIPS Group, National Radio Astronomy Observatory, Edgemont Road,
Charlottesville, VA 22903-2475, U.S.A.

\vskip 0.75in
\centerline{\hss\psfig{figure=/AIPS/TEXT/PUBL/FIG/MONKEY.PLT,height=2cm}\hss}
\vfill\eject
\end{document}
