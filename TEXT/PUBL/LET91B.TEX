%-----------------------------------------------------------------------
%;  Copyright (C) 1995
%;  Associated Universities, Inc. Washington DC, USA.
%;
%;  This program is free software; you can redistribute it and/or
%;  modify it under the terms of the GNU General Public License as
%;  published by the Free Software Foundation; either version 2 of
%;  the License, or (at your option) any later version.
%;
%;  This program is distributed in the hope that it will be useful,
%;  but WITHOUT ANY WARRANTY; without even the implied warranty of
%;  MERCHANTABILITY or FITNESS FOR A PARTICULAR PURPOSE.  See the
%;  GNU General Public License for more details.
%;
%;  You should have received a copy of the GNU General Public
%;  License along with this program; if not, write to the Free
%;  Software Foundation, Inc., 675 Massachusetts Ave, Cambridge,
%;  MA 02139, USA.
%;
%;  Correspondence concerning AIPS should be addressed as follows:
%;          Internet email: aipsmail@nrao.edu.
%;          Postal address: AIPS Project Office
%;                          National Radio Astronomy Observatory
%;                          520 Edgemont Road
%;                          Charlottesville, VA 22903-2475 USA
%-----------------------------------------------------------------------
%Body of Aips Letter for 15 April 1991
%last edited by Glen Langston on 1991 July 29
%last edited by Gareth Hunt   on 1991 July 19

\documentstyle{article}

\newcommand{\AIPRELEASE}{April 15, 1991}
\newcommand{\AIPVOLUME}{Volume XI}
\newcommand{\AIPNUMBER}{Number 2}
\newcommand{\RELEASENAME}{{\tt 15APR91}}
%macros and title page format for the AIPS letter.
\input AipsLetMac.tex

\normalstyle
\section{Recent Developments}
The structure and focus of the \AIPS\  programing group was redefined in
January 1991.  In summary, the \AIPS\  group, under the leadership of Geoff
Croes, has decided to re-write \AIPS\  in order to incorporate software
capabilities needed for the coming decade.
The new \AIPS\  software will be written in the language C++, and here after,
the new \AIPS\  will be called \AIPTOO.

A major consequence of this plan is that \AIPS\  has been frozen.  Current
software development in \AIPS\  is limited to improvements needed in the short
term for VLBA calibration and certain experimental imaging tasks.  Of course,
bugs will be fixed as soon as possible.  However, if bugs can be ``worked
around'' using existing \AIPS\  software, these bug fixes will have very low
priority.  Because most software developement effort must go towards
\AIPTOO\, no extensive software modifications to old \AIPS\ can be supported.

As a consequence of the \AIPTOO\ developement,
the manpower we have available to
support \AIPS\ has been dramatically reduced, therefore we have decided
to cease releasing \AIPS\ every quarter.
The 15APR91 version of \AIPS\
is the last planed release until 15APR92,
however, we may release another version in 15OCT91.
The decision as to when to release the next \AIPS\ version
depends on the amount of VLBA software development
that occurs over the next few months.

\section{Personnel}
There are three imminent changes to personnel.  Both Phil Diamond and Bill
Junor are leaving the \AIPS\ group to devote themselves to
the operation of the VLBA.
Bill will leave in September 1991, and Phil in January 1992.
Bill Junor's position is being filled by David
Adler, who will join the \AIPS\ group in August.

Gareth Hunt continues as head of the \AIPS\  user support in Socorro.  Glen
Langston heads \AIPS\  user support in Charlottesville as well as
continuing to develope imaging and self-calibration software for the VLBA.

\section{\AIPTOO}
The goals and design requirements for \AIPTOO\ will be described in a separate
document.  Because many man-years of work will be needed to create \AIPTOO\,
most of the \AIPS\  scientists will be working only on \AIPTOO.  The \AIPTOO\
development is headed by Geoff Croes.  The \AIPTOO\  software development is
planned as an international collaboration, with programmer support committed by
Jodrell Bank, Westerbork, the Australia Telescope, and the Tata Institute in
India.  Addtional U.S. support is committed by the Berkeley-Illinois-Maryland
Millimeter Array.  The NRAO scientists currently working on \AIPTOO\  are Bill
Cotton, Chris Flatters, Brian Glendenning.

\AIPTOO\  is still in the early stage of development, and
discussion on various topics is being carried out via electronic mail.
Several email ``exploders'' for discussion of issues related to \AIPTOO\  have
been created. The traffic to date has been logged, and is available via
anonymous {\tt ftp} from baboon.cv.nrao.edu (192.33.115.103) under the
pub/mailing-lists/aips++ directory. If after viewing the contributions to date
you are interested in joining the general AIPS++ mailing list, send email to
aips2-request@nrao.edu.

Here is a short example of retrieving the logged traffic via {\tt ftp}
from a unix machine.
If you are unable to use {\tt ftp}, email to aips2-request@nrao.edu and ask
for the traffic to date to be mailed to you.

\tablestyle
{\tt
\begin{tabular}{l}
\% {\it ftp 192.33.115.103}  \hskip 2in (if baboon.cv.nrao.edu doesn't work)\\
Connected to 192.33.115.103. \\
220 baboon FTP server (SunOS 4.1) ready. \\
Name (192.33.115.103:bglenden): {\it anonymous} \\
331 Guest login ok, send ident as password. \\
Password:{\it bglenden@nrao.edu}  \hskip 2in (use your email address) \\
230 Guest login ok, access restrictions apply. \\
ftp$>$ {\it cd pub/mailing-lists/aips++} \\
250 CWD command successful. \\
ftp$>$ {\it get aips2.log} \\
200 PORT command successful.\\
150 ASCII data connection for aips2.log (192.33.115.19,1301) (588302 bytes). \\
226 ASCII Transfer complete.\\
local: aips2.log remote: aips2.log \\
602684 bytes received in 4.5 seconds (1.3e+02 Kbytes/s) \\
ftp$>$ {\it quit} \\
221 Goodbye. \\
\end{tabular}
}
\normalstyle

\section{Reports from the 1991 \AIPS\ Site Survey}

Two reports prepared from the 1991 \AIPS\  Site Survey data are available
as \AIPS\  Memos.

The "1991 \AIPS\  Site Directory" (\AIPS\  Memo 69 by Alan Bridle and Joanne
Nance) lists hardware and \AIPS\  usage data for the 139 \AIPS\  sites whose
\AIPS\  activity was measurable from the survey data.  The directory lists
data for 204 computer systems at these 139 sites.  It also contains the
postal and E-mail addresses, and telephone numbers, for the Contact
People at each site.  Its main purpose is to help \AIPS\  Site Managers who
want to locate other \AIPS\  sites that have hardware, use patterns and
scientific interests similar to their own.  It was distributed in April
to the designated Contact People at all sites that are listed in the
Directory.  If you would like a personal copy, please use the \AIPS\
order form at the back of this \Aipsletter.

A report entitled "The 1990 \AIPS\  Site Survey" (\AIPS\  Memo 70 by Alan
Bridle and Joanne Nance) analyzes the demography and growth patterns of
the computing hardware in the worldwide \AIPS\  community.  Its main
purpose is to help people who plan \AIPS\  hardware development at the NRAO
and elsewhere to interpret recent trends in the \AIPS\  community.  It also
contains \AIPS\  performance estimates for a wide variety of computers on
which \AIPS\  is now running.  This report may be particularly interesting
to anyone seeking to make, or to justify, a new CPU or major peripheral
procurement for \AIPS\  use.  For your convenience, we reprint the summary
of its main conclusions below:

1.  There has been roughly three-fold growth in the average machine
(CPU) power per user throughout the \AIPS\  community since 1988.
"Affordable" machine power is growing faster than the number of \AIPS\
users, so we are now in a desirable growth phase in which the average
resource per \AIPS\  user can become a better match to the data-processing
needs of array telescopes.

2.  The \AIPS\  computing power is increasingly concentrated into machines
that use large fractions of their CPU time to run \AIPS\  --- the "share"
of the computing burden borne by such "highly active" \AIPS\  machines has
increased, and is now almost 80\% of the total.

3.  UNIX-based systems now provide 93\% of the concentrated machine power
used for \AIPS\  data processing.  The need for long-term support of \AIPS\
under VMS is now highly questionable.  The \AIPS\  sites that still use VMS
machines should therefore be encouraged to convert to UNIX as soon as
possible.

4.  Most of the growth in active \AIPS\  CPU power since 1988 has been in
scalar RISC workstations without classical vector processors, rather
than in vector- register machines (mini-supercomputers).  The number of
scalar machines used for \AIPS\  increased from 115 to 305 since 1988,
while the number of vector-register machines increased only from 18 to
22.  Only about a third of \AIPS\  data processing is now done with the aid
of classical vector hardware.  The use of stand-alone AP's continues to
decline.

5.  The fraction of \AIPS\  use that is devoted to VLA data processing has
increased to 74\% at the U.S.  sites outside the NRAO, but continues to
decrease at the NRAO and in other countries.  Most of the non-VLA use of
\AIPS\  is for other radio applications (VLBI, AT, MERLIN, WSRT etc.).  Use
of \AIPS\  for non-radio applications continues to be about 12\% of the
total.

6.  About 34\% of all VLA data processing by \AIPS\  in the U.S.  is
supported by NSF funds.  The NSF and NASA together support about 54\% of
the \AIPS\  CPU power in the U.S.  "Local initiatives" have almost matched
the total \AIPS\  CPU power supported by the two main federal sources of
funds for astronomy.  At \AIPS\  sites with any NSF funding for \AIPS\
computers, the NSF's contribution to the \AIPS\  CPU power averages 70\%.

7.  Computers at the NRAO now provide about 16\% of the total \AIPS\  CPU
power devoted to VLA data processing, down from 19\% in 1988.  The NRAO's
share of all active \AIPS\  CPU power has stabilized near 14\%.

8.  The total \AIPS\  machine power now in use for VLA data reduction
around the world falls short of that needed to meet the full scientific
potential of the VLA by a factor of about eight.  Machine powers in the
range from high-end "compute server" workstations to second- or
third-generation mini-supercomputers are particularly needed to augment
the available hierarchy."

Sites that are still running \AIPS\  under VMS should take particular note
of conclusion 3! This report has been distributed to the standard \AIPS\
memo recipients.  If you want a personal copy, please use the \AIPS\  order
form at the back of this \Aipsletter.

Please send any questions or comments about the \AIPS\  Site Survey or
these reports to Alan Bridle (abridle@polaris.cv.nrao.edu, 804-296-0375).

\section{Gripes}

A new system for maintaining the \AIPS\ "Gripes" has recently been
implemented and is now available for public use. Bill Cotton has created an
"AIPS Gripes Database" which is built within the EMACS editor. Each gripe
(approximately 4800 to date) is now entered into the database as an individual
file. These files can be accessed with various query modes (query by gripe
number, username, gripe date, etc.), thus making it fairly simple and
convenient to use.

The \AIPS\ Gripe Database currently resides on virtual machine called "gripe"
at NRAO-Charlottesville. A public account has been set up on this
machine; No username or password is required. To use the database from outside
NRAO, you may telnet to "gripe.cv.nrao.edu" (or to address 192.33.115.103).
From within NRAO, simply remote login to {\tt baboon.cv.nrao.edu}
with the account name "gripe".

To obtain a copy of the "User's Guide to the \AIPS\ Gripe Database" we have set
up an "anonymous {\tt ftp}" account on a Unix machine in Charlottesville. The
directory /local/ftp/pub/gripes on polaris.cv.nrao.edu (or 192.33.115.101)
contains an ascii file called GRIPEDOC which explains the basics of using the
database.

If you have any problems or comments, please contact either Bill Cotton
({\tt bcotton@nrao.edu}) or Dean Schlemmer ({\tt dschlemm@nrao.edu}).

\section{Documentation Available}
The \AIPS\ \Cookbook\ has been updated recently, and new
version chapters 4, 6 and 11 are available.
Several new \AIPS\ memos are also available.
\tablestyle
\begin{description}
\item{Memo 66:~} An Overview of \AIPS\ TV Servers.
\item{Memo 67:~} \AIPS\ DDT bench mark results for Sun's SPARCstation 2GX.
\item{Memo 68:~} Summary of \AIPS\ \Uv-data Calibration from VLA Arcive tape
to a \Uv-FITS tape.
\item{Memo 69:~} The 1991 \AIPS\ Site Directory.
\item{Memo 70:~} The 1991 \AIPS\ Site Survey.
\item{Memo 71:~} A Comparison of DDT results for IBM RS/6000 and Convex C-1.
\item{Memo 72:~} MAPIT: Automatic \AIPS\ Imaging and Self-Calibration.
\item{Memo 73:~} \AIPS\ DDT History.
\end{description}

\normalstyle

\section{Changes in \RELEASENAME\ of interest to programmers}
The changes to \RELEASENAME\ are summarized below and a few topics are
discussed in more detail.
There were 301 ``significant changes'' to \RELEASENAME.

\begin{description}
\myitem{TVs} Many improvements were applied to the \AIPS\ workstation
TV emulators.
\myitem{Calibration} A suite of procedures were created to
validate the VLA calibation software.
\myitem{\TVFLG} A few bugs werer found and fixed for graphical
editing of \UV-data.
\myitem{FQ ids} Several bugs were fixed in Multi FQ id \UV-data
processing.
\myitem{Decstation} The Decstation low level routines were added
to the \AIPS\ code heirarchy.
\myitem{Westerbork} Code was added to the \AIPS\ Contribution area
(\$APLCONTR) to read and process Westerbork data.
\myitem{Scratch} The \AIPS\ scratch file handling has been changed
to allow several CPUs to create files on the same disk.  Scratch
files from one CPU are no longer inadvertently distroyed by
other CPUs.
\myitem{Pseudo-AP} The Pseudo-Array processor include files
have been modified to allow system managers to easily increase
the array sizes.
Increasing the Pseudo-AP size increases mapping efficiency.
\myitem{Tapes} \AIPS\ now allows remote access of tapes.  In
particular, several workstations may now share a single exabyte tape
drive.
\myitem{\BPASS} Fix of Band Pass corrections for dual polarization
data bases.
\myitem{\MX} \MX\ mplimentation of the Zero-spacing flux has been
modified to allow a gaussian model for the source short spacing flux
distribution.
Data selection was modified to handle multi-frequency
\UV-data.
Also cleaning now stops at first negative CC if requested.
\myitem{\UVPRM}
\UVPRM\ has three functions, to find a reference
antenna, the maximum \UV-spacing in the \UV-data and to find
the UV-spacing at which the average flux of the source is
greater than the input FLUX.
Usefull for self-calibration.
\myitem{\CCFND}  \CCFND\ finds the last
clean component before a negative clean component then finds
the clean component FACTOR brighter than the negative component.
Useful for setting self-calibration parameters.
\myitem{\DECOR} New task to compute the decorrelation amoung a set of
selected correlations.  The output values have a real part that is the
ratio of the scalar amplitude average of the selected
correlations to the vector average.  This progam is especially
useful for looking for coherence problems in multi IF and/or
multi channel data.  This task allows application of calibration
and/or editing and will write a new multi source file.
\myitem{\SHOUV} \SHOUV\ is a new task that does column listings of closure
phases for selected triplets or listings with different
IF/channels in different columns.  These displays are very
useful for detecting and analysing problems in multi IF/channel
data sets.
\myitem{\GLENS} Gravitational LENS modeling program.  Makes models
of a Blandford elliptical galaxy or a point mass (black hole).
\myitem{\ANCAL} Added capability to give baseline dependent
calibration factors by writing a BL table if "BASELINE" cards
are present in the calibration file.  This implementation allows
only one set of baseline factors per uv data file.
\myitem{\DBCON} Fixed a problem with VLBI data that the AN
tables in the two files which may be inconsistent.
The ANntenna tables are merged if they have compatible AN table
headers and the data are to be written as one subarray.
\myitem{\MOMFT} First, second and third moments of flux density
distributionof images.
Gives a more reliable measure of compact source structure.
\myitem{\CVEL} \CVEL\ now works properly for dual polzn data.
Also for MkIII data corrections are applied for changing
reference stations.
\myitem{\ACFIT} \ACFIT\ now correctly handles dual polarization data.
\myitem{\SNPLT} Additional plotting options were added for VLBI data.
\myitem{\SNSMO}
New task to filter SN tables.  Much of the smoothing
functionality of \CLCAL\ is reproduced but this task included two
useful model for smoothing \FRING\ VLBI solutions.  Coupled
smoothings of delay (single and multiband), rate, amplitude and
phase.  The methods attempt to maintain coherence amoung the IFs
of a given Stokes' parameter (but not between Stokes').  This
task should be very useful for MkIII VLBI and VLBA data.
\myitem{\HORUS} Error in applying UNIFORM weighting corrected.
\myitem{\FILLM} A few more ``rare'' bugs fixed.
\myitem{\MKTIN} Extensive improvements and enhancements were
made by Athol Kemball and Bill Cotton.
Of particular note are improvements in handling Spectral Line Data.
The new \MKTIN\ produces an FQ table and allows multiple
observing bands.
Some changes in the Fourier transform conventions were
neccessary in handling multiple correlator module lag functions.
In particular the lag order convention needed to be reversed to
concatenate the output of the modules.

AC functions are treated separately and are corrected for bias
and clipping effects before being transformed to AC spectra.
Auto-correlation functions are typically generated by
cross-correlating half the XC lag range. Individual correlator
modules can however be placed in auto-correlation mode and this
may be used in the future.  Some changes in the code were
neccessary to deal with the two methods of generating AC functions.

Correlation functions which do not have the same number of
lags as the output file are skipped and a warning is printed.
This is neccessary to avoid 8-lag fringe finding scans which are
often included in the output tape.

Some checks were introduced to ensure that the A-tape file
structure is as expected by \MKTIN. The A-tape file headers are
decoded and the MK3 file parameters are extracted. A warning is
printed if the tape is a member of a multi-volume SAVEM set, as
this may cause loss of data.  The type 50 records within each
extent are required to be in the expected order and only
complete scan-baseline headers are used.

These checks should diminish the effect of tape errors which
may cause loss of synchronization and will also alert the user
to any future changes in the MK3 data format.

The data rejection criteria used by FRNGE at the Haystack
correlator were implemented in the new \MKTIN\ version. A
breakdown of the error statistics is available for each scan
which can optionally be indexed on correlator module serial
number. This feature was requested by the Bonn correlator group
for spectral line data.

The MK3 tape format is baseline-based and consequently there
is a high degree of redundancy in source- and antenna-based
parameters which is the format that AIPS requires. The new
version of \MKTIN\ assumes that source and antenna parameters in
files for different baselines may not be consistent and carries
out several checks in this regard. The data are not rejected or
changed in any way but a warning message is printed. The
correlator model input parameters (or closely associated
quantities) are also monitored throughout the run. A summary of
the correlator model is printed at the end of the run.
These consistency checks should help to isolate problems in
the data set caused by changes in the correlation strategy or
when merging data correlated at different times or different
correlators. These checks may also alert the user to any
unnanounced changes in the MK3 output tape format.

A correlator index is generated from the 20XX module
cross-reference table before processing each new type 51 data
extent. This allows video converters to be multiply assigned in
a given scan and should allow XC/AC mixing. The delay offsets
can be in arbitrary order in the cross-ref table.
\end{description}

\subsection{MAPIT}
A set of \AIPS\ procedures, called {\sc MAPIT},
 has been developed to automatically
image and self-calibrate \UV-data.  This set of procedures
works particularly well for VLA snapshots and observations of
radio sources with compact structure.  The {\sc MAPIT} proceedures
have been successfully used to produce images of VLBI observations.

\subsection{Remote tape dirves}

%This section still needs work. Gareth

	To get the remote tape stuff going do the following on
the machine that
is going to run \AIPS\ (not the machine with the tape drive).
I presume that it is a Sun.

Add the string -DREMOTE\_TAPES to the COMP line in CCOPTS.SH, \eg

\tablestyle
\begin{verbatim}
COMP="-v -c $OPT0 -pipe -PIC -DREMOTE_TAPES"
\end{verbatim}

or use the string as one of the arguments to COMRPL, \eg

\begin{verbatim}
COMRPL $APLBERK/ZRMTAP.C DEBUG -DREMOTE_TAPES NOPURGE
\end{verbatim}

%previously subroutines were
% COMRPL \$APLBERK/ZTAPIO.C
% COMRPL \$APLBERK/ZTPCL2.C
% COMRPL \$APLBERK/ZTPMI2.C
% COMRPL \$APLBERK/ZTPOP2.C
% COMRPL \$APLBERK/ZTPWA2.C
% COMRPL \$APLSUN/ZTAP2.C

\begin{tabular}{ll}
\% COMRPL \$APLBERK/ZRMTAP.C & Compile Berkeley Unix subroutines \\
\% COMRPL \$APLBERK/ZTAP*.C  & \hskip 1in `` \\
\% COMRPL \$APLBERK/ZTP*2.C  & \hskip 1in `` \\
& \\
\% COMLNK \$AIPPGM/AIPS.FOR  & Re-Link main \AIPS\ program \\
\% COMLNK \$APLPGM/AVTP.FOR  & Re-Link tape utility program \\
\% COMLNK \$APGNOT/FILL*.FOR & Re-Link tasks to read VLA archive tapes \\
\% COMLNK \$APGNOT/FIT*.FOR  & Re-Link tasks to write FITS format tapes \\
\% COMLNK \$APGNOT/IMLOD.FOR & Re-Link task to read FITS IMAGES \\
\% COMLNK \$APGNOT/MK3IN.FOR & Re-Link task to read Mark 3 VLBI data \\
\% COMLNK \$APLPGM/PRTTP.FOR & Re-Link task to print contents of a tape \\
\% COMLNK \$APGNOT/UVLOD.FOR & Re-Link task to read \Uv-data \\
\end{tabular}

\normalstyle
Assign the MT0x environment variable which you want to use (for tape
number x+1).  The may have to be done in ASSNLOCAL.CSH, but can probably be
done with a setenv or export before running the \AIPS\ script.  The latter is
possible with the default \AIPS\ script.  For example:

\begin{verbatim}
	setenv MT00 nrao1:/dev/rmt21   "for AIPS tape #1"
\end{verbatim}

If the remote tape is on a Convex, you will need to tpmount the tape from
the account from which you are running \AIPS\ on the Sun via rsh or equivalent.
If the remote tape is on a Sun or other compatible workstation, you can omit
this step.

In \AIPS, type

\begin{verbatim}
INTAPE n; TASK 'task'; GO
\end{verbatim}

No MOUNT command is needed.  Things should work.
\end{document}


