%-----------------------------------------------------------------------
%;  Copyright (C) 1995
%;  Associated Universities, Inc. Washington DC, USA.
%;
%;  This program is free software; you can redistribute it and/or
%;  modify it under the terms of the GNU General Public License as
%;  published by the Free Software Foundation; either version 2 of
%;  the License, or (at your option) any later version.
%;
%;  This program is distributed in the hope that it will be useful,
%;  but WITHOUT ANY WARRANTY; without even the implied warranty of
%;  MERCHANTABILITY or FITNESS FOR A PARTICULAR PURPOSE.  See the
%;  GNU General Public License for more details.
%;
%;  You should have received a copy of the GNU General Public
%;  License along with this program; if not, write to the Free
%;  Software Foundation, Inc., 675 Massachusetts Ave, Cambridge,
%;  MA 02139, USA.
%;
%;  Correspondence concerning AIPS should be addressed as follows:
%;          Internet email: aipsmail@nrao.edu.
%;          Postal address: AIPS Project Office
%;                          National Radio Astronomy Observatory
%;                          520 Edgemont Road
%;                          Charlottesville, VA 22903-2475 USA
%-----------------------------------------------------------------------
\input al82.mac
\input al8pt.mac
\letterbegin {X} {1} {January 15, 1990}

\subtitf{Limited Distribution of 15JAN90}
     The 15JAN90 release of \AIPS\ has a number of problems that
have been fixed in the 15APR90 release.  The most important of these
are in the processing and calibration of VLA spectral
line data, including mislabeling of all line frequencies.  There are also
problems in the reading and processing of MkIII VLBI data in
the 15JAN90 release.  For more details, see the article ``Changes of
Interest to Users: 15APR90'' in this \AIPSLetter.

     The VMS installation procedures for 15JAN90 have been tested and
debugged.  The Unix installation procedures have not yet been
thoroughly tested but are not known to contain serious problems.

   We will therefore ship the 15JAN90 release only to those sites that
tell us that they will not be
adversely affected by the known problems.   Sites that have
requested the 15JAN90 release and do not inform us (by E-mail,
telephone, or regular mail) that they wish to receive it, despite the
problems, will be shipped the 15APR90 release when it is available.

   The delay associated with waiting for 15APR90 will not be severe,
because we are modifying our quarterly procedures to try to ensure that the
tapes from a given release are shipped by the release date, rather
than the month later as has been the practice in the past.  We
anticipate shipping the 15APR90 release by 15 April.

\eject

\subtit{\AIPS\ VMS Support}

     The use of \AIPS\ on VMS systems at the NRAO has decreased
dramatically in recent years as faster non-VMS machines have become
available.  Few VMS related problems in \AIPS\ are
now found at the NRAO and we depend on outside users to find
and report them.  The NRAO support of VMS \AIPS\ is likely to diminish
further as ancient VMS \AIPS\ machines
are phased out of operation.  We hope to be able to distribute VMS
installation tapes for the foreseeable future but the general level of
support for VMS is likely to erode further.  Support of FPS array
processors will also become very problematical.  If anyone wishes to
be a beta test site for VMS and/or APs please contact us.

\subtit{\AIPS\ X-Windows TV}

     Recently there has been much activity at the NRL and the NRAO
directed towards the development of an X-Windows implementation of the
\AIPS\ TV display.  Such a TV display should work on a wide variety of
workstations now on the market.  This display will be similar to the
\AIPS\ TV display currently available on Suns and should work over
networks that support Berkley sockets.  The initial versions of this
software is currently being tested and we hope to distribute this
software in the 15OCT90 release.  For further details contact Chris
Flatters at NRAO-Socorro.

\subtit{\AIPS\ on Crays}

   We are currently working on installing \AIPS\ on Cray computers
under UNICOS, Cray's UNIX.  This effort has been based on work done on
Cray Research facilities with collaboration from the University of
Minnesota.  An AIPS system has been developed at Cray Research and put
into operation at the U. of Minn.  This system uses the virtual TV
with the display facility being provided by a SUN workstation.

This project is going well.  AIPS has been shown to work in this
environment with a relatively small number of limitations.  Of these
the most serious are the lack of interactivity on the display and
limited access to tapes.  The display difficulties can be overcome by
separately running AIPS on the workstation.  Real-time run times on
the Cray naturally depend on the machine loading but, for a lightly
loaded Cray X-MP running tasks well suited to the Cray, the ratio of
real times to those obtained using a SUN 3/60 favor the Cray by
factors of 100 to 200.

The NRAO intends to distribute at least the partial UNICOS
implementation on the installation tapes for the 15APR90 release.  We
plan to continue this work and expect some five Cray sites to be
running AIPS by late summer.  Some, but not all, of these sites will
entertain visitor processing and further announcements about this can
be anticipated in the NRAO Newsletter.  We are discussing the
possibility of having a one day Cray AIPS school sometime next summer.
If you are interested please contact Bob Burns.  A final decision will
be made by the next AIPSletter.  Technical details about \AIPS\ on
Crays may be obtained from Kerry Hilldrup at NRAO-Charlottesville.

\vfil\eject
\subtit{User Agreement}

   Starting with the 15OCT89 release of \AIPS\ all user sites will
need to provide the NRAO with a signed ``User Agreement'' form.  This
``User Agreement'' will be a no-cost item for sites engaged in basic
research in astronomy.  The need for an \AIPS\ ``user agreement'' has
arisen for several reasons.  The most important is that we want all
\AIPS\ sites to obtain their copies of \AIPS\ from the NRAO and
thereby to be made aware of the restrictions that apply to their use
of the code and to our support of it.  For sites doing astronomical
research, these restrictions are only to maintain the proprietary
nature of the code and to direct third parties who wish to receive the
code to the \hbox{NRAO}.  Once properly signed, a User Agreement
remains valid for 5 years and does not need to be renewed before this
time.  A copy of this agreement is printed in the back of this
\AIPS Letter.  The agreement should be signed by an individual in a
position to take responsibility that the user group will follow the
agreement.  This may be a department chairman or an administrative
officer.  Mail signed forms to Amy Shepherd, NRAO, Edgemont Road,
Charlottesville, VA 22903-2475

\subtit{Personnel}

     There are two personnel changes in the \AIPS\ group this quarter;
Eric Greisen has taken a one-year leave of absence and is currently
working with the Australia Telescope in Epping, NSW.  E-Mail addressed
to Eric at NRAO during his absence will be forwarded.  Dean Schlemmer
has joined the \AIPS\ group in Charlottesville as the Software
Administrator.  Dean will
become the primary contact person in the \AIPS\ group for the \AIPS\
user community.  His E-mail address and telephone number are {\tt
dschlemm@nrao.edu} and (804)296-0352.

\subtit{Document and Software Distribution by AIPSSERV}

   The NRAO maintains a mail-based file server, AIPSSERV, which is
available for use by \AIPS\ users to fetch files from {\tt CVAX}.
Detailed instructions for using this facility may be obtained by
sending an E-mail message containing the single word ``{\tt help}'' to
one of the following addresses:\hfil\break
{\tt aipsserv@nrao.edu},
{\tt aipsserv@nrao.bitnet},
{\tt ...!uunet!nrao.edu!aipsserv}
or {\tt 6654::aipsserv}.

   We intend to use this facility to distribute text files such as (a)
the data files needed as input to the ionospheric Faraday radiation
correction, (b) notes about probems
encountered in installation procedures, and (c) the contents of the
CHANGE.DOC files (documentation of software changes).  A general guide
to special files for distribution by AIPSSERV will be kept in file
DOC:README.  To obtain a text file in plain text form send AIPSSERV a
message of the form ``{\tt sendplain logical:filename.ext}'' where
{\tt logical} is the logical name of the directory and {\tt
filename.ext} is the name of the desired file.
For example to be sent a copy of the README. file send AIPSSERV the
message ``{\tt sendplain DOC:README.}''.
Multiple files may be
obtained by multiple ``sendplain'' commands one per line.

\vfil\eject
\subtit{Documentation}

   Several \AIPS\ publications are currently being revised and
will soon be available for distribution.  These items may be ordered
on the form at the end of this \AIPS Letter.

   The first two of these are the \AIPS\  \Cookbook\ chapters on
calibration of VLA and VLBI data using the \AIPS\  calibration package.
These describe in some detail the use of the calibration package of
tasks and procedures.  These are ready for distribution and
can be ordered using the form at the back of this issue.  Chapter 10
describes processing VLBI data in \AIPS\ and chapter 99 describes
calibrating VLA data.  These chapters will be sent automatically to
sites requesting the 15JAN90 release of \AIPS.  A full-scale version
of the \AIPS\ \Cookbook\ and an \AIPS\ Managers' manual are being
planned for later this year.

   The second document being revised is the programmers' manual
``Going \AIPS''.  This manual is being rewritten to reflect the many
changes made in the \AIPS\ software system during the recent overhaul.
Outstanding orders for these items will be held until the documents
are ready; these items may be ordered using the form at the end of
this issue.  Volume 1 is currently being printed and will be available
soon.

\subtit{Summary of Changes:  15 October 1989 --- 15 January 1990}

   We are no longer printing the contents of the software change
documentation file CHANGE.DOC.  The old version of CHANGE.DOC are kept
in area HIST with names CHANGED.yyr where yy are the last 2 digits of
the year and r is the release date code (A,B,C and D being 15JAN,
15APR, 15JUL and 15OCT).
Anyone wishing to see the details
previously given in these files may obtain them as described in the
article on AIPSSERV.  To obtain this documentation file for 15JAN90
send AIPSSERV the message ``{\tt sendplain DOC:CHANGED.90A}''.
A summary of the changes made to the \AIPS\
software is given in the following sections.

\smallhead{Changes of Interest to Users: 15JAN90}

A number of new tasks are introduced in this release.
{\tt RSTOR} will convolve CLEAN components with a Gaussian and
add them to an image.
{\tt SOLCL} will apply system temperature measurements for solar
observations made with the VLA.
{\tt UVMTH} will time average one uv data file and will add,
subtract, multiply or divide the averaged values to/from/by/into the
visibility data in another uv data file.
{\tt BLFIT} will solve for source and/or antenna positions from
residual phases in an {\tt SN} or {\tt CL} table.
{\tt TBIN} can read external FITS-like tables of the form written
by {\tt TBOUT} and convert them into \AIPS\ tables.
{\tt ACFIT} will determine the amplitude of antenna gains for spectral line
uv data by fitting a ``template'' spectrum to the observed
autocorrelation spectra.
A new adverb, {\tt FQTOL}, was added to {\tt DBCON} to allow user
control of the definition of {\tt FQ} ids.

   {\tt BPASS} can now divide line uv data by continuum data from another file.
The polarization calibration task {\tt PCAL} will now apply ionospheric
Faraday rotation corrections before determining the instrumental and
source polarizations.
Corrections for ionospheric Faraday rotation are now applied in any
routine that applied the polarization correction.  Faraday rotation
corrections are made using task {\tt FARAD} and ionospheric monotoring
data in files obtainable from AIPSSERV (``{\tt sendplain
AIPSIONS:TECB.yy}'' where yy are the last 2 digits of the year).
{\tt MK3IN} can now read data from polarization experiments done with the
MkIII VLBI system.
{\tt LISTR} now can have a fixed scaling for amplitude listing, separate
scaling for amplitude and RMS values and a Dec-10 like gain listing
option.
{\tt SNPLT} can now plot Doppler offsets from a {\tt CL} table.
{\tt UVFIX} can now process compressed format uv data and a correction has
been made in it's computation of the correct orientation of the field
at the standard epoch.
{\tt UVCOP} can now select either auto- or cross- correlation data to copy.
{\tt UVSUB} can now process images as large as 4096$\times$4096.
{\tt UVHGM} can now plot FQ id numbers.


\vfil\eject
A bug was fixed in {\tt VLBIN} in the lobe rotation of station ``B'' data in
MkII VLBI spectral line data.
Numerous bugs were fixed in {\tt CVEL} which corrects spectroscopic
interferometer data for the earth's rotation.
A bug was fixed in verb {\tt GET} which caused minimum match to fail if
there were more than 10 potential matches.
A bug was fixed in {\tt POPS} which caused it to abort if the user typed in
``DOWAIT=.FALSE.'' or similar constructs on machines using IEEE format
floating point.   A bug in the table
handling routines caused an error if {\tt RECAT} had been used; the routines
tried to find the files under the old slot numbers.

The \AIPS\ table access routines have been modified to recatalog
``forgotten'' \AIPS\ tables; reading a ``forgotten'' table with {\tt PRTAB}
or other task will cause it to be recataloged.
The Unix file destruction routines have been streamlined to speed up
the destruction of files.
A bug in {\tt UVFLG} was fixed that caused it to make no entries in an
{\tt FG} table if no source was specified.  A bug was fixed in the data
flagging routines which caused flags for some sources to be ignored
for some times.  {\tt WTMOD} no longer unflags bad data.  {\tt TVFLG}
had a number of bugs fixed.
Several problems were fixed in {\tt SPLIT} that caused incorrect
frequencies and u, v and w values to be written.  Several bugs were fixed
that caused {\tt MX} and {\tt UVSUB} to introduce stripes into images
if the 'GRID' option was used.  Bugs were fixed in {\tt FILLM} which
caused flagging of shadowed data to fail.
A logic error was fixed that caused {\tt UVFIX} to ignore user supplied
values of UT1-UTC and IAT-UTC.  {\tt ANCAL} now checks for blanked
values in the input {\tt CL} table before applying the new calibration.
{\tt APCLN} and {\tt RSTOR} had bugs fixed which could cause failure
when processing non square images.  {\tt UVFND} now warns about
compressed data before dying.



\smallhead{Changes of Interest to Programmers: 15JAN90}

A package of J2000 precession routines is now available; {\tt JPRECS}
is the highest level routine.
New routines {\tt HIMERG} and {\tt HIADDN} simplify the concatenating
of history files; two copies of the same file will not be written to
the output file.
Utility routine {\tt GETFQ} will get the information for a given FQID from
the {\tt FQ} table.
The axis labels for plots are now allowed to be 20 characters rather
than 8.
New routine {\tt PUTCOL} stores a given value into an \AIPS\ table entry.
Parsing routine {\tt GETNUM} now returns a value of {\tt DBLANK} when
it attempts to read a bad value.
A number of improvements were made to the DDT tests.
The terminal input routines {\tt INQ}, {\tt INQINT}, {\tt INQFLT} now
accept free format input.  {\tt APLVMS:ZMKTMP} was modified to handle
the large PIDs allowed under VMS 5.?.  The {\tt DOCWHO}, {\tt DOCGRIP}
and {\tt DOCPUBL} documentation directories which were  release
specific have been made nonrelease specific and have the new logicals
{\tt AIPSWHO}, {\tt AIPSGRIP}, and {\tt AIPSPUBL}.
Also the {\tt AIPSIONS}  directory was added for the ionospheric
monotoring data.

\smallhead{Changes of Interest to Users: 15APR90}

Several new tasks were added to 15APR90.  {\tt SPFLG} which edits data
in the time and frequency domain in a mannar similar to {\tt TVFLG}
was added.  {\tt TBAVG} will average all data at a given time over
baseline and write a new uv data file.  This is useful for measuring
the time variable flux densities of point sources.  {\tt UVBAS} fits a
``continuum'' baseline to a visibility spectrum and subtracts
it from the data.  This allows a much faster alternative to {\tt
UVSUB} for continuum subtraction for the cases in which this technique
can be applied.  {\tt MK3TX} will read text files from
Haystack format ``A'' and ``B'' tapes and write them to a disk file.


All of the \AIPS\ tasks and \AIPS\ itself can now optionally write
line printer output to a file as well as to the printer.  There was a
major upgrade to {\tt TVFLG}.  The new features include {\tt UNDO},
{\tt REDO} and new {\tt CLIP} options for flagging, saving flagging
commands in a temporary table ({\tt FC}) for undoing flags and for
recovering if the program or machine crashes, avoiding displaying long
time sequences with no data, editing autocorrelation data, and more.
The concept of windows was added to the \AIPS\ TV model.  This allows
more efficient use of workstations and greatly improves the speed of
operations such as {\tt TVMOVIE} on them.  The ``Resize'' button on
the workstation TV display now toggles between a full size display and
a smaller, user selected one.

\vfil\eject
{\tt UVFND} now has the options to display data with weights exceeding
a given value and to search both sides of the uv plane in the 'UVBX'
option.  {\tt SWPOL} will now switch the polarizations for selected
antennas; this is useful for VLBI data which was mislabeled at the
correlator.  The ionospheric monotoring data was added to the {\tt
AIPSIONS} directory.  Several new options were added to {\tt CLCOR}:
1) correct for an antenna position error, 2) insert MkIII ``manual''
phase cals and 3) correct for the difference between single- and
multi-band delays in MkIII VLBI data. {\tt PRTAB} will now handle up
to 1024 keyword/value pairs in a table header.  {\tt ANCAL} will now
process multiple IFs in a single run.  {\tt PRTUV} now works on
compressed data and a problem which occured when a file had more than
30 sources was fixed.  {\tt UVFLG} was changed to allow editing
``Channel 0'' data and then copying the {\tt FG} table to the line
data file without editing the channel ranges.  {\tt UVFLG} now also
has an option to set a mask to flag arbitrary combinations of
polarization correlation.  {\tt SDTUV} was modified to work on more
antennas.

The handling of frequencies for VLA spectral line data was improved.
{\tt FILLM} was mislabeling the frequencies of the data by half the
total bandwidth and had several other related problems.  The use of
frequencies was made consistent in {\tt FILLM}, {\tt SPLIT}, {\tt
UVCOP}, {\tt MX} and {\tt HORUS}.  Also a number of other corrections
were made in the calibration of spectral line data.  A problem in {\tt
SPLIT} writing compressed data was corrected.

A number of problems in the processing of MkIII VLBI data were fixed.
In {\tt MK3IN} the processing of phase cals had several serious bugs removed
and the specification of antennas and timeranges was made friendlier.  A
number of bugs in {\tt CALIB} and the calibration routines which occured for
MkIII data were fixed.  Bugs were fixed in {\tt BATER} which probably kept
it from working.  A frequency scaling problem in {\tt UVPLT} was fixed.

\smallhead{Changes of Interest to Programmers: 15APR90}

{\tt DGHEAD} now includes the source frequency offset if data from a
single source is selected. {\tt FUDGE} now processes compressed data.
New routines {\tt VISUNP} and {\tt VISPCK} simplify unpacking and
packing (compressed) uv data.  Several serious problems in {\tt
FILAI2}, used in \AIPS\ VMS installations, were fixed.  A makefile
(MAKEFILE) was added to {\tt YSVU} to assist in the creation of SSS
for Sun TVs.  The current logical \AIPS\ version (TST, NEW and OLD) was
added to DMSG.INC and to the accounting file.  This allows {\tt PRTAC}
to distinguish between different releases of \AIPS.

\vfill\eject
\pgskip
\pgskip
\input order.tex
\end



