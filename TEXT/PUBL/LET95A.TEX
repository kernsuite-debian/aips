%-----------------------------------------------------------------------
%;  Copyright (C) 1995
%;  Associated Universities, Inc. Washington DC, USA.
%;
%;  This program is free software; you can redistribute it and/or
%;  modify it under the terms of the GNU General Public License as
%;  published by the Free Software Foundation; either version 2 of
%;  the License, or (at your option) any later version.
%;
%;  This program is distributed in the hope that it will be useful,
%;  but WITHOUT ANY WARRANTY; without even the implied warranty of
%;  MERCHANTABILITY or FITNESS FOR A PARTICULAR PURPOSE.  See the
%;  GNU General Public License for more details.
%;
%;  You should have received a copy of the GNU General Public
%;  License along with this program; if not, write to the Free
%;  Software Foundation, Inc., 675 Massachusetts Ave, Cambridge,
%;  MA 02139, USA.
%;
%;  Correspondence concerning AIPS should be addressed as follows:
%;          Internet email: aipsmail@nrao.edu.
%;          Postal address: AIPS Project Office
%;                          National Radio Astronomy Observatory
%;                          520 Edgemont Road
%;                          Charlottesville, VA 22903-2475 USA
%-----------------------------------------------------------------------
%Body of \AIPS\ Letter for 15 January 1995

\documentstyle [twoside]{article}

\newcommand{\AMark}{AIPSMark$^{(93)}$}
\newcommand{\AMarks}{AIPSMarks$^{(93)}$}
\newcommand{\LMark}{AIPSLoopMark$^{(93)}$}
\newcommand{\LMarks}{AIPSLoopMarks$^{(93)}$}
\newcommand{\AM}{A_m^{(93)}}
\newcommand{\ALM}{AL_m^{(93)}}

\newcommand{\AIPRELEASE}{January 15, 1995}
\newcommand{\AIPVOLUME}{Volume XV}
\newcommand{\AIPNUMBER}{Number 1}
\newcommand{\RELEASENAME}{{\tt 15JAN95}}

%macros and title page format for the \AIPS\ letter.
\input LET94.MAC
\input psfig

\newcommand{\MYSpace}{-11pt}

\normalstyle

\section{The Good News $\ldots$}

The \RELEASENAME\ release of Classic \AIPS\ is now available.  Contact
Ernie Allen at any of the addresses given in the masthead to obtain a
copy.  As of this writing, 87 copies of the {\tt 15JUL94} release
have been given out electronically (21 tar.Z, 19 tar.gz) or on
magnetic tape (32 8mm, 10 4mm, 4 QIC, 1 9-track).  The increase in
magnetic tapes over network transmissions is due to the availability
of a full binary distribution, but only on magnetic tape.  36 sites
took advantage of this with the {\tt 15JUL94} release (20 Sun OS, 10
Solaris, 2 IBM AIX, 1 DEC Alpha, 3 Linux).

The new release contains 12 new tasks, including significant ones for
``on-the-fly'' imaging using the NRAO 12-meter telescope and for VLBI
spectral-line polarization calibration, fringe-rate imaging, and data
display.  Subtle effects in VLBA processing, resulting in not-so-subtle
data problems, were corrected in bandpass calibration and in delay and
rate smearing of amplitudes.  The number base used to name data files
was changed from 16 to 36, allowing for many more extension files of
each type per catalog entry, plus more simultaneous disks, tapes, etc.
This change will require the execution of {\tt UPDAT}, the old format
correction program which we have not used in five years (or so).

Completely updated chapters on displays and spectral-line data
reduction have been placed in the \Cookbook.  In addition, new
appendices, provided by people outside the \AIPS\ Group, have been
added to provide more genuinely ``cookbook'' views of continuum and
line processing.

\section{$\ldots$ and the Bad}
%\section{\AIPS\ User Agreements}

The \AIPS\ User Agreement has been required for non-NRAO sites to
obtain a copy of \AIPS\ and was designed to protect NRAO's
intellectual property rights.  It also helps protect all \AIPS\ users
from having their use of \AIPS\ compromised by such things as having
\AIPS\ locked  up in legal proceedings (which can last for years).
This Draconian agreement may no longer be essential --- the \AIPTOO\
group plans a GNU-like license --- but it appears that we are stuck
with it.  One aspect of this Agreement is that ``This Agreement shall
remain in effect for a period of five (5) years from the date
hereof.''  Since the first Agreements were signed late in 1989, they
are now expiring.  If your site has had \AIPS\ since then, it may have
to sign a new Agreement to receive the {\tt 15JAN95} and subsequent
release.

Each month, Ernie Allen notifies those sites whose licenses have
expired and requests that they execute a new one.  He also sends a
site information form, requesting \AIPS\ Managers to bring our
information on their site up to date.  We appreciate your cooperation
with these onerous details.
\vfill\eject

\section{AIPS++ Review}

The \AIPTOO\ project was reviewed by a panel consisting of Ray Offen,
Wim Brouw, James Coggins, Tim Cornwell, Dennis Gannon, and Bob
Hanisch.  They met, along with many ``observers'' and members of the
\AIPTOO\ project, in Charlottesville on December 12 and 13.  Their
full report and a reply from the Steering Committee have been checked
into an anonymous ftp area in Charlottesville.  The anonymous ftp
access is to {\tt aips2.cv.nrao.edu} in directory {\tt
pub/aips++/RELEASED/review-docs}.  The World-Wide Web URL is therefore
{\tt ftp://aips2.cv.nrao.edu/pub/aips++/RELEASED/review-docs}.

\section{\Cookbook\ Update Continues}

     The \AIPS\ \Cookbook\ was last updated for the {\tt 15OCT90}
release.  Because a lot has changed in \AIPS\ since then, we have
decided to modernize the \Cookbook.  We are doing this one chapter at
a time and are making each chapter available via the World-Wide Web as
soon as it is ready.  For details of the Web, see the publications
article in this \Aipsletter.  The chapters changed so far are
\vspace{-8pt}
\begin{itemize}
\item\ 1 --- {\it Introduction} --- Added new sections giving a
   project summary and a diagram of the structure of \AIPS.
\item\ 2 --- {\it Starting Up \AIPS} ---  Changed to describe
   workstation use, \AIPS\ in networked environments, and managing the
   TV server \hbox{{\tt XAS}}.
\item\ 3 --- {\it Basic \AIPS\ Utilities} --- Updated information about
   history files and disk allocation, added {\tt ABOUT} and {\tt
   APROPOS} to the help section, moved and updated tape mounting, and
   added a discussion on external disk files (Fits, text, $\ldots$).
\item\ 4 --- {\it Calibrating Interferometer Data} --- With much help
   from Rick Perley and Alan Bridle, rearranged and corrected
   everything, adding a substantial discussion of when and how to edit
   and bringing the description of {\tt TVFLG} up to date including a
   picture.
\item\ 7 --- {\it Displaying Your Data} --- Rewrote old chapters 7
   and 8, to make a coherent, current, and complete description of
   printing, plotting, TV, and graphical data displays.  Renumbered all
   chapters after 8.
\item\ 9 --- {\it Spectral-Line Software} --- Rewrote old chapter 10,
   replacing old outline format with a more coherent (and wordy)
   description of line analysis, emphasizing continuum subtraction and
   other more modern imaging techniques.
\item\ 14 --- {\it Current \AIPS\ Software} --- Replaced old lists with
   new ones produced for the {\tt ABOUT} verb.
\item\ A --- {\it Summary of \AIPS\ Continuum UV-data Calibration} ---
   Inserted a new appendix giving an updated version of Glen
   Langston's outline of continuum calibration.
\item\ B --- {\it A Step-by-Step Guide to Spectral-Line Data Analysis
   in \AIPS} --- Inserted a new appendix by Andrea Cox and Daniel
   Puche giving {\it their} outline view of spectral-line data
   reduction in \hbox{\AIPS}.
\item\ Z --- {\it System-Dependent \AIPS\ Tips} --- Replaced with whole
   new discussions including color printers, screen copying, film
   recorders, workstation environments.  A method for people to have
   NRAO make slides for them is described.
\end{itemize}
\vspace{-8pt}
\noindent There exists a draft of Chapter 10, {\it Reducing VLBI Data
in \AIPS}, which will be released in the relatively near future.  The
VLBI chapter is a major rewrite to account for the advent of the
\hbox{VLBA}.  It is our hope to update the other chapters, add a
chapter on single-dish data in \AIPS, and even to add an index to the
\Cookbook.

\vfill
\eject

\section{Improvements for Users in 15JAN95}

\subsection{Format change}

The first format change since the code overhaul (completed in 1989) is
included in the {\tt 15JAN95} release.  In a sense, it is a relatively
minor format change.  No file has had its {\it contents} changed in a
significant way, but virtually all data files have had their {\it
names} changed.  (The image header files had some previously unused
areas initialized in anticipation of a new system of coordinates.)
Before the format revision, file names had the form
$$
     \hbox{{\it tt\/{\tt C}mmmnn{\tt .}uuu{\tt ;1}}}
$$
where {\it tt} is the file type (\eg\ {\tt MA}), {\it mmm} is the
catalog slot number in hexadecimal, {\it nn} is the version number in
hexadecimal, and {\it uuu} is the user number in hexadecimal.  With
this form, user and slot numbers were required to be less than 4096
and version numbers were required to be less than 256.  The former
were worrisome, but the latter restricted spectral-line Cleans to 255
channels, a very serious limitation.  After the format revision, file
names have the form
$$
     \hbox{{\it tt\/{\tt D}mmmnnn{\tt .}uuu{\tt ;}}}
$$
where {\it mmm} is the catalog slot number in ``extended
hexadecimal,'' {\it nnn} is the version number in extended
hexadecimal, and {\it uuu} is the user number in extended hexadecimal.
This numbering system is base 36, using the characters 0 through 9,
followed by A through \hbox{Z}.  The new limit on user numbers, catalog
slots, {\it and} versions is 46655, which should be sufficient.

This switch from regular hexadecimal to extended has a number of other
consequences.  To assist users (and Unix procedures) in determining
numbers in this system, two procedures were written: {\tt EHEX~{\it
n}} converts {\it n} from decimal to base 36, while {\tt REHEX~{\it
m}} converts {\it m} from base 36 to decimal.  Logical names for
disks, tapes, display devices, and the like in \AIPS\ have used
hexadecimal for some time and now use extended hexadecimal.  This
means that users may have up to 35 disks in a single \AIPS\ session
and that a local environment may have up to 1295 workstations rather
than the previous limit of 255.

To perform the format conversion, only two simple steps are required.
After {\tt 15JAN95} is installed (complete with new system files for
all hosts, using {\tt SYSETUP} if necessary), but before any users are
allowed on it, the local \AIPS\ Manager runs the stand-alone program
{\tt UPDAT} over all user and disk numbers.  Then, each user renames
his or her RUN files changing the user-number extension from base 16
to base 36 with the help of \hbox{{\tt EHEX}}.  More details are
included in the installation documentation.

\subsection{VLBI data processing}

\subsubsection{Spectral-line polarization calibration}

A new task, {\tt SPCAL}, has been implemented to perform instrumental
feed calibration for spectral-line polarization VLBI data where the
program source has low or moderate linear polarization.  A subset of
velocity channels in the program source cross-power spectrum is used
in a composite fit for the feed terms.  This task may not be
appropriate for all spectral-line polarization VLBI data, but is part
of a continuing effort to expand the software available in this area
within \hbox{\AIPS}.  For further information please contact Athol
Kemball ({\tt akemball@nrao.edu}).

\subsubsection{Fringe-rate mapping}

A multiple-point fringe-rate mapping task, {\tt FRMAP}, has been
implemented in the {\tt 15JAN95} release.  This task uses a new
fringe-rate mapping algorithm which is less sensitive to erroneous
peaks in the fringe-rate spectra caused by missing data.  The map area
is subdivided into rectangular sub-regions and the number of lines
crossing each of these regions is used to define the initial component
positions.  Final positions are determined using a least-squares
minimization.  The task produces an output file listing the right
ascension, declination, and estimated flux density of each component.
A graphical display of the lines is also produced.  For further
information please contact Leonia Kogan ({\tt lkogan@nrao.edu}).

\vfill
\eject
\subsubsection{VLBA bandpass calibration}

In the VLBA correlator, fringe rotation is applied to both stations on
each correlated baseline, with the Earth's geocenter as the coordinate
reference point.  This differs from previous correlators where only
one station had fringe rotation applied.  As a consequence, the
differing VLBA coordinate reference introduces a time-variable
frequency offset in the autocorrelation and cross-correlation spectra
with respect to the recorded edge frequency at each station.  {\tt
BPASS} has been modified to take the time-variable offset in the data
into account in determining the bandpass solutions so that the {\tt
BP} entries are at a fixed frequency and can be averaged in time.  The
offset is taken into account again in applying the bandpass correction.

This mode is triggered in {\tt BPASS} if the array name keyword in the
{\tt AN} table is ``\hbox{VLBA}.''  The user is warned when this mode
is entered.  At present in {\tt BPASS}, only autocorrelation bandpass
determination incorporates the VLBA  frequency offset corrections
fully.  Cross-correlation bandpass determination for data from all
correlators is affected by the time-variable fringe rotation of at
least one of the stations on any baseline.  This effect appears only
to be significant at high frequencies ($> 22$ GHz) and for narrow
bandwidths ($< 1$ MHz).  Baseline-based methods may be necessary in
this case and this is a research area within \AIPS\ at present.
Please contact Phil Diamond ({\tt pdiamond@nrao.edu}) for further
information; an \AIPS\ Memo is planned.

\subsubsection{Baseline-oriented fringe fitting}

The baseline-oriented fringe-fitting tasks {\tt BLING} and {\tt BLAPP}
have been extensively modified for the {\tt 15JAN95} release of
\hbox{\AIPS}.  The good news is that {\tt BLING} should fail much less
often than before and will give more realistic error estimates and
that {\tt BLAPP} will detect antenn\ae\ that cannot be connected to the
reference antenna and will flag solutions for such antenn\ae\ rather
than assigning them random delays and rates.

If you use {\tt BLING} and {\tt BLAPP} you should note that the inputs
to {\tt BLING} have changed.  In particular, the coherence factor
(expressed as a percentage) is now used as the acceptance criterion
rather than the signal-to-noise ratio.  The default value deliberately
errs on the conservative side and you may need to use lower values for
weak sources.  You should also note that the {\tt BS} table now
carries more information than it did in earlier releases of \AIPS\ and
that, as a consequence, the {\tt 15JAN95} version of {\tt BLAPP}
cannot handle {\tt BS} tables from earlier versions of \hbox{{\tt
BLING}}.

The following suggestions should help you make the best use of the new
\hbox{{\tt BLING}}.
\vspace{-10pt}
\begin{itemize}
\item Run {\tt BLING} on a machine that does not penalize
  double-precision arithmetic (\eg\ an IBM RS/6000 or DEC Alpha)
  whenever possible; {\tt BLING} makes heavy use of double precision
  during the chi-squared fit and takes a relatively large performance
  hit on machines where double-precision arithmetic is slower than
  single-precision (\eg\ SPARCs).
\item Apply {\it a priori} amplitude calibrations before running
  \hbox{{\tt BLING}}.  If you   don't do this, the data weights will
  not reflect the expected noise in the data and {\tt BLING} will fail
  spectacularly (in some cases it may crash).
\item Don't turn on the fringe acceleration search ({\tt DPARM(7)} to
  {\tt DPARM(9)}).  This is a special option for space VLBI and will
  merely slow {\tt BLING} down and degrade the quality of the
  solutions for ground-based arrays.
\end{itemize}
\vspace{-10pt}
Further information about the revised editions of {\tt BLING} and {\tt
BLAPP} is available in \AIPS\ Memo 89 (see below).

\subsubsection{VLBA data handling}

As usual, {\tt FITLD} received a number of improvements for this
release.  The option to select spectral channels from each IF was
added as was the ability to remove the FFT artifacts from total-power
spectra.  (These artifacts are washed out in the correlator on
cross-power spectra.)  The pulse-cal table was defined and a new task,
{\tt PCLOD}, was written to read the ASCII-format pulse-cal tables
generated from the VLBA monitor system.  Task {\tt SNPLT} was enhanced
to enable it to display the contents of {\tt PC} tables and software
is now under development to apply these VLBA phase-cal tables to the
data.  The \uv-data display task {\tt SHOUV} was changed to display
closure phases with increased accuracy.

\eject
\subsubsection{VLBA amplitude correction}

An improvement has been made to the calibration of VLBA data within
\AIPS, by allowing an amplitude correction for data that have been
averaged in frequency in the VLBA correlator in the presence of
uncorrected residual delays.  This correction is significant only for
the higher channel bandwidths (\eg\ about 0.5--1 MHz, as obtained with
16 spectral channels or fewer and an 8 MHz or 16 MHz BBC filter
setting).  The correction is of order
$$
     \hbox{sinc}\, \left(\pi \times \hbox{Channel\_bandwidth/Hz} \times
            \hbox{Residual\_delay/s} \right) \, .
$$
Residual delays for the VLBA are generally low and, to date, this
effect has only been seen clearly in one dataset.  Data from MkIII and
MkII correlators are not affected.

This correction is applied only if the {\tt AN} table specifies that the
array name is ``VLBA'' and there exists an additional keyword {\tt
SPEC\_AVG} in the {\tt AN} table header which records the factor by
which the data have been pre-averaged in frequency.  A value greater
than unity is required to trigger the correction.  For full
polarization data, it is important to correct for any delay offsets
between the RCP and LCP delay solutions before applying this
correction.  Please contact Athol Kemball ({\tt akemball@nrao.edu})
for further information.

\subsection{Single-dish data in \AIPS}

Single-dish data analysis in \AIPS\ had fallen into disuse since the
programs that translated some Green Bank data into readable files were
lost with the Convexes.  The current ``single-dish binary-tables
FITS'' format involves many more than \AIPS' limit of 128 columns
(although few are used or even initialized) and does not tell the
truth about the actual contents of the data arrays.  It would appear
that this FITS format and the UniPops SDD format on which it is based
can only be used by software specialized to the observing mode of
interest.  For the {\tt 15JAN95} release, one such specialized task
was written.  It is called {\tt OTFUV} and translates SDD binary files
from the NRAO 12-meter when it is observing in ``on-the-fly'' mode (up
to four spectra so far every 0.1 seconds as the telescope is
continually re-pointed).  {\tt OTFUV} applies associated ``off'' and
gain scans to the data as they are read.

To make use of these voluminous data, a new data-gridding task {\tt
SDGRD} was written.  It is an efficient combination of data selection,
sorting (if needed), weighting, gridding, correcting, and cube
building.  In building this task, it was necessary to change the
single-dish data handling to support compressed data and flag tables.
Flagging tasks such as {\tt UVFLG} and {\tt TVFLG} needed modification
to support these data as well (see below).  {\tt DBCON} was changed so
that it would not trash single-dish data coordinates and {\tt PRTSD}
was changed to print more accurate times and to convert the ``beam''
into a pseudo-antenna number if it has the correct pattern ($257$
times an integer).  The new task {\tt BASRM} (see below) was intended
for VLBA users, but has obvious applications to single-dish
spectroscopy.

\subsection{$UV$ data calibration and manipulation}

\subsubsection{Calibration}

The new task {\tt BASRM} was written to copy a \uv\ data set, removing
a spectral baseline from total power data along the way.  Cross-power
data are untouched by this operation.  The user is able to fit and
remove an $n^{\uth}$-order polynomial from line-free channels of the
total-power spectra.

{\tt UVLSF} was improved to offer a couple of options for flagging
data based on the quality of the fits to the \uv-spectral baseline.
It was also given the option to shift data phases for a coordinate
shift before the baseline fit (and shift the phases back afterwards).

Two significant bugs were corrected in \hbox{{\tt CALIB}}.  The mean
gain modulus was computed separately for each sub-array and the last
sub-array's value was written to the table.  It has been corrected to
average over all sub-arrays.  The other bug was an error in the range
of a {\tt DO} loop causing it to loop over much too large an array
with spectral-line data.  This caused addressing problems and might,
on some computers, have caused mysterious corruptions.  {\tt SNPLT}
was multiplying by the mean gain when it should have been dividing.
{\tt ANCAL} was made rather more robust in its handling of missing and
blanked table and input entries.

\subsubsection{Single-source files and data editing}

For some reason, a number of \uv-data tasks forbade single-source files
from using flag and other calibration tables, while the rest happily
applied them to single-source files.  For the {\tt 15JAN95} release,
these arbitrary restrictions were removed from {\tt SPLIT}, {\tt
CALIB}, {\tt FRING}, and \hbox{{\tt HORUS}}.   {\tt TVFLG}, {\tt
SPFLG}, {\tt IBLED}, and {\tt UVFLG} were changed to use {\tt FLAGVER}
even for single-source files, creating a flag table if needed and {\tt
FLAGVER} $ > 0$.  They will write in a flag table if one already
exists.  Only if {\tt FLAGVER = 0} and no {\tt FG} table exists will
they actually flag the single-source data.

The universal problem in gridding irregularly-spaced data is where to
put the boundaries of the grid cells.  {\tt TVFLG} and {\tt SPFLG}
were basing that decision on the first data sample alone, which could
cause it to get severely out of step with later samples.  They have
been changed to examine a significant number of data samples at the
beginning of the data set and use the peak in the histogram of their
times (modulo the cell size) to control the positioning.  Both of them
had trouble averaging across times with no samples, which can arise
even with the improved grid construction.  The averaging routines
were improved to ignore missing data rather than terminating the
summation.

{\tt UVFLG} was changed to support compressed \uv\ data and to flag
single-dish data if requested rather than forcing the use of a flag
table.  The option to limit flagging (or unflagging) only to those
samples outside a specified range in amplitude and/or inside a
specified range in weight was also added.

\subsubsection{\UV\ display and data handling}

A new version of {\tt VBPLT} was written.  It is called {\tt VPLOT}
and is very much faster than the old task.  {\tt VPLOT} also offers
the new option of plotting both amplitude and phase simultaneously.

{\tt UVCOP} was improved to correct output tables for the various data
selection parameters applied.  A user-set option and other controls
were added to estimate the output file size to avoid the slow process
of creating an enormous output file when only a small file is
required.  {\tt UVDIF} was corrected to report differences in data
flagging.  A typo was causing it to ignore these important
differences.

\subsection{Imaging}

Two new tasks were written as aid to the imaging process.  {\tt CONPL}
plots the \AIPS-standard convolution functions and a selection of the
consequences of them (\ie\ the FFT of the function, the expected
signal-to-noise for uniformly sampled synthesis imaging, or the
convolution of the function with a Gaussian).  This turned out to be
surprisingly informative and should be of special interest to
single-dish users of \hbox{{\tt SDGRD}}.  The other new task, {\tt
IMCLP}, limits the values in an image to a specified range, replacing
pixel values outside the range with the closest value in the range.

Boxes to limit searches by Clean algorithms were renamed {\tt CLBOX}
in order to allow up to 50 such areas to be specified.  For {\tt
WFCLN} only, an option was added to read from a text file up to 50
boxes for each of the 16 simultaneous fields allowed.

\subsection{Image analysis and display}

There were several relatively minor changes in this area.  {\tt IMEAN}
now attempts to compute the true signal-free rms of an image by
fitting a Gaussian to the peak in the histogram.  It may require two
passes through the data to do this and does not depend on the
histogram plot option.  {\tt SAD} also fits this true noise level, but
now it adds it as a keyword to the output model fit ({\tt MF}) file.
A new task called {\tt MF2ST} converts selected components from this
{\tt MF} file into the stars-file format used by many plotting
programs.  The {\tt STARS} task and the {\tt ST} file itself were
changed to label the width columns as major and minor axes, to treat
the position angle correctly as East from North, and to stop dividing
by the cosine of the declination (since the width is not a right
ascension).

The new task {\tt SKYVE}, contributed by Mark Calabretta, will regrid
a Digitized Sky Survey image to a coordinate frame and projection
recognized by \hbox{\AIPS}.  The DSS is based on photographic material
obtained using the UK Schmidt Telescope and was produced by the Space
Telescope Science Institute.  DSS images may be extracted from the CD
set as FITS files by a program called {\tt getimage} and read into
{\tt AIPS} with {\tt IMLOD} or \hbox{{\tt FITLD}}.  {\tt SKYVE}
retrieves the plate solution parameters from the \AIPS\ history file
and regrids the image into a coordinate system recognized by
\hbox{\AIPS}.

\subsection{Miscellaneous changes of interest to users}

%\vspace{-10pt}
\begin{description}
\myitem{MOVE} New task to copy or move all of one catalog slot to
   another catalog slot belonging to the current user or to any other
   user.  It is faster than {\tt SUBIM} and {\tt UVCOP} and, unlike
   them, copies all files of any kind without examination or
   modification and can give the data to another user number.
\myitem{TVCPS} This task now selects landscape or portrait modes
   depending on the relative sizes of the input and output images.  It
   can be forced to use portrait mode with \hbox{{\tt APARM(9)}}.
\mylitem{TV pointing} The verbs {\tt TVPOS}, {\tt IMXY}, {\tt IMPOS},
   and probably others now prompt you to move the TV cursor and press
   any button.
\myitem{holography} {\tt HOLGR} was improved to offer a phase model
   appropriate to antenn\ae\ with sub-reflectors in addition to the
   older model appropriate to prime-focus antenn\ae.  {\tt UVPRT} was
   changed to simplify the automatic flagging and to allow the option
   to have multiple reference antenn\ae\ for a single output antenna.
\end{description}

\section{Improvements Primarily for Programmers in 15JAN95}

Other matters of interest to programmers in \RELEASENAME\ include
\vspace{-10pt}
\begin{description}
\myitem{functions} Most Fortran compilers take exception to the use of
   the function names {\tt IAND}, {\tt IOR}, and {\tt IEOR} since they
   think they know what they mean and that that meaning is not within
   the ANSI standard.  We have always provided subroutines to do these
   functions, but it is not clear that those subroutines were actually
   used in all cases.  Therefore, we changed all function references
   to {\tt ZAND}, {\tt ZOR}, and {\tt ZEOR}, respectively, and renamed
   our function subroutines.
\myitem{headers} New parameters were added to the image headers in
   anticipation of changing over to the new proposed standard for
   coordinates.  They are {\tt KRCOK} for a code to indicate that the
   other WCS parameters are usable , {\tt KDLON} for the value of {\tt
   LONGPOLE}, {\tt KDPRJ} for up to nine projection parameters ({\tt
   PROJP1} etc.), and {\tt KRPCM} for the {\tt PC} 7 by 7 pixel
   conversion matrix. There are still 17 free words in the header!
\myitem{ZCREA2} Unix systems require \AIPS\ to write to all records of
   a file in order to reserve the disk space requested.  \AIPS\ did
   this, but neglected to control what it wrote.  Now it writes zeros.
   Programmers should not depend on this since not all systems will be
   Unix forever.
\myitem{ZFIO}  This basic I/O routine attempts to read 1024 8-bit
   bytes in each operation.  In order to allow it to be used to read
   non-\AIPS\ binary data files, it was changed to accept (and report
   through the error return) a partial data record, presumably at the
   end of the file.
\myitem{ZDAOPN} This fundamental open routine for disk files attempted
   to open all files with both read and write access.  It has been
   changed to attempt a read-only open when exclusive use is not
   requested and the read/write open fails.  This allows us to
   circumvent privilege issues for files that only need to be read.
\myitem{TOUCH} Some algorithms keep scratch files open for very long
   times during enormous computations.  In fact, these times got so
   long that the files could be deleted by execution of {\tt TIMDEST}
   from another computer in the local network.  File locking does not
   always work across NFS and is not a reliable defense.  The new {\tt
   TOUCH} routine is designed to be called periodically to update the
   last access times of all standard scratch files within the calling
   program to defend them from precipitate deletion.
\end{description}

\vfill
\eject

\section{AIPS Publications and the World-Wide Web}

     There has been a virtual explosion in the use of the {\it
World-Wide Web\/} (WWW) protocol on the Internet.  It is a method for
sending hypertext over the network and has been made easy to use by
clients such as {\it NCSA Mosaic, Netscape, Arena,\/} and {\it
Lynx\/}.  NRAO is among the many institutions which now offer
informative Web pages and networks of additional information.  The
NRAO ``home'' page is at the Universal Resource Locator (URL) address
\begin{center}
\vskip -20pt
{\tt http://www.nrao.edu/}
\vskip -10pt
\end{center}
This page can lead you to information about each of the NRAO's sites
and telescopes, library system, major new initiatives, software
packages, phone directory, and Newsletter as well as information about
other astronomy resources on the Internet and about the Web's mark-up
language called {\tt html}.  The \AIPS\ group home page
may be found from the NRAO home page or addressed directly at URL
\begin{center}
\vskip -10pt
{\tt http://www.cv.nrao.edu/aips/}
\vskip -10pt
\end{center}
This page points at basic information (``What is AIPS?'' and a
``FAQ''), news items about \AIPS\ (such as ``15JAN95 Release imminent
$\ldots$''), the PostScript text of recent \AIPSLETTER s, patch
information for all releases after {\tt 15JAN91}, information about
known bugs, the latest \AIPS\ benchmark data from various computer
systems, copies of {\tt CHANGE.DOC} for every release since {\tt
15JAN90}, and {\it all} relevant \AIPS\ Memos, {\it every} chapter of
the \Cookbook, and all recent quarterly reports to the \hbox{NSF}.  We
recommend that you check this area occasionally since it changes with
time.  There is a new tool which allows you to read any help file from
the latest help area ({\tt 15JUL95} now).  This could be turned into a
full WAIS service if enough of you request it.  As we correct and
update the \Cookbook\ and as we write new \AIPS\ Memos and reports, we
place the documents in PostScript forms in the area used by WWW and we
have the {\tt html} listings, indices, tables of contents, and the
like updated to reflect the additions.  In this way, you do not have
to wait until {\tt 15JUL95} to get the latest \Cookbook\ chapters.  We
expect, for example, to have a revised version of the VLBI chapter in
the very near future.

Below is a list of the latest \AIPS\ Memos, of which only Memo 89 is
new with this release.
\begin{center}
\vspace{-6pt}
\begin{tabular}{ccl}
\hline
Memo  &        Date   & Title and author  \\
\hline\hline
  86 & 94/03/16 & Wide-field Polarization Correction of VLA Snapshot
                    Images at 1.4 GHz \\
     &          & \qquad W. D. Cotton, NRAO \\
  87 & 94/04/05 & The NRAO \AIPS\ Project --- a Summary \\
     &          & \qquad Alan H. Bridle, Eric W. Greisen, NRAO \\
  88 & 94/05/16 & The \AIPS\ Gripes Database \\
     &          & \qquad W. D. Cotton, Dean Schlemmer, NRAO \\
  89 & 94/11/17 & Baseline-Oriented Fringe Searches in \AIPS \\
     &          & \qquad Chris Flatters, NRAO \\
\hline
\end{tabular}
\end{center}
\vspace{-6pt}
A heavily revised edition of the Memo on Object-Oriented Programming
in \AIPS\ is available as file {\tt AIPSOOF.TEX} and, in PostScript
form, as \hbox{{\tt AIPSOOF.PS}}.

Since some Memos are not available electronically and others do not
yet have computer readable figures, you may wish to write for a paper
copy of these.  To do so, use an \AIPS\ order form or e-mail your
request to {\tt aipsmail@nrao.edu}.

If you cannot use the Web, you can still use \ftp\ to retrieve the
Memos, \Cookbook\ chapters, etc.:
\begin{description}
\vspace{-10pt}
\item{ 1.} {\tt ftp baboon.cv.nrao.edu}  or  {\tt 192.33.115.103}
\item{ 2.} Login under user name anonymous and use your e-mail address
           as a password.
\item{ 3.} {\tt cd pub/aips/TEXT/PUBL}
\item{ 4.} Read {\tt AAAREADME} for more information.
\item{ 5.} Read {\tt AIPSMEMO.LIST} for a full list of \AIPS\ Memos.
\end{description}

\AIPS\ Memos from Number 65 through 89 are present in this area as are
Numbers 27, 33, 35, 39, 46, 51, 54, 61, and 62.  We have been filling
in this list gradually, by finding and fixing old files in other areas
of the authors' disks, by scanning in text and figures, or by retyping
text and redrawing the figures.  The \Aipsletter s from 1991 through
the present are also available in this area.  Many of the Memos are in
both \TEX\ and PostScript forms, with the \TEX\ ones stored in a
subdirectory called \hbox{{\tt TEX}}.  Note that many, if not all of
these may be found on your home \AIPS\ system in an area called
\hbox{{\tt \$AIPSPUBL}}.  All Memos are available in paper form from
Ernie Allen at the addresses in the masthead.

The latest version of the \AIPS\ \Cookbook\ is also available (in the
form of PostScript files) in this area.  Initially the chapters from
the 1990 version of the \Cookbook\ were placed in this area;  whenever
one of these chapters is updated, the latest version will be available
immediately in this area.  Updated so far are chapters 1
(Introduction), 2 (Starting Up \AIPS), 3 (Basic \AIPS\ Utilities), 4
(Calibrating Interferometer Data), 7 (Displaying Your Data), 9
(Spectral-Line Software), 14 (Current \AIPS\ Software), A (Continuum
Calibration), B (Spectral-line Data Analysis), Z (System-Dependent
\AIPS\ Tips).  Chapter 10 (Reducing VLBI Data in \AIPS) is nearly
ready as well.  The remaining old chapters were revised to include
figures in the PostScript and improve the typesetting, but are full of
outdated information in addition to the good stuff.

\section{Patch Distribution}

Since \AIPS\ is now released only semi-annually (or even less
frequently), we make selected, important bug fixes and improvements
available via {\it anonymous} \ftp\ on the NRAO Cpu {\tt baboon}
({\tt 192.33.115.103}).  Documentation about patches to a release is
placed in the anonymous-ftp area {\tt pub/aips/}{\it release-name} and
the code is placed in suitable subdirectories below this.
(The patches and their documentation are also available on-line via
the World-Wide Web.)  Reports of significant bugs in {\tt 15JUL94}
\AIPS\ were less numerous than in the previous release, and many of
the patches were actually for new or improved code rather than bug
fixes.  The documentation file {\tt pub/aips/15JUL94/README.15JUL94}
mentions the following items:
\begin{description}
\vspace{-8pt}
\myitem{gains} {\tt CALIB.FOR} was corrected to average the gains over
   all sub-arrays rather than just the last one.  It was also
   corrected to average over the number of IFs being averaged.  It
   wrongly averaged over the number of spectral channels in an array
   that was not that large.  {\tt SNPLT.FOR} was multiplying by the
   mean gain when it should have been dividing.
\myitem{flagging} {\tt SPLIT.FOR}, {\tt CALIB.FOR}, {\tt FRING.FOR},
   and {\tt HORUS.FOR} refused to apply flag and other tables to
   single-source data sets.  Since most other tasks do what they are
   told, these arbitrary restrictions were removed.  Changed {\tt
   TVFLG.FOR}, {\tt SPFLG.FOR}, {\tt IBLED.FOR}, and {\tt UVFLG.FOR}
   to make and use flag tables on single-source data sets if
   requested.
\myitem{WFCLN} {\tt WFCLN.FOR} set the number of channels to average
   in ``SUM'' mode incorrectly.
\myitem{single-dish} {\tt SDGET.FOR}, {\tt DGETSD.FOR}, and {\tt
   PRTSD.FOR} were corrected to handle compressed data and to select
   IFs and channels correctly.  New tasks {\tt OTFUV}, {\tt SDGRD},
   and {\tt CONPL} were written to support on-the-fly mapping.
   Subroutines {\tt ZFIO.FOR}, {\tt ZFI2.C}, and {\tt ZDAOPN.C} were
   modified to handle partial data records and read/write privilege
   problems.  {\tt DBCON.FOR} was corrected to stop damaging
   single-dish coordinates by rescaling and re-labeling them as \uv\
   coordinates.
\myitem{UVDIF} {\tt UVDIF.FOR} failed to report differences in
   flagging between the two input data sets.
\myitem{UVLSF} {\tt UVLSF.FOR} was changed to flag data based on the
   quality of the fit baseline.
\mylitem{coordinates} {\tt DIRCOS.FOR} was corrected to rotate large
   differences between {\tt RA} and {\tt RA0} by $360^{\circ}$ rather
   than simply declare failure.
\myitem{linking} {\tt LINK} was corrected to use the desired versions
   of {\tt LIBR.DAT} depending on whether the module is to use shared
   or static libraries.
\end{description}
\vspace{-8pt}
Note that we did not revise the original {\tt 15JUL94} tape or \tar\
files for these patches.  No matter when you received your {\tt
15JUL94} ``tape,'' you must fetch and install these patches if you
require them.  See the publications article for an example of how to
fetch a patch.  Information on patches and how to fetch and apply them
is also available through the World-Wide Web pages for \hbox{\AIPS}.
As bugs in \RELEASENAME\ are found, the patches will be placed in the
{\tt ftp}/Web area for \hbox{{\RELEASENAME}}.  As usual, we will not
revise the original {\tt 15JAN95} tape or \tar\ files for any such
patches.  No matter when you receive your {\tt 15JAN95} ``tape,'' you
must fetch and install these patches if you require them.

\eject

\section{Preview of coming attractions}

\subsection{Improvements in plotting}

The task {\tt LWPLA} has been using incorrect sizes for characters to
try to correct for poor placement of the character strings by the plot
tasks.  These incorrect sizes lead to misplacement of the strings in
other ways.  The solution is to go through all plot tasks and adopt a
better and more standard way to place the labeling around the main
plot areas.  While doing so, we are also adding the adverb {\tt LTYPE}
to all plotting tasks which do not already have it and adding new
values for additional control over the labeling of plots.  We expect
to add grey-scale images to the contours drawn by {\tt KNTR} as well.
This work is already nearly done in the {\tt 15JUL95} release.

\subsection{Improvements in imaging}

At present, the various \AIPS\ tasks that make images from visibility
data all make different assumptions about the uniform weighting
algorithm and other data weighting options.  It is proposed to give
the user full control over (1) the \uv-cell size used in uniform
weighting, (2) the radius in these cells over which a sample is
counted, (3) the weight used in that counting (the data weight, $1.0$,
and possibly others), and (4) a parameter that limits how widely the
data weights may vary.  The first two options will allow for weights
inversely proportional to the approximate local density of data
samples in a detailed way.  The third is mainly to allow users all of
the current choices.  And the fourth, called ``robust weighting'' by
Dan Briggs, has both a good theoretical justification and some lovely
test results.

The current weighting and gridding routines have trouble when the
weight or \uv\ grid does not fit into the ``AP'' memory.  We intend to
build in a sort to the imaging task or tasks to be done when necessary
to avoid multiple passes and other complications in the weighting and
gridding.

\subsection{A plan for VLBI-specific modifications to Classic \AIPS}

A list of functions needed for full support of VLBI in \AIPS\ follows.
Priority 1, ``(P1),''items are needed as soon as possible to fill gaps
in the data-processing paths.  Priority 2 items are less time
critical, but are needed for full functionality.  Priority 3 items are
useful, but not essential.  No time scales have been set, but we hope
to complete most of the priority 1 and 2 items during 1995.

\begin{enumerate}
\item\ Loading of data
   \begin{enumerate}
   \item\ (P1) IF selection and time averaging in \hbox{{\tt FITLD}}.
   \item\ (P1) Application of external calibration tables while loading
              VLBA data in \hbox{{\tt FITLD}}.
   \item\ (P2) Enable {\tt FITLD} to deal with sub-arrays.
   \item\ (P2) Fix known problems with {\tt MK3IN} (amplitude offset).
   \item\ (P3) {\tt MK4IN} to accommodate new Haystack format.
   \end{enumerate}
\item\ More automated {\it a priori} calibration of VLBA data, \ie\
      using the information provided by the on-line system with less
      editing and in fewer steps.  It is essential that the correlator
      software start passing the necessary calibration information
      ($T_{sys}$ tables, gain curve tables, phase-cal tables, flagging
      tables, weather tables etc.) through the distribution system to
      be read by \hbox{{\tt FITLD}}.
   \begin{enumerate}
   \item\ (P1) Apply digital corrections properly, including sampler
              bias amplitude corrections.
   \item\ (P2) Proper application of phase-cal data.
   \item\ (P2) Upgrade {\tt ANCAL} to use {\tt SN} table; also add more
              versatile options to allow one-pass calibration even if
              some antennas are missing calibration data for both
              polarizations (plus other similar cases).  Also, deal
              with phased-array VLA calibration more generally.
   \item\ (P2) Opacity corrections.
   \end{enumerate}
\vfill\eject
\item\ Data editing
   \begin{enumerate}
   \item\ (P2) Modify {\tt IBLED} to be more station oriented.
   \item\ (P2) ``Nearest neighbor'' option in \hbox{{\tt IBLED}}.
   \item\ (P3) Table editing task with {\tt IBLED} functionality.
   \end{enumerate}
\item\ Data examination
   \begin{enumerate}
   \item\ (P3) RR/LL options in at least one or two listing tasks.
   \item\ (P3) 3-D visualization options (separate packages \eg\ ATNF,
              but some path from \AIPS\ perhaps).
   \item\ (P3) {\tt POSSM} option for adjacent (RR,LL,RL,LR) plots.
   \item\ (P3) Plots of polarization closure quantities.
   \item\ (P3) Ability to plot spectra from different times on same
              plot with offset --- useful for looking for time
              variations.
   \end{enumerate}
\item\ General \uv\ calibration
   \begin{enumerate}
   \item\ (P1) Parameterized bandpass determination, including
              cross-power bandpass for \hbox{VLBA}.
   \item\ (P2) BP editing, weighting as requested by Uson et al.
   \item\ (P2) Different solution intervals for amplitude and phase
              self-cal solutions, other possible improvements.
   \item\ (P2) Better convergence in {\tt UVFIT}; add other options for
              improved amplitude calibration/b-factor calculations.
   \item\ (P2) Phase-referencing software.
   \item\ (P2) Add polarization self-cal software.
   \item\ (P3) Add IF selection to \hbox{{\tt FRING}}.
   \item\ (P3) Further algorithm development for polarization calibration.
   \item\ (P3) A {\tt SPLIT}-like task that generates a multi-source
              output file.
   \end{enumerate}
\item\ Imaging
   \begin{enumerate}
   \item\ (P1) Add robust weighting option.
   \item\ (P2) On-the-fly sorting in mapping tasks.
   \item\ (P3) NNLS (Dan Briggs' algorithm).
   \end{enumerate}
\item\ Image Analysis
   \begin{enumerate}
   \item\ (P2) Kinematic modeling (written but needs tidying up).
   \item\ (P3) 3-D {\tt SAD} equivalent for extracting component
              properties.
   \item\ (p3) Zeeman fitting for B-field determination.
   \end{enumerate}
\item\ Space VLBI software
   \begin{enumerate}
   \item\ (P1) Interactive model-fitter.
   \item\ (P2) Data simulator for generating pre-{\tt FRING} data plus
              Space VLBI type errors.
   \item\ (P3) Define tables for passing Space VLBI information through
              the correlator.
   \end{enumerate}
\item\ System
   \begin{enumerate}
   \item\ (P1) Proper {\tt DDT} for VLBA including simulator.
   \item\ (P1) Upgrade scratch file requirements for {\tt FRING}, {\tt
              BPASS} etc.; try to avoid creating very large scratch
              files.  Check disk space before trying to create the
              scratch files. If input data are compressed write
              compressed scratch file.
   \item\ (P2) change of data structure to enable IF-dependent weights
              for compressed data.
   \item\ (P3) AIPS GUI?
   \end{enumerate}
\end{enumerate}

\vfill
\eject
\hphantom{A}
\vfill
\center{This page left blank.}
\vfill
\end{document}
