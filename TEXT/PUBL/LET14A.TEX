%-----------------------------------------------------------------------
%;  Copyright (C) 2014
%;  Associated Universities, Inc. Washington DC, USA.
%;
%;  This program is free software; you can redistribute it and/or
%;  modify it under the terms of the GNU General Public License as
%;  published by the Free Software Foundation; either version 2 of
%;  the License, or (at your option) any later version.
%;
%;  This program is distributed in the hope that it will be useful,
%;  but WITHOUT ANY WARRANTY; without even the implied warranty of
%;  MERCHANTABILITY or FITNESS FOR A PARTICULAR PURPOSE.  See the
%;  GNU General Public License for more details.
%;
%;  You should have received a copy of the GNU General Public
%;  License along with this program; if not, write to the Free
%;  Software Foundation, Inc., 675 Massachusetts Ave, Cambridge,
%;  MA 02139, USA.
%;
%;  Correspondence concerning AIPS should be addressed as follows:
%;          Internet email: aipsmail@nrao.edu.
%;          Postal address: AIPS Project Office
%;                          National Radio Astronomy Observatory
%;                          520 Edgemont Road
%;                          Charlottesville, VA 22903-2475 USA
%-----------------------------------------------------------------------
%Body of intermediate AIPSletter for 31 December 2014 version

\documentclass[twoside]{article}
\usepackage{graphics}

\newcommand{\AIPRELEASE}{June 30, 2014}
\newcommand{\AIPVOLUME}{Volume XXXIV}
\newcommand{\AIPNUMBER}{Number 1}
\newcommand{\RELEASENAME}{{\tt 31DEC14}}
\newcommand{\NEWNAME}{{\tt 31DEC14}}
\newcommand{\OLDNAME}{{\tt 31DEC13}}

%macros and title page format for the \AIPS\ letter.
\input LET98.MAC
%\input psfig

\newcommand{\MYSpace}{-11pt}

\normalstyle

\section{Happy 35$^{\rm th}$ birthday \AIPS}

\subsection{\Aipsletter\ publication}

We have discontinued paper copies of the \Aipsletter\ other than for
libraries and NRAO staff.  The \Aipsletter\ will be available in
PostScript and pdf forms as always from the web site listed above.
New issues will be announced in the NRAO eNews mailing and on the
bananas list server.

\subsection{Current and future releases}

We have formal \AIPS\ releases on an annual basis.  While all
architectures can do a full installation from the source files,
Linux (32- and 64-bit), Solaris, and MacIntosh OS/X (PPC and Intel)
systems may install binary versions of recent releases.  The last,
frozen release is called \OLDNAME\ while \RELEASENAME\ remains under
active development.  You may fetch and install a copy of these
versions at any time using {\it anonymous} {\tt ftp} for source-only
copies and {\tt rsync} for binary copies.  This \Aipsletter\ is
intended to advise you of improvements to date in \RELEASENAME\@.
Having fetched \RELEASENAME, you may update your installation whenever
you want by running the so-called ``Midnight Job'' (MNJ) which copies
and compiles the code selectively based on the changes and
compilations we have done.  The MNJ will also update sites that have
done a binary installation.  There is a guide to the install script
and an \AIPS\ Manager FAQ page on the \AIPS\ web site.

The MNJ serves up \AIPS\ incrementally using the Unix tool {\tt cvs}
running with anonymous ftp.  The binary MNJ also uses the tool {\tt
rsync} as does the binary installation.  Linux sites will almost
certainly have {\tt cvs} installed; other sites may have installed it
along with other GNU tools.  Secondary MNJs will still be possible
using {\tt ssh} or {\tt rcp} or NFS as with previous releases.  We
have found that {\tt cvs} works very well, although it has one quirk.
If a site modifies a file locally, but in an \AIPS-standard directory,
{\tt cvs} will detect the modification and attempt to reconcile the
local version with the NRAO-supplied version.  This usually produces a
file that will not compile or run as intended.  Use a copy of the task
and its help file in a private disk area instead.

\AIPS\ is now copyright \copyright\ 1995 through 2014 by Associated
Universities, Inc., NRAO's parent corporation, but may be made freely
available under the terms of the Free Software Foundation's General
Public License (GPL)\@.  This means that User Agreements are no longer
required, that \AIPS\ may be obtained via anonymous ftp without
contacting NRAO, and that the software may be redistributed (and/or
modified), under certain conditions.  The full text of the GPL can be
found in the \texttt{15JUL95} \Aipsletter, in each copy of \AIPS\
releases, and on the web at {\tt http://www.aips.nrao.edu/COPYING}.


\section{Improvements of interest in \RELEASENAME}

We expect to continue publishing the \Aipsletter\ approximately every
six months, but the publication is now primarily electronic.  There
have been several significant changes in \RELEASENAME\ in the last six
months.  Some of these were in the nature of bug fixes which were
applied to \OLDNAME\ before and after it was frozen.  If you are
running \OLDNAME, be sure that it is up to date; pay attention to the
patches and run a MNJ any time a patch relevant to you appears.  New
tasks in \RELEASENAME\ include {\tt BPWAY} to determine spectral
channel-dependent $uv$ data weights, {\tt DBAPP} to append multiple
$uv$ data sets at once, {\tt CENTR} to change the frequency reference
pixel to the center of the spectrum, {\tt HUINT} to adjust
hue-intensity images interactively and then save the result as image
files, and {\tt MODSP} to compute model images of spectral lines.  A
set of {\tt RUN} files, under the general name {\tt TDEPEND}, was
created to assist with imaging a time-dependent source.  The cube
model-fitting tasks {\tt XGAUS}, {\tt RMFIT}, and {\tt ZEMAN} were
given significant attention and a new \AIPS\ Memo was written to
document their usage.

Coming attractions include a completely overhauled VLB data-reduction
pipeline.  Amy Mioduszewski has re-written the {\tt RUN} file {\tt
  VLBAPIPE}, re-naming it {\tt VLBARUN}, and is now giving it final
testing.

{\tt 31DEC09} contains a significant change in the format of the
antenna files, which will cause older releases to do wrong things to
data touched by {\tt 31DEC09} and later releases.  {\tt 31DEC08}
contains major changes to the display software.  You are encouraged to
use a relatively recent version of \AIPS, whilst those with EVLA data
to reduce should get release \OLDNAME\ or, preferably, the latest
release.

\subsection{UV-data}

\begin{description}
\myitem{BPWAY} is a new task to determine relative weights for
          spectral channels based on an {\tt RFLAG}-like rms
          computation (over time).  It can do weight spectra averaged
          over one scan or over one source at a time.
\myitem{DBAPP} is a new task to append one or more similar $uv$ data
          sets. It is not as capable as {\tt DBCON} but can do
          multiple data sets at once.
\myitem{CENTR} is a new task to change the frequency reference pixel
          to the actual center channel of each spectrum.
\myitem{FQCENTER} is a new adverb in numerous tasks to change the
          frequency reference channel to the center of each band
          ($N_{\rm chan} + 1$).  This requires re-scaling all
          $(u,v,w)$ values and correcting values in both the {\tt FQ}
          and {\tt SU} tables.
\myitem{RLDLY} was changed to average over all possible reference
          antennas when {\tt REFANT = 0}.
\myitem{PCAL} was corrected for an error affecting data sets with axes
          in non-standard order.  It was changed to handle no data
          found situations with more grace and to make use of previous
          solutions an option rather than a requirement.
\myitem{DOPOL} applying the orientation-ellipticity mode of {\tt PCAL}
          was made to function properly.  Unfortunately, the solutions
          from {\tt PCAL} are not as good as they should be.
\myitem{BPASS} was corrected to use no spectral index when multiple
          sources have been averaged together ({\tt SOLINT=-1} with
          multiple sources) but to apply a correction when only one
          source was averaged.
\myitem{DOBAND} modes 2 and 4 apply the nearest-neighbor bandpass
          solution to the data.  The correction arrays were not
          initialized correctly at the beginning and after any long
          breaks in the data.
\myitem{REWAY} was corrected to handle very short scans without dying
          and to write out data records only when they contain some
          valid data.
\myitem{CLIP} was given new options to flag parallel-hand
          polarizations when cross-hand polarizations are clipped and
          to flag all polarizations when any one polarization is
          clipped.
\myitem{SPLAT} was changed to ``tidy up'' the calibration table as
          well as the source table.  Otherwise, the {\tt CL} table
          could contain sources no longer in the {\tt SU} table.  The
          output frequency, when averaging spectral windows with {\tt
            BIF} $> 1$, was corrected.  The task was also changed to
          defend itself from bad input values when doing spectral
          averaging.
\myitem{UVFND} was given the {\tt DOSCALE} option to normalize the
          amplitudes by the fluxes in the source table, including
          optionally the spectral index.  Channel averaging was
          corrected.
\myitem{CLCAL} was corrected for a bug which caused re-referencing of
          phases to fail and for a bug in time smoothing affecting
          very short scans.
\myitem{SETJY} was corrected for bugs in the velocity computations in
          {\tt OPTYPE = ' '} and {\tt 'SPEC'} modes.
\myitem{SNP2D} was tested to calibrate wide-band phase using a single
          spectral line.  Converting input {\tt SN} table phases into
          both a delay and an IF-dependent phase appears to provide an
          excellent adjustment after the standard calibrations have
          been applied.
\myitem{SPECR} was given the option to use simple interpolation
          methods rather than the FFT method, but the FFT is still
          recommended in most cases.
\myitem{RLCAL} was corrected to do time averaging properly, to stop
          dividing Q and U models by the I polarization flux, and to
          pay attention to all adverbs before dividing by a model.
\myitem{TI2HA} and {\tt HA2TI} were given the {\tt REFANT} option to
          control the antenna to which the hour angle refers.
          Previously it was the lower numbered antenna in each pair
          which made for confusion in sort order.
\myitem{FITLD} was changed to save the data file when an error occurs
          in reading ``trailing'' records in the FITS file.  Problems
          finding the desired records in the {\tt MC} and {\tt IM}
          table were also fixed.
\myitem{APCAL} was given the option of looping over all subarrays.
          It was changed to handle flagged table records without
          getting upset, to understand that subarray 0 means all, and
          to handle duplicate records in {\tt GC} tables properly.
\myitem{PCCOR} was changed to handle magic blank values in the input
          tables properly.
\myitem{SNSMO} was corrected in numerous ways to handle rates properly
          in Hz at the correct IF frequency, rather than handling them
          rather casually in the basic units (sec/sec).
\end{description}

\subsection{Display}

A new task {\tt HUINT} has been written.  Its goal is to do the same
display as the {\tt TVHUEINT} verb, but then to save the image as
displayed in rather arbitrary units and, optionally, to save an image
of the hue-intensity step wedge as well.   The results may then be
plotted in tasks such as {\tt KNTR} with the full flexibility allowed
by {\tt LWPLA} and its PostScript output.  The task {\tt TVHUI}
attempts to write out the image with real image units, suitably
scaled, and is caught up in the now obsolete need to replicate true
color in a non-true color environment.  The results from the new task
appear more pleasing than those from this older one.  Until now, if
one wanted to capture a {\tt TVHUEINT} display, one had to use a
screen capture tool (such as {\tt import}) and could not modify
anything about the plot afterwards.

\begin{description}
\myitem{BLSUM} now offers the option of weighting the spectral sums
        with the values of the blotch image.  This should be good for
        recombination lines with total intensity as the blotch.  The
        task also now offers the option of making real plot files
        instead of printer plots.
\myitem{SNPLT} now offers the option of plotting amplitude gains as
        $1/g(t)^2$ called {\tt POWR} and $-20 \log (g)$ as {\tt PODB}
        to display the antenna-based power.  Several aspects of rate
        plotting were changed to use correct frequencies in scaling
        from sec/sec to mHz.
\myitem{POSSM} was given improved labeling for parallel- versus
        cross-hand and auto- versus cross-correlations.
\myitem{VPLOT} now allows all $x$-axis types when plotting amplitude
        and phase in the same plot.
\myitem{PRTAB} was changed to use enough digits when an exponential
        format had to be used in a floating-format display.  It also
        now handles extra long character strings appropriately.
\myitem{LISTR} was corrected to use {\tt FACTOR} properly in phase,
        elevation, and parallactic angle displays.  Additional
        information on {\tt FACTOR} was added to the help file --- it
        can be confusing.  Some display overflows were previously not
        shown as such.
\myitem{CURVALUE}\hspace{2em} was given adverbs to control which
        graphics plane is used for its display and which memory plane
        is used for the image data to be displayed.
\myitem{Slice} labeling was corrected for a variety of issues
        including special code for when the slice is along an axis and
        code to encourage the use of the full display area.
\myitem{Coordinate} labeling was adjusted to be more flexible in
        dealing with the new, all-sky types of coordinates.
\end{description}

\subsection{Imaging}

{\tt IMAGR} was given several new options.  If {\tt OBOXFILE} is set
simply to an environment variable, \eg\ {\tt MYAREA:}, then an output
box file named in a unique way will be written to that directory.
Adverb {\tt LTYPE} specifies the type of labeling to be added to any
image display on the TV\@.  These displays can also plot ``stars''
which are specified by up to 100 {\tt S} cards in the {\tt BOXFILE}\@.
More important, perhaps, was the addition of the concept of
``UNClean'' boxes, namely rectangular or circular areas within which
no Clean components are allowed (at least in the facet for which the
UNClean box has been specified).  This concept has been used
profitably in the {\tt OBIT} package to add special facets with bright
sources carefully centered upon a pixel.  Users may use the concept
this way, but its introduction to \AIPS\ was required for the
time-dependent imaging procedures described next.  UNClean boxes are
supported interactively in the image display if at least one was
specified in the input {\tt BOXFILE}\@.

A general help file and a {\tt RUN} file, both called {\tt TDEPEND},
have been written to suggest a sequence to image a source which varies
during the course of a synthesis observation.  In general, the
assumption is that one has a variable source (\ie\ a star) in the
midst of a larger field containing other sources which do not vary.
Each time interval is imaged with a number of facets including a
special facet just for the star  and an UNClean box around the star in
any other facet in which it occurs.  {\tt UVSUB} is then used to
remove the star from the $uv$ data of the time interval and all
intervals of star-free data are concatenated.  The combined data are
then used to image the full field including self-cal and editing as
needed.  The process may then be iterated to make better images of the
star in each interval, a $uv$ file more free of the star, and then
further imaging of the full field.  Eventually, the final calibration
and flag tables may be applied to the star-only data in intervals and
time-dependent images made.  {\tt TDEPEND} has not been used ``in
anger'' as yet, so your editor is very interested in any experiences
you may have with this.

\subsection{Image analysis}

\subsubsection{{\tt XGAUS}, {\tt RMFIT}, {\tt ZEMAN}, modeling}

The analysis of image cubes in polarization, spectral line, and Zeeman
splitting received considerable attention in the last six months.
There is a new \AIPS\ Memo number 118 to describe the use of the
presently available tasks.  The abstract for this Memo appears later
in this \Aipsletter\@.  A new image modeling task {\tt MODSP} was
written.  It can add models to existing image cubes (I and V or RR and
LL) or create new ones.  Each model object is a spectral line with
specified frequency and spatial structure and separate RR and LL
fluxes.

The three cube fitting tasks received several improvements in common.
All three will now display, in the ``edit'' phase, small images
expanded by pixel replication.  This means that {\tt TVZOOM}, which
affects the menus as well as the image, is far less necessary.  All
three tasks now offer the option of using TV menus in the fitting
stages, rather than communicating extensively in the terminal window.
All three also use the adverb {\tt RMSLIMIT} to specify when a fit has
failed, changing the task from automatic fitting back into interactive
modes.

{\tt XGAUS} and {\tt RMFIT} now indicate a component which is not to
be fit by an X off the left side of the spectrum plot.  They both are
prepared to fit larger menus and titles when there are a large number
of components and/or the character size is increased (via {\tt
  CHARMULT})\@.  They both offer a {\tt DO FIT} menu option after the
user has entered a guess by the {\tt HAND} option, treating the user's
guess not as the answer but just an initial guess to be refined if
possible.  {\tt XGAUS} even allows the user to specify that certain
parameters are to be held fixed during the fit that immediately follows
the {\tt HAND} entry.

{\tt XGAUS} was corrected to require that the {\it average} of three
consecutive spectral points be above the cutoff for fitting.  This is
what is expected by {\tt ZEMAN} and it is important that both tasks do
the same thing.  {\tt XGAUS} was given another scheme to make its
initial guesses which turns out to be successful in many cases.  It
also had bugs preventing the writing out of the residual image.  The
plot of the initial guess in {\tt ZEMAN} was changed to be at the data
points, rather than on a fine and regular grid.  The latter made the
plot look like a step function for no good reason.

{\tt RMFIT} was changed to fit for a rotation measure ``thickness''
using any one of three models for the thickness (slab, Gaussian,
exponential).  Such thickness accounts for much of the loss of
polarization brightness at low frequencies.  {\tt MODIM} was changed
to be able to model such thicknesses.

\subsubsection{Other analysis changes}
\begin{description}
\myitem{FARS} did not do the convolution by the restoring function
    properly, making the answers from Cleaning incorrect.  This was
    fixed while simplifying the code for legibility and performance.
\myitem{IMFIT,} {\tt JMFIT}, and {\tt MAXFIT} again had trouble with
    negatives.  The tasks ran into problems fitting baselines in
    regions less than zero and failed to report all the values they
    found.  {\tt MAXFIT} was given the adverb {\tt DOINVERS} to allow
    it to fit a maximum or a minimum and was corrected to do these in
    regions of negative brightness.
\myitem{DSKEW} had a bug affecting the correction of images with any
    apparent rotation, especially those with a large rotation angle.
\myitem{\AIPS} Memo 117 was changed to describe the meaning of the
    $(x,y)$ coordinates in the Clean Components file correctly.
\end{description}

\subsection{General}

\begin{description}
\myitem{TGET file} \hspace{2em} has changed format to include the
    adverb names and their types and lengths.  This allows {\tt TGET}
    and {\tt VGET} to remark on changes in the {\tt INPUTS} since the
    corresponding {\tt TPUT} (or {\tt GO}) and {\tt VPUT} and to avoid
    messing up adverb values following the changes.  The old format
    {\tt TGET} files will still be used if the task or verb is not
    found in the new format file.  {\tt TGINDEX} will list the
    contents of both format files, listing only the most recent
    occurrence of each task.
\myitem{DISKU} was changed to have the {\tt DOALL} adverb control the
    smallest size catalog entry to be listed in detail and the format
    of that listing was made much more useful.  This enables users to
    find their largest data files when it becomes time to clear some
    disk space.
\myitem{\Cookbook} was brought up to date in April and later converted
    to use Palatino fonts rather than Computer Modern fonts.  This
    \Aipsletter\ also uses these more elegant fonts.  Appendix L on
    reducing EVLA low frequency data was contributed by Minnie Mao.
\myitem{On-line} help information in interactive tasks involving menus
    is provided by special help files, \eg\ {\tt HLPXGAUS}\@.  The
    parsing of these files was changed to allow certain string
    substitutions for increased clarity, namely ``{\tt <n>}'' has the
    current value of n substituted.
\myitem{Maximum} image size was increased to 131072 to match the
    maximum number of spectral channels.  FFTs of this size are then
    supported, but will be expensive.  Square arrays of this size are
    too large for Fortran compilers by a factor of more than 32.
\end{description}

\section{Patch Distribution for \OLDNAME}

Important bug fixes and selected improvements in \OLDNAME\ can be
downloaded via the Web beginning at:

\begin{center}
\vskip -10pt
{\tt http://www.aoc.nrao.edu/aips/patch.html}
\vskip -10pt
\end{center}

Alternatively one can use {\it anonymous} \ftp\ to the NRAO server
{\tt ftp.aoc.nrao.edu}.  Documentation about patches to a release is
placed on this site at {\tt pub/software/aips/}{\it release-name} and
the code is placed in suitable sub-directories below this.  As bugs in
\NEWNAME\ are found, they are simply corrected since \NEWNAME\ remains
under development.  Corrections and additions are made with a midnight
job rather than with manual patches.  Since we now have many binary
installations, the patch system has changed.  We now actually patch
the master version of \OLDNAME, which means that a MNJ run on
\OLDNAME\ after the patch will fetch the corrected code and/or
binaries rather than failing.  Also, installations of \OLDNAME\ after
the patch date will contain the corrected code.

The \OLDNAME\ release has had a number of important patches:
\begin{enumerate}
   \item\ {\tt BPASS} failed to apply a spectral index correction when
          {\tt SOLINT=-1} even when there was only one calibration
          source. {\it 2014-01-14}
   \item\ {\tt LISTR} failed to read source information while printing
          "gains". {\it 2014-01-14}
   \item\ {\tt DOOSRO} run file had a typo in its first line.  {\it
           2014-01-27}
   \item\ {\tt UVFND} in averaging channels used the real part as both
          the real and imaginary parts. {\it 2014-02-07}
   \item\ {\tt DOBAND} modes 2 and 4 had an initialization problem.
          {\it 2014-02-09}
   \item\ {\tt DOOSRO} had a POPS language error {\it 2014-02-10}
   \item\ {\tt KNTR} and {\tt PCNTR} had minor issues which blocked
          display of true-color images. {\it 2014-02-11}
   \item\ {\tt PCCOR} did not handle blanked values from the {\tt PC}
          table cable-cal measurements {\it 2014-02-11}
   \item\ {\tt FITLD} had trouble finding the correct records for
          {\tt MC} and {\tt IM} tables. {\it 2014-03-11}
   \item\ {\tt DSKEW} did bad things when the input image had a
          non-zero value of rotation. {\it 2014-03-17}
   \item\ {\tt CLCAL} failed to re-reference the SN tables when
          requested. {\it 2014-03-19}
   \item\ {\tt SNP2D} needed clarification of {\tt BCHAN} and {\tt
          BIF} and to write reference channel phases as well as
          delays.   {\it 2014-03-27}
   \item\ {\tt TIORD} had a bad format. {\it 2014-02-31}
   \item\ {\tt XGAUS} and {\tt RMFIT} called the function routine
          incorrectly numerous times, mostly with remarkably benign
          results. {\it 2014-04-02}
   \item\ {\tt PRTAB}  could abort when string data were very long.
          {\it 2014-04-02}
   \item\ {\tt LISTR} did not show the correct scaling for angles in
          the {\tt GAIN} listing. {\it 2014-04-11}
   \item\ {\tt TVFLG} and {\tt SPFLG} interpreted flagged rows in the
          flag command table as a serious error in a couple of
          places. {\it 2014-05-29}
   \item\ {\tt DTSUM} omitted the highest numbered antenna from its
          matrix list if it did not have autocorrelation records. {\it
            2014-06-13}
   \item\ {\tt CALIB} in phase-only solutions could return amplitude
          gains other than 1.0 when only 2 baselines occurred in an
          interval.  {\it 2014-06-27}
\end{enumerate}
\vfill\eject

\section{Recent \AIPS\ Memoranda}

All \AIPS\ Memoranda are available from the \AIPS\ home page.  \AIPS\
Memo 117 describing \AIPS' usage of the FITS format was modified
slightly in the first part of 2014.

\begin{tabular}{lp{5.8in}}
{\bf 118} & {\bf Modeling Spectral Cubes in \AIPS}\\
   &  Eric W. Greisen, NRAO\\
   &  June 19, 2014\\
   &  \AIPS\ has done Gaussian fitting along the $x$-axis of image
   cubes with task {\tt XGAUS} since the 1980s.  That task has
   recently been overhauled to be much easier to use and much more
   capable.  In like fashion, new tasks {\tt ZEMAN} and {\tt RMFIT}
   have been developed. The former fits the standard leakage and
   scaling terms for Stokes V cubes, including a new option to do this
   for each of the Gaussians found by {\tt XGAUS}\@.  The latter fits
   polarization models to Stokes Q and U cubes, using the output of
   Faraday Rotation Measure Synthesis (\AIPS\ task {\tt FARS}) to
   assist with initial guesses.  The models can contain multiple
   components each with a polarization flux, angle, rotation measure,
   and rotation measure ``thickness.'' The present memo will describe
   the functions of these tasks in some detail with numerous
   graphical examples.  This memo also discusses a number of tasks
   which make visibility and image model files.
\end{tabular}


\section{\AIPS\ Distribution}

We are now able to log apparent MNJ accesses and downloads of the tar
balls.  We count these by unique IP address.  Since some systems
assign the same computer different IP addresses at different times,
this will be a bit of an over-estimate of actual sites/computers.
However, a single IP address is often used to provide \AIPS\ to a
number of computers, so these numbers are probably an under-estimate
of the number of computers running current versions of \AIPS\@. In
2014, there have been a total of 646 IP addresses so far that have
accessed the NRAO cvs master.  Each of these has at least installed
\AIPS\ and 257 appear to have run the MNJ on \RELEASENAME\ at
least occasionally.  During 2014 more than 248 IP addresses have
downloaded the frozen form of \OLDNAME, while more than 566 IP
addresses have downloaded \RELEASENAME\@.  The binary version was
accessed for installation or MNJs by 321 sites in \OLDNAME\ and 464
sites in \RELEASENAME\@.  A total of 1105 different IP addresses have
appeared in one of our transaction log files.  Some of these numbers
are a bit higher than those of 2013 at a comparable date, while the
{\tt cvs} number is curiously low.

\centerline{\resizebox{!}{3.6in}{\includegraphics{FIG/PLOTIT14a.PS}}}

\vfill\eject
% mailer page
% \cleardoublepage
\pagestyle{empty}
 \vbox to 4.4in{
  \vspace{12pt}
%  \vfill
\centerline{\resizebox{!}{3.2in}{\includegraphics{FIG/Mandrill.eps}}}
%  \centerline{\rotatebox{-90}{\resizebox{!}{3.5in}{%
%  \includegraphics{FIG/Mandrill.color.plt}}}}
  \vspace{12pt}
  \centerline{{\huge \tt \AIPRELEASE}}
  \vspace{12pt}
  \vfill}
\phantom{...}
\centerline{\resizebox{!}{!}{\includegraphics{FIG/AIPSLETS.PS}}}

\end{document}
