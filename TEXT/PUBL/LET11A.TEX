%-----------------------------------------------------------------------
%;  Copyright (C) 2011
%;  Associated Universities, Inc. Washington DC, USA.
%;
%;  This program is free software; you can redistribute it and/or
%;  modify it under the terms of the GNU General Public License as
%;  published by the Free Software Foundation; either version 2 of
%;  the License, or (at your option) any later version.
%;
%;  This program is distributed in the hope that it will be useful,
%;  but WITHOUT ANY WARRANTY; without even the implied warranty of
%;  MERCHANTABILITY or FITNESS FOR A PARTICULAR PURPOSE.  See the
%;  GNU General Public License for more details.
%;
%;  You should have received a copy of the GNU General Public
%;  License along with this program; if not, write to the Free
%;  Software Foundation, Inc., 675 Massachusetts Ave, Cambridge,
%;  MA 02139, USA.
%;
%;  Correspondence concerning AIPS should be addressed as follows:
%;          Internet email: aipsmail@nrao.edu.
%;          Postal address: AIPS Project Office
%;                          National Radio Astronomy Observatory
%;                          520 Edgemont Road
%;                          Charlottesville, VA 22903-2475 USA
%-----------------------------------------------------------------------
%Body of intermediate AIPSletter for 31 December 2011 version

\documentclass[twoside]{article}
\usepackage{graphics}

\newcommand{\AIPRELEASE}{June 30, 2011}
\newcommand{\AIPVOLUME}{Volume XXXI}
\newcommand{\AIPNUMBER}{Number 1}
\newcommand{\RELEASENAME}{{\tt 31DEC11}}
\newcommand{\NEWNAME}{{\tt 31DEC11}}
\newcommand{\OLDNAME}{{\tt 31DEC10}}

%macros and title page format for the \AIPS\ letter.
\input LET98.MAC
%\input psfig

\newcommand{\MYSpace}{-11pt}

\normalstyle

\section{General developments in \AIPS}

\subsection{\Aipsletter\ publication}

We have decided to discontinue paper copies of the \Aipsletter\ other
than for libraries and NRAO staff.  The \Aipsletter\ will be available
in PostScript and pdf forms as always from the web site listed above.
It will be announced in the NRAO e-News mailing and on the bananas
list server.

\subsection{Current and future releases}

We have formal \AIPS\ releases on an annual basis.  While all
architectures can do a full installation from the source files,
Linux (32- and 64-bit), Solaris, and MacIntosh OS/X (PPC and Intel)
systems may install binary versions of recent releases.  The last,
frozen release is called \OLDNAME\ while \RELEASENAME\ remains under
active development.  You may fetch and install a copy of these
versions at any time using {\it anonymous} {\tt ftp} for source-only
copies and {\tt rsync} for binary copies.  This \Aipsletter\ is
intended to advise you of improvements to date in \RELEASENAME\@.
Having fetched \RELEASENAME, you may update your installation whenever
you want by running the so-called ``Midnight Job'' (MNJ) which copies
and compiles the code selectively based on the changes and
compilations we have done.  The MNJ will also update sites that have
done a binary installation.  There is a guide to the install script
and an \AIPS\ Manager FAQ page on the \AIPS\ web site.

The MNJ serves up \AIPS\ incrementally using the Unix tool {\tt cvs}
running with anonymous ftp.  The binary MNJ also uses the tool {\tt
rsync} as does the binary installation.  Linux sites will almost
certainly have {\tt cvs} installed; other sites may have installed it
along with other GNU tools.  Secondary MNJs will still be possible
using {\tt ssh} or {\tt rcp} or NFS as with previous releases.  We
have found that {\tt cvs} works very well, although it has one quirk.
If a site modifies a file locally, but in an \AIPS-standard directory,
{\tt cvs} will detect the modification and attempt to reconcile the
local version with the NRAO-supplied version.  This usually produces a
file that will not compile or run as intended.

\AIPS\ is now copyright \copyright\ 1995 through 2011 by Associated
Universities, Inc., NRAO's parent corporation, but may be made freely
available under the terms of the Free Software Foundation's General
Public License (GPL)\@.  This means that User Agreements are no longer
required, that \AIPS\ may be obtained via anonymous ftp without
contacting NRAO, and that the software may be redistributed (and/or
modified), under certain conditions.  The full text of the GPL can be
found in the \texttt{15JUL95} \Aipsletter, in each copy of \AIPS\
releases, and on the web at {\tt http://www.aips.nrao.edu/COPYING}.

\vfill\eject

\section{Improvements of interest in \RELEASENAME}

We expect to continue publishing the \Aipsletter\ approximately every
six months along with the annual releases.  Henceforth, this
publication will be primarily electronic.  There have been several
significant changes in \RELEASENAME\ in the last six months.  Some of
these were in the nature of bug fixes which were applied to \OLDNAME\
before and after it was frozen.  If you are running \OLDNAME, be sure
that it is up to date to sometime this year (a MNJ after June 8 would
be best). New tasks in \RELEASENAME\ include {\tt REWGT} to scale data
weights by subarray and/or IF, {\tt ACLIP} to do clip operations on
auto-correlation data, {\tt RLCAL} to solve for right minus left phase
differences as a function of time when good source models are
available, {\tt SY2TY} to make {\tt TY} tables from SysPower tables,
{\tt SYCOP} to copy good {\tt SY} table IFs to bad ones, {\tt QUOUT}
and {\tt QUFIX} to determine right-left calibrations from
well-calibrated images, {\tt RFLAG} to flag data based on rmses
measured over short time intervals and on deviations from the average
spectrally, {\tt IM2CC} to convert model images into Clean components
attached to suitable image facets, and {\tt AFARS} to extract useful
parametric images from the Faraday Rotation synthesis ({\tt FARS})
output image cubes.  The new verb {\tt HUEWEDGE} draws a step wedge in
two dimensions on the TV and even labels it to match a {\tt TVHUEINT}
display.  The OBIT programs {\tt BDFIn} and {\tt BDFList} have been
made accessible to \AIPS\ users through ``verbs'' {\tt BDF2AIPS} and
{\tt BDFLIST}.  These replace earlier stand-alone, question-and-answer
scripts.  This worked so well that further ``verbs'' {\tt OBITMAP},
{\tt OBITIMAG}, {\tt OBITSCAL}, and {\tt OBITPEEL} have been written
to make available increasingly complex portions of the OBIT program
{\tt Imager}.  A new pipeline procedure named {\tt DOOSRO} has been
written to reduce EVLA data taken with less difficult observing
parameters (``Open shared risk'').  A service program to list assigned
user numbers called {\tt USERNO} was re-created.

{\tt 31DEC09} contains a significant change in the format of the
antenna files, which will cause older releases to do wrong things to
data touched by {\tt 31DEC09} and later releases.  {\tt 31DEC08}
contains major changes to the display software.  Older versions may
use the {\tt 31DEC08} display ({\tt XAS}), but {\tt 31DEC08} code may
not use older versions of {\tt XAS}\@.   Magnetic tape logical unit
numbers changed with {\tt 31DEC04}\@.  You are encouraged to use a
relatively recent version of \AIPS, whilst those with EVLA data to
reduce must get the latest release.

\subsection{Lion}

Apple has announced a new major revision of OS X called Lion (10.7).
The VAO in Socorro has made a new, beautifully equipped iMac available
to us for testing.  It was found to install and run the existing {\tt
  MACINT} load modules with no difficulty.  We then tried using the
latest Intel compiler (12.0.4 dated 20110503) with the x86\_64
architecture.  This version ran a little bit faster than the 32-bit
version, but not enough faster to justify making a new {\tt MAC64}
``architecture'' with a new NRAO computer to support it.

\subsection{EVLA UV-data calibration and handling}

A great many changes have been stimulated by the commissioning effort
now underway with the new EVLA WIDAR correlator.  A new appendix (E)
to the \Cookbook\ has been written to describe EVLA data reduction.
It has been available on the web and is now incorporated into the
\AIPS\ release.  Note that this document is a work in progress and it
will of necessity evolve with time.

\subsubsection{RFLAG}

RFI is frequently characterized by rapid variability in both time and
frequency.  {\tt RFLAG} looks for this and flags the data accordingly.
Using a floating buffer in time (usually just 3 or 5 integrations), it
measures the rms over time in each channel individually.  If the rms
exceeds a cutoff, the channel is flagged over that time range.  {\tt
  RFLAG} also finds the mean and rms of the real and imaginary parts
of the visibility over the channels in each spectral window at each
time.  The task uses ``robust'' methods which exclude outliers while
finding the ``real'' mean and rms.  Any channel that deviates by more
than $N$ times the rms will also be flagged.  To assist in setting the
time cutoff and the value of $N$, {\tt RFLAG} will plot the histograms
of values of the time rms and the spectral deviations instead of
writing a flag table.  {\tt RFLAG} flags spectral channels
individually, while {\tt FLAGR} deletes whole spectral windows
depending on trends in the average and rms of the spectral channels.

\subsubsection{Polarization and right minus left matters}

{\tt RLDLY} finds the difference in delay between right and left
polarizations.  This can be large even if little or no delay
correction is needed for the parallel-hand data.  Bugs were fixed that
caused the answers to be compromised especially when dividing the IFs
into halves to get one AC and one BD delay.  It was changed to copy
and correct the {\tt CL} table that it applied to the data rather than
the highest {\tt CL} version.  It was changed to get correct phase
offsets for each IF, so that phases will remain continuous across the
full band.

After correcting the right minus left delay, one can solve for antenna
polarization using {\tt PCAL}\@.  The model computation in {\tt PCAL}
was found to be in error for Clean component models of Q and U and was
corrected.  Unfortunately, since {\tt PCAL} does not solve for a right
minus left phase difference but the data contain this difference, the
use of a fixed model for the calibrator in {\tt PCAL} cannot work.

After {\tt PCAL}, the D-terms are applied to the data from a source of
known polarization position angle and the remaining right minus left
phases are found and corrected.  {\tt RLDIF} is the main task for this
and it had to be fixed to change the spectral-line D-term file ({\tt
  PD}) and the continuum mode tables ({\tt SU}, {\tt AN}, {\tt CL})
properly.  {\tt RLDIF} can now read a text file with the Q and U
spectrum in order to do its calibration.  {\tt RLDIF} had incorrect
position angles for known sources 3C48, 3C138, and 3C147.  Handling of
blanked solutions and smoothing was also improved.

{\tt RLDIF} can be instructed to correct the relevant \AIPS\ tables in
either continuum or spectral modes.  It can also, in continuum mode
only, leave the corrections to be run later in {\tt CLCOR}\@.  That
task corrects the antenna table (D-term solutions) and {\tt CL} table
and now also corrects the source table Q and U values.

{\tt RLDIF} assumes that the right minus left phase difference is
constant with time.  At high precision, this is often not the case.  A
new task {\tt RLCAL} was written to compare visibility data with
models of Q and U to determine and correct this phase difference as a
function of time.  It should be possible to feed this back into the
original data set to improve the polarization and {\tt RLDIF}
solutions and therefore the new set of images.

Dynamic scheduling of observations means that the user cannot count on
having observations of the best polarization calibration sources in
every data set.  Two new tasks were written to assist the user
desperately trying to calibrate the data anyway.  {\tt QUOUT} prepares
a text file of Q and U for a calibration source from another,
well-calibrated data set or from a Q/U image cube.  {\tt RLDIF} can
read this file to allow this previously unknown calibration source to
be used.  The new task {\tt QUFIX} assumes that the user has a
well-calibrated Q/U image cube from one day and a second Q/U image
cube from a day lacking the polarization angle calibrator.  Comparing
the position angles over areas in the two cubes at each spectral
channel, the task determines and applies the right-minus left
correction.

\subsubsection{OBIT in \AIPS}

The software package known as OBIT, written by Bill Cotton, offers
capabilities not available elsewhere.  Two stand-alone procedures have
been available to translate data from the Science Data Model/Binary
Data Format file tree into \AIPS\ for those users who have OBIT
available.  These question-and-answer scripts, with requirements on
the users' setup, have been replaced with {\tt AIPS} verbs.  The
adverb values are prepared in the usual way inside {\tt AIPS} and the
functions invoked by entering the verb names {\tt BDFLIST} and {\tt
  BDF2AIPS}\@.  They invoke OBIT programs and leave log files in the
user's home area.  The {\tt DOWAIT} adverb is very useful here,
causing {\tt AIPS} to display the output from the OBIT programs as it
occurs and returning control of the {\tt AIPS} window to the user only
when the process has finished.

\subsubsection{EVLA SysPower table}

The units of the EVLA SysPower table were examined and changed on
February 24.  They were a Psum and Pdif {\it divided} by a ``gain''
normally recorded as having value 512.  In fact, this was a fiction
with the real gain involving other factors as well.  Beginning
February 24, the Psum and Pdif are the true values {\it times} a gain
(as advertised) and the true gain is recorded.  {\tt TYAPL} was
changed to use the new scaling for observations beginning February 24
and to use the former scaling for older data.  {\tt TYAPL} was also
changed to offer the option of scaling the visibilities but leaving
the weights untouched.  In computing weights, it now uses the {\tt
  INTTIM} random parameter when available as the integration time and
ignores the incoming data weights.  Those incoming data weights can
include channel averaging, which {\tt TYAPL} already includes from the
channel width in the header, causing the resulting weights to be in
error if the incoming weights are used.  {\tt SPLAT} was corrected to
insure that the {\tt INTTIM} values would be correct.

{\tt TYSMO}, which prepares {\tt SY} tables for {\tt TYAPL}, now uses
Tsys rather than Tsys/Tcal for clipping and applies a flag table
optionally to the {\tt SY} and {\tt TY} data as they are read.
Blanking of all parameters in the {\tt SY} table when any one is bad
is now handled properly.

Two new tasks were written to allow manipulation of {\tt SY} tables.
{\tt SYCOP} averages {\tt SY} table values from ``good'' IFs and
replaces the values in ``bad'' IFs with them.  RFI can cause bad {\tt
  SY} values in heavily affected frequency ranges.  For experimentation
purposes, {\tt SY2TY} was written to translate {\tt SY} tables into
{\tt TY} tables for use by tasks that only understand the latter.

\subsubsection{Miscellaneous EVLA-driven matters}

\begin{description}
\myitem{DOOSRO} is a new {\tt RUN} file which enables a pipeline to
      reduce EVLA data taken in forms not too much larger than the old
      VLA.
\myitem{REWAY} now has the option to use the {\tt INTTIM} random
      parameter rather than the incoming weights or 1.0.  The output
      weights do not seem to change much with this option.
\myitem{NOIFS} had a significant error in scaling the $uvw$'s causing
      a scale change in the images corresponding to half of an IF
      bandwidth.
\myitem{SOUSP} can update the fluxes in the {\tt SU} table with the
      fit spectral-index curve under control of a {\tt DOCONFRM}
      option.
\myitem{UVFLG} now has {\tt QUAL} and {\tt CALCODE} to modify the
      sources it uses.  Assumptions about how to recognize missing
      antennas were corrected.
\myitem{UVFND} now has adverbs {\tt QUAL}, {\tt CALCODE}, {\tt
      ANTENNAS}, and {\tt BASELINE} to modify which data are examined.
\myitem{VLANT} does work for EVLA data as well as VLA data, using
      different tables and coordinate conventions.  The help file used
      to discourage EVLA users.
\end{description}

\subsection{More general $uv$-data matters}

{\tt CALIB} was given a {\tt NORMALIZ} adverb to control how the
amplitude gains are normalized.  Choices are a global normalization or
separate normalizations over subarray, over subarray and IF, or over
subarray, IF, and polarization.  The average gain modulus can be
either the mean of the median.  {\tt CALIB} was changed to make output
data sets consistent with \AIPS\ structures even when the input files
are not entirely normal.  It was also given {\tt DOAPPLY} to control
whether the {\tt SN} table is applied to the data to write out a
single-source calibrated data set.  The counting of failed solutions
was upgraded to separate all polarizations, IFs, and antennas with no
data from those that failed for other, more interesting, reasons.


\begin{description}
\myitem{Many} more simultaneous flags can now be handled.  The normal
      limit was raised from 6000 to 30000 and the special {\tt UVCOP}
      limit was raised from 120000 to 200000.
\myitem{Stokes} selection was not taken into account in copying
      tables.  If a task selected only LL polarization, the RR values
      from the {\tt CL} and {\tt SN} tables would have been applied to
      it.  All table copying now takes the Stokes selection into
      account.
\myitem{FRING} did not test for failure to get all the pseudo-array
      memory it needs and so wandered into the weeds in some large
      cases.  The exhaustive search method had an error which caused
      the answers to be correct only for the first IF.
\myitem{Index} tables are automatically created and filled by many
      tasks as they copy $uv$ data.  The method was changed to use the
      index table of the input file as a guide for the scan break
      times for the output file.  This lets the initial scan structure
      (if known as with {\tt BDF2AIPS}) to be preserved.  An error
      when dealing with multiple subarrays was corrected.
\myitem{FLAGR} can do robust rather than simple averaging on the
      spectra which will alter what is flagged.  A combination of {\tt
        RFLAG} and {\tt FLAGR} should be very powerful.
\myitem{REWGT} is a new task to scale data weights with factors
      depending on subarray or IF.
\myitem{TIORD} now tests for baseline numbering order as well as time
      order.
\myitem{DBCON} now appends {\tt SY} tables and will append all
      versions of tables found in the inputs to corresponding versions
      in the output.  {\tt CALCODE} is now used to distinguish
      sources.  An index table is made, but it ignores those in the
      input files.
\myitem{OOP} tasks processing $uv$ data could write things to the
      wrong files if there was a file of the same output name on the
      disk with a higher sequence number than that of the current
      output.
\myitem{3C138} model for X-band had a position error of 0.25 arc
      seconds.
\myitem{Multi} source files are those with a {\tt SOURCE} random
      parameter.  Tasks that confused the presence of a source table
      as a sure indicator of a multi-source file were corrected.
\myitem{APCAL} now allows the {\tt ANTENNAS} adverb to convey a list
      of antennas to be omitted in the usual way.  Previous usage
      allowed only one omitted antenna and was very non-standard.
\myitem{SPLAT} was corrected to use all 16 characters of the source
      names and to support source de-selection properly in the
      SOURCES adverb.
\myitem{UVCOP} was changed to apply flags to copied {\tt SY} tables
      (it already did {\tt TY} and {\tt SN}) and corrected to use all
      characters of the source name.  You now have to tell {\it not}
      to flag tables should you desire that action.
\myitem{SETJY} will now issue messages when the current standard
      calibrator flux differs by more than 3\%\ from the previous
      standards.  Spectral index is reported more clearly in both the
      messages and the history file.
\myitem{PRTUV} uses the index table when possible to speed up finding
      the beginning of the desired data.
\end{description}

\subsection{Imaging}

\AIPS\ imaging task {\tt IMAGR} received only a little attentions in
the last six months.  The formula for {\tt FACTOR} used with
multi-scale imaging was changed to a more stable one.  When doing
baseline-length time averaging on the fly, {\tt IMAGR} had an error
addressing multiple spectral channels in a bandwidth synthesis.  In
order to speed the search for the next facet, {\tt IMAGR} actually
images multiple facets each time with the number controlled by the
size of the work file and limited by user parameter {\tt
  IMAGRPRM(18)}\@.  This has been found to slow the task down after
reaching the stage in which all facets are similar in their peak
values.  Therefore, {\tt IMAGR} was changed to reduce the number of
simultaneously imaged facets as the Clean progresses, but to raise
that number again if it encounters difficulties in finding the next
facet to Clean.

The OBIT software system written and maintained by Bill Cotton at NRAO
in Charlottesville contains a variety of interesting programs besides
the SDM/BDF reading tasks mentioned above.  Perhaps of greatest
interest is the program {\tt Imager} which images a full field of view
with options for outlying sources, multi-scale Clean,
self-calibration, and even peeling.  In addition to a variety of
algorithmic enhancements, {\tt Imager} can run with multiple threads
allowing significant performance advantages over \AIPS' single-thread
programs.  {\tt AIPS} now offers four verbs {\tt OBITMAP}, {\tt
  OBITIMAG}, {\tt OBITSCAL}, and {\tt OBITPEEL} which expose
increasingly complicated interfaces to {\tt Imager}.  All of them
prepare a ``run'' file in the user's {\tt \$HOME} area and then order
{\tt ObitTalk} to execute the run file, writing a log file in the {\tt
  \$HOME} area.  If {\tt DOWAIT=2}, {\tt AIPS} will wait for {\tt
  Imager} to finish and then display the messages under control of
{\tt DOCRT}\@.  With {\tt DOWAIT = 1}, which is recommended, {\tt
  AIPS} will display the log file messages as they are generated and
resume when {\tt Imager} is finished.  Note that this is a powerful
and complex program and is not maintained by the \AIPS\ Group.  It
requires that the OBIT package be installed and maintained at your
site.  This should not be difficult.

\subsection{Display}

\begin{description}
\myitem{HUEWEDGE} \hspace{0.3in} is a new verb to display a wedge in
      two dimensions suitable for accompanying a hue-intensity display
      (such as total hydrogen controls intensity and velocity controls
      color).  The double wedge can be labeled in both axes.
\myitem{PRTAB} was given an option to limit the display to those rows
      that match (or not match) a specified value in a specified
      column.
\myitem{POSSM} was changed to use the correct frequencies when
      applying shifts to the visibilities, to determine the desired
      Stokes before setting pointers for plotting auto-correlations,
      to select the desired Stokes for tables and plot all valid data
      even if one of the Stokes is fully blanked.  {\tt FACTOR} got an
      additional meaning suppress connecting lines when $> 100$.
\myitem{OFMGET} was changed to allow it to get only one of the colors
      of an OFM, rather than all three.
\myitem{LWPLA} used an incorrect {\tt 'LG'} transfer function; the
      correct one is much better and matches that used elsewhere.
\myitem{TVROAM} was enhanced to initialize the look-up-tables of all
      channels to match those of a specified channel and to allow
      modification of all LUTs during roaming (on button {\tt A})\@.
\myitem{SNPLT} was changed to read the table to be plotted in order to
      determine how many records are used for each antenna.  It then
      allocates enough memory dynamically to plot even the largest
      {\tt SY} tables.  A {\tt DOSCAN} option was added to plot marks
      at scan boundaries as defined by the index table.
\myitem{FRPLT} was changed to allow more times, to determine sensible
      defaults for start and stop times, to initialize all line
      drawings to avoid plotting from a non-data point, to handle
      missing data sensibly rather than over-plotting, to label $x$
      axes correctly when plotting more than one IF and/or
      polarization, to do time binning carefully rather than assuming
      exact times, to use {\tt FACTOR} to control the plotting of
      symbols and/or connecting lines, and to display antenna numbers
      as well as names on each plot panel.
\end{description}

\subsection{Image analysis}

\subsubsection{FARS}

{\tt FARS} is an experimental task written by Leonia Kogan to perform
``Faraday rotation synthesis'' on a group of Q and U polarization
images taken over a range of frequencies.  It performs a Fourier
transform at each spatial coordinate of the set of $\lambda^2$.  It
can then do a one-dimensional Clean along this axis.  The final output
is a cube over rotation measure and two spatial axes of the real part
and a second cube of the imaginary part of this operation.  A variety
of minor improvements were made to {\tt FARS} over the last six
months, allowing full Cleaned output, supporting {\tt BLC}/{\tt TRC}
properly, computing frequencies from the {\tt FQ} table or header as
appropriate, and so on.

To simplify this operation, a new task {\tt AFARS} was written.  It
writes out an image at each spatial coordinate of the rotation measure
at which the maximum amplitude occurs.  It also writes an image of
that maximum amplitude or of the phase at that position.

Further simplification is provided by a new {\tt RUN} file {\tt
  DOFARS}\@.  This procedure runs {\tt RSPEC} (see below), transposes
both Q and U cubes with {\tt TRANS}, runs {\tt FARS}, and then
transposes the output cubes, leaving both $zxy$ and $xyz$ cubes
behind.

\subsubsection{Other analysis changes}
\begin{description}
\myitem{RSPEC} was changed to offer the option to output a
      signal-to-noise image rather than plotting/writing the spectrum
      of noise.  To support {\tt FARS}, the option to write a text
      file of normalized weights as a function of channel was added.
\myitem{IMFIT,} {\tt JMFIT} and {\tt SAD} were changed to use the
      actual Clean beam for the particular channel and to display
      results in Jy/headerbeam and Jy/actualbeam.
\myitem{IM2CC} is a new task to convert a model image (Jy/pixel) over
      a wide area into multiple facets with Clean component files.
      This should make it easier to use models generated with other
      software systems in \AIPS\ tasks such as {\tt CALIB}\@.
\end{description}

\subsection{General}

\begin{description}
\myitem{RMS} computations are found generally to be quite demanding
      and {\tt DOUBLE PRECISION} is now used in all such computations
      within \AIPS\@.
\myitem{CookBook} chapter 4 was updated to reflect the revisions in
      {\tt TVFLG}.  Appendix E changed to describe {\tt PCAL} and {\tt
        RLDIF} more correctly, to describe the use of Tsys in {\tt SY}
      tables, and the flagging of tables in {\tt SNEDT} and {\tt
        TYSMO}.
\myitem{USERNO} is a re-created program to process user number lists
      into an interesting format for printing.
\myitem{ZMI2.C} file size error messages needed additional format
      control for 64-bit computers.
\end{description}

%\vfill\eject

\section{Patch Distribution for \OLDNAME}

Important bug fixes and selected improvements in \OLDNAME\ can be
downloaded via the Web beginning at:

\begin{center}
\vskip -10pt
{\tt http://www.aoc.nrao.edu/aips/patch.html}
\vskip -10pt
\end{center}

Alternatively one can use {\it anonymous} \ftp\ to the NRAO server
{\tt ftp.aoc.nrao.edu}.  Documentation about patches to a release is
placed on this site at {\tt pub/software/aips/}{\it release-name} and
the code is placed in suitable sub-directories below this.  As bugs in
\NEWNAME\ are found, they are simply corrected since \NEWNAME\ remains
under development.  Corrections and additions are made with a midnight
job rather than with manual patches.  Since we now have many binary
installations, the patch system has changed.  We now actually patch
the master version of \OLDNAME, which means that a MNJ run on
\OLDNAME\ after the patch will fetch the corrected code and/or
binaries rather than failing.  Also, installations of \OLDNAME\ after
the patch date will contain the corrected code.

The \OLDNAME\ release has had a number of important patches:
\begin{enumerate}
  \item\ {\tt UVCOP} used only 12 of the 16 characters of the {\tt
       SOURCE} adverbs. {\it 2011-01-18}
  \item\ {\tt DBCON} was vulnerable to errors in table headers causing
       absurd disk file size expansions. {\it 2011-01-18}
  \item\ {\tt FRING} got wrong answers from the exhaustive search
       method for IFS higher than 1. {\it 2011-01-21}
  \item\ OOP-based $uv$ tasks could write to the wrong output file.
       {\it 2011-01-21}
  \item\ {\tt CALIB} failed to build the output header correctly for
       non-standard single-source data sets. {\it 2011-01-24}
  \item\ {\tt RLDIF} did not handle one source, continuum corrections
       properly. {\it 2011-01-24}
  \item\ {\tt SWAPR} ignore autocorrelation data. {\it 2011-01-24}
  \item\ {\tt SNPLT} did not handle PC table phases correctly. {\it
       2011-03-23}
  \item\ {\tt IMAGR} did not average multi-channel data correctly when
       doing on-the-fly baseline-length time averaging. {\it 2011-04-15}
  \item\ {\tt NX} tables were written incorrectly when there were
       multiple subarrays. {\it 2011-04-25}
  \item\ {\tt IMEAN} did not write the text file output correctly.
       {\it 2011-06-08}
  \item\ {\tt NOIFS} did not scale the UVW's correctly
       {\it 2011-06-14}
  \item\ {\tt CVEL} did not die on fatal errors such as missing
       bandpasses {\it 2011-06-23}
  \item\ {\tt FITLD} and friends did not read table header
       character-valued keywords correctly {\it 2011-06-24}
\end{enumerate}

\vfill\eject

% mailer page
% \cleardoublepage
\pagestyle{empty}
 \vbox to 4.4in{
  \vspace{12pt}
%  \vfill
\centerline{\resizebox{!}{3.2in}{\includegraphics{FIG/Mandrill.eps}}}
%  \centerline{\rotatebox{-90}{\resizebox{!}{3.5in}{%
%  \includegraphics{FIG/Mandrill.color.plt}}}}
  \vspace{12pt}
  \centerline{{\huge \tt \AIPRELEASE}}
  \vspace{12pt}
  \vfill}
\phantom{...}
\centerline{\resizebox{!}{!}{\includegraphics{FIG/AIPSLETS.PS}}}

\end{document}

\section{\AIPS\ Distribution}

We are now able to log apparent MNJ accesses and downloads of the tar
balls.  We count these by unique IP address.  Since some systems
assign the same computer different IP addresses at different times,
this will be a bit of an over-estimate of actual sites/computers.
However, a single IP address is often used to provide \AIPS\ to a
number of computers, so these numbers are probably an under-estimate
of the number of computers running current versions of \AIPS\@. In
2011 {\it 2010}, there have been a total of 1309 {\it 1433} IP
addresses so far that have accessed the NRAO cvs master.  Each of
these has at least installed \AIPS\ and ??? {\it 340} appear to have
run the MNJ on \RELEASENAME\ at least occasionally.  During 2011 more
than 181 {\it 205} IP addresses have downloaded the frozen form of
\OLDNAME, while more than 583 {\it 692} IP addresses have downloaded
\RELEASENAME\@.  The binary version was accessed for installation or
MNJs by 297 {\it 369} sites in \OLDNAME\ and 537 {\it 644} sites in
\RELEASENAME\@.  The attached figure shows the cumulative number of
unique sites, cvs access sites, and binary and tar-ball download sites
known to us as a function of week --- so far --- in 2011.  These
numbers are quite noticably less than those reported one year ago at
this time for last year's releases.

\centerline{\resizebox{!}{3.1in}{\includegraphics{FIG/PLOTIT11a.PS}}}

%\hphantom{.}
%\vfill
%\centerline{This page deliberately left blank}
\vfill
\eject
