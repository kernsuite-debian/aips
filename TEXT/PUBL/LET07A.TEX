%-----------------------------------------------------------------------
%;  Copyright (C) 2007
%;  Associated Universities, Inc. Washington DC, USA.
%;
%;  This program is free software; you can redistribute it and/or
%;  modify it under the terms of the GNU General Public License as
%;  published by the Free Software Foundation; either version 2 of
%;  the License, or (at your option) any later version.
%;
%;  This program is distributed in the hope that it will be useful,
%;  but WITHOUT ANY WARRANTY; without even the implied warranty of
%;  MERCHANTABILITY or FITNESS FOR A PARTICULAR PURPOSE.  See the
%;  GNU General Public License for more details.
%;
%;  You should have received a copy of the GNU General Public
%;  License along with this program; if not, write to the Free
%;  Software Foundation, Inc., 675 Massachusetts Ave, Cambridge,
%;  MA 02139, USA.
%;
%;  Correspondence concerning AIPS should be addressed as follows:
%;          Internet email: aipsmail@nrao.edu.
%;          Postal address: AIPS Project Office
%;                          National Radio Astronomy Observatory
%;                          520 Edgemont Road
%;                          Charlottesville, VA 22903-2475 USA
%-----------------------------------------------------------------------
%Body of intermediate AIPSletter for 31 December 2006

\documentclass[twoside]{article}
\usepackage{graphics}

\newcommand{\AIPRELEASE}{June 30, 2007}
\newcommand{\AIPVOLUME}{Volume XXVII}
\newcommand{\AIPNUMBER}{Number 1}
\newcommand{\RELEASENAME}{{\tt 31DEC07}}
\newcommand{\NEWNAME}{{\tt 31DEC07}}
\newcommand{\OLDNAME}{{\tt 31DEC06}}

%macros and title page format for the \AIPS\ letter.
\input LET98.MAC
%\input psfig

\newcommand{\MYSpace}{-11pt}

\normalstyle

\section{General developments in \AIPS}

\subsection{FILLM}

\OLDNAME\ contains a revision of {\tt FILLM} which is essential to
support the new data form to be produced by the VLA beginning in July
2007.  VLA users will have to upgrade their copy of \AIPS\ to
\RELEASENAME\ or \OLDNAME\ by that time.  See below for more
information about {\tt FILLM}, including important patches to the
\OLDNAME\ version of {\tt FILLM}\@.  Changes which are not in the
\OLDNAME\ version may also be important to current observers.

\subsection{Current and future releases}

We have formal \AIPS\ releases on an annual basis.  While all
architectures can do a full installation from the source files,
Linux, Solaris, and MacIntosh OS/X (PPC and Intel) systems may install
binary versions of recent releases.  The next release is called
\RELEASENAME\ and remains under active development.  You may fetch and
install a copy of this version at any time using {\it anonymous} {\tt
  ftp} for source-only copies and {\tt rsync} for binary copies.  This
\Aipsletter\ is intended to advise you of improvements to date in this
new release. Having fetched \RELEASENAME, you may update your
installation whenever you want by running the so-called ``Midnight
Job'' (MNJ) which uses {\tt cvs} and transaction files to copy and
compile the code selectively based on the code changes and
compilations we have done.  The MNJ will also update sites that have
done a binary installation using {\tt cvs} and {\tt rsync}.  There is
a guide to the install script and an \AIPS\ Manager FAQ page on the
\AIPS\ web site.

The MNJ serves up \AIPS\ incrementally using the Unix tool {\tt cvs}
running with anonymous ftp.  The binary MNJ also uses the tool {\tt
rsync} as does the binary installation.  Linux sites will almost
certainly have {\tt cvs} installed; other sites may have installed it
along with other GNU tools.  Secondary MNJs will still be possible
using {\tt ssh} or {\tt rcp} or NFS as with previous releases.  We
have found that {\tt cvs} works very well, although it has one quirk.
If a site modifies a file locally, but in an \AIPS-standard directory,
{\tt cvs} will detect the modification and attempt to reconcile the
local version with the NRAO-supplied version.  This usually produces a
file that will not compile or run as intended.

\AIPS\ is now copyright \copyright\ 1995 through 2007 by Associated
Universities, Inc., NRAO's parent corporation, but may be made freely
available under the terms of the Free Software Foundation's General
Public License (GPL)\@.  This means that User Agreements are no longer
required, that \AIPS\ may be obtained via anonymous ftp without
contacting NRAO, and that the software may be redistributed (and/or
modified), under certain conditions.  The full text of the GPL can be
found in the \texttt{15JUL95} \Aipsletter, in each copy of \AIPS\
releases, and on the web at {\tt
  http://www.aoc.nrao.edu/aips/COPYING}.
\vfill\eject

\section{NRAO is 50 years old}

The NRAO is celebrating is $50^{\rm th}$ year of existence.  All
sites, including the high site in Chile, were connected by television
conferencing for brief remarks and a ceremonial cutting of cakes at
each site.  A symposium was held June 17--21 in Charlottesville,
Virginia to celebrate the event entitled {\it Frontiers of
Astrophysics: A Celebration of NRAO's 50th Anniversary: Astrophysics
Across the Electromagnetic Spectrum}.  To obtain more information,
see {\tt http://www.nrao.edu/50/}.  It is hoped that the PowerPoint
talks will appear on the web and that the whole symposium, which was
said to be very good, will appear in a book.

Among the many letters received by the NRAO congratulating us on this
anniversary was one which we thought would be of interest to \AIPS\
users.  It reads:

``Fifty years is a long time for the Italian radio astronomical
community, which started 43 years ago with the completion of the
Northern Cross in Bologna.  It is probably fair to say that early
interaction between Italians and US radio astronomers was at most
sporadic at that time, with very little collaboration.

The construction of the VLA changed all this, and from the 1970s on,
contacts have been intense.  A whole generation of Italian radio
astronomers could make use of advanced instrumentation and became
strong admirers of the ``American way'' of doing science.  Curiously
enough perhaps it was not the hardware the most influential of all but
a seemingly minor software development, which became known in the
whole world as AIPS\@.  The possibility to bring the software back
home and have it run on extremely powerful computers (VAX) really
changed the way in which scientific research was done.  NRAO has made
this revolution possible.

In the early eighties one still had to go to the VLA site to do the
reduction of your own data: down there it took no more than a few
minutes to get 100 CLEAN components, and this breathtaking speed could
only be reached with the super-modern VAXes at the site.  Many Italian
astronomers have beautiful memories of nights spent at the VLA site,
working in the AIPS-cage (after having struggled to reserve time on
the VAX --- only the feeding of sharks must have been a more ferocious
and bloody show) and consuming gallons of (according to Italian
taste\footnote{Editor's note: not just Italian$\ldots$}) awful coffee.

The next 50 years of NRAO will no doubt be quite different in nature,
with personal contacts reduced because of high-speed electronic
connections.  Anyway, the future of radio astronomy is bright, owing
to intense international collaborations and the new generation radio
telescopes.  Obviously the role of NRAO in the planned new generation
instruments will be fundamental.  The international radio astronomical
community can only rejoice.''

Signed for The Director and the Staff of the Istituto di
Radioastronomia INAF by Dr.ssa Luigina Feretti.

\section{Improvements of interest in \RELEASENAME}

We expect to continue publishing the  \Aipsletter\ approximately every
six months along with the annual releases.  There have been quite a
few changes in \RELEASENAME\ in the last six months.  Many have been
the usual bug fixes, but there have also been a number of new verbs,
tasks, and procedures as well as significant improvements in existing
tasks.  New tasks include {\tt OOSUB} to apply a model to visibility
data with spectral-dependent corrections, {\tt VLANT} to correct
phases for corrections in the antenna locations at the VLA, {\tt
CUBIT} (written by Judith Irwin) to fit models to full spectral cubes,
and {\tt UVDI1} to subtract the average of one $uv$ data set from
another.  New verbs include {\tt VLA}, {\tt EVLA}, {\tt VLBA} and {\tt
HSA} which read an antenna table and place suitable antenna numbers on
the \POPS\ stack.

{\tt 31DEC04} through \RELEASENAME\ use a new numbering scheme for
magnetic tape logical unit numbers that is incompatible with previous
versions.  Thus all tape tasks and the server {\tt TPMON} must be from
one of these four releases.  Other than this, \RELEASENAME\ is
compatible in all major ways with the with the {\tt 15OCT98} and later
releases.  There are significant incompatibilities with older versions.

\vfill\eject

\subsection{UV data calibration and handling}

It has come to our attention that users now trust \AIPS\ altogether
too much.  These users have assumed that all \AIPS\ tasks apply a flag
table if it is present.  But that is not the case.  The simple rules
are: if {\tt FLAGVER} is not an adverb to a task, flag tables are not
applied in that task.  If {\tt DOCALIB} is not an adverb to a task,
then {\tt SN} and {\tt CL} tables are not applied in that task.  If
{\tt DOBAND} is not an adverb to a task, then bandpass tables are not
applied in that task.  We have begun a review of all \uv-data tasks to
determine which tasks would benefit by having flagging and calibration
options.  A number of changes have been made to the calibration system
to enable the changes to the old, non-calibration tasks.  Note that
the package of data reading-flagging-calibration routines have
limitations, primarily to restrict the output data to one sub-array and
one frequency ID\@.  Thus the newly modified tasks, while acquiring
new capabilities, may have some of their previous capabilities
curtailed.

The most sweeping change to the calibration routines is to allow the
data to be in sort orders other than {\tt 'T*'} where possible.  For
non-time sorted data, flagging is limited to 6000 flags applying to
the full data set.  Time sorted data work when no more than 6000 flags
apply to a single time.  If the data are not in time order, then
time-dependent calibrations, {\tt DOCAL} true and {\tt DOBAND} greater
than 1, are not allowed.  Time independent bandpasses and polarization
calibration are allowed in all sort orders.  The tasks that apply
flagging with their own flag routines, {\tt UVCOP}, {\tt AVSPC}, {\tt
  CVEL} and {\tt SDVEL}, were also modified to work with greater
restrictions on data that is not in time order.  The labeling and
selection of polarizations in the calibration package was generalized
to allow linear polarization (with many limitations), to allow a few
additional {\tt STOKES} values, such as {\tt 'CROS'} and {\tt 'RLLR'},
and to allow specification that the I Stokes value is to be ``formal''
I\@.  This last means that Stokes I is flagged unless {\it both} RR
and LL or XX and YY are present.  The new {\tt STOKES} values {\tt
  'F'} and {\tt 'FQU'} request this.  See the revised {\tt STOKES} help
file for details.

Tasks {\tt FUDGE}, {\tt UVLSF} and {\tt UVLSD} have been rearranged to
offer the full suite of calibration adverbs.  {\tt FUDGE} is a
``paraform'' task used as a model for the creation of many ``simple''
one-data-set-in, one-data-set-out tasks and provides a pattern for
changing many of the older, non-calibration tasks into the more
complete form.  The other two tasks really do not need the calibration
adverbs, but flagging, particularly on a channel-dependent basis, will
improve their performance.

\subsubsection{UVMTH}

Recently it was discovered that the VLA had developed a ``correlator
bias.''  This appears as a baseline, spectral channel, polarization,
and IF dependent offset in the visibilities.  When imaged, an object
appears at the phase reference position (that used during observing)
which cannot be Cleaned like true sources.  In the case of the VLA, it
appears that this bias has been present from September 2006 until late
in June 2007.  It is known to affect spectral-line observations at
25-MHz bandwidth, producing an ``object'' of some 100's of
micro-Jansky.  It is believed not to affect continuum data at this
bandwidth.  We have been unable to determine whether spectral-line
observations at narrower bandwidths are similarly affected.  The
effect is seen only when there are adequate integration time and
bandwidth to reduce the noise below these levels and when there is no
strong source at the field center masking the offending object.  It is
thought that the VLA bias is only mildly time dependent, allowing the
bias to be measured over significant periods during an observing run
and applied to other times during that run.  Measurements on
consecutive days did show significant changes in the bias.  {\tt
  UVMTH} is a newly-upgraded task that can help VLA observers deal
with this problem.  It may also be useful with GMRT data, for which
there is also evidence of correlator bias.

{\tt UVMTH} averages the visibilities in a \uv\ data set on each
baseline and correlator and adds them to, subtracts them from,
multiplies them by, or divides them into the visibilities in another
data set.  It was written in 1989 and then mostly forgotten.  With the
reports of correlator bias, this task was revised to apply all of the
data selection, calibration, and flagging options to the first input
\uv\ data set while averaging each baseline and correlator over all
{\tt TIMERANG}\@.  The result is applied to the second data set which
is read, modified, and written with no application of data selection
(other than sub-array and frequency ID), flagging, or calibration.
For the VLA correlator bias problem (and a similar one sometimes seen
in GMRT data), the user should specify the same data set for both
inputs.  {\tt SOURCES} should select those pointings with no
significant source at the center of the field and {\tt FLAGVER} should
select flags for at least these sources, but all calibration should be
turned off.  The default subtraction operation should be selected.
The new task {\tt UVDI1} was written to do a similar operation.  It
requires a single-source data set for the averaging and does not apply
flagging or calibration to either data set.  It can time-average the
output difference data, however.  {\tt UVAVG} has a similar operation
which subtracts a full average of a data set from itself with optional
averaging.  At present, this {\tt SUBT} operation functions correctly
only on single-source data,  and there is some thought to removing the
option when calibration adverbs are added to {\tt UVAVG}\@.

\subsubsection{{\tt FILLM}}

{\tt FILLM} is the old task that converts the VLA on-line data format
into an \AIPS\ format.  It applies to data from the now old ModComp
controlled system and from the new control system (through IDCAF) as
long as the old correlator is in use.  At the end of 2006, it was
revised to handle the data scaling difference between the ModComp
output and the new IDCAF output.  Two significant bug fixes were
patched into the {\tt 31DEC06} release as well.  One caused the loss of
the first data record from the next input disk file.  At best this
meant simply a loss of a single integration time.  But, if the mode
changed at the file boundary, that change might have gone
unrecognized.  If the number of polarizations or IFs changed, then
errors would be recognized quickly when writing the {\tt CL} table.  A
lingering fear that data might have gotten mixed without recognition
of a problem remains.  The second ``error'' was a poor decision to use
an embedded blank in station names for the VLA\@.  This did not
confuse \AIPS, but other software packages were not so lucky.  The
names are now {\tt VLA:\_}{\it xxx} and {\tt EVLA:}{\it xxx}, where
{\it xxx} is the station name, such as {\tt W1} or {N72}\@.

Changes which did not make it to patch include the addition of support
for Z mode observing.  In that mode, true zero-lag data are retained,
offering the possibility of new tasks to re-appodize the spectra.  The
master pad is now correctly recognized as a valid station.  EVLA
antennas do not have a ``front-end'' ${\rm T}_{\rm sys}$, so the
back-end one is used for them at all times.  {\tt FILLM} was changed
to write the data with the ``normal'' \AIPS\ axis order: Stokes,
frequency, and IF\@.  Previously, it wrote whatever order was
convenient to the observing mode, usually frequency, Stokes, and IF,
which caused trouble when comparing freshly loaded data sets to data
that had had some \AIPS\ processing down to them.  Support was added
for the ten ``official'' EVLA band codes and separation frequencies.
The new telescopes are able to, and do, observe at frequencies which
were not understood previously.  The default handling of {\tt
TIMERANG} was changed to that which users now expect and the
mischievous adverb {\tt DOALL} was removed.  The bad exponent message
was curtailed, and a summary of bad exponents was added at the end.
It was discovered that {\tt FILLM} would load two data files into one
output, without distinguishing them, even when the two were taken with
different configurations.  Code sensitive to any change in antenna
position greater than 0.02 ns was added to prevent this.  Note that
this test cannot be made more sensitive for ModComp-recorded data.
The antenna positions recorded by the ModComps are changed
continuously for earth tides and polar wander.  The routine that
attempted to correct antenna locations for the shift that must
occur when Pie Town is used at Q band was found to be defective.  It
was generalized so that it will work without reference to band.

\subsubsection{{\tt VLANT}}

When an antenna at the VLA is moved to a new station, it is pressed
into service after a brief check of its pointing parameters.  At a
later date, a significant period of time is used to observe a wide
variety of calibration sources in order to determine improved
estimates of the true locations of the antennas.  These data are
reduced by the data analysts using a procedure that invokes \AIPS\
tasks {\tt CALIB} and {\tt LOCIT} and suitable changes are then made
to the antenna database at the telescope.  These changes, as
increments, are also added to a hand-maintained text file available to
users.  The web site contains an access tool to these text files and a
complicated prescription for how the user should add up corrections
and apply them to his/her data.  The changes were to be made, one
antenna at a time, via {\tt CLCOR} operation {\tt ANTP}\@.

We suspect that users seldom did this; the corrections, while useful,
were usually small enough that they were not worth the trouble for all
but high-frequency data.  To rectify this, a new task called {\tt
VLANT} was written.  The analysts' text files for all available
previous years were cleaned up to remove various formatting issues and
a few obvious errors and then included with \AIPS\@.  {\tt VLANT}
fetches the latest year's text file from the web, if necessary, in
order to have all of the analysts' recent changes.  (Note that this
implies that \AIPS\ must be running on a machine that has web access,
at least for data that require the latest text file.)  Then {\tt
  VLANT} performs the complicated prescription for all antennas and
does the modification of the {\tt AN} table and a new {\tt CL} table
for all antennas at once.  Now there is no excuse for not applying the
best known antenna locations to a users' data.  Note, that antenna
position information continues to be updated until the antenna is
moved from its pad.  This can be many months in some cases.

\subsubsection{{\tt OOSUB}}

{\tt OOSUB} is a version of {\tt UVSUB} written using the object-based
classes of such tasks as {\tt IMAGR}\@.  It offers the
frequency-dependent options found in {\tt IMAGR} to correct for
primary-beam  and spectral-index effects.  It also has a special
option not available anywhere else.  This function allows the user to
omit from the {\tt OOSUB} all {\tt CC}s that are inside the highest
frequency beam, outside the highest frequency beam, inside the lowest
frequency beam, or outside the lowest frequency beam.  The special
option will allow the subtraction of partial models to enable better
imaging of the inner portions of the data and will probably have other
uses when we get clever.  It will be interesting to experiment with
modifications of {\tt PEELR} to, for example, subtract a model in a
frequency-dependent manner, but add it back without the frequency
dependencies.

\subsubsection{Other \uv-editing matters}

\begin{description}
\myitem{TVFLG} and {\tt SPFLG} were changed to use modern rules for
       {\tt FLAGVER}, rather than their curious rules.  Now, if a flag
       table is used on the input, it is copied to a new output flag
       table and the newly-created flags are appended.  A fatal error
       when looping over channels and IFs to apply an existing clip
       pattern was corrected.  The error arose only when one of the
       channels had no data.
\myitem{WIPER} was changed to label sub-windows as well as those
       displaying all of the data.  It now will display up to 2
       antenna pairs contributing to the sample under the cursor, with
       a plus sign added if there are more than 2.  It will now accept
       any reasonable {\tt STOKES} specification, displaying all that
       are requested and flagging things appropriately.  The handling
       of hour angles, elevations, azimuths, and parallactic angles
       was corrected.
\myitem{FLAGR} was changed to look for averaging intervals containing
       only one time (at the ends of scans) which can be reasonably
       included in the previous interval within that scan.  Otherwise,
       they are flagged since they cannot have an rms.  The data
       buffer sizes were increased to allow for four polarizations
       with large numbers of antennas and IFs.  Other changes to
       improve reliability were also made.
\myitem{EDITR} was made more forgiving about format errors when
       reading user-provided numbers.  This forgiveness also
       affects {\tt EDITA}, {\tt SNPLT}, {\tt SCMAP}, and {\tt
       SCIMG}\@.
\end{description}

\subsubsection{Other \uv-display matters}

\begin{description}
\myitem{LISTR} and {\tt RLDIF} were changed to average all requested
       spectral channels in the {\tt MATX} display and to loop
       appropriately over IF\@.  {\tt RLDIF} was corrected to use
       vector averaging between scans as well as within scans.
       {\tt LISTR}'s {\tt GAIN} display required a correction to
       its format.
\myitem{PRTUV} was given the option to print real and imaginary parts
       of the visibility rather than amplitude and phase.  The
       alternate display of random parameters was corrected and
       generalized.
\myitem{VPLOT} was given a new option {\tt 'IFRA'} to plot the ratio
       of two IFs.
\myitem{UVPLT} had all restrictions on {\tt STOKES} removed; all
       requested polarizations will be plotted together.  It had
       corrections for VLBI to the computation of hour angle, azimuth,
       elevation, and parallactic angle.  The display when using such
       axes was changed to show the usual information as well as the
       {\tt REFANT}\@.
\myitem{DTSUM} confused antenna number with position in the antenna
       file.  The displays of which baselines are in the data were
       revised to do the total summary, followed optionally by a
       scan-by-scan summary.  More user and internal controls over
       the displays were added.
\myitem{ANBPL} computed elevation incorrectly for single-source files
       since the source position information was never read.  This
       oversight was corrected.
\myitem{SNPLT} subtracted 6 hours erroneously from all hour angles.
\end{description}

\subsubsection{Other \uv-related matters}

\begin{description}
\myitem{SPLIT} could lose the value of {\tt DOCALIB}, {\tt DOPOL},
       {\tt DOBAND}, and {\tt DOBL} if a source in the list had no
       data found.  All sources later in the list would be written
       out, but with no calibration applied and no warnings.  The code
       was corrected and a patch issued.
\myitem{QUACK} was given new options to limit its function by Stokes
       and IF\@.
\myitem{INDXR} was changed to work with single-source files as well as
       its usual multi-source files.  {\tt CALIB} will use this
       information to good advantage in self-calibration.  {\tt INDXR}
       was changed to support the new EVLA band definitions.
\myitem{CALIB} extended index table times by 5 seconds in each
       direction.  This is disastrous when using a short averaging
       time within scans, so corrected the round-off protection to 0.1
       sec.
\myitem{LOCIT} was given the option of using the difference in the
       phases of the 2 IFs to solve for antenna positions.  This means
       that baseline errors are multiplied by the small frequency
       difference of the 2 IFs rather than the full observing
       frequency.  An optional slope in phase with time was also
       added.  All of this was needed to find the VLA master pad!
\myitem{VBGLU} was improved to reformat flag tables for the new IF
       nomenclature.  {\tt VBGLU} was changed to refuse to glue
       multi-frequency ID data sets.  This is better than making a
       mess of the job.  Gluing of the {\tt STATE} column in {\tt PC}
       tables was corrected.  A multi-dimensioned character column is
       confusing to FITS tables, or at least to \AIPS\@.
\myitem{CVEL} had a programming error cause {\tt CALINI} to issue
       numerous warnings erroneously and another error that could
       cause an abort (with some compilers) before the extension files
       were copied.
\myitem{BLCAL} is now recommended with some VLA data.  It was
       corrected to work with normal multi-source data sets.  It did
       bad things when there was non-zero delays in the {\tt CL}
       table due to a misunderstanding about units.  The task was also
       changed to use dynamic memory instead of rather small buffers.
       Fixed scaling of calibrator source models to be properly IF
       dependent.  (This change was also made a wide variety of other
       tasks.)
\myitem{VLBAUTIL} \hspace{1.5em} was changed to try a variety of
       download methods to fetch files from the web; {\tt wget} is not
       universally available.  It was also changed to split frequency
       IDs before calling {\tt MSORT}\@.  That task could take a
       semi-infinite time to run when two frequency IDs observed at
       the same time were appended.  Such data usually need to be
       separated, then sorted and {\tt VBGLU}ed.
\myitem{UVFIX} was given the option of suppressing differential
       aberration in its computations.  An error of the reference
       channel in frequency was made for uncompressed data when the
       source position was changed.
\myitem{CLCOR} had a second-order error in phase corrections for a
       change in source position which has been corrected.
\myitem{SMOOTH} was left out of 17 tasks that otherwise apply
       calibration.  The oversight was corrected.
\myitem{FITLD} was changed to support bandpass tables within the IDI
       convention.
\myitem{MULTI} used parameters to the precession routine other than
       those used normally in \AIPS\@.  The apparent coordinates in
       the output source table were somewhat affected.
\end{description}

\subsection{Miscellaneous}

\subsubsection{New verbs}

Four new, related verbs have been written.  They are named VLA, EVLA,
VLBA, and HSA and place a temporary numeric array on the \POPS\ stack
for processing by other verbs.  The verbs look in the specified
antenna file to make a list of the antenna numbers that match (or do
not match) the specified array.  The temporary numeric array is a new
concept in \POPS\ and required a number of modifications to the
language and its fundamental operators.  Verbs such as {\tt MAX} and
{\tt PRINT} can now handle the output of such verbs.  There are,
however, a number of limitations to the use of these verbs, the worst
being that one cannot use two of them in the same command.  If you
want to set an adverb array to, for example, the list of EVLA
antennas, you must include an equals sign; thus {\tt ANTENNAS =
  EVLA}\@.

\subsubsection{Other matters}

\begin{description}
\myitem{TABED} was changed to default the output file name parameters
       individually to those of the input.  An inadvertent {\tt
         OUTDISK} with otherwise blank {\tt OUTNAME} sent the output
       table to very unexpected places.
\myitem{DFQID} failed to renumber frequency IDs in the tables.
\myitem{DQUAL} failed to renumber the sources in tables.
\myitem{FXTIM} failed to correct the {\tt TIME} and {\tt TIME RANGE}
       in tables.
\myitem{TAPLT} was corrected to recognize blanked values.  The scaling
       was changed for binned plots to ignore empty bins and plot them
       at the bottom anyway.  The Y function string displayed was
       corrected.
\myitem{KNTR} was changed to allow {\tt ZINC} less than zero to plot
       the planes in the opposite order.  Error reporting, especially
       for crazy polarization lines, was improved.
\end{description}

\subsection{Analysis}

\subsubsection{CUBIT}

Judith Irwin of the University of Toronto wrote {\tt CUBIT} quite some
time ago to fit spectral-line cubes with models of density as a
function of radius in, and height above, a galactic plane and of
velocity as a function of radius.  She has continued to develop the
code and help file over the years and has now kindly made it available
to all \AIPS\ users.  We have modified the code to modernize its
structure and to replace modest fixed arrays with dynamic memory,
allowing cubes of any size to be fit.  This task is not easy to use ---
there are a great many parameters to fit and it takes patience to find
good minima in the very large Chi-squared space.  The results are
rather more general than those found fitting solely moment images.
If you use this task to analyze your data cube and make reference to
the results in a paper, you should reference the original paper
describing this task:  Irwin, Judith A. 1994, ``Arcs and bridges in
the interacting Galaxies NGC 5775/NGC 5774,'' ApJ, 429, 618--633.

\subsubsection{Other analysis matters}

\begin{description}
\myitem{FLATN} was changed to allow it to make images of the expected
       noise or weight (one over noise squared) from mosaic
       observations.  Repaired bug which caused the {\tt NOISE}
       adverb in mosaicing to be ignored, leaving all pointings with
       the same basic weight.  {\tt FLATN} will now use the header
       parameter {\tt ACTNOISE}, if present, as the default for {\tt
       NOISE}\@.
\myitem{VTESS} was changed to test whether the incoming images for a
       mosaic are on the same geometry as required.  Changed {\tt
         UTESS}, {\tt LTESS}, and {\tt STESS} to apply these same
       tests.   Corrected a bug in {\tt STESS} and added a note to its
       help file pointing out that {\tt FLATN} now does its job
       without all the problems with image geometry.
\myitem{SAD} was enhanced to be able to use the optional noise image
       when selecting the islands to be fit and during the decision
       process in which fits are accepted or rejected.  Previously,
       the noise image was only used to modify the uncertainties
       assigned to the results of the fit.
\myitem{SMODEL} was applied without the $w \times z$ term which is
       important at large angles.  The extended source portions of
       {\tt SMODEL} were also ignored by low-level OOP routines.
\myitem{Models} of calibration sources must be scaled by the source
       total flux, which is different for each IF in the data.  {\tt
         CALIB} did this correctly, but most other model application
       tasks ignored this nicety.
\myitem{UVCON} was corrected for problems with its computation of
       elevation and its creation of antenna files.
\end{description}

\subsection{Imaging}

\subsubsection{{\tt IMAGR}}

The TV menu was rearranged to put the choices that stop things well
away from the choices that are frequently used.  This should reduce
the premature terminations.  Added two new TV options - to turn off
Cleaning of a facet now and to turn it back on.  Changed the beam
histogram computation to avoid edges and corners which can be
pathological due to accumulated round-off errors.  Moved the forcing of
{\tt DO3DIM} to true for {\tt OVERLAP=2} mode to before the images are
created.  It is too late after that.  Fixed the positioning of the
inscribed circle which may be drawn optionally to guide the eye in
setting boxes.  Cleaned up the handling of the on-line help which had
stopped working for some options and were too demanding on the
programmer for the ones containing field numbers.

%\subsubsection{Other imaging matters}

\begin{description}
\myitem{SETFC} had an error in the formula for phase error which was
       corrected in the help file and code.  The output is unchanged
       because the default phase error was changed accordingly.  Fixed
       errors handling a radius of zero.  Changed the channel to avoid
       channel 1 which is often fully flagged.  Changed the round up
       levels to recommend larger images rather sooner.
\end{description}

\subsection{General and programming matters}

\AIPS\ does much of its high-performance computing with a cpu-based
emulation of an array processor.  General mathematical routines
function on data located in a large memory accessed via pointers.
Such routines are usually highly optimized by compilers.  Previously,
the size of this memory was determined at compile time and set by the
local \AIPS\ Manager.  A fundamental conflict arose in choosing this
size between those people at a site with older, small-memory computers
and those with large problems to solve.  Furthermore, since the size
of the memory was set independent of the problem, the programmer had
to write algorithms which could adapt to almost any size of memory.
Such algorithms tend to be limited and/or much more difficult to
write.

In {\tt 31DEC07}, the pseudo-AP has been changed to use dynamically
allocated memory.  The programmer specifies the amount of memory
needed when the AP is allocated by {\tt QINIT}\@.  Pointers in regular
code remain normal integers since we assume that we will not allocate
more than 8 Gbytes.  However, the actual subscripts to {\tt APCORE}
are now required to be {\tt LONGINT}s which are the sum of an offset
called {\tt PSAPOF} and the calling routine's pointers.  Computers
which use 8 bytes for address pointers in C require us to use {\tt
INTEGER*8} ``{\tt LONGINT}s\@.''  The pre-processor automatically
handles the conversion of {\tt LONGINT} in the \AIPS\ Fortran code
into the form needed by the local computer.

We had hoped that this change would improve performance by allocating
more memory in many cases to avoid the use of disk inside algorithms.
The standard {\tt Y2K} tests, however, showed no change in
performance, suggesting that 20-Mbyte APs are already adequate for
these tests.  Linux disk I/O does a great deal of in-core caching as
well, which would reduce the differences bewteen ``disk'' and
in-memory algorithms.  The main purpose of this change was to allow
for the creation of new algorithms which would be easier to create and
more efficient by never invoking disk files.

\begin{description}
\myitem{MNJ} reports now contain a list of files noted by the {\tt
       cvs} update, with a character showing whether the file was new,
       updated, different at the remote site, etc.
\myitem{install.pl} was changed to cause {\tt cvs} to ignore more
       of the files it should ignore in its reporting.  The options of
       the {\tt rsync} calls were altered to improve reliability.
       Calls to {\tt chgrp} were added to change the group ownership
       of files to that requested by the installer.
\myitem{Servers} for messages and Tektronix plot emulation are
       programs that run inside an  {\tt xterm} or other suitable
       window program.  The binary releases require the location of
       supplied libraries when these programs start in the spawned
       {\tt xterm}.  The files {\tt UNIXSERVERS} and {\tt XASERVERS}
       were changed to have shell- and architecture-dependent case
       statements to try to set this information where it can be used.
       The situation with these servers seems to have improved with
       this addition, but mysterious troubles continue to be reported.
\end{description}

\section{Patch Distribution for \OLDNAME}

Important bug fixes and selected improvements in \OLDNAME\ can be
downloaded via the Web beginning at:

\begin{center}
\vskip -10pt
{\tt http://www.aoc.nrao.edu/aips/patch.html}
\vskip -10pt
\end{center}

Alternatively one can use {\it anonymous} \ftp\ to the NRAO server
{\tt ftp.aoc.nrao.edu}.  Documentation about patches to a release is
placed on this site at {\tt pub/software/aips/}{\it release-name} and
the code is placed in suitable subdirectories below this.  As bugs in
\NEWNAME\ are found, they are simply corrected since \NEWNAME\ remains
under development.  Corrections and additions are made with a midnight
job rather than with manual patches.  Since we now have many binary
installations, the patch system has changed.  We now actually patch
the master version of \OLDNAME, which means that a MNJ run on
\OLDNAME\ after the patch will fetch the corrected code and/or
binaries rather than failing.  Also, installations of \OLDNAME\ after
the patch date will contain the corrected code.

The \OLDNAME\ release had a number of important patches:
\begin{enumerate}
\item\ {\tt CALIB} handled scan times badly when averaging within a
       scan {\it 2007-01-02}
\item\ {\tt UPDCONTROL} in the MNJ used obsolete syntax for {\tt sort}
       {\it 2007-01-10}
\item\ {\tt UVFIX} did not contain the latest leap second {\it
       2007-02-27}
\item\ {\tt CLCOR} operation {\tt SUND} did not work {\it 2007-02-27}
\item\ {\tt VBGLU} did not handle the {\tt PC} table {\tt STATE}
       column correctly {\it 2007-02-27}
\item\ {\tt CVEL} had a bad call sequence which could cause an abort
       {\it 2007-02-27}
\item\ {\tt SNSMO} had a bad call sequence which could cause bad
       smoothing of rates including flagging them {\it 2007-04-26}
\item\ {\tt UVFIX} had a frequency error for uncompressed data only
       and {\tt CLCOR} had a minor conceptual error both affecting
       phases after a position shift {\it 2007-04-26}
\item\ {\tt BLCAL} and {\tt UVFND} set the integration time to one
       day, causing bad amplitude calibration when there were rates
       and delays {\it 2007-04-26}
\item\ {\tt aips.l} man page was lost {\it 2007-05-04}
\item\ {\tt FILLM} skipped a record at ends of file which could lose a
       data sample and possibly cause confusion if the mode changed
       {\it 2007-05-24}
\item\ {\tt SPLIT} lost the calibration flags when a source was not
       found so that later data did not have calibration applied {\it
       2007-06-10}
\item\ {\tt SNPLT} lost data from phase plots of {\tt PC} tables due
       to failure to check for wraps and got hour angles wrong by 6
       hours {\it 2007-06-14}
\item\ {\tt FILLM} used a blank in the middle of some station names,
       confusing other software packages {\it 2007-06-16}
\end{enumerate}


\section{\AIPS\ Distribution}

We are now able to log apparent MNJ accesses and downloads of the tar
balls.  We count these by unique IP address.  Since dial-up
connections may be assigned different IP addresses at different times,
this will be a bit of an over-estimate of actual sites/computers.
However, a single IP address is often used to provide \AIPS\ to a
number of computers, so these numbers are probably an under-estimate
of the number of computers running current versions of \AIPS\@.
In 2007, there have been a total of 750 IP addresses so far that have
accessed the NRAO cvs master.  Each of these has at least installed
\AIPS\ and 186 appear to have run the MNJ on \RELEASENAME\ at least
occasionally.  During 2007 more than 180 IP addresses have downloaded
the frozen form of \OLDNAME, 117 in binary form, while more than 517
IP addresses have downloaded \RELEASENAME, 346 in binary form.
% The attached figure shows the cumulative number of unique sites, cvs
% access sites, and binary and tar-ball download sites known to us as a
% function of week --- so far --- in 2007.

%\vspace{12pt}

%\centerline{\resizebox{!}{2.9in}{\includegraphics{FIG/PLOTIT7a.PS}}}
\vfill\eject
%\pgskip
%\section{Preview of coming attractions}
%\hphantom{.}
%\vfill
%\centerline{This page deliberately blank.}
%\vfill\eject

% mailer page
% \cleardoublepage
\pagestyle{empty}
 \vbox to 4.4in{
  \vspace{12pt}
%  \vfill
\centerline{\resizebox{!}{3.2in}{\includegraphics{FIG/Mandrill.eps}}}
%  \centerline{\rotatebox{-90}{\resizebox{!}{3.5in}{%
%  \includegraphics{FIG/Mandrill.color.plt}}}}
  \vspace{12pt}
  \centerline{{\huge \tt \AIPRELEASE}}
  \vspace{12pt}
  \vfill}
\phantom{...}
\centerline{\resizebox{!}{!}{\includegraphics{FIG/AIPSLETS.PS}}}

\end{document}
