%-----------------------------------------------------------------------
%;  Copyright (C) 1995, 1996
%;  Associated Universities, Inc. Washington DC, USA.
%;
%;  This program is free software; you can redistribute it and/or
%;  modify it under the terms of the GNU General Public License as
%;  published by the Free Software Foundation; either version 2 of
%;  the License, or (at your option) any later version.
%;
%;  This program is distributed in the hope that it will be useful,
%;  but WITHOUT ANY WARRANTY; without even the implied warranty of
%;  MERCHANTABILITY or FITNESS FOR A PARTICULAR PURPOSE.  See the
%;  GNU General Public License for more details.
%;
%;  You should have received a copy of the GNU General Public
%;  License along with this program; if not, write to the Free
%;  Software Foundation, Inc., 675 Massachusetts Ave, Cambridge,
%;  MA 02139, USA.
%;
%;  Correspondence concerning AIPS should be addressed as follows:
%;          Internet email: aipsmail@nrao.edu.
%;          Postal address: AIPS Project Office
%;                          National Radio Astronomy Observatory
%;                          520 Edgemont Road
%;                          Charlottesville, VA 22903-2475 USA
%-----------------------------------------------------------------------
\input COOK82.MAC
\def\titlea{15-Jan-1996 (revised 18-Jan-1996)}
\def\chapt{Z}
\def\Chapt{46}
\sect{Appendix Z.  SYSTEM-DEPENDENT \AIPS\ TIPS}
\pd

     Although \AIPS\ attempts to be system independent, some aspects
of its use depend inevitably on the specific site.  These vary from
procedural matters (\eg\ assignment of workstations and location of
sign-up sheets, tape drives, and workstations or other terminals) to
the hardware (\eg\ names and numbers of workstations and tape and disk
drives, the parameters of television and array processor devices) to
the peculiar (\eg\ the response of the computer to specific keys on
the terminal, the presence of useful job control procedures).  This
appendix contains information specific to the NRAO's individual \AIPS\
installations.  It is intended that non-NRAO installations replace
this appendix with one describing their own procedures, perhaps using
this version as a template.  The general description of using \AIPS\
on workstations was given in Chapter 2 and will not be repeated here.

    Within the NRAO, \AIPS\ is installed on two main architectures ---
Sun workstations and IBM RS/6000 systems.  All our old CONVEX C--1 and
DEC VAX 11/750 and 11/780 systems have been decommissioned.  Currently
the fastest \AIPS\ machines in the Observatory are the IBM/RS6000
model 580 systems {\tt rhesus} in Charlottesville, and {\tt kiowa} and
its clones at the Array Operations Center.  \AIPS\ also runs in
Charlottesville on a DEC Alpha workstation and on a 486/66 personal
computer.  The former is available for general use and would be an
interesting computer if it had more memory.

\subsect{Z.1. NRAO workstations  --- general information}

     All NRAO workstations run some version of the Unix operating
system, AIX on IBMs and SunOS (or Solaris) on Suns.  Unix systems are
intrinsically sensitive to the difference between upper and lower
case.  Be sure to use the case indicated in the comments and advice
given in the following notes.  \AIPS\ itself is case-insensitive,
however; conversion of lower-case characters to upper-case occurs
automatically.  (Unix systems have a variety of characters for the
prompt at monitor (job-control) level, and allow users to set their
own as well.  We will use {\tt \$} as the prompt in the text below.)

\subsubsect{Z.1.1.  The ``midnight'' jobs}

    The versions of \AIPS\ on all NRAO Sun, IBM, DEC, and Intel
systems are kept up to date continually with the master versions on
the Charlottesville Sun called {\tt baboon}.  This is achieved by
automated jobs that start running at very antisocial hours of the
early morning.  Any changes formally made to the {\us TST} version of
\AIPS\ are copied to the relevant computers and recompiled/relinked.
At the time of writing, midnight jobs run on {\tt rhesus} (CV IBM),
{\tt tamarin} (CV Solaris 2.3), {\tt tarsier} (CV Intel Linux), {\tt
pongo} (CV Dec Alpha), {\tt aguila} (AOC Sun), {\tt laphroaig} (AOC
Solaris 2.3), {\tt kiowa} (AOC IBM), and {\tt miranda} (VLA Sun).  At
``quarterly'' ({\it sic\/}) update time, {\tt NEW} is also updated via
the midnight jobs while it is still unfrozen.

There are at least two sites outside of the NRAO that are also served
by the \indx{midnight job} and this number may be greater by the time
you read this.

\vfill\eject
\subsubsect{Z.1.2.   Generating color hard copy}

\subsubsectt{Z.1.2.1.  Color printers}

     Color printers are, these days, simply printers that understand
the color extensions to the PostScript language used to describe
plots.  The NRAO owns two Tektronix 4511a \indx{color printer}s, one
in Charlottesville and one at the AOC in Socorro.  You may display
your PostScript file on these printers simply by typing
\dispx{{\tt \$}\qs lpr\qs -Ppscolor\qs {\it filename} \CR}{where {\it
        filename} is the name of your file.}
\dispx{{\tt \$}\qs lp\qs -d pscolor\qs {\it filename} \CR}{for Solaris
        systems on Suns}
\dispe{{\it Just before} doing this, make sure that the desired type
of special Tektronix paper (plain or transparency) is in the cassette
mounted in the printer.  The paper size is $8.5 \times 11$ inches,
which is the default for \AIPS\ tasks {\tt TVCPS} and \hbox{{\tt
LWPLA}}.  If you do not wish to save the plot as a disk file, you may
also print it directly from within \hbox{\AIPS}.  The color printer is
one of the printer choices when you start up {\tt AIPS}, but you
probably want to select a regular PostScript printer as your default
printer.  \AIPS\ print routines will re-direct PostScript files that
actually contain color commands to {\tt pscolor}, but will not
re-direct ordinary print jobs to some printer other than {\tt
pscolor}.  There are some special instructions for the color printer
at the AOC in \Sec Z.3.4.}

\subsubsect{Z.1.2.2.  Software to copy your screen}

     To obtain a color hardcopy of what is on your screen, there are
three software options you can choose.  These are {\tt TVCPS}, {\tt
xv}, and {\tt xgrab}.  Having created a PostScript file, you can print
it on color printers and film recorders at the NRAO, or copy the file
via e-mail or {\tt ftp} to some other site for printing.

     The {\tt \tndx{TVCPS}} task in \AIPS\ will create a color
Encapsulated \indx{PostScript} file from whatever is displayed on the
\AIPS\ TV server (\hbox{{\tt XAS}}).  If you use the {\tt
\tndx{OUTFILE}} adverb, this file is saved with whatever name you
specify (see \Sec 2.5).  If you specify a black-and-white output to
{\tt TVCPS}, then the output can be sent to any PostScript printer.
Color PostScript must be sent to {\tt pscolor} or the Solitaire.  You
can, of course, edit the save file (if you are a PostScript wizard)
and can insert the file (since it is encapsulated) in another
document.  See the Charlottesville Workstation Guide for a short
chapter on PostScript

     The {\tt \tndx{xv}} program (in version 2) is a Unix utility
program available on most systems at the \hbox{NRAO}.  (Version 3 of
{\tt xv} is significantly improved, but, due to copyright
restrictions, is not yet generally available at the \hbox{NRAO}.)  It
is mainly intended for image display of GIF, JPG, TIFF, and other
format files.  When you start {\tt xv}, click the right button mouse
anywhere in the {\tt xv} window to bring up the control window.  One
of its features is a screen grab which is controlled by the ``Grab''
button in the lower right corner of the control window.  {\it Before}
you press this, arrange your windows and icons so that you can see
exactly what it is you want to grab (\eg\ the {\tt XAS} server).  Now
press the ``grab'' button.  The bell will ring and the cursor will
change to a white cross symbol; move it to the top left of the area
you want to grab. Then press and hold down the left mouse button, and
drag the mouse cursor until it is at the bottom right of the area you
want to grab. As you do this, you will see a box pattern on the screen
outlining the area selected.  Once you are done selecting the area,
release the mouse cursor.  When {\tt xv} has finished grabbing the
screen, it will beep twice, and whatever you grabbed appears in the
main {\tt xv} window.  You can now use the ``save'' button of the
control window to save this as any format you want.  Once nice feature
of this is the ``save as Postscript'' option.  It allows you to scale,
rotate, and position the image in relation to the page.  Its user
interface is better than most image utilities.

     The {\tt xv} program can either be started by typing {\tt xv}
from the Unix command line, or by choosing the XV option from the
programs menu (which is obtained by pressing the middle mouse button
on the root window of the X-Windows display).

     Finally, the {\tt \tndx{xgrab}} program provides similar
functionality to the ``grab'' feature of {\tt xv}, with fewer output
formats but much more control over how the grabbing is done.  By
default it allows three seconds in between starting the grab and when
it actually starts to read the screen; this can be useful for setting
things up.  Also, it un-maps itself from the window when grabbing so
you don't have to worry about getting it out of the way.
Unfortunately, there appears to be some problems with its encapsulated
PostScript output.

\subsubsect{Z.1.2.3.  Color film recorders}

     The NRAO owns two Solitaire \indx{film recorder}s, one in
Charlottesville and one (currently non-functional) at the AOC in
Socorro.  These recorders support two film sizes, 35mm and $4 \times
5$ (inches that is).  Our normal film type, for which the recorder is
very well calibrated, is Ektachrome 100+ Professional.  The film
recorder has 3 scan resolutions, 2048, 4096 and 8192 lines.  Any one
of these may be used with any input image.  The higher resolution
scans take rather longer to do, but will clearly do a better job on
larger input images.

     Actually, the Solitaire is driven by software called Freedom of
Press ({\tt freedom}) from Custom Applications, Inc.~which
interprets/filters \indx{PostScript} input files into a form
understood by the Solitaire.  (The Freedom of Press software has a
great many other options as well, with a wide variety of output
formats.  See the chapter on the Solitaire in the Charlottesville
``Sun Workstation Notes'' for details.)  For PostScript files that
need no special processing by {\tt freedom}, you may simply ``print''
them to special queues set up for the Solitaire.  These queues are, of
course, actually filtered by {\tt freedom} on the way to the film
recorder. The apparent plot size for the Solitaire is $7.33 \times 11$
inches for 35mm film and $8\times10$ inches for $4\times5$ film.
\AIPS\ tasks {\tt LWPLA} and {\tt \tndx{TVCPS}} Should be given these
dimensions in their inputs for plots intended for NRAO's film
recorders.  These tasks can print directly to the Solitaire queues, if
they are among the choices offered when you start {\tt AIPS}, but we
recommend that you send the plots to disk text files instead.  In this
way, you may send the plots to the film recorder in the queue of your
choice {\tt after} making sure that the device has the correct film
type mounted and ready for use.

     The details of how you may have the film type changed, have the
film processed, get the resulting pictures/slides back, have the unit
cleaned and repaired, and the like differ between the two sites and
will be discussed in the appropriate Sections below.

     In the hopes of producing both better pictures and a usable NRAO
image library, \indx{Pat Smiley} of the \indx{Charlottesville Graphics
department} has volunteered to coordinate the making of photographs
using the Charlottesville film recorder.  To avail yourself of this
offer, you must (1) copy the picture file to the appropriate disk and
(2) notify Pat that you have done this.  In particular, to copy the
file from a Charlottesville computer:
\disps{{\tt \$} cp {\it file\_name}
    /net/solitaire/solitaire/ftp/pub/slides \CR}
\dispe{Use anonymous ftp to copy the file from more remote computers:}
\displx{{\tt \$} ftp ftp.cv.nrao.edu \CR}{or 192.33.115.79}
\displx{{\tt Connected to tarsier.cv.nrao.edu.}}{connect message
               returned}
\displx{{\tt 220 tarsier FTP server ($\ldots$) ready.}}{server
               message returned}
\displx{{\tt Name (ftp.cv.nrao.edu:$\ldots$):} anonymous \CR}{reply
               with {\us anonymous} or {\us ftp}.}
\displx{{\tt 331 Guest login ok, send ident as password.}}{login
               message and suggestion.}
\displxx{{\tt Password: } {\it your\_address} \CR}{reply with your
               account and e-mail address, \eg\ {\tt
               anewman@nrao.edu}; it will not be visible on the
               screen.}
\displx{{\tt 230 Guest login ok, access restrictions apply.}}{login is
               successful}
\displx{{\tt ftp> } cd pub/slides/slides \CR}{change to slides
               directory}
\displxx{{\tt ftp> } put {\it file\_name} \CR}{to copy {\it file\_name}
               from your computer to the Solitaire slides area.}
\displx{{\tt ftp> } quit \CR}{to exit and log out.}
\dispe{Note that {\it file\_name} must be a file in PostScript format
suitable for the film recorder, such as the files produced by the
\AIPS\ task \hbox{{\tt TVCPS}}.  To assist Pat in previewing your
picture, its name should be no longer than eight characters plus an
extension of no more than three characters, all preferably in lower
case.}

     Having copied the file, you should then send Pat an e-mail
message ({\tt psmiley@nrao.edu}) telling her your name, postal and
e-mail addresses, file name, any special requirements you and your
picture may have, and whether you would be willing to allow Pat to
include your picture in the NRAO image library.  There are a number of
reasons to ask Pat for this service, even if you are at the
\hbox{AOC}.  She is able to get 35mm slides processed in one day and
mailed off, if needed, by Federal Express the next day.  Special
requests, such as 4x5 film for very detailed (\eg\ multi-panel) images,
larger format overheads to bring out special detail or Black-and-white
negative films for non-color prints, can be handled a bit more slowly.
Pat cleans the film recorder before making pictures, giving the best
possible quality in the photos.  (The devices have been found to be
``dusty'' intrinsically.)  You may choose to allow Pat to keep your
images in the NRAO image library.  If you do, she will need extra
information from you about the image and the NRAO telescope used, as
well as a release date and a statement allowing NRAO to distribute the
image.  Besides giving your image and science greater currency and
helping the NRAO in its publicity and educational r\^oles, leaving
your image in the NRAO image library will allow you to request
duplicates and prints at any time at no charge.\iodx{Pat Smiley}

\subsubsect{Z.1.3.   Gripe, gripe, gripe, $\ldots$}

    With the effort currently going into \AIPTOO, the ``designated
AIP'' program has been discontinued mainly due to lack of people and
time.  At the AOC in Socorro, contact the \AIPS\ Manager if you have a
problem.  Likewise in Charlottesville.

     Suggestions and complaints entered on NRAO computers with the
{\tt \tndx{GRIPE}} verb (see \Sec 11.1) are collected regularly and,
at least, read.  From {\tt 15JAN96} and later releases, they are sent
immediately by e-mail from any \AIPS\ computer to several addresses in
the \AIPS\ programming group.  The most urgent are addressed and,
sometimes, answered. All gripes are entered into a database which
resides on {\tt zia.aoc.nrao.edu} (otherwise known as {\tt
146.88.1.4}).  Users may read the contents of this database in as much
detail as they can stand.  To do so, login to the account called {\tt
gripe} on {\tt zia}.  This is a ``captive'' account, requiring no
password, and allowing you only to execute an especially prepared
version of the text editor {\tt emacs}.  When you tire and exit the
special {\tt emacs}, you will be logged out of {\tt zia}.

     After you log in, you will be presented with a selection/options
menu.  Fill in and/or alter some of the selection criteria to limit
which gripes you will view.  Then, select display option {\tt index},
and, {\tt only} when you are fully ready, hit a \hbox{\CR}.  You will
be shown a descriptive list of the selected gripes.  If you wish to
read one of them in detail, move the cursor to it and hit \hbox{\CR}.
The space bar gets you the text of the next gripe and typing the
letter {\tt q} returns you to the index.  Another {\tt q} returns you
to the selection/option form.  Typing a {\tt ?} in any of the displays
will provide you with information on all the options available at that
level of the system.

\AIPS\ Memo No.~88 describes the system in some detail.  This memo may
be available on your \AIPS\ system as file {\tt
\$AIPSPUBL/AIPSMEMO88.PS} in PostScript form.  It is also available to
the ``World-Wide Web'' (start with ``URL'' {\tt
http://info.cv.nrao.edu/aips/aips-home.html}) so that it may be
examined and retrieved over the Internet.  The file used is also
available via anonymous {\tt ftp} on host {\tt baboon.cv.nrao.edu} (or
{\tt 192.33.115.103}) as a PostScript file named \hbox{{\tt
/pub/aips/TEXT/PUBL/AIPSMEMO88.PS}}.

\vfill\eject
\subsubsect{Z.1.4.  Solving problems at the NRAO}

     Below are details specific to the Charlottesville and Socorro
systems for handling some of the problems which may arise in
\hbox{\AIPS}.

\subsubsect{Z.1.4.1.  Booting the workstations}

     Modern workstations, especially the powerful IBM RS/6000s, are
complex Unix systems which may have remote users within the NRAO and
guests from elsewhere on the Internet.  Users should {\it never}
attempt to boot the system on their own.  If the machine appears to be
dead, find or call one of the people listed on the bulletin boards in
the \AIPS\ Caige for this purpose.

\subsubsect{Z.1.4.2. Printout fails to appear}

     Check the \AIPS\ output messages that appeared shortly after you
submitted your \indx{print job}, whether it be from {\tt PRTMSG} or
{\tt LWPLA}, or some other task.  You should see the output of the
Unix command to show the printer queue status.  If anything went wrong
with the print submission, an error message should be obvious.  If
not, check the output of the {\tt \tndx{lpq}} (or {\tt \tndx{lpstat}}
for Solaris 2.{\it x}) command, see what print queue was involved, and
check it again from the Unix command level (not from inside \hbox{{\tt
AIPS}}).

     {\tt AIPS} will delete spooled files about 5 minutes after they
are submitted.  If the print queue is stalled (due, say, to a jammed
printer) or backed up with a lot of jobs, it is possible that the file
was deleted before it was gobbled up by the print spooler.  This time
delay has been made a locally-controlled parameter, so it is possible
to set it to values higher than 5 minutes.  At this writing, the
Charlottesville systems are using a 20-minute delay time.

     Finally, check to see if the printout was (a) diverted to the
``big'' printer ({\tt lp27} in room 213 at the AOC or {\tt ps3dup} in
the Charlottesville library) because it was too long for the smaller
printers, (b) you forgot which printer you had selected on {\tt aips}
startup, or, at the AOC, (c) someone has taken the output and filed
it in the ``today'' file bin on the left side of the post directly
behind the {\tt psnet} printer.

\subsubsect{Z.1.4.3.  Stopping excess printout}

     To find out what jobs are in the spooling queue for the relevant
printer, type, at the monitor level:
\dispx{{\tt \$\qs} lpq \CR}{to list default print queue}
\dispx{{\tt \$\qs} lpstat \CR}{to list default print queue under
            Solaris}
\dispe{or to display a specific queue}
\dispx{{\tt \$\qs} lpq\qs -P{\it ppp} \CR}{to show printer {\it ppp}}
\dispx{{\tt \$\qs} lpstat\qs {\it ppp} \CR}{to show printer {\it
            ppp} under Solaris}
\dispe{where {\it ppp} {\it might} be {\tt psnet} at the AOC or {\tt
ps3dup} in Charlottesville.  If the file is still in the queue as job
number {\it nn}, you can type  simply}
\dispx{{\tt \$\qs} \tndx{lprm} -P{\it ppp} {\it nn} \CR}{to remove the
            job}
\dispx{{\tt \$\qs} cancel {\it nn} \CR}{to remove the job under Solaris}
\dispe{{\tt lprm} and {\tt cancel} will announce the names of any files
that they remove and are silent if there are no jobs in the queue which
match the request.}

     Unfortunately, it is now very difficult to stop long print jobs.
The large memories of modern printers mean that more than one print
job can already be resident in the printer while your long unwanted
job is being printed.  Therefore, turning off the printer is not an
option.  Try to be more careful and not generate excess printout in
the first place (save a tree).

     A nice option available for most \AIPS\ print tasks or verbs
is adverb {\tt \tndx{OUTPRINT}} which allows you to divert the output
to a text file.  Then you can use an editor like {\tt emacs} to
examine the file in detail before printing.  The Unix command {\tt
\tndx{wc} -l {\tt file}} will count the number of lines in a text file
called {\tt file} for you; note that {\tt -l} is the letter ell, not
the number one. Beginning with the {\tt 15JUL94} release, \AIPS\
provides a ``filter'' program to convert plain (or Fortran) text files
to PostScript for printing on PostScript printers.  The command
\dispx{{\tt \$\qs} \tndx{F2PS} -{\it nn} {\tt <} {\it file} {\tt |}
             lpr -P{\it ppp}}{ }
\dispe{will print text file {\it file} on PostScript printer {\it
ppp}.  The parameter {\it nn} is the number of lines per page used
inside \AIPS; it is likely to be 97 if direct printing comes out in
``portrait'' form or 61 if the direct print outs come out in
``landscape'' form.}

     It is not unusual for \AIPS\ jobs to be in the 1 Mbyte or more in
length, which will take 5--10 minutes to print.  For large text files,
it is quite likely that the {\tt ZLPCL2} shell script will divert the
job to a ``big''  printer (in Socorro, {\tt lp27} in room 213).
However, graphics files are not subject to such restrictions.

     If you plan on generating large or very complex plot files which
you intend to print, please select the {\tt psnet} printer at the AOC
or the {\tt ps3dup} printer in Charlottesville.  Since they are,
effectively, on the ethernet, the bandwidth to it is usually an order
of magnitude faster than any serial line.  You --- and others
--- will have to spend less time waiting for jobs to come out of the
printer. If you are submitting jobs which you know are several Mbyte
in size, we ask that you wait until after local business hours to
avoid tying up the printer.

\subsubsect{Z.1.4.4. {\tt CTRL Z} problems}

     The last process placed in the background via {\tt \tndx{CTRL Z}}
can be brought back to the foreground by typing {\us \tndx{fg} \CR} in
response to the monitor level {\tt \%} or {\tt \$} (or whatever) prompt
Alternatively, the user can type {\us\qs \tndx{jobs} \CR}, which
displays all background processes associated with the current login
and can bring a specific process to the foreground by typing {\us
fg\qs {\it \% m} \CR}, where {\it m} is the job number as displayed by
the {\tt jobs} command as {\tt [{\it m}]}.  For example, if a user
initiated his {\tt AIPS}{\it n} by typing {\us aips\qs new\qs pr=4\CR}
and:
\dispxx{{\tt \^{ }Z}}{{\tt CTRL Z} typed by accident (or
             intentionally).}
\dispxx{{\tt Stopped}}{{\tt aips new} is put in the background as
             ``stopped'' and user is returned to the Unix level.}
\dispx{{\tt \$\qs} jobs \CR}{to display status of background jobs.}
\dispxx{{\tt [1]\qs + Stopped\qquad aips new}}{info from Unix, where
              {\tt [1]} means job 1, ``Stopped'' is job 1's state and
              ``aips new'' is the command used to start up job 1.}
\dispx{{\tt \$\qs} fg {\it m} \CR}{to return job {\it m} to the
              foreground.}
\dispxx{{\tt aips new}}{appears on the screen just to tell the user
              to which job he is talking (\ie\ it does {\it not}
              re-execute {\tt aips new}).  You should now be talking
              to your {\tt AIPS}{\it n} again.}
\dispx{\CR}{to get {\tt AIPS}{\it n} {\tt >} prompt.}
\pd

\vfill\eject
\subsubsect{Z.1.4.5. ``File system is full'' message}

     The message {\tt write failed, file system is full} will appear
when the search for scratch space encounters a disk or disks without
enough space.  This is only a problem when none of the disks available
for scratch files has enough space, at which point the task will shut
down.

\subsubsect{Z.1.4.6.  Tapes won't mount}

   Occasionally, both local and remote \indx{tape mount}s may not work
successfully.  The source of the problem is often your failure to load
the tape physically into the device or to wait until the device is
ready to read the tape.  DATs and Exabytes, in particular, go through
lots of clicking and whirring before they are really ready.  An error
message like
\disps{{\tt AIPS 1: ZMOUN2: Couldn't open tape device /dev/nrst0}}
\dispe{(or some other tape-device name gibberish) is to be expected in
this case.}

     If you attempt to mount a remote tape and get the messages:
\disps{{\tt AIPS 1: ZMOUNR: UNABLE TO MOUNT REMOTE TAPE DEVICE, ERROR
              96}}
\disps{{\tt AIPS 1: AMOUNT: TAPE IS ALREADY MOUNTED BY \tndx{TPMON}}}
\dispe{it means that your {\tt AIPS} and the tape d\ae mon that you
are using disagree on whether the tape is already mounted in software.
The most probable reason for this is that you are attempting to mount
someone else's tape (check your inputs and the labels on the device
closely) --- or that the previous user of the device dismounted the
tape from the hardware but neglected to do it from software.  In this
case, you have two choices: (1) find the culprit and have him do a
software dismount, or (2) find an \AIPS\ Manager to kill the confused
d\ae mon and restart it.  (If you are using tape device {\it n} on
computer {\it host\_name}, then you need to stop the process called
{\tt TPMON}{\it m}, where $m = n + 1$ on computer {\it host\_name} and
then start it again by running {\tt /AIPS/START\_TPSERVERS} on that
computer.  This should be done by an \AIPS\ Manager.)}

     If you attempt to mount a remote tape and see, instead, the
messages:
\disps{{\tt ZVTPO2 connect (INET): Connection refused}}
\disps{{\tt AIPS 1: ZMOUNR: UNABLE TO OPEN SOCKET TO REMOTE MACHINE,
              ERROR  1}}
\disps{{\tt AIPS 1: ZMOUNT: ERROR     1 RETURNED BY ZMOUN2/ZMOUNR}}
\dispe{then the tape d\ae mons are not running on the remote machine.
Log into the remote machine as user {\tt aips} and type:}
\disps{{\tt /AIPS/\tndx{START\_TPSERVERS}}}
\dispe{After a minute or two, you should see some messages from
{\tt STARTPMON} about starting {\tt TPMON} d\ae mons.  Alternatively,
you could exit from {\tt AIPS} and start back up again, including {\tt
tp={\it host\_name}} on the {\tt aips} command line; see \Sec 2.2.3.
If the tape still doesn't mount after doing this, see the \AIPS\
Manager.}

\subsubsect{Z.1.4.7.  I can't use my data disk!}

     If at some point during your work you find you are prevented from
reading or writing files on a data disk, it could be that your \AIPS\
number does not have access to that area.  If you encounter the message:
\disps{{\tt AIPS 2: CATOPN: ACCESS DENIED TO DISK  8 FOR USER 1783}}
\dispe{it means that user 1783 has not been given access to write
(or read) on disk 8.  This can be seen by typing {\tt \tndx{FREESPAC}}
in the \hbox{{\tt AIPS}} session, listing the mounted disks. If you
see a data disk listed with an access of {\tt Not you}, it means your
\AIPS\ number has not been enabled for that disk. If you feel that you
should have access to that particular disk, see Jon Spargo (at the
AOC) or an \AIPS\ Manager about enabling your user number.}

\vfill\eject

\subsect{Z.2.  \AIPS\ at the NRAO in Charlottesville}

     The \Indx{Charlottesville} \AIPS\ Caige is located in Room 111 on
the first floor of the Edgemont Road Office Building.  There are three
public workstations available there: ``{\tt rhesus''} an IBM RS/6000
580, ``{\tt ringtail}'' an IBM RS/6000 560, and ``{\tt gibbon}'' a Sun
\hbox{IPX}.  For normal plots and print jobs there is an HP LaserJet
printer called {\tt ps1} just outside the Caige in the corridor.  This
printer is on a slow serial line and prints on only one side of the
paper.  Large graphics plots from {\tt LWPLA} should be sent to the
network HP printer in the library (called {\tt ps3}), while long print
jobs should be (and will automatically be) sent to this printer in
duplex mode (called {\tt ps3dup}).  There is a Tektronix 4511a color
printer known as {\tt pscolor} in the \AIPS\ Caige for special
plotting and color displays (\eg\ from {\tt TVCPS}) on either paper or
transparencies.  Color slides may also be made by directing the output
in PostScript form to {\tt 35mm2k}, a Solitaire film recorder across
the hall.

     The IBM workstations run under relatively current releases of
AIX, IBM's version of the Unix operating system, while the Suns run
under versions of SunOS, now called Solaris.  They are equipped with
color display screens, $1280 \times 1024$ on IBMs and $1152 \times
900$ on Suns.  All display in 8-bit pseudocolor, although {\tt
ringtail} has the ability to display 24-bit ``true'' color images for
applications that use it (not \AIPS' \hbox{{\tt XAS}}).  Each system
has substantial amounts of disk space.  At this writing, {\tt rhesus}
has the most with 8 \AIPS\ ``disks'' (12.2 Gbytes), {\tt ringtail} has
4 (7.8 Gbytes), and {\tt gibbon} has 3 (1.8 Gbytes).  All three
systems have at least one 4mm DAT and one 8mm Exabyte (low density)
tape drive, while {\tt rhesus} has two of each plus one 800/1600/6250
bpi 9-track tape drive.

\subsubsect{Z.2.1.  Using the Charlottesville workstations}

\subsubsectt{Z.2.1.1.  Signing up for \AIPS\ time in Charlottesville}

     The sign-up sheet for \AIPS\ on {\tt rhesus} is found on the
notice board just to the left of the entrance to the \AIPS\ Caige, Room
111.  If you wish to be certain of the availability of {\tt rhesus},
it is advisable to sign up for your \AIPS\ time in advance.  To
promote fair and efficient use of the system, users are asked to
restrict the amount of time that they reserve.  Formal rules are not
now in effect, but may be imposed should the need arise.

     \AIPS\ on the RS/6000 systems supports up to eight simultaneous
interactive users, plus two batch queues.  The user signed up for {\tt
AIPS1} has priority for the use of the cpu, console (display), and
local tape drives, but is expected to be reasonable about sharing
these resources, particularly the tape drives.  Visitors to
Charlottesville should call Jim Condon, in advance of their arrival,
to avoid conflicts with other visitors and to arrange for sign-up time.

     Sign-up time for {\tt ringtail} is no longer available, although
you may use it on a {\it caveat emptor\/} basis.  It is implicitly
reserved for anyone wishing to use the advanced visualization
facilities available through the {\it AVS\/} or {\it Data Explorer\/}
software systems.  If you are using \AIPS\ on it, and someone needs to
run either of these systems, you can be bumped off the system at that
time.

\subsubsect{Z.2.1.2.  Managing \indx{workstation} windows in
         Charlottesville}

     After you have logged on to any Sun or IBM as user-id {\tt aips},
the X-Window system should appear.  Unlike the AOC where there are
different window managers for IBMs and Suns, at Charlottesville the
{\tt aips} account always uses the {\tt \tndx{twm}} \indx{window
manager} (or possibly {\tt tvtwm} which is a superset of {\tt twm}).
\AIPS\ may also be run from your own account --- we prefer that you do
that --- but your account must be included in the {\tt aipsuser} group
and your startup procedures must start some X-Windows window manager,
preferably {\tt twm} or {\tt tvtwm}.

\eject

     The window manager allows you to create, destroy, modify, and
select an active window on the screen.  When you first start up (in
the {\tt aips} account, there will be at least two {\tt xterm}
terminal emulator windows on the screen, one green and one blue.  The
green one is for console messages and is usually quite small.  The
blue one is intended to be the main work area and it is in this window
that you probably will start up {\tt AIPS} itself.  There may be
additional windows like a clock and an {\tt xload} load meter.

     The default behavior of the \indx{window manager} is ``focus
follows pointer.''  What this means is that in order to use one of the
windows, you merely move the cursor with the mouse so that it is
within the main part of the window and start typing.  Most windows
will have a title bar on the top; you can ``grab'' this title bar by
moving the mouse cursor onto it and holding and keeping down the left
mouse button; then moving the mouse will also move the window.  There
is a small  square symbol on the left of the title bar that, if
pressed with the left mouse button, will iconify the window (you can
grab and drag icons too).  Clicking once with the left \indx{mouse}
button on any icon will de-iconify, or open it.\iodx{workstation}

The mouse buttons bring up a series of menus when they are pressed on
the background, or root window.  The left mouse button shows various
window operations such as ``redraw screen'', ``raise'', ``move'', and
of course ``Exit X11''.  The middle mouse button brings up a menu of
useful programs, including a desk calculator, the {\tt emacs} editor,
{\tt xv}, {\tt xgrab}, and more.  The right mouse button brings up the
System menu; the most useful item on this is a ``pull-right'' option
``X Terminal Emulator'' which, when you press it and ``pull right'' by
dragging the mouse a little to the right, shows another menu with
different foreground/background colors.  Choose one of these and you
will get an {\tt xterm} terminal emulator with those color
combinations.  There is also a pull-right menu for remote login to
various other useful machines in \Indx{Charlottesville} and other NRAO
sites.

     To start {\tt AIPS}, choose the {\tt xterm} window that you want
to work with (or create a new one), and type, \eg\ from {\tt rhesus}
\disps{{\tt aips tst pr=3 da=ringtail tp=gibbon}}
\dispe{This example command selects the TST version of {\tt AIPS} (the
default anyway), chooses printer number 3 (on the ground floor) as the
default printer for text and graphics output, makes the data areas
from {\tt ringtail} accessible (via NFS) in addition to local data
areas on {\tt rhesus}, and makes sure that the {\tt TPMON} daemons for
remote tape access are running on remote host {\tt gibbon}.  These
options are explained in some detail in \Sec 2.2.3 and may be viewed
in even greater detail  by typing {\us HELP AIPS \CR} inside {\tt
AIPS} or by typing {\us man aips \CR} from the Unix command line.}

\subsubsect{Z.2.1.3.  Data disk management in Charlottesville}

     A public 150-Mbyte disk partition on an IBM RS6000 580 called
polaris was set aside for use as the main disk where message and
{\tt SAVE}/{\tt GET} files are kept.  \AIPS\ looks for these files
only on \indx{disk} ``1,'' whichever data area that happens to be.  If
the same disk area is always disk 1, no matter what computer you are
using, then you would always get the same set of message and {\tt
SAVE}/{\tt GET} files.  The fly in this ointment is that the Network
File System has to wait while disk reads and writes are completed,
while reads and writes on local disks may be done asynchronously using
large memory buffers in the Unix operating system.  Thus, the writing
of messages, in particular, is virtually instantaneous on local disks,
but costs about one second per message over \hbox{\indx{NFS}}.

     The same consideration applies to disks used for image and $uv$
data files.  It is not too expensive to read such files over NFS, but
you should only write data to disks on the computer you are using.
You should also restrict all scratch files to be on local disks, using
the adverb {\tt \tndx{BADDISK}} to inhibit all NFS disks.  Note, that
the computer you are using does not have to be the one which you are
sitting in front of.  You may do an {\tt rlogin} or {\tt telnet} from
an {\tt xterm} window on, say, the Sun on your desk to, for example,
{\tt rhesus}.  Then, when you run {\tt AIPS} in that window, the local
disks will be the ones attached to {\tt rhesus}.  Under most
circumstances, the {\tt aips} procedure will be able to figure out
which Sun you are actually typing on and use it for the \AIPS\ TV,
message, and graphics servers (all both executing and displaying on
your Sun).

\subsubsect{Z.2.2.  Using the tape drives in \Indx{Charlottesville}}

    For a general discussion of magnetic tapes, including the {\it
required\/} software mount, see \Sec 2.4.  The following describes how
to deal with the physical \indx{tape} drives themselves.
\iodx{magnetic tape}

\subsubsect{Z.2.2.1.  Mounting and removing tapes on 9-track drives}

    If the front door of the tape unit is not open, it may be in use.
Ask around before doing anything.  When you're certain that it's okay,
press the ``offline'' button, then press the ``Unload/Open'' button.
The door should drop open for you.

    Once the door is open, get ready to put your \indx{9-track tape}
in.  Put a write ring in the tape only if you intend to write on the
tape during this \AIPS\ session.  These drives do not support
auto-load rims, so all rims must be removed from the tape before
loading.  Slide the tape spool in sideways with the label facing up
until it sits on the central hub (it will feel like it's balanced).
Then close the door.  After the display indicates that the drive is
ready, you can perform the software mount from \hbox{{\tt AIPS}}.

\subsubsect{Z.2.2.2.  Mounting tapes on Exabyte and DAT drives}

     \indx{Exabyte (8mm)} and \indx{DAT (4mm)} drives have a window or
opening through which a mounted tape may be seen.  Before touching
anything, look in the window or opening to see if there is already a
tape in the drive.  If there is, ask around to make sure that the tape
is no longer in use.  Remember that the user of the drive may be in an
office as much as two floors away and that Unix does not provide much
protection.  If you dismount a remote user's tape and mount your own,
that user may well write on it, thinking that he is writing on his own
tape, without knowing that he is destroying all your data.

     On most drives, there will be a single button on the front panel
of the device somewhere.  When the device becomes available, press
this button to open the door.  If there was already a tape in the
drive, it will be ejected after some whirring and clanking and a few
seconds.  If a tape is ejected, remove it.  Now put your tape in the
drive, label facing upwards.  On Exabytes, push the door closed
gently.  For DAT drives, lightly push the tape into the drive until
the device ``grabs'' the tape and pulls it in the rest of the way.
Exabyte and DAT tapes have a small slide in the edge of the tape which
faces out which takes the place of the write ring of 9-track tapes.
For 8mm (Exabyte) tapes push the slide to the right (color black
shows) for writing and to the left (red or white shows) for reading.
With 4mm DAT tapes, the slide also goes to the right for writing (but
white or red shows) and to the left for reading (black shows).

     It is necessary to wait until the mechanism in the drive has
``settled down'', \ie\ when the noises and flashing lights have stopped,
before you can access the drive.  The first access is, of course, the
software {\tt MOUNT} command from inside \hbox{{\tt AIPS}}.

\vfill\eject
\subsubsect{Z.2.3.  Color hard copy in Charlottesville}

     Having created a PostScript file containing color commands or
pictures (see \Secs Z.1.2.), you can print it either on a Tektronix
4511a \indx{color printer} known as {\tt pscolor} in the \AIPS\ Caige (with
plain or transparency ``paper'') or send it to the Solitaire film
recorder.  Special Tektronix ``paper'' is loaded into cassettes, one
of which is inserted into the printer.  One cassette is used for plain
paper and the other for transparency paper.  The latter is rather
expensive, so be reasonable in its use.  Additional boxes of paper may
be found inside the cabinet on which the printer rests.  When the
little green light on the right hand side of the printer is blinking,
the printer is either receiving data or computing on data already
received.  The computer inside the printer is actually rather fast, but
a color picture is usually a large data file which takes a while to
transmit and display.  When the printer starts actually printing, it
must move the paper in and out three times, onece for each color.
When it is done, it will eject the paper fully.

     The Solitaire \indx{film recorder} is in the room across the hall
from the \AIPS\ Caige.  To see details on its use, consult the
appropriate chapter of the Charlottesville ``Sun Workstation Notes.''
In those notes, it tells you how to change film and to swap film
holders.  We suggest that you do {\it not} do these operations.
Instead, have someone from the Graphics Department (George Kessler or
\indx{Pat Smiley}) preferably or a computer technician (Warren
Richardson or Gene Runion) to assist you.  Cleaning and repairs to the
device should be done only by Gene and Warren.  We have found that
pictures made by the device are very sensitive to dust and particles
of paint (from the device itself) and, hence, think it best if the
device is handled only by trained personnel.

     The Solitaire has an LCD display which usually shows the status
of the film recorder.  On the left is the film counter, telling how
many frames have been exposed since a film load or a counter reset.
Towards the middle is the machine's status, which usually displays
Idle, Calibrating, Computing Geometry, or Exposing.  When the machine
is idle, output can be dumped to it (assuming there is no PostScript
in the process of being interpreted for the film recorder already).
Image recorder calibration is done every half hour, as well as when
any job is sent to the recorder.  Computing Geometry is displayed
whenever a new film holder has been placed onto the Image Recorder.
On the upper right, G2 is normally displayed, which means that it's
listening to GPIB address 2 (GPIB is similar to SCSI).  Under G2, the
number of frames left on the roll is shown.  The Solitaire has two
status lights on the console; a green light for exposing (the light
will vary in intensity with the actual light beam), and a red light to
signify that it's not ready.

     You should check the status display, then send your PostScript
file to the film recorder, and then check the display again, all to
make sure that the desired operation has taken place.  The queues are
named {\tt 35mm2k}, {\tt 35mm4k}, {\tt 35mm8k}, {\tt 4x5-2k}, {\tt
4x5-4k}, and {\tt 4x5-8k}, all with the obvious meaning.  Have the
film holder changed to the desired one before submitting a job to its
queue.  After you have exposed the film, ask someone from the Graphics
Department to remove the film and have it processed.  Rolls of 35mm
slides taken to be processed before 10 a.m.~can be back as early as 2
p.m.  All other processing takes until the next business day.  If you
are not in a hurry, you should wait for more pictures on a roll to be
exposed.  However, the cost of processing a whole roll for one
picture, on an occasional basis, is not significant.

     Even though you are in Charlottesville, you should consider
\indx{Pat Smiley}'s offer of help with your photographs.  See the
comments and instructions given in \Sec Z.1.2.3.

\vfill\eject

\subsect{Z.3.  \AIPS\ at the NRAO AOC in Socorro}

   The computing power at the NRAO Array Operations Center
(``\Indx{AOC}'') in Socorro is no longer concentrated solely in the
\AIPS\ ``Caiges'' --- there are now several public Sun IPX
workstations scattered throughout the building, most with at least one
Exabyte and one DAT tape drive and 2 Gbyte or more of disk space.
There are also eight public IBM RS/6000 machines, ranging from a 320
(Zuni) to several 580s.  Each IBM is equipped with at least one
Exabyte tape drive, and many have a DAT drive as well.  The disk space
on the IBMs ranges from 4 to 16 Gbytes.  Several printers are
accessible, most of which are located in the computer room, ranging
from a DEC LP27A line printer to a Tektronix color printer ideal for
color prints or overheads.  Color slides may also be made by sending
your picture files to Charlottesville; see \Sec Z.1.2.3.

\subsubsect{Z.3.1.  Using the AOC workstations}

\subsubsectt{Z.3.1.1.  Signing up for \AIPS\ time in Socorro}

   Jon Spargo (505-835-7305) is responsible for the assignment of time
on all public AOC machines; reservations must be made at least two
weeks in advance.  Last-minute sign-ups for vacant time and any
exchanges of assigned time must also be cleared by Jon.  Normally,
sign-ups will be limited to a maximum of two weeks, although
exceptions are made occasionally.  Visitors have exclusive rights to
their assigned workstations during their visit, although tape drives
(when not in use) are fair game for remote users.  Any data left
on disk after a user's scheduled time will be deleted promptly.  See
``Dr.~Delete's Public Workstation Allocation Board'' next to room 258
for the machine assignments.

\subsubsect{Z.3.1.2.  Managing \indx{workstation} windows at the AOC}

    The Suns and IBMs have different window managers at the
\hbox{AOC}.

     To start {\tt AIPS} on either architecture, choose the window
that you want to work with (or create a new one), and type:
\disps{{\tt aips tst pr=7}}
\dispe{This example command selects the {\tt TST} version of {\tt
AIPS} (the default) and chooses printer number 7 (the {\tt psnet}
printer in the computer room) as the default printer for text and
graphics output.  Other options (such as cross mounting data disks
from other machines and checking the status of remote tape daemons)
can be chosen on the command line as well. These options are explained
in some detail in \Sec 2.2.3 and may be viewed in even greater detail
by typing {\us HELP AIPS \CR} inside {\tt AIPS} or by typing {\us man
aips \CR} from the Unix command line.}

\subsubsect{Z.3.1.2.1.  IBM workstation windows at the AOC}

     The IBMs use the \indx{Motif} \indx{window manager} ({\tt mwm}).
Before you begin on your assigned IBM, you should have a black screen
with a login prompt.  If a window system is already running, that
means someone else has not completely logged out of the machine.
Check the windows to see if any processes are running; if not, press
{\tt ALT}, {\tt DELETE}, and {\tt BACKSPACE} keys (all at the same
time) to exit the window system.  Then log in as user-id {\tt aips},
and the window system will start afresh.  If you wish to use your own
user account, you must make sure you are in the {\tt aipsuser} group
(type {\us id \CR} and check your group listings).

     You will be greeted by several windows: a cordial lawyer-ese
greeting from IBM (feel free to kill the window), a window-manager
screen, and a blue window which might exhibit some strange
characteristics at first. If you type {\us stty sane \CR}, the window
will behave in a much more civil manner.

     The default behavior of the window manager is ``focus follows
\indx{mouse} pointer.''  What this means is that in order to use one
of the windows, you merely move the cursor with the mouse so that it
is within the main part of the window and start typing.  Click once on
a border and that window will pop to the front.  Most windows will
have a title bar on the top.  You can ``grab'' this title bar by
moving the mouse cursor onto it and holding and keeping down the left
mouse button; then moving the mouse will also move the window.  You
will see the outline of the window as it is being moved.  A window can
be resized by moving the cursor to a window corner and holding the
left mouse button and ``dragging'' the corner.  A window can be
resized in width or height only by grabbing a side of the window.  A
window can be closed (iconified) by clicking once on the ``dot'' on
the right side of the title bar.  A window icon can be opened by
moving the cursor to the icon and clicking twice rapidly with the left
mouse button.  Clicking once on the square to the extreme right of the
title bar will enlarge the window to fill the workstation screen;
clicking again will reduce it to the previous size.  Clicking once
(and holding) on the ``dash'' to the extreme left of the title bar
will give a menu to move, resize, open and close the window. {\it
Note, depending on the nature of the window, if you click on this dash
{\rm twice} in rapid succession, the window will disappear and all
processes in it will die.}  Don't do this unless you really mean it.

     There are some shortcuts to these window functions as well;
pressing {\tt ALT} and {\tt F3} at the same time will push the
highlighted window (the one with the cursor in it) behind all the
other windows.  Pressing {\tt ALT} and {\tt F4} at the same time will
close (\ie\ kill) a window.  {\tt ALT} and {\tt F9} will minimize
(iconify) the current window.

     The left mouse button brings up a series of menus when it is
pressed on the background, or root, window.  The first item will open
an {\tt xterm} window well-suited for the running of \hbox{\AIPS}.
The second item starts an \AIPS\ session; this window will ignore
{\tt CTRL-Z} commands and will disappear when {\tt AIPS} is exited, so
it might not be well suited to all applications.  The next item will
open an {\tt emacs} window for editing text files.  There is also an
option to open a window suitable for running \hbox{{\tt OBSERVE}}.
Finally, one of the options will allow you to exit the windowing
system.

\subsubsect{Z.3.1.2.2.  SUN \indx{workstation} windows at the AOC}

     The \Indx{AOC} Suns use the {\tt \indx{OpenWindows}} system.  At
the login prompt, type {\tt aips} and enter the password.  You will
then be asked if you wish to start {\tt OpenWindows} --- answer yes,
and the window manager will begin.  You will see several windows open
up --- a virtual desktop, a clock, a performance meter, a console
window, and an {\tt xterm}.  The {\tt xterm} is well suited for
running \AIPS, but does not allow scrolling back a large distance;
opening a {\tt command tool} enables a larger scroll range.

     A whole host of {\tt OpenWindow} options can be chosen by
clicking once with the right mouse button on the background (\ie\ not
in a window).  If you go to {\tt Programs} and drag the cursor to the
right, you will see a second menu with various types of windows
(xterm, command tool, shell tool), tools (mail tool, print tool,
calculator, etc.) and managers (file manager, calendar manager, etc.).
The next option of the main menu, {\tt Utilities} allows you to
refresh the screen or lock the terminal.  Note that if you change your
window setup, you should {\it not} {\tt save the workspace}, since
this will affect all other users of the {\tt aips} account.  Finally,
you can exit {\tt OpenWindows} by dragging down to {\tt Exit}; note
that it will not, however, log you out of the \AIPS\ account.

     Windows can be moved or resized in much the same manner as the
IBM windows.  Clicking (and holding) once on the title bar with the
left mouse button allows you to drag the window to another location.
Clicking once (and holding) at a window corner with the left mouse
button allows resizing of a window.  Each of these jobs can also be
done from a menu for each window; click once on the title bar with the
right mouse button to see the options.

     An icon can be opened into a window by clicking on it twice in
rapid succession with the left mouse button. A window can be closed by
clicking once on the box to the left side of the title bar.

\subsubsect{Z.3.1.3.  Data disk management at the \Indx{AOC}}

     A public 2-Gbyte \indx{disk} partition on the Auspex server ({\tt
Arana}) was set aside for use as the area where message and
{\tt SAVE}/{\tt GET} files are kept.  \AIPS\ looks for these files
only on disk ``1,'' whichever data area that happens to be.  If the
same disk area is always disk 1, no matter what computer you are
using, then you will always get the same set of message and {\tt
SAVE}/{\tt GET} files.  The public Sun workstations are set up so that
the {\tt Arana} data disk is always mounted as the first disk.  The
drawback is that the Network File System \hbox{(\indx{NFS})} has to
wait while disk reads and writes are completed, while reads and writes
on local disks may be done asynchronously using large memory buffers
in the Unix operating system.  Thus, the writing of messages, in
particular, is virtually instantaneous on local disks, but is time
consuming over \hbox{NFS}.  It is for this reason that the {\tt Arana}
data disk is no longer automatically mounted on the IBM workstations;
if you wish to mount the {\tt Arana} disk, you must do so with the
{\tt da=arana} option at \AIPS\ startup.  If you don't want to mount
the {\tt Arana} data disk, you can create your own disk access file
which excludes all external disks.  Talk to your assigned \AIPS\
friend or the \AIPS\ Manager for more details.

     The same consideration applies to disks used for image and $uv$
data files.  It is not too expensive to read such files over
\hbox{NFS}, but you should {\it only} write data to disks on the
computer you are using.  You should also restrict all scratch files to
be on local disks, using the adverb {\tt BADDISK} to inhibit all
NFS-mounted disks.  It also helps to ``spread around'' your data on
multi-disk machines; if you are reading files from one disk, write the
output file to a different disk.

\subsubsect{Z.3.2.  Using the tape drives at the AOC}

    For a general discussion of magnetic tapes, including the {\it
required\/} software mount, see \Sec 2.4.  The following describes how
to deal with the individual \indx{tape} drives at the \hbox{AOC}.
\iodx{magnetic tape}

\subsubsect{Z.3.2.1.  Mounting and removing tapes on 9-track drives}

 \iodx{9-track tape}
   The Sun workstation {\tt Sol} in the second- (ground-) floor
computer room has a nine-track HP drive. If there is no tape in the
drive, the readout panel will read \hbox{{\tt READY}}.  If you want to
make sure there is no tape in the drive before pushing any buttons,
the top panel can be lifted (push the button on the right side of the
box) and the reel inspected.  Pushing the {\tt REW/UNLD} button will
open the drive.  Put a write ring in the tape only if you intend to
write on the tape; if you only want to read, make sure there is no
write ring.  Take the outer plastic rim off the tape before loading.
Slide the tape in sideways with the label facing up until it sits on
the central hub (it will feel like it's balanced), then close the
door.  The display panel will read {\tt LOADING}, with the {\tt
ONLINE} light flashing.  When the tape is ready to be accessed, the
panel will read {\tt BOT}, with the {\tt ONLINE} indicator on (not
flashing).  After the display indicates that the drive is ready, you
can perform the software mount from \hbox{{\tt AIPS}}.  As you access
the tape, the panel will read {\tt IDLE} in between operations.  When
you are finished with the tape, dismounting the tape in the {\tt AIPS}
window will free the tape lock and unload the tape by opening the
drive door.  Please close the door after you've removed the tape to
keep out dust and dirt.

     Two more nine-track drives are located on the Sun workstation
{\tt Iolani}, also located in the second-floor computer room.  This
machine is primarily for use of the tape archivists, and the drives
should only be used if urgency reigns and {\tt Sol} is in use.
Contact the data archivists (currently Theresa McBride and Gayle
Rhodes) before using these drives.

\subsubsect{Z.3.2.2.  Mounting tapes on Exabyte drives at the AOC}

     There are several different types of \indx{Exabyte (8mm)} drives
at the  \hbox{\Indx{AOC}}.  All of the public Sun workstations have
dual-density Exabyte drives, either a TTi 8501 or a COCOMP drive.  The
TTi 8501 drive has a window through which the tape (or lack thereof)
can be seen; the COCOMP drives have a screen displaying the status of
the tape.  Before opening the drive to insert your tape, make sure
there is not a tape present (on the COCOMP drives, the {\tt current
tape operation} reads {\tt NotRdy}).  If there is a tape present, ask
around to make sure that the tape is no longer in use.  Remember that
the user of the drive may be in an office as much as two floors away
and that Unix does not provide much protection.  If you dismount a
remote user's tape and mount your own, that user may well write on it,
thinking that he is writing on his own tape, without knowing that he
is destroying all your data.

     To open the drive, push the button on the lower left of the tape
door.  If there was already a tape in the drive, it will be ejected
after some whirring and clicking and a few seconds.  If a tape is
ejected, remove it.  Exabyte tapes have a small slide in the edge of
the tape which faces out which takes the place of the write ring of
9-track tapes.  For 8mm (Exabyte) tapes push the slide to the right
(color black shows) for writing and to the left (red or white shows)
for reading.  Now put your tape in the drive, label facing upwards;
gently push the door closed.

     It is necessary to wait until the mechanism in the drive has
``settled down'' before you can access the drive.  On both the TTi
8501 and COCOMP drives the flashing green light will stay lit (and the
whirring and clicking will stop) when the tape is ready to be
accessed.  The COCOMP {\tt current tape operation} will read
\hbox{{\tt IDLE}}.  Each drive will also show some indication of the
tape remaining (or capacity); the number should be on the order of
2200 for low-density tapes and 4500 for high-density.  If it is a new
tape, the display will be at 4500; if you mount at low density, the
display will not change to the appropriate value until you start
writing.  Once the green light on the drive has stopped flashing, the
tape is ready to be accessed.  The first step is the software {\tt
MOUNT} command from inside \hbox{{\tt AIPS}}.

     The IBM workstations also have varying types of Exabyte drives.
Most have an internal Exabyte drive (in the cpu box) which work at
low density only; they have windows allowing you to see if a tape is
currently mounted.  Kiowa's internal drive is dual density, but has no
display (the green light will be lit or flashing quickly if a tape is
being accessed).  Several of the IBMs also have external Exabytes.
The TTi 8501 on Hopi works as those on the Suns.  Several IBMs have
TTi 8505 dual-density drives (about half the height of the TTi 8501
drives); these drives don't have viewing windows (although you can
push the door in {\it gently} to look for a tape).  If there is a tape
in the drive, the green light will be lit (or flashing quickly) and
the {\tt tape remaining} indicator will be non-zero.  If there is no
tape in the drive, the green light will either not be lit or blink
very slowly.  {\it Note:\/} please use only ``data-grade'' tapes on
the IBMs.

\subsubsect{Z.3.2.3.  Mounting tapes on DAT drives at the AOC}

\iodx{DAT (4mm)}
   The Falcon DAT tape (4mm) drives (manufactured by HP) at the AOC
have a window which has a cover.  When the cover is down and the two
green lights under the door are not lit, there is no DAT tape in the
drive.  When a tape is in the drive, it is visible from the outside
(\ie\ the door does not come down and cover the tape) and the two
lights will be on (continually lit if the tape is not being accessed;
flashing if the tape is in use).  If the drive is not in use, insert
the tape, label up, and push it gently into the opening; the drive
will eventually ``grab'' it and bring it in the rest of the way.  With
4mm DAT tapes, the write-protection works differently from Exabytes;
the slide goes to the right for writing (but white or red shows) and
to the left for reading (black shows).  When the green lights stop
flashing, the tape is ready to be accessed.

\subsubsect{Z.3.3.  Color hard copy at the \Indx{AOC}}

     Having created a PostScript file containing color commands or
pictures (see \Secs Z.1.2.), you can print it on a Tektronix Phaser II
PXi \indx{color printer} known endearingly as {\tt Farbdrucker} in the
second- (ground-) floor computer room.  Two queues have been set up
for this printer; one for color paper ({\tt pscolor}) and one for
color overhead transparencies ({\tt psoverhd}).  Once the file has
been routed to the appropriate queue using the {\tt psprint} or {\tt
lpr} command, go to the terminal to the right of the color printer
(labelled {\tt Farbdrucker only!!!}).  To log in, press the {\tt F2}
key for the {\tt pscolor} queue, {\tt F3} for the {\tt psoverhd}
queue.  You will then be shown the print queue, asked if you want to
route the files to the printer, and asked to check that the proper
cartridge is in place (paper or transparencies).  When the printer
starts actually printing, it must move the paper in and out three
times, one for each color.  When it is done, it will eject the paper
fully.  Please note that color paper and transparencies are not cheap,
so be reasonable in their use.  To obtain additional paper or
transparencies for empty cartridges, see Theresa McBride.

     The AOC no longer has a working film recorder.  Send PostScript
picture files to \indx{Pat Smiley} at the \indx{NRAO} in
Charlottesville for processing and mailing; see \Sec Z.1.2.3.  Fast
turn-around and special processing, if needed, are both available
through Pat.  Use {\tt TVCPS} or {\tt TVRGB} to produce color image
picture files or {\tt LWPLA} to translate plot files into
black-and-white picture files.

%\vfill\eject

\subsect{Z.4.  \AIPS\ at the NRAO Very Large Array site}

     \AIPS\ is installed on a Sun IPX workstation  called {\tt
Miranda} in the control room at the \Indx{VLA} site.  While there is
not enough disk space to allow large-scale data reduction, there is
adequate space to allow running on-line {\tt FILLM} to inspect your
data as it is acquired.  {\tt Miranda} is equipped with a 800-Mbyte
data disk and a low-density Exabyte (8mm) tape drive.  There is also a
nine-track tape drive on the operator's Sun ({\tt Banshee}) which can
be used remotely, but check with the operator before using it.  The
\AIPS\ QMS printer is located in the computer room next to the control
room.

\subsubsect{Z.4.1.  Using the VLA workstation}

\subsubsectt{Z.4.1.1.  Signing up for \AIPS\ time at the VLA}

     There is no official sign-up for visitor use of {\tt Miranda}; if
the workstation is not in use, feel free to use it.  However, if there
is more than one observing team at the site at any one time, priority
goes to the current observer.

     If you are planning on using on-line {\tt FILLM}, you must
contact George Martin (505--835--7287) prior to your arrival.  It
would also be wise to contact George or Gustaaf van Moorsel
(505--835--7396) to get some documentation on on-line {\tt FILLM}
before heading out to the site.

\vfill\eject
\subsubsect{Z.4.1.2.  Managing \indx{workstation} windows at the VLA}

    {\tt Miranda} uses the {\tt \indx{OpenWindows}} system and will
behave in a manner very similar to the AOC Suns.  If the window system
is not running, type {\tt aips} at the login prompt, then enter the
password. You will be asked if you wish to start {\tt OpenWindows} ---
answer yes, and the window manager will begin.  You will see several
windows open up, currently some command tools, a clock, a performance
meter, a console window, and a file manager.  Further information on
dealing with {\tt OpenWindows} options can be found in \Sec Z.3.1.2.2.

\subsubsect{Z.4.1.3.  Data disk management at the VLA}

     Unlike the AOC, there are no ``optional'' disks to be mounted on
{\tt Miranda}.  The two data areas which appear when you start {\tt
aips} are actually just two partitions on the same disk.

     If, when you start {\tt aips}, you find little or no space on the
disks, contact the \AIPS\ Manager to clear off some space.  Due to the
limited size of the data \indx{disk}, there is an automatic time
destroy running on {\tt Miranda} which will delete data which has not
been used in three days.  Sometimes even this interval is too long to
keep a usable amount of space on the disk and the \AIPS\ Manager is
forced to take draconian measures.  {\it Caveat emptor:\/} if you want
a copy of the data on the {\tt Miranda} disk, copy it onto a tape
immediately!

\subsubsect{Z.4.2.  Using the tape drives at the VLA}

    For a general discussion of magnetic tapes, including the {\it
required\/} software mount, see \Sec 2.4.  The following describes how
to deal with the Exabyte drive at the VLA site.

   The Exabyte drive at the VLA is a COCOMP drive.  Unlike the COCOMP
drives at the AOC, there is no display panel covering the window.
Before opening the drive to insert your tape, make sure there is not
already a tape in the drive.  If there is a tape present, check with
any other observers and the telescope operator to make sure that the
tape is no longer in use.

     To open the drive, push the button on the lower left of the tape
door.  If there was already a tape in the drive, it will be ejected
after some whirring and clicking and a few seconds.  If a tape is
ejected, remove it.  Exabyte tapes have a small slide in the edge of
the tape which faces out which takes the place of the write ring of
9-track tapes.  For 8mm (Exabyte) tapes push the slide to the right
(color black shows) for writing and to the left (red or white shows)
for reading.  Now put your \indx{tape} in the drive, label facing
upwards; push the door closed gently.\iodx{Exabyte (8mm)}
\iodx{magnetic tape}

     It is necessary to wait until the mechanism in the drive has
``settled down'' before you can access the drive.  On the COCOMP drive
the flashing green light will stay lit (and the whirring and clicking
will stop) when the tape is ready to be accessed.  Once the green
light on the drive has stopped flashing, the tape is ready for the
software {\tt MOUNT} command from inside \hbox{{\tt AIPS}}.

    On-line {\tt FILLM} requires several ``pseudo-tape'' devices in
the \AIPS\ software, so the first ``real'' tape device on {\tt
Miranda} is tape number 6.  Therefore, to access the local tape on
{\tt Miranda}, you need to use {\tt INTAPE} (and {\tt OUTTAPE} when
needed) {\tt = 6}.  If you need to access one of the tapes on {\tt
Banshee} remotely, make sure to use one of the remote tape device
numbers, either {\tt INTAPE} 7 or 8.

\vfill\eject
\subsect{Z.5.  Additional \indx{recipe}s}

\subsubsect{Z.5.1.  Delightful banana cheesecake}

\hanstart{50pt}
\enumerate{1}{Preheat oven to $350\deg$.}
\enumrnext{Combine 1.5 cups crushed {\tt cereal} (3 cups un-crushed
   Multi-Bran Chex suggested), 1/3 cup melted {\tt margarine} or
   butter, and 1/4 cup packed {\tt brown sugar}; mix well.}
\enumrnext{Press firmly onto bottom and sides of greased 9-inch pie
   plate.  Bake 8--10 minutes, then cool completely.}
\enumrnext{Arrange 1.5 cups sliced {\tt \indx{bananas}} onto sides and
   bottom of cooled crust.}
\enumrnext{Combine 16 oz.~softened light or regular {\tt cream
   cheese}, 1.5 cups {\tt powdered sugar}, and 3/4 teaspoon {\tt
   vanilla extract}.}
\enumrnext{Mix well, then fold in 2 cups light or regular {\tt whipped
   topping}.  Pour over sliced bananas.}
\enumrnext{Cover and refrigerate for 4 hours or until set.}
\enumrnext{Garnish with 1/2 cup sliced {\tt bananas}.}
\hangnxtpar{\hfill Thanks to Ralston Purina Company.}
\par\hanend

\subsubsect{Z.5.2.  Banana poundcake}

\hanstart{50pt}
\enumerate{1}{Mix in large bowl until blended:}
\par\hanend
\vskip 4pt
\hbox to \hsize{\hskip 100pt\baselineskip 14pt\vbox{\halign{\rt{{$#$}}\quad%
&\lft{#}\cr
1 {1\over3}&cups mashed {\tt \indx{bananas}} (4 medium)\cr
1          &pkg.~($18{1\over2}$ oz.) {\tt yellow cake mix}\cr
1          &pkg.~($3{3\over4}$ oz.) instant {\tt vanilla pudding mix}\cr
1\over3    &cup {\tt salad oil}\cr
1\over2    &cup {\tt water}\cr
1\over2    &teaspoon {\tt cinnamon}\cr
1\over2    &teaspoon {\tt nutmeg}\cr
4          &{\tt eggs} at room temperature\cr}}\hfil}
\hanstart{50pt}
\enumerate{2}{Beat at medium speed for 4 minutes.}
\enumrnext{Turn batter into greased and lightly floured 10-inch tube pan.}
\enumrnext{Bake in $350\deg$ oven for 1 hour or until cake tester inserted
in cake comes out clean.}
\enumrnext{Cool in pan 10 minutes, then turn out onto rack and cool
completely.}
\enumrnext{If desired, dust with confectioners sugar before serving.}
\hangnxtpar{Thanks to the United Fresh Fruit and Vegetable Association.}
\par\hanend

\vfill\eject
\end
