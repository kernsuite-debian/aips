%-----------------------------------------------------------------------
%;  Copyright (C) 1995
%;  Associated Universities, Inc. Washington DC, USA.
%;
%;  This program is free software; you can redistribute it and/or
%;  modify it under the terms of the GNU General Public License as
%;  published by the Free Software Foundation; either version 2 of
%;  the License, or (at your option) any later version.
%;
%;  This program is distributed in the hope that it will be useful,
%;  but WITHOUT ANY WARRANTY; without even the implied warranty of
%;  MERCHANTABILITY or FITNESS FOR A PARTICULAR PURPOSE.  See the
%;  GNU General Public License for more details.
%;
%;  You should have received a copy of the GNU General Public
%;  License along with this program; if not, write to the Free
%;  Software Foundation, Inc., 675 Massachusetts Ave, Cambridge,
%;  MA 02139, USA.
%;
%;  Correspondence concerning AIPS should be addressed as follows:
%;          Internet email: aipsmail@nrao.edu.
%;          Postal address: AIPS Project Office
%;                          National Radio Astronomy Observatory
%;                          520 Edgemont Road
%;                          Charlottesville, VA 22903-2475 USA
%-----------------------------------------------------------------------
%Body of Aips Letter for 15 January 1991
%last edited by Glen Langston on 1991 July 25
%last edited by Gareth Hunt   on 1991 July 19

\documentstyle{article}

\newcommand{\AIPRELEASE}{January 15, 1991}
\newcommand{\AIPVOLUME}{Volume XI}
\newcommand{\AIPNUMBER}{Number 1}
\newcommand{\JANNO}{15JAN91}
\newcommand{\APRNO}{15APR91}
\newcommand{\JULNO}{15JUL91}
\newcommand{\OCTNO}{15OCT91}
\newcommand{\RELEASENAME}{\JANNO}
%macros and title page format for the AIPS letter.
\input AipsLetMac.tex


\normalstyle
\section{Recent Developments}
The structure and focus of the \AIPS\  programming group was redefined in
January 1991.  In summary, the \AIPS\  group, under the leadership of Geoff
Croes, has decided to re-write \AIPS\  in order to incorporate software
capabilities needed for the coming decade.

The \AIPS\ re-write is discussed in the \APRNO\ {\tt \AIPSLETTER}.
The new \AIPS\  software will be written in the language C++, and here after,
the new \AIPS\  will be called \AIPTOO.

\section{Personnel}
There has been one recent personnel change: Patrick Murphy, formerly of NRAO
Tucson (and before that, the VLA),
has joined the \AIPS\ group and moved to NRAO, Charlottesville.  Pat is
in charge of helping \AIPS\  system managers install new releases and is also
updating the \AIPS\  interface to the various flavors of operating systems.
Welcome Pat!  (Pat can be contacted at (804) 296-0372 or e-mail address
{\tt pmurphy@orangutan.cv.nrao.edu or pmurphy@192.33.115.11})

\section{New VLA data and Old Releases of \AIPS}

Scientists using the VLA often calibrate their \UV-data at NRAO using the
latest release of \AIPS\, then take the data back to there home institution for
further processing.  This can cause problems if the the home institution has
not recently installed the latest release of \AIPS.  The problem occurs because
the VLA has started allowing many new observing modes (with many simultaneous
observing frequencies) which are not supported in old releases of \AIPS.

In particular, the newest releases of \AIPS\ support
{\it multi-IF} VLA observations in a different manner than
previous \AIPS\ releases.
({\it Multi-IF} observations are observations in which two separate band
passes observed simultaneously \eg\ two 5O MHz observations at
4885 and 4835 MHz).
Old releases of \AIPS\ were geared toward \UV-data calibrated with the
DEC 10 at the VLA site.
The continuum \UV-data calibrated by the DEC 10 could contain only a single
frequency.  (Called AC or BD \UV-data in DEC 10 speak).
New \AIPS\ tasks place the \UV-data for these {\it multi-IF}
observations in one file to facilitate some of the new data processing
algorithms.
Old versions of \AIPS\ either ignore the second {\it IF} or incorrectly
handle processing the second frequency.
(\AIPS\ tasks \UVMAP\ and \ASCAL\ {\bf ignore} the
second {\it IF}.
In \JANNO, tasks \MX\ and \CALIB\
handle {\it multi-IF} data correctly, and
replace \UVMAP and \ASCAL.)

If observers are taking data back to sites running old versions of \AIPS\
they should use the task \UVCOP to copy each {\it IF} into a
separate \UV-data set.
The old \AIPS\ tasks are believed to handle the single frequency
data correctly.
Note that spectral line data from a single {\it IF} is believed
to be processed correctly in older versions of \AIPS.

\section{Getting \AIPS\ over the InterNet}

Over the past few months, we have been investigating the feasibility
of using InterNet for distribution of the Unix \AIPS\ source.  Several
sites have obtained the \JANNO\ release of \AIPS\ in this
manner, and the experience so far has been very positive.  The
procedure is quite simple: the requester uses the standard {\tt ftp}
program to copy a compressed Unix {\tt tar} file from NRAO to their
computer.  The requester can also retrieve copies of the Installation
guide and reference manual or can request that they be sent via
electronic mail.  The tar file is about 36 Mega-bytes in size, and
compresses down to 10 Mega-bytes.  The requester can choose between a
single monolithic file and the split version (52 files).

As \AIPS\ code is proprietary, we are unable to simply make it
available over anonymous ftp.  The procedure currently in use is to
use a password-based account; the password is changed for each
requester.

We are now adding ``{\tt ftp}'' as an option for distribution medium on
the \AIPS\ order form at the end of this {\tt \AIPSLETTER}.  Please note that
this applies {\it only to the Unix version\/} of \AIPS.  If you wish
to make use of this option, you should either use the form or send an
e-mail equivalent to {\tt aipsmail@nrao.edu}.  Your site does need a
current \AIPS\ license in order for us to process {\tt ftp} orders.
Contact Pat Murphy for more information concerning \AIPS\ via InterNet.

\normalstyle
\section{Changes in \RELEASENAME}
There were 403  ``significant'' changes applied to \JANNO.
A selection of these changes are summarized below.
\subsection{Changes of interest to Users:}
\begin{description}
\myitem{FQ ids} Bugs were removed from several tasks which handled
several {\it IF}s simultaneously.
\myitem{\CALIB} \CALIB was split into two tasks,
\CALIB to do amplitude and phase calibration as well as
self-calibration.
\FRING was created for fring fitting VLBI data.
Several Calibration bugs were fixed.
\myitem{\MKTIN} Major improvement to \AIPS\ ability to process
Mark III VLBI data.
\myitem{\MX} \MX has been modified to accept un-sorted \UV-data.
This significantly reduces disk space needed to process observations.
\myitem{Print} All printing tasks were modified so that output
could be re-directed to a file.
\myitem{TVs} Improvements were made to the three workstation TV emulators.
\myitem{\IBLED} An Interactive Baseline EDitor is available.
This task is particularly suited to VLBI data and
provides a graphics interface to \UV-data editing.
\myitem{\FILLM} Numerous improvements to \AIPS\ ability to read
VLA archive tapes.
\myitem{\DAYFX} A new task to fix occasionally garbled Day numbers
in \UV-data read by FILLM.
\myitem{\CVEL} Numerous improvements and bug fixes were applied.
\myitem{\VBPLT} Several improvements to VLBI Baseline Plotting tasks.
\myitem{IBM} Subroutine libraries were added for the IBM 6000/RISC
computers.
\myitem{HORUS} Errors in applying UNIFORM Weighting in HORUS were
corrected.
\end{description}
\normalstyle

%text describing the \MX AP size changes.
\input MXPSAP.tex
\end{document}
