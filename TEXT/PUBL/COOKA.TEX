%-----------------------------------------------------------------------
%;  Copyright (C) 1995
%;  Associated Universities, Inc. Washington DC, USA.
%;
%;  This program is free software; you can redistribute it and/or
%;  modify it under the terms of the GNU General Public License as
%;  published by the Free Software Foundation; either version 2 of
%;  the License, or (at your option) any later version.
%;
%;  This program is distributed in the hope that it will be useful,
%;  but WITHOUT ANY WARRANTY; without even the implied warranty of
%;  MERCHANTABILITY or FITNESS FOR A PARTICULAR PURPOSE.  See the
%;  GNU General Public License for more details.
%;
%;  You should have received a copy of the GNU General Public
%;  License along with this program; if not, write to the Free
%;  Software Foundation, Inc., 675 Massachusetts Ave, Cambridge,
%;  MA 02139, USA.
%;
%;  Correspondence concerning AIPS should be addressed as follows:
%;          Internet email: aipsmail@nrao.edu.
%;          Postal address: AIPS Project Office
%;                          National Radio Astronomy Observatory
%;                          520 Edgemont Road
%;                          Charlottesville, VA 22903-2475 USA
%-----------------------------------------------------------------------
% Summary of standard aips calibration process.
% last edited by Eric Greisen for CookBook inclusion
\documentstyle{article}
\newcommand{\lastedit}{{\it 92 January 16}}
\newcommand{\Chapt}{21}
\large
\newcommand{\doFIG}{T}
\newcommand{\figyes}{T}
\parskip 3mm
\textwidth 6.5in
\linewidth 6.5in
\marginparsep 0in
\oddsidemargin 0in
\evensidemargin 0in
\topmargin -.5in
\headheight 0in
\headsep 0.25in
\textheight 9.25in
\headheight 0.25in
\pretolerance=10000
\parindent 0in
\input psfig

\newcommand{\beq}{\begin{equation}}       % start equation
\newcommand{\eeq}{\end{equation}}
\newcommand{\beddes}{\begin{description} \leftmargin 2cm} % description list
\newcommand{\eeddes}{\end{description}} % description list
\newcommand{\backs}{$\backslash$}
\newcommand{\myitem}[1]{\item{\makebox[2cm][l]{\bf {#1}}}}
\newcommand{\mybitem}[1]{\item{\makebox[0.65cm][l]{\sc {#1}}}}
\newcommand{\AIPS}{{$\cal AIPS$~}}
\newcommand{\COOKBOOK}{{\cal C{\it ook\/}B{\it ook\/}}}
\newcommand{\Cookbook}{{\cal C{\it ook\/}B{\it ook\/}}}
\newcommand{\APEIN}[1]{{\large \tt #1}}
\newcommand{\uvdata}{{\it uv}-data}
\newcommand{\IF}{{\normalsize \sc IF}~}
\newcommand{\SN}{{\normalsize \sc SN}~}
\newcommand{\SU}{{\normalsize \sc SU}~}
\newcommand{\CL}{{\normalsize \sc CL}~}
\newcommand{\TCTES}{{\normalsize \sc 3C48}}
\newcommand{\TCPCL}{{\normalsize \sc 3C286}}
\newcommand{\TCOTE}{{\normalsize \sc 3C138}}
\newcommand{\FREQID}{{\normalsize \sc FREQID}}
\newcommand{\normalstyle}{\baselineskip 8mm \parskip 1mm \large}
\newcommand{\tablestyle}{\baselineskip 4mm \parskip 0mm \normalsize }
%%% ------------------- open index file  ----------------------
\newwrite\indxfile
\immediate\openout\indxfile=\jobname.ndx

%                             basic macro
\newcommand{\WINDEX}[3]{\write\indxfile{:#1:\Chapt:\thepage:#2:#3}}

%                             text as is in output and index
\newcommand{\indx}[1]{#1\WINDEX{#1}{R}{R}}
\newcommand{\Indx}[1]{#1\WINDEX{#1}{R}{B}}
%                             text as is in output and {\tt index}
\newcommand{\tndx}[1]{#1\WINDEX{#1}{T}{R}}
\newcommand{\Tndx}[1]{#1\WINDEX{#1}{T}{B}}
%                             index only as is
\newcommand{\iodx}[1]{\WINDEX{#1}{R}{R}}
\newcommand{\Iodx}[1]{\WINDEX{#1}{R}{B}}
%                             text as is in output and {\tt index}
\newcommand{\todx}[1]{\WINDEX{#1}{T}{R}}
\newcommand{\Todx}[1]{\WINDEX{#1}{T}{B}}

\begin{document}
\pagestyle{myheadings}
\newcommand{\HEADING}{{\it \AIPS Continuum Calibration Summary} \hfill Page~~}
\markboth{\HEADING}{\HEADING}
\vskip -.5cm
\pretolerance 10000
\normalstyle
\listparindent 0cm
\labelsep 0cm
\centerline{\huge{\it Summary of \AIPS Continuum UV-data Calibration}}

\centerline{{\it From VLA Archive Tape to a UV FITS Tape}}
\normalsize
\centerline{{\it \AIPS Memo 76 Updated}}
%\centerline{\lastedit}
\centerline{Glen Langston}

\normalstyle
\centerline{\bf BASIC CALIBRATION}

The Gentle User enters the Computer room with a VLA archive tape
containing a scientific breakthrough.  The user's sources are named
\APEIN{source1} and \APEIN{source2}.  The interferometer phase is
calibrated by observations of \APEIN{cal1} and \APEIN{cal2}.  The flux
density scale is calibrated by observing \TCTES\ (=0137+331) and
polarization is calibrated with observations of \TCPCL\ and/or \TCOTE.
Mount the tape on drive number {\it n}, log in and start \AIPS. Example
input: \APEIN{AIPS NEW}.  Mount the tape:
\APEIN{INTAPE={\it n}; DENS=6250; MOUNT}.
\beddes
\myitem{\tndx{PRTTP}} Find out what is on the tape, get project number
   and bands.~
\APEIN{TASK='PRTTP'; PRTLEV=-2; NFILES=0; INP; GO; WAIT; REWIND}.
\myitem{\tndx{FILLM}} Load your data from tape.
Select only one band at a time to process.
\APEIN{TASK='FILLM'; VLAOBS='?'; BAND='\ '; NFILES=?; INP; GO}.
(Replace all \APEIN{?}'s with appropriate values.) \APEIN{FILLM} will
load your visibilities (\uvdata) into a large file for each band and
create 6 \AIPS tables each.  The tables have two letter names
described below.\iodx{extension files}
\tablestyle
\beddes
\mybitem{HI} Human readable history of things done to your data.
Use PRTHI to read it.
\mybitem{AN} Antenna location and polarization tables.  Antenna
polarization \indx{calibration} is placed here.
\mybitem{NX} Index into visibility file based source name and
observation time.  Not modified by calibration.
\mybitem{SU} Source table contains the list of sources observed
and indexes into the frequency table.  The flux densities of the
calibration sources are entered into this table.
\mybitem{FQ} Frequencies of observation and bandwidth with index
into visibility data. Not modified.
\mybitem{CL} Calibration table describing the antenna based gains.
Version 1 should never be modified.
The CL table contains entries at regular time intervals (i.e. 2
minutes) for each antenna.
{\bf The ultimate goal of calibration is to create a good \CL version 2.}
Use PRTAB to read tables.
\eeddes
\normalstyle
\myitem{\tndx{PRTAN}} Print out the antenna locations.
\APEIN{TASK='PRTAN'; PRTLEV=0; INP; GO}.
Choose a good {\it Reference} antenna (called {\it R})
near the center of the array (\APEIN{REFANT=R}).
Check the VLA operator log to make sure the antenna was OK
during the entire observation.
\myitem{\tndx{QUACK}} Flag the bad points at the beginning of each
scan, even the ones with good amplitudes could have bad phases.
Creates a Flag Table (\APEIN{FG}).  You want to use
\APEIN{FG} table version 1 for all tasks.
\APEIN{TASK='QUACK'; FLAGV=1; OPCOD='~'; APARM=0; SOUR='~'; INP; GO}
deletes the first six seconds of each scan, which may not be enough.
\tablestyle
\beddes
\mybitem{FG} A flag table marks bad data. FG tables contain an index
into the UV data based on time range, antenna number, frequency and
\IF number.
\eeddes
\normalstyle
\myitem{\tndx{LISTR}} Lists your UV data in a variety of ways.  Make a list
of your observations.
\APEIN{TASK='LISTR'; OPTYP='SCAN'; DOCRT=-1; SOUR='~'; CALC='*';}
\APEIN{TIMER=0; INP; GO}.
NOTE: IF you have observed in a such a way as to create more than one
\FREQID,  you must run through the entire calibration
once for EACH \FREQID.
For new users, it is better to use \APEIN{UVCOP} to copy each
\FREQID\ into separate files and calibrate each file separately.  This
is required if you are doing polarization calibration.
\myitem{\tndx{UVCOP}} Skip this step if your data consists of only one
   \FREQID. Copy different \FREQID s into separate files.
\APEIN{TASK='UVCOP'; FREQID=?; CLRON; OUTDI=INDI; INP; GO}.
The result will be a \APEIN{??.UVCOP}~ file.
\myitem{\tndx{SETJY}} Sets the flux of your flux calibration source in
the \SU table.
\APEIN{TASK='SETJY'; SOUR='3C48','~'; OPTYP='CALC'; FREQID=1; INP; GO}.
Adjust flux density for partial resolution following the rules in the
VLA Calibration Source Manual or the \AIPS\ \COOKBOOK.
\myitem{\tndx{TASAV}} As insurance, make a copy of all your tables.
\APEIN{TASK='TASAV'; CLRON; OUTDI=INDI; INP; GO}.
\myitem{\tndx{CALIB}} \APEIN{CALIB} is the heart of the \AIPS calibration
package.  \APEIN{RUN \tndx{VLAPROCS}}, an \AIPS {\it runfile}, to create
procedures \APEIN{\tndx{VLACALIB}, \tndx{VLACLCAL}} and \APEIN{VLARESET}.
The procedure \APEIN{VLACALIB} runs \APEIN{CALIB}.  Set the UV and
Antenna limits for \TCTES.  For L, C and X band 5\% and 5 degree
errors are OK; for other bands the limits are higher.  \APEIN{CALIB}
places antenna amplitude and phase corrections into an \SN table for
the time of observation of phase calibration sources.
\tablestyle
\beddes
\mybitem{SN} Solution table contains antenna based amplitude and phase
corrections for the time of observations of the calibration sources.
These \SN table results are latter interpolated for all times of
observation and placed in a \CL table.  Only the \CL table corrections
will be applied to the program sources.
\eeddes
\normalstyle
\APEIN{TASK='VLACAL';}~ \APEIN{CALS='3C48','~';}~
\APEIN{CALCODE='*';}~ \APEIN{REFANT={\it R};}~
\APEIN{UVRA=?; SNVER=1; DOCALIB=-1}~
\APEIN{MINAMP=10; MINPH=10; INP; VLACAL}.
The task \APEIN{CALIB} lists antenna pairs which deviate significantly
from the solution.  If you have lots of errors, then carefully examine
your data using \APEIN{\tndx{TVFLG}} or \APEIN{LISTR}. (See \AIPS\
\COOKBOOK\ \iodx{COOKBOOK} for a lengthy discussion on flagging.)~
If one antenna is bad over a limited time range, use \APEIN{UVFLG}
to flag that antenna for the time from just after the previous good
\APEIN{cal} observation to before the next good \APEIN{cal} observation.
\myitem{\tndx{UVFLG}} Flag bad UV-data.
\APEIN{TASK='UVFLG'; ANTEN=?,0; BASELI=?,0; TIMER=?; FLAGV=1;
SOUR='~'; OPCOD='~'; INP; GO}.~
If in doubt about any data, \APEIN{FLAG THEM!}  If you have flagged
the primary calibrator, return to \APEIN{CALIB} above and try again.
\myitem{\tndx{CALIB}} Now calibrate the antenna gain based on the rest of the
cal sources.  Look in the Calibrator manual for UV limits; if there
are limits, \APEIN{VLACAL} must be run separately for these sources.
\APEIN{TGET VLACAL; CALS='cal1','cal2','~'; ANTEN=0; BASELI=0;
UVRANGE=?,?; INP; VLACAL}.
Flag bad antennas listed.  Each execution of \APEIN{CALIB} replaces
previous corrections in the \SN table or appends new corrections.
If unsatisfied with a \APEIN{VLACAL} execution, all effects
of it are removed by running \APEIN{VLACAL} again
for the same sources (but different \APEIN{ADVERBS}
or after flagging bad data).
\myitem{\tndx{GETJY}} Sets the flux of phase \indx{calibration}
    sources in the \SU table.
\APEIN{TASK 'GETJY'; SOUR='cal1,'cal2','~';}~
\APEIN{CALS='3C48','~'; BIF=0; EIF=0; INP; GO}.~
\APEIN{GETJY} over-writes existing \SU table
entries, and is not affected by previous executions.
\myitem{\tndx{TASAV}} Good time to save your tables.
\APEIN{TGET TASAV; INP; GO}.~
\myitem{\tndx{CLCAL}} Read the antenna amplitude and phase corrections from
the
\SN table and interpolate the corrections into a new \CL table.~
\APEIN{CLCAL} applies calibration source corrections to the
program sources.
Each execution of \APEIN{CLCAL} adds to output \CL table version 2.
\APEIN{CLCAL} is run using the procedure \APEIN{VLACLCAL}.~
\APEIN{TASK='VLACLC';}~ \APEIN{SOUR='source1','cal1','~';}~
\APEIN{CALS='cal1','~'; OPCODE='CALI';}~
\APEIN{TIMER=0;}~
\APEIN{INTERP='2PT';}~ \APEIN{INP; VLACLC}.~
Run \APEIN{CLCAL} for the second source using the second calibrator.
\APEIN{TGET VLACLC; SOUR='source2','cal2','~'; CALS='cal2','~'; INP;
VLACLC}.~
Move the \SN table corrections for \TCTES\ into the \CL table.
\APEIN{TGET VLACLC; SOUR='3C48','~';CALS='3C48','~'; INP; VLACLC}.
(\TCTES\ could also be calibrated with \APEIN{cal1} or \APEIN{cal2}.)
\eeddes
%\clearpage

\begin{figure}[t]
\if\doFIG\figyes
    \centerline{\psfig{figure=FIG/MEMO76A.PLT,height=3.6in}\hss
        \psfig{figure=FIG/MEMO76B.PLT,height=3.6in}}
\else
    \vskip 3.6in
\fi
{\it \hskip 1.5in a) \hfill b) \hskip 1.5in}

{\bf Figure:}
{\it a)} Un-calibrated {\it uv}-data and  {\it b)} calibrated {\it
uv}-data from a C-band snapshot of \TCTES.  Default VLA gains are a
tenth of the actual gains and can show significant scatter.  Only wild
{\it uv} points $\sim$50\% greater than the average can be detected
before calibration.
\end{figure}

\begin{description}
\myitem{\tndx{LISTR}} Make a matrix listing of the Amplitude and RMS
   of \indx{calibration} sources with calibration applied.  Look for
   wild points.~
\APEIN{TASK='LISTR'; OPTYP='MATX'; SOUR='cal1','cal2','~'; DOCAL=1;}~
\APEIN{DOCRT=-1; DPARM=3,1,0; UVRA=0; ANTEN=0; BASELI=0; BIF=1; INP; GO}.~
If only a few points are bad, flag them and continue.
If too many are bad, delete \CL table 2 and the \SN
tables using \APEIN{VLARESET}.
Then return to the first \APEIN{CALIB} step.
If the data look good, run \APEIN{LISTR} again for \IF two.
\APEIN{TGET LISTR; BIF=2; INP; GO}
\myitem{\tndx{UVPLT}} Plot the \uvdata\ in a variety of ways.  Make a Flux
versus Time plot first.  Choose \APEIN{XINC} so the plot will have no
more than 1000 points.
\APEIN{TASK='UVPLT'; SOUR='source1','~'; XINC=10; BPARM(1)=11;
DOCAL=1; BIF=1; INP; GO}.~
Look at the plot with \APEIN{LWPLA, TKPL, TVPL} or \APEIN{TXPL}.~~
Plot other \IF. Flag wild points. Plot Flux versus baseline.
\APEIN{TGET UVPLT; BPARM=0; INP; GO}.
\eeddes

Calibration is now complete for continuum, un-polarized observations.
Write the calibrated data to tape with \APEIN{FITTP} if you don't want
to calibrate the polarization.  To create images from the \uvdata\ use
\APEIN{SPLIT} to calibrate the multi-source data and create a single
source \uvdata\ set.  (\APEIN{FITTP} and \APEIN{SPLIT} are described
at the end of the polarization calibration process.)

\clearpage
\centerline{\bf POLARIZATION CALIBRATION}
For \indx{polarization} observations, the following steps are
required.  For 21cm or longer wavelength observations, ionospheric
Faraday rotation corrections may be needed.  See \APEIN{FARAD} in the
\AIPS\ \COOKBOOK, but don't expect much help anymore.
\beddes
\myitem{\tndx{TASAV}} As added insurance, save your tables again.
\APEIN{TGET TASAV; INP; GO}.
\myitem{\tndx{LISTR}} Print the parallactic angles of the calibration
sources.
\APEIN{TGET LISTR; SOUR='~'; CALC='*';
OPTYP='GAIN'; DPARM=9,0; INP; GO}~
\myitem{\tndx{PCAL}} Intrinsic antenna polarization calculation.
\APEIN{PCAL} will be successful
only if cal.~sources are observed at several parallactic angles.
\APEIN{PCAL} will modify the \APEIN{AN} and \APEIN{SU} tables.
\APEIN{TASK='PCAL'; CALS='cal1','cal2','~'; BIF=1; EIF=2;
DOCAL=1; REFANT={\it R}; INP; GO}
\myitem{\tndx{LISTR}} Now determine the absolute linear polarization angle.
Make a matrix listing of the angle of \TCPCL.
\APEIN{TGET LISTR; SOUR='3C286','~'; DOCAL=1; BIF=1; DOPOL=1;}~
\APEIN{GAINUSE=2; OPTYP='MATX'; DPARM=1,0; STOKES='POLC'; INP; GO}.~
Record the matrix average angle, $\phi_1$, for \IF 1.
The observed angles are different for each frequency and \IF.
Record the matrix average angle, $\phi_2$, for \IF 2
(\APEIN{BIF=2; INP; GO}).
\myitem{\tndx{CLCOR}} Now apply the angle corrections to CL table 2.  The
relative phase of Left and Right circular polarization produces the
linear polarization angle and the phase correction is applied to L.
The phase difference (twice the angle of linear polarization) for
\TCPCL\ is $66^o$ and for \TCOTE, $\phi=-18^o$ at L band, perhaps
$-24^o$ at higher frequencies.
\APEIN{TASK='CLCOR'; STOKES='L'; SOUR='~'; OPCOD='POLR'; BIF=1; EIF=2;
\mbox{CLCORPRM=66-$\phi_1$,66-$\phi_2$,0}; GAINVER=2; INP; GO}.~
Run \APEIN{LISTR} again to check the phases.~
\APEIN{TGET LISTR; INP; GO}.~
If the phases are wrong, return to \APEIN{PCAL} and \APEIN{LISTR},
then do another \APEIN{CLCOR}.

\vskip 10pt
\centerline{\bf BACKUP AND IMAGING}

\myitem{\tndx{FITTP}} Writes the output {\it uv}-data to tape.
\APEIN{DISMOUNT} your archive; \APEIN{MOUNT} your output tape.
\APEIN{TASK='FITTP'; DOEOT=1; OUTTAP=INTAP; INP; GO}.~  Use
\APEIN{DOEOT=-1} when at the beginning of a new tape.
\myitem{\tndx{SPLIT}} The \AIPS calibration process only modifies the tables
associated with the multi-source \uvdata\ set.  \APEIN{SPLIT} selects
individual sources, reads the \CL table and multiplies the
visibilities by the corrections to produce a calibrated single-source
\uvdata\ set.
\APEIN{TASK='SPLIT'; SOUR='~'; CALC='~'; UVRA=0; TIMER=0; DOCAL=1;}
\APEIN{FLAGVER=1; GAINUSE=2; DOPOL=1; DOBAN=-1; BIF=0; EIF=0; STOKES='~';}
\APEIN{BLVER=-1; APARM=0; DOUVCOM=1; CHANSEL=0; INP; GO}
\myitem{Mapping} Use your favorite Fourier Transform task
(e.g. \APEIN{UVMAP}, \APEIN{HORUS}, \APEIN{MX} or \APEIN{WFCLN}) to
produce images from the calibrated data.  A set of \AIPS procedures
(called \APEIN{MAPIT}) has been developed to automatically Fourier
Transform, deconvolve and self-calibrate the \uvdata.  See \AIPS Memo
72.
\eeddes

\end{document}


