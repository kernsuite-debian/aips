% -*- latex -*-
%-----------------------------------------------------------------------
%;  Copyright (C) 2000
%;  Associated Universities, Inc. Washington DC, USA.
%;
%;  This program is free software; you can redistribute it and/or
%;  modify it under the terms of the GNU General Public License as
%;  published by the Free Software Foundation; either version 2 of
%;  the License, or (at your option) any later version.
%;
%;  This program is distributed in the hope that it will be useful,
%;  but WITHOUT ANY WARRANTY; without even the implied warranty of
%;  MERCHANTABILITY or FITNESS FOR A PARTICULAR PURPOSE.  See the
%;  GNU General Public License for more details.
%;
%;  You should have received a copy of the GNU General Public
%;  License along with this program; if not, write to the Free
%;  Software Foundation, Inc., 675 Massachusetts Ave, Cambridge,
%;  MA 02139, USA.
%;
%;  Correspondence concerning AIPS should be addressed as follows:
%;          Internet email: aipsmail@nrao.edu.
%;          Postal address: AIPS Project Office
%;                          National Radio Astronomy Observatory
%;                          520 Edgemont Road
%;                          Charlottesville, VA 22903-2475 USA
%-----------------------------------------------------------------------
%Body of AIPSletter for 15 April 2000

\documentclass[twoside]{article}
\usepackage{graphics}

\newcommand{\AIPRELEASE}{April 1, 2000}
\newcommand{\AIPVOLUME}{Volume XX}
\newcommand{\AIPNUMBER}{Number 1}
\newcommand{\RELEASENAME}{{\tt 31DEC99}}
\newcommand{\OLDNAME}{{\tt 15OCT99}}
\newcommand{\NEXTNAME}{{\tt 31DEC99}}

%macros and title page format for the \AIPS\ letter.
\input LET98.MAC
%\input psfig

\newcommand{\MYSpace}{-11pt}

\normalstyle

\section{General developments in \AIPS}

\subsection{Current and future releases}

We do not expect to have any further formal releases of \AIPS\@.
Instead, \AIPS\ has been available as the \RELEASENAME\ version since
October 1999.  This version remains under development by the (reduced)
\AIPS\ Group to correct and improve this ``\AIPS\ for the Ages''
indefinitely.  You may fetch and install a complete copy of this
version at any time.  Having done so, you may update your installation
whenever you want either as a whole or by running the so-called
``midnight job'' which uses transaction files to copy and compile the
code selectively based on the code changes and compilations we have
done.  We expect that most users will take the source-only version of
\AIPS\ over the Internet (via \emph{anonymous} ftp), but we can ship
it on more traditional media if requested (contact Ernie Allen at the
address given in the masthead).  \AIPS\ is now copyright \copyright\
1995 through 2000 by Associated Universities, Inc., NRAO's parent
corporation, but may be made freely available under the terms of the
Free Software Foundation's General Public License (GPL)\@.  This means
that User Agreements are no longer required, that \AIPS\ may be
obtained via anonymous ftp without contacting NRAO, and that the
software may be redistributed (and/or modified), under certain
conditions.  The full text of the GPL can be found in the
\texttt{15JUL95} \Aipsletter.

\subsection{Personnel}

The \AIPS\ Group, none of whom work full time on \AIPS, has consisted
of Chris Flatters, Ketan Desai, and Leonia Kogan doing programming and
user support at the AOC in Socorro and Eric Greisen and Pat Murphy
performing similar roles in Charlottesville.  Ernie Allen handles data
base and shipping chores for the Group.  This is a pretty minimal
group to support such a large and widespread software package.
Unfortunately, this group has been eroding.  Pat is also the Computer
Division Head for Charlottesville, a nearly full-time job.  Ketan
has now left the group to find greener pastures working in the New
York City area.  We wish him well in his new endeavors, but his
departure has left us below minimum strength.

     Two actions are being taken to address this.  First, Eric is
moving to Socorro for the period from April to the middle of October.
Second, despite NRAO's current financial problems, a position to
replace Ketan has been announced.  The full text of this announcement
appears in the following pages.  All things being equal, a
VLBI-knowledgeable radio astronomer with \AIPS\ and software support
experience would be our ideal candidate.

\vfill
\eject

\section{\AIPS\ position available}

\begin{center}
\begin{tabular}{l}
Assistant Scientist - \AIPS\ Project \\
NATIONAL RADIO ASTRONOMY OBSERVATORY \\
Personnel Office \\
P.O. Box 0 \\
Socorro, NM 87801 \\
Email Submission: {\tt alewis@nrao.edu} \\
Attention: A. Lewis
\end{tabular}
\end{center}

The Astronomical Image Processing System (\AIPS) is a software package
for reducing data from single-dish and array radio telescopes in the
US and throughout the world.  More information on \AIPS\ is available
from the \AIPS\ home page at {\tt http://www.cv.nrao.edu/aips/}.

The \AIPS\ software group has a job vacancy at the National Radio
Astronomy Observatory Array Operations Center in Socorro, New Mexico.
The duties will be to provide \AIPS\ programming support for local and
visiting astronomers, participate in software development in support
of NRAO instruments and programs, and to coordinate with the
scientific staff on data-reduction software issues. We anticipate that
up to 20 hours a week of the applicant's time will be required to
perform these duties; the remainder will be available for independent
research. We are particularly interested in applicants with VLBI
experience.

Substantial experience using and programming in \AIPS\ is expected.
Experience with Unix system administration and/or scientific user
support is desirable. A strong background in radio astronomy is
required. This position has an initial term of three years; further
extension may be possible.

Socorro is a small historic town (pop.~9000) in the mid Rio Grande
valley of New Mexico. More information can be obtained from the
Socorro Home Page at {\tt http://www.socorro.com}

Further information on NRAO is available from the NRAO Home Page
{\tt http://www.nrao.edu}.  Please contact Tony Beasley ({\tt
tbeasley@nrao.edu}) for more information on this position.

Applications should be addressed to A. Lewis at the above address.
First consideration of applications will occur May 01 2000. AAE/EOE.

\section{Simulating \UV\ data for an array design}

The recently written task {\tt UVCON} generates a \uv\ database for an
interferometric array whose configuration is specified by the user.
Visibilities corresponding to a specified model with Gaussian noise
appropriate for the specified antenna characteristics are calculated.
The output is a standard \AIPS\ \uv\ data file.  This task replaces
the old simulating procedure which required use of the \AIPS\ tasks
{\tt UVSIM}, {\tt UVSUB}, and {\tt UVMOD} and verb {\tt PUTHEAD}\@.
The array geometry can be specified in four different coordinate
systems: earth-centered equatorial, local tangent plane, geodetic, and
array-centered equatorial.   The antenna diameters, efficiency, noise
temperature, and number of levels in the digitizer are used to
calculate the noise level for the given visibility.  An atmospheric
contribution to the noise level is now available.

Having simulated the \uv\ data, the whole power of \AIPS\ can be used
for imaging, plotting, printing, and editing.  In particular, the
standard \AIPS\ tasks {\tt UVPLT}, {\tt IMAGR}, {\tt KNTR}, and {\tt
LWPLA} can be used for creating the \uv\ coverage, beam pattern, and
simulated image files in PostScript format.

The task has successfully used for the array configuration design of
the VLA expansion project by R. Perley and ALMA project by Kogan, L.,
ALMA Memo \#247, 1999, ALMA Memo \#296, 2000; by Conway, J., ALMA Memo
\#291, 2000;  and by Heddle, S., and Webster, A., ALMA Memo \#290, 2000.
The last memo ``describes some progress that has been made towards
automation of simulations using straightforward shell scripts applied
while running Linux, with Classic \AIPS\ as the core simulation
tool.''  All Memos may be obtained on the WWW at {\tt
http://www.alma.nrao.edu/memos/index.html}.

\vfill\eject
\section{Improvements of interest to users in \RELEASENAME}

Although we do not intend to have any more formal, frozen releases, we
expect to continue publishing the  \Aipsletter\ approximately every
six months so long as there are things to report and an audience to
care.  Among the new thing in {\tt 31DEC99} over the past six months
are two tasks: {\tt SCIMG}, a multi-field iterative imaging and
self-calibration program and {\tt FGPLT}, a program to illustrate
when data are flagged.  There are major new options in the main
imaging task {\tt IMAGR}, significant enhancements to interactive
editing tasks, and new verbs to help manage slices.  Several
significant bugs were corrected as well.

{\tt 31DEC99} is compatible in all major ways with the other 1999 and
{\tt 15OCT98} releases.  There are significant incompatibilities
with older versions of \AIPS\@.

\subsection{Imaging}

\subsubsection{IMAGR}

A multi-resolution variant of the Clean algorithm has been added to
{\tt IMAGR}\@.  It takes advantage of the multiple field capabilities
to model the source simultaneously as a collection of Gaussian objects
of user-specified widths.  The same part of the sky is imaged at
several different resolutions corresponding to these widths.  Clean
then selects the strongest field and removes some number of components
from it.  The only differences from normal Clean are in the tapers
applied to the image and the subtraction of a circular Gaussian rather
than a point from the \uv\ data.  The present algorithm is based on a
description of a ``multi-scale'' Clean given by Mark Holdaway and Tim
Cornwell at an imaging meeting in Socorro in 1999, but it is likely to
differ from their algorithm in significant details.  The {\tt IMAGR}
version offers two simple adverbs, {\tt NGAUSS} and {\tt WGAUSS}, to
specify the use of this new technique and offers numerous ``knobs'' in
some of the {\tt IMAGRPRM} adverb to adjust its behavior.  Users
should try this new technique with caution --- use {\tt DOTV} true;
please report your experiences to {\tt egreisen@nrao.edu}.

The output class names from all imaging tasks were changed; the
previous usage of 0-relative, extended hexadecimal was confusing to
users.  The new names are {\it x{\tt IM}nnn}, {\it x{\tt CL}nnn}, and
{\it x{\tt BM}nnn}, where {\it x} is the Stokes code and {\it nnn} is
the field number from {\tt 001} to {\tt 512}.  All calibration tasks
can recognize either old- or new-style names for modeling, but earlier
versions of \AIPS\ will not recognize the new names.  Writers of
procedures should remember that \AIPS\ file naming adverbs support a
variety of powerful wild-card conventions.

Another new option is encoded in the new {\tt ALLOKAY} adverb.  This
allows the user to reassure {\tt IMAGR} that the pre-existing beam
images and even the pre-existing {\tt IMAGR} work file do not need to
be recomputed.  For restarts with multiple beam images and with large
numbers of components, this can be a considerable savings in time.
The previous version of {\tt IMAGR} limited a user to 256 Clean boxes
per field no matter how many fields were in use.  This has been
changed to allow up to 2048 boxes per field up to 64 fields.  If there
are more fields, then the maximum in any field is (131072 / {\tt
NFIELD}).  When imaging more than one field, a field status line will
appear above the menu on the TV display.

A number of subtle bugs --- with unsubtle consequences --- were
corrected.  A compiler peculiarity in round off affected only Linux
systems (see programmer section) and caused the uniform weighting to
fail on rare occasions.  If the first field was small enough to be
gridded in core, the creation of scratch files for later, larger
fields was compromised.  The routine that decided how big to make the
gridding and gridded subtraction scratch files did not use the same
formulation as the routines that use the files.  In some cases, this
could result in a failure due to a too-small scratch file.  The Clean
component filtering routine had an error causing it to over-write
memory near but not in the designated array.  This too could appear as
a data- and architecture-dependent failure.

\subsubsection{Other imaging changes}

\begin{description}
\myitem{SCIMG} is a new task to image and self-calibrate a \uv-dataset
               iteratively.  It allows large numbers of fields and
               most of the multi-field options of {\tt IMAGR}, but
               does not include (as yet) the experimental variations
               on Clean found there.  {\tt SCMAP} is single-field
               version, retained for simplicity.
\myitem{MAPPR} is a new standard procedure which provides a simplified
               access to {\tt IMAGR}\@.  It is for a single field
               image on pre-calibrated data without the experimental
               options.
\myitem{CMETHOD}\hspace{0.5cm}  adverb allows the user to have the
               task select DFT or gridded subtraction depending on
               which is faster.  This option was added to {\tt FRING}
               and {\tt KRING}\@.  It was found that the formulas used
               were no longer appropriate to modern workstations.
               Much worse was the fact that in almost all
               calibration-like tasks, the grid size was not set when
               the decision was made.  This was a serious bias in favor
               of gridded subtraction.  We made timing measurements,
               changed the formula, and saw to it that all parameters
               are known before the decision is made.
\myitem{SETFC} did not use the correct coordinates for multi-source
               files and did not handle ``negative'' right ascensions
               and even negative declinations correctly.  The task was
               changed to use, for the center fly's eye, a grid of
               field centers in a {\tt -SIN} projection like geometry.
               The attempt to be linear in RA and declination did not
               work near the poles.
\end{description}

\subsection{Interferometric data editing}

\subsubsection{{\tt EDITR}, {\tt EDITA}, et al.}

The object-based {\tt EDIT} class in \AIPS\ was changed to allow for
very rapid editing of multiple-IF, multiple-polarization data.  This
edit class is found directly in the tasks {\tt EDITR}, {\tt EDITA},
and {\tt SNEDT} and is available as a menu option in the
self-calibration and imaging tasks {\tt SCIMG} and {\tt SCMAP}\@.  The
biggest change was to allow, under control of the {\tt CROWDED}
adverb, the simultaneous display of both polarizations and/or all IFs
in the data.  The tasks begin with a ``crowded'' display, but the
user may select interactively whatever combination of single or
multiple IFs and polarizations are desired for the current display and
editing.  A new menu option, {\tt NEXT CORRELATOR}, was added to
assist in this selection.  The crowded display posed significant
complications in determining what data sample is being selected by the
TV cursor.  In some cases, multiple samples are plotted on the same TV
pixel and, in these cases, all samples on that pixel are selected when
the pixel is selected.  Getting all samples deleted, including the
affects of the {\tt ALL-IF}, {\tt ALL-POL}, and antenna selection
flags, posed several interesting challenges.

Tasks {\tt EDITA}, {\tt EDITR}, and {\tt SNEDT} had adverbs {\tt
REASON} and {\tt ANTUSE} added to allow the user to select the initial
``reason'' string and initially displayed antennas.  Both of these may
then be changed interactively.  The ability to specify amplitude
displays from 0 to maximum rather than minimum to maximum was added.
If these tasks start with {\tt EXPERT} true, certain display limits
and flag settings are now different than in the menu-driven mode.  The
computation of differences between the data and a running mean of the
data was postponed until just before it is needed for display.  This
avoids potentially long waits for these data which may never be
displayed with the current settings.

\subsubsection{{\tt SPFLG}}

{\tt SPFLG} was substantially revised so that the horizontal axis
contains all selected spectral channels for all selected IFs.  This
avoids have to display one IF after another while editing.  It is
tedious enough to have to loop over polarization and baseline without
this extra loop and modern TV displays are usually wide enough to
encompass the longer rows.  The task is aware of which IF and channel
are being selected and does appropriate displays and deletions
depending on the setting of the {\tt ALL-IF} flag.  The {\tt SWITCH
ALL-BL FLAG} option from current vs all baselines to a 4-position
switch with all baselines to each 1 of the antennas included as
choices.

During this upgrade, a number of error in {\tt SPFLG} (and {\tt
TVFLG}) were uncovered.  An error in the declaration of the {\tt FC}
table common and I/O buffer caused the flag listing to be peculiar and
caused Linux systems at least to die on {\tt UNFLAG} operations.  An
error recording the current Stokes caused later {\tt REDO} flag
operations to behave incorrectly.  {\tt TVFLG} now requires the user
to type an IF or channel number only if there are more than two of
them.

\vfill\eject
\subsection{VLBI and interferometric data handling}

\subsubsection{Phase re-referencing}

The subroutine used by {\tt CALIB}, {\tt CLCAL}, {\tt FRING}, and {\tt
KRING} (at least) to bring the {\tt SN} tables to a common phase
reference was rewritten.  {\tt CALREF} changes the reference antenna
for a single IF, polarization, and subarray.  It interpolates between
or extrapolates from tabulated values of the phase, delay, and rate
corrections for the old reference antenna relative to the new one that
have been extracted from the table and smoothed using boxcar functions
in time.  It uses the smoothed rate corrections to provide time
derivatives for the phase and delay corrections during
interpolation/extrapolation.  {\tt CALREF} now reports an error if the
old and new reference antennae cannot be related to one another; it no
longer fails if there is only one solution linking the two.  Delays
are now interpolated using the rate as the first derivative of delay
with respect to time.  The correct interpolation tables are now used
if the smoothing times are small and smoothing times now conform to
the header comments.  It no longer applies a spurious delay
correction, it avoids overflows in computing {\tt  ATAN2} properly and
excess arguments (including a large work array) have been removed.

Both {\tt FRING} and {\tt KRING} had errors which caused them to fail
to change the reference antenna for the second polarization.  The
calling tasks now allocate needed memory dynamically to handle the
table being modified.  They issue warnings but continue when {\tt
CALREF} finds no connection between the old and new reference antennae.
This is often triggered by the use of different reference antennae in
different subarrays.

\subsubsection{FILLM and DOCALIB = 2}

{\tt FILLM} has offered the option to set the weights using the
antenna temperatures recorded with the data.  This option was
deprecated since the absolute calibration of the system temperatures
is unreliable for the frequency and date of observation.  Ketan Desai
found that the ``nominal sensitivity'' is recorded with the data.
This is the system level information used to scale the visibilities
approximately to deciJanskys prior to calibration.  If one uses these
sensitivities also to set the weights, and then applies the flux
calibration to the those weights, the resulting weights should
represent $1/\sigma^2$ which is desired in averaging and imaging.
At present, the user has to instruct {\tt FILLM} to use this option
with the third bit in {\tt CPARM(2)}, \ie\ set {\tt CPARM(2) = 8 +}
other options.  All calibration tasks were changed to have {\tt
DOCALIB = 2} mean to apply the calibration to the visibilities and the
weights, while {\tt DOCALIB = 1} calibrates only the visibilities in
{\it all} tasks.  Previously, some VLBI-oriented tasks applied weight
calibration, {\tt SPLIT} and {\tt SPLAT} offered the choice, and the
rest never calibrated the weights.

\subsubsection{DBCON}

{\tt DBCON} has the thankless task of trying to combine two datasets
that may actually be rather different.  One thing it can do is
combine data sets with different numbers of polarizations.  A patch
was issued correcting errors in handling that case.  Position shifting
is done with the best estimate of the frequency of each channel;
obsolete information about options to avoid this was removed.
Previously, {\tt DBCON} retained the two sets of frequency identifier
numbers unchanged and only issued a mild warning if this seemed likely
to be wrong.  Now, it renumbers the {\tt FQ} values for the second
data set to match (or add to) the first data set.

\subsubsection{Other \uv-data changes}

\begin{description}
\myitem{UVCOP} failed to subtract the 5-day offset correctly when
               making a single subarray dataset from a multiple
               subarray one.  It produced times a bit greater than -5
               days.  It also was able to get the wrong subarray due
               to an inadequate round up.
\myitem{UVFIX} corrected phases using the first channel as the
               reference channel rather than the correct channel.
               The computation of projected baselines was changed to
               the aberrated source direction which seems to improve
               some astrometric results.  (Unexplained variations in
               the ``plate solutions'' with epoch for VLA images still
               remain.)
\myitem{UJOIN} was changed to put the IF selection in the copied
               tables.  The flagging of data under the {\tt DOWEIGHT}
               was corrected and the meaning of {\tt DOWEIGHT} was
               changed.  Additional explanatory  messages and frequency
               checks were added and the help file was made clearer.
\myitem{CALIB} and {\tt BPASS} were changed to check for
               inconsistencies in the user parameters and to refuse to
               proceed when in doubt as to the intentions.
\myitem{TI2HA} was changed to correct times for subarray number (under
               user control) and to use the output name adverbs.
\myitem{UVMTH} was changed to handle compressed data, to use whatever
               dynamic memory is needed, and to handle the two data
               sets independently.  It and {\tt DIFUV} were made more
               immune to differences between the two input data sets.
\myitem{TBIN} had trouble with flagged rows in the table when reading
               the tables some columns at a time.  In that mode it
               must, and now does, flag the row only when writing the
              last columns.
\end{description}

\subsection{Data display}

\subsubsection{{\tt  FGPLT}}

{\tt FGPLT} is a new task to display the times when \uv-data samples
are flagged.  The vertical axis is correlator from the selected
portion of the data set.  Horizontal lines are drawn to illustrate the
times at which each correlator is flagged in the {\tt FG} table.
Since the VLBA Correlator now generates the initial {\tt FG} table for
its output, this task was conceived as a means to evaluate the
observing and correlation success of a data set.  General users should
find it interesting when evaluating the consequences of the various
non-interactive editing tasks in \AIPS\@.  Tasks like {\tt CLIPM} and
{\tt UVLMN}, in particular, write {\tt FG} tables which could be
explored with {\tt FGPLT}\@.

\subsubsection{Other data display changes}

\begin{description}
\myitem{VPLOT} was changed to allow the plotting of more data points,
               to allow flagging all channels based on the data of one
               spectral channel, and to plot a wider choice of
               parameters with time averaging.
\myitem{adverbs} for numerous tasks were changed.  The adverb {\tt
               NCOUNT} had migrated to so many functions that users
               were tripping over themselves.  New adverbs {\tt
               BPRINT}, {\tt EPRINT}, {\tt NPRINT}, and {\tt NPLOTS}
               were created and put in place of the overworked
               adverbs.
\myitem{general} The tick mark algorithm was given a little more work
                to insure that at least two labeled tick marks appear
                even in small plots.
\myitem{Clean box} display and setting had several minor bugs
                corrected.  The most mysterious caused boxes outside
                the display area to move during the {\tt FILEBOX}
                operation.
\myitem{POSSM} had two errors corrected which were due to using the
                wrong version of the data header.  One caused
                frequencies in plots 2 -- {\tt NPLOTS} on each page to
                be wrong; the first plot was correct.
\end{description}

\vfill\eject
\subsection{Analysis and documentation}

\subsubsection{Slices}

\AIPS\ has been using an {\tt xterm} or similar XWindows terminal
emulation program to emulate the old Tektronix display device.  While
this works very well for display, it appears to be unreliable on all
systems for cursor reading.  It works just enough to be interesting
and then returns incorrect values for the cursor position.  The most
significant use of the TK cursor reading was in setting the inputs
for fitting Gaussians to slices.  To get around the TK unreliability,
we have written a parallel set of TV verbs for all the TK slice
operations.  These are {\tt TVSLICE}, {\tt TVASLICE}, {\tt TVMODEL},
{\tt TVAMODEL}, {\tt TVRESID}, {\tt TVARESID}, {\tt TVGUESS}, and
{\tt TVAGUESS} for plotting slices, models, residuals, and guesses and
{\tt TVSET} and {\tt TV1SET} for setting the guess parameters.

In testing the new verbs, the display of fit values and uncertainties
by {\tt SLFIT} was improved.  The internal and ``scientific'' units
are both displayed and labeled.  Note that the units of the adverbs to
{\tt SLFIT} remain the ``scientific'' (actually plot) units, namely
those displayed on the {\tt TKSLICE} or {\tt TVSLICE} plot, \eg\ mJy
and asec.

\subsubsection{Other analysis changes}

\begin{description}
\myitem{CONFI} is being used to study array configurations for ALMA
               and other telescopes.  New configuration optimization
               and constraint options were added in support of this.
               It was revised to handle the new ALMA site topography
               file that has both better resolution and a much larger
               coverage.  Control over which antennas are to be held
               fixed was generalized.
\myitem{UVHOL} was revised to allow removal of samples at both ends of
               each holography scan.
\myitem{JMFIT} was corrected to report the offset term in the correct
               units; it was correctly computed.  The default handling
               of {\tt GMAX} was corrected to match the help file and
               {\tt IMFIT}\@.
\myitem{IRING} used an incorrect convention for position angle.  It
               has been corrected to measure East from North even in
               rotated images.
\myitem{BSMOD} was changed to generate single-dish, beam-switched
               continuum datasets without requiring an input
               dataset..
\myitem{BSCLN} is a new task to Clean beam-switched continuum images.
               The one-dimension task {\tt BSTST} also has new Clean
               options.  The Cleaning is described in ADASS talks at\\
               {\tt ftp://ftp.cv.nrao.edu/egreisen/SWpaper.ps.gz} and\\
               {\tt ftp://ftp.cv.nrao.edu/egreisen/SWposter.my.ps.gz}.
\end{description}

\subsubsection{Documentation}

Unfortunately, the files used by {\tt ABOUT} require manual
intervention to update.  We remember to do this on occasion and have
done so at least once for {\tt 31DEC99} already.  There are enough
changes to software since the last major revision of the \Cookbook\
that we also have revised virtually all chapters to the {\tt 31DEC99}
version.  They are available from the WWW Table of Contents at
{\tt http://www.cv.nrao.edu/aips/cook.html} which also has summaries
of the changes made on a chapter-by-chapter basis.  The main reasons
for changes included {\tt VPUT}, {\tt VGET}, {\tt FITAB}, {\tt TAPES},
{\tt DEFAULT}, numerous new options in {\tt EDITR} and {\tt IMAGR},
{\tt PRINTER}, new adverb and class names, {\tt CLIPM}, {\tt UVMLN},
and TV connectivity and start-up changes.

{\it Going AIPS\/} has not been revised since 1990.  But it is still
surprisingly useful.  It has been made available on a
chapter-by-chapter basis from the WWW Table of Contents at\\
{\tt http://www.cv.nrao.edu/aips/goaips.html}.\\  We are toying with the
idea of upgrading some of the material.

We have run out of pre-printed covers for the \Cookbook\ and for {\it
Going AIPS} and do not plan to print more.  PostScript and jpg
versions of the covers are available from the \AIPS\ WWW pages.

\subsection{Programmer tidbits}

\begin{description}
\myitem{Round off} is handled differently by different
           compilers/architectures.  In particular, the sequence {\tt
           A = B * C} followed by {\tt I = A + 0.5} gets a different
           result than {\tt I = B * C + 0.5}.  Therefore, when two
           computations are required to get the same result, \ie\ in
           computing offsets into an array, they must be expressed in
           the same fashion.
\myitem{Linux} compiler options were changed to test for unused and
           uninitialized variables.  The egcs compiler is not fooled
           by a {\tt DATA} statement into thinking that a variable is
           used.
\myitem{Long long} variables in C, such as those used for byte offsets
           into data files in those systems allowing files larger than
           2 Gbyte, require a different format identifier ({\tt
           \%lld} rather than {\tt \%d}).  This seems to work on all
           architectures but alphas where it will not be needed.
\myitem{Y2K police} have been busily reading all the precursor
           comments throughout the \AIPS\ code.  To forestall further
           complaints, we have changed the precursor comments a few
           places to match the already correct code.  No actual
           problems have been found.
\end{description}

\section{New AIPS Installation Wizard}

In order to make the installation of the {\tt 31DEC99} version of
\AIPS\ easier, the whole installation process has been overhauled and
re-thought.  While the old {\tt INSTEP1} procedure was a significant
improvement over what came before, it is now eight years old and is
difficult to maintain or improve incrementally.  The new approach is
inspired somewhat by the intermediate {\tt update.pl} perl script that
some of you have used to install {\tt 31DEC99}, and also by what one
typically encounters with modern software installs.  The {\tt
update.pl} script is limited in that it expects you to have an older
version of \AIPS\ already installed.

Based on the experience of {\tt update.pl}, we decided to stick with
perl as the language in which to write the new wizard.  Given that
perl is now ubiquitous, comes with Linux in most, if not all,
distributions, and is otherwise freely available under the Artistic
License, it is our hope that the requirement of having perl (version
5) should not cause undue hardship for our users.  You may find copies
of perl at {\tt http://www.perl.org/}.

Some features of the new {\tt install.pl} script include:
\begin{itemize}
 \item\ Gathers all information before doing anything.
 \item\ Offers to retrieve and unpack the distribution if not already
        done.
 \item\ Does system ``Search and Discovery'' with significant
        improvements over previous installation codes.
 \item\ Allows you to save configuration and continue later.
 \item\ Permits ``back'', ``next'', and ``quit'' at each step.
 \item\ Does {\it not\/} use a GUI; uses short plain text menus instead.
\end{itemize}

The choice not to use a Graphical User Interface for this wizard was
partly pragmatic and partly philosophical.  Adding such a feature
would have entailed much additional work, requiring resources (people,
time) not currently available to the \AIPS\ Group.  Any added
functionality would have been questionable at best given the cost in
such resources.

At the time of writing, this wizard is still in preparation.  We
expect to have it available sometime in April 2000; please watch the
\AIPS\ home page at {\tt http://www.cv.nrao.edu/aips/} for news on its
testing and release.

\vfill\eject

\section{Patch Distribution for {\tt 15OCT99}}

As before, important bug fixes and selected improvements in
\OLDNAME\ can be downloaded via the Web at:

\begin{center}
\vskip -10pt
{\tt http://www.cv.nrao.edu/aips/15OCT99/patches.html}
\vskip -10pt
\end{center}

Alternatively one can use {\it anonymous} \ftp\ on the NRAO cpu {\tt
aips.nrao.edu}.  Documentation about patches to a release is placed
in the anonymous-ftp area {\tt pub/aips/}{\it release-name} and the
code is placed in suitable subdirectories below this. Information on
patches and how to fetch and apply them is also available through the
World-Wide Web pages for \hbox{\AIPS}.  As bugs in \RELEASENAME\ are
found, they are simply corrected since \RELEASENAME\ remains under
development.  Corrections and additions are made with a midnight job
rather than with manual patches.  Remember, no matter when you
received your \OLDNAME\ ``tape,'' {\it you must} fetch and install
its patches if you require them.  Also note that we do not expect to
make any more patches to \OLDNAME\@.

The \OLDNAME\ release had only a few important patches announced.
These were:
\begin{enumerate}
\item\ {\tt DBCON} fails combining data sets with different numbers of
       polarizations. 1999-11-23.
\item\ {\tt IMAGR} fails when gridding large fields after gridding
       small fields. 1999-12-14
\item\ {\tt IMAGR} fails when filtering Clean components for large
       fields.  2000-02-04
\end{enumerate}

\section{\AIPS\ Distribution}

By March 30, 2000, 76 sites have taken copies of {\tt 31DEC99} via
ftp.  A few of them have begun taking updates regularly via the
``midnight job.''  We recommend this to all serious \AIPS\ users.

A total of 301 copies of the {\tt 15OCT99} release were distributed to
276 non-NRAO sites.  Of these, 145 were in source code form and 156
were distributed as binary executables. The figures on computers using
\AIPS\ are affected by the percentage of \AIPS\ users that register
with NRAO\@.  Of 276 non-NRAO sites receiving {\tt 15OCT99} only 42
(15\%!!) have registered.  We remind serious \AIPS\ users that
registration is required in order to receive user support.  The first
table below shows the breakdown of how the copies of {\tt 15OCT99}
were distributed and includes both source-code distributions and
binary distributions.  The second table below is based on the
disappointing number of registered installations of {\tt 15OCT99}\@.
It lists total numbers of computers and indicates that the
distribution over operating systems was heavily weighted toward
Solaris with Linux as a distant second.  However, when asked about
``primary'' architecture, 48\%\ of our users answered Linux and 33\%\
answered some flavor of Sun OS\@.  This indicates how far Linux has
penetrated as the system used on the more powerful computers used by
astronomers today.

\begin{center}
\begin{tabular}{|l|r|r|r|r|r|r|} \hline\hline
              &{ftp} & {CDrom} &{8mm} & {4mm} & {ZIP} & {Floppy} \\ \hline
{\tt 15OCT98} & 242   &      71 &   8  &    1  &    0  &       0  \\ \hline
{\tt 15APR99} & 290   &      69 &   0  &    2  &    0  &       0  \\ \hline
{\tt 15OCT99} & 227   &      72 &   0  &    2  &    0  &       0  \\ \hline\hline
\end{tabular}
\end{center}

\begin{center}
\begin{tabular}{|l|r|r|r|r|r|r|r|r|} \hline\hline
{OS} & \texttt{15OCT99} & \texttt{15OCT99} &
             \texttt{15APR99}  & \texttt{15OCT98} & \texttt{15APR98}
           & \texttt{15OCT97} & \texttt{15APR97}  & \texttt{15OCT96} \\
           & {No.} & {\%} & {\%} & {\%} & {\%} & {\%} & {\%} & {\%} \\
\hline
Solaris         &  84 & 56 & 53 & 61 & 66 & 50 & 66 & 46  \\
PC Linux        &  43 & 29 & 27 & 27 & 19 & 23 & 16 & 19  \\
HP-UX           &  10 &  7 &  2 &  5 &  2 &  3 &  6 &  4  \\
SGI             &   9 &  6 & 13 &  1 &  3 &  1 &  1 &  5  \\
SunOS 4         &   2 &  1 &  1 &  1 &  4 & 14 &  5 & 13  \\
Alpha Linux     &   2 &  1 &  0 &  0 &    &    &    &     \\
Dec Alpha       &   1 &  1 &  4 &  3 &  7 &  9 &  6 & 10  \\
IBM /AIX        &   0 &  0 &  0 &  2 &  1 &  0 &  0 &  4  \\
Total           & 151 &    &    &    &    &    &    &     \\
\hline\hline
\end{tabular}
\end{center}

\section{Recent \AIPS\ Memoranda}

The following new memorandum is available from the \AIPS\ home page.

\begin{tabular}{lp{5.8in}}
103 &   Weighting data in AIPS\\
   &    Ketan M. Desai\\
   &    March 21, 2000\\
   &    This memo describes the ``correct'' calculation of weights for
        VLA and VLBA data based primarily on phenomological
        considerations with a light sprinkling of theoretical
        grounding.\\
\end{tabular}

\end{document}
