%-----------------------------------------------------------------------
%;  Copyright (C) 1995-1997
%;  Associated Universities, Inc. Washington DC, USA.
%;
%;  This program is free software; you can redistribute it and/or
%;  modify it under the terms of the GNU General Public License as
%;  published by the Free Software Foundation; either version 2 of
%;  the License, or (at your option) any later version.
%;
%;  This program is distributed in the hope that it will be useful,
%;  but WITHOUT ANY WARRANTY; without even the implied warranty of
%;  MERCHANTABILITY or FITNESS FOR A PARTICULAR PURPOSE.  See the
%;  GNU General Public License for more details.
%;
%;  You should have received a copy of the GNU General Public
%;  License along with this program; if not, write to the Free
%;  Software Foundation, Inc., 675 Massachusetts Ave, Cambridge,
%;  MA 02139, USA.
%;
%;  Correspondence concerning AIPS should be addressed as follows:
%;         Internet email: aipsmail@nrao.edu.
%;         Postal address: AIPS Project Office
%;                         National Radio Astronomy Observatory
%;                         520 Edgemont Road
%;                         Charlottesville, VA 22903-2475 USA
%-----------------------------------------------------------------------
\documentstyle [twoside]{article}
%
\newcommand{\AIPS}{{$\cal AIPS\/$}}
\newcommand{\POPS}{{$\cal POPS\/$}}
\newcommand{\AMark}{AIPSMark$^{(93)}$}
\newcommand{\AMarks}{AIPSMarks$^{(93)}$}
\newcommand{\LMark}{AIPSLoopMark$^{(93)}$}
\newcommand{\LMarks}{AIPSLoopMarks$^{(93)}$}
\newcommand{\AM}{A_m^{(93)}}
\newcommand{\ALM}{AL_m^{(93)}}
\newcommand{\eg}{{\it e.g.},}
\newcommand{\ie}{{\it i.e.},}
\newcommand{\daemon}{d\ae mon}
\newcommand{\boxit}[3]{\vbox{\hrule height#1\hbox{\vrule width#1\kern#2%
\vbox{\kern#2{#3}\kern#2}\kern#2\vrule width#1}\hrule height#1}}
%
\newcommand{\memnum}{98}
\newcommand{\whatmem}{\AIPS\ Memo \memnum}
%\newcommand{\whatmem}{{\bf D R A F T}}
\newcommand{\memtit}{AIPSTerminal for Linux PC's}
\title{
%   \hphantom{Hello World} \\
   \vskip -35pt
%   \fbox{AIPS Memo \memnum} \\
   \fbox{{\large\whatmem}} \\
   \vskip 28pt
   \memtit \\}
\author{ Robert~L.~Millner and Patrick~P.~Murphy\\
National Radio Astronomy Observatory\\Charlottesville, Virginia, USA}

%
\parskip 4mm
\linewidth 6.5in                     % was 6.5
\textwidth 6.5in                     % text width excluding margin 6.5
\textheight 8.91 in                  % was 8.81
\marginparsep 0in
\oddsidemargin .25in                 % EWG from -.25
\evensidemargin -.25in
\topmargin -.5in
\headsep 0.25in
\headheight 0.25in
\parindent 0in
\newcommand{\normalstyle}{\baselineskip 4mm \parskip 2mm \normalsize}
\newcommand{\tablestyle}{\baselineskip 2mm \parskip 1mm \small }
%
%
\begin{document}

\pagestyle{myheadings}
\thispagestyle{empty}

\newcommand{\Rheading}{\whatmem \hfill \memtit \hfill Page~~}
\newcommand{\Lheading}{~~Page \hfill \memtit \hfill \whatmem}
\markboth{\Lheading}{\Rheading}
%
%

\vskip -.5cm
\pretolerance 10000
\listparindent 0cm
\labelsep 0cm
%
%

\vskip -30pt
\maketitle
\vskip -30pt


\normalstyle

%
%\date{November~ 4,~1997}
\vspace{2mm}
\begin{abstract}

There has been some interest in accessing \AIPS\ on a home Linux PC.
Many home machines lack the resources to run \AIPS\ well and people may
not wish to move large amounts of data to and from home.  A small
install package was created for Red Hat Linux which contains a minimal
subset of \AIPS\ to run the TV, Message and Tek servers.  With this
package, a user may run the servers from home, dial into the network and
access \AIPS\ on their workstation in an efficient manner.

\end{abstract}

\section{Introduction}

The AIPSTerminal package is built from the current revision of \AIPS\ and
packaged as a Red Hat RPM file.  The package can be installed anywhere
on the Linux box, either by the superuser or a normal user and occupies
a few hundred kilobytes.  It is useful for providing the ability to
access \AIPS\ from home in a manner that is more efficient than having X
clients constantly push bitmaps across a modem link.  Even though it is
possible to create RPM's for other operating systems, only an
Intel/Linux version was done.  We can provide support if there is demand
for other configurations that run \AIPS\ .

The install package may be retrieved from the \AIPS\ ftp site.  Those
who maintain their own modifications to \AIPS\ may wish to retrieve the
source RPM and build packages of their own.

\section{Installing and Running}

The package runs on whatever system will be compatible with the same
version of \AIPS\ .  Either the {\tt g77} or {\tt f2c} libraries must be
installed.  Red Hat Linux version 4.2 is the current base for
compilation though it should operate in any recent Linux environment
(\ie\ Debian 1.3, Caldera 1.1).  Red Hat's RPM is needed to handle the
package.

The \AIPS\ manager at your site will need to have additional guest image
catalog files to handle AIPSTerminal sessions.  Instructions for doing
this are covered in the installation notes for \AIPS\ .

\subsection{Retrieve the Package}

The package can be retrieved from {\tt ftp://aips.cv.nrao.edu/pub/aips/AIPSTerminal}.  The install package is
named {\tt AIPSTerminal-VERSION-REV.i386.rpm}, where {\tt VERSION} is
the corresponding \AIPS\ version and {\tt REV} is the packaging
revision.  There may be incompatibilities between versions so it is best
to choose the one which matches the \AIPS\ version you use.  Packaging
revisions will contain small bug-fixes but no major code changes.  The
{\tt AIPSTerminal-VERSION-REV.src.rpm} package is the skeleton to build
your own.

\subsection{Installing}

To install the package, become root and use RPM (\ie\ {\tt rpm -U
AIPSTerminal-15OCT97-1.0.i386.rpm}).  The package will install by
default into {\tt /usr/local/AIPSTerminal}.  You may install it
elsewhere by using the {\tt --prefix} option (\ie\ {\tt rpm --prefix
/opt -U AIPSTerminal-15OCT97-1.0.i386.rpm}).  You may also install the
package without becoming root if you have your own personal installation
of RPM by using the {\tt --prefix} option.  Documentation for the
package is located in {\tt /usr/local/AIPSTerminal} instead of {\tt
/usr/doc}.

\subsection{Running}

The package contains a script called {\tt AIPSTerminal} which will set
up the local \AIPS\ environment and run the servers .  You may wish to
copy it to {\tt /usr/local/bin} or somewhere else in your path.  Running
the script is the simplest way to start the servers.  They can also be
started manually by setting up the local \AIPS\ environment using the
standard method with {\tt /usr/local/AIPSTerminal} as your {\tt
AIPS\_ROOT}.

\section{Building a Customized Version}

It is unlikely that you will need to build your own package.  I
recommended that you become familiar with using RPM to build packages
first before building this one.

After installing the source RPM, examine the {\tt AIPSTerminal.spec}
file for version changes.  You may need to select a different \AIPS\
version by changing a few places at the top of the file.  You will
certainly need to change the value of {\tt BuildRoot}.  I tried to keep
the {\tt AIPSTerminal.spec} file more or less version independent so the
changes should be small.  Some of the patches are version dependent and
will need to be modified.  The \AIPS\ versioning scheme doesn't lend
itself well to sorting.  Because of this, I included a serial number for
each revision which is based on the release version.  Check the {\tt
serial} tag to be sure that it matches the version (\ie\ {\tt 15APR98}
becomes {\tt 19980415} ).

The package preparation stage will require an existing \AIPS\
installation to copy files from.  Setup the \AIPS\ environment for the
version you are building from before trying to make the package.  The
{\tt prep} script will check the file {\tt AIPS.filelist} for files and
directories to copy.  This file will also be parsed and fed to RPM to
generate the final package.  The attribute commands are there to allow
you to make the package without needing to become root.  Any files that
are created by the procedure instead of copied from the \AIPS\
distribution are listed in the {\tt files} directive.

RPM will use the {\tt AIPSROOT.DEFINE} script to set a proper value for
{\tt AIPS\_ROOT} in all of the package's scripts when it is installed.
If you add a script that has its own {\tt AIPS\_ROOT}, make sure that it
is modified by the package.

A common scheme for marking locally modified revisions is to append your
initials or your organization's acronym to the revision.  Please use
this scheme or one like it if you redistribute the package.

\section{Contact}

Please refer questions about the package itself to Robert Millner,
rmillner@nrao.edu.  Questions pertaining to \AIPS\ should be directed to
the Designated Aip, daip@nrao.edu.  For more information on Red Hat
Linux or RPM, see {\tt http://www.redhat.com}.

\end{document}

