%-----------------------------------------------------------------------
%;  Copyright (C) 1995
%;  Associated Universities, Inc. Washington DC, USA.
%;
%;  This program is free software; you can redistribute it and/or
%;  modify it under the terms of the GNU General Public License as
%;  published by the Free Software Foundation; either version 2 of
%;  the License, or (at your option) any later version.
%;
%;  This program is distributed in the hope that it will be useful,
%;  but WITHOUT ANY WARRANTY; without even the implied warranty of
%;  MERCHANTABILITY or FITNESS FOR A PARTICULAR PURPOSE.  See the
%;  GNU General Public License for more details.
%;
%;  You should have received a copy of the GNU General Public
%;  License along with this program; if not, write to the Free
%;  Software Foundation, Inc., 675 Massachusetts Ave, Cambridge,
%;  MA 02139, USA.
%;
%;  Correspondence concerning AIPS should be addressed as follows:
%;          Internet email: aipsmail@nrao.edu.
%;          Postal address: AIPS Project Office
%;                          National Radio Astronomy Observatory
%;                          520 Edgemont Road
%;                          Charlottesville, VA 22903-2475 USA
%-----------------------------------------------------------------------
%Body of \AIPS\ Letter for 15 October 1992

\documentstyle [twoside]{article}

\newcommand{\AIPRELEASE}{October 15, 1992}
\newcommand{\AIPVOLUME}{Volume XII}
\newcommand{\AIPNUMBER}{Number 2}
\newcommand{\RELEASENAME}{{\tt 15OCT92}}

%macros and title page format for the \AIPS\ letter.
\input LETMAC92A.TEX

\newcommand{\MYSpace}{-11pt}

\normalstyle
\section{AIPSLETTER Number, Personnel Changes}

This \AIPSLETTER\ is the second of two letters in volume XII of the
series.  The letter for the {\tt 15APR92} release of \AIPS\ was
erroneously labeled volume X, number 1.

%\section{Personnel}

Gustaaf van Moorsel has rejoined the \AIPS\ group, this time stationed
in Socorro, \hbox{NM}.  His considerable experience will be a great
help in providing user support and on-the-spot debugging at the AOC.

Darrell Schiebel has joined NRAO in Charlottesville to work on the
\AIPTOO\ project.

Ger van Diepen is visiting Charlottesville for one year from Dwingeloo
also to work on the \AIPTOO\ project.

\section{15OCT92 Available}

The \RELEASENAME\ version of \AIPS\ is now available.  It contains
significant improvements in ``television'' displays for workstations
and in various tasks of particular interest to the VLBI community.
These improvements will be described in some detail below.

The last VMS machine directly accessible to the NRAO \AIPS\ group has
now been dismantled.  This means that we cannot write a VMS-{\tt
BACKUP}-format release tape.  VMS users can get the code via \ftp\ in
Unix compressed \tar\ format or, by special arrangement, in plain text.
Tapes in the Unix compressed \tar\ format are available in several
shapes (half-inch reels, DATs, Exabytes).  A ``compressed-FITS-text''
format could be made available on tape if there is sufficient demand.
The program to read such a tape was never fully developed for VMS,
however.  It lacks one easy and two less easy Z routines for VMS.  We
could write the remaining code, but we have no way to test it.
Nonetheless, we will attempt this route if our users seem to need it.

So far as we know, \AIPS\ will still work on FPS array processors.
However, we will no longer support those devices, and may develop code
using Q routines that have no FPS version.

We had hoped to offer support for several additional flavors of Unix
in this release.  That hope will now have to be placed on the next
release.  Of particular urgency is Sun's shift to the Bell-UNIX-based
operating system they call Solaris.  Hewlett Packard (HP-UX) and
Silicon Graphics versions are also anticipated.  Both of these will
require assistance from non-NRAO sites.

\clearpage

\section{Calibration Improvements in 15OCT92}

\subsection{Changes to Calibration Tables and Time Systems}

   In the \RELEASENAME\  release there were significant changes made
to the contents of the calibration tables and to the time system used
for \uv\ data.  These changes were driven by the needs of the VLBA and
result in smaller sizes for CL tables.

   While forward compatibility is not a problem, these changes have
resulted in some incompatibility between \RELEASENAME\  (and later)
versions of \AIPS\ with previous releases.  Due to a bug in the older
software, the new CL tables cannot be applied in systems older than
\hbox{{\tt 15APR92}}.  New CL tables can be applied by older versions
if a patch is made to {\tt \$APLSUB/CLREFM.FOR} (see the Patch
Distribution article below). Alternatively, VLA data can be
recalibrated in older systems by deleting the new-format CL and SN
tables and creating a new, old-format CL table using \hbox{{\tt
INDXR}}.  The final {\tt CALIB} and {\tt CLCAL} (and {\tt CLCOR} for
polarization data) steps of the calibration need to be repeated.  Your
editing and determination of secondary calibrator fluxes (FG and SU
files) are not affected by these changes.

   The CL and SN tables were changed to include a multi-band delay
column for each polarization.  This multi-band delay is not a part of
the residual calibration but is kept for model accountability.  Also,
the phase for each IF was changed to be at the reference channel for
that IF rather than at the over-all reference frequency.  This means
that this phase is no longer a function of the single band delay.

   The system temperature values were removed from the CL table.
Separate TY tables have been implemented to contain system-temperature
information.

   The model accountability in the CL table has been changed.  The
geometric portion of the model delay is given as a time series
polynomial (column {\tt GEODLY}) and columns {\tt GEOPHA} and {\tt
GEORAT} have been removed.  There is a single atmospheric group delay
and first time derivative per antenna/time ({\tt ATMOS} and {\tt
DATMOS}).  Each polarization has a separate clock epoch and rate error
({\tt CLOCK}{\it n}, {\tt DCLOCK}{\it n}).  Each polarization also has
a dispersive term ({\tt DISP}{\it n} and {\tt DDISP}{\it n}) which is
the phase delay at a wavelength of 1 meter and scaled as wavelength
squared.

   The time system associated with \AIPS\ time labels for \uv\ data
was simply assumed to be \hbox{IAT}.  Our new scheme will allow any
time system with a fixed offset from \hbox{UTC}.  This offset is
carried in the AN table along with a character string label specifying
the time system.

\subsection{VLBI Polarization Calibration and Imaging}

   The \RELEASENAME\ release contains a number of enhancements to the
VLBI polarization calibration and imaging capabilities of
\hbox{\AIPS}.  \AIPS\ Memo Number 79 describes several techniques for
polarization calibration and suggests a method of calibration and
imaging of VLBI polarimetric data.

   A procedure called {\tt CROSSPOL} was implemented to simplify the
calibration of the cross-polarized delay and IF-peculiar
right-minus-left (R--L) phase difference.  This procedure is defined
by entering {\tt RUN CROSSPOL}, and it has both inputs and explain
documentation.  The end result of the procedure is an SN table
containing the appropriate cross-polarized delay and phase corrections
to be applied to a CL table.

   {\tt CROSSPOL} begins by copying a user-selected subset of his/her
data, usually a single baseline observation of one or more strong
calibrator sources.  A parallel-hand fringe fit is run on this subset
to correct the R and L delays and phases.  Then, the R and L
polarizations are swapped for one of the antennas using task {\tt
SWPOL} and {\tt FRING} is run again to determine cross-polarized
single- and multi-band delays (and any IF-dependent peculiar R-L phase
differences).  Finally task {\tt POLSN} munges the output of this last
{\tt FRING}, producing an SN table which is copied back to the
original data (where it can be applied using \hbox{{\tt CLCAL}}.

   Imaging of data with asymmetric coverage (some data samples with
one, but not both, cross-polarizations) is supported in this release.
Procedure {\tt CXPOLN} simplifies the process, using a combination of
the new task {\tt UVPOL} and the old familiar {\tt MX} to produce
complex dirty images and beams.  New task {\tt CXCLN} will then do a
complex H\"ogbom CLEAN deconvolution of the Q and U images.  This
procedure is defined by entering {\tt RUN CXPOLN}, and it has both
inputs and explain documentation.

\subsection{VLBI data input}

A number of changes as well as bug fixes were made to {\tt MK3IN}, our
task to read Haystack-format MkIII VLBI correlator tapes.  {\tt MK3IN}
now offers the option to split upper and lower sidebands into separate
IFs.  In order to support this, the maximum number of IFs allowed in
\AIPS\ was raised to 28.  {\tt MK3IN} now shifts all frequency
channels by one-half of a frequency cell to avoid the zero frequency
and no channels are automatically blanked.  This produces 8 frequency
channels for double-sideband data.  {\tt AIPS} verb {\tt TELL} can now
be used to terminate {\tt MK3IN} gracefully.

\subsection{Data editing}

The baseline-oriented interactive editing task {\tt IBLED} was
substantially rewritten to be considerably faster and to handle the
flag command table, more than one polarization, multiple frames in the
data, the master file, workstation windows, and numerous other matters
correctly.  It now also offers the option to flag on
``decorrelation'', defined as the ratio of the vector- to the
scalar-averaged amplitudes, and on the ratio of any 2 IFs present in
the data.  It also offers the one- versus all-source option for each
flag prepared and will also determine a ``best'' time interval by
default.  {\tt TVFLG} and {\tt SPFLG} were also changed to pick up some
of the better changes to {\tt IBLED}, including on-line help for all
menu items, cataloging even temporary master files, more flexible and
correct handling of Stokes flag masks, and more efficient application
of flagging to single-source files.  {\tt SPFLG} was given a new menu
option to show the next baseline.

\subsection{Total electron content data for {\tt FARAD}}

The \AIPS\ task {\tt FARAD}, which corrects for ionospheric Faraday
rotation requires total electron content data for the period of
observation.  These data are distributed with \AIPS\ in the directory
with the logical name \hbox{{\tt \$AIPSIONS}}.  The data files follow
the naming convention {\tt TEC{\it s}.{\it yy}} where {\it s} is a
station code and {\it yy} is the last two digits of the year.
Currently data are only available from Boulder (station code B), the
closest ionosonde to the VLA, and are complete up to May 1992.

The data are also available from baboon.cv.nrao.edu by anonymous ftp
(from area {\tt pub/aips/TEXT/IONS}).  We will announce the
availability of new ionospheric data on bananas and the
alt.sci.astro.aips newsgroup as soon as we receive updates from
Boulder.

\section{Additional Improvements in 15OCT92}

\subsection{TV servers and programs}

Almost all TV applications verbs and tasks were changed to take
full advantage of the new asynchronous capabilities of {\tt XAS} (see
below).  As a result, displays change less smoothly, but much faster.
Verbs {\tt GREAD} and {\tt GWRITE} were added to allow users to read
and set the colors of the TV graphics overlay planes.  Two new tasks
were created to allow nearly full color displays even on workstations
(see \AIPS\ Memo Number 82 for additional details).  One of these,
{\tt TVHUI}, does a TV display, with optional output file, in which
one image controls intensity, a second image controls hue, and a
third, optional, image controls saturation.  All levels of a single TV
memory are used and the results are surprisingly good (at least
without the saturation image) on workstations with only about 200
levels.  This task should be used with spectral-line moment images,
continuum polarization images, and numerous other possibilities.  The
second task is called {\tt TVRGB} and is used to display independent
red, green, and blue images using a single TV memory.  A
``median-cut'' algorithm is used in the histogram of the images colors
to optimize use of the limited number of levels available on the TV.
This task can be used to display ``true-color'' images and outputs
from {\tt TVHUI} and {\tt RGBMP}, or to compare images such as
comparing optical and radio images of a source or even the same image
with significantly different scalings.  Both tasks offer a menu of
interactive enhancement and display options.

A number of changes have been made in the {\tt 15OCT92} version of the
\AIPS\ television driver {\tt XAS}, resulting in considerably improved
performance.  First, the {\tt DISPLAY} variable was changed from {\it
host}{\tt :0} to simply {\tt :0}, where possible.  This should prompt
the X server to use Unix sockets rather than Internet sockets, with
some improvement in performance.  Second, the ``blit'' of the image
from {\tt XAS}'s memory to the display was changed to be as large as
possible on each display update.  Previously, only a row at a time was
blitted when the image was zoomed and/or contained graphics overlays.
Third, the {\tt XAS} memory was changed to use, optionally, the X
extension called ``shared memory.''  This greatly improves blit speed
after an initial overhead to synchronize the memories.  Fourth, the
application code was provided with the option to ask {\tt XAS} to
delay updating the display until instructed to do so.  This allows
multiple graphics planes to be turned on with a single screen update,
a full image to be loaded with a single blit to the display rather
than one blit per row, multiple line segments of a plot to be drawn
with a single blit to the display rather than very many small blits,
and so forth.  This option, implemented with subroutine {\tt YHOLD},
is dangerous in that it requires considerable care on the part of the
application programmer to make certain the the display is brought up
to date whenever required.  As some protection against programmer
error, subroutine {\tt TVCLOS} forces synchronization.  Also the new
{\tt XAS} allows the user to set (via his or her {\tt .Xdefaults}
file) a maximum number of commands to be done asynchronously before
{\tt XAS} itself forces an update of the screen.

The older television drivers, {\tt XVSS} and {\tt SSS}, were changed
to allow the {\tt YHOLD} command and {\tt SSS} even supports it.  A
bug in the screen addressing for all three drivers was corrected.
This caused too many pixels to be sent to the display frequently, but
no serious error.  The output buffer in {\tt XAS}, which was too small
to support reading a full image row of the TV, was increased.  The
new {\tt XAS} was changed to allow users to set the maximum image
intensity used.  This allows one to choose between high dynamic range
(up to 237 levels) and the distraction caused by using different
workstation color tables when the cursor is inside and outside the
\AIPS\ TV window.  The {\tt XAS} help file was also improved.

\subsection{Tape programs}

The performance of Berkeley sockets over Ethernet connections depends
critically on the size of the data blocks being transmitted.  If they
are too small (requiring many connections) or too large (requiring
subdivision into multiple segments), then the throughput is reduced by
factors of two or more.  For the \RELEASENAME\ version, we changed the
remote tape open and I/O routines to raise the assigned buffer sizes
to the maximum size we could ever need (29000 bytes).  This was the
simplest of several solutions to give optimal performance, \ie\
performance equal to that for directly-connected tape devices.  See
\AIPS\ Memo Number 80 for details.

A number of other changes were made to tape programs.  {\tt FITTP} was
given the ability to write out the extra keywords now recorded with
\AIPS\ image and \uv\ headers.  A special form is used so that \AIPS\
FITS readers ({\tt FITLD}, {\tt IMLOD}, {\tt UVLOD}) can now read the
parameters and put them back into the \AIPS\ headers' keywords.  A
``new'' tape operation was created to allow {\tt FITTP} to write
anywhere on a tape.  Previously, at least on Exabytes, it was
restricted to writing either at the beginning of tape or at the
end-of-information.   {\tt BAKLD} and {\tt BAKTP} were made to work
better and to work on IBMs.  Several minor inadequacies were corrected
in {\tt FILLM}.  Serious bugs in writing the frequency into the
antenna file and in setting the final file size were corrected.  In
addition, support for ``on-line'' uses of {\tt FILLM} and for {\tt
AIPS}' {\tt SHOW} and {\tt TELL} verbs was added.  Additionally, the
procedure which starts the remote tape d\ae mons was made more
powerful, the tape Z routines were given more explanatory error
messages, bugs reading labeled tapes were corrected, and mounting of
remote tapes on Convexes was made somewhat more reliable.

\subsection{New programs for optical data}

Several new tasks were added to \RELEASENAME\ \AIPS\ primarily to
assist in the reduction of optical data.  New task {\tt STFND} uses a
threshold-area algorithm for finding stars.  New task {\tt GSCAT}
reads the STScI Guide Star tapes to find stars in the neighborhood of
a specified coordinate.  New task {\tt GSTAR} converts the files
created by {\tt GSCAT} to \AIPS\ ST tables to be used for plotting,
for locating optical counterparts, and/or for transforming celestial
and image coordinates.  New task {\tt STRAN} attempts to identify
stars on optical images automatically using the list of stars in an ST
file.  Also, the median-window-filter task, {\tt MWFLT}, was improved
to do either a high- or a low-pass filtering.

\clearpage

\subsection{Miscellaneous changes}

Other corrections and improvements made to \AIPS\ for the
\RELEASENAME\ release include:
\begin{description}
\myitem{ACFIT} Added the option to write the baseline-corrected,
   total-power spectra to an output file.
\myitem{CLCOR} Added option to correct blanked phases using any valid
   ones with specified relationships between the IFs.
\myitem{DBCON} Changed to correct the $u,v,w$ to reference channel
   one, allowing correct imaging of multi-channel continuum data which
   are first {\tt SPLIT} and then {\tt DBCON}ed.  Also fixed to handle
   more channels and/or IFs.
\myitem{HYB} New procedure to simplify ``hybrid mapping'' using {\tt
   CALIB} and {\tt MX} with user editing of the CC file and a contour
   plot.
\myitem{LISTR} Corrected bug that caused the wrong columns to be
   printed when examining tables, corrected bugs handling times in
   tables and multi-band delays, and changed it to examine the full
   data set to determine the scaling for {\tt GAIN} listings.
\myitem{LWPLA} Changed to produce Postscript suitable for
   encapsulating and to offer additional user controls over plot
   orientation, fonts, and the like.  Corrected handling of vectors
   that go partly off the page in all plot programs.  The error was
   serious only for {\tt LWPLA} apparently.
\myitem{POSSM} Corrected its handling of flagged data and of times,
   changed it to plot a line at 0 in phase, real, and imaginary plots,
   changed it to use the full TV screen when {\tt DOTV} is true, and
   to be properly interactive when doing multiple plots to the
   \hbox{TV}.  Corrected the {\tt EXTLIST} display of {\tt POSSM} plot
   files as well.
\myitem{UVCOP} Corrected the handling of the reference frequency in
   the antennas files.
\myitem{UVPRT} New task to print \uv\ data in the format of {\tt
   PRTUV}, but with application of all standard calibrations and a
   special mode for holography data.
\myitem{VBPLT} Corrected to use data from the desired subarray and CC
   file, to apply baseline-dependent calibrations, and to show the CC
   file used.
\end{description}

\subsection{Miscellaneous changes for programmers}

Several rather technical but useful changes were made to this release
which are primarily of interest to programmers.
\begin{description}
\myitem{ZACTV9} Rewrote the task activation to use POSIX standards, to
    create the task as a grandchild process thereby avoiding the
    accumulation of zombie processes, and to avoid creating link files
    in load-module areas when tasks are not being debugged.
\myitem{ZABOR2} Took advantage of the change to {\tt ZACTV9} to allow
    tasks being run in the presence of the debugger to trap exceptions
    normally unless they are being run in debug mode and to allow all
    tasks to trap floating-point exceptions.
\myitem{COMLNK} Changed compilation procedures to use list files to
    determine which programs should be optimized at which levels on
    which hosts.
\myitem{TSKHLP} New subroutine to allow run-time help information from
    inside tasks read from files named {\tt \$HLPFIL/HLP{\it
    taskname}.HLP} and displayed with \hbox{{\tt MSGWRT}}.
\myitem{APLNOT} Cleaned up this area deleting unused subroutines which
    caused unresolved external reference messages to appear when
    linking with shared libraries.  Also changed several tasks to
    provide the large buffers in the tasks rather than declaring them
    in the subroutines where they are actually needed.  Large buffers
    declared in subroutines appear in all load modules, whether they
    are used or not, when shared libraries are used.  Some large
    buffers and a few unavoidable unresolved references still remain.
\myitem{TEXT} The \AIPS\ text areas containing ionospheric data, TV
    OFM tables, and publications (\Cookbook, \AIPSLETTER s, \AIPS\
    Memos, old {\tt CHANGE.DOC} files) will be included on the release
    tape this time.  They were inadvertently omitted in recent years.
\end{description}

%\clearpage
\section{Patch Distribution}

Since \AIPS\ is now released only semi-annually, we have developed a
method of distributing important bug fixes and improvements via {\it
anonymous} \ftp\ on the NRAO Cpu {\tt baboon} (192.33.115.103).
Documentation about patches to a release is placed in the
anonymous-ftp area {\tt pub/aips/}{\it release-name} and the code is
placed in suitable subdirectories below this.  Reports of significant
bugs in {\tt 15APR92} \AIPS\ have been relatively few; however, the
documentation file {\tt pub/aips/15APR91/README.15APR92} mentions the
following items:
\begin{description}
\myitem{INSTEP1} This initial installation procedure could clobber
     existing files if restarted.
\myitem{START\_AIPS} \hskip 0.8cm {\tt AIPS} start-up script failed
     when there is only one printer at a site or when there is a
     numeric digit in the host name of a TV server.
\myitem{ZVTP*.C} {\tt \$APLBERK} versions of remote tape open and I/O
     routines altered for significantly better performance.
\myitem{MX} Spectral-line data with poor phase calibration may fail to
     converge in cleaning due to an error in the test for divergence.
\myitem{XTRAN} Raised the size of the images and the number of stars
     that can be used to transform an (optical) image to standard
     coordinates.
\end{description}
Note that we do not revise the original {\tt 15APR92} tape or \tar\
files for these patches.  No matter when you received your {\tt
15APR92} tape, you must fetch and install these patches if you require
them.  See the {\tt 15APR92} \AIPSLETTER\ for an example of how to
fetch and apply a patch.

One more patch has been added to the {\tt 15APR91} area in addition to
those listed in the previous \AIPSLETTER.  This is
\begin{description}
\myitem{SPLIT} An error in {\tt \$APLNOT/CLREFM.FOR} prevented modern
     CL tables from being used by older versions.
\end{description}

As bugs to \RELEASENAME\ are found, the patches will be placed in the
\ftp\ area for \RELEASENAME.

\section{VLBI Summer School Announced}

The NRAO will hold a summer school in Socorro, NM June 23--30, 1993.
Continuing the series of NRAO summer schools, this school will
emphasize VLBI theory and techniques.  In particular, we wish to
provide students and other future users with an opportunity to become
familiar with the capabilities and user aspects of the VLBA. The
program will include lectures and demonstrations by NRAO staff members
and by several invited speakers from the VLBI community.

If you are interested in attending the summer school you should contact
Terry Romero at:
\vspace{\MYSpace}
\begin{center}
\begin{tabular}{l}
   NRAO Summer School on VLBI Techniques \\
   National Radio Astronomy Observatory \\
   P. O. Box 0  \\
   Socorro, NM 87801 \\
   \\
   or \\
   \\
   email: tromero@nrao.edu \\
   fax: (505) 835-7027 \\
\end{tabular}
\end{center}

\clearpage

\section{Improved Tape Support Planned}

We are planning to put greater support for a variety of tape devices
into the \AIPS\ tape Z routines.  We would very much like to hear from
all sites using tape devices on workstations except those using
ordinary Exabytes with standard SCSI controller.  Please let us
know the manufacturer, model, controller, software driver, and any
changes you have had to make to the \AIPS\ Z routines for your tape
devices.  Please address your comments to eallen at the addresses in
the masthead.  He will forward them appropriately.

\section{Latest \AIPS\ Memos}

Below is a list of the latest \AIPS\ Memos.
\begin{center}
\begin{tabular}{ccl}
\hline
MEMO  &        DATE   & TITLE and AUTHOR  \\
\hline\hline
 *76 & 91/11/27 & Summary of \AIPS\ Continuum UV-data Calibration, \\
     &          & from VLA Archive Tape to UV FITS Tape \\
     &          & (supersedes memo 68) \\
     &          & \qquad Glen Langston, NRAO \\
  77 & 92/09/03 & Summary of DDT Accuracy Results \\
     &          & \qquad Ernest Allen \&\ Glen Langston, NRAO \\
  78 & 92/06/01 & Object-Oriented Programming in AIPS Fortran \\
     &          & \qquad W. D. Cotton, NRAO \\
  79 & 92/06/09 & Polarization Calibration of VLBI Data \\
     &          & \qquad W. D. Cotton, NRAO \\
  80 & 92/06/30 & Remote Tapes in AIPS \\
     &          & \qquad Eric W. Greisen, NRAO \\
  81 & 92/08/26 & Tape and TV Performance in AIPS \\
     &          & \qquad Eric W. Greisen, NRAO \\
  82 & 92/09/24 & Replacing the Convexes --- New Color Algorithms in
                    AIPS \\
     &          & \qquad Eric W. Greisen, NRAO \\
\hline
\end{tabular}
\end{center}

To order, use an \AIPS\ order form or e-mail your request to
aipsmail@nrao.edu.  Memos can also be gotten via anonymous \ftp,
except for figures which may be missing in those denoted by an asterisk.

To use \ftp\ to retrieve the memos:
\begin{description}
\item{ 1.} {\tt ftp baboon.cv.nrao.edu}  or  192.33.115.103
\item{ 2.} Login under user name anonymous and use your e-mail address
           as a password.
\item{ 3.} {\tt cd pub/aips/TEXT/PUBL}
\item{ 4.} Read {\tt AAAREADME} for more information.
\item{ 5.} Read {\tt AIPSMEMO.LIST} for a full list of \AIPS\ Memos.
\end{description}

\AIPS\ Memos from Number 69 through 82 are present in this area as
well as a few of the earlier ones.  All are avaiable in paper form
from Ernie Allen at the addresses in the masthead.  Note that the
anonymous ftp areas for memos, the \Cookbook, and other text files
have been changed to parallel the areas in the main \AIPS\ directory
tree.

\clearpage

\section{Computing at the NRAO}

The computing environment at the NRAO, both in Charlottesville and in
Socorro, has been changing fairly rapidly.  All of the old VAX 780s
with their FPS array processors have been dismantled and carted away.
The Convex C-1 computers will be phased out soon.  We anticipate that
the Charlottesville C-1 and one of the two C-1s at the AOC will be
turned off around January 1, 1993.  The other C-1 will be operated for
some period of time without a maintenance contract, perhaps up to six
months.  This will affect \AIPS\ users since we will no longer have a
Convex cpu to test Convex-specific matters, we will no longer have any
hardware TV devices (\eg\ IIS Models 70 and IVAS), and we will be
operating with rather fewer half-inch tape drives.

To replace the Convexes, NRAO has recently purchased five IBM
RS/6000 Model 560 workstations, each with 64 Mbytes of memory, an
Exabyte tape, and 3 Gbytes of local disk.  One of these is in
Charlottesville and the other four are at the \hbox{AOC}.  These
workstations achieved a rating of 2.56 \AIPS marks, where a Convex C-1
achives about 1.0.  \AIPS marks are based on the total time to run
the large {\tt DDT} test, as 5000 divided by this total time in
seconds minus 60\%\ of the real time of \hbox{{\tt ASCAL}}.  In
addition, the NRAO has bought one large workstation for data
visualization for each site.  These are also IBM RS/6000 Model 560s,
but equipped with a Model 7235 POWERgraphics GTO subsystem plus 256
Mbytes of memory and 3 Gbytes of local disk.  Both are supplied with
the AVS data visualization software package.

The ``basic'' workstations are already in routine use for data
reduction with \AIPS\ and other software.  However, we are still
trying to decide how best to use the visualization stations.  To
discuss this, we are holding a meeting October 19--20 involving people
from both sites and from at least one non-NRAO site (University of
Illinois).

We have also purchased a significant number of new tape devices,
primarily Exabytes and DATs.  In Charlottesville, there are now six
tape devices on the IBM RS/6000 Model 530 called {\tt lemur}, two each
of half-inch, Exabyte, and DAT types.  These replace the six half-inch
drives of the VAX and Convex.  A public Sun is also equipped with both
an Exabyte and a DAT drive.  At the VLA site, a Sun has been purchased
to copy half-inch data tapes from the on-line system to Exabyte tapes
for export from the site.  The full VLA archive is being copied from
(old) half-inch tapes to Exabytes.  The AOC anticipates having three
half-inch drives on Sun workstations for this project and for reading
half-inch tapes generally.  And all of the public workstations at the
AOC have an Exabyte drive and many will also have a DAT drive.  These
replace the eight high-speed half-inch drives currently on the AOC
Convexes.

All of the IBM workstations in Charlottesville (3) and the AOC (6)
plus one Sun IPX in Charlottesville and seven Sun IPX's at the AOC are
for public, but assigned, use.  All public workstations have an
Exabyte tape drive and 2 or more Gbytes of local disk.  Visitors
should call John Spargo at the AOC and Jim Condon in Charlottesville
well in advance ($\geq 2$ weeks) to arrange for use of these
workstations.

At the VLA site, there is now a Sun workstation called {\tt miranda}
for visiting observers.  A near-real-time version of {\tt FILLM} is
available to fill data into \AIPS\ as it arrives.  An Exabyte tape
drive is also available on {\tt miranda} to export data in FITS format
for further reduction at the AOC or elsewhere.  To sign up for {\tt
miranda} and reserve disk space (data files will have a lifetime of
three days unless other advance arrangements are made), contact George
Martin at the AOC at least one working day before the observing run.

\section{And a Brag}

On May 6, 1992, the VLBA Correlator obtained its first astronomical
fringes using the W3(OH) masers at 18-cm wavelength.  Within 12 hours,
these data were read into \AIPS\ 2000 miles away.  The data were
promptly sorted and plots produced of both the visibility functions
and the spectra.  {\tt AIPS}' verbs even allowed the few minor rough
spots in this data transfer to be covered over so that correctly
annotated plots were produced.


\end{document}
