%-----------------------------------------------------------------------
%;  Copyright (C) 2008
%;  Associated Universities, Inc. Washington DC, USA.
%;
%;  This program is free software; you can redistribute it and/or
%;  modify it under the terms of the GNU General Public License as
%;  published by the Free Software Foundation; either version 2 of
%;  the License, or (at your option) any later version.
%;
%;  This program is distributed in the hope that it will be useful,
%;  but WITHOUT ANY WARRANTY; without even the implied warranty of
%;  MERCHANTABILITY or FITNESS FOR A PARTICULAR PURPOSE.  See the
%;  GNU General Public License for more details.
%;
%;  You should have received a copy of the GNU General Public
%;  License along with this program; if not, write to the Free
%;  Software Foundation, Inc., 675 Massachusetts Ave, Cambridge,
%;  MA 02139, USA.
%;
%;  Correspondence concerning AIPS should be addressed as follows:
%;          Internet email: aipsmail@nrao.edu.
%;          Postal address: AIPS Project Office
%;                          National Radio Astronomy Observatory
%;                          520 Edgemont Road
%;                          Charlottesville, VA 22903-2475 USA
%-----------------------------------------------------------------------
%Body of intermediate AIPSletter for 31 December 2006

\documentclass[twoside]{article}
\usepackage{graphics}

\newcommand{\AIPRELEASE}{June 30, 2008}
\newcommand{\AIPVOLUME}{Volume XXVIII}
\newcommand{\AIPNUMBER}{Number 1}
\newcommand{\RELEASENAME}{{\tt 31DEC08}}
\newcommand{\NEWNAME}{{\tt 31DEC08}}
\newcommand{\OLDNAME}{{\tt 31DEC07}}

%macros and title page format for the \AIPS\ letter.
\input LET98.MAC
%\input psfig

\newcommand{\MYSpace}{-11pt}

\normalstyle

\section{General developments in \AIPS}

\subsection{FILLM}

{\tt 31DEC06} contains a revision of {\tt FILLM} which is essential to
support the new data form that has been produced by the VLA since late
June 2007.  However, a bug affecting the scaling of cross-hand data in
the new form was not corrected in {\tt 31DEC06}\@.  Therefore, VLA
users will have to upgrade their copy of \AIPS\ to \RELEASENAME\ or
\OLDNAME\ to read such data correctly.  There are additional changes
only in the \RELEASENAME\ version of {\tt FILLM}, which may be
important to current observers.

\subsection{Current and future releases}

We have formal \AIPS\ releases on an annual basis.  While all
architectures can do a full installation from the source files,
Linux, Solaris, and MacIntosh OS/X (PPC and Intel) systems may install
binary versions of recent releases.  The last release is called
\OLDNAME; \RELEASENAME\ remains under active development.  You may
fetch and install a copy of this version at any time using {\it
anonymous} {\tt ftp} for source-only copies and {\tt rsync} for binary
copies.  This \Aipsletter\ is intended to advise you of improvements
to date in \RELEASENAME\@. Having fetched \RELEASENAME, you may
update your installation whenever you want by running the so-called
``Midnight Job'' (MNJ) which copies and compiles the code selectively
based on the changes and compilations we have done.  The MNJ will also
update sites that have done a binary installation.  There is a guide
to the install script and an \AIPS\ Manager FAQ page on the \AIPS\ web
site.

The MNJ serves up \AIPS\ incrementally using the Unix tool {\tt cvs}
running with anonymous ftp.  The binary MNJ also uses the tool {\tt
rsync} as does the binary installation.  Linux sites will almost
certainly have {\tt cvs} installed; other sites may have installed it
along with other GNU tools.  Secondary MNJs will still be possible
using {\tt ssh} or {\tt rcp} or NFS as with previous releases.  We
have found that {\tt cvs} works very well, although it has one quirk.
If a site modifies a file locally, but in an \AIPS-standard directory,
{\tt cvs} will detect the modification and attempt to reconcile the
local version with the NRAO-supplied version.  This usually produces a
file that will not compile or run as intended.

\AIPS\ is now copyright \copyright\ 1995 through 2008 by Associated
Universities, Inc., NRAO's parent corporation, but may be made freely
available under the terms of the Free Software Foundation's General
Public License (GPL)\@.  This means that User Agreements are no longer
required, that \AIPS\ may be obtained via anonymous ftp without
contacting NRAO, and that the software may be redistributed (and/or
modified), under certain conditions.  The full text of the GPL can be
found in the \texttt{15JUL95} \Aipsletter, in each copy of \AIPS\
releases, and on the web at {\tt
  http://www.aips.nrao.edu/COPYING}.
\vfill\eject
\section{Patch Distribution for \OLDNAME}

Important bug fixes and selected improvements in \OLDNAME\ can be
downloaded via the Web beginning at:

\begin{center}
\vskip -10pt
{\tt http://www.aoc.nrao.edu/aips/patch.html}
\vskip -10pt
\end{center}

Alternatively one can use {\it anonymous} \ftp\ to the NRAO server
{\tt ftp.aoc.nrao.edu}.  Documentation about patches to a release is
placed on this site at {\tt pub/software/aips/}{\it release-name} and
the code is placed in suitable subdirectories below this.  As bugs in
\NEWNAME\ are found, they are simply corrected since \NEWNAME\ remains
under development.  Corrections and additions are made with a midnight
job rather than with manual patches.  Since we now have many binary
installations, the patch system has changed.  We now actually patch
the master version of \OLDNAME, which means that a MNJ run on
\OLDNAME\ after the patch will fetch the corrected code and/or
binaries rather than failing.  Also, installations of \OLDNAME\ after
the patch date will contain the corrected code.

The \OLDNAME\ release has had a number of important patches:
\begin{enumerate}
\item\ {\tt REBYTE} did not handle tables with long rows ({\tt IM} and
       possibly {\tt BP}) correctly {\it 2008-01-09}
\item\ {\tt FITLD} did not translate {\tt WX} (weather) tables
       correctly {\it 2008-01-18}
\item\ {\tt DFT} model division did not set weights correctly {\it
       2008-03-05}
\item\ {\tt FILLM} did not scale and weight cross-hand data for some
       baselines correctly {\it 2008-03-05}
\item\ {\tt VISDFT} did not do multi-scale model division and
       subtraction correctly {\it 2008-04-29}
\item\ {\tt FILLM} did not set the {\tt CORRCOEF} keyword correctly
       for recent data {\it 2008-06-19}
\end{enumerate}


\section{\AIPS\ Distribution}

We are now able to log apparent MNJ accesses and downloads of the tar
balls.  We count these by unique IP address.  Since some systems
assign the same computer different IP addresses at different times,
this will be a bit of an over-estimate of actual sites/computers.
However, a single IP address is often used to provide \AIPS\ to a
number of computers, so these numbers are probably an under-estimate
of the number of computers running current versions of \AIPS\@. In
2008, there have been a total of 981 IP addresses so far that have
accessed the NRAO cvs master.  Each of these has at least installed
\AIPS\ and {\bf 214} appear to have run the MNJ on \RELEASENAME\ at
least occasionally.  During 2008 more than 155 IP addresses have
downloaded the frozen form of \OLDNAME, while more than 611 IP
addresses have downloaded \RELEASENAME\@.  The binary version was
accessed for installation or MNJs by 218 sites in \OLDNAME\ and 529
sites in \RELEASENAME\@.  The attached figure shows the cumulative
number of unique sites, cvs access sites, and binary and tar-ball
download sites known to us as a function of week --- so far --- in
2008.

\vspace{12pt}

\centerline{\resizebox{!}{2.8in}{\includegraphics{FIG/PLOTIT8a.PS}}}
\vfill\eject

\section{Improvements of interest in \RELEASENAME}

We expect to continue publishing the  \Aipsletter\ approximately every
six months along with the annual releases.  There have been quite a
few changes in \RELEASENAME\ in the last six months.  A significant
effort has been made to upgrade the capabilities of the ``TV'' display
({\tt XAS}) and to make imaging and model computation more efficient
through reduction in disk I/O\@.  Although both of these should
improve performance significantly for some users, neither causes any
significant change in user inputs.  A new verb {\tt SETMAXAP} allows
the user to guide \AIPS\ in the matter of the amount of memory which
it is safe to use within individual tasks.  The model computation
changes make this guidance significant.  There are two new tasks to
support on-line flagging tables which presently are written only for
the VLA\@.  These are {\tt PRTOF} to print such tables and {\tt OFLAG}
to use these tables to generate entries in flag ({\tt FG}) tables.
A new task {\tt FIXAL} and procedure {\tt FXALIAS} have been written
to deal with a temporary aliasing problem on EVLA-EVLA baselines.  A
new diagnosis task {\tt TIORD} was written to check $uv$ data sets to
make sure that they are in strict time order, reporting any failures.

\RELEASENAME\ contains major changes to the display software.  Older
versions may use the \RELEASENAME\ display ({\tt XAS}), but
\RELEASENAME\ code may not use older versions of {\tt XAS}\@.  {\tt
31DEC04} through \RELEASENAME\ use a new numbering scheme for magnetic
tape logical unit numbers that is incompatible with previous versions.
Thus all tape tasks and the server {\tt TPMON} must be from one of
these five releases.  Other than these issues, \RELEASENAME\ is
compatible in all major ways with the with the {\tt 15OCT98} and later
releases.  There are significant incompatibilities with older versions.

\subsection{TV display}

The \AIPS\ ``television'' display d\ae mon {\tt XAS} was significantly
modified to allow greater numbers of image memories and wider dynamic
ranges.  This upgrade was done by the addition of new operation codes
so that older versions should be able to use the new display program
without modification.  The number of TV memories, each the size of the
display screen, was changed from 4 to 16.  This change will allow for
much larger {\tt TVROAM}s up to 4 x 4 planes or even 16 x 1 or 1 x 16
planes and for larger {\tt TVMOVIE}s using a combination of  more
spectral channels and larger sub-images of each channel.  The \AIPS\
TV allows the user to do very complex combinations of images; the
larger number of memories will allow ``layering'' up to 16
simultaneous images in the display.

The range of data values in each memory has been changed from 0--255
to 0-8191.  The data go through look-up tables called LUTs which
previously had 256 input values and output values in the range 0-255.
The new range is 8192 inputs and output values in the range 0-2046.
These outputs are summed over those image memories which are ``on''
and enter the output function memory look-up tables (OFMs) which now
have 32752 possible input values ($2047 \times 16$).  The output data
range is still 0-255, which is all the display screen can actually
handle.  This extended dynamic range should allow for greater display
flexibility after the image has been loaded to the TV memory and
should also allow mathematical combinations of images, such as {\tt
TVHUEINT}, of greater accuracy.

Other than the ability to ask for {\tt TVCHAN} up to 16, there are few
changes visible at the user level.  Tasks and verbs that use two
graphics or grey-scale channels now use new adverbs {\tt GR2CHAN} and
{\tt TV2CHAN}, respectively, to specify the second display.  The verbs
{\tt GRON}, {\tt GROFF}, {\tt TVON} and {\tt TVOFF} that used to take
a decimal-coded immediate argument, \eg\ {\tt TVON 12} meant turn on
channels 1 and 2, now take a binary-coded immediate argument.  Thus,
in the new system, {\tt TVON 12} turns on channels 3 and 4 ($12 = 4 +
8 = 2^{3-1} + 2^{4-1}$).  Other verbs and tasks which allowed {\tt
TVCHAN} and {\tt GRCHAN} to have decimal-coded multiple values now no
longer support that option.  Task {\tt TVCPS} now displays whatever
graphics channels are visible rather than requiring the user to
specify which of the visible and invisible ones it should use.  \AIPS\
{\tt INPUTS} and {\tt GO} now know the number of TV memories and the
size of the TV screen locally and use those limits when checking
adverb values.

\vfill\eject

\subsection{Imaging and model computation}

When \AIPS\ computed the visibilities from a source model, it used to
have to re-read the $uv$ data for every facet and, for
frequency-dependent models, every spectral channel.  The latter occur
in tasks {\tt IMAGR} and {\tt OOSUB} which allow the user to correct
models for the primary beam pattern and for known images of spectral
index.  All tasks involved with model computation deal with multiple
facets, including calibration tasks such as {\tt CALIB}, {\tt FRING},
and so forth, as well as  some modes in {\tt IMAGR}\@.  The basic
routines that compute models in both DFT and gridded forms have been
changed so that they can allocate a large ``pseudo-AP'' memory and
compute the models for as many channels and facets as possible for
each read through the $uv$ data.  This should greatly improve
performance in large bandwidth-synthesis imaging problems involving
wide fields and/or wide bandwidths.  We will continue to look for more
ways to reduce the disk traffic.

A new verb has been added to \AIPS, named {\tt SETMAXAP}, to allow the
user to specify the maximum amount of memory for an \AIPS\ task to
use.  This allows users on small-memory machines, or machines doing
many simultaneous operations, to limit the algorithms described above
to reasonable memory sizes.  If they are not limited, severe paging
problems could cause tasks to take nearly infinite times to complete.
Most \AIPS\ tasks have algorithms that adapt to available
memory, so a limited allocation will still work and should be faster
than a blindly page-faulting version.  We have to leave this parameter
in the users' hands since we cannot find an operating system service
to provide this information.  Standard memory allocation will quite
happily allow memory sizes in excess of available physical memory and
will only fail if they exceed available swap space.

During the testing of the new gridded and DFT subtractions, a number
of disturbing things were noticed.  It became clear that the order in
which facets were subtracted from the $uv$ data mattered, especially
in the less than totally accurate (but much faster) gridded
subtraction.  Steps were taken in the code to retain the original
$u,v,w$ values rather than to rotate them for one facet and then
rotate them back before starting the next.  In single-precision,
numerical error could accumulate in this operation and subtle changes
in cell position can change how the gridded algorithm does its
interpolation for those few points that are almost exactly on cells.
A better-known error is seen in imaging with large numbers of samples,
either many visibilities or many spectral channels in bandwidth
synthesis.  The images, including the beams, frequently show large
excursions in the corners.  This is the result of a random-walk
accumulation of numerical error in gridding followed by a very large
correction (in the corners) for the Fourier transform of the gridding
function.

We have experimented with making the pseudo-AP operate fully in double
precision.  We did not add options to send in double precision values,
but just took advantage of the improved accuracy internally.  Indeed,
the problems described above were greatly reduced by this.  There are
two costs in switching the pseudo-AP to double precision.  The first
is that the number of data words available is halved, which reduces the
number of facets and channels which can be handled simultaneously.
This will add to the real time in large problems (where the errors are
more important).  The second is that the cpu and real times increased
by a noticeable amount, probably due to double-precision arithmetic
being a bit slower and due to cache limitations when twice the memory
is required for an operation.  It was concluded that one may make a
few algorithmic changes to avoid the worst of the errors (\eg\
avoiding image corners, caching $u,v,w$ values for re-use) and that
the scientific results of the computations will not be enhanced by the
greater accuracy.  It is important to pay attention, however, to
details.  If a single-precision number is used to count a very large
number of small numbers, then it will reach a maximum beyond which the
small numbers will not contribute even if there are still very many of
them.  Double precision, or smarter algorithms, are the solutions in
such cases.

\subsection{UV data calibration and handling}

\subsubsection{FILLM}

A significant error, mentioned above in the patches listing, was
discovered in the task that translates the current VLA data format
into \AIPS\ $uv$ files.  {\tt FILLM} reverses the direction of some
baselines in the data to make them consistent when numbered with the
actual antenna numbers; they are written by the VLA on-line system in
a consistent fashion based on ``dcs'' address instead.  Unfortunately,
the nominal sensitivity (${\rm T}_{\rm sys}$) correction factors were
not swapped.  During the ModComp era (prior to June 27, 2007), this
affected only Solar data and the weights of cross-hand data.  The
latter is likely to be of almost no significance, but Solar data
should have encountered troubles calibrating polarization.  The new,
post-ModComp system writes the data as correlation coefficients,
forcing {\tt FILLM} to scale the data as well as the weights by the
nominal sensitivity.  This means that some baselines will have the
cross-hand data incorrectly scaled for data taken after June 27, 2007
using versions of {\tt FILLM} prior to March 6, 2008.  Parallel-hand
data were correctly scaled and weighted.  Solar data since June 27,
2007 should now be correct.

The {\tt 31DEC08} version of {\tt FILLM} also had a number of
improvements.  The option to average data on input was extended to
include correct averaging of the output tables ({\tt TY}, {\tt PO},
{\tt OF}, {\tt WX}, and {\tt OT}), to check integration time as well
as elapsed time to terminate an average, and to restart an integration
if the first time sample(s) had no valid data.  A few more holes, in
which {\tt FILLM} could miss changes of mode or the number of spectral
channels at file boundaries, were plugged.  The task now writes an
on-line flagging table in a new, more general and extensive, format.
The task was changed to use an improvement in the on-line format which
actually tells the reader which receiver was used.  For older data,
{\tt FILLM} will split two IFs from the same receiver if they are at
frequencies on opposite sides of one of the ``official'' band
boundaries.  Messages will now be issued when the task omits (or
includes) pointing- and tilt-mode data.  The writing of the header
keyword {\tt CORRCOEF} was corrected.  It indicates whether the data
are correlation coefficients ($+1$) or visibilities ($-1$) and is used
by the {\tt TYAPL} task.

Two new tasks were written to do useful things with the on-line
flagging table written by {\tt FILLM}\@.  They are {\tt PRTOF} which
displays the table, interpreting the flag bit patterns into
meaningful words, and {\tt OFLAG} which may be used to apply
selectively the information in the {\tt OF} table.  Note that both of
these read only the new {\tt OF} table.  {\tt FILLM} used to write an
{\tt OF} table once in a while, containing almost nothing of any use.

\subsubsection{EVLA-EVLA spectral aliasing}

At the current time, the VLA is being operated using some antennas
with the old VLA electronics and some antennas with the upgraded EVLA
electronics, all of which feed the old correlator.  Due to the absence
of some prohibitively expensive filters, data from below baseband
is aliased into the observing band on EVLA-EVLA baselines only.  For
narrow-band, spectral-line observations, this causes a serious error
in the band shape on those baselines only.  Extensive study suggests
that the form of the data on these baselines is
$$
     V(n) = A_c e^{2\pi i \phi} + f(n)\, A_c e^{-2\pi i \phi} + V_l(n)
$$
where $A_c$ is the amplitude of the continuum, $\phi$ is the
uncalibrated phase of the continuum, $V_l$ is any spectral-line signal
as a function of channel number $n$, and $f(n)$ is the strength of the
aliasing.  It appears that, to first order anyway, $f(n)$ is
independent of time, direction, IF, polarization, and antenna and is a
real function with no phase term.

We have written a new task {\tt FIXAL} which fits observations of
calibrator sources to determine $f(n)$ and then fits that function to
line-free channels in the main data set to determine $A_c$ and $\phi$
to correct the data for the aliasing.  A procedure {\tt FXALIAS} was
written to assist in the operation.  It runs {\tt BPASS} using only
VLA-VLA and VLA-EVLA baselines, applies the bandpass to all data with
{\tt SPLAT}, separates the bandpass calibrators with {\tt UVCOP}, and
then runs {\tt FIXAL}\@.  Note that this operation must be done on
totally uncalibrated data --- if any phase correction has been
applied, the above formula will have been rendered incorrect.

Users should note two things.  This problem is temporary.  When the
new WIDAR correlator is used, the problem will disappear.  However,
the problem will remain with us until then and will remain in the VLA
data archive forever.  {\it At present, the new task and procedure
should be regarded as experimental.}  They appear to work most of the
time and to remove most of the problem.  There are niggling bits left
and there seem to be isolated cases in which they do not work well.

\subsubsection{Other \uv-editing matters}

\begin{description}
\myitem{EDITA} and {\tt EDITR} were changed to handle flagged table
               rows without dying, to ``restore area'' with the same
               complex logic used in ``flag area'', to know which
               antennas have data and ignore those that don't, to
               handle phase plot ranges better, to keep track of
               source name/number even when only one source is
               included, and to apply the {\tt FG} table correctly to
               {\tt TY} tables before plotting them.
\myitem{SNFLG} now counts the data to determine the required dynamic
               memory correctly and benefits from the changes to the
               {\tt EDIT} class described for {\tt EDITA}\@.
\myitem{TVFLG} now supports channel averaging with {\tt NCHAVG} and
               {\tt CHINC} adverbs and allows auto-correlations to
               have phase (which they do in cross-hands).
\myitem{WIPER} now has interactive options {\tt FLAG BASELIN} and {\tt
               UNFLAG BASEL} to eliminate/restore all points from a
               user-entered antenna pair.  Up to 10 antenna pairs are
               remembered for each plotted point.  Phase plots handle
               wraps better.
\end{description}

\subsubsection{Other \uv-display matters}

\begin{description}
\myitem{LISTR} now supports two gain conversions {\tt EFST} (effective
               system temperature) and {\tt SEFD} (system equivalent
               flux density) and honors the {\tt FREQID} specification
               on gain listings.
\myitem{VPLOT} can now average spectral channels under control of
               adverb {\tt AVGCHAN} and can plot channels and IFs
               separately or together under control of the {\tt
               CROWDED} adverb.  It can now draw connected lines in
               color when requested and selects better phase ranges
               for plotting when possible.
\myitem{UVPLT} was improved to plot phases with the least wrap
               problem, to implement the fixed scale within a fixed
               range option, to bin phases in a vector fashion rather
               than scalar, and to scale $u,v,w$ by frequency.
\end{description}

\subsubsection{Other \uv-related matters}

\begin{description}
\myitem{FITLD} was corrected for an error that caused weather tables
               to be garbled and for an error in the logic that
               decided which antennas to include in the next record
               being written in the {\tt CL} table.  Code was added
               for the new software ``DifX'' correlator to read a new
               keyword and, based on it, to avoid one of the VLBA
               digital corrections, while making all of the others.
\myitem{SPLIT} and {\tt SPLAT} now write a new index table
               automatically.  The index table can help {\tt CALIB} do
               a better job of averaging within scans even in
               single-source files and running {\tt INDXR} is a
               nuisance.
\myitem{DELZN} was changed to use the antenna name rather than number
               since the latter can vary between data sets.
\myitem{CLCOR} was changed to offer a moving-source correction option
               using either fixed rates or an {\tt INFILE} and to use
               the new format of the output from {\tt DELZN}\@.
\myitem{Weights} in model division should be multiplied by the model
               amplitude squared.  The gridded method was corrected
               some time ago, but the DFT method was only corrected in
               March.
\myitem{TIORD} is a modest task intended simply to report all points
               in a data set at which the data are not in strict time
               order.  {\tt INDXR} is very fussy about this, but quits
               at the first such point without supplying any useful
               information.
\end{description}

\subsection{Analysis}

\begin{description}
\myitem{RMSD}  was changed to offer the histogram-fitting method of
               rms determination, to allow control of the number of
               iterations in the robust method, and to allow circular
               as well as rectangular apertures for the computations.
\myitem{CONVL} was changed to offer the option of doing a
               cross-correlation of two images and to find and fit the
               maximum in the result.
\myitem{IMEAN} and {\tt IMSTAT} can now do their thing inside or
               outside the specified window.
\myitem{MFPRT} was upgraded to offer a variety of new output options
               including full user control of which columns are
               displayed and more mnemonic column labels.
\myitem{IRING} was changed to offer more plot options including choice
               and size of symbol, error bars, connecting of points,
               and a rescale and relabel option for the $x$ axis.
               This allows conversion to, for example, kpc rather than
               arcsec.  It now also offers a text-file output option
               (for use with {\tt PLOTR} or other plot programs) and
               uses a plot type fully understood by {\tt EXTLIST} and
               {\tt PLGET}\@.
\myitem{ISPEC} had displays of total flux, sum of plotted points, and
               number of non-blanked points added.  Previously, one
               had to add up the numbers by hand.
\myitem{XGAUS} had a number of bugs corrected, one of which caused it
               to go catatonic when a retry was requested.  Additional
               descriptions were added to the help file to clarify what
               it is attempting to do.
\myitem{IMFIT} and {\tt JMFIT} reported the major and minor axis sizes
               in arc seconds after conversion from pixels, but
               reported the pixel-fit position angle rather than the
               one the corresponds to CCW from North in coordinate
               space.  {\tt SAD} reported this position angle
               correctly.
\end{description}

\subsection{Miscellaneous matters}

\begin{description}
\myitem{TABED} had the operations of delete, clip, and unflag added.
\myitem{PEELR} now supports the {\tt SOLMODE} option.
\myitem{I/O}   count is now displayed in megabytes at the end of each
               task.  If one enters {\tt SETDEBUG 1}, then all
               following tasks will display separately the total count
               and size of {\tt ZMIO} and {\tt ZFIO} reads and writes
               from which the total is computed.
\myitem{\Cookbook} was reviewed thoroughly and numerous upgrades and
               corrections were made about January 1.
\end{description}

\vfill\eject

% mailer page
% \cleardoublepage
\pagestyle{empty}
 \vbox to 4.4in{
  \vspace{12pt}
%  \vfill
\centerline{\resizebox{!}{3.2in}{\includegraphics{FIG/Mandrill.eps}}}
%  \centerline{\rotatebox{-90}{\resizebox{!}{3.5in}{%
%  \includegraphics{FIG/Mandrill.color.plt}}}}
  \vspace{12pt}
  \centerline{{\huge \tt \AIPRELEASE}}
  \vspace{12pt}
  \vfill}
\phantom{...}
\centerline{\resizebox{!}{!}{\includegraphics{FIG/AIPSLETS.PS}}}

\end{document}
