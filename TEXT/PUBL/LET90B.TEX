%-----------------------------------------------------------------------
%;  Copyright (C) 1995
%;  Associated Universities, Inc. Washington DC, USA.
%;
%;  This program is free software; you can redistribute it and/or
%;  modify it under the terms of the GNU General Public License as
%;  published by the Free Software Foundation; either version 2 of
%;  the License, or (at your option) any later version.
%;
%;  This program is distributed in the hope that it will be useful,
%;  but WITHOUT ANY WARRANTY; without even the implied warranty of
%;  MERCHANTABILITY or FITNESS FOR A PARTICULAR PURPOSE.  See the
%;  GNU General Public License for more details.
%;
%;  You should have received a copy of the GNU General Public
%;  License along with this program; if not, write to the Free
%;  Software Foundation, Inc., 675 Massachusetts Ave, Cambridge,
%;  MA 02139, USA.
%;
%;  Correspondence concerning AIPS should be addressed as follows:
%;          Internet email: aipsmail@nrao.edu.
%;          Postal address: AIPS Project Office
%;                          National Radio Astronomy Observatory
%;                          520 Edgemont Road
%;                          Charlottesville, VA 22903-2475 USA
%-----------------------------------------------------------------------
\input al82.mac
\input al8pt.mac
\letterbegin {X} {2} {April 15, 1990}

\subtitf{New \AIPS\ Distribution Procedures}
   As of the 15APR90 release of \AIPS\ we are adopting new distribution
procedures.  The preparation for the release now begins a month before
the release date rather than on the release date.  We are also testing
the installation procedures before each release is shipped to minimize
the difficulties in installing \AIPS .  We hope that this procedure
will allow us to begin shipping more reliable versions of \AIPS\ by
the release date.

\subtit{Recent Developments}

   Much of the software activity in the last quarter has been in
debugging calibration software; this is reflected in the many bug
fixes described in a later article.  Other areas that are being
actively developed are 1) imaging and deconvolving uv data in
arbirtary order, 2) an interactive baseline-based uv data editing
routine for VLBI data, 3) continued development of AIPS X-windows
displays (discussed further in another article), 4) \AIPS\ under Unicos
(also discussed in another article), and 5) processing of MkIII VLBI
data (also discussed later).

   The programmer documentation, Going \AIPS\ , has been revised to
reflect the state of \AIPS\ as of the 15APR90 release and is now
available.  The user documentation, the \Cookbook , is currently being
revised.  Chapters describing the processing of VLBI data (Chapter 10)
and calibrating VLA data (Chapter 99) are currently available.  All
documentation may be ordered using the order form at the end of this
\AIPS Letter.

\subtit{X-windows Support}
The \AIPS\ group has ported the SunView screen server to the X Window
System$^TM$ using the XView toolkit. A beta test version of XVSS will
be distributed as part of the 15JUL90 release of \AIPS. Sites may
obtain this version earlier using AIPSSERV. Availability via AIPSSERV
was announced on the Bananas mail exploder.

XVSS will require the XView toolkit, which is freely available via
anonymous ftp from sites which distribute the X Window System, and
a window manager that handles virtual color maps according to the
final version of the Inter-Client Communications Conventions Manual
(the ICCCM). Some people may disagree with our use of the XView
toolkit but it did allow us to move the screen server to the X Window
System in a very short time. The XView toolkit has been tested on
Sun hardware and on the DECstation 3100; it should work on any Unix
system that is compatible with BSD Unix.

The NRL group has been working on an alternative screen server that uses no
toolkits. This does not have some of the features of XVSS but is
noticably faster and may be preferred by users with slower workstations
such as Sun 3s. We will distribute this alongside XVSS when it becomes
available. Further details will be announced when available.
(X Window System is a trademark of the Massachussetts Institute of
Technology.)

\subtit{MkIII VLBI data}

   There have been a number of recent improvements in the AIPS task,
MK3IN, that reads Haystack format VLBI ``A'' tapes.  This task is now
more reliable and may produce useable output but is still incompletely
tested.  Much of the current activity is in determining appropriate
calibration procedures to remove the instrumental artifacts of the MkIII
system as well as normal amplitude, delay and rate calibration.  The
software distributed in the 15APR90 release may be adequate for
processing MkIII data but users should carefully examine the data at
each step of the processing.  Use of bandpass calibration is
recommended even for continuum data; especially if both sidebands are
used.  A number of other problems have been noted with data read by
MK3IN but it is not yet known if these are due to errors in the data
or in MK3IN.

\subtit{User Agreement}

   Starting with the 15OCT89 release of \AIPS\ all user sites
need to provide the NRAO with a signed ``User Agreement'' form.  This
``User Agreement'' is a no-cost item for sites engaged in basic
research in astronomy.  The need for an \AIPS\ ``user agreement'' has
arisen for several reasons.  The most important is that we want all
\AIPS\ sites to obtain their copies of \AIPS\ from the NRAO and
thereby to be made aware of the restrictions that apply to their use
of the code and to our support of it.  For sites doing astronomical
research, these restrictions are only to maintain the proprietary
nature of the code and to direct third parties who wish to receive the
code to the \hbox{NRAO}.  Once properly signed, a User Agreement
remains valid for 5 years and does not need to be renewed before this
time.  A copy of this agreement is printed in the back of this
\AIPS Letter.  The agreement should be signed by an individual in a
position to take responsibility that the user group will follow the
agreement.  This may be a department chairman or an administrative
officer.  Mail signed forms to Amy Shepherd, NRAO, Edgemont Road,
Charlottesville, VA 22903-2475

\subtit{\AIPS\ under Unicos}

The 15APR90 release includes an implementation under the Cray Research
Inc. (CRI) UNICOS operating system.  UNICOS is derived from UNIX System V and
supports varying degrees of Berkeley UNIX features depending on the revision
level.  The UNICOS implementation was developed on Cray X-MPs starting under
UNICOS 5.0.13 with cft77 3.1 using the CRI facilities at Mendota Heights.
The requirements for Y-MPs, and especially Cray 2s, may differ.  Systems
running revision levels of UNICOS prior to 5.0.13 also may encounter problems,
particularly with regard to the availability of Berkeley UNIX features.  The
UNICOS port is still in a state of flux, but has been developed to the point
that it has passed the DDT and has been put into production at the University
of Minnesota using a virtual TV interface to the Sunview Screen Server (SSS)
program via TVMON.  Areas still under development include the magnetic tape
interface (FITS-disk files work), various data conversion Z-routines, dynamic
pseudo-AP memory management, and interactive virtual TV functions.  Further
optimization, including auto-tasking and perhaps data storage, also are under
development.
\def\xyz {\hfill & \hfill}
A summary of timings for the large DDT problems on a lightly loaded Cray
X-MP/416 (using only a single processor) compared to an empty Sun 3/60 is
shown below.  The typically scalar DDT problems appear in the upper portion
of the table.  Those that make use of the pseudo-AP library appear in the
lower portion.  All times are in seconds. \par \vskip 0.2in
\settabs 10 \columns
\+ &&& \hfill Cray$^2$ \hfill &&& \hfill Sun$^3$ \hfill &&& \cr
\+ &&& \hfill X-MP \hfill &&& \hfill 3/60 \hfill && \hfill Sun: & Cray \cr
\vskip 0.1in
\+ &&& \hfill WALL- \hfill &&& \hfill WALL- \hfill &&& \hfill WALL- \hfill \cr
\+ TASK & (N)$^1$ & \hfill CPU \xyz CLOCK \xyz C:W \xyz CPU \xyz CLOCK \xyz
C:W \xyz CPU \xyz CLOCK \hfill \cr
\vskip 0.1in
\+ CCMRG & ~(1) & \hfill ~~4 \xyz ~~5 \xyz .80 \xyz ~~~24 \xyz ~~~45 \xyz .53
\xyz ~~6.0 \xyz ~~9.0 \hfill \cr
\+ COMB  & ~(8) & \hfill ~32 \xyz ~40 \xyz .80 \xyz ~~464 \xyz ~~752 \xyz .62
\xyz ~14.5 \xyz ~18.8 \hfill \cr
\+ UVSRT & ~(2) & \hfill ~38 \xyz ~60 \xyz .63 \xyz ~~300 \xyz ~1022 \xyz .29
\xyz ~~7.9 \xyz ~17.0 \hfill \cr
\+ UVDIF & ~(2) & \hfill ~14 \xyz ~24 \xyz .58 \xyz ~~532 \xyz ~~562 \xyz .95
\xyz ~38.0 \xyz ~23.4 \hfill \cr
\+ SUBIM & ~(2) & \hfill ~~6 \xyz ~~8 \xyz .75 \xyz ~~100 \xyz ~~156 \xyz .64
\xyz ~16.7 \xyz ~19.5 \hfill \cr \vskip 0.1in
\+ APCLN & ~(1) & \hfill 398 \xyz 526 \xyz .76 \xyz 26105 \xyz 26653 \xyz .98
\xyz ~65.6 \xyz ~50.7 \hfill \cr
\+ APRES & ~(1) & \hfill ~13 \xyz ~15 \xyz .87 \xyz ~1898 \xyz ~2116 \xyz .90
\xyz 146.0 \xyz 141.1 \hfill \cr
\+ ASCAL & ~(1) & \hfill ~48 \xyz ~55 \xyz .87 \xyz 15894 \xyz 16024 \xyz .99
\xyz 331.1 \xyz 291.3 \hfill \cr
\+ MXMAP & ~(1) & \hfill ~13 \xyz ~16 \xyz .81 \xyz ~1491 \xyz ~1721 \xyz .87
\xyz 114.7 \xyz 107.6 \hfill \cr
\+ MXCLN & ~(1) & \hfill 459 \xyz 486 \xyz .94 \xyz 33175 \xyz 34087 \xyz .97
\xyz ~72.3 \xyz ~70.1 \hfill \cr
\+ UVMAP & ~(1) & \hfill ~11 \xyz ~40 \xyz .28 \xyz ~1308 \xyz ~1579 \xyz .83
\xyz 118.9 \xyz ~39.5 \hfill \cr
\+ VTESS & ~(1) & \hfill ~60 \xyz ~69 \xyz .87 \xyz ~7196 \xyz ~8291 \xyz .87
\xyz 119.5 \xyz 120.2 \hfill \cr \vskip 0.1in
\+ TOTAL & (22) & & \hfill 1344 \hfill &&& \hfill 93008 \hfill &&& \hfill
~69.2 \hfill \cr  \par \vskip 0.2in
$^1$Number of times problem executed (times represent sum of executions).
\par $^2$Cray X-MP/416 (SN218):  15APR90 AIPS: UNICOS 5.0.13 with cft77 3.1;
256 Kword pseudo-AP; hardware gather/scatter; SCILIB CFFT2 for FFTs; no
auto-tasking; Q-routines optimized ``full'', all optimization suppressed on
rest; scratch files through 128 Mword SSD via logical device cache. \par
$^3$Sun 3/60G (KONG): 15OCT89 AIPS; SunOS 3.5; MC68881 co-processor; 64
Kword ``vanilla'' pseudo-AP; Q-routines compiled at highest optimization
level, default optimization level on rest; all files on local disk. \par

\subtit{Supercomputing AIPS Workshop}

The NRAO, in cooperation with Cray Research Inc. (CRI), plans to hold a
workshop on the installation, operation, and management of AIPS on
supercomputers.  The meeting will be primarily directed to those planning a
new installation of AIPS on their supercomputer but will also be a workshop
for those currently running AIPS on supercomputers.  CRI has graciously agreed
to host the workshop, so it will take place at their facilities in Mendota
Hights (Minneapolis), Minnesota, on September 10 and 11, 1990.  Included in
the program will be a tour of their research facilities.  Participants should
plan to arrive in Minneapolis on Sunday evening September 9.  Activities will
terminate by noon on Tuesday September 11.

For detailed information contact Bob Burns at the NRAO in Charlottesville,
804-296-0229,  E-mail bburns@nrao.edu.

\subtit{Document and Software Distribution by AIPSSERV}
   The NRAO maintains a mail-based file server, AIPSSERV, which is
available for use by \AIPS\ users to fetch files from {\tt CVAX}.
Detailed instructions for using this facility may be obtained by
sending an E-mail message containing the single word ``{\tt help}'' to
one of the following addresses:\hfil\break
{\tt aipsserv@nrao.edu},
{\tt aipsserv@nrao.bitnet},
{\tt ...!uunet!nrao1!aipsserv}
or {\tt 6654::aipsserv}.

   We intend to use this facility to distribute text files such as (a)
the data files needed as input to the ionospheric Faraday radiation
correction, (b) notes about probems
encountered in installation procedures, and (c) the contents of the
CHANGE.DOC files (documentation of software changes).  A general guide
to special files for distribution by AIPSSERV will be kept in file
DOC:README.  To obtain a text file in plain text form send AIPSSERV a
message of the form ``{\tt sendplain logical:filename.ext}'' where
{\tt logical} is the logical name of the directory and {\tt
filename.ext} is the name of the desired file.
For example to be sent a copy of the README. file send AIPSSERV the
message ``{\tt sendplain DOC:README.}''.
Known problems and workarounds for 15APR90 \AIPS\ will be
distributed as files DOC:VMSNOTES.90B and DOC:UNIXNOTES.90B for the
VMS and UNIX  installations.
Multiple files may be
obtained by multiple ``sendplain'' commands one per line.

\subtit{Summary of Changes:  15 January 1990 --- 15 April 1990}

   We are no longer printing the contents of the software change
documentation file CHANGE.DOC.  The old version of CHANGE.DOC are kept
in area HIST with names CHANGED.yyr where yy are the last 2 digits of
the year and r is the release date code (A,B,C and D being 15JAN,
15APR, 15JUL and 15OCT).  Anyone wishing to see the details
given in these files may obtain them as described in the article on
AIPSSERV.  To obtain this documentation file for 15APR90 send AIPSSERV
the message ``{\tt sendplain DOC:CHANGED.90B}''.  A summary of the
changes made to the \AIPS\ software is given in the following
sections.

\vfil\eject
\smallhead{Changes of Interest to Users: 15APR90}

Several new tasks were added to 15APR90.  {\tt SPFLG} which edits data
in the time and frequency domain in a mannar similar to {\tt TVFLG}
was added.  {\tt TBAVG} will average all data at a given time over
baseline and write a new uv data file.  This is useful for measuring
the time variable flux densities of point sources.  {\tt UVBAS} fits a
``continuum'' baseline to a visibility spectrum and subtracts
it from the data.  This allows a much faster alternative to {\tt
UVSUB} for continuum subtraction for the cases in which this technique
can be applied.  {\tt MK3TX} will read text files from
Haystack format ``A'' and ``B'' tapes and write them to a disk file.


All of the \AIPS\ tasks and \AIPS\ itself can now optionally write
line printer output to a file as well as to the printer.  There was a
major upgrade to {\tt TVFLG}.  The new features include {\tt UNDO},
{\tt REDO} and new {\tt CLIP} options for flagging, saving flagging
commands in a temporary table ({\tt FC}) for undoing flags and for
recovering if the program or machine crashes, avoiding displaying long
time sequences with no data, editing autocorrelation data, and more.
The concept of windows was added to the \AIPS\ TV model.  This allows
more efficient use of workstations and greatly improves the speed of
operations such as {\tt TVMOVIE} on them.  The ``Resize'' button on
the workstation TV display now toggles between a full size display and
a smaller, user selected one.

{\tt UVFND} now has the options to display data with weights exceeding
a given value and to search both sides of the uv plane in the `UVBX'
option.  {\tt SWPOL} will now switch the polarizations for selected
antennas; this is useful for VLBI data which was mislabeled at the
correlator.  The ionospheric monotoring data was added to the {\tt
AIPSIONS} directory.  {\tt PRTAB} will now handle up
to 1024 keyword/value pairs in a table header.  {\tt ANCAL} will now
process multiple IFs in a single run.  {\tt PRTUV} now works on
compressed data and a problem which occured when a file had more than
30 sources was fixed.  {\tt UVFLG} was changed to allow editing
``Channel 0'' data and then copying the {\tt FG} table to the line
data file without editing the channel ranges.  {\tt UVFLG} now also
has an option to set a mask to flag arbitrary combinations of
polarization correlation.  {\tt SDTUV} was modified to work on more
antennas and a number of other errors corrected.  A problem with
determining subarray number was fixed in {\tt UVCOP} which now
also copies all catalog header keyword/value pairs.
Errors in {\tt UVFIX} which caused it to fail on compressed data were
fixed.  A bug
in the calibration routines was causing the gain stabilization to
fail.  A bug in TKSLICE was causing it to go into an infinite loop.  A
bug in {\tt XMOM} caused it to fail under all circumstances.  Several
bugs in {\tt SNPLT} causing problems with OPTYPE='SUM' were fixed.  A
number of labeling problems in {\tt KNTR} were fixed.  A number of
problems in the virtual TV ``Z'' routines which caused remote TVs over
internet to fail were fixed.  The default behavior in {\tt TACOP} when
no output file was specified is now more sensible.

   A bug that had disabled {\tt PLCUB} was fixed. {\tt UVPLT} now
correctly scales u, v, and w for the actual frequency of that data.  A
bug in the FITS reader which caused only a single character to be read
from character entries in a FITS ASCII table was fixed.  Several
errors in {\tt SLCOL} were fixed which caused BLC and TRC to be
ignored.

The handling of frequencies for VLA spectral line data was improved.
{\tt FILLM} was mislabeling the frequencies of the data by half the
total bandwidth and had several other related problems.  The use of
frequencies was made consistent in {\tt FILLM}, {\tt SPLIT}, {\tt
UVCOP}, {\tt MX} and {\tt HORUS}.  Also, a number of other corrections
were made in the calibration of spectral line data and in {\tt FILLM}.
A problem in {\tt SPLIT} writing compressed data was corrected.
Bugs were fixed in {\tt BATER} which probably kept it from working.

A number of problems in the processing of MkIII VLBI data were fixed.
In {\tt MK3IN} the processing of phase cals had several serious bugs
removed and the specification of antennas and timeranges was made
friendlier.  Also, there were several errors in the correction for the
fractional bit error.  A number of bugs in {\tt CALIB} and the
calibration routines which occured for MkIII data were fixed.
Several new options were added to {\tt CLCOR} that are useful for
MkIII data: 1) correct for an antenna position error, 2) insert MkIII
``manual'' phase cals and 3) correct for the difference between
single- and multi-band delays in MkIII VLBI data.


\smallhead{Changes of Interest to Programmers: 15APR90}

{\tt DGHEAD} now includes the source frequency offset if data from a
single source is selected. {\tt FUDGE} now processes compressed data.
New routines {\tt VISUNP} and {\tt VISPCK} simplify unpacking and
packing (compressed) uv data.  Several serious problems in {\tt
FILAI2}, used in \AIPS\ VMS installations, were fixed.  A makefile
(MAKEFILE) was added to {\tt YSVU} to assist in the creation of SSS
for Sun TVs.  A number of errors in the DeAnza ``Y'' routines were
corrected.
The current logical \AIPS\ version (TST, NEW and OLD) was
added to DMSG.INC and to the accounting file.  This allows {\tt PRTAC}
to distinguish between different releases of \AIPS.  {\tt AXSTRN} now
gives frequency labels in exponential notation if the range of the
format is exceeded.  New routine {\tt KEYCOP} copies catalog header
keywords from one file to another.  A bug in the logic for averaging
frequencies in {\tt MX} was fixed.

\smallhead{Changes of Interest to Users: 15JUL90}

   A number of new verbs were added to manipulate AIPS tables entries;
{\tt GETTHEAD}, {\tt PUTTHEAD}, {\tt TABGET}, and {\tt TABPUT} can
read and write table header keywords and table entries.

   The `CALC' option in {\tt SETJY} was updated to use the lastest
spectra for 3C48, 3C138, 3C147 and 3C286.  Also, the QUAL adverb was
added.  {\tt VLBIN} can now handle lower sideband data.  {\tt POSSM}
can now plot multiple IFs on the same plot.  A number of tasks had
task specific adverb arrays replacing APARM, BPARM ... .  Some of
these are {\tt CLCOR}, {\tt UVCOP}, and {\tt UVFIX}.  A number of bugs
in {\tt BPASS} were fixed.  More bugs were also removed from {\tt
FILLM}.  {\tt POSSM} no longer tried to plot flagged channels.

   The gridding routine UVGRID, used by {\tt MX} can now grid data in
arbitrary order.  This eliminates the need for sorting data if natural
weighting is being used and the 'DFT' model calculation is used.  {\tt
UVCOP} can now select by source, qualifier, FQ id, and IF range.
{\tt PRTTP} now has a terse output format for FITS files.  {\tt MX}
again works for 16 fields.  A number of problems were fixed in {\tt
LWPLA}, the task that sends plots to postscript printers.


\smallhead{Changes of Interest to Programmers: 15JUL90}

   The flagging (FG) tables now include a frequency (FQ) identifier.
{\tt SOURNU} now uses the correct null (selects all) value of QUAL.
The new routines {\tt CLSEL}, {\tt SNSEL}, {\tt SUSEL}, and {\tt
FGSEL} can copy CL, SN, SU and FG tables selecting a range of IFs.
An X-windows AIPS TV {\tt XVSS} was added and is in the beta test stage.
There was a major change in the directory structure for the virtual TV
routines to accommodate the X-windows versions.  The maximum number of
AIPS files that can be open simultaneously was increased to 20 in all
3 categories.

\vfill\eject
\pgskip
\pgskip
\input order.tex
\end



