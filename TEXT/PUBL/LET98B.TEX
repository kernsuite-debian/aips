% -*- latex -*-
%-----------------------------------------------------------------------
%;  Copyright (C) 1998
%;  Associated Universities, Inc. Washington DC, USA.
%;
%;  This program is free software; you can redistribute it and/or
%;  modify it under the terms of the GNU General Public License as
%;  published by the Free Software Foundation; either version 2 of
%;  the License, or (at your option) any later version.
%;
%;  This program is distributed in the hope that it will be useful,
%;  but WITHOUT ANY WARRANTY; without even the implied warranty of
%;  MERCHANTABILITY or FITNESS FOR A PARTICULAR PURPOSE.  See the
%;  GNU General Public License for more details.
%;
%;  You should have received a copy of the GNU General Public
%;  License along with this program; if not, write to the Free
%;  Software Foundation, Inc., 675 Massachusetts Ave, Cambridge,
%;  MA 02139, USA.
%;
%;  Correspondence concerning AIPS should be addressed as follows:
%;          Internet email: aipsmail@nrao.edu.
%;          Postal address: AIPS Project Office
%;                          National Radio Astronomy Observatory
%;                          520 Edgemont Road
%;                          Charlottesville, VA 22903-2475 USA
%-----------------------------------------------------------------------
%Body of AIPSletter for 15 October 1998

\documentstyle [twoside]{article}

\newcommand{\AIPRELEASE}{October 15, 1998}
\newcommand{\AIPVOLUME}{Volume XVIII}
\newcommand{\AIPNUMBER}{Number 2}
\newcommand{\RELEASENAME}{{\tt 15OCT98}}
\newcommand{\OLDNAME}{{\tt 15APR98}}
\newcommand{\NEXTNAME}{{\tt 15APR99}}

%macros and title page format for the \AIPS\ letter.
\input LET98.MAC
\input psfig

\newcommand{\MYSpace}{-11pt}

\normalstyle

\section{General developments in \AIPS}

\subsection{Current and next release}

The \AIPRELEASE\ release of Classic \AIPS\ is now available.  It may
be obtained via \emph{anonymous} ftp or by contacting Ernie Allen at
the address given in the masthead.  {\tt 15OCT98} is also available
on CDrom as well as the more traditional tape media.  \AIPS\ is now
copyright \copyright 1995 through 1998 by Associated Universities,
Inc., NRAO's parent corporation, but may be made freely available
under the terms of the Free Software Foundation's General Public
License \hbox{(GPL)}.  This means that User Agreements are no longer
required, that \AIPS\ may be obtained via anonymous ftp without
contacting NRAO, and that the software may be redistributed (and/or
modified), under certain conditions.  The full text of the GPL can be
found in the \texttt{15JUL95} \Aipsletter. Details on how to obtain
\AIPS\ under the new licensing system appear later in this
\Aipsletter.

The next release of \AIPS\ will be \texttt{15APR99}.  It is possible
to get early access to, and remain current with, this release by
running a ``midnight job''; see the \AIPS\ home page for further
details.  Note that this allows your site to receive the latest
improvements and bug fixes, at the cost of also receiving the latest
bugs.  The latter can and will be fixed as rapidly as possible when
the programmers are notified of them at \texttt{daip@nrao.edu}.

\subsection{Highlights}

This release contains a new version of \texttt{XAS} which will emulate
an IIS Model 70 doing full color displays on those terminals which
support ``24-bit TrueColor'' X-Windows visuals.  This enables stunning
overlays of multi-spectral images, interactive hue-intensity displays,
roam and other split-screen algorithms, and more.  The ability to have
more than one TV on a single TV-host has been developed, as has the
ability to adapt the screen-full function of print routines to the
current size of the workstation window.

The automatic transfer of calibration information from the VLBA
correlator to \AIPS\ appears to be working in the new release.  Users
will no longer have to extract these data laboriously from log files.
VLBI fringe fitting and model fitting also received significant
attention for this release.  Interactive data editing using the TV
graphics planes also improved significantly.

\vfill\eject

\section{Improvements of interest to users in \RELEASENAME}

\emph{The {\tt 15APR98} introduced numerous changes which are not
compatible with previous releases.  Disk files written by previous
versions are read transparently by {\tt 15APR98} and {\tt 15OCT98}
(including {\tt SAVE}/{\tt GET} files), but users must not attempt to
read disk files written by either of these versions with earlier
versions.  {\tt 15APR98} and {\tt 15OCT98} {\tt AIPS} cannot start
previous versions of tasks and the TV displays of the versions are
incompatible.  The TV displays of {\tt 15OCT98} and {\tt 15APR98} are
also not fully compatible.  {\tt SAVE}/{\tt GET} files for {\tt
15OCT98} cannot be read by {\tt 15APR98} and earlier releases; {\tt
15OCT98} will translate old {\tt SAVE}/{\tt GET} files when it first
reads them.}

\subsection{VLBI data processing}

\subsubsection{{\tt FITLD}: Calibration transfer}

While VLBA calibration data usually has to be extracted from log files
by an observer, it has long been a goal of the VLBA project to
extract these data at the correlator automatically and to include them
as part of the data set that is sent out to the observer.

This capability has been tested within NRAO with the result that
several problems were discovered.  Most of these arose from
misunderstandings between the VLBA online group and the \AIPS\ group
concerning the formats of the binary tables that were to carry the
calibration data. We believe that these problems have now been
resolved and the {\tt 15OCT98} version of {\tt FITLD} should be
capable of reading calibration data from the VLBA when it is made
available to observers outside \hbox{NRAO}.

The process for handling calibration transfer through {\tt FITLD} is
not yet fully automatic.  Since VLBA data sets often require several
files to be concatenated, much of the calibration data may be
duplicated.  Some \AIPS\ tasks may encounter problems as a result of
the duplicated information.  Users should, therefore, run {\tt TAMRG}
on the calibration tables to remove the redundant records.  The {\tt
EXPLAIN} file for {\tt FITLD} suggests suitable inputs to {\tt TAMRG}
for this procedure.  This will be handled in a more convenient way in
a future release of \hbox{\AIPS}.

Other correlators that produce data in the FITS Interferometry Data
Interchange format that is used by the VLBA should be aware that {\tt
FITLD} now checks the format of incoming tables much more thoroughly
than it did before (the detection of problems in calibration transfer
had been hampered by inadequate checks); this means that {\tt FITLD}
may now reject some data that would have been passed by previous
versions.  The difficulties that arose during the testing of VLBA
calibration transfer also made it clear that the existing definition
of the FITS Interferometry Data Interchange format (VLBA Correlator
Memo No.~108) was no longer adequate. A new definition of this format
is now available in draft form. This may be obtained by anonymous ftp
from {\tt ftp.aoc.nrao.edu} as either
\begin{center}
\begin{tabular}{l}
/pub/staff/cflatter/FITS-IDI.ps (PostScript) or \\
/pub/staff/cflatter/FITS-IDI.pdf (Adobe PDF)
\end{tabular}
\end{center}
It may also be obtained through the Web page at
{\tt http://www.aoc.nrao.edu/$\sim$cflatter/FITS-IDI.html}. A list of
errata is maintained at {\tt
http://www.aoc.nrao.edu/$\sim$cflatter/FITS-IDI-errata.html}.

\subsubsection{{\tt FITLD}: Lower sideband and other corrections}

When reading data in FITS Interferometry Data Interchange format (as
used by the VLBA correlator), {\tt FITLD} reverses the order of lower
sideband frequency channels so that they are sorted into ascending
order of frequency.  This requires {\tt FITLD} to recalculate the
reference frequencies for the lower sidebands so that the frequency
channels are labelled correctly.  This calculation has been performed
erroneously with the result that frequencies for lower sidebands may
be up to a channel bandwidth in error.  Frequencies in {\tt FITLD}
were also rounded to integer multiple of 10 kHz.  This led to shifts
of a few kHz in the labelling of lower sideband frequencies.  Both
problems have been corrected in \hbox{{\tt15OCT98}}.

{\tt FITLD} was making source tables with multiple occurrences of
essentially the same source and with multiple occurrences of the same
source number attached to different sources.  The VLBA Correlator
treats the same source, with different {\tt CALCODE} as a different
source.  This causes \AIPS\ to have things in the source table that it
cannot properly address and is simply due to the VLBA antennas passing
on {\tt CALCODE}s while non-VLBA ones do not.  {\tt FITLD} was changed
to drop {\tt CALCODE} as a data selection adverb and to allow for
source renumbering even while reading the first input file.  The list
handling was changed to include current additions to the source list
when checking the next input source and to allow for the case in which
2 or more source numbers would both be translated into the same output
source number.

\subsubsection{Fringe fitting}

The FFT stage of {\tt KRING} received a lot of attention since the
last release.  Extrapolation of delay and rate solutions is now
allowed both forward and backward in time; a pre-FFT stage solution
for each antenna is sought as follows.  {\tt KRING} looks backwards to
find the last good solution whose listed SNR exceeds some multiple of
the FFT SNR threshold.  If such a solution is found, its delay and
rate solutions are adopted for the current solution and a new phase is
computed.  If the SNR of this solution exceeds the FFT SNR threshold
by the above-mentioned, {\tt KRING} skips the FFT stage.  This pre-FFT
stage is carried out for each antenna.  After going through the entire
data set, {\tt KRING} will, optionally, do a ``backwards'' pass in
which it looks forward instead of backward for solutions to
extrapolate.

A test suite of simulated data is available to check that {\tt KRING}
is producing the correct theoretical answers.  The calculation of
signal-to-noise ratio during the FFT stage has now been revised.  The
new calculation is derived in \AIPS\ Memo No.~101 (see below).  A
problem with multi-band delays was also corrected.  Users are advised
to run trial data sets through {\tt KRING} to compare with their
results from {\tt FRING} and provide feedback if there are any qualms
about this new, in principle better, task.

The new calculation of SNR implemented in {\tt KRING} will replace the
SNR calculation in {\tt FRING} in the {\tt 15APR99} release.  For this
release, the calculation of the minimum frequency increment needed was
improved.  Cases in which the lower and upper sidebands each had a
0-frequency offset channel caused particular problems.

\subsubsection{Miscellaneous corrections}

\begin{description}
\myitem{PCCOR} was corrected to read enough of the file to find a good
   record for the FQ number of the present data.  It calculates the
   highest and lowest frequencies to use independent of their order in
   the {\tt PC}-table record.
\myitem{delay calibration} \hspace{38pt} A long-standing error in a
   low-level routine caused delay corrections not to be applied if
   they were zero for the first IF channel.  Values for the other IFs
   were not checked.
\myitem{ACCOR} was corrected to prevent it from using flagged data,
   which it used happily before.
\myitem{RESEQ} is used to re-sequence antenna numbers, primarily for
   space VLBI data.  It was corrected for numerous gross errors, all
   of which caused it to fail completely.
\end{description}

\subsection{Interferometric data handling}

\subsubsection{{\tt SPLAT} and {\tt SPLIT}}

     A new task called {\tt SPLAT} was developed.  It has all of the
capabilities of {\tt SPLIT} plus two new options.  (1) It can assemble
the calibrated and edited data for selected sources into new
multi-source output file.  (2) It can average the calibrated data in
time and in groups of frequency channels.  The task can be useful in
applying the preliminary calibration of fringe rate and delay.  The
output of {\tt SPLAT} will have much less data because of the
averaging.

     {\tt SPLIT} was also changed to allow boxcar averaging of
adjacent groups of channels to produce a spectral-line output data set
of fewer channels.  The task was corrected to handle the option of
looping over subarrays even when there is no data for the source in
some of the subarrays.  Logic errors in the counting of flagged and
passed samples were corrected.  The task (and {\tt UVCOP}) were
changed to get rid of the random parameter called {\tt REMOVED}, a
disk-space wasting concept.

\subsubsection{Interactive data editing}

The {\tt EDIT} package in \AIPS\ uses the graphics channels of the TV
display to provide tools for editing \uv\ data based on the data
themselves (tasks {\tt EDITR} and {\tt SCMAP}) and on the contents of
system temperature or calibration tables (tasks {\tt EDITA} and
\hbox{{\tt SNEDT}}).  These capabilities were introduced in the
previous release and have been, with significant user input, greatly
improved in this release. The class now allows crowded data displays
in the edit area, \ie\ displays with more than one data sample at the
same horizontal TV pixel.  Flagging a single ``time'' in such a
display is actually flagging a time range.  To aid the flagging of
crowded displays, a conservative ``flag point'' and a more daring
``flag fast'' (no TV button required) edit modes were added.  The
option to display a second observable from the primary
antenna/baseline was added as were menu items to select the next
antenna and baseline.  An ``expert'' mode was added in which the
keyboard is used as the command input device rather than the TV
cursor/mouse.  You may switch easily between the expert and TV menu
modes.  The color assignments were also changed to get a more legible
display.

\subsubsection{Bandpass calibration problems}

A number of significant bugs were found in the bandpass solution tasks
{\tt CPASS} and especially \hbox{{\tt BPASS}}.  In the latter, the
worst was an addressing error that caused, in the presence of
channel-dependent flagging, good weights to be assigned to bad
channels with the highest numbered channels being lost.  The fit
routine did not provide enough information to invoke the interpolation
option when a channel was fully flagged and the interpolation method,
when invoked, did not do quite the right thing.  {\tt BPASS} and the
bandpass application routines did the correct thing when calibrator
autocorrelation spectra were applied to source autocorrelations and
when calibrator cross-correlations were applied to source
cross-correlations.  However, in the other two cases which sometimes
apply, the amplitudes were wrong by a square or square root.  The
bandpass application routines did not correctly handle the case when
an antenna is missing from the {\tt BP} table but present in the data.
Both tasks suffered from errors determining whether the data-set was
single or multi-source and what the maximum antenna number was.  When
an antenna was omitted from the {\tt AN} file, the maximum antenna
number would be wrong and the output {\tt BP} table corrupted.  (Note
--- this error was corrected in a significant number of tasks
affecting fringe fitting, polarization calibration, display tasks, and
more.)  {\tt CPASS} was corrected to write usable weights for all
polarizations and IFs, not just the first.

\subsubsection{{\tt FLGIT} and {\tt UVFLG}}

{\tt FLGIT} is a new task intended primarily to flag channel-dependent
interference from spectral-line observations.  It has been used with
considerable success for 74-MHz and P-Band observations of continuum
sources made with numerous spectral channels to allow for RFI excision
and to avoid bandwidth smearing.  {\tt FLGIT} has two modes.  For
both, it applies calibration under the usual options.  Then it flags
channels with excessive amplitude and either fits a linear baseline to
the real and imaginary spectra using a selected set of channels or
runs a median window filter through the spectrum.  Any channels with
residuals greater than user-specified limits are flagged.  The output
of the task is a calibrated (if requested) and edited data set.  A
flag table may be applied on input; none is generated.

Spurious fringes appear due to pulse cal injection.  If the fringe
rate is close to zero, these spurious fringes are not removed by the
fringe rate stopping procedure.  The time interval required to
suppress the spurious fringes was calculated by Kogan in VLBA Test
Memo No.~58, 1998.  A new option has been added to {\tt UVFLG} to flag
the calculated time interval around which the fringe rate is zero.
The length of this interval depends on the suppression factor given in
{\tt APARM(5)}, while {\tt APARM(6)} $>0$ causes a display of the
source, time, baseline, and $U,V$ of expected zero fringe rate ($U=0$).
The user should pay special attention to the PCAL tones at the
vicinity of the printed points.

\subsubsection{Miscellaneous improvements}

\begin{description}
\myitem{TI2HA} is a new task which converts the recorded times to hour
   angles.  This will allow time averaging of data taken with the same
   array on multiple days.
\myitem{PHNEG} is a new task to reverse the sign of the phase.
\myitem{UVCOP} was changed to allow up to 50000 flags to apply to a
   single time, which is ten times the normal limit.  The {\tt
   SUBARRAY} adverb was also added.
\myitem{FITTP} always writes \uv\ data as floating point so the {\tt
   DOTWO} option was dropped.
\myitem{DBCON} had an error when re-weighting two data sets in the
   same sort order, causing the data to be compromised.
\myitem{Autocorrelation phases} \hspace{70pt} actually are non-zero for
   cross-hand polarizations.  A number of tasks were fixed to allow
   for this.
\myitem{Calibration times} \hspace{45pt} used to be required to be the
   same for all antennas in an {\tt SN} or {\tt CL} table.  For this
   release, the calibration application routines were changed to allow
   each antenna to occur at its own set of times.  {\tt INDXR} now
   uses this option for {\tt USUBA}-like VLBI data sets and {\tt
   CLCAL} will write subarray 0 to the output {\tt CL} table for such
   data sets if requested.
\myitem{UVSRT} now has a much bigger sort buffer which should make it
   run faster.
\end{description}

\subsection{Imaging and modeling}

\subsubsection{Array design}

The new task {\tt CONFI} implements the algorithm to optimize antenna
array configuration by minimizing side lobes described by L. Kogan,
MMA Memo No.~171, May, 1997.  The task finds the array configuration
which has the minimum worst side lobe over a specified circular area
of the sky.  The initial configuration is taken from the input file or
calculated as\par
\vspace{-8pt}
\begin{center}
\begin{tabular}{l}
  a homogeneous configuration on the circumference; \\
  a configuration on the circumference with a progressively increasing
     spacing;  \\
  a homogeneous configuration on two circumferences; or \\
  a homogeneous configuration on the three circumference;
\end{tabular}
\end{center}
\vspace{-8pt}
The optimization is constrained in topography, in ``donuts,'' or in
two circles.  The process of finding the configuration is iterative.
At each iteration the position and value of the worst side lobe in the
given sky area is found.  Then a small correction of each element
position is provided to minimize the value of the worst side lobe.
The configuration found and relevant \uv\ coverage are plotted on the
same plot (on TV or in a file).  The coordinates of the final
configuration are recorded in the output file in normalized form or in
meters.  The task can be useful in the design of MMA, LSA, and other
arrays.

\subsubsection{Interferometric imaging}

{\tt IMAGR} was corrected to survive a missing {\tt AN} file and was
changed to use the current total iteration count over all fields to
set cutoffs and other limits.  Bill Cotton and Jim Condon have made
available a new service via the World-Wide Web for users of
{\tt IMAGR} for low frequency radio interferometer data.  This Web
service at {\tt http://www.cv.nrao.edu/~bcotton/NVSSAIPS.html} uses
the NRAO/VLA Sky Survey (NVSS) catalog to produce a list of
potentially confusing sources around a specified celestial position.
The result is in the form of an \AIPS\ RUN file that may be cut and
pasted into an \AIPS\ window that sets up the input adverbs for \AIPS\
task {\tt IMAGR} to add extra fields around these confusing sources.
Selection of source is by region of the sky and peak 20-cm flux
density after correction for the antenna gain at the frequency of
observation.  The NVSS survey was made with the NRAO VLA array at
20-cm wavelength (1.4 GHz) with a resolution of 45", full sky coverage
north of -40 degrees declination, and a limiting source peak
brightness of about 2.5 mJy (worse near the galactic plane).  The NVSS
is nearing completion and the catalog contains more than 1.8 million
sources.

{\tt FLATN} is the task which regrids multiple fields produced by {\tt
IMAGR} onto a single image and geometry.  It had minor annoyances
corrected, a new option to control the center coordinate added, and
another new option to apply radial and 3D corrections to the fields as
they are regridded.  {\tt OHGEO} had this option corrected for images
made with {\tt IMAGR}'s {\tt DO3DIMAG} option.

{\tt MWFLT} wrote its output rows one row too soon leading to errors
in the source declinations.
\eject

\subsubsection{Single-dish imaging}

The 12m telescope can process data so that the ``off'' and ``gain''
scans have scan numbers in a different range than those of the main
source.  {\tt OTFUV} was changed to read the off and gain scan numbers
from the main data while applying the scan number range ({\tt BCOUNT}
and {\tt ECOUNT}) as a limit only to the primary source.

\subsubsection{Modeling}

{\tt OMFIT}'s error bar analysis has now been tested using a suite of
simulated data to check that the analysis works correctly even when
the weights are not properly scaled.  A new model component that
allows time-variable position offsets has been added.  This model
could easily be generalized to deal with time-variable fluxes if there
is user demand.  {\tt OMFIT} now self-calibrates multiple Stokes and
IFs separately.  Also, it can solve for station-based time-independent
amplitude offsets, finding separate offsets for each Stokes and
\hbox{IF}.  This capability is not offered by either {\tt CALIB} or by
\hbox{{\tt DIFMAP}}.  These offsets are, of course, supplied with
proper error bars.  A PERL script is now available for converting {\tt
DIFMAP} model files into {\tt OMFIT} input files.  This script has
been used to convert an initial model developed using {\tt DIFMAP}
into an {\tt OMFIT} input model which was then used to refine the
model and calculate error bars.  For more details, please write to
kdesai@nrao.edu.

The Gaussian fitting tasks {\tt IMFIT}, {\tt JMFIT}, and {\tt SAD}
were given a new option to apply radial bandwidth smearing to the
Clean beam that is used as an initial guess for components and that is
used in deconvolving the fit component.  They were corrected to handle
images which lack proper coordinate information in a happier fashion.
Task {\tt CCMOD} had numerous corrections made to its model
computations, data management, and documentation.

\subsection{Data display}

\subsubsection{Truecolor TV display}

In {\tt 15OCT98}, the {\tt XAS} display server implements two
different models of TV display.  In the familiar model, the chosen
single image is loaded into 8-bit display memory by {\tt XAS} using
most but not all of the 256 values available.  A hardware look-up
table is then applied to control the pseudo-color display.  In the new
model, images in the {\tt XAS} memory are converted through separate
(software) look-up tables, summed, converted through another look-up
table, and only then moved to a 24-bit display buffer.  This is slower
than the first mode, since every time one wants to change a look-up
table, the full images have to be recomputed and re-loaded to the
display memory.  But, it is rather faster than was expected and the
new model allows different images to be used for each of the 8-bit
segments of the display memory, permitting for example, separate red,
green, and blue images to be overlayed.  This ``truecolor'' display
allows us to look at full-color images, to compare one-color images
from multiple sources (\eg\ optical, radio, X-ray), and to implement
full displays in which the intensity of the display is controlled by
one image while the color is controlled by another image.  This last,
done by verb {\tt TVHUEINT}, is very useful for zero and first moment
images of spectral cubes, for polarization intensity and angle images,
for intensity/depolarization displays, and so forth.  The logic
inherent in having separate look-up tables for separate parts of the
image, which is not available when using the hardware LUT, allows the
24-bit mode of {\tt XAS} to offer split screen functions as well.  The
verb {\tt TVROAM} will display images much larger than the TV display,
allowing you to any region the size of the display interactively.
Split screen also allows adjusting one image's LUT while looking at
both images in {\tt TVBLINK} operations.

The 24-bit mode is the default if your display is capable of it.
However, you may turn off this default with a parameter in your {\tt
.Xdefaults} file; see {\tt HELP XAS} for details of this and other
parameters.  {\tt XAS} now contains both the image catalog and the
status parameter information for its display.  Therefore, all \AIPS\
programs will know which mode the TV is in and, when appropriate, take
advantage of that mode.  In 24-bit mode (and 8-bit for that matter),
you are strongly encouraged to use shared memory.  For Solaris
systems, this will require adding a line to the file {\tt /etc/system}
saying, for example, {\tt set shmsys:shminfo\_shmmax=4194304}.  This is
4 Megabytes which is enough for normal, smaller-size screens even in
24-bit mode. 5 or 6 Megabytes may be needed for larger display
screens.  (1--2 Megabytes are needed for shared memory with 8-bit
pseudocolor.)
\eject

\subsubsection{Multiple TV displays}

An often-requested feature that has been missing from \AIPS\ is the
ability to run more than a single instance of the {\tt XAS} TV Server
---  and the corresponding tektronix, message, and TV lock d\ae mons
--- on a single host.  This is particularly desirable in environments
where a central compute server is used with X terminals, low-end
desktops, and/or PC's running non-Unix systems.  Users reducing
multiple projects have also requested the ability to have separate
TV and message servers for their separate streams of thought.  In this
release one can now do precisely this with a simple extra option on
the command line: \par
\vspace{4pt}
\centerline{\tt \$ aips tv=local \hfil}\par

will cause the system to use Unix-based sockets instead of the more
traditional Internet (or Inet) sockets for communication between {\tt
AIPS} or its tasks and the TV servers.  The limit on concurrent
instances of such ``local'' TVs is now 35 per host; this is in
addition to the one usual Inet socket based TV.  Full details are
available in \AIPS\ Memo No.~99 (see below) and in the \AIPS\ manual
page and on-line help file.  These documents tell you how to run more
than one instance of the local, Unix-socket servers and how to connect
to an instance that is already running.  Note that the servers run on
the same machine that is being used for running {\tt AIPS}  and its
tasks and can receive commands only from that computer.  However, the
windows that display the TV images, tektronix plots, and messages will
appear on the device to which the {\tt \$DISPLAY} variable is pointed.
These windows are accessible only to the servers on the main computer.

\subsubsection{Printing to the CRT}

All tasks that offer the alternative of printing to a line printer or
to the workstation window (``CRT'') pause when the CRT ``page'' is
full and ask the user for permission to continue.  Previously, the
page-full condition was defined simply by a number stored in the
system parameter ({\tt SP}) file for the computer.  Since the CRT is
almost always a workstation window of variable size these days, the
use of a fixed number is annoying.  With {\tt 15OCT98}, all \AIPS\
printing offers the option to use the current size of the window for
the next page of printing.  This is requested by setting the fixed
page size $\leq 0$ in the system parameter file.  The previous
behavior is also available; simply set the page size to the desired
positive number of lines.  But try the new option; you will like it.

\subsubsection{Miscellaneous improvements}

\begin{description}
\myitem{KLEENEX} is a new verb to exit from {\tt AIPS} while ordering
   all servers (TV, TV lock, message, Tek) to terminate.  This verb is
   strongly recommended to users of the Common Desktop Environment and
   other screen managers that attempt to checkpoint your environment
   when you exit.
\myitem{UVPLT} was changed to loop over IF and subarray as well as
   spectral channel if requested.  Bandwidth synthesis may now be
   plotted fully.
\myitem{SNPLT} now plots flagged samples with a different symbol than
   good samples.
\myitem{STARS} was confused about which axis position angle is
   measured from (should be North to East) and had no defense against
   mixing units in transposed images.
\end{description}

\section{Improvements in system matters in \RELEASENAME}

\subsection{\AIPS\ Manager related items}

\begin{description}
\myitem{ME files} The \POPS\ memory was made about four times bigger
   for compiling complex procedures.  The code was changed to use
   parameters to set the memory size and to reformat older, smaller
   \POPS\ memory structures as they are read in (with {\tt GET} or
   when starting \hbox{{\tt AIPS}}).  During installation, old {\tt
   ME} files in the {\tt DA00} areas will have to be deleted so that
   they can be replaced with the new, larger ones.  The {\tt FILAIP}
   program and {\tt SYSETUP} procedure will normally be used to make
   the new files.
\myitem{TC files} The {\tt SHOW} and {\tt TELL} communication files
   were found to be inadequate both for the number of commands that
   they could hold and for the number of users that they could
   support.  A new {\tt TC} file was created with a slightly different
   name and a new, larger structure.  Several low-level routines were
   changed so that they remove obsolete entries in this file, leaving
   room for new, potentially useful entries.  {\tt FILAIP} and {\tt
   SYSETUP} will create the new files during installation.
\myitem{DADEVS} The manner in which disks are selected for use in a
   given {\tt AIPS} run was revised.  Now, required disks are always
   included first, local disks are second, and optional disks selected
   on the command line ({\tt DA=}$\ldots$) are third.  Because NFS is
   very slow when writing message files ($\approx 1$ second per line),
   site-wide {\tt DADEVS.LIST} files should almost never specify a
   required disk.  If a user wants to have his messages and other
   basic files remain independent of the computer being used, then
   that user may specified a required disk in a personal copy of the
   disk list in a {\tt \$HOME/.dadevs} file.
\myitem{Readline} A new version of the GNU command-line-editing
   package is being shipped with \hbox{{\tt 15OCT98}}.  It appears to
   be more reliable than the older version and resolves a number of
   problems encountered on 64-bit architectures (alphas).
\myitem{AXLINUX} A new architecture has been added to \AIPS\ with this
   release, namely {\tt AXLINUX}.  This architecture represents the
   Linux operating system running on Alpha hardware (from DEC/Compaq
   or other vendors).  The port itself was relatively easy, apart from
   some minor incompatibilities with an old version of readline (see
   above).  However, the performance attainable with this combination
   has not been as good as we hoped.  As the GNU compilers become more
   mature within the Alpha environment, we expect this to change (as
   it did, rather dramatically, for \AIPS\ on Intel/Linux about a year
   ago).
\myitem{Page full} The \AIPS\ Manager should consider running {\tt
   SETSP} to change the CRT line count to $0$ for all computers so
   that the new CRT page-full handling will be invoked; see above.
\end{description}

\subsection{Programming considerations}

A number of wide-spread problems were found in the \AIPS\ code that
need to be mentioned here:

\begin{description}
\myitem{Antenna numbers} \hspace{39pt} are given in the {\tt AN} file
   as data in each record.  There is no requirement that the number of
   records match the highest antenna number or even that they be in
   numerical order.  A new routine, {\tt ANMAXA}, was written to
   return the maximum antenna number in the {\tt AN} file.  {\tt
   GETANT} was changed so that the Common parameter {\tt NSTNS} is the
   maximum antenna number and {\tt TELNO($i$) = $i$} at all times.
   Numerous tasks in bandpass and polarization calibration and fringe
   fitting failed badly in {\tt AN} files had missing or
   out-or-standard-order records.
\myitem{Multi-source} \hspace{20pt} detection in \uv\ tasks was
   confused by the presence of an {\tt SU} file.  A multi-source file
   is one with a {\tt SOURCE} random parameter ({\tt ILOCSU} $\geq 0$
   after a {\tt UVPGET} call).  A source table must be present in this
   case, but it is allowed to be present in the other case as well,
   even if its meaning is not entirely clear.  {\tt ILOC} pointers
   indicate valid data when they are $\geq 0$; numerous tasks tested
   for $> 0$ only.
\myitem{\uv\ scaling} is based on the header reference frequency.
   \uv\ values must be scaled from that frequency to the frequency of
   the data being displayed, copied, or otherwise used.  Several
   display tasks neglected to do this, although all imaging and
   calibration ones got it right.
\myitem{Baseline selection} \hspace{35pt} via the {\tt ANTENNAS} and
   {\tt BASELINE} adverbs was clarified for this release.  The cleaned
   up meanings are described in the help files.  Subroutine {\tt
   SETANT} was written to convert the user adverbs into useful Fortran
   variables, while {\tt REQBAS}, {\tt AN10RS} and {\tt BASLST} were
   revised to follow the new conventions.  These adverbs are still not
   implemented inside the {\tt UVGET} calibration package.
\myitem{Y2K} compliance was asserted for {\tt 15APR98}, but, as will
   be the case lots of places beside astronomy, a few things were
   overlooked.  The VLBA correlator was used to produce a data set
   ``observed'' and correlated in the year 2000.  The correlator
   programmers did a good job of using the new date format, including
   using it in various date-related keywords in tables.  This had been
   overlooked in \AIPS\ and is now implemented for FITS reading (now)
   and writing (after 31-Dec-1998) in \hbox{{\tt 15OCT98}}.
\myitem{Formatting} \hspace{10pt} of displayed strings was made
   simpler with the addition of a new subroutine {\tt REFRMT} which
   allows you to use a wide format (\eg\ {\tt I6}) for something that
   might be that wide but is usually only one or two digits.  A call
   to {\tt REFRMT} squeezes out the consecutive blanks.  There is a
   mechanism to keep consecutive blanks where desired as well.
\end{description}

\section{Patch Distribution}

As before, important bug fixes and selected improvements in
\RELEASENAME\ can be downloaded via the Web at:

\begin{center}
\vskip -10pt
{\tt http://www.cv.nrao.edu/aips/15OCT98/patches.html}
\vskip -10pt
\end{center}

Alternatively one can use {\it anonymous} \ftp\ on the NRAO cpu {\tt
aips.nrao.edu}.  Documentation about patches to a release is placed
in the anonymous-ftp area {\tt pub/aips/}{\it release-name} and the
code is placed in suitable subdirectories below this. Information on
patches and how to fetch and apply them is also available through the
World-Wide Web pages for \hbox{\AIPS}.  As bugs in \RELEASENAME\ are
found, the patches will be placed in the {\tt ftp}/Web area for
\hbox{{\RELEASENAME}}.  No matter when you receive your \RELEASENAME\
``tape,'' {\it you must} fetch and install these patches if you
require them.

The \OLDNAME\ release had a number of important patches.  These were
\begin{enumerate}
\item\ {\tt FITLD} makes incorrect source tables and numbers.
   1998-04-19 and 1998-04-23
\item\ {\tt TEKSRV} error for SGI and DEC computers. 1998-04-23
\item\ {\tt DBCON} reweights the wrong thing. 1998-04-28
\item\ {\tt BPASS} handles flagged channels and interpolation wrongly.
   1998-05-05, corrected 1998-06-18
\item\ {\tt IMAGR} can get wrong frequency in output image.
   1998-05-05, corrected 1998-05-06
\item\ Digital Unix version 4.0D tape problems. 1998-06-05
\item\ Y2K addition to FITS table readers. 1998-06-05
\item\ {\tt DBCON} has trouble with multiple sources. 1998-07-05
\item\ {\tt FITLD} labels LSB data incorrectly. 1998-07-06 and
   1998-07-21 and 1998-08-18
\item\ {\tt FTPGET} for SUL gets library name wrong. 1998-07-29
\item\ {\tt M3TAR} handles double-precisions wrongly.. 1998-08-18
\end{enumerate}

\section{\AIPS\ Distribution}

A total of 278 copies of the {\tt 15APR98} release were distributed to
260 sites.  Of these, 128 were in source code form and 150 were
distributed as binary executables.  This is rather more than those of
{\tt 15OCT97} (107 copies), {\tt 15OCT96} (222 copies), and {\tt
15APR97} (148 copies), perhaps reflecting the lower rate of
developments in previous releases and the new capabilities of
\hbox{{\tt 15APR98}}.  The figures on computers using \AIPS\ are
affected by the percentage of \AIPS\ users that register with
\hbox{NRAO}.  Of 260 non-NRAO sites receiving {\tt 15APR98} only 76
(29\%) have registered.  We remind serious \AIPS\ users that
registration is required in order to receive user support.
The first table below shows the breakdown of how the copies of {\tt
15APR98} were distributed and includes both source-code distributions
and binary distributions.  The second table below is based on
registered installations of {\tt 15APR98} and suggests that the
distribution over operating systems was heavily weighted toward
Solaris with Linux as a distant second.

\begin{center}
\begin{tabular}{|r|r|r|r|r|r|} \hline\hline
{ftp} & {CDrom} &{8mm} & {4mm} & {ZIP} & {Floppy} \\ \hline
224   &      45 &   6  &    3  &    0  &       0  \\ \hline\hline
\end{tabular}
\end{center}

\begin{center}
\begin{tabular}{|l|r|r|r|r|r|} \hline\hline
{Operating System} & {No.} & \texttt{15APR98}  & \texttt{15OCT97}  &
                             \texttt{15APR97}  & \texttt{15OCT96}  \\
                &       & {\%}  & {\%}   & {\%}   &  {\%}  \\ \hline
Solaris/SunOS 5 &    243   & 66 & 50 & 66 & 46  \\
PC Linux        &     70   & 19 & 23 & 16 & 19  \\
HP-UX           &      6   &  2 &  3 &  6 &  4  \\
Dec Alpha       &     26   &  7 &  9 &  6 & 10  \\
SunOS 4         &     14   &  4 & 14 &  5 & 13  \\
SGI             &     10   &  3 &  1 &  1 &  5  \\
IBM /AIX        &      2   &  1 &  0 &  0 &  4  \\
Total           &    371   &    &    &    &   \\ \hline\hline
\end{tabular}
\end{center}

%\section{DDT or How to speed up your \AIPS}

\section{Recent \AIPS\ Memoranda}

The following memoranda are available from the \AIPS\ home page.

\begin{tabular}{lp{6in}}
99 &    Multiple TV Servers on a Single Host \\
   &    Patrick~P.~Murphy \\
   &    May 29, 1998  \\
   &    Since the network-related overhaul of \AIPS\ in 1992, it has
        not been possible to have more than one instance of the ``TV''
        (image display) server or its ancillary servers on a single
        host.  This has proven to be an inconvenience in many
        environments, and debilitating for \AIPS\ users where X
        terminals or their equivalent are widely deployed.  With
        recent changes in the {\tt 15OCT98} version of \AIPS, this
        restriction has now been lifted.  This memo describes how the
        use of Unix domain sockets enables one to have multiple
        instances of the TV and other servers on a single host,
        displaying either to a single X11 display or multiple
        displays. \\
   &    \\
100 &   The Creation of \AIPS \\
   &    Eric W. Greisen \\
   &    July 27, 1998 \\
   &    At this writing, the \AIPS\ package of software has been in
        active development and use for over 19 years.  The present
        manuscript is an attempt to summarize the discussions and
        earlier software packages that led to the creation of \AIPS\
        and to describe what \AIPS\ was like during its formative
        years. \\
   &    \\
101 &   The Calculation of SNR in {\tt KRING}'s FFT stage \\
   &    Ketan M. Desai \\
   &    October 8, 1998 \\
   &    This memo describes the SNR calculation during the FFT stage
        of \hbox{{\tt KRING}}.  I describe the assumed statistical
        properties of the visibilities used for the fringe search and
        show how {\tt KRING}'s SNR estimate differs from that in
        \hbox{{\tt FRING}}.  I also discuss the probability of false
        detection, a statistic users have often requested be reported
        while fringe-fitting. \\
   &    \\
\end{tabular}

\section{\AIPS\ on CDrom}

Starting with {\tt 15APR98}, we have made \AIPS\ available on CDrom.
The initial tests using recordable CD's were very successful, and
resulted in a CD with source code, two binary versions (Linux and
Solaris), and a GNU-zipped version of the documentation ({\tt TEXT})
area.  It was possible to either perform a full installation on disk
(\ie\ copying the binaries from CD to local disk), or to run from the
\hbox{CD}.  In the latter case, the ``footprint'' on the local disk
was under 10 Megabytes!  (This figure does not include user data,
obviously.)  Furthermore, the setup script was given the ability to
switch between a ``run from CD'' installation and a ``full''
installation.  The fact that 34 Linux and 11 Solaris versions of the
CDrom were distributed suggests that this functionality, and the
availability of \AIPS\ on this new medium, will be of considerable
use to the Astronomical Community.

\eject

 \cleardoublepage\pagestyle{empty}
 \centerline{\hss\psfig{figure=FIG/AIPSORDER.PS,height=23.3cm}\hss}
 \vfill\eject
 \vbox to 4.4in{
 \vfill
% \centerline{\hss\psfig{figure=FIG/Mandrill.eps,height=2.6in}\hss}
 \vfill}
 \phantom{...}
 \centerline{\hss\psfig{figure=FIG/AIPSLETM.PS,width=\linewidth}\hss}

\end{document}
