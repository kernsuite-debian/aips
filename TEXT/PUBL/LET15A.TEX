%-----------------------------------------------------------------------
%;  Copyright (C) 2015
%;  Associated Universities, Inc. Washington DC, USA.
%;
%;  This program is free software; you can redistribute it and/or
%;  modify it under the terms of the GNU General Public License as
%;  published by the Free Software Foundation; either version 2 of
%;  the License, or (at your option) any later version.
%;
%;  This program is distributed in the hope that it will be useful,
%;  but WITHOUT ANY WARRANTY; without even the implied warranty of
%;  MERCHANTABILITY or FITNESS FOR A PARTICULAR PURPOSE.  See the
%;  GNU General Public License for more details.
%;
%;  You should have received a copy of the GNU General Public
%;  License along with this program; if not, write to the Free
%;  Software Foundation, Inc., 675 Massachusetts Ave, Cambridge,
%;  MA 02139, USA.
%;
%;  Correspondence concerning AIPS should be addressed as follows:
%;          Internet email: aipsmail@nrao.edu.
%;          Postal address: AIPS Project Office
%;                          National Radio Astronomy Observatory
%;                          520 Edgemont Road
%;                          Charlottesville, VA 22903-2475 USA
%-----------------------------------------------------------------------
%Body of intermediate AIPSletter for 31 December 2014 version

\documentclass[twoside]{article}
\usepackage{graphics}

\newcommand{\AIPRELEASE}{June 30, 2015}
\newcommand{\AIPVOLUME}{Volume XXXV}
\newcommand{\AIPNUMBER}{Number 1}
\newcommand{\RELEASENAME}{{\tt 31DEC15}}
\newcommand{\NEWNAME}{{\tt 31DEC15}}
\newcommand{\OLDNAME}{{\tt 31DEC14}}

%macros and title page format for the \AIPS\ letter.
\input LET98.MAC
%\input psfig

\newcommand{\MYSpace}{-11pt}

\normalstyle

\section{Happy 36$^{\rm th}$ birthday \AIPS}

\subsection{\Aipsletter\ publication}

We have discontinued paper copies of the \Aipsletter\ other than for
libraries and NRAO staff.  The \Aipsletter\ will be available in
PostScript and pdf forms as always from the web site listed above.
New issues will be announced in the NRAO eNews mailing and on the
bananas list server.

\subsection{Current and future releases}

We have formal \AIPS\ releases on an annual basis.  While all
architectures can do a full installation from the source files,
Linux (32- and 64-bit), Solaris, and MacIntosh OS/X (PPC and Intel)
systems may install binary versions of recent releases.  The last,
``frozen'' release is called \OLDNAME\ while \RELEASENAME\ remains
under active development.  You may fetch and install a copy of these
versions at any time using {\it anonymous} {\tt ftp} for source-only
copies and {\tt rsync} for binary copies.  This \Aipsletter\ is
intended to advise you of improvements to date in \RELEASENAME\@.
Having fetched \RELEASENAME, you may update your installation whenever
you want by running the so-called ``Midnight Job'' (MNJ) which copies
and compiles the code selectively based on the changes and
compilations we have done.  The MNJ will also update sites that have
done a binary installation.  There is a guide to the install script
and an \AIPS\ Manager FAQ page on the \AIPS\ web site.

The MNJ serves up \AIPS\ incrementally using the Unix tool {\tt cvs}
running with anonymous ftp.  The binary MNJ also uses the tool {\tt
rsync} as does the binary installation.  Linux sites will almost
certainly have {\tt cvs} installed; other sites may have installed it
along with other GNU tools.  Secondary MNJs will still be possible
using {\tt ssh} or {\tt rcp} or NFS as with previous releases.  We
have found that {\tt cvs} works very well, although it has one quirk.
If a site modifies a file locally, but in an \AIPS-standard directory,
{\tt cvs} will detect the modification and attempt to reconcile the
local version with the NRAO-supplied version.  This usually produces a
file that will not compile or run as intended.  Use a copy of the task
and its help file in a private disk area instead.

\AIPS\ is now copyright \copyright\ 1995 through 2015 by Associated
Universities, Inc., NRAO's parent corporation, but may be made freely
available under the terms of the Free Software Foundation's General
Public License (GPL)\@.  This means that User Agreements are no longer
required, that \AIPS\ may be obtained via anonymous ftp without
contacting NRAO, and that the software may be redistributed (and/or
modified), under certain conditions.  The full text of the GPL can be
found in the \texttt{15JUL95} \Aipsletter, in each copy of \AIPS\
releases, and on the web at {\tt http://www.aips.nrao.edu/COPYING}.


\section{Improvements of interest in \RELEASENAME}

We expect to continue publishing the \Aipsletter\ approximately every
six months, but the publication is now primarily electronic.  There
have been several significant changes in \RELEASENAME\ in the last six
months.  Some of these were in the nature of bug fixes which were
applied to \OLDNAME\ before and after it was frozen.  If you are
running \OLDNAME, be sure that it is up to date; pay attention to the
patches and run a MNJ any time a patch relevant to you appears.  New
tasks in \RELEASENAME\ include {\tt SNFIT} to fit Gaussians to
drift-scan gain solutions, {\tt UVFRE} to re-grid visibility data in
frequency space to match a second data set, {\tt HOLOG} to transform
and analyze holography data, {\tt PANEL} to convert {\tt HOLOG} output
to panel adjustment tables, {\tt STACK} to combine multiple images
without regard for coordinates, and {\tt TVHLD} to display and save
histogram-equalized versions of images.  New verb {\tt TVLAYOUT} also
assists in displaying holography results and new procedure {\tt
  DOVLAMP} assists in converting EVLA SysPower data to gains for VLB
usage.

{\tt 31DEC14} contains a change to the ``standard'' random parameters
in $uv$ data and adds columns to the {\tt SN} table.  Note, however,
that the random parameters written to FITS files have not been changed.
Older releases of \AIPS\ cannot handle the new {\it internal} $uv$
format and might be confused by the {\tt SN} table as well.  {\tt
  31DEC09} contains a significant change in the format of the antenna
files, which will cause older releases to do wrong things to data
touched by {\tt 31DEC09} and later releases.  You are encouraged to
use a relatively recent version of \AIPS, whilst those with EVLA data
to reduce should get release \OLDNAME\ or, preferably, the latest
release.

\subsection{UV-data}

\subsubsection{Dispersion and fringe fitting}

Wide-band observations, particularly those at low frequencies, are
affected by ionospheric dispersion (phases change linearly with
wavelength) as well as delay errors (phases change linearly with
frequency).  The calibration ({\tt CL}) table has contained
dispersion columns, used by {\tt TECOR}, for a long time.  The current
release was changed to have dispersion columns also in the solution
({\tt SN}) table. {\tt CLCAL} was also changed to smooth and apply
dispersions from the {\tt SN} to the {\tt CL} table.  It was corrected
to do the advertised smoothing of the multi-band delay as well.

{\tt SNSMO} was changed to smooth, clip, and re-reference dispersions
as well as other {\tt SN}-table data.  {\tt SNSMO} was also changed to
handle phases more carefully, keeping track of groups of IFs so that
the phase relationship between IFs in a group is not lost when delays
are smoothed.  A new {\tt SMOTYPE = 'VLDE'} is used to request this
improved smoothing method.

If there is dispersion, then the single-band delays found by {\tt
  FRING} will depend on frequency even if the real delay is actually
the same for all bands.  The single-band delays may then be converted
to a single multi-band delay plus a dispersion by a relatively simple
least-squares routine.  {\tt OPTYPE = 'DISP'} was added to {\tt MBDLY}
to perform this computation on {\tt SN} tables computed as single-band
delays by {\tt FRING} and {\tt KRING}\@.  An option was also added to
{\tt FRING} to perform this same computation following the fit for
single-band delays.  {\tt FRING} was also corrected to handle cases in
which the spectral increment is of opposite sign in different IFs
(\eg\ ALMA)\@.

The alternative fringe-fitting task {\tt KRING} contains a
least-squares routine designed, among other things, to solve for a
multi-band delay plus dispersion.  This routine was adjusted and the
output into the {\tt SN} table corrected to produce correct phases,
delays, and dispersions.  These corrections affect the output
substantially even when not solving for dispersion.  Code was added to
average the data over time when not solving for rates.  This produces
a great improvement in the run time.  The task was also changed to
work for single-source files as well as multi-source.  Unfortunately,
the least-squares routine can return wrong answers some of the time.
Why it does so remains under investigation, so care should be
exercised in using {\tt KRING} outputs.  The task does offer the only
routine to use all the phases when solving for dispersion; {\tt MBDLY}
and {\tt FRING} only use the single-band delays to find dispersion.

\vfill\eject
\subsubsection{PIPEAIPS}
The {\tt RUN} file procedure {\tt DOOSRO} (formerly called {\tt
  VLARUN}) has been renamed {\tt PIPEAIPS}\@. When given the expected
format of data, {\tt PIPEAIPS} will take most connected element
interferometer data and calibrate them according to known standard
procedures. Optionally it will make images automatically and provide
some means of quality control.  It handles both continuum and spectral
line data flexibly as well as a wide range of observational setups.
Polarization calibration however is not built in since that currently
requires much hands-on massaging.

Like any pipeline, {\tt PIPEAIPS} will only do what it is designed to
do.  The anticipated typical use of {\tt PIPEAIPS} is to calibrate
data from the VLA, but also ALMA, ATCA, WSRT, etc.~in a first pass to
provide feedback on bad data:  This can be done blindly, without any
prior flagging if desired.  After (additional) flagging, the procedure
can be (re)run with imaging switched on.  The resulting data can be
inspected, the pipeline can be rerun, and/or the calibrated data can
be re-imaged with the final imaging parameters.

As an example, most of the old VLA continuum data in the archive was
imaged in a single run of (then) {\tt VLARUN}, \ie\ without additional
flagging or self-calibration.  These data sets and their images, which
constitute the NRAO VLA Archive Survey (NVAS), are publicly available
from the NRAO data archive or more directly via {\tt
  https://archive.nrao.edu/nvas} and are a great resource for a quick
look or as starting point for further imaging (\eg\ by combining
multiple of these calibrated data sets on the same source in the same
frequency band).

For inquiries or help with {\tt PIPEAIPS}, please contact
{\tt lsjouwer@nrao.edu} or {\tt daip@nrao.edu}.

\subsubsection{Other matters}
\begin{description}
\myitem{UV} data inside \AIPS\ now have random parameters {\tt
       ANTENNA1}, {\tt ANTENNA2}, and {\tt SUBARRAY} rather than the
       {\tt BASELINE} parameter that combined all three.  All \AIPS\
       tasks can handle either of these forms and the FITS writers
       {\tt FITTP} and {\tt FITAB} will continue to write data out
       using the {\tt BASELINE} form when possible.  The new format
       allows for more than 255 antennas, as planned for some of the
       array designs now being considered.
\myitem{Calibration} routines required several bug fixes.  The routine
       that fetched the next gain solution used a time increment of
       0.4 seconds which is way too long when data are at intervals of
       0.1 seconds.  The computation of wavelengths for polarization
       and dispersion calibration assumed a frequency reference pixel
       of 1.0 rather than using the actual value, which is now usually
       at the center of each band.  Dispersions were applied to the
       data only when both dispersion values were non-zero, thereby
       leaving out all baselines to the reference antenna.  There was
       also a bug in the I/O initialization routine which only
       mattered if the data file on disk was exactly the minimum size
       required.
\myitem{BPASS} did not use left-handed antenna coordinates in shifting
       VLBI data, making bandpass solutions for high frequency, narrow
       channel data incorrect.  The low-level routines were changed to
       use right-handed coordinates (like the rest of \AIPS) and the
       bandpass application routines were changed accordingly.
\myitem{Sorting} of data in the OOP routines (such as {\tt IMAGR})
       uses the pseudo-AP, but the communication between low-level
       routines was badly flawed.  Sorting with {\tt OOSRT} and inside
       {\tt IMAGR} has now been fixed, but must not have been used
       much previously.
\myitem{FITLD} was changed to set the {\tt RDATE} table keyword to the
       user-specified reference date (if any) including setting the
       GST at midnight and dealing with date offsets in computing the
       source apparent positions.  The handling of {\tt DIGICOR} was
       changed, to tell the truth about what the task is doing and to
       allow non-VLBA correlators to get the corrections.
\myitem{UVSUB} now offers the option to divide cross-hand polarization
       data by the parallel-hand model using {\tt OPCODE = 'DIV4'}\@.
\myitem{MORIF} was given the option to combine IFs as well as to
       sub-divide them.
\myitem{UVFRE} is a new task to re-grid the frequency structure of one
       data set (as best one can) to match that of another data set.
\myitem{UVIMG} was re-written to grid on any user-selected axes with
       channel averaging and increments and including multiple IFs.
       It can grid the data with a variety of gridding functions or
       output the count of samples instead.  The output may be an
       interpolated or a convolved image.
\myitem{SOUSP} was given the option of correcting a range of {\tt SN}
       table versions if the new fluxes replace the previous fluxes in
       the source table.
\myitem{DOVLAMP} is a new {\tt RUN}-file procedure to produce
       amplitude calibration for phased-VLA data used in VLBI\@.
\myitem{CLCOR} was changed to allow the {\tt 'EOPS'} operation to
       request day offset 0, which is useful when the first scans occur
       just before midnight.
\myitem{TECOR} was changed to use values from the JPL files at time 0
       hours rather than the data at time 24 hours of the previous
       day.  The two are not the same and the 0 hours one is likely to
       have been computed with additional information.
\myitem{SNFIT} is a new task to fit a primary-beam model to {\tt SN}
       tables generated for a source that moves through the primary
       beam (\eg\ drift or driven scans).
\myitem{TVFLG,} {\tt SPFLG}, and {\tt FTFLG} were corrected to prevent
       them from trying to load too much data when switching from one
       of multiple ``pieces'' back to a single (time-averaged) piece.
\myitem{LISTR} will now list autocorrelation data in {\tt 'LIST'} with
       an extra factor of 0.01 in the scaling, when {\tt DOACOR} is
       true.
\end{description}

\subsection{Holography and other analysis}

The analysis of holography data was given attention.  Two tasks, kept
by Rick Perley, were brought into \AIPS\ and adjusted to simplify
inputs and meet plot standards.  {\tt HOLOG} is a simplified version
of {\tt HOLGR} to read the output of {\tt UVHOL} and do FFTs to make
images.  Ten different types of image may be obtained including
regridded amplitude and phase, regridding weights, amplitude and phase
of the antenna illumination, amplitude and phase of the point-spread
function, focus model phase corrections, surface deviations, and
interpolated antenna power pattern.  Then {\tt PANEL} takes the
illumination amplitude as a mask and the surface deviations as the
image to make plots of the raw data, adjustment image, and residual
image and tables of the adjustments to be applied at each panel
corner.  Verb {\tt TVLAYOUT} was written to overlay the panel layout
on images from {\tt HOLOG} loaded to the TV and tasks {\tt KNTR} and
{\tt GREYS} were given enhanced capability to overlay the panel layout
on their plots.

\begin{description}
\myitem{STACK} is a new task to do image ``stacking'' in which small
        images of selected positions are added to see if an object
        below the noise level of any one image will appear.
        Basically, it just does a weighted average (or median) of a
        set of images from a cube or a range of sequence numbers.
\myitem{MCUBE,} {\tt FQUBE} and {\tt STUFFER} were changed to deal
        with missing sequence numbers between {\tt INSEQ} and {\tt
          IN2SEQ} without quitting.
\myitem{OMFIT} now quotes uncertainties $\sqrt{2}$ less than before.
        The new values match uncertainties found by other means more
        closely.
\myitem{UVFIT} now says whether it converged or not.
\end{description}
\vfill\eject

\subsection{Display}

\begin{description}
\myitem{TVHLD} is a new task to replace the old tasks {\tt TVHXF} and
        {\tt TVHLD}, both of which required well-equipped IIS Model 70
        display devices.  It loads an image on the TV with ``histogram
        equalization.''  An interactive menu is used to control the
        intensity range of the equalization and to select various
        functions to be performed on the normal histogram before it is
        used to do the equalization.  Some of these functions enhance
        the higher levels of the image relative to the lower levels
        more than others.  The usual image display/enhancement options
        are also offered.  When done, the equalized image may be
        written to a cataloged image file or the task may exit without
        writing an image.
\myitem{TVMOVIE} was given the option {\tt DOALL} to allow the movie
        to run over the planes of more than one axis.  Since {\tt
          UVIMG} can now write cubes with both a frequency and an IF
        axis of more than one pixel, this option became advisable.
\end{description}

\subsection{Imaging}

{\tt IMAGR} no longer returns control to the user when {\tt DOTV} is
true.  This allows it to check with the user when a requested {\tt
  TVBOX} would wipe out more than one previously existing box.  If
{\tt OBOXFILE} is blank at the start, the task now creates a temporary
file in the user's home area.  This temporary file, or a
user-specified one, may be replaced later with a {\tt TELL}
operation.  {\tt IMAGR} was also corrected to avoid, but also handle,
a situation in which the dynamic memory allocated is inadequate for
the number of channels or facets in a pass of gridded subtraction.
A bug in {\tt IMAGR}'s baseline-length time averaging was also
corrected.  Previous usage of that option could have been severley
compromised.

{\tt SCIMGF} and {\tt SCMAP} were changed to support image labeling
including ``stars,'' the more general {\tt OBOXFILE} usage described
above, and to check with the user whenever {\tt TVBOX} would delete
more than one pre-existing box.

\subsection{General}

The Max OS/X operating system, in version 10.10 called ``yosemite,''
changed the standard directory {\tt /tmp} into {\tt /private/tmp}.
Changes to the {\tt START\_AIPS} and {\tt START\_TVSERVERS} scripts as
well as to start-up code in {\tt XAS} itself were required to handle
this so that the simple command {\tt aips} would again bring up the TV
as expected.  The background of characters written to grey-scale TV
planes now remains black when the image plane is enhanced.

The \POPS\ language processor would occasionally produce a strange
result when strings were being concatenated in procedures.  A wrong
pointer was returned to the concatenation subroutine.  The bug was
fixed by ignoring this pointer.  The data were already stored in
memory and did not need to be stored a second time in the same place.

Task {\tt MOVE} has been changed to use large I/O buffers, reducing
the I/O count and achieving a substantial improvement in performance
on some I/O systems.  A factor of 10 in real time was obtained on a
Lustre system; local Linux systems had much less improvement in real
time but had a substantial reduction in cpu time.

A nuber of sites have tried to compile \AIPS\ using relatively modern
versions of {\tt gfortran} and {\tt gcc} and have encountered
difficulties.  We have found that the default {\tt gfortran} included
with RedHat 6 Linux (version 4.4.7) does seem to work although the
default version of the debugger ({\tt gdb} version 7.2) does not play
well with that version.  Later versions of {\tt gfortran} produce code
in which (at least) Fortran text-file I/O and optimised pseudo-AP code
fail.

\vfill\eject
\section{Recent \AIPS\ Memoranda}

All \AIPS\ Memoranda are available from the \AIPS\ home page.  \AIPS\
Memo 117 describing \AIPS' usage of the FITS format was modified
in 2015 for changes to the UV and {\tt SN} table formats.

\begin{tabular}{lp{5.8in}}
{\bf 117} & {\bf \AIPS\ FITS File Format}\\
   &  Eric W. Greisen, NRAO\\
   &  April 24, 2015, revision\\
   &  \AIPS\ has been writing images and $uv$ data in FITS-format files
  for a very long time.  While these files have been used widely in
  the community, there is a perception that a detailed document in
  still required.  This memo is an attempt to meet that perception.
  \AIPS\ FITS files for $uv$ are conventions layered upon the standard
  FITS format to assist in the interchange of data recorded by
  interferometric telescopes, particularly by radio telescopes such as
  the EVLA and VLBA\@.

  \vspace{4pt}

  The Memo was revised for changes to the {\tt SU} table ({\tt RAOBS}
  and {\tt DECOBS} columns), the {\tt FQ} table ({\tt BANDCODE}
  column), the {\tt AN} table (clarification of coordinate systems),
  clarifications of when source tables and columns or random
  parameters are required, the {\tt CC} table (clarify meaning of {\tt
    DELTAX} and {\tt DELTAY}), the $uv$ data format (possible addition
  of {\tt SUBARRAY}, {\tt ANTENNA1} and {\tt ANTENNA2} random
  parameters), and the {\tt SN} table (addition of {\tt DISP} and
  {\tt DDISP} columns).
\end{tabular}


\section{\AIPS\ Distribution}

We are now able to log apparent MNJ accesses and downloads of the tar
balls.  We count these by unique IP address.  Since some systems
assign the same computer different IP addresses at different times,
this will be a bit of an over-estimate of actual sites/computers.
However, a single IP address is often used to provide \AIPS\ to a
number of computers, so these numbers are probably an under-estimate
of the number of computers running current versions of \AIPS\@. In
2015, there have been a total of 662 IP addresses so far that have
accessed the NRAO cvs master.  Each of these has at least installed
\AIPS\ and 229 appear to have run the MNJ on \RELEASENAME\ at
least occasionally.  During 2015 more than 233 IP addresses have
downloaded the frozen form of \OLDNAME, while more than 614 IP
addresses have downloaded \RELEASENAME\@.  The binary version was
accessed for installation or MNJs by 262 sites in \OLDNAME\ and 500
sites in \RELEASENAME\@.  A total of 1131 different IP addresses have
appeared in one of our transaction log files.  The \RELEASENAME\
numbers are rather higher than last year; the others are about the
same as last year at this time.

\centerline{\resizebox{!}{3.75in}{\includegraphics{FIG/PLOTIT15a.PS}}}

\vfill\eject
\section{Patch Distribution for \OLDNAME}

Important bug fixes and selected improvements in \OLDNAME\ can be
downloaded via the Web beginning at:

\begin{center}
\vskip -10pt
{\tt http://www.aoc.nrao.edu/aips/patch.html}
\vskip -10pt
\end{center}

Alternatively one can use {\it anonymous} \ftp\ to the NRAO server
{\tt ftp.aoc.nrao.edu}.  Documentation about patches to a release is
placed on this site at {\tt pub/software/aips/}{\it release-name} and
the code is placed in suitable sub-directories below this.  As bugs in
\NEWNAME\ are found, they are simply corrected since \NEWNAME\ remains
under development.  Corrections and additions are made with a midnight
job rather than with manual patches.  Since we now have many binary
installations, the patch system has changed.  We now actually patch
the master version of \OLDNAME, which means that a MNJ run on
\OLDNAME\ after the patch will fetch the corrected code and/or
binaries rather than failing.  Also, installations of \OLDNAME\ after
the patch date will contain the corrected code.

The \OLDNAME\ release has had a number of important patches:
\begin{enumerate}
   \item\ {\tt DOFARS} procedure and inputs retained adverbs no longer
          used by {\tt FARS}. {\it 2015-01-02}
   \item\ {\tt RMFIT} failed to copy the {\tt FQ} table to the output
          residual images. {\it 2015-01-13}
   \item\ {\tt FTFLG} messed up antenna numbers in the output {\tt FG}
          table when a single antenna or baseline was used.
          {\it 2015-01-16}
   \item\ {\tt VLBATECR} had trouble around the end of the year
          deciding what files needed to be downloaded. {\it
            2015-01-19}
   \item\ {\tt VLBARUN} requires {\tt convert} to make the html output
          files.  Added tests and a control on {\tt DOTV}.
          {\it 2015-01-19}
   \item\ {\tt BPASS} had an array and history writing code which
          could not handle more than 32 antennas. {\it 2015-01-20}
   \item\ {\tt MORIF} messed up bandpass and other spectral tables.
          {\it 2015-01-29}
   \item\ {\tt TYSMO} did not apply clipping unless median-window
          filtering was also requested. {\it 2015-02-11}
   \item\ {\tt BPASS} did not understand that low-level VLB routines
          expected left-handed antenna coordinates.
          {\it 2015-02-21}
   \item\ UV data disk I/O had an issue if the full file fit into the
          first of the two buffers and the disk file was exactly the
          right size! {\it 2015-04-24}
   \item\ {\tt START\_AIPS, START\_TVSERVERS} needed a grammar change to
          support Mac yosemite systems. {\it 2015-04-24}
   \item\ Basic calibration subroutines mis-computed the channel
          wavelengths and did not always do the dispersion correction.
          {\it 2015-05-19}
   \item\ {\tt XAS} needed a change to support Mac yosemite systems
          and also to write black character backgrounds.
          {\it 2015-05-20}
   \item\ {\tt KRING} help file adverb list did not match that of the
          Fortran. {\it 2015-06-17}
   \item\ {\tt IMAGR} did not do the baseline-based time averaging
          properly for the last samples written to the work file.
          {\it 2015-06-22}
\end{enumerate}
\vfill\eject

% mailer page
% \cleardoublepage
\pagestyle{empty}
 \vbox to 4.4in{
  \vspace{12pt}
%  \vfill
\centerline{\resizebox{!}{3.2in}{\includegraphics{FIG/Mandrill.eps}}}
%  \centerline{\rotatebox{-90}{\resizebox{!}{3.5in}{%
%  \includegraphics{FIG/Mandrill.color.plt}}}}
  \vspace{12pt}
  \centerline{{\huge \tt \AIPRELEASE}}
  \vspace{12pt}
  \vfill}
\phantom{...}
\centerline{\resizebox{!}{!}{\includegraphics{FIG/AIPSLETS.PS}}}

\end{document}
