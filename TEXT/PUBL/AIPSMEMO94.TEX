%-----------------------------------------------------------------------
%;  Copyright (C) 1997
%;  Associated Universities, Inc. Washington DC, USA.
%;
%;  This program is free software; you can redistribute it and/or
%;  modify it under the terms of the GNU General Public License as
%;  published by the Free Software Foundation; either version 2 of
%;  the License, or (at your option) any later version.
%;
%;  This program is distributed in the hope that it will be useful,
%;  but WITHOUT ANY WARRANTY; without even the implied warranty of
%;  MERCHANTABILITY or FITNESS FOR A PARTICULAR PURPOSE.  See the
%;  GNU General Public License for more details.
%;
%;  You should have received a copy of the GNU General Public
%;  License along with this program; if not, write to the Free
%;  Software Foundation, Inc., 675 Massachusetts Ave, Cambridge,
%;  MA 02139, USA.
%;
%;  Correspondence concerning AIPS should be addressed as follows:
%;         Internet email: aipsmail@nrao.edu.
%;         Postal address: AIPS Project Office
%;                         National Radio Astronomy Observatory
%;                         520 Edgemont Road
%;                         Charlottesville, VA 22903-2475 USA
%-----------------------------------------------------------------------
\documentstyle [twoside]{article}
%
\newcommand{\AIPS}{{$\cal AIPS\/$}}
\newcommand{\AMark}{AIPSMark$^{(93)}$}
\newcommand{\AMarks}{AIPSMarks$^{(93)}$}
\newcommand{\AM}{A_m^{(93)}}
\newcommand{\whatmem}{\AIPS\ Memo \memnum}
\newcommand{\boxit}[3]{\vbox{\hrule height#1\hbox{\vrule width#1\kern#2%
\vbox{\kern#2{#3}\kern#2}\kern#2\vrule width#1}\hrule height#1}}
%
\newcommand{\memnum}{94}
\newcommand{\memtit}{\AIPS\ Benchmarks for the Silicon Graphics
Origin200}
\title{
%   \hphantom{Hello World} \\
   \vskip -35pt
%   \fbox{AIPS Memo \memnum} \\
   \fbox{{\large\whatmem}} \\
   \vskip 28pt
   \memtit \\}
\author{Athol Kemball and Chris Flatters}
%
\parskip 4mm
\linewidth 6.5in                     % was 6.5
\textwidth 6.5in                     % text width excluding margin 6.5
\textheight 8.91 in                  % was 8.81
\marginparsep 0in
\oddsidemargin .25in                 % EWG from -.25
\evensidemargin -.25in
\topmargin -.5in
\headsep 0.25in
\headheight 0.25in
\parindent 0in
\newcommand{\normalstyle}{\baselineskip 4mm \parskip 2mm \normalsize}
\newcommand{\tablestyle}{\baselineskip 2mm \parskip 1mm \small }
%
\begin{document}
\pagestyle{myheadings}
\thispagestyle{empty}

%\newcommand{\Rheading}{\whatmem \hfill \memtit \hfill Page~~}
%\newcommand{\Lheading}{~~Page \hfill \memtit \hfill \whatmem}
%\markboth{\Lheading}{\Rheading}
%
%

\vskip -.5cm
\pretolerance 10000
\listparindent 0cm
\labelsep 0cm
%
%

\vskip -30pt
\maketitle
\vskip -30pt
\normalstyle

\begin{abstract}

We have run \AIPS\ benchmarks on the Silicon Graphics Origin200, an entry
level, shared-memory multiprocessor system.  We achieved a single-user
\AMark\ of $13.7$ and were able to run four simultaneous DDTs (on a
four-processor system) with only a slight loss of speed.

\end{abstract}

%\renewcommand{\topfraction}{0.85}
\renewcommand{\floatpagefraction}{0.75}
%\addtocounter{topnumber}{1}
\typeout{bottomnumber = \arabic{bottomnumber} \bottomfraction}
\typeout{topnumber = \arabic{topnumber} \topfraction}
\typeout{totalnumber = \arabic{totalnumber} \textfraction\ \floatpagefraction}

\section{Introduction}

The Silicon Graphics Origin line of servers are shared-memory,
symmetric multiprocessing computers based on the MIPS R10000
processor.  Options range from the Origin200, which can be configured
with up to 4 CPUs, to the CRAY Origin2000, which can be configured
with up to 128 CPUs.

An Origin200 server, belonging to SGI's benchmarking group,  was made
available to us for testing.  This machine was configured with four
R10000 processors, running at 180 MHz and 1 GByte of main memory.
Each processor had 1 MByte of secondary cache, which is standard for
this system. The disk system comprised seven 4.5 GB disks with two
SCSI II controllers concatenated in software to form a striped
volume of $\sim 30$ GB.

The tests were run using {\tt 15OCT96} with some components from
{\tt 15APR97} used for the VLB DDT tests.  The AIPS executables were
compiled on a Power Challenge machine using version 7.0 of the SGI
compilers under IRIX 6.2 (the test machine was running IRIX 6.4).
The executables were compiled with level-2 optimization ({\tt -O2}
for version 4 of the MIPS instruction set ({\tt -mips4}) using 32-bit
addressing ({\tt -n32}).

\section{Basic DDT Results}

The large DDT was run twice, we had exclusive use of the machine for
the second run while the machine had one other user during the first
run.  The timing results for both runs are as follows.

\begin{tabular}{|l|r|r|r|r|}
\hline
\multicolumn{1}{|c|}{\bf{Test}} & \multicolumn{2}{c|}{\bf{Run 1}}
	& \multicolumn{2}{c|}{\bf{Run 2}} \\
\cline{2-5}
& \multicolumn{1}{|c|}{CPU time} & \multicolumn{1}{c|}{Elapsed time}
	& \multicolumn{1}{c|}{CPU time}
	& \multicolumn{1}{c|}{Elapsed time} \\
\hline
$T_{large}$/sec & --- & 302.0 & --- & 292.0 \\
$\AM$	  	& --- & 13.2 & --- & 13.7 \\
\hline
$T_{\tt APCLN}$/sec & 52.22 & 53.0 & 52.01 & 53.0 \\
$T_{\tt CALIB}$/sec & 17.63 & 18.0 & 17.25 & 18.0 \\
$T_{\tt MXCLN}$/sec & 73.27 & 75.0 & 72.77 & 74.0 \\
$T_{\tt VTESS}$/sec & 20.91 & 30.0 & 20.52 & 22.0 \\
\hline
\end{tabular}

The $u,v$ datasets generated during the DDT were not found to differ
significantly from the master datasets.  The number of correct bits in
the image tests are shown here.

\begin{tabular}{|l|r|r|r|r|}
\hline
\multicolumn{1}{|c|}{\bf Image} &  \multicolumn{1}{c|}{Max. Error}
	& \multicolumn{1}{c|}{RMS Error} \\
\hline
UVMAP & 14.8164 & 21.0378 \\
UVBEAM & 20.6107 & 26.0573 \\
APCLN & 11.7986 & 17.1534 \\
APRES & 13.7520 & 20.7065 \\
MXMAP & 16.4649 & 22.5102 \\
MXBEAM & 20.9163 & 27.0501 \\
MXCLN & 12.2863 & 16.7303 \\
VTESS & 22.7466 & 30.9920 \\
\hline
\end{tabular}

\section{DDT Throughput Tests}

In order to test the amount of degradation we might expect from multiple
simultaneous users we ran 4 DDTs in parallel.  Each was run from a
separate terminal window using a different AIPS user number so that
there would be no contention for message file locks.  We did not have
exclusive use of the machine for this test: one other CPU intensive
process was started while the test was running.

\begin{tabular}{|l|r|r|r|r|}
\hline
\multicolumn{1}{|c|}{\bf Test} &
\multicolumn{4}{c|}{\bf Results for Each DDT} \\
\cline{2-5}
 & \multicolumn{1}{c|}{1} & \multicolumn{1}{c|}{2} &
   \multicolumn{1}{c|}{3} & \multicolumn{1}{c|}{4} \\
\hline
$T_{large}$/sec & 357.0 & 354.0 & 329.0 & 314.0 \\
$\AM$		& 11.2 & 11.3 & 12.2 & 12.7 \\
\hline
$T_{\tt APCLN}$/sec 	& 53.0 & 53.0 & 59.0 & 56.0 \\
$T_{\tt CALIB}$/sec	& 19.0 & 18.0 & 24.0 & 19.0 \\
$T_{\tt MXCLN}$/sec	& 73.0 & 75.0 & 86.0 & 80.0 \\
$T_{\tt VTESS}$/sec	& 21.0 & 21.0 & 25.0 & 29.0 \\
\hline
\end{tabular}

The worst-case degradation from the single-user time was 22.3\%; the
mean degradation was 15.9\%.

\section{VLB DDT Results}
 The performance of {\tt BLING} was tested using simulated data
generated by {\tt DTSIM}, as implemented in the run file {\tt VLBDDT}
in the {\tt 15APR97} version of {\tt AIPS}. For one processor the
elapsed time per single baseline in the {\tt VLBDDT} test was 95
seconds. This compares with a previously determined value of 185
seconds on a system with an $\AM$ of $\sim 6$, and is thus consistent
with the overall DDT results presented above.

\section{Acknowledgements}
 We would like to thank Steve Simonds and Ed Hernandez of the
Albuquerque SGI office for making the benchmark tests possible.



\end{document}


