%-----------------------------------------------------------------------
%;  Copyright (C) 1995
%;  Associated Universities, Inc. Washington DC, USA.
%;
%;  This program is free software; you can redistribute it and/or
%;  modify it under the terms of the GNU General Public License as
%;  published by the Free Software Foundation; either version 2 of
%;  the License, or (at your option) any later version.
%;
%;  This program is distributed in the hope that it will be useful,
%;  but WITHOUT ANY WARRANTY; without even the implied warranty of
%;  MERCHANTABILITY or FITNESS FOR A PARTICULAR PURPOSE.  See the
%;  GNU General Public License for more details.
%;
%;  You should have received a copy of the GNU General Public
%;  License along with this program; if not, write to the Free
%;  Software Foundation, Inc., 675 Massachusetts Ave, Cambridge,
%;  MA 02139, USA.
%;
%;  Correspondence concerning AIPS should be addressed as follows:
%;          Internet email: aipsmail@nrao.edu.
%;          Postal address: AIPS Project Office
%;                          National Radio Astronomy Observatory
%;                          520 Edgemont Road
%;                          Charlottesville, VA 22903-2475 USA
%-----------------------------------------------------------------------
%Body of \AIPS\ Letter for 15 July 1994

\documentstyle [twoside]{article}

\newcommand{\AMark}{AIPSMark$^{(93)}$}
\newcommand{\AMarks}{AIPSMarks$^{(93)}$}
\newcommand{\LMark}{AIPSLoopMark$^{(93)}$}
\newcommand{\LMarks}{AIPSLoopMarks$^{(93)}$}
\newcommand{\AM}{A_m^{(93)}}
\newcommand{\ALM}{AL_m^{(93)}}

\newcommand{\AIPRELEASE}{July 15, 1994}
\newcommand{\AIPVOLUME}{Volume XIV}
\newcommand{\AIPNUMBER}{Number 2}
\newcommand{\RELEASENAME}{{\tt 15JUL94}}

%macros and title page format for the \AIPS\ letter.
\input LET94.MAC
\input psfig

\newcommand{\MYSpace}{-11pt}

\normalstyle

\section{The Good News $\ldots$}

The \RELEASENAME\ release of Classic \AIPS\ is now available.  Contact
Ernie Allen at any of the addresses given in the masthead to obtain a
copy.  As of this writing, 80 copies of the {\tt 15JAN94} release
have been given out electronically (32 tar.Z, 31 tar.gz) or on
magnetic tape (10 8mm, 4 4mm, 2 QIC, 1 9-track).    A new port to
Silicon Graphics computers was developed for this release, and the
magnetic-tape portions of the ports of \AIPS\ to DEC's alpha computer
(OSF/1 operating system) and the Linux system on PCs were debugged.
In addition to the ports, we have written a number of new tasks,
enhanced the transmission and calibration of VLBA data, and improved
the automatic source finding and fitting task \hbox{{\tt SAD}}.  The
large job of rewriting and updating the \Cookbook\ was begun, with the
new and remaining old chapters being made available on-line through
the World-Wide Web.

The NRAO is advertising a support scientist position in the \AIPTOO\
group at the Array Operations Center in Socorro, New Mexico.  If you
are interested, or know anyone who might be interested, please contact
Ina Cole at the NRAO in Socorro (505--835--7309 or {\tt
icole@nrao.edu}).  The position we advertised in the Classic \AIPS\
group has been filled by Athol Kemball, an expert in VLBI
spectral-line polarimetry and an experienced \AIPS\ programmer.  We
would like to thank all those who applied for the position.  The high
caliber of the applicants was very gratifying.

The IAU FITS Working Group approved the binary tables extension to the
FITS format conventions on June 10, 1994.  At long last, this gives
official approval at the highest possible level to the tables format
we have been using in \AIPS\ since about 1987.  This format was
designed initially by Bill Cotton to cope with the need for greater
efficiency than one could obtain with ASCII tables and with the need
for columns in tables to contain arrays as well as scalars.  Inside
\AIPS\ these tables have been called {\tt A3DTABLE}, but the new
official name is \hbox{{\tt BINTABLE}}.  (This change is the current
reason for the adverb {\tt DONEWTAB} in {\tt FITTP} since {\tt
BINTABLE} is understood only by the {\tt 15JAN94} and later versions
of \AIPS.)

The {\tt DDT} test has been run on a Silicon Graphics Indigo 1
computer (33 MHz MIPS R3000 cpu, 16 Mbyte memory) and an Indigo 2
(150 MHz MIPS R4400 cpu, 32 Mbyte memory) by Jeff Pedelty (GSFC).  The
\AMarks\ achieved were {\bf 1.18} and {\tt 2.66}, respectively.  Both
of these should increase with larger memory and SCSI-2 fast/wide
disks.  A new record \AMark\ of {\bf 3.89} was acheived on a DEC Alpha
3000 model 600 with 128 Mbyte of memory, 250 Mbytes of swap space, and
OSF/1 version 1.3.  The tests were carried out by Scott E. Aaron and
Jing Ping Ge of Brandeis.

\section{$\ldots$ and the Bad}
%?\TOO

\vfill\eject

\section{Improvements for Users in 15JUL94}

%This release is characterized more by numerous program improvements
%and bug fixes, rather than by significant volumes of new code.

\subsection{VLBI data processing}

Two of the new tasks in \RELEASENAME\ are designed to correct format
limitations in VLBA data.  {\tt VBGLU} is used to bind together MkIII
format VLBA data which were correlated in more than one pass.  The
result is a data set with a significant number of IFs.  {\tt FXPOL} is
a task to correct the \AIPS\ header and tables for dual-polarization
VLBA data sets.  The correlator treats the two polarizations as
separate IFs, while \AIPS\ would prefer to treat them as separate
Stokes.  {\tt FITLD}, which translates correlator FITS files into
\AIPS, was improved in a number of ways.  The format of the {\tt IM}
table was updated, a progress message giving source name and times was
added, user control over the frequency tolerance for file
concatenation was added, and suitable cleanup operations were added to
remove unused entries in tables and empty data files.  Task {\tt
INDXR} was enhanced to use the {\tt IM} table if available when
creating a new base calibration table ({\tt CL} table version 1).

The task which solves for residual delays and rates, {\tt FRING},
received considerable attention.  Bugs were fixed which would (1) stop
the task because the (reported) solution interval of 0.0 was too
large, (2) cause bad solutions due to aliasing in default (Nyquist)
search windows, (3) bias solutions toward zero in narrow search
windows, and (4) stop the task for singularities in the least-squares
fit which can actually be managed.  Warning messages about failed
solutions were also added.  {\tt ANCAL}, which enters system
temperature and antenna gain corrections in the {\tt CL} table, was
enhanced to read in system temperatures for more IFs and to read in
antenna temperatures for many more times for each antenna.  The option
that allows users to specify system temperatures in arbitrary order
({\tt INDEX}) was made operational with several indexing errors
corrected.  {\tt ACFIT}, which determines antenna gains from
spectral-line auto-correlation data, was corrected for an error in
comparing the times of the template to the specified time range and
was enhanced by new adverbs to allow the user to specify the system
temperatures for each IF in the template spectrum.  Shifts in the
fringe-rate spectra plotted by {\tt FRPLT} were corrected.  {\tt
WTMOD} was deterred from modifying weights back to positive (no longer
flagged) which it was doing in some modes.

\subsection{$UV$ data calibration and manipulation}

\RELEASENAME\ \AIPS\ will now support 90-antenna arrays rather than
mere 45-antenna arrays.  This allows \AIPS\ to support the new
real-time solar imaging array built by the Japanese.  The gain
normalization was found to have have two errors: unsampled antennas
were included (at gain 1.0) in the average of the gains (at least when
{\tt APARM(9) = 1}) and the square root of the correct factor was
applied to the data.  Significant bugs in the handling of parallactic
angle were found affecting VLBI polarization corrections and the {\tt
RAPR} option in \hbox{{\tt PCAL}}.  This may explain why the linear
approximation solutions did not seem to work well.  Significant bugs
were also corrected throughout the calibration package in the
deselection of antennas, sources, and calibrators.  There were modes
in which things deselected at the beginning of the task would become
the only things selected at later stages.  Several tasks were changed
to make sure that data in tables are consistent with the data in the
headers.  {\tt SPLIT} and {\tt DBCON} were changed to change the
frequency in the antenna table in the same way that the frequency in
the header was changed.  {\tt SPLIT} was also changed to fix the {\tt
FQ} table for channel averaging and to remove the polarization
correction from the antenna table after applying it to the data.  This
is intended to keep the programs from applying the same correction
twice.  {\tt UVCOP} now removes unused sources from the output source
table.

A bug in {\tt CALIB} caused L1 solutions to fail if one of IFs had
no data for an antenna present at other IFs.  {\tt UVFIX} was improved
to make all phase corrections functions of frequency and to include
leap seconds by default.  A source-position correction option was
added to {\tt CLCOR} and its internal units for the zenith angle gain
correction were made consistent with the documentation.  {\tt UVFLG}
was changed to be able to read more flagging commands and to alter
user-specified times by only 0.5 seconds and then only when the two
ends of a time range are equal.  {\tt LISTR} will now include
auto-correlation data, if present, in {\tt MATX} listings.  {\tt
UVFND} now tests for data flagging in all cases.  The interference
detection task {\tt RFI} was improved to catch interference of steady
amplitude and variable phase and was corrected in its handling of the
amplitude-proportional part of the cutoff level.  {\tt SOLCL}, which
applies solar system temperature measurements to the calibration
table, was improved to avoid writing over {\tt CL} table one and make
a {\tt CL} version two automatically when needed.

\vfill\eject
\subsection{Imaging}

The relatively new imaging and deconvolution with corrections task
{\tt WFCLN} required a number of corrections for this release.  The
default handling of clean boxes was not as a user would have it and
the meanings of the channel selection and averaging adverbs were not
consistent throughout the task.  The option to restart with a
pre-existing work file was turned off since it appears that it can
never be done in a quick mode and it may be the cause of a number of
mysterious failures.  The imaging task {\tt HORUS} was found to be
using only one IF by default, which was a (nasty) surprise to continuum
users.  It was changed to do {\tt SUM} mode on continuum data by
default, to inform users whenever their data selection adverbs are
overridden, and to prevent a failure due to channel-based data
editing.

A minor bug in {\tt SUBIM} causing it to fail to correct the data
extrema in the output header was fixed.  Fourier-transform task {\tt
FFT} was corrected to stop trashing rectangular images.  {\tt VLABP},
which applies wide-field instrumental polarization corrections to VLA
snapshot images, was fixed to use the correct geometry in its scratch
files and to write its output all in the correct place even when some
of the input adverbs have default values.  {\tt OHGEO}, which corrects
and alters image geometry, had a serious bug corrected.  To quote the
{\tt CHANGE.DOC} entry: ``Apparently correct behavior in the distant
past is mystifying.''

Holographic imaging of antenna surfaces is supported in \AIPS\ by two
tasks, {\tt UVPRT} which translates the pseudo-\uv\ data into text
files and {\tt HOLGR} which reads the text files, grids the data,
solves for offset of the antenna vertex plus corrections for pointing,
focus, and feed offset, and Fourier transforms the data to make images
of the telescope.  These programs were substantially revised for this
release.  {\tt UVPRT} was given a new holography mode with a number of
user-controlled and automatic data editing functions to limit the data
written to the text files.  {\tt HOLGR} was changed to offer new
tapering options, to do uniform weighting by default, to compute the
weight and observation beam images from all of the weighting and
tapering, to ignore samples outside the grid, to parameterize the
image dimensions and to use equivalences to avoid using too much
virtual memory, to give the user some control over the resolution of
the image of the telescope's beam pattern, and to allow images up to
512 on a side.

\subsection{Image analysis and display}

Task {\tt SAD} is intended to ``search and destroy'' Gaussian
components in images.  It encountered the first real image from the
VLA D-Array Survey and received a variety of improvements to allow it
to cope.  Corrections were made in the units of the clean beam
position angle, in disk access while subtracting components, in the
estimate of the error in the position angle, and in the corrections of
the component widths and position angle for projective geometry.
Solutions were made more reliable by changes to fit the components in
descending flux order, to limit initial guesses to reasonable widths,
and to determine the extrema and rms of the residuals around each
component only after all components are subtracted.  New options were
added to include a flux-dependent component in warning levels and to
allow the user to specify conditions under which components are
rejected and to display information about those rejections.
Anticipating spectral-line (or time domain) usage, a column for data
cube plane number was added to the model fit table file.  A new task
called {\tt MFPRT} was written to display the contents of the model
fit file in a format suitable for certain non-\AIPS\ modeling
software.

To provide external software with better access to values from \AIPS,
task {\tt IMEAN} was changed to write histogram values into its output
text file and task {\tt SLICE} was changed to write slice values to a
text file.  Task {\tt IMERG} uses an interesting FFT-based algorithm
to merge images of differing resolution.  It was corrected to remove
several coding errors which prevented it from doing anything.  For
single-dish users of \AIPS, task {\tt PRTSD} was corrected in its
handling of spectral channels (it applied {\tt BCHAN} twice) and in
its formats.  {\tt IMLIN}, which fits and removes ``continuum'' from
data cubes, was corrected to handle missing data gracefully, a
condition which arises naturally in single-dish imaging.

A new display verb, {\tt TVSTAR}, was written to plot positions from a
``star'' ({\tt ST}) file directly on a TV.  The color PostScript task
{\tt TVCPS} was enhanced with options to invert the coloring, to add a
user-specified string at the bottom of the picture, and to read the
data from disk rather than from the TV (while still using the TV
transfer functions).  Both it and {\tt LWPLA} were corrected to write
only correct encapsulated PostScript.  Three-color display tasks {\tt
TVHUI} and {\tt TVRGB} were corrected in their enhancement algorithms,
in their image coincidence checking, and in their handling of blanked
pixels.  {\tt KNTR}, which plots multiple contour drawings per page,
was corrected to plot the specified axis labeling type.

\subsection{Magnetic tapes}

Because people are now transferring all of their old data tapes to a
relatively few DAT or Exabyte tapes, we have improved {\tt PRTTP} to
attempt to report on their contents.  In particular, it was changed to
skip over null files (if requested), to report on unknown formats, to
write any error messages into the print file as well as the message
file, to keep going as long as possible despite errors in data, to
write something to the output text file (on {\tt PRTLEV = -3}) for
every format, and to handle large numbers of sources in VLA archive
files.  The tape-mounting software was masking errors from
higher-level code, causing the tape d\ae mon {\tt TPMON} to think
tapes were mounted when they were not.

Tape routines were tested on a variety of new computer systems.
Improved low-level routines were developed for Sun Solaris, DEC, and
Linux.  The latter two are ``byte-swapped'' architectures.  Errors
translating ModComp and FITS-standard floating-point numbers to local
floating point were discovered on these while running {\tt FILLM} and
{\tt FITLD}, respectively.  {\tt FILLM} was also corrected to accept
an ``end-of-information'' return code as a signal to quit normally; it
was only accepting end-of-file.  A bug in the Sun-specific low-level
code caused users at the NRAO to write tapes with no filemarks between
FITS files (over a two-week period).  Task {\tt TCOPY} was enhanced to
recognize and, optionally, correct this problem.

\subsection{Miscellaneous changes of interest to users}

Other corrections and improvements made to \AIPS\ for the
\RELEASENAME\ release include:
\vspace{-10pt}
\begin{description}
\myitem{POSSM} Changed task to add an option to reverse the direction
   of the plot added, to fix a one-channel offset in the axis
   labeling (yet again), and to fix a weighting problem for IFS $ >
   1$.
\myitem{CHKNAME} \hskip 2em New verb to determine whether the file
   specified by {\tt INNAME}, {\it et al.} exists and to return a
   status adverb.  It should be useful in large procedures.
\myitem{EXTDEST} Changed to require confirmation before deleting
   version 1 of either a {\tt CL} or an {\tt HI} file.  One should
   almost never do either of these.
\myitem{TELL} Batch tasks may now actually receive parameters from the
   {\tt TELL} verb.  Thus, an interactive user can alter the parameters
   of, or request a graceful termination of, his/her batch tasks.
\myitem{{\bf helps}} Updated the general help files to point at {\tt
   ABOUT} and continued to keep current the files used by {\tt ABOUT}
   and \hbox{{\tt APROPOS}}.
\end{description}

\section{Improvements Primarily for Programmers in 15JUL94}

\subsection{Ports to new operating systems}

     The new operating system for this release is Silicon Graphics.
We extend our thanks to Jeffrey A. Pedelty of the Goddard Space Flight
Center for doing the initial port and then making his computer
available to us for later testing.  It is a relatively standard Bell
port with a few differences.  In particular, clock ticks are
represented by a different standard symbol ({\tt HZ}) and value (100),
the Unix pipes routine {\tt popen} has a non-standard type requiring a
cast to the usual type, and terminal characteristics are set by
methods other than those used previously.  So far as we know, all of
the port is working, but we have not tested the magnetic tape
routines.  These routines have been brought up to the form and
standards of the best supported systems using the SGI includes and
data structure.  However, they are likely to have minor problems and
will probably appear in due course in the patch area for
\hbox{\RELEASENAME}.

\subsection{Miscellaneous changes for programmers}

Byte-swapped computers are making a renaissance, so programmers should
remain aware of the difficulties they pose and the \AIPS\ parameter
({\tt BYTFLP} in Fortran, {\tt Z\_bytflp} in C) carried around to
describe byte and word swapping.  The relatively generic routines for
converting FITS-standard ({\tt ZR64RL}) and ModComp ({\tt ZRM2RL}, {\tt
ZDM2DL}) floating-point binary numbers to local format all had to be
corrected for this release.

Magnetic tapes remain a software porting nightmare.  Most versions of
{\tt ZTAP2} were improved for this release, removing excess
operations.  This can be dangerous as EOFs were removed excessively
for 2 weeks on Suns.  The Solaris Z routines were upgraded to current
standards.  The DEC versions of the tape routines were made to work,
corrected to handle {\tt errno} 28 which DEC uses for EOI, corrected
in their error handling on back-files, and enhanced to use another
{\tt ioctl} to obtain additional status information.  The Linux Z
routines were also made to work.  In that case, there are a number of
caveats.  Linux systems are really opposed to writing more than one
file and virtually require that to be at the previous end of
information.  By adding a variety of slow and normally un-needed tape
operations, we have been able to overcome most of the limitations of
the Linux tape driver.  Unfortunately, we cannot prevent it from
writing an extra end of file whenever a tape opened for write is
closed.  Fortunately, in \AIPS, such closes only come when the tape is
actually at EOI already.  If we rewound the tape before the close,
Linux would put an EOF at the start of the tape, destroying its entire
contents.  As it is, Linux users should either avoid {\tt DOEOT TRUE},
or learn to tolerate a large number of null files on their tapes.

The area of printing also received some attention for this release.
To remove our dependence on non-standard filters to convert Fortran
print files to plain text and then to PostScript, we have written our
own filter {\tt F2PS} to convert Fortran or plain text to PostScript.
We have also written a filter called {\tt F2TEXT} to convert Fortran
output to plain text printers.  These filters take, as an argument,
the number of lines per page assumed by the \AIPS\ task and attempt to
do something reasonable with it.  \AIPS\ deletes print and plot files
from the {\tt /tmp} area after a delay interval sufficient to get the
file plotted.  Since ``sufficient'' varies, we changed the system to
make this interval available to the \AIPS\ Manager via \hbox{{\tt
SETPAR}}.  The delay may then be set and changed simply for all hosts
in your network together or individually.

Other matters of interest to programmers in \RELEASENAME\ include
\vspace{-10pt}
\begin{description}
\myitem{{\bf large disks}} Several versions of {\tt ZCREA2} were fixed
   to avoid integer overflow problems on disk systems larger than 2
   Gbytes.  Disk free space can no longer be measured in bytes!
\myitem{{\bf disk access}} The disk-use permission parameters have
   been extended to include \AIPS\ disk 1.  Previously, all users were
   allowed to read and write disk 1, but the new network-wise \AIPS\
   allows users to define an otherwise forbidden disk as disk 1.
   \AIPS\ user number 1 now has privilege on all disks and no longer
   needs to be listed in \hbox{{\tt \$NET0/NETSP}}.
\myitem{{\bf standards}} One of the compilers has forced us to require
   that commas separate all format items in {\tt FORMAT} statements.
   This is the ANSI standard, but most compilers have been forgiving.
\myitem{{\bf image size}} The verbs {\tt GO}, {\tt INPUTS}, and
   friends were changed to apply an upper limit of 8192 for image
   corners, ignoring any limits typed in the help files.  Code should
   anticipate images of this size in the future.
\myitem{{\bf AIX}} The print routines {\tt ZLASC2} and {\tt ZLPCL2}
   were not interpreting error code from pipes correctly on IBM
   systems.  This led the {\tt /tmp} area to overflow.
\myitem{{\bf swap space}} \AIPS\ tasks are now requiring large amounts
   of swap space, partly from large arrays of their own using.  When
   ``shared libraries'' are used, however, the swap requirements
   increase to include all subroutines in the shared libraries.  Some
   attempts to remove subroutines from libraries has been made and
   other attempts should be considered.  Large buffers should be coded
   into tasks and passed down to subroutines.  Otherwise those buffers
   are included in the swap-space requirements of all tasks that use
   that subroutine library.  The current swap-space requirements for
   {\tt TVFLG} under Sun OS 4.1.2, for example, are 13.24 Mbytes when
   linked statically and 24.76 Mbytes when linked with shared
   libraries.  For {\tt FITTP}, which uses neither the TV libraries
   nor the calibration parts of {\tt \$APLNOT}, the difference is even
   more extreme, with the static binary requiring only 0.85 Mbytes
   while the dynamic binary requires 21.4 Mbytes.  These numbers show
   that the saving in disk space for the load modules, currently about
   50 Mbytes, is less than the amount of extra swap space required for
   \AIPS\ on {\it each} workstation in the system.
\end{description}

\vfill\eject
\section{AIPS Publications and the World-Wide Web}

     There has been a virtual explosion in the use of the {\it
World-Wide Web\/} (WWW) protocol on the Internet.  It is a method for
sending hypertext over the network and has been made easy to use by
clients such as {\it NCSA Mosaic\/} and {\it Lynx\/}.  NRAO is among
the many institutions which now offer informative ``home pages'' and
networks of additional information.  The NRAO home page is at the
Universal Resource Locator (URL) address
\begin{center}
\vskip -20pt
{\tt http://info.aoc.nrao.edu/}
\vskip -10pt
\end{center}
This page can lead you to information about each of the NRAO's sites
and telescopes, library system, major new initiatives, software
packages, phone directory, and Newsletter as well as information about
other astronomy resources on the Internet and about {\it Mosaic} and
its mark-up language called {\tt html}.  The \AIPS\ group home page
may be found from the NRAO home page or addressed directly at URL
\begin{center}
\vskip -10pt
{\tt http://info.cv.nrao.edu/aips/}
\vskip -10pt
\end{center}
This page points at basic information (``What is AIPS?'' and a
``FAQ''), news items about \AIPS\ (such as ``15JUL94 Release imminent
$\ldots$''), the PostScript text of recent \AIPSLETTER s, patch
information for all releases after {\tt 15JAN91}, information about
known bugs, the latest \AIPS\ benchmark data from various computer
systems, copies of {\tt CHANGE.DOC} for every release since {\tt
15JAN90}, and {\it all} relevant \AIPS\ Memos, {\it every} chapter of
the \Cookbook, and all recent quarterly reports to the \hbox{NSF}.  We
recommend that you check this area occasionally.  As we correct and
update the \Cookbook\ and as we write new \AIPS\ Memos and reports, we
place the documents in PostScript forms in the area used by WWW and we
have the {\tt html} listings, indices, tables of contents, and the
like updated to reflect the additions.  In this way, you do not have
to wait until {\tt 15JAN95} to get the latest \Cookbook\ chapters.  We
expect, for example, to have revised versions of the Spectral-line and
VLBI chapters in the relatively near future.

Below is a list of the latest \AIPS\ Memos.  Memos 86, 87, and 88 are
new in this \Aipsletter, although 88 is simply a \TeX-set and
slightly updated version of several old internal memos.
\begin{center}
\vspace{-6pt}
\begin{tabular}{ccl}
\hline
Memo  &        Date   & Title and author  \\
\hline\hline
  84 & 93/11/12 & A Proposed Package to Support the Use of the X
                  Window System in AIPS Tasks \\
     &          & \qquad Chris Flatters, NRAO \\
  85 & 94/02/01 & DDT Revised and \AMark\ Measurements\\
     &          & \qquad Eric W. Greisen, NRAO \\
  86 & 94/03/16 & Wide-field Polarization Correction of VLA Snapshot
                    Images at 1.4 GHz \\
     &          & \qquad W. D. Cotton, NRAO \\
  87 & 94/04/05 & The NRAO \AIPS\ Project --- a Summary \\
     &          & \qquad Alan H. Bridle, Eric W. Greisen, NRAO \\
  88 & 94/05/16 & The \AIPS\ Gripes Database \\
     &          & \qquad W. D. Cotton, Dean Schlemmer, NRAO \\
\hline
\end{tabular}
\end{center}
\vspace{-6pt}
A heavily revised edition of the Memo on Object-Oriented Programming
in \AIPS\ is available as file {\tt AIPSOOF.TEX} and, in PostScript
form, as \hbox{{\tt AIPSOOF.PS}}.

Since some Memos are not available electronically and others do not
yet have computer readable figures, you may wish to write for a paper
copy of these.  To do so, use an \AIPS\ order form or e-mail your
request to aipsmail@nrao.edu.

To use \ftp\ to retrieve the Memos, \Cookbook\ chapters, etc.:
\begin{description}
\vspace{-10pt}
\item{ 1.} {\tt ftp baboon.cv.nrao.edu}  or  {\tt 192.33.115.103}
\item{ 2.} Login under user name anonymous and use your e-mail address
           as a password.
\item{ 3.} {\tt cd pub/aips/TEXT/PUBL}
\item{ 4.} Read {\tt AAAREADME} for more information.
\item{ 5.} Read {\tt AIPSMEMO.LIST} for a full list of \AIPS\ Memos.
\end{description}

\AIPS\ Memos from Number 65 through 88 are present in this area as are
Numbers 27, 33, 35, 39, 46, 51, 54, 61, and 62.  We have been filling
in this list gradually, by finding and fixing old files in other areas
of the authors' disks, by scanning in text and figures, or by retyping
text and redrawing the figures.  The \Aipsletter s from 1991 through
the present are also available in this area.  Many of the Memos are in
both \TEX\ and PostScript forms, with the \TEX\ ones stored in a
subdirectory called \hbox{{\tt TEX}}.  Note that many, if not all of
these may be found on your home \AIPS\ system in an area called
\hbox{{\tt \$AIPSPUBL}}.  All Memos are available in paper form from
Ernie Allen at the addresses in the masthead.

The latest version of the \AIPS\ \Cookbook\ is also available (in the
form of PostScript files) in this area.  Initially the chapters from
the 1990 version of the \Cookbook\ were placed in this area;  whenever
one of these chapters is updated, the latest version will be available
immediately in this area.  Updated so far are chapters 1
(Introduction), 2 (Starting Up \AIPS), 3 (Basic \AIPS\ Utilities), 4
(Calibrating Interferometer Data), 15 (Current \AIPS\ Software), and Z
(System-Dependent \AIPS\ Tips).  Chapter 10 (Spectral-Line Software)
is nearly ready as well.  The remaining old chapters were revised to
include figures in the PostScript and improve the typesetting, but are
full of outdated information in addition to the good stuff.

\section{Patch Distribution}

Since \AIPS\ is now released only semi-annually (or even less
frequently), we make selected, important bug fixes and improvements
available via {\it anonymous} \ftp\ on the NRAO Cpu {\tt baboon}
({\tt 192.33.115.103}).  Documentation about patches to a release is
placed in the anonymous-ftp area {\tt pub/aips/}{\it release-name} and
the code is placed in suitable subdirectories below this.
(The patches and their documentation are also available on-line via
the World-Wide Web.)  Reports of significant bugs in {\tt 15JAN94}
\AIPS\ were more numerous than recent previous releases.  The
documentation file {\tt pub/aips/15JAN94/README.15JAN94} mentions the
following items:
\begin{description}
\vspace{-8pt}
\myitem{FRING} {\tt FRING.FOR} was corrected to be consistent in its
    use of buffers and to avoid telling the user incorrectly to use a
    {\tt SOLINT} of 0.0.
\myitem{\    } {\tt FRING.FOR} was corrected to avoid aliasing in the
   default (Nyquist) search windows.  This aliasing appears to have
   been responsible for bad solutions for residual delay and rate.
\myitem{\    } {\tt FRING.FOR} was corrected for a bias toward zero in
   the solutions for delay using narrow search windows.  This caused
   non-closing errors in the amplitudes degrading the dynamic ranges
   of the resulting images.
\myitem{ANCAL} {\tt KEYIN.FOR} was changed to increase the line length
   in text files.  This is needed in {\tt ANCAL} to read
   $T_{\hbox{sys}}$ information for large numbers of IFs.
\myitem{\    } {\tt ANCAL.FOR} was corrected for a number of indexing
    errors in the processing of $T_{\hbox{sys}}$ and $T_{\hbox{ant}}$
    entries.  {\tt ANCAL.HLP} was corrected to fix misleading
    information about $T_{\hbox{sys}}$ groups.
\myitem{SPLIT} {\tt SPLIT.FOR} was changed to keep frequencies
    consistent in all parts of the output (the antenna tables were not
    changed previously) and to reset the polarization corrections to
    prevent them from being applied twice.
\myitem{MX} {\tt MX.FOR} was corrected for a data-selection bug in
   making spectral-line images while ``averaging'' multiple IFs.  The
   wrong data were selected in all but the first output channel.
   Fortunately, this is not a likely mode of imaging.
\myitem{FFT} {\tt FFT.FOR} was corrected to stop mangling rectangular
    images.
\myitem{KNTR} {\tt KNTR.FOR} was corrected to use all the input
    adverbs and to do the desired type of axis labeling.
\myitem{true color} {\tt TVHUI.FOR} and {\tt TVRGB.FOR} were corrected
    to use the TV cursor for image enhancement in the usual way, to be
    more correct and complete in checking image alignment, and to be
    more forgiving about blanked pixels.
\myitem{IMERG} {\tt IMERG.FOR} was not previously able to run.
\myitem{PRTSD} {\tt PRTSD.FOR} was corrected to display the requested
    channel (it was offsetting twice) and to improve minor formatting
    problems.
\myitem{FILLM} {\tt FILLM.FOR} failed to test for end-of-information,
    depending on end-of-file only.  SUN returns EOI for double
    end-of-files on 9-track tapes, leading to an infinite loop of
    error messages.
\myitem{PRTTP} {\tt PRTTP.FOR} was enhanced to handle null files, to
    do something useful with unrecognized formats, to force all error
    messages to the printer file, to try to continue despite
    data-format errors, and to be rather more bomb proof in general.
\myitem{DEC} {\tt ZTAP2.C}, {\tt ZTPWA2.C}, and {\tt ZMOUN2.C} for
   DEC Ultrix and DEC Alpha systems only were changed to recognize
   beginning and end of tape correctly and to do a few other things
   better.  Affects all tape tasks (including {\tt AIPS}), but only
   for DEC systems.
\myitem{DEC, PC} {\tt ZDM2DL.C} and {\tt ZRM2RL.C} were corrected to
   run correctly on byte-swapped (little-endian) computers.  {\tt
   FILLM.FOR} was corrected to assure correct alignment of buffers.
\myitem{big disks} {\tt ZCREA2.C} for several systems was changed
    to handle disk systems with more than 2 Gbytes of free space.
    There was no problem for more reasonable systems and in some of
    other versions of \hbox{{\tt ZCREA2}}.
\myitem{printing} {\tt ZLASC2.C} and {\tt ZLPCL2.C} did not work
    correctly on IBM AIX systems, causing {\tt /tmp} to fill up among
    other things.
\myitem{HP} {\tt LIBR.DAT} for HP systems only needed a small
    addition in order to link edit the {\tt \$AIPNOT} area correctly.
\end{description}
\vspace{-8pt}
Note that we did not revise the original {\tt 15JAN94} tape or \tar\
files for these patches.  No matter when you received your {\tt
15JAN94} ``tape,'' you must fetch and install these patches if you
require them.  See the {\tt 15APR92} \AIPSLETTER\ for an example of
how to fetch and apply a patch.  Information on patches and how to
fetch and apply them is also available through the World-Wide Web
pages for \AIPS.
As bugs in \RELEASENAME\ are found, the patches will be placed in the
\ftp\ area for \hbox{{\RELEASENAME}}.  As usual, we will not revise
the original {\tt 15JUL94} tape or \tar\ files for any such patches.  No
matter when you receive your {\tt 15JUL94} ``tape,'' you must fetch
and install these patches if you require them.

\section{\Cookbook\ Update Begun}

     The \AIPS\ \Cookbook\ was last updated for the {\tt 15OCT90}
release.  Because a lot has changed in \AIPS\ since then, we have
decided to upgrade the \Cookbook.  We are doing this one chapter at a
time and are making each chapter available via the World-Wide Web as
soon as it is ready.  For details of the Web, see the publications
article in this \Aipsletter.  The chapters changed so far are
\vspace{-8pt}
\begin{itemize}
\item\ 1 --- {\it Introduction} --- New sections giving a project
   summary and a diagram of the structure of \AIPS\ were added.
\item\ 2 --- {\it Starting Up \AIPS} ---  Changed to describe
   workstation use, \AIPS\ in networked environments, and managing the
   TV server {\tt XAS}.
\item\ 3 --- {\it Basic \AIPS\ Utilities} --- Updated information about
   history files and disk allocation, added {\tt ABOUT} and {\tt
   APROPOS} to the help section, moved and updated tape mounting, and
   added a discussion on external disk files (Fits, text, $\ldots$).
\item\ 4 --- {\it Calibrating Interferometer Data} --- With much help
   from Rick Perley and Alan Bridle, rearranged and corrected
   everything, adding a substantial discussion of when and how to edit
   and bringing the description of {\tt TVFLG} up to date including a
   picture.
\item\ 15 --- {\it Current \AIPS\ Software} --- Replaced old lists with
   new ones produced for the {\tt ABOUT} verb.
\item\ Z --- {\it System-Dependent \AIPS\ Tips} --- Replaced with whole
   new discussions including color printers, screen copying, film
   recorders, workstation environments.  A method for people to have
   NRAO make slides for them is described.
\end{itemize}
\vspace{-8pt}
\noindent There exist drafts of both Chapters 10 {\it Spectral-Line
Software} and 11 {\it Reducing VLBI Data in \AIPS} which will be
released in the relatively near future.  Elias Brinks has helped with
the spectral-line chapter, which will be more up to date with less
emphasis on {\tt HORUS} and more on {\tt MX} and {\tt WFCLN}, and more
attention for the {\tt IMLIN} and {\tt UVLIN} type of continuum
subtraction.  The VLBI chapter is a major rewrite to account for the
advent of the \hbox{VLBA}.  It is our hope to update the other
chapters, add a chapter on single-dish data in \AIPS, and even to add
an index to the \Cookbook.
\eject
\end{document}

%\vfill
%\centerline{\psfig{figure=FIG/monkey.ps,height=4.05in}}
%\vskip 6pt
%\centerline{Panel of \AIPS\ users tests new recipes}
%\eject
