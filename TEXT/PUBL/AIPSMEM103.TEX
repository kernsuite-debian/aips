\documentstyle [twoside]{article}
%-----------------------------------------------------------------------
%;  Copyright (C) 2000
%;  Associated Universities, Inc. Washington DC, USA.
%;
%;  This program is free software; you can redistribute it and/or
%;  modify it under the terms of the GNU General Public License as
%;  published by the Free Software Foundation; either version 2 of
%;  the License, or (at your option) any later version.
%;
%;  This program is distributed in the hope that it will be useful,
%;  but WITHOUT ANY WARRANTY; without even the implied warranty of
%;  MERCHANTABILITY or FITNESS FOR A PARTICULAR PURPOSE.  See the
%;  GNU General Public License for more details.
%;
%;  You should have received a copy of the GNU General Public
%;  License along with this program; if not, write to the Free
%;  Software Foundation, Inc., 675 Massachusetts Ave, Cambridge,
%;  MA 02139, USA.
%;
%;  Correspondence concerning AIPS should be addressed as follows:
%;         Internet email: aipsmail@nrao.edu.
%;         Postal address: AIPS Project Office
%;                         National Radio Astronomy Observatory
%;                         520 Edgemont Road
%;                         Charlottesville, VA 22903-2475 USA
%-----------------------------------------------------------------------
\newcommand{\DoMemo}{T}
\newcommand{\doFig}{T}
\newcommand{\showwork}{T}
\newcommand{\workyes}{F}
%
\newcommand{\memnum}{103}
\newcommand{\whatmem}{\AIPS\ Memo \memnum}
\newcommand{\AIPS}{{$\cal AIPS\/$}}
\newcommand{\VPOPS}{{$\cal VPOPS\/$}}
\newcommand{\RANCID}{{$\cal RANCID\/$}}
\newcommand{\memtit}{Weighting data in AIPS}
%
\newcommand{\POPS}{{$\cal POPS\/$}}
\newcommand{\Cookbook}{${{\cal C}ook{\cal B}ook\/}$}
\newcommand{\TEX}{\hbox{T\hskip-.1667em\lower0.424ex\hbox{E}\hskip-.125em X}}
\newcommand{\AMark}{AIPSMark}
\newcommand{\AMarks}{AIPSMarks}
\newcommand{\figyes}{T}
\newcommand{\uv}{{\it uv}}
\newcommand{\eg}{{\it e.g.},}
\newcommand{\ie}{{\it i.e.},}
\newcommand{\daemon}{d\ae mon}
\newcommand{\Aipsletter}{{${\cal AIPSL}etter\/$}}
\newcommand{\ust}{{\rm st}}
\newcommand{\uth}{{\rm th}}
\newcommand{\und}{{\rm nd}}
\newcommand{\urd}{{\rm rd}}
%
\newcommand{\boxit}[3]{\vbox{\hrule height#1\hbox{\vrule width#1\kern#2%
\vbox{\kern#2{#3}\kern#2}\kern#2\vrule width#1}\hrule height#1}}
%
\title{
   \vskip -35pt
   \if T\DoMemo
      \fbox{{\large\whatmem}} \\
      \fi
   \vskip 28pt
   \memtit\\}
\author{Ketan M. Desai}
%
\parskip 4mm
\linewidth 6.5in
\textwidth 6.5in                     % text width excluding margin
\textheight 8.81 in
\marginparsep 0in
\oddsidemargin .25in                 % EWG from -.25
\evensidemargin -.25in
\topmargin 0.3in
\headsep 0.25in
\headheight 0.25in
\parindent 0in
\newcommand{\normalstyle}{\baselineskip 4mm \parskip 2mm \normalsize}
\newcommand{\tablestyle}{\baselineskip 2mm \parskip 1mm \small }
%
%
\begin{document}

\pagestyle{myheadings}
\thispagestyle{empty}

\if T\DoMemo
   \newcommand{\Rheading}{\whatmem \hfill \memtit \hfill Page~~}
   \newcommand{\Lheading}{~~Page \hfill \memtit \hfill \whatmem}
\else
   \newcommand{\Rheading}{K. M. Desai\hfill \memtit \hfill Page~~}
   \newcommand{\Lheading}{~~Page \hfill \memtit \hfill K. M. Desai}
   \fi
\markboth{\Lheading}{\Rheading}
%
%

\vskip -.5cm
\pretolerance 10000
\listparindent 0cm
\labelsep 0cm
%
%

\vskip -30pt
\maketitle
%\vskip -30pt
\normalstyle

\begin{abstract}
This memo describes the 'correct' calculation of weights for VLA and
VLBA data based primarily on phenomological considerations with a
light sprinkling of theoretical grounding.
\end{abstract}

\section{Introduction}

Weight calibration is important for radio-synthesis data.
One commonly weighting scheme is for data weights to reflect the
inverse variance of the data points.  We have the advantage that
modern synthesis radio interferometry is based upon the correlation of
millions of digitized signal bits.  Large number statistics are on our
side.

The unnormalized correlator coefficient is the number of bits
correlated $\rho = r$.  The {\it rms} number of bits correlated when
there is no signal present is the noise $\sigma = \sqrt{n}$.  The
normalized correlation coefficient is $\hat{\rho} = {r\over n}$.  The
normalized noise is $\hat{\sigma} = {\sqrt{n}\over n}$.  So, the
normalized weight $\hat{w} = {1\over\hat{\sigma}^2} = n$.

The key point is that, once the visibility and weights are made to be
in proper relation to one another, no adjustments should be made to
one without a corresponding adjustment to the other.  AIPS does make
adjustments to the visibilities and weights in the proper way whenever
the logical DOWTCL in SELINI.INC is set.  Unfortunately, this is
rarely done.  That must change if proper weights are to propagate from
end to end.

\section{Weighting for the VLA}

The VLA currently delivers $v_d = \hat{\rho}\Sigma$.  $\Sigma$ is a
scaling factor that attempts to convert the correlation coefficient
into a visibility in units of deciJy (for historical
software-motivated reasons).  The number of bits correlated is $n =
\Delta t\Delta\nu$.  The delivered weight by the VLA is $w_d = 1 =
{\Delta t\over [10 sec]}$.  Users seem to want the currently delivered
visibility to remain unchanged so, $v_w = v_d$.  We want a weight $w_w
= {1\over\sigma^2} = {1\over\hat{\sigma}^2\Sigma^2} = {\Delta t\Delta\nu\over\Sigma^2}$.  This means
$w_w = {w_d 10 \Delta\nu\over\Sigma^2}$.

\subsection{What about the correlator efficiency $\eta_c$?}

The correlator efficiency factor represents a decrease in sensitivity
due to digitization noise.  This factor is accounted for in the
nominal sensitivity because it is also used to convert from the
correlator coefficient to the deciJy scale. It is not necessary to
consider this factor again for VLA data.

\section{Weighting for the VLBA}

The VLBA correlator delivers correlator coefficients along with a
scaling factor necessary to properly normalize them.  To the extent
that the visibility scaling factor simply normalizes the correlation
coefficients, it is irrelevant for correct calculation of the weights.
Any factors that represent an 'incompleteness' in the number of bits
used for correlation are relevant.  This certainly should include the
doom factors.  Also, this includes a tape-playback factor.  The weight
that FITLD should produce is:

$$w_w = \eta_{playback}\eta_{doom}\Delta t\Delta\nu$$

Currently the delivered weight is $w_d = \eta_{playback}\Delta t$.  So,

$$w_w = \eta_{doom}w_d\Delta_\nu$$.

\section{Other considerations}

These musings do not properly address the strong signal case
and proper treatment of the Van Vleck corrections.

\section{Acknowledgements}

I would like to acknowledge helpful discussions with Barry Clark,
Michael Rupen, Bryan Butler, and Ken Sowinski.  Also, whoever
implemented the 'solar' mode of FILLM did most of the hard work for
figuring out the VLA weight calibration procedure.

\end{document}
