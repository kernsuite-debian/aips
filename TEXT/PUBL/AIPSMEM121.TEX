%-----------------------------------------------------------------------
%;  Copyright (C) 2016
%;  Associated Universities, Inc. Washington DC, USA.
%;
%;  This program is free software; you can redistribute it and/or
%;  modify it under the terms of the GNU General Public License as
%;  published by the Free Software Foundation; either version 2 of
%;  the License, or (at your option) any later version.
%;
%;  This program is distributed in the hope that it will be useful,
%;  but WITHOUT ANY WARRANTY; without even the implied warranty of
%;  MERCHANTABILITY or FITNESS FOR A PARTICULAR PURPOSE.  See the
%;  GNU General Public License for more details.
%;
%;  You should have received a copy of the GNU General Public
%;  License along with this program; if not, write to the Free
%;  Software Foundation, Inc., 675 Massachusetts Ave, Cambridge,
%;  MA 02139, USA.
%;
%;  Correspondence concerning AIPS should be addressed as follows:
%;          Internet email: aipsmail@nrao.edu.
%;          Postal address: AIPS Project Office
%;                          National Radio Astronomy Observatory
%;                          520 Edgemont Road
%;                          Charlottesville, VA 22903-2475 USA
%-----------------------------------------------------------------------
\documentclass[twoside]{article}
\usepackage{palatino}
\renewcommand{\ttdefault}{cmtt}
% Highlight new text.
\usepackage{color}
\usepackage{alltt}
\usepackage{graphicx,xspace,wrapfig}
\usepackage{pstricks}  % added by Greisen
\definecolor{hicol}{rgb}{0.7,0.1,0.1}
\definecolor{mecol}{rgb}{0.2,0.2,0.8}
\definecolor{excol}{rgb}{0.1,0.6,0.1}
\newcommand{\Hi}[1]{\textcolor{hicol}{#1}}
%\newcommand{\Hi}[1]{\textcolor{black}{#1}}
\newcommand{\Me}[1]{\textcolor{mecol}{#1}}
%\newcommand{\Me}[1]{\textcolor{black}{#1}}
\newcommand{\Ex}[1]{\textcolor{excol}{#1}}
%\newcommand{\Ex}[1]{\textcolor{black}{#1}}
\newcommand{\No}[1]{\textcolor{black}{#1}}
\newcommand{\hicol}{\color{hicol}}
%\newcommand{\hicol}{\color{black}}
\newcommand{\mecol}{\color{mecol}}
%\newcommand{\mecol}{\color{black}}
\newcommand{\excol}{\color{excol}}
%\newcommand{\excol}{\color{black}}
\newcommand{\hblack}{\color{black}}
%
\newcommand{\AIPS}{{$\cal AIPS\/$}}
\newcommand{\eg}{{\it e.g.},}
\newcommand{\ie}{{\it i.e.},}
\newcommand{\etal}{{\it et al.}}
\newcommand{\tablerowgapbefore}{-1ex}
\newcommand{\tablerowgapafter}{1ex}
\newcommand{\keyw}[1]{\hbox{{\tt #1}}}
\newcommand{\sub}[1]{_\mathrm{#1}}
\newcommand{\degr}{^{\circ}}
\newcommand{\vv}{v}
%\newcommand{\vv}{\varv}
\newcommand{\eq}{\hbox{\hspace{0.6em}=\hspace{0.6em}}}
\newcommand{\newfig}[2]{\includegraphics[width=#1]{data.fig#2}}
%\newcommand{\putfig}[1]{\includegraphics{data.fig#1.eps}}
%\newcommand{\putfig}[1]{\includegraphics{#1.eps}}
\newcommand{\putfig}[1]{\includegraphics{#1}}
\newcommand{\whatmem}{\AIPS\ Memo \memnum}
\newcommand{\boxit}[3]{\vbox{\hrule height#1\hbox{\vrule width#1\kern#2%
\vbox{\kern#2{#3}\kern#2}\kern#2\vrule width#1}\hrule height#1}}
%
\newcommand{\memnum}{121}
\newcommand{\memtit}{Editing on a $uv$ grid in \AIPS}
\title{
   \vskip -35pt
   \fbox{{\large\whatmem}} \\
   \vskip 28pt
%   \vskip 10pt
%   \fbox{{\Huge \Me{D R A F T}}}
%   \vskip 10pt
   \memtit \\}
\author{Eric W. Greisen}
%
\parskip 4mm
\linewidth 6.5in                     % was 6.5
\textwidth 6.5in                     % text width excluding margin 6.5
\textheight 9.0 in                  % was 8.81
\marginparsep 0in
\oddsidemargin .25in                 % EWG from -.25
\evensidemargin -.25in
\topmargin -0.4in
%\topmargin 0.2in
\headsep 0.25in
\headheight 0.25in
\parindent 0in
\newcommand{\normalstyle}{\baselineskip 4mm \parskip 2mm \normalsize}
\newcommand{\tablestyle}{\baselineskip 2mm \parskip 1mm \small }
%
%
\begin{document}

\pagestyle{myheadings}
\thispagestyle{empty}

\newcommand{\Rheading}{\whatmem \hfill \memtit \hfill Page~~}
\newcommand{\Lheading}{~~Page \hfill \memtit \hfill \whatmem}
\markboth{\Lheading}{\Rheading}
%
\vskip -.5cm
\pretolerance 10000
\listparindent 0cm
\labelsep 0cm
%
%

\vskip -30pt
\maketitle

\normalstyle
\begin{abstract}
  Since its beginning, \AIPS\ has offered users methods to identify
  and delete bad data samples.  Modern interferometers observe over
  wide bandwidths and so must include spectral regions polluted by
  radio-frequency interference (``RFI'')\@.  Numerous tasks, both
  interactive and automated, have appeared to assist the modern user
  in this operation.  A new one, called {\tt UFLAG}, has been written
  to allow the user to explore the data as it appears when gridded in
  the $uv$ plane.  This memo is intended to assist users in navigating
  the new task.
\end{abstract}

\renewcommand{\floatpagefraction}{0.75}
\typeout{bottomnumber = \arabic{bottomnumber} \bottomfraction}
\typeout{topnumber = \arabic{topnumber} \topfraction}
\typeout{totalnumber = \arabic{totalnumber} \textfraction\ \floatpagefraction}

\section{Introduction}

The (pre-upgrade) VLA produced, by modern standards, only modest
amounts of data.  The visibility samples could even be examined
individually with printer displays, such as those offered by {\tt
  LISTR}\@.  As data volumes grew, interactive tasks to examine data
using a ``television'' display were required.  The first of these,
{\tt TVFLG} allows the user numerous options to view single channel
data on a grey-scale display with baseline number and time axes.  A
similar task, {\tt SPFLG} appeared not long after to allow similar
options on a display of spectral channel versus time, one baseline at
a time.  Graphical editors appeared in time to edit tables ({\tt
  SNEDT}), to edit $uv$ data based on table values ({\tt EDITA}), and
to edit $uv$ data based on its values ({\tt EDITR}), all of which are
limited to a single antenna or baseline in the edit window at any one
time.  A powerful editing task ({\tt WIPER}) makes a two-dimensional
plot on any two physical parameters of a $uv$ data set keeping track
of the number of samples contributing to each cell in the grid and
which baselines have contributed to each cell.  The display is,
however, simply graphical, a cell is either occupied or it isn't.
Nonetheless, bad data may be identified in many cases (\eg\ in
plotting amplitude versus baseline length), marked as bad, and then,
as the data are re-read, marked with flags in an output flag table.

Recently, Golap and Momjian\footnote{Golap, K., Momjian, E. 2016, EVLA
  Memo \#198, ``MSUVBIN: A Way to Combine, Average, Flag Visibility
  Data''}  have discussed the advantages of making a $uv$ grid of the
data in order to average over multiple data sets, save storage space,
and flag grid cells differing excessively from their neighbors.  From
early days, \AIPS\ had ways to investigate the cause of bad grid cells
(which make stripes in images) through tasks {\tt FFT} to Fourier
transform the suspect image, {\tt COMB} to make an amplitude image
from the real and imaginary images, {\tt UVFND} to display the data
contributing to bad cells, and finally {\tt UVFLG} to add the flag
commands to a flag table.  There is nothing automatic or simple about
that process.  So it was decided to investigate an interactive task to
produce a generalized shortcut of the process including the capability
of doing much of the flagging itself.  With an interactive task, one
may also investigate possible automatic or semi-automatic algorithms
for identifying and flagging bad visibility samples.

\vfill\eject

\section{Gridded UV data exploration: {\tt UFLAG}}

{\tt UFLAG}, new in the {\tt 31DEC16} version of \AIPS, is the result.
It is a ``clone'' of {\tt WIPER} in that it grids the $uv$ data into
an image with $u$ and $v$ as axes and then displays an editing menu.
Also, at the end it re-reads the $uv$ data, finds the cells to which
they contribute, and, if the cells are flagged, writes appropriate
flag commands to the output flag table.  However, in the case of {\tt
  UFLAG}, the images are of the scalar and vector average amplitudes
of the data falling into the grid cell, the vector average phase, and
the  difference between scalar and vector amplitudes.  A menu item
allows you to select which of these you wish to examine and to use in
editing at any particular time.  {\tt UFLAG} keeps track of which
baselines contribute to each grid cell (up to 25) and will display the
first two in the list during some interactive flagging operations.
{\tt UFLAG} also constructs lists of which visibility samples
contribute to each grid cell with no limit in the length of the lists
(a linked list structure is used).  This allows the task to re-examine
each visibility that contributed to a particular cell, allowing the
user to flag some, rather than all, of the visibilities contributing
to a cell.  Of course, one may also flag all visibilities that
contribute to a cell (as in {\tt WIPER})\@.

One should note that the grid constructed by {\tt UFLAG} is made by
putting each visibility into a single cell with full weight.  This
``pill-box'' convolution, while necessary for flagging, is a rather
poor convolution for imaging purposes.  High-quality convolutions,
such as the spheroidal function used by default in \AIPS, grid a
sample into multiple cells in the grid with non-unity weights and make
a rather confusing image when used for editing.  The old {\tt FFT}
method suffered from this issue unless the image was heavily Cleaned
before the FFT or was constructed with the pill-box convolution.

For calibration sources, the phase image in {\tt UFLAG} can be of
considerable interest, while for target sources it is likely to be
more confusing than useful.  However, in both cases, the difference
image between the scalar and vector images will reveal grid cells in
which the visibilities differed significantly in their phase.  Note,
however, this difference image is exactly zero where only one
visibility sample contributes to a cell, no matter how egregious that
visibility.

\subsection{Inputs}

There is a large number of input adverbs to {\tt UFLAG} but many of
them are familiar adverbs used in the normal fashion.  Such adverbs
select the input data set, the sources included, the Stokes value (one
at a time is required), the frequency ID, times, antennas, spectral
windows (``IFs''), calibration, and input flagging.  This section will
only highlight the new adverbs and those used in somewhat non-standard
ways.

Channel selection is handled in the usual way with {\tt BCHAN} and
{\tt ECHAN}, but every channel so selected appears in the single $uv$
plane grid at its correct (frequency-dependent) location.  That may be
modified by setting {\tt NCHAV} to pre-average channels before they
are gridded.  This is a non-standard use of {\tt NCHAV}, also
practiced in {\tt WIPER}, but quite unlike the usage in {\tt IMAGR} in
particular.  {\tt CHINC} may also be used to limit the channels
included in the grid.  Adverb {\tt OUTFGVER} may be used to specify a
particular output flag table.  It is probably best to leave this at
zero to create a new flag table.  Note that {\tt UFLAG} will copy the
input flag table ({\tt FLAGVER}, if any) to a new output table, but
not to a pre-existing one.  Note also that {\tt UFLAG} is capable of
restarting, using the new flag table as an input flag table, and then
creating a new output flag table.  For this reason, leaving {\tt
  OUTFGVER} zero is probably a good idea.  Among other reasons, if you
do not like the results of {\tt UFLAG}, it is easy to delete a new
flag table, but harder to edit the {\tt UFLAG} entries out of an
existing table.

The user controls the size of a cell in the $uv$ plane grid with the
familiar {\tt CELLSIZE} (image cell size in arc seconds in $x$ and
$y$) and {\tt IMSIZE} (number of image pixel in $x$ and $y$) adverbs.
The cell size in wavelengths is then $(180 * 3600) / (\pi * {\rm
  cellsize} * {\rm imsize})$.  Note that there is no requirement in
{\tt UFLAG} that {\tt IMSIZE} be powers of two or that the grid cell
sizes be the same.  {\tt UFLAG} reads the data to determine the
minimum and maximum values of $u$ and maximum value of $| v |$, using
the Hermitian property of visibilities to require that $v \geq 0$.
This then sets the grid image size, including some pixels around the
edge to improve the display and keep the tick marks from interfering
with the data. Adverb {\tt DOWEIGHT} controls whether the data weight
or 1.0 is used to weight the data into the grid.  Adverb {\tt
  DOCENTER} simply controls the position of the {\tt CURVALUE}-like
display during interactive, pixel-based editing operations.  A value
of -1 places it at the upper left, +1 at the upper right, and 0 at the
upper center.  Adverb {\tt PRTLEV} controls the amount of messages
appearing with $> 0$ printing information about each cell addressed in
auto-flagging operations and $> 1$ also producing, at the end or a
restart, matrix displays of correlations flagged.

Adverb {\tt DOALL} is a bit pattern that you may use to control which
spectral channels, spectral windows, and polarizations are ultimately
flagged when a flag command is generated.  A value of zero is
conservative while higher values cause additional data to be flagged.

The one new adverb is {\tt ISCALIB} to tell the task whether the
source included should be treated as a target or a calibration source.
The latter are assumed the have amplitudes centered about some
constant value and phases centered on zero.  The auto flagging
operations flag calibration sources for both high and low amplitudes
as well as phases too far from zero.  For target sources, the auto
flagging routines ignore phase and low amplitudes, flagging high
amplitudes only.  Adverb {\tt APARM} is used to control the setting of
examination and clip levels in the {\tt AUTO FLAG VIS} operation.  See
below for a description of this operation and the usage of {\tt
  APARM}\@.

\subsection{Running {\tt UFLAG}: menus and display options}

{\tt UFLAG} begins by reading the data set to make sure that some data
are found and to determine the size of the $uv$ grid to be displayed.
Then it compares that grid size to the size of the TV display screen.
If the grid is larger than the screen, the task displays the grid
image in a pixel-averaged form and requires the user to select a
sub-image small enough to fit of the screen without pixel averaging.
If the grid is small enough, the display will use pixel replication to
blow up the display, but it always uses the same pixel replication
factor on both axes.  The selection of a sub-image is done by the
usual {\tt TVWINDOW} operation: point at the bottom left corner, hit
button A or B to select the top right corner, and finally hit button C
or D to exit, selecting the sub-image.  The sub-image (or full image)
will then be displayed including axis labels.  The {\tt VECTOR}
average image is selected for the first display.  A grey-scale step
wedge will appear above the $uv$ plane grid.  It is labeled at the
left with the lowest value, at the right with the highest value, and
in the center with the data type and polarization displayed.  The
median value of the displayed image and the simple mean and rms of the
image (ignoring unsampled cells) are also displayed above the step
wedge.  All labeling uses the first graphics plane which is usually
yellow.

The following menu appears in graphics plane two (usually green) at
the left of the screen whenever you need to select some operation.

\begin{center}
\begin{tabular}{|l|l|}\hline
 {\tt ABORT          } & Exit the task, writing no more flag tables \\
 {\tt FLAG + EXIT    } & Exit the task, writing flag tables \\
 {\tt FLAG + REPEAT  } & Write current flagging information, restart
                         with that flag table as input \\
 {\tt FLAG + SWITCH  } & As {\tt FLAG+REPEAT}, but also switch Stokes \\
 {\tt                } & \\
 {\tt OFF ZOOM       } & Reset zoom factor to one \\
 {\tt TVZOOM         } & Set zoom factor and center for some displays \\
 {\tt OFF TRANS      } & Initialize black-and-white transfer function \\
 {\tt OFF COLOR      } & Turn off any pseudo-coloring \\
 {\tt TVTRANSF       } & Adjust black-and-white transfer function \\
 {\tt TVPSEUDO       } & Color contours of a variety of types \\
 {\tt TVPHLAME       } & Flame-like pseudo-coloring \\
 {\tt SET PIXRANGE   } & Type in grey-scale intensity range for
                         display for the image type being viewed\\
 {\tt LOAD {\it xxxx}} & Re-load image with {\it xxxx} transfer
                         function \\
 {\tt VIEW {\it zzzz}} & Re-load with image type {\it zzzz} \\
 {\tt VIEW ONLY      } & To set the intensity range visible in the
                         display, then may flag \\
 {\tt VIEW ALL       } & To view all intensities now available \\
 {\tt FLAG PTS OFF   } & To turn the display of flagged cells off or
                         back on\\
\hline
\end{tabular}
\end{center}

This menu includes the familiar image display options.  For more
accurate flagging, you may zoom the image interactively including the
center about which the zoom is done.  The image is shown in this zoom
during enhancement operations, window setting, flagging/unflagging
operations, and the visibility data ``examination'' function.
Otherwise the image is shown without zoom.  Note that {\tt SET WINDOW}
allows you to select a sub-image which will be shown in a
pixel-replicated fashion up to the replication factor that still fits
on the TV display.  This may be a better way to see a portion of the
image ``zoomed.''  Other familiar display options allow you to enhance
the image interactively in black and white and in pseudo-colorings of
various sorts and then to turn these enhancements off again.

{\tt SET PIXRANGE} allows you to type in the desired intensity range
to be displayed with actual data values above the range getting full
intensity and those below the range getting the minimum value.  The
range set is for the image type currently displayed and does not
affect the other types.  It may be reset by entering a minimum value
$\geq$ the maximum value.  {\tt LOAD} {\it xxxx} allows you to re-load
the image with the specified transfer function.  The value of {\it
  xxxx} that appears in the menu is the next one in the sequence {\tt
  LIN}, {\tt LOG}, {\tt SQRT}, and {\tt LOG2}\@.  Thus, if the display
has been loaded with a linear transfer function (as is done when it
starts), then {\tt LOG} will appear in the menu.

Similarly, {\tt VIEW} {\it zzzz} allows you to select which image to
view and to use in editing from the sequence {\tt PHASE}, {\tt
  SCAL-VEC}, {\tt SCALAR}, and {\tt VECTOR}\@.  Again, if {\tt VECTOR}
has been loaded (which is how the task begins), then {\tt PHASE} will
appear in the menu.  The option {\tt FLAG PTS OFF} appears when
flagged grid cells may be shown with marks in graphics plane four
(usually cyan).  Since large numbers of these can be distracting, the
option allows you to turn off the graphics plane.  If you do, the menu
option changes to {\tt FLAG PTS ON}\@.  The remaining options in this
menu will be discussed in later sections.


The following menu appears in graphics plane two (usually green) at
the right of the screen whenever you need to select some operation.
Some of these options appear only when appropriate conditions occur

\begin{center}
\begin{tabular}{|l|l|}\hline
 {\tt SET WINDOW     } & Select a sub-image for more detailed viewing \\
 {\tt RESET WINDOW   } & Return to viewing the full image \\
 {\tt                } & \\
 {\tt FLAG POINT     } & Flags cells one at a time on buttons A, B, C \\
 {\tt FLAG AREA      } & Flags rectangles of cells set in the usual
                         manner \\
 {\tt FLAG FAST      } & Flags any cell at which the cursor is
                         pointed \\
 {\tt FLAG BASELINE  } & Records flagging for a pair of antennas and
                         marks cells appropriately \\
 {\tt UNFLAG POINT   } & Unflags flagged cells one at a time with
                         buttons \\
 {\tt UNFLAG AREA    } & Unflags flagged cells in rectangles\\
 {\tt UNFLAG FAST    } & Unflags any flagged cell at which cursor is
                         pointed\\
 {\tt UNFLAG BASEL   } & Removes record of flagging for a pair of
                         antennas and unmarks cells appropriately\\
 {\tt EXAMINE VIS    } & Displays all visibilities contributing to
                         interactively selected cells, allows some\\
 {\tt                } & of them to be flagged with the remaining
                         cell value (if any) restored\\
 {\tt USER FLAG VIS  } & Auto-flags visibilities based on value ranges
                         entered on the terminal\\
 {\tt AUTO FLAG VIS  } & Auto-flags visibilities based on value ranges
                         set by the image statistics \\
 {\tt FLAG ALL       } & Flags all cells in the {\tt VIEW ONLY} range \\
 {\tt FLAG ALL WIN   } & Flags all cells in the {\tt VIEW ONLY} range
                         in the current sub-image\\
 {\tt UNDO FLAG ALL  } & Unflags a previous {\tt FLAG ALL} range\\
 {\tt UNDO FLAG WIN  } & Unflags a previous {\tt FLAG ALL} range but
                         only in the current sub-image \\ \hline
\end{tabular}
\end{center}

Most of these functions will be discussed in later sections.  The {\tt
  SET WINDOW} function is the familiar {\tt TVWINDOW} interaction in
which the first button A or B push sets the lower left corner, then
the next button A or B sets the upper right corner, until you have
selected the desired sub-window for viewing.  Then buttons C or D
exit the function with the selected sub-image promptly loaded to the
TV display.  The largest pixel replication factor which allows the
sub-image to fit on the screen will be used.  Certain menu operations
are limited to the viewed sub-image and the use of pixel replication
makes it easier to select desired cells for examination or flagging.
{\tt RESET WINDOW} selects the full image, if it will fit on the
display without pixel averaging.  If it does not fit, a pixel-averaged
image will be displayed and you will be directed to choose a sub-image
that will fit.

The {\tt FLAG ALL} option appears only when a {\tt VIEW ONLY}
operation is in effect for the current display image type.  {\tt
  FLAG ALL WIN} appears in the same circumstance when the viewed image
is a sub-image.  {\tt UNDO FLAG ALL} appears only when the current
display image type has had a {\tt FLAG ALL} operation performed.  {\tt
  UNDO FLAG WIN} appears in the same circumstance when a sub-image is
displayed.  See below for further explanation of the {\tt VIEW ONLY}
function.

A sample of the initial display is shown in Figure~\ref{fig:Target}
\footnote{For legibility in printed copies, the figures shown in this
  memo have been constructed with {\tt CHARMULT 2} to double the
  character size and with $uv$ cells somewhat larger than desirable in
  order to avoid overlapping the menus and data.}.

\begin{figure}
\begin{center}
\resizebox{6.5in}{!}{\putfig{Target.eps}}
\caption{Initial {\tt UFLAG} display.  This example is a target source
  showing all 64 spectral channels of one spectral window and is badly
  affected by RFI.  The black-and-white transfer function has been
  adjusted to show all visibilities in light grey with the bright
  white cells in need of flagging.}
\label{fig:Target}
\end{center}
\end{figure}

\subsection{Basic flagging}

Interactive flagging operations in the second menu include {\tt FLAG
  POINT} which allows you to point at a cell to be flagged and then
hit buttons A, B, or C to flag that cell.  Button D exits the
operation.  During this operation, a display at the top of the window
shows the cell coordinates in pixels and in $u$ and $v$, the cell
value, and up to two baselines which contribute to the cell (with a
{\tt +} if there are more).  This is illustrated in
Figure~\ref{fig:Flag}.  The value code letter is {\tt S} for scalar
amplitude, {\tt A} for vector amplitude, {\tt P} for phase, and {\tt
  D} for the difference between scalar and vector amplitude.  A faster
version of this function is offered by {\tt FLAG FAST}\@.  The fast
operation is started by pointing at a sample and hitting button A, B,
or C.  From then, any cell at which you point with the cursor will be
flagged until you hit button D.  Note that you ``point'' at a cell by
holding down the left mouse button.  The fast operation then consists
of clicking the left mouse button over all cells you wish to delete.
The coordinates and value for the cells selected are displayed.

\begin{figure}
\begin{center}
\resizebox{6.5in}{!}{\putfig{Flag.eps}}
\caption{{\tt UFLAG} display during {\tt FLAG POINT} using a sub-image
  of the data set shown in Figure~\ref{fig:Target}.  Five cells have
  already been flagged.}
\label{fig:Flag}
\end{center}
\end{figure}

If a whole rectangular area of the grid needs to be flagged, use the
{\tt FLAG AREA} operation.  It begins by letting you point at the
lower left corner.  Then hit button A to select the bottom left and to
start setting the upper right corner. After that, hit button A to
switch between setting the two corners, button B to flag the current
area and continue to set another area, and button C to flag the
current area and exit.  Hit button D to exit without flagging the
current area.

The {\tt UNFLAG POINT}, {\tt UNFLAG FAST}, and {\tt UNFLAG AREA}
operate in the same way to undo the flagging of selected cells.  Doing
an {\tt UNFLAG} operation on a cell which has not been flagged has no
effect.

The {\tt FLAG BASELINE} menu item lets you type into the terminal the
numbers of two antennas whose baseline you wish to flag.  If one of
the numbers is zero, all of the baselines to the other antenna number
are marked to be flagged.  When you do this, the images are changed
by flagging only those cells affected solely by the flagged
baseline(s).  The flag is written to the output flag table when you
exit as a flag for all sources, spectral channels, spectral windows,
and times for the specified baseline(s).  This is pretty drastic and
so should be reserved for special, egregious cases. {\tt UNFLAG
  BASELINE} undoes all (or part) of a previous {\tt FLAG BASELINE}.
You can, for example, flag all baselines to antenna 18 and then unflag
baseline 1-18.

\subsection{VIEW ONLY flagging}

The {\tt VIEW ONLY} operation introduces a form of clipping.  As
illustrated in Figure~\ref{fig:Viewonly}, selecting this menu item
produces an interaction using the step wedge at the top of the
displayed image.  Hit button A or B to switch between setting of the
top value and the bottom value of the image value range you wish to
view.  The current value range is shown at the top of the TV display.
During the interaction, the black-and-white image transfer function is
used to make all values outside the range completely black.  If you
hit button C, these values are used to control a full image load
making all excluded image values invisible.  Hit button D instead to
exit the operation without setting a ``view-only'' range and state.

\begin{figure}
\begin{center}
\resizebox{6.5in}{!}{\putfig{Viewonly.eps}}
\caption{{\tt UFLAG} display during {\tt VIEW ONLY} using a sub-image
  of the data set shown in Figure~\ref{fig:Target}.  Flagged cells are
  not visible and cells with value less than 4.4 or greater than
  314 Jy (!) are also not visible.}
\label{fig:Viewonly}
\end{center}
\end{figure}

When a view-only state is set, the menu items {\tt FLAG ALL} and, if
you are viewing a sub-image, {\tt FLAG ALL WIN} are shown as
illustrated in Figure~\ref{fig:VOstate}.  Note that in the view only
state, the only cells shown are those in the range that was set and
that this includes flagged cells as well as unflagged cells.
Selecting {\tt FLAG ALL} will proceed to flag all pixels in the
selected value range over the full image, while {\tt FLAG ALL WIN}
will only flag cells in the selected sub-image.  If one of these is
selected, the operation and its values are recorded in a list
separated by image type.

\begin{figure}
\begin{center}
\resizebox{6.5in}{!}{\putfig{VOstate.eps}}
\caption{{\tt UFLAG} display after {\tt VIEW ONLY} using a sub-image
  of the data set shown in Figure~\ref{fig:Target}.  Five cells have
  already been flagged but only 4 are in the view-only range.  Note
  the addition of {\tt FLAG ALL} and {\tt FLAG ALL WIN} menu options.}
\label{fig:VOstate}
\end{center}
\end{figure}

You may turn off the view-only state with the menu item {\tt VIEW
  ALL}\@.  The view-only state is also reset whenever the {\tt FLAG
  ALL} or {\tt FLAG ALL WIN} option is selected or when a new image
type is selected.  If a {\tt FLAG ALL} operation has been performed
for the currently selected image type, menu items {\tt UNDO FLAG ALL},
and, if you are viewing a sub-image, {\tt UNDO FLAG WIN} will appear.
The {\tt UNDO FLAG} operations use the list created in either of the
{\tt FLAG ALL} operations to allow you to undo one or more of the
operations over the entire image or only in the sub-image.  If there
have been more than one {\tt FLAG ALL} operations on the current image
type, the program will show you the list of the operations and ask you
which you wish to undo.  Enter a number in the terminal window.  If
you {\tt UNDO FLAG ALL}, then the selected operation will be removed
from the list, but, if you only undo it in the sub-image, then it will
remain in the list.

\subsection{Flagging input visibilities}

All flagging operations considered so far cause every visibility
sample contributing to a flagged cell to be marked as bad in the
output flag table.  This may be a serious over-kill.  The gridding
operation has also recorded the visibility number of every visibility
that contributes to every cell in the grid.  This enables {\tt UFLAG}
to review the visibilities at selected cells and potentially mark them
as flagged individually rather than as a whole.  The three menu
operations described below do this in different ways, but all three
end up writing records in a ``flag command'' ({\tt FC}) table attached
to the input $uv$ data set.  Visibilities flagged in this way remain
flagged.  They are removed from the linked lists.  The only way to
restore them is to exit with {\tt ABORT} which destroys the {\tt FC}
table without applying it.
\vfill\eject

The first of these operations, {\tt EXAMINE VIS}, is a ``manual''
editing function.  It brings up an interactive display much like {\tt
  FLAG PIXEL}, complete with display of the cell coordinates and
value.  Hit button A, B, or C to examine a cell.  The task then
re-reads the input data set at all samples contributing to the
selected cell and applying all flags and calibration used when making
the original grid.  It lists the samples including a sample number,
time, baseline, IF, channel range, amplitude, and phase.  Then {\tt
  UFLAG} asks which one to flag.  Enter a number on the terminal to
delete that visibility.  The question is repeated until you enter a
zero.  Then, if you asked for some visibilities to be deleted, the
remaining ones (if any) are averaged and the image grid values and TV
display changed to the new values.  Suitable {\tt FC} table records
are written for later use.  If no samples remain, the image cell is
marked as unsampled.  The interactive selection of cells to examine
continues until button D is hit.

The second menu choice to examine the input visibilities individually
is called {\tt USER FLAG VIS}\@.  It begins by asking several
questions to which you need to enter values on the terminal.  The
first question is ``Examine pixels in range (2 values)'' which wants
the lower limit and upper limit of values in the current image type
that are to be examined.  Then it asks ``Flag amplitudes below and
above (2 values)'' which wants the lower and upper limits for
visibility amplitudes in Jy which are considered good.  All those
below and above these limits will be flagged.  Note that 0 or $-1$ is
perfectly acceptable as the lower limit if you do not want to flag low
visibilities (\eg\ target sources).  The third question is ``'Flag
phases below and above (2 values)'' which wants similarly the lower
and upper limits for good phases.  Again, note that an answer such as
-200\ 200 is probably the correct answer for non-calibration sources.
The final question is ``'Examine flagged (-1), unflagged (1), or both
(0)'' which determines which cells will be examined.  {\tt UFLAG} then
examines every cell {\it in the current sub-image} and, for those
cells selected by the examination range and flag state parameter,
performs a detailed examination.  It reads every visibility record
contributing to the chosen cell and, for every visibility outside the
allowed amplitude and phase ranges, causes the visibility to be
recorded in the {\tt FC} table and removed from the linked lists.  The
remaining visibilities (if any) are averaged and the images updated.
If no visibilities remain, the cell in the images is marked as
unsampled and, briefly, the cell is also marked in cyan as flagged.
When the operation completes, the cyan marks disappear since the cell
is now viewed as unsampled rather than flagged.

The third menu choice, {\tt AUTO FLAG VIS}, performs an operation
similar to {\tt USER FLAG VIS} except that it uses the image
statistics (displayed above the step wedge) to set the cutoff limits.
These statistics are not adequate to set limits when the phase and
scalar minus vector image types are displayed, so {\tt AUTO FLAG VIS}
does nothing when these types are selected.  For non-calibrator
sources (adverb {\tt ISCALIB} $\leq 0$), all unflagged cells above
the mean plus {\tt APARM(1)} times the rms are examined.  Any
visibility amplitude greater than the median plus  {\tt APARM(2)}
times the rms is eliminated.  For calibrator sources ({\tt ISCALIB} $>
0$), all unflagged cells below the mean minus {\tt APARM(1)} times the
rms and above the mean plus {\tt APARM(1)} times the rms are examined.
Visibilities having amplitudes more than {\tt APARM(2)} times the
rms from the median are eliminated as are those with phases more than
{\tt APARM(3)} times the rms divided by the median (in radians) away
from zero.  The default values for {\tt APARM} are 3, 3, and 4 and are
used whenever the input value of the particular {\tt APARM({\it i})}
is less than 1.0  The operation is performed only in the current
sub-image.  As with the previous two operations, any remaining
visibilities in the cell are averaged and the images updated.  If none
remain, the cell is temporarily marked as flagged and also as
unsampled.

A sample figure for these operations is not really possible --- a movie
would be more appropriate as one watches the bright pixels be
corrected in sequence.  These operations are painstaking and so are
rather slow.

\subsection{Exiting and re-starting {\tt UFLAG}}

There are four menu operations to exit or re-start {\tt UFLAG}\@.  The
simplest one is {\tt ABORT} which exits the task deleting the {\tt FC}
table (if any) and writing no output flag table.  All three of the
others begin by writing the output flag table if any flags were
generated.  The input flag table ({\tt FLAGVER $\geq 0$}) is copied to
the output flag table if it is a new flag table.  Then the data set is
re-read, the location in the grid of each re-read sample is
determined, the counts at each cell are decremented, and, if
appropriate, a record written to the output flag table.  Then all {\tt
  FLAG BASELINE} flags are written to the output flag table.  If there
is an {\tt FC} table, its rows are translated into commands written to
the output flag table.  If {\tt PRTLEV $> 1$}, an elaborate summary of
flagging by correlator is printed.  Then the {\tt FC} table is
deleted.  At this point, menu choice {\tt FLAG + EXIT} simply exits
the task.

Menu choice {\tt FLAG + REPEAT} instead resets the input flag table
({\tt FLAGVER}) to the output flag table, sets the output flag table
to be a new one, and loops to construct the grid images over again.
{\tt FLAG + SWITCH} also does this, but, additionally, it switches the
polarization in pairs.  Thus RR and LL switch, RL and LR switch, I
and V switch, Q and U switch, XX and YY switch, and XY and YX switch.
Note that changing the input and output flag tables depends on the
output flag table having been a new table, or at least the highest
numbered one.  Really, you should always set {\tt OUTFGVER} to zero.

\section{Conclusions}

\begin{figure}
\begin{center}
\resizebox{6.5in}{!}{\putfig{Preflagged.eps}}
\caption{{\tt UFLAG} display using an input flag table generated by a
  sequence of \AIPS\ flagging tasks.  Scalar average is shown since it
reveals a region of higher than normal amplitudes while the vector
average looks quite okay because of very noisy phases.}
\label{fig:Preflagged}
\end{center}
\end{figure}

It is clear that {\tt UFLAG} is a useful addition to the arsenal of
\AIPS\ data flagging tasks.  Just how useful remains to be seen.  If
an image shows stripes, then {\tt UFLAG} is both easier to use and
more powerful than the traditional Fourier-transform approach and
should show the bad cells that are causing the stripes.  However,
flagging every sample contributing to a grid cell is in many cases
excessive.  It has been asserted that there are situations with noisy
input data where stripes appear while imaging.  The data are said to
be too noisy to flag individually, but with enough observing sessions,
the $uv$ grid can reveal bad cells.  {\tt UFLAG} is able to address
this situation easily, but the author has yet to see an example.  The
operations that address the visibilities individually are really only
slightly better informed clipping operations which might be addressed
more simply with \AIPS\ task {\tt CLIP}\@.

Of more concern than these points is the fact that each visibility
sample falls in exactly one cell.  This has been seen to disguise
really bad visibilities if there are enough good samples also
contributing to the cell or if the bad visibility amplitudes in the
cell have widely varying phase.  The averaged cell will probably have
an incorrect amplitude and phase, but perhaps not so incorrect as to
stand out from its neighbors.  A slight change in the grid sampling,
however, may suddenly make some of the bad samples very visible while
disguising others.  Thus {\tt UFLAG} should be run more than once
using different $u$ and $v$ sampling.

Perhaps a useful sequence to flag data sets would be {\tt FTFLG} to
flag spectral channels that are generally bad, one or two passes of
{\tt RFLAG} using the cross-hand data, a pass of {\tt RFLAG} using the
parallel-hand data, and then a {\tt CLIP}\@.  Since {\tt RFLAG} gets
its power from examining data over short time intervals, it is
probably best to run {\tt CLIP} afterwards to catch any egregious
samples missed by {\tt RFLAG}\@.  If it is run first, the contiguity
of the samples may be lost allowing more bad data to slip through.
{\tt UFLAG} using the output flag table from this sequence could be a
very powerful way to look for any residual problems either before, or
after, some initial imaging.

Figure~\ref{fig:Preflagged} shows an example of a data set that has
passed through {\tt FTFLG}, {\tt RFLAG}, and clipping using {\tt
  WIPER}\@.  The vector average image looks fine, the phase image
is a big mess (to human eyes), but the scalar and scalar minus vector
images reveal a region of the $uv$ plane that differs significantly
from the rest.  Often this can occur from phase discrepancies between
the samples contributing to the grid cells.  In this case however,
phases are noisy everywhere in this target source while this region
has higher than normal visibility amplitudes.  Many of the grid cells
have hundreds of samples contributing from only one or two baselines
and a few spectral channels.  A {\tt VIEW ONLY} followed by a {\tt
  FLAG ALL} would probably do almost the same as an {\tt AUTO FLAG
  VIS} and do it very much faster.  An {\tt AUTO FLAG VIS} was run on
a modest sub-image near the center of this image and, eventually,
produced several hundred thousand entries in the {\tt FC} table.  Note
too that any algorithm that depends on a cell standing out from its
immediate neighbors will fail in the present situation.

\section{Acknowledgments}

The author wishes to acknowledge Emmanuel Momjian for numerous useful
conversations and for providing the data set used both to test {\tt
  UFLAG} and to make the figures for this Memo.  It is a CASA tutorial
data set initially observed on the EVLA by Sanjay Bhatnagar 23 August
2010.  The observations are of a weak, extended supernova remnant
G55.7+3.4 and the spectral window appearing in all figures is centered
upon 1.32 GHz.

\end{document}
