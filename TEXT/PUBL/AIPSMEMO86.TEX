%-----------------------------------------------------------------------
%;  Copyright (C) 1995
%;  Associated Universities, Inc. Washington DC, USA.
%;
%;  This program is free software; you can redistribute it and/or
%;  modify it under the terms of the GNU General Public License as
%;  published by the Free Software Foundation; either version 2 of
%;  the License, or (at your option) any later version.
%;
%;  This program is distributed in the hope that it will be useful,
%;  but WITHOUT ANY WARRANTY; without even the implied warranty of
%;  MERCHANTABILITY or FITNESS FOR A PARTICULAR PURPOSE.  See the
%;  GNU General Public License for more details.
%;
%;  You should have received a copy of the GNU General Public
%;  License along with this program; if not, write to the Free
%;  Software Foundation, Inc., 675 Massachusetts Ave, Cambridge,
%;  MA 02139, USA.
%;
%;  Correspondence concerning AIPS should be addressed as follows:
%;          Internet email: aipsmail@nrao.edu.
%;          Postal address: AIPS Project Office
%;                          National Radio Astronomy Observatory
%;                          520 Edgemont Road
%;                          Charlottesville, VA 22903-2475 USA
%-----------------------------------------------------------------------
% NOTE: You must have wfpoln.sty in working directory

\documentstyle[11pt,FIG/AM86.STY]{article}
\input psfig


	\pagestyle{myheadings}
	\setlength\textwidth{36pc}
	\setlength\textheight{52pc}
	\setlength\topmargin{-1cm}
        \setlength\oddsidemargin{0.5in}
        \setlength\evensidemargin{0in}
\newcommand{\AIPS}{{$\cal AIPS\/$}}

\newcommand{\memnum}{86}
\newcommand{\whatmem}{\AIPS\ Memo \memnum}

\begin{document}

   \hphantom{Hello World}
   \vskip -25pt
\centerline{\fbox{{\large\whatmem}}}
   \vskip 28pt

\centerline{\LARGE\bf Widefield Polarization Correction of}
\medskip
\centerline{\LARGE\bf VLA Snapshot Images at 1.4 GHz}

\author{\large W. D. Cotton}
\bigskip
\affil{National Radio Astronomy Observatory, Charlottesville, VA}
\medskip
\centerline{16 March, 1994}

\begin{abstract}

\noindent
This memo describes measurments of the off--axis polarization pattern
of the VLA antennas at frequencies near 1.4 GHz and techniques for
correcting these effects in snapshot images.
\end{abstract}


\section {Introduction}

   Napier 1989 has shown that the instrumental polarization response
will vary across the primary beam of the antenna pattern;
the determination and removal of this effect is necessary for
accurate, widefield polarization imaging.
This memo describes efforts made to this end as part of
the VLA D array sky survey.  The antenna polarization pattern was only
measured at selected frequencies near 1.4 GHz and corrections
only applied to snapshot observations but the same techniques could be
used at other frequencies and adapted to extended synthesis
observations.  These corrections apply only to linear polarization;
corrections for circular polarization are signifigantly more difficult.

\section {On--axis Polarization Calibration}

   The standard practice for calibrating linear polarization data from
with the VLA (Fomalont and Perley 1989), as with other synthesis
arrays, is to determine the spurious instrumental response to an
unpolarized signal from a calibrator source on the axis of the antenna
pattern in terms of the ``leakage'' (or ``D--term'') model.
The on--axis instrumental polarization is then
removed from all data by subtracting the spurious instrumental
response determined  from the antenna ``D--terms'' and the measured
total intensity.
The effects of the parallactic angle (apparent rotation of
the feed as seen by the source) are removed from the corrected data by
rotating the phase of the observed cross-polarized visibilities by
plus or minus (LR or RL correlations) the sum of  the parallactic
angles of the antennas.
This rotation by the sum of the parallactic angles corrects the
response to the source polarization but will cause any residual
spurious instrumental response to display a polarization angle that
rotates with parallactic angle.
Because the instrumental polarization response varies
with location in the antenna pattern this procedure is adequate only
near the axis of the antenna pattern.

\begin{figure}
%\centerline{\psfig{figure=08SEP93.BEAM.PS,height=7in}}
\centerline{\psfig{figure=FIG/AM86.FIG.1,height=7in}}
\caption{
The off-axis instrumental polarization derived from 6 sources on 08
September 1993.  The contours are every percent of instrumental
polarization; the vectors have lengths proportional to the
instrumental polarization and orientations of the instrumental
polarization.
Two 50 MHz bands are averaged.
The individual scans in this composite were indistinguishable.
}
\label{avgpbeam}
\end{figure}

\section {VLA Residual Antenna Polarization Pattern}

   In order to explore the off--axis polarization pattern, observations
of the regions around bright sources were made in modified
holography mode and gridded into an image of the antenna pattern
in Stokes' I, Q, U, and V.
The details of this process are given in a later section.
A representation of the fractional linear polarization beam derived
from several sources is shown in Figure \ref{avgpbeam}.
These observations were made at a variety of observing geometries and
with a mixture of polarized and unpolarized sources and are consistent
at the level of uncertainty of the individual source measurments.
This pattern therefore appeared to be stable with observing geometry
at a single epoch.
For reference, the total intensity and circular polarization
antenna patterns are shown in Figure \ref{IVbeam}.

\begin{figure}
%\centerline{\hbox{\psfig{figure=IBEAM.PS,height=3.5in}}
%   \hbox{\psfig{figure=VBEAM.PS,height=3.5in}}}
\centerline{\hbox{\psfig{figure=FIG/AM86.FIG.2A,height=3.5in}}
   \hbox{\psfig{figure=FIG/AM86.FIG.2B,height=3.5in}}}
\caption{
Left: The normalized total intensity response of the VLA antennas
derived from 6 sources.
The contours are every 10 percent.
\hfill\break
Right: The fractional circularly polarized response of the VLA
antennas as Stokes' V derived from 6 scans.  The left-- and right--handed
responses were normalized at zero offset by the standard calibration
process.
The contours are every 5 percent; dashed contours are negative.
}
\label{IVbeam}
\end{figure}


\subsection {Frequency Effects}

   The data shown in Figure \ref{avgpbeam} are the average of two 50
MHz bandpasses centered at 1.365 and 1.435 GHz and therefore refer to
an average frequency of 1.4 GHz.  The pattern shown in Figure
\ref{avgpbeam} varies quite strongly with observing frequency.  The
linear polarization patterns in the individual 50 MHz bands are
compared in Figure \ref{compbeam}.  It therefore appears that the beam
polarization characteristics must be determined for each frequency
configuration to be corrected.

\begin{figure}
%\centerline{\hbox{\psfig{figure=PBEAM1.PS,height=3.5in}}
%   \hbox{\psfig{figure=PBEAM2.PS,height=3.5in}}}
\centerline{\hbox{\psfig{figure=FIG/AM86.FIG.3A,height=3.5in}}
   \hbox{\psfig{figure=FIG/AM86.FIG.3B,height=3.5in}}}
\caption{
The off--axis instrumental polarization in the two 50 MHz bands used.
Signifigant differences are apparent over the 70 MHz separation in the
two frequencies
}
\label{compbeam}
\end{figure}


\subsection {Stability in Time}

   If the off--axis instrumental polarization is determined primarily
by the antenna configuration then it should be relatively stable in
time.  This expectation is confirmed in Figure \ref{laterbeam} as well
as in tests of the correction process described later.  The data shown
in Figure \ref{laterbeam} were obtained using the same frequencies
and techniques as those in Figure \ref{avgpbeam} but on a different
source and approximately four months later.  Excluding the outer two
rows in the images shown in Figures \ref{avgpbeam} and \ref{laterbeam}
the average and RMS of the difference is 0.1\%.
This comparison includes all regions with an antenna gain above 30\%.
The amplitude of the difference of the polarization patterns is also
shown in Figure \ref{laterbeam}.
Much of the difference seen at high negative azimuth offsets is due to
strong interference during these observations requiring heavy editing
of one of the two 50 MHz bands.

\begin{figure}
%\centerline{\hbox{\psfig{figure=14JAN94.BEAM.PS,height=3.5in}}
%   \hbox{\psfig{figure=DIFFBEAM.PS,height=3.5in}}}
\centerline{\hbox{\psfig{figure=FIG/AM86.FIG.4A,height=3.5in}}
   \hbox{\psfig{figure=FIG/AM86.FIG.4B,height=3.5in}}}
\caption{
Left: The off--axis instrumental polarization derived from a single
scan on a single source on 14 January 1994 shown as in Figure 1.
This is the average of two 50 MHz bands.
\hfill\break
Right: The amplitude of the difference between the polarization image
shown above with that in Figure 1.  The innermost contour is at 0.5\%
and the rest are every percent.  See text for an explanation of the
values at high negative azimuth offset.
}
\label{laterbeam}
\end{figure}

\section {Measurment of the Antenna Polarization Pattern}

   The antenna polarization pattern was determined using a
modification of the holography mode of the VLA control system.  The
major modification was that all antennas were moved to the same
off-source position rather than leaving ``reference'' antennas
pointing on-source.
Thus, the polarized power pattern rather than the voltage pattern was
measured.
Prior to being gridded the data are fully amplitude, phase and
on--axis polarization calibrated and converted to i, q, u and v.
The data for each offset are time and frequency averaged and
atmospheric phase effects are removed and the values converted to
fractional polarization by dividing the polarized responses by the
``i'' correlation and then averaged over baseline.
A correction for the expected ``q'' and ``u'' correlations due to
the measured polarization of the source is then applied.
Finally, the erroneous effects of the on--axis polarization
calibration must be undone by rotating (q + ju) by minus the sum the
parallactic angles of the antennas.

The images shown in this memo are 11 $\times$ 11 rasters with 5
arcminute spacings in azimuth and elevation.
The actual rasters are made in true azimuth and elevation offset but
the beam images need to be projected onto a plane whose tangent point is
at zero offset.
For holography mode data, AIPS task FILLM (which reads the VLA archive
tapes) returns the projected azimuth and elevation offsets as the ``u''
and ``v'' coordinates attached to the measured visibilities.
Since the observations are not made on the grid being used for the image,
the observations must be interpolated onto the grid.
This process in incorporated into the AIPS task MAPBM.

   The transformation from azimuth and elevation to the projected
coordinates is given by the following (M. Kestevan private
communication):
$$ u = -cos(E)sin(A-A_0) $$
$$ v = -cos(E)sin(E_0)cos(A-A_0)+sin(E)cos(E_0) $$
where $E$ is the elevation of a given pointing, $E_0$ the elevation of
the source, $A$ is the azimuth of the pointing and $A_0$ the azimuth
of the source.

\section {Correction of Snapshot Images}

   The beam polarization images derived by the procedure above are
of fractional polarization and the spurious instrumental response as a
function of sky position is the product of this complex, fractional
polarization image and the total intensity image of the source.
However, there are two effects of the observing geometry that
complicate this correction.

   The first effect is that the instrumental polarization pattern is
fixed to the antenna and therefore rotates on the sky as the antenna
tracks the source.  For a snapshot observation this rotation
(parallactic angle) can be considered constant and the antenna
polarization image can be rotated to the orientation of
the observation by a rotation in the plane of the sky.
The amount of this rotation is the parallactic angle of the
observation.

   The second effect is that the standard on-axis polarization
calibration will have rotated the polarization vector of the
instrumental polarization by minus the sum of the parallactic angles to
remove the effects of parallactic angle from the source polarization.
This effect can be removed by rotation the instrumental polarization
image by plus the sum of the parallactic angle in the Q-U plane.

   As both of these effects vary with the observing parallactic angle
and hence with time, this correction is only straightforward for a
snapshot observation where the parallactic angle can be considered
constant.
For a snapshot image, the off-axis instrumental polarization can be
applied by making the two rotations described above to the beam
polarization image, multiplying the result times the total intensity
image and subtracting from the observed source polarization image.


   The correction of snapshot images is implemented in AIPS task
VLABP.  As part of the testing of the calibration for the VLA D array
survey, a number of observations were made of strong sources with
various pointing offsets.  These observations were processed in the
same manner as normal survey observations including widefield
polarization corrections using task VLABP.  The calibration images
were those shown in Figure \ref{avgpbeam} and were made from
observations obtained in early September 1993.  The test observations
were made in January 1994;  the results of these tests are given in
Table 1.  This table gives the offset in Right Ascension and
Declination, the parallactic angle of the observation, and the antenna
gain.
The final two columns give the amplitude of the residual instrumental
polarization without (``uncorrected'') and with (``corrected'')
off-axis polarization calibration expressed as a percentage of the
total intensity.
The source used (3C295) was unpolarized so no correction was needed
for source polarization.

   The measurments given in Table 1 are clustered at a number of
levels in the primary beam.  The typical antenna gain, the number of
measurments and the average and worst cases for each of these clusters
are given in Table 2 for images without (``raw'') and with (``corr.'')
off-axis polarization calibration.  As in Table 1 the values are the
amplitude residual instrumental polarization expressed as a percentage
of the total intensity.
Widefield polarization calibration appears to reduce the instrumental
polarization typically by a factor of 3 -- 5.

\begin{table} [t]
\begin{center}
\caption{Measurments of Sources}
\medskip
\begin{tabular}{cccccc}
\hline \hline
$\Delta$ RA [']  & $\Delta$ Dec ['] & Parallactic angle
[Deg] & gain & Uncorrected[\%] & Corrected [\%]\\
\hline
0 & 0 & 180 & 1.00 & 0.02 & 0.03 \\
0 & 5 & 156 &  0.91 & 0.39 & 0.05 \\
0 & 10 & 155 & 0.74 & 0.68 & 0.22 \\
0 & 15 & 155 & 0.51 & 1.22 & 0.25 \\
0 & 20 & 154 & 0.28 & 2.37 & 0.27 \\
0 & 25 & 154 & 0.12 & 4.34 & 0.78 \\
0 & -5 & 153 & 0.92 & 0.36 & 0.13 \\
0 & -10 & 152 & 0.74 & 1.00 & 0.21 \\
0 & -15 & 151 & 0.51 & 1.98 & 0.37 \\
0 & -20 & 151 & 0.28 & 3.70 & 0.66 \\
0 & -25 & 150 & 0.12 & 6.66 & 1.48 \\
 0 & 0   & -159 & 1.000 & 0.03 & 0.03 \\
 5  & 0  & -152 & 0.93 & 0.13 & 0.15 \\
 10 & 0  & -152 & 0.77 & 0.09 & 0.21 \\
 15 & 0  & -153 & 0.53 & 0.09 & 0.23 \\
 20 & 0  & -154 & 0.29 & 0.57 & 0.17 \\
 25 & 0  & -155 & 0.12 & 1.69 & 0.45 \\
 -5 & 0  & -154 & 0.92 & 0.32 & 0.14 \\
 -10 & 0 & -155 & 0.75 & 1.02 & 0.30 \\
 -15 & 0 & -155 & 0.51 & 2.38 & 0.49 \\
 -20 & 0 & -156 & 0.28 & 4.74 & 0.80 \\
 -25 & 0 & -156 & 0.03 & 20.2 &  \\
 0 & 5   & -157 & 0.91 & 0.40 & 0.01 \\
 0 & 10  & -158 & 0.75 & 1.12 & 0.18 \\
 0 & 15  & -158 & 0.51 & 2.38 & 0.36 \\
 0 & 20  & -158 & 0.28 & 4.50 & 0.61 \\
 0 & 25  & -158 & 0.11 & 7.70 & 1.20 \\
 0 & -5  & -160 & 0.93 & 0.14 & 0.12 \\
 0 & -10 & -162 & 0.76 & 0.57 & 0.17 \\
 0 & -15 & -163 & 0.52 & 1.58 & 0.36 \\
 0 & -20 & -164 & 0.30 & 3.96 & 0.96 \\
 0 & -25 & -165 & 0.12 & 7.52 & 1.90 \\
 7 & 7     & -164 & 0.86 & 0.47 & 0.13 \\
 14 & 14   & -164 & 0.56 & 1.33 & 0.39 \\
 21 & 21   & -165 & 0.24 & 3.17 & 1.08 \\
 -7 & 7    & -166 & 0.86 & 0.04 & 0.08 \\
 -14 & 14  & -166 & 0.55 & 0.43 & 0.07 \\
 -21 & 21  & -166 & 0.23 & 1.72 & 0.13 \\
 7 & -7    & -169 & 0.86 & 0.44 & 0.17 \\
 14 & -14  & -170 & 0.58 & 1.47 & 0.36 \\
 21 & -21  & -171 & 0.26 & 3.48 & 0.69 \\
 -7 & -7   & -170 & 0.86 & 0.45 & 0.17 \\
 -14 & -14 & -171 & 0.57 & 1.30 & 0.20 \\
 -21 & -21 & -172 & 0.24 & 2.90 & 0.53 \\
\hline
\end{tabular}
\medskip

\end{center}
\end{table}

\begin{table} [b]
\begin{center}
\caption{Summary of Corrections}
\medskip
\begin{tabular}{cccccc}
\hline \hline
Gain & Number & avg. raw[\%] & max. raw[\%] & avg. corr.[\%] & max.
					    corr.[\%] \\
\hline
0.90 & 10 & 0.31 & 0.47 & 0.12 & 0.17 \\
0.75 &  6 & 0.75 & 1.12 & 0.22 & 0.30 \\
0.50 & 10 & 1.42 & 2.38 & 0.31 & 0.49 \\
0.25 & 10 & 3.10 & 4.74 & 0.59 & 1.08 \\
0.12 &  5 & 5.58 & 7.70 & 1.16 & 1.90 \\
%\hline
\end{tabular}
\medskip
\end{center}
\end{table}


\section {Conclusions}

   The effects of off--axis instrumental polarization can be a serious
problem when widefield polarization observations are made.
The uncorrected residual instrumental polarization can be several
percent in regions of signifigant antenna gain.
Routines MAPBM and VLABP have been implemented in the AIPS image
processing system which can make beam polarization images from
quasi--holography mode observations and apply these corrections to
widefield snapshot images.  This correction appears to reduce the
instrumental polarization by factors of 3 to 5 over regions of
interest in the primary beam.  In the tests presented here, the
residual instrumental polarization after full beam correction did not
exceed 0.5\% within the half power of the primary antenna pattern.


\medskip
\centerline{\Large Acknowledgements}
\medskip
   I would like to thank Mike Kestevan and Ken Sowinski for their help
in understanding the holography mode of the VLA and Mark Holdaway for
discussions on antenna beam polarization
{\footnotesize

\medskip
\noindent


% Abbreviations for regular journals, IAU style.  Use as : e.g., \MN185,207.
% adapt the format for your own additional journal
%
\def\AandA#1,#2.{{\it Astron. Astrophys.}, {\bf #1}, #2.}
\def\AandAS#1,#2.{{\it Astron. Astrophys. Suppl.}, {\bf #1}, #2.}
\def\AJ#1,#2.{{\it Astron. J.}, {\bf #1}, #2.}
\def\APJ#1,#2.{{\it Astrophys. J.}, {\bf #1}, #2.}
\def\APJS#1,#2.{{\it Astrophys. J. Suppl.}, {\bf #1}, #2.}
\def\APJL#1,#2.{{\it Astrophys. J. (Letters)}, {\bf #1}, #2.}
\def\APLETT#1,#2.{{\it Astrophys. Letters}, {\bf #1}, #2.}
\def\ARAA#1,#2.{{\it Ann. Rev. Astron. Astrophys.}, {\bf #1}, #2.}
\def\ASS#1,#2.{{\it Astrophys. Space Sci.}, {\bf #1}, #2.}
\def\BAAS#1,#2.{{\it Bull. Am. Astron. Soc.}, {\bf #1}, #2.}
\def\CJP#1,#2.{{\it Can. J. Phys.}, {\bf #1}, #2.}
\def\IEEE#1,#2.{{\it Proc. Inst. Elec. Electron. Engrs.}, {\bf #1}, #2.}
\def\MAG#1,#2.{{\it Mitt. Astron. Ges.}, {\bf #1}, #2.}
\def\MEMRAS#1,#2.{{\it Mem. Roy. Astron. Soc.}, {\bf #1}, #2.}
\def\MN#1,#2.{{\it Monthly Notices Roy. Astron. Soc.}, {\bf #1}, #2.}
\def\NAT#1,#2.{{\it Nature}, {\bf #1}, #2.}
\def\PASJ#1,#2.{{\it Publ. Astron. Soc. Japan}, {\bf #1}, #2.}
\def\PASP#1,#2.{{\it Publ. Astron. Soc. Pacific}, {\bf #1}, #2.}
\def\PhysRevA#1,#2.{{\it Phys. Rev.~A}, {\bf #1}, #2.}
\def\RADIOSCI#1,#2.{{\it Radio Sci.}, {\bf #1}, #2.}
\def\SCI#1,#2.{{\it Science}, {\bf #1}, #2.}
\def\RevModPhys#1,#2.{{\it Rev. Mod. Phys.}, {\bf #1}, #2.}
\def\SovAst#1,#2.{{\it Soviet Astron.}, {\bf #1}, #2.}
\def\SovAstL#1,#2.{{\it Soviet Astron. Lett.}, {\bf #1}, #2.}
\def\AstZh#1,#2.{{\it Astron. Zh.}, {\bf #1}, #2.}
\def\Pisma#1,#2.{{\it Pis'ma Astron. Zh.}, {\bf #1}, #2.}

%use these abbreviations as examples for books.

\def\SOCORRO#1.{{in {\it Parsec-Scale Radio Jets}, ed. J.~A. Zensus
    and T.~J. Pearson (Cambridge: Cambridge University Press), p.~#1.}}
\def\BIGBEAR#1.{{in {\it Superluminal Radio Sources}, ed. J.~A. Zensus
    and T.~J. Pearson (Cambridge: Cambridge University Press), p.~#1.}}
\def\SYN#1.{{in {\it Synthesis Imaging in Radio Astronomy}, ed. R.~A.
Perley, F.~R. Schwab, and A.~H. Bridle (San Francisco: Astronomical
Society of the Pacific), p.~#1.}}

\section* {References}

\tref{Fomalont, E.~B. and Perley, R.~A. 1989, \SYN 83.}

\tref{Napier, P.~J. 1989, \SYN 39.}
%\tref{Biretta, J. A., Moore, R. L., and Cohen, M. H.
%%1986, \APJ 308,93.}
%
%\tref{Krichbaum, T.~P. 1990$a$, Ph.~D.~thesis, Universit\"at Bonn.}
%
%\tref{Krichbaum, T.~P. 1990$b$, \SOCORRO 83.}
%
%\tref{Moore, R. L., Readhead, A. C. S., and B{\aa\aa}th, L.
%1983, \NAT 306,44.}
%
%\tref{Unwin, S. C., Cohen, M. H., Pearson, T. J., Seielstad, G. A., Simon, R.
%      S., Linfield, R. P., and Walker, R. C.
%1983, \APJ 271,536.}
%
%\tref{Tang, G., Bartel, N., Ratner, M. I., Shapiro, I. I., B{\aa \aa}th, L. B.,
%and R\"onn\"ang, B. 1990, \SOCORRO 32.}
%
%\tref{Zensus, J.  A., and Pearson, T.  J.  (ed.) 1987, {\it Superluminal
%Radio Sources} (Cambridge: Cambridge University Press).}
%
%\tref{Zensus, J. A., and Pearson, T. J. (ed.) 1990, {\it Parsec-Scale Radio Jets}
%(Cambridge: Cambridge University Press).}
%
} % end of footnotesize

\end{document}


