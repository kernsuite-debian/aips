%-----------------------------------------------------------------------
%;  Copyright (C) 2014-2015, 2017
%;  Associated Universities, Inc. Washington DC, USA.
%;
%;  This program is free software; you can redistribute it and/or
%;  modify it under the terms of the GNU General Public License as
%;  published by the Free Software Foundation; either version 2 of
%;  the License, or (at your option) any later version.
%;
%;  This program is distributed in the hope that it will be useful,
%;  but WITHOUT ANY WARRANTY; without even the implied warranty of
%;  MERCHANTABILITY or FITNESS FOR A PARTICULAR PURPOSE.  See the
%;  GNU General Public License for more details.
%;
%;  You should have received a copy of the GNU General Public
%;  License along with this program; if not, write to the Free
%;  Software Foundation, Inc., 675 Massachusetts Ave, Cambridge,
%;  MA 02139, USA.
%;
%;  Correspondence concerning AIPS should be addressed as follows:
%;          Internet email: aipsmail@nrao.edu.
%;          Postal address: AIPS Project Office
%;                          National Radio Astronomy Observatory
%;                          520 Edgemont Road
%;                          Charlottesville, VA 22903-2475 USA
%-----------------------------------------------------------------------
\documentclass[twoside]{article}
\usepackage{palatino}
\renewcommand{\ttdefault}{cmtt}
% Highlight new text.
\usepackage{color}
\usepackage{alltt}
\usepackage{graphicx,xspace,wrapfig}
\usepackage{pstricks}  % added by Greisen
\definecolor{hicol}{rgb}{0.7,0.1,0.1}
\definecolor{mecol}{rgb}{0.2,0.2,0.8}
\definecolor{excol}{rgb}{0.1,0.6,0.1}
\newcommand{\Hi}[1]{\textcolor{hicol}{#1}}
%\newcommand{\Hi}[1]{\textcolor{black}{#1}}
\newcommand{\Me}[1]{\textcolor{mecol}{#1}}
%\newcommand{\Me}[1]{\textcolor{black}{#1}}
\newcommand{\Ex}[1]{\textcolor{excol}{#1}}
%\newcommand{\Ex}[1]{\textcolor{black}{#1}}
\newcommand{\No}[1]{\textcolor{black}{#1}}
\newcommand{\hicol}{\color{hicol}}
%\newcommand{\hicol}{\color{black}}
\newcommand{\mecol}{\color{mecol}}
%\newcommand{\mecol}{\color{black}}
\newcommand{\excol}{\color{excol}}
%\newcommand{\excol}{\color{black}}
\newcommand{\hblack}{\color{black}}
%
\newcommand{\AIPS}{{$\cal AIPS\/$}}
\newcommand{\eg}{{\it e.g.},}
\newcommand{\ie}{{\it i.e.},}
\newcommand{\etal}{{\it et al.}}
\newcommand{\tablerowgapbefore}{-1ex}
\newcommand{\tablerowgapafter}{1ex}
\newcommand{\keyw}[1]{\hbox{{\tt #1}}}
\newcommand{\sub}[1]{_\mathrm{#1}}
\newcommand{\degr}{^{\circ}}
\newcommand{\vv}{v}
%\newcommand{\vv}{\varv}
\newcommand{\eq}{\hbox{\hspace{0.6em}=\hspace{0.6em}}}
\newcommand{\newfig}[2]{\includegraphics[width=#1]{data.fig#2}}
%\newcommand{\putfig}[1]{\includegraphics{data.fig#1.eps}}
\newcommand{\putfig}[1]{\includegraphics{#1.eps}}
\newcommand{\whatmem}{\AIPS\ Memo \memnum}
\newcommand{\boxit}[3]{\vbox{\hrule height#1\hbox{\vrule width#1\kern#2%
\vbox{\kern#2{#3}\kern#2}\kern#2\vrule width#1}\hrule height#1}}
%
\newcommand{\memnum}{118 \Hi{Revised}}
\newcommand{\memtit}{Modeling Spectral Cubes in \AIPS}
\title{
   \vskip -35pt
   \fbox{{\large\whatmem}} \\
   \vskip 28pt
%   \vskip 10pt
%   \fbox{{\Huge \Me{D R A F T}}}
%   \vskip 10pt
   \memtit \\}
\author{Eric W. Greisen}
%
\parskip 4mm
\linewidth 6.5in                     % was 6.5
\textwidth 6.5in                     % text width excluding margin 6.5
\textheight 9.0 in                  % was 8.81
\marginparsep 0in
\oddsidemargin .25in                 % EWG from -.25
\evensidemargin -.25in
\topmargin -0.4in
%\topmargin 0.2in
\headsep 0.25in
\headheight 0.25in
\parindent 0in
\newcommand{\normalstyle}{\baselineskip 4mm \parskip 2mm \normalsize}
\newcommand{\tablestyle}{\baselineskip 2mm \parskip 1mm \small }
%
%
\begin{document}

\pagestyle{myheadings}
\thispagestyle{empty}

\newcommand{\Rheading}{\whatmem \hfill \memtit \hfill Page~~}
\newcommand{\Lheading}{~~Page \hfill \memtit \hfill \whatmem}
\markboth{\Lheading}{\Rheading}
%
\vskip -.5cm
\pretolerance 10000
\listparindent 0cm
\labelsep 0cm
%
%

\vskip -30pt
\maketitle

\normalstyle
\begin{abstract}
  \AIPS\ has done Gaussian fitting along the $x$-axis of image cubes
  with task \keyw{XGAUS} since the 1980s.  That task has recently been
  overhauled to be much easier to use and much more capable.  In like
  fashion, new tasks \keyw{ZEMAN} and \keyw{RMFIT} have been developed.
  The former fits the standard leakage and scaling terms for Stokes V
  cubes, including a new option to do this for each of the Gaussians
  found by \keyw{XGAUS}\@.  The latter fits polarization models to
  Stokes Q and U cubes, using the output of Faraday Rotation Measure
  Synthesis (\AIPS\ task \keyw{FARS}) to assist with initial guesses.
  The models can contain multiple components each with a polarization
  flux, angle, rotation measure, and rotation measure ``thickness.''
  The present memo will describe the functions of these tasks in some
  detail with numerous graphical examples.  This revision also
  discusses \Hi{changes made for the {\tt 31DEC15} release and two new
  tasks to plot spectra with model fits and} a number of tasks which
  make visibility and image model files.
\end{abstract}

\renewcommand{\floatpagefraction}{0.75}
\typeout{bottomnumber = \arabic{bottomnumber} \bottomfraction}
\typeout{topnumber = \arabic{topnumber} \topfraction}
\typeout{totalnumber = \arabic{totalnumber} \textfraction\ \floatpagefraction}

\section{Introduction}

{\tt XGAUS} was written, and re-written, a long time ago in the hopes
that converting a spectral cube, with many spectral channels at each
spatial location, into one or more sets of images of peak, center,
width, and integral would simplify the data presentation.  One could
even hope that the separate Gaussians fit might have some separate
physical reality.  The task could fit a single Gaussian to each
spatial position without much human guidance, but detailed human
interactivity was required for more complex spectra.  Use of the task
involved an endurance contest between the human and the power company,
requiring both luck and skill on the part of the human. The task had
no way to back up and fix fits generated at previous spatial positions
and either wrote out all of the fit parameters in images or discarded
all work done on the cube.  Clearly, this was not a desirable
situation and would-be users tended to prefer fitting routines in
other packages such as {\tt GYPSY}\@.

There has always been some interest in fitting rotation-measures to
polarization images taken at different frequencies.  Until the advent
of the Jansky VLA however, the images available for fitting rotation
measures were at a modest number of scattered frequencies and could
generally be fit by rather simple algorithms.  This is no longer the
case and better fitting algorithms were required.  Initially there was
considerable interest in a method known as Faraday Rotation Measure
Synthesis.\footnote{Brentjens, M. A. and de Bruyn, A. G. 2005, {\it
Astron.\ \& Ap.}, 1217.}  The method was implemented in \AIPS\ by
Leonid Kogan as task \keyw{FARS} with supplementary tasks and
additions by the author.  However, it was found that the ``synthesized
beam width'' in this method was too wide to separate most
scientifically interesting cases, even if the data were taken to
frequencies as low as 1 GHz.  Data from lower frequencies would be
needed to produce better component discrimination, but most radio
sources are completely depolarized at these low frequencies.  A more
direct fitting method is therefore required.

The author undertook a re-write of \keyw{XGAUS} in order to make it a
much more useful and user-friendly task and to provide a software
framework for a new task to do polarization fitting.  The suggestion
was then made to create a task to fit standard Zeeman-splitting
parameters as well.  The following Memo attempts to describe in detail
the function of all three of the tasks that have resulted.  This memo
\Hi{has been revised for the {\tt 31DEC15}} version of these tasks and
does not attempt to describe differences with the {\tt 31DEC13}
\Hi{and {\tt 31DEC14} versions}, in which these tasks first appeared.
\Hi{Two new tasks, {\tt AGAUS} and {\tt ZAMAN}, have appeared in the
  {\tt 31DEC17} release of \AIPS\ to fix Gaussians and Zeeman
  splitting in absorption spectra.  They are described in \AIPS| Memo
  122.}

\Hi{Two new tasks have been written to prepare plot files from the
fits done by these tasks.}  There are also tasks in \AIPS\ which
modify $uv$-data files to add model components both spatially and
spectrally.  In addition there are tasks which add polarization and
spectral models to images.  These will be described in a final
section.

\section{Gaussian fitting: {\tt XGAUS}}

Gaussian fitting is the process of determining the peak brightness,
center position, and width of a number of one-dimensional Gaussians,
plus a baseline, which best matches a one-dimensional slice of input
data.  {\tt XGAUS} will do this for every voxel in a three-dimensional
image cube, taking each row as the one-dimensional slice.  It then
produces images over the second and third input axes of the fit
parameters.  Normal usage has the first axis of the cube as a spectral
axis, either in frequency or velocity units, and the second and third
axes are celestial coordinates.  This usage will be assumed in the
later discussion, but it is not required.  Mathematically, the function
fit is
\begin{equation}
  T(x,y,z) = A + Bx + \sum_1^{\tt NGAUSS} T_i e^{-\alpha (x - x_i)^2 /
     \sigma_i^2}
\end{equation}
where $x$ is the coordinate along the first axis of the image cube,
$A(y,z) + B(y,z)x$ is the spectral baseline, $T_i(y,z)$ are the peak
values of each of the {\tt NGAUSS} components, $x_i(y,z)$ are the
center channels of the components, $\sigma_i(y,z)$  are the full
widths at half maximum of the components and $\alpha = 4 \ln(2)$ to
impart this meaning to the $\sigma_i$.

\subsection{Inputs}

The usual {\tt INNAME} {\it et al.} adverbs define the cube to be fit
and the {\tt OUTNAME} {\it et al.} adverbs define the (eventual)
output name.  The {\tt INVERS} adverb controls which {\tt XG} table is
used by the task, with $\leq 0$ meaning a new table.  {\tt BLC} and
{\tt TRC} define the pixel ranges to be used in the current execution,
where {\tt BLC(1)} and {\tt TRC(1)} control the spectral channels that
will be fit and {\tt BLC(2)}, {\tt TRC(2)}, {\tt BLC(3)}, and {\tt
TRC(3)} control the area in celestial coordinates over which the
fitting is done.  \Hi{New {\tt XG} table files are created for the
entire input image cube, but are filled with peak brightnesses limited
by {\tt BLC(1)} through {\tt TRC(1)}\@.  Therefore, it would be wise
to use as much of the first axis as is reliable when creating a new
table.  The second and third values of these adverbs then limit the
area over which fitting is done during this execution of {\tt
  XGAUS}\@.}  Adverbs {\tt YINC} and {\tt ZINC} control the stride
taken in the first pass through the cube; a second pass will then fit
all voxels not fit in the first pass.  {\tt FLUX} controls with
spectra will be fit during this execution; all positions with 3
consecutive channels averaging above {\tt FLUX} will be fit.  The
initial guess for the linear baseline \Hi{is now always zero.  Adverb
  {\tt ORDER} controls the order of the baseline with $<0$ meaning
  none, 0 meaning a constant, and $\ge 1$ meaning a constant plus a
  slope.}  {\tt DOOUTPUT} controls what files are written --- this may
be changed interactively so leave it zero at this point.  Set {\tt
  DOTV = 2} to use TV menus to prompt you.  Even when fitting only 1
Gaussian component, it is best to watch what is happening so you
should never set this adverb false.  {\tt DORESID} controls whether
residuals are plotted on the fit spectra; such plots often provide
clues when more Gaussians are needed for the best fits.  In general it
is best to set {\tt PIXRANGE} to zero to see the full range of image
values, but, if you are fitting weak Gaussians in the presence of very
strong ones, you might wish to cut off the highest values.  (This may
make it harder to set initial guesses however since you can't point at
the peak of the clipped component.)  Set {\tt LTYPE} to your favorite
type of labeling, set {\tt PIXVAL} to zero to see all positions being
fit, and leave {\tt NITER} zero since 100 is more than enough
iterations. Set {\tt NGAUSS} to the number of Gaussians to be fit in
this execution.  It may be changed if you re-start on a pre-existing
{\tt XG} table since those tables contain room for the maximum number
of Gaussians allowed (8 at this writing).  {\tt RMSLIMIT} is an upper
limit for the rms of a fit before the fit is viewed as ``failed''
which causes the TV and interaction to be turned back on after you
have turned it off.  You should get a good idea of an appropriate
value from your initial uses of {\tt XGAUS} or from your knowledge of
the noise in your data cube.

The task begins by creating an {\tt XG} table and populating each row
with the largest average brightness over three consecutive channels in
the corresponding row \Hi{in the range {\tt BLC(1)} through {\tt
TRC(1)}\@.}.  Then it reads the table every {\tt YINC} rows and {\tt
ZINC} planes and, for those with an average brightness greater than
{\tt FLUX} attempts a fit.  Your interaction with this fit will be
described below.  After the first pass, the task loops over every row
and plane fitting those positions which have enough brightness and
which have not already been fit.  Finally, after all pixels above {\tt
  FLUX} have been fit, the task goes into an ``edit'' mode.  It
constructs images of each fit parameter and the integral flux of the
Gaussian and of the uncertainties in these parameters.  You may view
these images, select positions explicitly or by their parameter values
or rms and revisit the fits of those positions.  This stage will be
described in detail below.

At any time you may exit the task and then re-start it using the
same {\tt XG} table.  Good reasons for doing this include fitting
smaller regions with each pass using the appropriate number of
Gaussians for that region.  Doing small regions which will have
similar parameter values helps a great deal with the initial guessing
done by the task (mostly using the previous solution).  You might also
fit the cube initially with a high value of {\tt FLUX} and then
re-start with a lower value to extend the areas fit.

\subsection{Fitting}

\begin{figure}
\begin{center}
\resizebox{6.0in}{!}{\putfig{XGAUS.init}}
\caption{First OH spectrum to fit, initial ``guess'' is all zero.}
\label{fig:XGAUS.init}
\end{center}
\end{figure}

The fitting process starts with a plot of the spectrum with the data
and axis labels in graphics channel one (usually yellow) and the
initial guess as X's in graphics channel two (usually green).  Then
you are offered a menu of options, either in your \AIPS\ terminal
window ({\tt DOTV = 1}) or, as shown in the present figures, on the TV
({\tt DOTV = 2}).  The first spectrum to be fit (from an OH 1720 MHz
maser source) is illustrated in Figure~\ref{fig:XGAUS.init} showing
that the first guess for more than one Gaussian is not useful.  The
menu that appears at this point is\\

\begin{figure}
\begin{center}
\resizebox{6.0in}{!}{\putfig{XGAUS.guess}}
\caption{First OH spectrum to fit: better initial ``guess'' has been
  entered.}
\label{fig:XGAUS.guess}
\end{center}
\end{figure}

\begin{center}
\begin{tabular}{|l|l|}\hline
   {\tt DO FIT}   & {\tt \hphantom{A}} \\
   {\tt RE-GUESS} & {\tt E} \\
   {\tt BAD}      & {\tt B} \\
   {\tt QUIT}     & {\tt Q} \\ \hline
\end{tabular}
\end{center}
You select a menu option by moving the cursor to the desired option
with the mouse and registering that move with the TV by clicking the
left mouse button.  The selected menu item will change color as shown
in the figure.  If you press TV ``button'' {\tt D} at this point
(actually keyboard character D), helpful information about the
selected item will appear on your terminal window.  If you press one
of TV ``buttons'' {\tt A}, {\tt B}, or {\tt C} (actually keyboard
characters A, B, C), the selected function will be performed.  The
option to {\tt QUIT} (or {\tt Q} on the terminal) causes the task to
quit at this point.  You may re-start later.  The option {\tt BAD}
({\tt B} on the terminal) will mark this position as failed and go on
to the next position.  The option {\tt DO FIT} will cause the task to
attempt the non-linear fit with the current initial guess.  The
selected option in Figure~\ref{fig:XGAUS.init} is {\tt RE-GUESS} which
causes the task to prompt you first to {\it ``Position cursor at
  center \&\ height of Gaussian component 1''}.  Move the cursor to
the peak of component 1 and press any TV button.  This selects the
peak value and center of component 1.  Then the task prompts you to
{\it ``Position cursor at half-width of Gaussian component 1''}.  Move
the cursor horizontally to the approximate position of the half-power
point of component 1 and press any TV button.  The horizontal position
of the cursor then sets the initial guess of the full width of the
component.  These prompts are then repeated for components two through
{\tt NGAUSS}\@.  If you do not want to fit a particular component at
this position, move the cursor outside the rectangular border line
(\ie\ outside the data area of the plot) before pressing the TV
button for that component; it will be omitted.

\begin{figure}
\begin{center}
\resizebox{6.0in}{!}{\putfig{XGAUS.good}}
\caption{First OH spectrum to fit: good fit obtained.}
\label{fig:XGAUS.good}
\end{center}
\end{figure}

After the improved initial guess has been entered the plot is changed
to show the new guess, as illustrated in Figure~\ref{fig:XGAUS.guess}.
This guess is good so the {\tt DO FIT} option is selected.  After TV
``buttons'' {\tt A}, {\tt B}, or {\tt C} are pressed the task attempts
the fit with the current initial guess.  Then the plot is changed with
the addition of the fit function in graphics plane 4 (usually cyan)
and the residual in graphics plane 3 (usually pink).  This is
illustrated in Figure~\ref{fig:XGAUS.good}.  A different menu appears
at this point containing\\
\begin{center}
\begin{tabular}{|l|l|}\hline
   {\tt GOOD}     & {\tt \hphantom{A}} \\
   {\tt DO FIT}   & {\tt D} \\
   {\tt RE-GUESS} & {\tt E} or {\tt R} \\
   {\tt TVOFF}    & {\tt T} \\
   {\tt HAND}     & {\tt H} \\
   {\tt BAD}      & {\tt B} \\
   {\tt 1}        & {\tt 1} \\
   {\tt 2}        & {\tt 2} \\
   {\tt QUIT}     & {\tt Q} \\ \hline
\end{tabular}
\end{center}
The option to {\tt QUIT} (or {\tt Q} on the terminal) causes the task
to quit at this point.  You may re-start later.  The option {\tt BAD}
({\tt B} on the terminal) will mark this position as failed and go on
to the next position.   The option {\tt RE-GUESS} ({\tt E} or {\tt R}
on the terminal) will loop back to prompt you for a new guess and
repeat the fit.  Options {\tt 1}, {\tt 2}, $\ldots$, {\tt NGAUSS} will
loop back to plot an initial guess with the selected number of
Gaussians.  ({\tt NGAUSS=2} in the current example.)  Option {\tt
  HAND} ({\tt H} on the terminal) will prompt you to enter using the
terminal the Gaussian parameters for each component.  Enter on one
line for each component, the peak value of the Gaussian (in image
units), the center (in pixels with respect to the reference pixel),
and width (in pixels).  Appropriate ranges of parameters in these
units can be seen from the display of the current fit values.  You may
also enter flags to cause one or more parameter values to be fixed
should you re-fit the current spectrum.  The flags are entered after
the 3 parameter values, flags $\leq 0$ mean to hold the corresponding
parameter fixed and omitted flags are taken as 1.  {\tt XGAUS} will
then repeat the display in Figure~\ref{fig:XGAUS.good} to see if you
made a good guess.  Immediately after a {\tt HAND} operation only, the
option {\tt DO FIT} is offered to go back with the hand-entered values
as the initial guess for a new fit.  Option {\tt TVOFF} allows you to
turn off interactivity, allowing the task to run using its own initial
guesses until it finds a completely unreasonable solution or one with
an rms greater than {\tt RMSLIMIT}\@.  When that happens, you are
shown the offending fit parameters and the task resumes with the plot
of Figure~\ref{fig:XGAUS.good} to allow you to try to fix things.
Option {\tt GOOD} (and other initial character on the terminal) tells
the task that you are (reasonably) happy and that it should go on to
the next position.

\begin{figure}
\begin{center}
\resizebox{6.0in}{!}{\putfig{XGAUS.HIgood}}
\caption{First 1-Gaussian HI spectrum to fit: good fit obtained
  automatically.}
\label{fig:XGAUS.HIgood}
\end{center}
\end{figure}

When fitting only a single Gaussian, {\tt XGAUS} makes an initial
guess based mostly on finding a real peak in the spectrum.  This is
quite reliable, so turning off the TV interaction may save a great
deal of effort, although there will possibly be bad positions to be
fixed up in the next stage of this task.  An example of an HI cube, on
its first position, is shown in Figure~\ref{fig:XGAUS.HIgood}.
%\vfill\eject

\subsection{Editing and output}

\begin{figure}
\begin{center}
\resizebox{6.0in}{!}{\putfig{XGAUS.edit}}
\caption{OH maser field, edit view}
\label{fig:XGAUS.edit}
\end{center}
\end{figure}

Eventually all positions selected by {\tt FLUX} and {\tt BLC} and {\tt
  TRC} will have been fit.  At this point, the task computes images of
the fit parameters plus the ``flux'' (area under the Gaussian) and
their uncertainties.  It then offers a lengthy menu of options which
will allow you to view these images and revisit positions that seem to
have produced incorrect fits.  If {\tt NGAUSS} $> 1$, options to swap
portions of image $n$ with corresponding portions of image $m$ are
also offered.  This ``edit'' menu is illustrated in
Figure~\ref{fig:XGAUS.edit}.  Note that the size of the \Hi{signal
  portion of the} OH image is quite small.  \Hi{When an appropriate
  sub-image is selected for display, as was done for this figure},
{\tt XGAUS} replicates pixels in both directions to make the image
large enough to see.  Note too that, for legibility in all figures in
this memo, the \AIPS\ TV was run with double-sized characters.

There are three kinds of editing implemented here.  In the first, the
user establishes the parameter extrema which should be viewed as
acceptable.  The extrema currently set are shown in the title lines.
Then, {\tt XGAUS} may be told to flag all solutions not meeting these
criteria, or, more profitably perhaps, to revisit those positions to
see if a better fit can be obtained.  The other editing methods are
similar, but act on a list of pixel positions.  These may be entered
by typing in values or by clicking on suspect pixel positions in the
{\tt CURVALUE} function described below.  The contents of the list may
be viewed, the solutions at the positions may be flagged, or they may
be revisited to attempt for a better solution, or the solutions at the
listed positions may be swapped between components $n$ and $m$.  The
menu will offer only appropriate swaps between components, thus 1 and
2 in our {\tt NGAUSS=2} example.  Swapping may be required if {\tt
  XGAUS} gets confused as to which component you want to call number 1
and which number 2.  After the flagging, revisiting, or swapping, the
list is cleared.    The first column of the menu shows the following:
%\vfill\eject

\begin{center}
\begin{tabular}{|l|l|}\hline
 {\tt EXIT           } & Exit {\tt XGAUS}, writing output images if
                         {\tt DOOUTPUT} is now $> 0.$\\
 {\tt SET MIN S/N    } & Set minimum amplitude S/N(s) for okay
                         solutions\\
 {\tt SET MAX RES    } & Set maximum residual for okay solutions\\
 {\tt SET PEAK RANGE } & Set peak value range(s) for okay solutions\\
 {\tt SET OFFX RANGE } & Set offset range(s) for okay solutions\\
 {\tt SET WIDTH RANGE} & Set width range(s) for okay solutions\\
 {\tt SET MAX ERR WID} & Set maximum error(s) in width for okay
                         solutions\\
 {\tt REDO ALL       } & Re-do all solutions which are not okay
                         following the above criteria\\
 {\tt FLAG ALL       } & Mark bad all solutions which are not okay\\
 {\tt OFF ZOOM       } & Turn of TV zoom\\
 {\tt OFF TRANSFER   } & Turn off black \&\ white and color TV
                         enhancements\\
 {\tt SET DOOUTPUT   } & Increment {\tt DOOUTPUT} in loop 0-3 --- with
                         1 and 3 causing residual\\
                       & images and 2 and 3 causing parameter images
                         to be written on {\tt EXIT}\\
 {\tt ADD TO LIST    } & Type in output pixel coordinates to add to
                         list\\
 {\tt SHOW LIST      } & Display coordinates in list\\
 {\tt REDO LIST      } & Re-do solutions for all pixels in list\\
 {\tt FLAG LIST      } & Flag solutions for all pixels in list\\
 {\tt SWAP LIST 1-2  } & Swap solutions for components 1 and 2 for all
                         pixels in list\\ \hline
\end{tabular}
\end{center}

The second (and potentially third and even more) menu columns
contain {\tt NGAUSS} sets of functions

\begin{center}
\begin{tabular}{|l|l|}\hline
 {\tt SHOW IMAGE A1 } & Enter image interaction with peak value of
                     component 1\\
 {\tt SHOW IMAGE C1 } & Enter image interaction with center pixel of
                     component 1\\
 {\tt SHOW IMAGE W1 } & Enter image interaction with width of
                     component 1\\
 {\tt SHOW IMAGE F1 } & Enter image interaction with "flux" of
                     component 1\\
 {\tt SHOW IMAGE EA1} & Enter image interaction with uncertainty in
                     peak value of component 1\\
 {\tt SHOW IMAGE EC1} & Enter image interaction with uncertainty in
                     center pixel of component 1\\
 {\tt SHOW IMAGE EW1} & Enter image interaction with uncertainty in
                     width of component 1\\
 {\tt SHOW IMAGE EF1} & Enter image interaction with uncertainty in
                     "flux" of component 1\\ \hline
\end{tabular}
\end{center}

\begin{figure}
\begin{center}
\resizebox{6.0in}{!}{\putfig{XGAUS.HIpeak}}
\caption{HI galaxy: peak value image interaction}
\label{fig:XGAUS.HIpeak}
\end{center}
\end{figure}

On very crowded menus, the word {\tt SHOW} may be omitted.  When you
select one of these functions most of the following operations will
appear in yet another menu.  This menu is illustrated in
Figure~\ref{fig:XGAUS.HIpeak}.  Only one of the {\tt LOAD AS} options
will appear, with the next one in the sequence offered when the
current one has been invoked.  \Hi{The {\tt SET WINDOW} option allows
you to select a sub-image to view in greater detail, while {\tt RESET
WINDOW} returns to viewing the full image.}  The {\tt SWAP $n$-$m$}
options will appear as needed when {\tt NGAUSS} $ > 1$ and are used to
impose your selection of component number $n$ and component $m$ in
case {\tt XGAUS} got confused.  The {\tt NEXT WINDOW} option appears
when needed to display an image too large to fit on the display
screen.  These options mostly invoke familiar functions from \AIPS\ to
control the {\tt FUNCTYPE} used in loading the image to the display,
to enhance the image intensities, to color the enhanced image
intensities, and to zoom the display.

\vfill\eject
\begin{center}
\begin{tabular}{|l|l|}\hline
 {\tt RETURN     } & Return to the above menus, image stays displayed\\
 {\tt LOAD AS SQ } & Re-load image with square root transfer function\\
 {\tt LOAD AS LG } & Re-load image with log transfer function\\
 {\tt LOAD AS L2 } & Re-load image with extreme log transfer function\\
 {\tt LOAD AS LN } & Re-load image with linear transfer function\\
\Hi{{\tt SET WINDOW}} & \Hi{Set a sub-image to view}\\
\Hi{{\tt RESET WINDOW}} & \Hi{Return too viewing the full image}\\
 {\tt OFF TRANSF } & Turn off enhancement done with {\tt TVTRANSF}\\
 {\tt OFF COLOR  } & Turn off color enhancements\\
 {\tt TVTRANSF   } & Black \&\ white image enhancement\\
 {\tt TVPSEUDO   } & Color enhancement of numerous sorts\\
 {\tt TVPHLAME   } & Color enhancement of flame type, multiple colors\\
 {\tt TVZOOM     } & Interactive zooming and centering of image\\
 {\tt CURVALUE   } & Display value under cursor, mark pixels for list\\
 {\tt SWAP 1-2   } & Swap solutions for components 1 and 2 interactively\\
 {\tt NEXT WINDOW} & Move to next window into large images\\ \hline
\end{tabular}
\end{center}

\begin{figure}
\begin{center}
\resizebox{6.0in}{!}{\putfig{XGAUS.HIcent}}
\caption{HI galaxy: center channel image interaction}
\label{fig:XGAUS.HIcent}
\end{center}
\end{figure}

%\begin{figure}
%\begin{center}
%\resizebox{6.0in}{!}{\putfig{XGAUS.HIwidth}}
%\caption{HI galaxy: width image interaction}
%\label{fig:XGAUS.HIwidth}
%\end{center}
%\end{figure}

Two operations in this menu are different from the usual.  {\tt
  CURVALUE} provides the capability of selecting positions for the
edit ``list.''  During the {\tt CURVALUE} operation position the
cursor over the desired pixel and press buttons {\tt A}, {\tt B}, or
{\tt C} to add that pixel to the list.  The {\tt SWAP $n$-$m$}
operation uses a TV blotch operation like that in the \AIPS\ verb {\tt
  TVSTAT} and task {\tt BLSUM}\@.  You are to mark with a ``blotch''
region those pixels in the present image which are to have their
solutions swapped with those of the selected component.  Instructions
will appear in the message window as you proceed.  Begin by
positioning the cursor at a pixel to be the first vertex of a
connected sequence of vertices and press TV button {\tt A}\@.  Move to
the next vertex and press button {\tt A} again and repeat until you
have marked all vertices for this region.  Then press button {\tt D}
if you are done with this region or button {\tt C} if you need to
re-position one of the vertices.  In this case, move the cursor to the
vertex to be re-positioned, press button {\tt A} and drag the vertex to
the corrected position.  Press button {\tt A} or {\tt B} to fix that
vertex and go on to reset another vertex or {\tt D} to end this region
and swap the solutions.  You may do this as many times as needed.

When you have finished getting the images just the way you want them,
you may write them out as \AIPS\ image files.  Select the {\tt SET
  DOOUTPUT} option until its value, shown at the top of the screen, is
what you want.  In {\tt XGAUS}, values 1 and 3 cause a residual image
cube to be written, while values 2 and 3 cause images of the parameter
values and their uncertainties to be written.  The baseline and slope
images and their uncertainties are given class {\tt CONST}, {\tt
  SLOPE}, {\tt DCONST}, and {\tt DSLOPE}, while the parameter images
and their uncertainties get class {\tt AMPL$n$}, {\tt CENTR$n$}, {\tt
  WIDTH$n$}, {\tt DAMPL$n$}, {\tt DCENT$n$}, and {\tt DWIDT$n$}, and
the flux and its uncertainty get class {\tt FLUX$n$}, and {\tt
  DFLUX$n$}, where $n$ is the component number.

%\vfill\eject
\section{Zeeman splitting: {\tt ZEMAN}}

When an intrinsically unpolarized spectral line is emitted in the
presence of a magnetic field, the right and left circular
polarizations have their frequencies shifted in opposite directions by
an amount proportional to the magnetic field (at least for modest
magnetic fields).  The traditional analysis of data to measure this
splitting works only for those cases in which the separation of
polarizations is a small fraction of the line width.  In that case,
the function that has been traditional is
\begin{equation}
     V(x,y,z) = A(y,z) I(x,y,z) + 0.5 B(y,z)\,\, \frac{dI(x,y,z)}{dx}
\label{eqn:Zeman}
\end{equation}
where $V(x,y,z)$ is the V Stokes polarization component. $I(x,y,z)$ is
the unpolarized I Stokes component, $x$ is the spectral axis value,
$(y,z)$ is the celestial coordinate value, and $A(y,z)$ and $B(y,z)$
are the parameters to be found by a linear least-squares method.
\footnote{Sault, R.J., Killeen, N. E. B.., Zmuidzinas, J., Loushin, R.
  1990, {\it Ap.~J.}, {\bf 74}, 437-461.}  The \AIPS\ task {\tt ZEMAN}
offers this model, with two choices for the method by which the
derivative of $I$ with $x$ is computed.  If the total intensity
spectrum has had {\tt XGAUS} applied, another function may be used
instead:
\begin{equation}
     V(x,y,z) = A(y,z) I(x,y,z) + 0.5 \sum_1^{\tt NGAUSS} B_i(y,z)\,
        \,\frac{dG_i(x,y,z)}{dx}
\label{eqn:ZemanGaus}
\end{equation}
where $G_i(x,y,z)$ is the spectrum of the $i$'th Gaussian component
and one solves for $A(y,z)$ and {\tt NGAUSS} component frequency
separations $B_i(y,z)$.  In the example shown below, the present data
lack the spatial resolution to separate two spectral components, but
this second method easily finds two different magnetic field values.
This source has been observed with much better spatial resolution
which separates the two components.  The present field values match
those found in the published, high-resolution image.  Note that {\tt
  ZEMAN} returns $B$ in units matching the input $x$ axis (usually Hz
or m/sec).  Any association with magnetic field values is left to the
user.

\subsection{Inputs}

The inputs for {\tt ZEMAN} are very similar to those for {\tt
  XGAUS}\@.  {\tt INNAME} {\it et al.}~specify the V polarization
cube which must be in transposed form with frequency as the first
axis.  {\tt IN2NAME} {\it et al.}~specify the corresponding I
polarization cube, similarly transposed.  The axes of the two images
must match.  {\tt BLC} and {\tt TRC} define the spectral and celestial
coordinate regions \Hi{of interest} for the fit and {\tt FLUX} gives
the lower limit in the I image for the average of three consecutive
channels for the spectra to be fit.  {\tt INVERS} specifies the input
version of the {\tt ZE} table in which the results are stored.  Zero
means to make a new one, other values mean to re-visit an existing
solution.  Adverb {\tt OPTYPE} specifies which of the above formul\ae\
are solved.  {\tt OPTYPE = 'GAUS'} says to use the {\tt XG} file {\tt
  IN2VERS} attached to the input I polarization cube to solve using
Equation~\ref{eqn:ZemanGaus}.  {\tt OPTYPE = '2SID'} says to solve
Equation~\ref{eqn:Zeman}, evaluating the derivative by
$$
\frac{dI(x,y,z)}{dx} = 0.5 \left[\, I(x+1,y,z) - I(x-1,y,z) \,\right]
$$
while any other {\tt OPTYPE} value says to evaluate the derivative
with
$$
\frac{dI(x,y,z)}{dx} = I(x,y,z) - I(x-1,y,z)
$$
while solving Equation~\ref{eqn:Zeman}.  {\tt DOOUTPUT} controls what
files are written --- this may be changed interactively so leave it
zero at this point.  Set {\tt DOTV = 2} to use TV menus to prompt you.
Even though the fitting operation is linear and so needs little or no
guidance, it is best to watch what is happening so you should never
set this adverb false.  If all seems well, you can turn off the TV
after watching a few of the solutions.  $A(y,z)$ is always fit but
should be very close to zero if you are fitting a cube which has
already had the leakage term fit and removed.  {\tt RMSLIMIT} is an
upper limit for the rms of a fit before the fit is viewed as
``failed'' which causes the TV and interaction to be turned back on
after you have turned it off.  You should get a good idea of an
appropriate value from your initial uses of {\tt ZEMAN} or from your
knowledge of the noise in your data cube.

\subsection{Fitting}

\begin{figure}
\begin{center}
\resizebox{6.0in}{!}{\putfig{ZEMAN.init}}
\caption{First OH spectrum to fit, initial ``guess'' has Gaussians but
   the two $B_i(y,z)$ are zero.}
\label{fig:ZEMAN.init}
\end{center}
\end{figure}

The fitting process starts with a plot of the I polarization spectrum
across the top and the V polarization across the bottom.  The data
and axis labels are plotted in graphics channel one (usually yellow)
and the initial guess as lines in graphics channel two (usually
green).  The Gaussians, when {\tt OPTYPE='GAUS'}, are shown as a
smooth line on the I polarization plot, while the initial guess,
plotted in the V polarization, is plotted at the locations of the data
samples.  Then you are offered a menu of options, either in your
\AIPS\ terminal window ({\tt DOTV = 1}) or, as shown in the present
figures, on the TV ({\tt DOTV = 2}).  The first spectrum to be fit
(from an OH 1720 MHz maser source) is illustrated in
Figure~\ref{fig:ZEMAN.init} showing that the first guess has zero for
the two $B_i(y,z)$ but a reasonable guess at the $A(y,z)$.  Thus, the
green line mimics the shape of the I spectrum, not the V spectrum.
The menu that appears at this point is\\
\begin{center}
\begin{tabular}{|l|l|}\hline
   {\tt DO FIT}   & {\tt \hphantom{A}} \\
   {\tt BAD}      & {\tt B} \\
   {\tt QUIT}     & {\tt Q} \\ \hline
\end{tabular}
\end{center}
Note that there is no {\tt RE-GUESS} since there is no need to enter
them with linear least squares.  You select menu options in the same
way as {\tt XGAUS}, with ``button'' {\tt D} giving appropriate
real-time help information and buttons {\tt A}, {\tt B}, and {\tt C}
selecting the highlighted option.  The option to {\tt QUIT} (or {\tt
  Q} on the terminal) causes the task to quit at this point.  You may
re-start later.  The option {\tt BAD} ({\tt B} on the terminal) will
mark this position as failed and go on to the next position.  The
option {\tt DO FIT}, currently highlighted, will cause the task to
attempt the linear fit with the current initial guess.  The result of
that fit is shown in Figure~\ref{fig:ZEMAN.good}.  The second pixel
starts with the $B_i(y,z)$ of the last fit
(Figure~\ref{fig:ZEMAN.init2}) and so is a much better initial guess.

\begin{figure}
\begin{center}
\resizebox{5.5in}{!}{\putfig{ZEMAN.good}}
\caption{First OH spectrum to fit, plot after fitting.}
\label{fig:ZEMAN.good}
\end{center}
\end{figure}

\begin{figure}
\begin{center}
\resizebox{5.5in}{!}{\putfig{ZEMAN.init2}}
\caption{Second OH spectrum to fit, initial ``guess'' has Gaussians and
   values for $B_i(y,z)$.}
\label{fig:ZEMAN.init2}
\end{center}
\end{figure}

\vfill\eject
\subsection{Editing and output}

\begin{figure}
\begin{center}
\resizebox{6.0in}{!}{\putfig{ZEMAN.edit}}
\caption{{\tt ZEMAN} fit and image editing screen.}
\label{fig:ZEMAN.edit}
\end{center}
\end{figure}

Eventually all positions selected by {\tt FLUX} and {\tt BLC} and {\tt
  TRC} will have been fit.  At this point, the task computes images of
the fit parameters $A(y,z)$ and $B_i(y,z)$ and their uncertainties.
It then offers a menu of options which will allow you to view these
images and revisit positions that seem to have produced incorrect
fits.  This ``edit'' menu is illustrated in
Figure~\ref{fig:ZEMAN.edit}.  Note that the size of the OH \Hi{region
of interest} is quite small \Hi{as illustrated here.  If {\tt SET
  WINDOW} is used to select a small sub-image, then} {\tt ZEMAN} will
blow it up by pixel replication to a reasonable size \Hi{as in the
  following figure}.

There are two kinds of editing implemented here.  In the first, the
user establishes the parameter extrema which should be viewed as
acceptable.  The extrema currently set are shown in the title lines
and include the maximum rms, the range of allowed values for ``gain''
($A(y,z)$), the range of allowed values for ``field'' ($B(y,z)$ in
pixels), and the maximum uncertainty in the field.  Then, {\tt ZEMAN}
may be told to flag all solutions not meeting these criteria, or, more
profitably perhaps, to revisit those positions to see why a poor fit
was obtained.  Note that, unlike the other tasks in this memo, the
linear nature of the fit in {\tt ZEMAN} means that only one solution
is possible at each celestial coordinate.  The other editing method is
similar, but acts on a list of pixel positions.  These may be entered
by typing in values or by clicking on suspect pixel positions in the
{\tt CURVALUE} function described below.  The contents of the list may
be viewed, the solutions at the positions may be flagged, and they may
be revisited to see why they are suspect.  Note that there are no
swapping of solutions in this task; {\tt XGAUS} establishes which
component is which.  After the flagging or revisiting, the list is
cleared.  The first column of the menu includes:
\vfill\eject

\begin{center}
\begin{tabular}{|l|l|}\hline
 {\tt EXIT           } & Exit {\tt XGAUS}, writing output images if
                         {\tt DOOUTPUT} is now $> 0.$\\
 {\tt SET MAX RES    } & Set maximum residual for okay solutions\\
 {\tt SET GAIN RANGE } & Set gain value range(s) for okay solutions\\
 {\tt SET FIELD RANGE } & Set field range(s) for okay solutions\\
 {\tt SET MAX ERR FLD} & Set maximum error(s) in field for okay
                         solutions\\
 {\tt REDO ALL       } & Re-do all solutions which are not okay\\
 {\tt FLAG ALL       } & Mark bad all solutions which are not okay\\
 {\tt OFF ZOOM       } & Turn of TV zoom\\
 {\tt OFF TRANSFER   } & Turn off black \&\ white and color TV
                         enhancements\\
 {\tt SET DOOUTPUT   } & Increment {\tt DOOUTPUT} in loop 0-3 --- with
                         1 and 3 causing residual\\
                       & images and 2 and 3 causing parameter images
                         to be written on {\tt EXIT}\\
 {\tt ADD TO LIST    } & Type in output pixel coordinates to add to
                         list\\
 {\tt SHOW LIST      } & Display coordinates in list\\
 {\tt REDO LIST      } & Re-do solutions for all pixels in list\\
 {\tt FLAG LIST      } & Flag solutions for all pixels in list\\ \hline
\end{tabular}
\end{center}

The second menu column contains
\begin{center}
\begin{tabular}{|l|l|}\hline
 {\tt SHOW IMAGE G } & Enter image interaction with gain\\
 {\tt SHOW IMAGE EG } & Enter image interaction with uncertainty in
                     the gain\\
 {\tt SHOW IMAGE F1 } & Enter image interaction with field of
                     component 1\\
 {\tt SHOW IMAGE EF1} & Enter image interaction with uncertainty in
                     field of component 1\\
 {\tt SHOW IMAGE F2} & Enter image interaction with field of component
                     2\\
 {\tt SHOW IMAGE EF2} & Enter image interaction with uncertainty in
                     field of component 2\\ \hline
\end{tabular}
\end{center}
There is 1 or, if {\tt OPTYPE='GAUS'} as in the illustrated cases,
the maximum of {\tt NGAUSS} of the {\tt F}$n$ and {\tt EF}$n$ options.
When you select one of the {\tt SHOW} options, the options in yet
another menu appear along with a display of the selected image.  As
illustrated in Figure~\ref{fig:ZEMAN.field1}, these options are
\begin{center}
\begin{tabular}{|l|l|}\hline
 {\tt RETURN     } & Return to the above menus, image stays displayed\\
 {\tt LOAD AS SQ } & Re-load image with square root transfer function\\
 {\tt LOAD AS LG } & Re-load image with log transfer function\\
 {\tt LOAD AS L2 } & Re-load image with extreme log transfer function\\
 {\tt LOAD AS LN } & Re-load image with linear transfer function\\
\Hi{{\tt SET WINDOW}} & \Hi{Set a sub-image to view}\\
\Hi{{\tt RESET WINDOW}} & \Hi{Return too viewing the full image}\\
 {\tt OFF TRANSF } & Turn off enhancement done with {\tt TVTRANSF}\\
 {\tt OFF COLOR  } & Turn off color enhancements\\
 {\tt TVTRANSF   } & Black \&\ white image enhancement\\
 {\tt TVPSEUDO   } & Color enhancement of numerous sorts\\
 {\tt TVPHLAME   } & Color enhancement of flame type, multiple colors\\
 {\tt TVZOOM     } & Interactive zooming and centering of image\\
 {\tt CURVALUE   } & Display value under cursor, mark pixels for list\\
 {\tt NEXT WINDOW} & Move to next window into large images\\ \hline
\end{tabular}
\end{center}
Only one of the {\tt LOAD AS} options will appear, namely the next
after the current transfer function from the list of linear, square
root, log, and more extreme log transfer functions.  {\tt CURVALUE}
provides the capability of selecting positions for the edit ``list.''
During the {\tt CURVALUE} operation position the cursor over the
desired pixel and press buttons {\tt A}, {\tt B}, or {\tt C} to add
that pixel to the list.  The other options are familiar as {\tt AIPS}
verbs.  Instructions for interaction will appear on the terminal and
button {\tt D} in the menu may be used to obtain a helpful display on
the terminal.

\begin{figure}
\begin{center}
\resizebox{6.0in}{!}{\putfig{ZEMAN.field1}}
\caption{{\tt ZEMAN} image of $B_1(y,z)$ with display options.}
\label{fig:ZEMAN.field1}
\end{center}
\end{figure}

When you have finished getting the images just the way you want them,
you may write them out as \AIPS\ image files.  Select the {\tt SET
  DOOUTPUT} option until its value, shown at the top of the screen, is
what you want.  In {\tt ZEMAN}, values 4, 5, 6, and 7 cause a residual
(data-model) image cube to be written, while values 2, 3, 6, and 7
cause images of the gain and field values and their uncertainties to
be written and values 1, 3, 5, and 7 cause an image of the V cube to
be written with the gain times the I-polarization cube subtracted.
The corrected V-polarization image gets the class specified by {\tt
  OUTCLASS}, the residual image gets class {\tt VRESID}, the gain and
its uncertainty get classes {\tt GAIN} and {\tt DGAIN}, and the
field(s) and their uncertainties get classes {\tt FIELD}$n$ and {\tt
  DFELD}$n$, with $n = 1$ to the maximum {\tt NGAUSS}\@.
%\vfill\eject

\section{Polarization fitting}

Faraday rotation occurs when a polarized source of radiation
$F_0(\Phi_1)$ has its radiation pass through a region of electrons (a
``Faraday screen'').  The complex polarization ($Q + iU$) is rotated
in position angle by an amount proportional to the wavelength squared,
\ie\ it is multiplied by $\exp (i 2 \phi_1 \lambda^2)$.  If this
radiation is then joined by another source of radiation $F_1(\Phi_2)$
and the two then pass through additional screens $\phi_2$, $\phi_3$,
and $\phi_4$ after which there is a third source of radiation and one
final screen, the net polarization observed  is then
\begin{eqnarray}
P(\lambda^2) & = & F_0(\Phi_1)\, e^{i2\Phi_1\lambda^2} + F_1(\Phi_2)\,
                   e^{i2\Phi_2\lambda^2} + F_2(\Phi_3)\,
                   e^{i2\Phi_3\lambda^2},\\
\noalign{\mbox{where}}
\Phi_1 & = & \phi_1 + \phi_2 + \phi_3 + \phi_4 + \phi_5 \nonumber\\
\Phi_2 & = & \phi_2 + \phi_3 + \phi_4 + \phi_5 \nonumber\\
\Phi_3 & = & \phi_5\nonumber\\
\noalign{\mbox{or, in the limit:}}
P(\lambda^2) & = & \int F(\Phi)\, e^{i2\Phi\lambda^2} d\Phi \label{eqn:FARS}
\end{eqnarray}

``Faraday Rotation Measure Synthesis'' is then the Fourier analysis of
Equation~\ref{eqn:FARS} as initially described by Brentjens and de
Bruyn.\footnote{Brentjens, M. A. and de Bruyn, A. G. 2005, {\it
    Astron.\ \& Ap.}, 1217.}  \AIPS\ task {\tt FARS} takes as input
two matching spectral cubes, one in Q and one in U polarization.  It
does a suitable Fourier transform in wavelength squared and can do a
one-dimensional complex Clean and restoration at each spatial
coordinate.  A sample of a two-component model is shown in
Figure~\ref{fig:FARS model}.  Although the model contains data from
frequencies from 1 GHz to 3 GHz, the figure shows that it is not
possible with this analysis to separate the two components even though
they are at a physically realistic and interesting separation.

\begin{figure}
\begin{center}
\resizebox{6.0in}{!}{\putfig{FARS}}
\caption{{\tt FARS} output with model data.  Source has components at
100 and 140 radians per $m^2$.  Blue is without Clean, green is Clean
restored with very narrow components, and pink is Clean with a normal
restoration.  Note that Clean finds components at 120 plus about 68
and 172 radians per $m^2$.}
\label{fig:FARS model}
\end{center}
\end{figure}

Therefore a new fitting task, called {\tt RMFIT}, was written.  Its
jobs is to do a non-linear fit of
\begin{eqnarray}
  Q(i) & = & \sum  P_j \cos (2\theta_j + 2 R_j \lambda(i)^2)
              F(\beta_j,, \lambda(i)^2) \\
  U(i) & = & \sum  P_j \sin (2\theta_j + 2 R_j \lambda(i)^2)
              F(\beta_j,, \lambda(i)^2)
\end{eqnarray}
where $P_j$ is the polarized flux of component $j$, $\theta_j$ is the
polarization angle of component j, $R_j$ is the rotation measure of
component j, $\lambda(i)$ is the wavelength of spectral channel $i$,
$\beta_j$ is a ``thickness'' parameter and $F$ is one of
\begin{eqnarray*}
F(\beta_j,\lambda(i)^2) & = & \left(
    \frac{\lambda_1^2}{\lambda(i)^2}\right) ^ {\frac{\beta_j}{2}} \\
  & = & \frac{\sin (\beta_j \lambda(i)^2)}{\beta_j \lambda(i)^2} \\
  & = & e ^ {-\ln(2)\, \left(\,\frac{\beta_j \lambda(i)^2}{1.8954}\,\right)^2} \\
  & = & e ^ {-\ln(2)\, \frac{\beta_j \lambda(i)^2}{1.8954}}
\end{eqnarray*}
where $\lambda_1$ is the wavelength at the fiducial frequency of 1 GHz
and the thickness models will be referred to by the names spectral
index, slab, Gaussian, and exponential, respectively.  The peculiar
number $1.8954$ was chosen so that the last three models reach the 0.5
point at the same value of $\beta_j$.  Rotation measure thickness, in
general, makes the polarized signal decrease at longer wavelengths.
It is one of the reasons that radio sources at low frequency tend to
have little or no polarized flux density.

\subsection{Inputs}

{\tt RMFIT} expects five input cubes which contain, in order, the Q
polarization, the U polarization, the I polarization, the FARS output
real or amplitude, and the FARS output imaginary or phase images.  All
five must align appropriately, the first three must be transposed so
that frequency (or frequency ID number) is the first axis (as they go
into {\tt FARS}), and the last two have the rotation measure axis
first (as they come out of {\tt FARS})\@.  The {\tt FARS} images are
required to allow the user to make initial guesses for the model.  Do
some Cleaning in {\tt FARS} and restore with a ``Clean beam''
reasonably close to the default.  The I polarization image is
optional and may be a continuum image rather than a spectral cube.

The rest of the inputs to {\tt RMFIT} are much like those of {\tt
  XGAUS}\@.  {\tt INVERS} defines the version of the {\tt RM} table to
be used to hold the model fit, with zero causing a new table to be
used.  {\tt BLC} and {\tt TRC} define the pixel ranges to be used in
the current execution, where {\tt BLC(1)} and {\tt TRC(1)} control the
spectral channels that will be fit and {\tt BLC(2)}, {\tt TRC(2)},
{\tt BLC(3)}, and {\tt TRC(3)} control the area in celestial
coordinates over which the fitting is done.  {\tt BLC(1)} and {\tt
TRC(1)} do not apply to the {\tt FARS} images\Hi{, but they do control
which channels are used to find the peak polarization and unpolarized
brightnesses for new {\tt RM} tables.  The {\tt RM} table now contains
all pixels on axes 2 and 3 of the input images, so there should never
be a problem resuming fitting with a new region of interest.}  Adverbs
{\tt YINC} and {\tt ZINC} control the stride taken in the first pass
through the cube; a second pass will then fit all voxels not fit in
the first pass.  Adverbs {\tt PCUT} and {\tt ICUT} control which
voxels are fit.  Any spectra having average total polarization
($\sqrt{Q^2 + U^2}$) greater than {\tt PCUT} and, if an I image is
given, total intensity greater than {\tt ICUT} will be fit.

 {\tt DOOUTPUT} controls what files are written --- this may be
 changed interactively so leave it zero at this point.  Set {\tt DOTV
   = 2} to use TV menus to prompt you.  The badly named adverb {\tt
   NGAUSS} controls the maximum number of components to fit and is
 limited to 4.  Adverb {\tt DOSPIX} the choice of thickness model to
 be fit, with 1 for spectral index, 2 for slab, 3 for Gaussian, 4 for
 exponential, and anything else for no thickness.  Polarization cubes
 may have quite different noise levels in different spectral channels.
 Adverb {\tt DOWEIGHT} controls whether a weighted fit is done.  The
 weights may be found by robust rms methods from the Q and U images or
 read from a text file specified by adverb {\tt INFILE}\@.  {\tt
   RMSLIMIT} is an upper limit for the rms of a fit before the fit is
 viewed as ``failed'' which causes the TV and interaction to be turned
 back on after you have turned it off.  You should get a good idea of
 an appropriate value from your initial uses of {\tt RMFIT} or from
 your knowledge of the noise in your data cube.

A recommended strategy, when creating an {\tt RM} table, would be to
set the \Hi{the spectral region {\tt BLC(1)} through {\tt TRC(1)} to
encompass all reliable spectral channels so that the full range of
polarized and unpolarized brightnesses will be found.}  The task
begins by creating this table\Hi{, including the entire input
image,} and populating each row with the average unpolarized and
polarized brightnesses in each spectrum.  Then it reads the table
every {\tt YINC} rows and {\tt ZINC} planes and, for those with
brightnesses exceeding {\tt PCUT} and {\tt ICUT}, attempts a fit.
Your interaction with this fit will be described below.  After the
first pass, the task loops over every row and plane fitting those
positions which have enough brightness and which have not already been
fit.  Finally, after all pixels above the cutoffs have been fit, the
task goes into an ``edit'' mode.  It constructs images of each fit
parameter and Q and U at zero wavelength and of the uncertainties in
these parameters.  You may view these images, select positions
explicitly or by their parameter values or rms and revisit the fits of
those positions.  This stage will be described in detail below.

At any time you may exit the task and then re-start it using the
same {\tt RM} table.  Good reasons for doing this include fitting
smaller regions with each pass using the appropriate number of
components for that region.  Doing small regions which will have
similar parameter values helps a great deal with the initial guessing
done by the task (mostly using the previous solution).  You might also
fit the cube initially with a high value of {\tt PCUT} and then
re-start with a lower value to extend the areas fit.

\begin{figure}
\begin{center}
\resizebox{6.0in}{!}{\putfig{RMFIT.init3}}
\caption{Model field first pixel {\tt FARS} rotation measure spectrum
  to fit.}
\label{fig:RMFIT.init}
\end{center}
\end{figure}

\subsection{Fitting}

The fitting process starts with a plot of the {\tt FARS} amplitude and
phase as a function of rotation measure at the first coordinate with
adequate signal.  The data and axis labels are plotted in graphics
channel one (usually yellow) and the initial guess as X's in graphics
channel two (usually green).  Then you are offered a menu of options,
either in your \AIPS\ terminal window ({\tt DOTV = 1}) or, as shown in
the present figures, on the TV ({\tt DOTV = 2}).  The first spectrum
to be fit is illustrated in Figure~\ref{fig:RMFIT.init} showing
that the first guess for more than one component is not useful.  The
menu that appears at this point is\\

\begin{center}
\begin{tabular}{|l|l|}\hline
   {\tt DO FIT}   & {\tt \hphantom{A}} \\
   {\tt RE-GUESS} & {\tt E} \\
   {\tt BAD}      & {\tt B} \\
   {\tt QUIT}     & {\tt Q} \\ \hline
\end{tabular}
\end{center}
You select a menu option by moving the cursor to the desired option
with the mouse and registering that move with the TV by clicking the
left mouse button.  The selected menu item will change color as shown
in the figure.  If you press TV ``button'' {\tt D} at this point
(actually keyboard character D), helpful information about the
selected item will appear on your terminal window.  If you press one
of TV ``buttons'' {\tt A}, {\tt B}, or {\tt C} (actually keyboard
characters A, B, C), the selected function will be performed.  The
option to {\tt QUIT} (or {\tt Q} on the terminal) causes the task to
quit at this point.  You may re-start later.  The option {\tt BAD}
({\tt B} on the terminal) will mark this position as failed and go on
to the next position.  The option {\tt DO FIT} will cause the task to
attempt the non-linear fit with the current initial guess.  The
selected option in Figure~\ref{fig:RMFIT.init} is {\tt RE-GUESS} which
causes the task to prompt you first to {\it ``Position cursor at
  center (RM) of component 1''}.  Move the cursor to the peak of
component 1 and press any TV button.  This selects the rotation
measure value of component 1.  This prompt is repeated for components
two through {\tt NGAUSS}\@.  To omit the fit for a particular
component at this position, move the cursor off the plot before
pressing the TV button for that component.

After a fit has been attempted, the U spectrum is plotted in the top
half of the screen and the Q spectrum is plotted in the lower half.
The data and labels are in graphics channel 1 (usually yellow), the
initial guess is plotted in channel 2 (usually green), and the fit
model is plotted in channel 4 (usually cyan).  The fit values and
rmses are displayed on your terminal.  The user chose to fit only two
components initially, getting the unhappy result shown in
Figure~\ref{fig:RMFIT.2bad}.

\begin{figure}
\begin{center}
\resizebox{6.0in}{!}{\putfig{RMFIT.2bad}}
\caption{Model field first pixel Q and U spectra, fit with only two
  components (there are actually three).}
\label{fig:RMFIT.2bad}
\end{center}
\end{figure}

The menu that appears at this point offers the option to {\tt QUIT}
(or {\tt Q} on the terminal) causes the task to quit at this point.
You may re-start later.  The option {\tt BAD} ({\tt B} on the
terminal) will mark this position as failed and go on to the next
position.   The option {\tt RE-GUESS} ({\tt E} or {\tt R} on the
terminal) will loop back to prompt you for a new guess and repeat the
fit.  Options {\tt 1}, {\tt 2}, $\ldots$, {\tt NGAUSS} will loop back
to plot an initial guess with the selected number of Gaussians.
Option {\tt HAND} ({\tt H} on the terminal) will prompt you to enter
using the terminal the parameters for each component.  Enter on one
line for each component, the polarization brightness (in image units),
the polarization angle (in degrees), the rotation measure and, if you
are fitting a thickness, the thickness width (in radians per meter
squared).  {\tt RMFIT} will then repeat the display in
Figure~\ref{fig:RMFIT.2bad} to see if you made a good guess.
Immediately after a {\tt HAND} operation only, the option {\tt DO FIT}
is offered to go back with the hand-entered values as the initial
guess for a new fit.  Option {\tt GOOD} (and other initial character
on the terminal) tells the task that you are (reasonably) happy and
that it should go on to the next position.  The menu includes

\begin{center}
\begin{tabular}{|l|l|}\hline
   {\tt GOOD}     & {\tt \hphantom{A}} \\
   {\tt DO FIT}   & {\tt D} \\
   {\tt RE-GUESS} & {\tt E} or {\tt R} \\
   {\tt TVOFF}    & {\tt T} \\
   {\tt HAND}     & {\tt H} \\
   {\tt BAD}      & {\tt B} \\
   {\tt PIXRANGE} & {\tt P} \\
   {\tt 1}        & {\tt 1} \\
   {\tt 2}        & {\tt 2} \\
   {\tt 3}        & {\tt 3} \\
   {\tt QUIT}     & {\tt Q} \\ \hline
\end{tabular}
\end{center}

In the present case, the user has wisely decided to enter a new guess,
this time specifying all three components.  The happy result is shown
in Figure~\ref{fig:RMFIT.3good}.  Option {\tt TVOFF} allows you to
turn off interactivity, allowing the task to run using its own initial
guesses until it finds a completely unreasonable solution or one with
an rms greater than {\tt RMSLIMIT}\@.  When that happens, you are
shown the offending fit parameters and the task resumes with the plot
of Figure~\ref{fig:RMFIT.tvrestart}, already prompting you to enter a
new guess, to allow you to try to fix things.

\begin{figure}
\begin{center}
\resizebox{5.0in}{!}{\putfig{RMFIT.3good}}
\caption{Model field first pixel Q and U spectra, fit with three
  components.}
\label{fig:RMFIT.3good}
\end{center}
\end{figure}

\begin{figure}
\begin{center}
\resizebox{5.0in}{!}{\putfig{RMFIT.tvrestart}}
\caption{Model field {\tt FARS} rotation measure spectrum at
  coordinate for which the fit fails.}
\label{fig:RMFIT.tvrestart}
\end{center}
\end{figure}

\vfill\eject
\subsection{Editing and output}

Eventually all positions selected by {\tt PCUT}, {\tt ICUT}, {\tt
  BLC}, and {\tt TRC} will have been fit.  At this point, the task
computes images of the fit parameters plus the Q and U polarization at
zero wavelength and their uncertainties.  It then offers a lengthy
menu of options which will allow you to view these images and revisit
positions that seem to have produced incorrect fits.  If {\tt NGAUSS}
$> 1$, options to swap portions of image $n$ with corresponding
portions of image $m$ are also offered.  This ``edit'' menu is
illustrated in Figure~\ref{fig:RMFIT.edit2} for the {\tt NGAUSS = 2}
and {\tt DOSPIX = 1} case.  {\tt RMFIT} will replicate pixels in both
directions if needed to make small images large enough to see, but it
can also handle images larger than the TV display area.
Where,the {\tt SET SPINX RANGE} appears only when {\tt DOSPIX = 1} and
the {\tt SET THICKNESS MAX} appears only when {\tt DOSPIX =} 2, 3, or
4, the first column of the menu shows the editing functions.

\begin{figure}
\begin{center}
\resizebox{6.0in}{!}{\putfig{RMFIT.edit2}}
\caption{Model field, showing rotation measure for component one, for
which there are pixels needing correction.}
\label{fig:RMFIT.edit2}
\end{center}
\end{figure}

There are three kinds of editing implemented here.  In the first, the
user establishes the parameter extrema which should be viewed as
acceptable.  The extrema currently set are shown in the title lines.
Then, {\tt RMFIT} may be told to flag all solutions not meeting these
criteria, or, more profitably, to revisit those positions to see if a
better fit can be obtained.  The other editing methods are similar,
but act on a list of pixel positions.  These may be entered by typing
in values or by clicking on suspect pixel positions in the {\tt
  CURVALUE} function described below.  The contents of the list may be
viewed, the solutions at the positions may be flagged, or they may be
revisited to attempt for a better solution, or the solutions at the
listed positions may be swapped between components $n$ and $m$.  The
menu will offer only appropriate swaps between components, thus 1 and
2 in our {\tt NGAUSS=2} example.  Swapping may be required if {\tt
  RMFIT} gets confused as to which component you want to call number 1
and which number 2, as illustrated in Figure~\ref{fig:RMFIT.edit2}.
After the flagging, revisiting, or swapping, the list is cleared.
\vfill\eject

\begin{center}
\begin{tabular}{|l|l|}\hline
 {\tt EXIT           } & Exit {\tt XGAUS}, writing output images if
                         {\tt DOOUTPUT} is now $> 0.$\\
 {\tt SET MIN S/N    } & Set minimum amplitude S/N(s) for okay
                         solutions\\
 {\tt SET MIN P1     } & Set minimum polarization at 1 GHz for okay
                         solutions\\
 {\tt SET MAX RESID  } & Set maximum residual for okay solutions\\
 {\tt SET RM RANGE   } & Set rotation measure value range(s) for okay
                         solutions\\
 {\tt SET MAX THETA ER} & Set maximum error(s) in polarization angle
                         for okay solutions\\
 {\tt SET SPINX RANGE} & Set spectral index range(s) for okay
                         solutions\\
 {\tt SET THICKNESS MAX} & Set thickness maximum value(s) for okay
                         solutions\\
 {\tt REDO ALL       } & Re-do all solutions which are not okay\\
 {\tt FLAG ALL       } & Mark bad all solutions which are not okay\\
 {\tt OFF ZOOM       } & Turn of TV zoom\\
 {\tt OFF TRANSFER   } & Turn off black \&\ white and color TV
                         enhancements\\
 {\tt SET DOOUTPUT   } & Increment {\tt DOOUTPUT} in loop 0-3 --- with
                         1 and 3 causing residual\\
                       & images and 2 and 3 causing parameter images
                         to be written on {\tt EXIT}\\
 {\tt ADD TO LIST    } & Type in output pixel coordinates to add to
                         list\\
 {\tt SHOW LIST      } & Display coordinates in list\\
 {\tt REDO LIST      } & Re-do solutions for all pixels in list\\
 {\tt FLAG LIST      } & Flag solutions for all pixels in list\\
 {\tt SWAP LIST 1-2  } & Swap solutions for components 1 and 2 for all
                         pixels in list\\ \hline
\end{tabular}
\end{center}

The second (and potentially third) menu columns contain {\tt NGAUSS}
sets of functions
\vfill\eject

\begin{center}
\begin{tabular}{|l|l|}\hline
 {\tt SHOW IMAGE P1\_1} & Enter image interaction with total
                     polarization at 1 GHz of component 1\\
 {\tt SHOW IMAGE TH\_1} & Enter image interaction with polarization
                     position angle of component 1\\
 {\tt SHOW IMAGE RM\_1} & Enter image interaction with rotation
                     measure of component 1\\
 {\tt SHOW IMAGE BE\_1} & Enter image interaction with RM thickness
                     ($\beta$) of component 1\\
 {\tt SHOW IMAGE SP\_1} & Enter image interaction with spectral index
                     of component 1\\
 {\tt SHOW IMAGE Q0\_1} & Enter image interaction with Q at 0
                     wavelength of component 1\\
 {\tt SHOW IMAGE U0\_1} & Enter image interaction with U at 0
                     wavelength of component 1\\
 {\tt SHOW IMAGE EP1\_1} & Enter image interaction with uncertainty in
                     total polarization at 1 GHz\\
                         & of component 1\\
 {\tt SHOW IMAGE ETH\_1} & Enter image interaction with uncertainty in
                     polarization position angle\\
                         & of component 1\\
 {\tt SHOW IMAGE ERM\_1} & Enter image interaction with uncertainty in
                     rotation measure of component 1\\
 {\tt SHOW IMAGE UBE\_1} & Enter image interaction with uncertainty in
                     RM thickness ($\beta$) of component 1\\
 {\tt SHOW IMAGE ESP\_1} & Enter image interaction with uncertainty in
                     spectral index of component 1\\
 {\tt SHOW IMAGE EQ0\_1} & Enter image interaction with uncertainty in
                     Q at 0 wavelength of component 1\\
 {\tt SHOW IMAGE EU0\_1} & Enter image interaction with uncertainty in
                     U at 0 wavelength of component 1\\ \hline
\end{tabular}
\end{center}
The {\tt SP} options will appear only if {\tt DOSPIX = 1} was specified
and the {\tt BE} options will appear only if {\tt DOSPIX =} 2, 3, or 4
was specified.

\begin{figure}
\begin{center}
\resizebox{6.0in}{!}{\putfig{RMFIT.badpix}}
\caption{Model field rotation measure of component 2 showing pixels in
the upper ``sources'' which were solved incorrectly.}
\label{fig:RMFIT.badpix}
\end{center}
\end{figure}

When you select one of these functions most of the following
operations will appear in yet another menu.  This menu is illustrated
in Figure~\ref{fig:RMFIT.badpix}.  Only one of the {\tt LOAD AS}
options will appear, with the next one in the sequence offered when
the current one has been invoked.  The {\tt SWAP $n$-$m$} options will
appear as needed when {\tt NGAUSS} $ > 1$.  The {\tt NEXT WINDOW}
option appears when needed to display an image too large to fit on the
display screen.  These options mostly invoke familiar functions from
\AIPS\ to control the {\tt FUNCTYPE} used in loading the image to the
display, to enhance the image intensities, to color the enhanced image
intensities, and to zoom the display.

\begin{center}
\begin{tabular}{|l|l|}\hline
 {\tt RETURN     } & Return to the above menus, image stays displayed\\
 {\tt LOAD AS SQ } & Re-load image with square root transfer function\\
 {\tt LOAD AS LG } & Re-load image with log transfer function\\
 {\tt LOAD AS L2 } & Re-load image with extreme log transfer function\\
 {\tt LOAD AS LN } & Re-load image with linear transfer function\\
\Hi{{\tt SET WINDOW}} & \Hi{Set a sub-image to view}\\
\Hi{{\tt RESET WINDOW}} & \Hi{Return too viewing the full image}\\
 {\tt OFF TRANSF } & Turn off enhancement done with {\tt TVTRANSF}\\
 {\tt OFF COLOR  } & Turn off color enhancements\\
 {\tt TVTRANSF   } & Black \&\ white image enhancement\\
 {\tt TVPSEUDO   } & Color enhancement of numerous sorts\\
 {\tt TVPHLAME   } & Color enhancement of flame type, multiple colors\\
 {\tt TVZOOM     } & Interactive zooming and centering of image\\
 {\tt CURVALUE   } & Display value under cursor, mark pixels for list\\
 {\tt SWAP 1-2   } & Swap solutions for components 1 and 2 interactively\\
 {\tt NEXT WINDOW} & Move to next window into large images\\ \hline
\end{tabular}
\end{center}

Two operations in this menu are different from the usual.  {\tt
  CURVALUE} provides the capability of selecting positions for the
edit ``list.''  During the {\tt CURVALUE} operation position the
cursor over the desired pixel and press buttons {\tt A}, {\tt B}, or
{\tt C} to add that pixel to the list.  The {\tt SWAP $n$-$m$}
operation uses a TV blotch operation like that in the \AIPS\ verb {\tt
  TVSTAT} and task {\tt BLSUM}\@.  You are to mark with a ``blotch''
region those pixels in the present image which are to have their
solutions swapped with those of the selected component.  Instructions
will appear in the message window as you proceed.  Begin by
positioning the cursor at a pixel to be the first vertex of a
connected sequence of vertices and press TV button {\tt A}\@.  Move to
the next vertex and press button {\tt A} again and repeat until you
have marked all vertices for this region.  Then press button {\tt D}
is you are done with this region or button {\tt C} is you need to
re-position one of the vertices.  In this case, move the cursor to the
vertex to be re-positioned, press button {\tt A} and drag the vertex to
the corrected position.  Press button {\tt A} or {\tt B} to fix that
vertex and go on to reset another vertex or {\tt D} to end this region
and swap the solutions.  You may do this as many times as needed.

\begin{figure}
\begin{center}
\resizebox{6.0in}{!}{\putfig{RMFIT.3cedit}}
\caption{Model field Q polarization at 0 wavelength of component 2
  showing edit menu for {\tt NGAUS = 3} and {\tt DOSPIX = 0}.}
\label{fig:RMFIT.3cedit}
\end{center}
\end{figure}

The images displayed were edited, first by selecting pixels in the
image displayed in Figure~\ref{fig:RMFIT.badpix} with {\tt CURVALUE}
and then re-fitting the spectra with {\tt REDO LIST}\@.  Additional
editing was needed for poor solutions at the upper right of the
central ``source.''  {\tt SET RM RANGE} and {\tt REDO ALL} were used.
A more crowded edit menu is then shown for three components with
{\tt DOSPIX = 0} in Figure~\ref{fig:RMFIT.3cedit}.

When you have finished getting the images just the way you want them,
you may write them out as \AIPS\ image files.  Select the {\tt SET
  DOOUTPUT} option until its value, shown at the top of the screen, is
what you want.  In {\tt RMFIT}, values 1 and 3 cause residual image
cubes to be written, while values 2 and 3 cause images of the
parameter values and their uncertainties to be written.  The Q and U
residual images get classes {\tt QRESID} and {\tt URESID},
respectively.  The polarization at 1 GHz, polarization angle, rotation
measure, and spectral index or thickness (if any) images and their
uncertainties get classes {\tt PPOL$n$}, {\tt THETA$n$}, {\tt
  ROTME$n$}, {\tt SPIX$n$}, {\tt THICK$n$}, {\tt DPPOL$n$}, {\tt
  DTHET$n$}, {\tt DROTM$n$}, {\tt DSPIX$n$}, and {\tt DTHIC$n$},
respectively, and Q and U at zero wavelength get classes {\tt
  Q0\_$n$}, {\tt U0\_$n$}, {\tt DQ0\_$n$}, and  {\tt DU0\_$n$}, where
$n$ is the component number.
\vfill\eject

\hicol
\section{Post-fit plotting}

The images produced by {\tt XGAUS}, {\tt ZEMAN} and {\tt RMFIT} may be
displayed using all the usual tools such as {\tt KNTR}, {\tt PROFL},
and numerous other tasks.  However, the display of the spectral data
and the various fits to them required new tasks.

\begin{figure}
\begin{center}
\resizebox{6.0in}{!}{\putfig{XG2PL}}
\caption{\Hi{Example spectrum from a circular area for a Gaussian and
  Zeeman fit.}}
\label{fig:XG2PL}
\end{center}
\end{figure}

{\tt XG2PL} plots a spectrum for a single pixel or for a rectangular
or circular region about a single pixel.  For each pixel included in
the average, the task reads the I polarization image to obtain the
data and the appropriate line on the {\tt XG} table to obtain the {\tt
  XGAUS} solution for that pixel.  It then computes the spectrum of
each component in the model, plus the sum of the components, and the
residual (data-model).  Each of these then enter into the average of
that parameter.  Finally, the task plots a user-selected number of the
parameters and, optionally, prints all of them to a text file.  The
plot may appear on the TV or be placed in a standard plot file
attached to the I polarization image.

Optionally, {\tt XG2PL} will also add the spectrum of the V
polarization data and the results of the fitting done by {\tt ZEMAN}
at the same pixels as the I polarization Gaussians.  The task reads
the V polarization image for the data spectrum and the{\tt ZE} table
for the Zeeman-splitting model (either using the Gaussians or the
simpler ones using the I polarization data).  It computes the spectrum
of each component (including the gain term in each), the net model
(sum of the components but including the gain term only once), and the
residual.  Finally, the task plots a user-selected number of the
parameters and, optionally, prints all of them to a text file.  In
general, the I polarization spectrum appears in the upper part of the
plot and the V polarization spectrum appears in the lower portion.
Either portion may be omitted under control of the adverbs.  The
output of {\tt XG2PL} is illustrated in Figure~\ref{fig:XG2PL}.

The inputs for {\tt XG2PL} begin with the I polarization image which
is required and then the V polarization image which may be omitted.
{\tt INVERS} and {\tt IN2VERS} give the version numbers of the {\tt
  XG} and {\tt ZE} tables, respectively.  {\tt APARM} provides the
central pixel coordinates, the plot intensity ranges for I and for V,
the size of the rectangle (or circle) over which to average, and a
flag limiting the average to those pixels having a model fit,  {\tt
  BPARM} is a set of flags selecting which parameters are plotted.
{\tt CPARM} selects the channel range to be plotted, the relative size
of the V and I plots, the type of the horizontal axis (channels,
frequency, velocity), whether the channels are plotted in reverse
order, and whether the data are plotted with stepped or directly
connected lines.  {\tt OUTTEXT} specifies the output text file, if
any.  The usual {\tt XYRATIO}, {\tt LTYPE}, {\tt DOTV}, and {\tt
  GRCHAN} adverbs control the scale, labeling, choice of TV versus
plot file, and, if TV, which graphic channel(s) are used.  If {\tt
  GRCHAN} is zero, graphics channel 1 is used for data and labeling
(usually yellow), channel 2 is used for the full model (usually
green), channel is used for the residual (usually pink), and channel 4
is used for the model component(s) (usually cyan).

\begin{figure}
\begin{center}
\resizebox{6.0in}{!}{\putfig{RM2PL}}
\caption{\Hi{Example spectrum from a circular area for a 2-component
  rotation measure fit.}}
\label{fig:RM2PL}
\end{center}
\end{figure}

{\tt RM2PL} plots a spectrum for a single pixel or for a rectangular
or circular region about a single pixel.  For each pixel included in
the average, the task reads the Q and V polarization images to obtain
the data and the appropriate line on the {\tt RM} table to obtain the
{\tt RMFIT} solution for that pixel.  It then computes the spectrum of
each component in the model, plus the sum of the components, and the
residual (data-model).  Each of these then enter into the average of
that parameter.  Finally, the task plots a user-selected number of the
parameters and, optionally, prints all of them to a text file.  The
plot may appear on the TV or be placed in a standard plot file
attached to the Q polarization image.  The user may choose to omit all
of the Q or all of the U polarization plot.  The output of {\tt RM2PL}
is illustrated in Figure~\ref{fig:RM2PL}.

The inputs for {\tt RM2PL} begin with the Q polarization image which
is required and then the U polarization image which may be omitted.
{\tt INVERS} gives the version number of the {\tt RM} table.  {\tt
  APARM} provides the central pixel coordinates, the plot intensity
ranges for Q and for U, the size of the rectangle (or circle) over
which to average, and a flag limiting the average to those pixels
having a model fit.  {\tt BPARM} is a set of flags selecting which
parameters are plotted.  {\tt CPARM} selects the channel range to be
plotted, the relative size of the U and Q plots, the type of the
horizontal axis (channels, frequency, wavelength, wavelength squared),
whether the channels are plotted in reverse order, and whether the
data are plotted with stepped or directly connected lines.  {\tt
  OUTTEXT} specifies the output text file, if any.  The usual {\tt
  XYRATIO}, {\tt LTYPE}, {\tt DOTV}, and {\tt   GRCHAN} adverbs
control the scale, labeling, choice of TV versus plot file, and, if
TV, which graphic channel(s) are used.  If {\tt GRCHAN} is zero,
graphics channel 1 is used for data and labeling (usually yellow),
channel 2 is used for the full model (usually green), channel is used
for the residual (usually pink), and channel 4 is used for the model
component(s) (usually cyan).

\hblack
\section{Model creation}

\subsection{Visibility data models}

There are two tasks which will modify an existing visibility (``$uv$'')
data set, adding models and noise to the existing data or replacing
those data.  {\tt UVMOD} can apply a full set of calibration and data
selection adverbs to the input $uv$ data set.  Then the data are
scaled by {\tt FACTOR}, which can be zero to eliminate the input data
entirely.  Finally, a model is added.  The model can be a simple set
of up to 4 {\tt NGAUSS} components specified by adverbs {\tt CTYPE},
{\tt FMAX}, {\tt FPOS}, and {\tt FWIDTH}\@.  Alternatively, adverb
{\tt INLIST} may specify a text file containing up to 9999 components.
Each non-comment line in this file specifies
\begin{center}
\begin{tabular}{|r|l|}\hline
 1. & I-polarization flux (Jy at \Hi{header frequency})\\
 2. & X-shift from reference (arc seconds)\\
 3. & Y-shift from reference (arc seconds)\\
 4. & Major axis (arc seconds)\\
 5. & Minor axis (arc seconds)\\
 6. & Position angle (degrees CCW from North)\\
 7. & Type code: 1 Gaussian, 2 solid disk, 3 solid rectangle,\\
    & 4 optically thin sphere, 5 exponential, else point\\
 8. & Spectral index (see note below)\\
 9. & Spectral index curvature\\
10. & Q-polarization flux (Jy at \Hi{header frequency})\\
11. & U-polarization flux (Jy at \Hi{header frequency})\\
12. & V-polarization flux (Jy at \Hi{header frequency}) \\ \hline
\end{tabular}
\end{center}
where trailing zeros may be omitted.  If spectral index is used, note
the definition of spectral index and flux.  If $x$ is the logarithm,
base 10, of the frequency \Hi{divided by the header frequency}, then
$$ F = F_0 10^{(\alpha x + \beta x^2)} $$
where $\alpha$ is the spectral index and $\beta$ is its curvature.
Thus the flux you must specify in the model is \Hi{the flux at the
frequency of the data}.

{\tt SPMOD} is a very similar task designed to do spectral-line
modeling rather than spectral index and polarization.  In this case
{\tt INLIST} is required and each of the up to 9999 non-comment lines
contains
\begin{center}
\begin{tabular}{|r|l|}\hline
 1. & RR-polarization flux (Jy)\\
 2. & X-shift from reference (arc seconds)\\
 3. & Y-shift from reference (arc seconds)\\
 4. & Major axis (arc seconds)\\
 5. & Minor axis (arc seconds)\\
 6. & Position angle (degrees CCW from North)\\
 7. & Type code: 1 Gaussian, 2 solid disk, 3 solid rectangle,\\
    & 4 optically thin sphere, 5 exponential, else point\\
 8. & Spectral line center (channels, but see note below)\\
 9. & Spectral line width (FWHM channels)\\
10. & LL-polarization flux (Jy)\\ \hline
\end{tabular}
\end{center}
where trailing zeros may be omitted.  The LL flux will be set equal to
the RR flux if it is omitted.  But note that both the RR and LL flux
may be specified to have any value positive, zero, or negative.  This
will allow Zeeman-splitting to be modeled with any degree of
splitting.  Spectral channels are counted from {\tt BCHAN} through
{\tt ECHAN} as $1$ to $N$ in the first spectral window ({\tt BIF}),
then $N+1$ through $2 N$ in the next spectral window, and so forth.
This allows you to put lines in any spectral window you desire.

\subsection{Image data models}

There are three tasks designed to modify an image adding a specified
model and noise.  {\tt IMMOD} is intended for continuum images.  The
existing image may be scaled (or eliminated) and noise added.  It adds
{\tt NGAUSS} components up to four using adverbs {\tt OPCODE} to
specify type, and {\tt FMAX}, {\tt FPOS}, and {\tt FWIDTH} to specify
peak brightness and pixel position and size.  Alternatively, adverb
{\tt INLIST} may be used to specify up to 9999 components, one per
line in the text file. Each non-comment line specifies
\begin{center}
\begin{tabular}{|r|l|}\hline
 1. & Peak brightness (Jy/beam)\\
 2. & Component $X$ center (pixels)\\
 3. & Component $Y$ center (pixels)\\
 4. & Major axis (pixels)\\
 5. & Minor axis (pixels)\\
 6. & Position angle (degrees CCW from North)\\ \hline
\end{tabular}
\end{center}

The more recent task {\tt MODIM} is specifically designed for
polarization modeling and has been used extensively in testing {\tt
  RMFIT}\@.  It takes in three image cubes, one for I polarization,
one for Q, and one for U.  Alternatively, it can create new images
with no input images used.  Up to 9999 components are taken from the
text file specified by {\tt INLIST}\@.  Each non-comment line in this
file specifies {\it every one of}
%\vfill\eject
\begin{center}
\begin{tabular}{|r|l|}\hline
 1. & I-polarization brightness (Jy/beam at 1 GHz)\\
 2. & Q-polarization brightness (Jy/beam at 1 GHz)\\
 3. & U-polarization brightness (Jy/beam at 1 GHz)\\
 4. & Spectral index (see note below)\\
 5. & Rotation measure (radians $m^{-2}$)\\
 6. & Rotation measure thickness (radians $m^{-2}$)\\
 7. & Component $X$ center (pixels)\\
 8. & Component $Y$ center (pixels)\\
 9. & Major axis (pixels)\\
10. & Minor axis (pixels)\\
11. & Position angle (degrees CCW from North)\\
12. & Type code: 1 Point, 2 Gaussian, 3 solid disk, 4 solid rectangle,\\
    & 5 optically thin sphere, 6 exponential, else point\\ \hline
\end{tabular}
\end{center}
Note that all parameters must be specified for every component.
Adverb {\tt OPTYPE} specifies the type of rotation measure thickness
model, with values {\tt 'SLAB'}, {\tt 'GAUS'}, and {\tt 'EXP '}\@.
The brightness of a component at frequency $\nu$ in GHz is $F = F_1
\nu^\alpha$\@.  Input (if any) and output images are in the
transposition that places celestial coordinates first and frequency on
the third axis.

A third modeling task, similar to {\tt MODSP}, has recently appeared.
This task models I and V image cubes from models similar to those of
{\tt MODIM}, but suitable for spectral lines.  It takes in two image
cubes, one for I or RR polarization and one for V or LL polarization.
Alternatively, it can create new images with no input images used.  Up
to 9999 components are taken from the text file specified by {\tt
  INLIST}\@.  Each non-comment line in this file specifies

\begin{center}
\begin{tabular}{|r|l|}\hline
 1. & R-polarization brightness at line center (Jy/beam)\\
 2. & L-polarization brightness at line center (Jy/beam)\\
 3. & Component $X$ center (pixels)\\
 4. & Component $Y$ center (pixels)\\
 5. & Major axis (pixels)\\
 6. & Minor axis (pixels)\\
 7. & Position angle (degrees CCW from North)\\
 8. & Type code: 1 Point, 2 Gaussian, 3 solid disk, 4 solid rectangle,\\
    & 5 optically thin sphere, 6 exponential, else Gaussian\\
 9. & Line center (channels)\\
10. & Line width (FWHM of Gaussian in channels)\\
11. & Change of line center per $X$ pixel (channels/pixel)\\
12. & Change of line center per $Y$ pixel (channels/pixel)\\
13. & Change of line width per $X$ pixel (channels/pixel)\\
14. & Change of line width per $Y$ pixel (channels/pixel)\\ \hline
\end{tabular}
\end{center}

Note that parameters 1--10 must be specified for every component.  The
remaining position and width derivatives will be taken as zero if
omitted.  Input (if any) and output images are in the transposition
that places celestial coordinates first and frequency on the third
axis.  Input images may be in either I and V polarizations or RR and
LL, output images will be in I and V.  Input images are scaled by {\tt
  FACTOR} before the model is added.  {\tt MODSP} allows the spectral
channel parameters to be linear functions of position around the
center of the component.  Thus, the line center channel and width are
given by
\begin{eqnarray*}
  C(x,y) & = & C_i + \frac{dC_i}{dx} (x - X_i) +
                     \frac{dC_i}{dy} (y - Y_i) \\
  W(x,y) & = & W_i + \frac{dW_i}{dx} (x - X_i) +
                     \frac{dW_i}{dy} (y - Y_i) \\
\end{eqnarray*}
where, for component $i$, $C_i$ is parameter 9 above,
$\frac{dC_i}{dx}$ is parameter 11, $\frac{dC_i}{dy}$ is parameter 12,
$X_i$ is parameter 2, $Y_i$ is parameter 3. $W_i$ is parameter 10,
$\frac{dW_i}{dx}$ is parameter 13, and $\frac{dW_i}{dy}$ is parameter
14.

\end{document}
