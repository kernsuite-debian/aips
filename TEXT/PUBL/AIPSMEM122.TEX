%-----------------------------------------------------------------------
%;  Copyright (C) 2017
%;  Associated Universities, Inc. Washington DC, USA.
%;
%;  This program is free software; you can redistribute it and/or
%;  modify it under the terms of the GNU General Public License as
%;  published by the Free Software Foundation; either version 2 of
%;  the License, or (at your option) any later version.
%;
%;  This program is distributed in the hope that it will be useful,
%;  but WITHOUT ANY WARRANTY; without even the implied warranty of
%;  MERCHANTABILITY or FITNESS FOR A PARTICULAR PURPOSE.  See the
%;  GNU General Public License for more details.
%;
%;  You should have received a copy of the GNU General Public
%;  License along with this program; if not, write to the Free
%;  Software Foundation, Inc., 675 Massachusetts Ave, Cambridge,
%;  MA 02139, USA.
%;
%;  Correspondence concerning AIPS should be addressed as follows:
%;          Internet email: aipsmail@nrao.edu.
%;          Postal address: AIPS Project Office
%;                          National Radio Astronomy Observatory
%;                          520 Edgemont Road
%;                          Charlottesville, VA 22903-2475 USA
%-----------------------------------------------------------------------
\documentclass[twoside]{article}
\usepackage{palatino}
\renewcommand{\ttdefault}{cmtt}
% Highlight new text.
\usepackage{color}
\usepackage{alltt}
\usepackage{graphicx,xspace,wrapfig}
\usepackage{pstricks}  % added by Greisen
\definecolor{hicol}{rgb}{0.7,0.1,0.1}
\definecolor{mecol}{rgb}{0.2,0.2,0.8}
\definecolor{excol}{rgb}{0.1,0.6,0.1}
\newcommand{\Hi}[1]{\textcolor{hicol}{#1}}
%\newcommand{\Hi}[1]{\textcolor{black}{#1}}
\newcommand{\Me}[1]{\textcolor{mecol}{#1}}
%\newcommand{\Me}[1]{\textcolor{black}{#1}}
\newcommand{\Ex}[1]{\textcolor{excol}{#1}}
%\newcommand{\Ex}[1]{\textcolor{black}{#1}}
\newcommand{\No}[1]{\textcolor{black}{#1}}
\newcommand{\hicol}{\color{hicol}}
%\newcommand{\hicol}{\color{black}}
\newcommand{\mecol}{\color{mecol}}
%\newcommand{\mecol}{\color{black}}
\newcommand{\excol}{\color{excol}}
%\newcommand{\excol}{\color{black}}
\newcommand{\hblack}{\color{black}}
%
\newcommand{\AIPS}{{$\cal AIPS\/$}}
\newcommand{\eg}{{\it e.g.},}
\newcommand{\ie}{{\it i.e.},}
\newcommand{\etal}{{\it et al.}}
\newcommand{\tablerowgapbefore}{-1ex}
\newcommand{\tablerowgapafter}{1ex}
\newcommand{\keyw}[1]{\hbox{{\tt #1}}}
\newcommand{\sub}[1]{_\mathrm{#1}}
\newcommand{\degr}{^{\circ}}
\newcommand{\vv}{v}
%\newcommand{\vv}{\varv}
\newcommand{\eq}{\hbox{\hspace{0.6em}=\hspace{0.6em}}}
\newcommand{\newfig}[2]{\includegraphics[width=#1]{data.fig#2}}
%\newcommand{\putfig}[1]{\includegraphics{data.fig#1.eps}}
\newcommand{\putfig}[1]{\includegraphics{#1.eps}}
\newcommand{\whatmem}{\AIPS\ Memo \memnum}
\newcommand{\boxit}[3]{\vbox{\hrule height#1\hbox{\vrule width#1\kern#2%
\vbox{\kern#2{#3}\kern#2}\kern#2\vrule width#1}\hrule height#1}}
%
\newcommand{\memnum}{122}
\newcommand{\memtit}{Modeling Absorption-line Cubes in \AIPS}
\title{
   \vskip -35pt
   \fbox{{\large\whatmem}} \\
   \vskip 28pt
%   \vskip 10pt
%   \fbox{{\Huge \Me{D R A F T}}}
%   \vskip 10pt
   \memtit \\}
\author{Eric W. Greisen}
%
\parskip 4mm
\linewidth 6.5in                     % was 6.5
\textwidth 6.5in                     % text width excluding margin 6.5
\textheight 9.0 in                  % was 8.81
\marginparsep 0in
\oddsidemargin .25in                 % EWG from -.25
\evensidemargin -.25in
\topmargin -0.4in
%\topmargin 0.2in
\headsep 0.25in
\headheight 0.25in
\parindent 0in
\newcommand{\normalstyle}{\baselineskip 4mm \parskip 2mm \normalsize}
\newcommand{\tablestyle}{\baselineskip 2mm \parskip 1mm \small }
%
%
\begin{document}

\pagestyle{myheadings}
\thispagestyle{empty}

\newcommand{\Rheading}{\whatmem \hfill \memtit \hfill Page~~}
\newcommand{\Lheading}{~~Page \hfill \memtit \hfill \whatmem}
\markboth{\Lheading}{\Rheading}
%
\vskip -.5cm
\pretolerance 10000
\listparindent 0cm
\labelsep 0cm
%
%

\vskip -30pt
\maketitle

\normalstyle
\begin{abstract}
  \AIPS\ does Gaussian fitting of spectral lines with
  recently-overhauled task \keyw{XGAUS} and can fit V polarization
  image cubes for Zeeman-splitting with the relatively new task
  \keyw{ZEMAN}\@.  Both of these tasks are designed for emission
  spectra in which the noise is not a function of spectral channel.
  In absorption, however, the noise in optical depth becomes high when
  the optical depth is high.  Therefore, new tasks \keyw{AGAUS} and
  \keyw{ZAMAN} have been written to provide similar functions but with
  mathematics suitable for absorption lines.  This memo describes the
  new tasks in some detail and includes a description of a new,
  simplified modeling task \keyw{MODAB} which may also be useful.
  That task has shown that the results of these four tasks are biased
  by the presence of the Zeeman splitting and need modest correction
  if they are meant to describe the actual pre-splitting line widths
  and magnetic field.
\end{abstract}

\renewcommand{\floatpagefraction}{0.75}
\typeout{bottomnumber = \arabic{bottomnumber} \bottomfraction}
\typeout{topnumber = \arabic{topnumber} \topfraction}
\typeout{totalnumber = \arabic{totalnumber} \textfraction\ \floatpagefraction}

\section{Introduction}

{\tt XGAUS} was written a long time ago in the hopes that converting a
spectral cube, with many spectral channels at each spatial location,
into one or more sets of images of peak, center, width, and integral
would simplify the data presentation.  One could even hope that the
separate Gaussians fit might have some separate physical reality.
{\tt XGAUS} was extensively re-written beginning in 2013 and is now a
quite friendly and powerful task.  Accompanying this task is {\tt
  ZEMAN} which first appeared in 2013 and which solves for Zeeman
splitting of the right and left polarizations using a new Gaussian
model or the older methods involving the derivative of the unpolarized
spectra.  These tasks are described in detail in \AIPS\ Memo 118
(revised)\footnote{Greisen, Eric W., 2013+, ``Modeling Spectral Cubes
  in \AIPS'', AIPS Memo 118, NRAO, Socorro, NM.}

However, these tasks are designed for emission lines for which the
noise is not a significant function of spectral channel.  In
absorption, the optical depth is the physical parameter which one
might expect to be Gaussian.  However, the noise is a significant
function of spectral channel whenever the optical depths become large.
This calls for a new fitting routine to use Gaussian models in optical
depth, but to fit the actual spectra for which the noise is
approximately independent of spectral channel.  This new task is named
{\tt AGAUS} and is very similar to {\tt XGAUS}, but with enough
differences that combining the two functions in one task is
impractical.

A revision of {\tt ZEMAN}, called {\tt ZAMAN}, has also been written
to solve for Zeeman splitting using the appropriate mathematics for
absorption-line Gaussians.  That mathematics will be presented below
in some detail.  A simple modeling task was written to prepare image
cubes of known properties in order to test these new tasks.  It has
made it clear that the answers from {\tt XGAUS} and {\tt AGAUS} are
affected by the Zeeman splitting as are the answers from {\tt ZEMAN}
and {\tt ZAMAN}\@.  The nature of the bias in the answers will be
discussed below and correction methods suggested.

\section{Absorption-line Gaussian fitting: {\tt AGAUS}}

In absorption, we assume that the optical depth is proportional to the
number of atoms/molecules in the line of sight and that the radial
velocities of these absorbers is approximately Gaussian.  Thus the
function we seek to fit is
\begin{equation}
   T(x,y,z) = (A(y,z) + B(y,z) * x) \times \exp (-\tau(x,y,z))
\end{equation}
where
\begin{equation}
   \tau(x,y,z) = \sum_1^{\tt NGAUSS} \tau_i e^{-\alpha (x - x_i)^2 /
     \sigma_i^2}
\end{equation}
where $x$ is the coordinate along the first axis of the image cube,
$A(y,z) + B(y,z)x$ is the spectral baseline (the continuum),
$\tau_i(y,z)$ are the peak optical depths of each of the {\tt NGAUSS}
components, $x_i(y,z)$ are the center channels of the components,
$\sigma_i(y,z)$  are the full widths at half maximum of the components
and $\alpha = 4 \ln(2)$ to impart this meaning to the $\sigma_i$.

{\tt AGAUS} will fit this formulation to every row of a Stokes I image
cube.  Note that this cube must contain the full continuum emission of
the source ($A(y,z)$) in order to be described by these equations.  It
then produces images over the second and third input axes of the fit
parameters.  Normal usage has the first axis of the cube as a spectral
axis, either in frequency or velocity units, and the second and third
axes are celestial coordinates.  This usage will be assumed in the
later discussion.

\subsection{Inputs}

The usual {\tt INNAME} {\it et al.} adverbs define the cube to be fit
and the {\tt OUTNAME} {\it et al.} adverbs define the (eventual)
output name.  The {\tt INVERS} adverb controls which {\tt XG} table is
used by the task, with $\leq 0$ meaning a new table.  {\tt BLC} and
{\tt TRC} define the pixel ranges to be used in the current execution,
where {\tt BLC(1)} and {\tt TRC(1)} control the spectral channels that
will be fit and {\tt BLC(2)}, {\tt TRC(2)}, {\tt BLC(3)}, and {\tt
TRC(3)} control the area in celestial coordinates over which the
fitting is done.  New {\tt XG} table files are created for the
entire input image cube, but are filled with peak brightnesses limited
by {\tt BLC(1)} through {\tt TRC(1)}\@.  Therefore, it would be wise
to use as much of the first axis as is reliable when creating a new
table.  The second and third values of these adverbs then limit the
area over which fitting is done during this execution of {\tt
  AGAUS}\@.  Adverbs {\tt YINC} and {\tt ZINC} control the stride
taken in the first pass through the cube; a second pass will then fit
all voxels not fit in the first pass.  {\tt FLUX} controls which
spectra will be fit during this execution; all positions with 3
consecutive channels averaging above {\tt FLUX} will be fit.  The
initial guess for the linear baseline is taken as the peak value with
no slope.  {\tt ORDER} controls the order of the baseline with $\leq
0$ meaning a constant and $\ge 1$ meaning a constant plus a slope.
{\tt DOOUTPUT} controls what files are written --- this may be changed
interactively so leave it zero at this point.  Set {\tt DOTV = 2} to
use TV menus to prompt you.  Even when fitting only 1 Gaussian
component, it is best to watch what is happening so you should never
set this adverb false.  {\tt DORESID} controls whether residuals are
plotted on the fit spectra.  Such plots often provide clues when more
Gaussians are needed for the best fits.  However, the plots are
self-scaled from the data only and so may not include the residuals
which should be near zero.  Set {\tt PIXRANGE} if you must see the
residuals.  In general it is best to set {\tt PIXRANGE} to zero to see
the full range of image values in best detail, but, if you are fitting
weak Gaussians in the presence of very strong ones, you might wish to
cut off the lowest values.  This will not interfere with setting
initial guesses as the plot of apparent optical depth is always
self-scaled.  Set {\tt LTYPE} to your favorite type of labeling, set
{\tt PIXVAL} to zero to see all positions being fit, and leave {\tt
  NITER} zero since 100 is more than enough iterations. Set {\tt
  NGAUSS} to the number of Gaussians to be fit in this execution.  It
may be changed if you re-start on a pre-existing  {\tt XG} table since
those tables contain room for the maximum number of Gaussians allowed
(8 at this writing).  {\tt RMSLIMIT} is an upper limit for the rms of
a fit before the fit is viewed as ``failed'' which causes the TV and
interaction to be turned back on after you have turned it off.  You
should get a good idea of an appropriate value from your initial uses
of {\tt AGAUS} or from your knowledge of the noise in your data cube.

Just as in {\tt XGAUS}, {\tt AGAUS} begins by creating an {\tt XG}
table and populating each row with the largest average brightness over
three consecutive channels in the corresponding row in the range {\tt
  BLC(1)} through {\tt TRC(1)}\@..  Then it reads the table every {\tt
  YINC} rows and {\tt ZINC} planes and, for those with an average
brightness greater than {\tt FLUX}, attempts a fit.  Your interaction
with this fit will be described below.  After the first pass, the task
loops over every row and plane fitting those positions which have
enough brightness and which have not already been fit.  Finally, after
all pixels above {\tt FLUX} have been fit, the task goes into an
``edit'' mode.  It constructs images of each fit parameter and the
integral flux of the Gaussian, and of the uncertainties in these
parameters.  You may view these images, select positions explicitly or
by their parameter values or rms and revisit the fits of those
positions.  This stage will be described in detail below.

At any time you may exit the task and then re-start it using the
same {\tt XG} table.  Good reasons for doing this include fitting
smaller regions with each pass using the appropriate number of
Gaussians for that region.  Doing small regions which will have
similar parameter values helps a great deal with the initial guessing
done by the task (mostly using the previous solution).  You might also
fit the cube initially with a high value of {\tt FLUX} and then
re-start with a lower value to extend the areas fit.

\subsection{Fitting}

\begin{figure}
\begin{center}
\resizebox{6.0in}{!}{\putfig{AGAUS.init}}
\caption{First OH spectrum to fit, initial ``guess'' is all zero.}
\label{fig:AGAUS.init}
\end{center}
\end{figure}

\begin{figure}
\begin{center}
\resizebox{6.0in}{!}{\putfig{AGAUS.guess}}
\caption{First OH spectrum to fit: optical depth spectrum to mark the
  Gaussians.}
\label{fig:AGAUS.guess}
\end{center}
\end{figure}

The fitting process starts with a plot of the spectrum with the data
and axis labels in graphics channel one (usually yellow) and the
initial guess as X's in graphics channel two (usually green).  Then
you are offered a menu of options, either in your \AIPS\ terminal
window ({\tt DOTV = 1}) or, as shown in the present figures, on the TV
({\tt DOTV = 2}).  The first spectrum to be fit (from an OH absorption
source) is illustrated in Figure~\ref{fig:AGAUS.init} showing
that the first guess for more than one Gaussian is not useful.  The
menu that appears at this point is\\

\begin{center}
\begin{tabular}{|l|l|}\hline
   {\tt DO FIT}   & {\tt \hphantom{A}} \\
   {\tt RE-GUESS} & {\tt E} \\
   {\tt BAD}      & {\tt B} \\
   {\tt QUIT}     & {\tt Q} \\ \hline
\end{tabular}
\end{center}

You select a menu option by moving the cursor to the desired option
with the mouse and registering that move with the TV by clicking the
left mouse button.  The selected menu item will change color as shown
in the figure.  If you press TV ``button'' {\tt D} at this point
(actually keyboard character D), helpful information about the
selected item will appear on your terminal window.  If you press one
of TV ``buttons'' {\tt A}, {\tt B}, or {\tt C} (actually keyboard
characters A, B, C), the selected function will be performed.  The
option to {\tt QUIT} (or {\tt Q} on the terminal) causes the task to
quit at this point.  You may re-start later.  The option {\tt BAD}
({\tt B} on the terminal) will mark this position as failed and go on
to the next position.  The option {\tt DO FIT} will cause the task to
attempt the non-linear fit with the current initial guess.  The
selected option in Figure~\ref{fig:AGAUS.init} is {\tt RE-GUESS} which
causes the task to change the plot to a spectrum of apparent optical
depth, such as that shown in Figure~\ref{fig:AGAUS.guess}.  It will
then prompt you first to {\it ``Position cursor at center \&\ height
  of Gaussian component 1''}.  Move the cursor to the peak of
component 1 and press any TV button.  This selects the peak value and
center of component 1.  Then the task prompts you to {\it ``Position
  cursor at half-width of Gaussian component 1''}.  Move the cursor
horizontally to the approximate position of the half-power point of
component 1 and press any TV button.  The horizontal position of the
cursor then sets the initial guess of the full width of the component.
These prompts are then repeated for components two through {\tt
  NGAUSS}\@.  If you do not want to fit a particular component at
this position, move the cursor outside the rectangular border line
(\ie\ outside the data area of the plot) before pressing the TV
button for that component; it will be omitted.

\begin{figure}
\begin{center}
\resizebox{6.0in}{!}{\putfig{AGAUS.guessed}}
\caption{First OH spectrum to fit: better initial ``guess'' has been
  entered.}
\label{fig:AGAUS.guessed}
\end{center}
\end{figure}

\begin{figure}
\begin{center}
\resizebox{6.0in}{!}{\putfig{AGAUS.fit}}
\caption{First OH spectrum to fit: good fit obtained.}
\label{fig:AGAUS.good}
\end{center}
\end{figure}

After the improved initial guess has been entered the plot is changed
to show the new guess, as illustrated in Figure~\ref{fig:AGAUS.guessed}.
This guess is good so the {\tt DO FIT} option is selected.  After TV
``buttons'' {\tt A}, {\tt B}, or {\tt C} are pressed the task attempts
the fit with the current initial guess.  Then the plot is changed with
the addition of the fit function in graphics plane 4 (usually cyan)
and the residual in graphics plane 3 (usually pink).  This is
illustrated in Figure~\ref{fig:AGAUS.good}.  The individual Gaussians
are also plotted in a graphics plane which is only visible where no
other graphics plane has been turned on.

A different menu appears at this point containing\\
\begin{center}
\begin{tabular}{|l|l|}\hline
   {\tt GOOD}     & {\tt \hphantom{A}} \\
   {\tt DO FIT}   & {\tt D} \\
   {\tt RE-GUESS} & {\tt E} or {\tt R} \\
   {\tt TVOFF}    & {\tt T} \\
   {\tt HAND}     & {\tt H} \\
   {\tt BAD}      & {\tt B} \\
   {\tt 1}        & {\tt 1} \\
   {\tt 2}        & {\tt 2} \\
   {\tt 3}        & {\tt 3} \\
   {\tt QUIT}     & {\tt Q} \\ \hline
\end{tabular}
\end{center}
The option to {\tt QUIT} (or {\tt Q} on the terminal) causes the task
to quit at this point.  You may re-start later.  The option {\tt BAD}
({\tt B} on the terminal) will mark this position as failed and go on
to the next position.   The option {\tt RE-GUESS} ({\tt E} or {\tt R}
on the terminal) will loop back to prompt you for a new guess and
repeat the fit.  Options {\tt 1}, {\tt 2}, $\ldots$, {\tt NGAUSS} will
loop back to plot an initial guess with the selected number of
Gaussians.  ({\tt NGAUSS=3} in the current example.)  Option {\tt
  HAND} ({\tt H} on the terminal) will prompt you to enter using the
terminal the Gaussian parameters for each component.  Enter on one
line for each component, the peak value of the Gaussian (in optical
depth), the center (in pixels with respect to the reference pixel),
and width (in pixels).  Appropriate ranges of parameters in these
units can be seen from the display of the current fit values.  You may
also enter flags to cause one or more parameter values to be fixed
should you re-fit the current spectrum.  The flags are entered after
the 3 parameter values, flags $\leq 0$ mean to hold the corresponding
parameter fixed and omitted flags are taken as 1.  {\tt AGAUS} will
then repeat the display in Figure~\ref{fig:AGAUS.good} to see if you
made a good guess.  Immediately after a {\tt HAND} operation only, the
option {\tt DO FIT} is offered to go back with the hand-entered values
as the initial guess for a new fit.  Option {\tt TVOFF} allows you to
turn off interactivity, allowing the task to run using its own initial
guesses until it finds a completely unreasonable solution or one with
an rms greater than {\tt RMSLIMIT}\@.  When that happens, you are
shown the offending fit parameters and the task resumes with the plot
of Figure~\ref{fig:AGAUS.good} to allow you to try to fix things.
Option {\tt GOOD} (and other initial character on the terminal) tells
the task that you are (reasonably) happy and that it should go on to
the next position.

When fitting only a single Gaussian, {\tt AGAUS} makes an initial
guess based mostly on finding a real peak in the spectrum.  This is
quite reliable, so turning off the TV interaction may save a great
deal of effort, although there will possibly be bad positions to be
fixed up in the next stage of this task.
\vfill\eject

\subsection{Editing and output}

\begin{figure}
\begin{center}
\resizebox{6.0in}{!}{\putfig{AGAUS.edit}}
\caption{OH absorption source peak of component 1 displayed on edit
  view}
\label{fig:AGAUS.edit}
\end{center}
\end{figure}

Eventually all positions selected by {\tt FLUX} and {\tt BLC} and {\tt
  TRC} will have been fit.  At this point, the task computes images of
the fit parameters plus the ``flux'' (area under the Gaussian) and
their uncertainties.  It then offers a lengthy menu of options which
will allow you to view these images and revisit positions that seem to
have produced incorrect fits.  If {\tt NGAUSS} $> 1$, options to swap
portions of image $n$ with corresponding portions of image $m$ are
also offered.  This ``edit'' menu is illustrated in
Figure~\ref{fig:AGAUS.edit}.  Note that the size of the signal portion
of the OH image is quite small.  When an appropriate sub-image is
selected for display, as was done for this figure, {\tt AGAUS}
replicates pixels in both directions to make the image large enough to
see.  Note too that, for legibility in all figures in this memo, the
\AIPS\ TV was run with double-sized characters.

There are three kinds of editing implemented here.  In the first, the
user establishes the parameter extrema which should be viewed as
acceptable.  The extrema currently set are shown in the title lines.
Then, {\tt AGAUS} may be told to flag all solutions not meeting these
criteria, or, more profitably perhaps, to revisit those positions to
see if a better fit can be obtained.  The other editing methods are
similar, but act on a list of pixel positions.  These may be entered
by typing in values or by clicking on suspect pixel positions in the
{\tt CURVALUE} function described below.  The contents of the list may
be viewed, the solutions at the positions may be flagged, or they may
be revisited to attempt for a better solution, or the solutions at the
listed positions may be swapped between components $n$ and $m$.  The
menu will offer only appropriate swaps between components, thus 1, 2,
and 3 in our {\tt NGAUSS=3} example.  Swapping may be required if {\tt
  AGAUS} gets confused as to which component you want to call number 1
and which number 2.  After the flagging, revisiting, or swapping, the
list is cleared.    The first column of the menu shows the following:
\vfill\eject

\begin{center}
\begin{tabular}{|l|l|}\hline
 {\tt EXIT           } & Exit {\tt AGAUS}, writing output images if
                         {\tt DOOUTPUT} is now $> 0.$\\
 {\tt SET MIN S/N    } & Set minimum amplitude S/N(s) for okay
                         solutions\\
 {\tt SET MAX RES    } & Set maximum residual for okay solutions\\
 {\tt SET PEAK RANGE } & Set peak value range(s) for okay solutions\\
 {\tt SET OFFX RANGE } & Set offset range(s) for okay solutions\\
 {\tt SET WIDTH RANGE} & Set width range(s) for okay solutions\\
 {\tt SET MAX ERR WID} & Set maximum error(s) in width for okay
                         solutions\\
 {\tt REDO ALL       } & Re-do all solutions which are not okay
                         following the above criteria\\
 {\tt FLAG ALL       } & Mark bad all solutions which are not okay\\
 {\tt OFF ZOOM       } & Turn of TV zoom\\
 {\tt OFF TRANSFER   } & Turn off black \&\ white and color TV
                         enhancements\\
 {\tt SET DOOUTPUT   } & Increment {\tt DOOUTPUT} in loop 0-3 --- with
                         1 and 3 causing residual\\
                       & images and 2 and 3 causing parameter images
                         to be written on {\tt EXIT}\\
 {\tt ADD TO LIST    } & Type in output pixel coordinates to add to
                         list\\
 {\tt SHOW LIST      } & Display coordinates in list\\
 {\tt REDO LIST      } & Re-do solutions for all pixels in list\\
 {\tt FLAG LIST      } & Flag solutions for all pixels in list\\
 {\tt SWAP LIST 1-2  } & Swap solutions for components 1 and 2 for all
                         pixels in list\\
 {\tt SWAP LIST 1-3  } & Swap solutions for components 1 and 3 for all
                         pixels in list\\
 {\tt SWAP LIST 2-3  } & Swap solutions for components 2 and 3 for all
                         pixels in list\\ \hline
\end{tabular}
\end{center}

The second (and potentially third and even more) menu columns
contain {\tt NGAUSS} sets of functions

\begin{center}
\begin{tabular}{|l|l|}\hline
 {\tt SHOW IMAGE A1 } & Enter image interaction with peak value of
                     component 1\\
 {\tt SHOW IMAGE C1 } & Enter image interaction with center pixel of
                     component 1\\
 {\tt SHOW IMAGE W1 } & Enter image interaction with width of
                     component 1\\
 {\tt SHOW IMAGE F1 } & Enter image interaction with "flux" of
                     component 1\\
 {\tt SHOW IMAGE EA1} & Enter image interaction with uncertainty in
                     peak value of component 1\\
 {\tt SHOW IMAGE EC1} & Enter image interaction with uncertainty in
                     center pixel of component 1\\
 {\tt SHOW IMAGE EW1} & Enter image interaction with uncertainty in
                     width of component 1\\
 {\tt SHOW IMAGE EF1} & Enter image interaction with uncertainty in
                     "flux" of component 1\\ \hline
\end{tabular}
\end{center}

\begin{figure}
\begin{center}
\resizebox{6.0in}{!}{\putfig{AGAUS.image1c}}
\caption{Center channel component 1 in image interaction stage}
\label{fig:AGAUS.image}
\end{center}
\end{figure}

On very crowded menus, the word {\tt SHOW} may be omitted.  When you
select one of these functions most of the following operations will
appear in yet another menu.  This image interaction menu is
illustrated in Figure~\ref{fig:AGAUS.image}.  Only one of the {\tt
  LOAD AS} options will appear, with the next one in the sequence
offered when the current one has been invoked.  The {\tt SET WINDOW}
option allows you to select a sub-image to view in greater detail,
while {\tt RESET WINDOW} returns to viewing the full image.  The {\tt
  SWAP $n$-$m$} options will appear as needed when {\tt NGAUSS} $ > 1$
and are used to impose your selection of component number $n$ and
component $m$ in case {\tt AGAUS} got confused.  The {\tt NEXT WINDOW}
option appears when needed to display an image too large to fit on the
display screen.  These options mostly invoke familiar functions from
\AIPS\ to control the {\tt FUNCTYPE} used in loading the image to the
display, to enhance the image intensities, to color the enhanced image
intensities, and to zoom the display.

Two operations in this menu are different from the usual.  {\tt
  CURVALUE} provides the capability of selecting positions for the
edit ``list.''  During the {\tt CURVALUE} operation position the
cursor over the desired pixel and press buttons {\tt A}, {\tt B}, or
{\tt C} to add that pixel to the list.  The {\tt SWAP $n$-$m$}
operation uses a TV blotch operation like that in the \AIPS\ verb {\tt
  TVSTAT} and task {\tt BLSUM}\@.  You are to mark with a ``blotch''
region those pixels in the present image which are to have their
solutions swapped with those of the selected component.  Instructions
will appear in the message window as you proceed.  Begin by
positioning the cursor at a pixel to be the first vertex of a
connected sequence of vertices and press TV button {\tt A}\@.  Move to
the next vertex and press button {\tt A} again and repeat until you
have marked all vertices for this region.  Then press button {\tt D}
if you are done with this region or button {\tt C} if you need to
re-position one of the vertices.  In this case, move the cursor to the
vertex to be re-positioned, press button {\tt A} and drag the vertex to
the corrected position.  Press button {\tt A} or {\tt B} to fix that
vertex and go on to reset another vertex or {\tt D} to end this region
and swap the solutions.  You may do this as many times as needed.

\begin{center}
\begin{tabular}{|l|l|}\hline
 {\tt RETURN     } & Return to the above menus, image stays displayed\\
 {\tt LOAD AS SQ } & Re-load image with square root transfer function\\
 {\tt LOAD AS LG } & Re-load image with log transfer function\\
 {\tt LOAD AS L2 } & Re-load image with extreme log transfer function\\
 {\tt LOAD AS LN } & Re-load image with linear transfer function\\
 {\tt SET WINDOW } & Set a sub-image to view\\
 {\tt RESET WINDOW} & Return too viewing the full image\\
 {\tt OFF TRANSF } & Turn off enhancement done with {\tt TVTRANSF}\\
 {\tt OFF COLOR  } & Turn off color enhancements\\
 {\tt TVTRANSF   } & Black \&\ white image enhancement\\
 {\tt TVPSEUDO   } & Color enhancement of numerous sorts\\
 {\tt TVPHLAME   } & Color enhancement of flame type, multiple colors\\
 {\tt TVZOOM     } & Interactive zooming and centering of image\\
 {\tt CURVALUE   } & Display value under cursor, mark pixels for list\\
 {\tt SWAP 1-2   } & Swap solutions for components 1 and 2 interactively\\
 {\tt SWAP 1-3   } & Swap solutions for components 1 and 3 interactively\\
 {\tt SWAP 2-3   } & Swap solutions for components 2 and 3 interactively\\
 {\tt NEXT WINDOW} & Move to next window in large images\\ \hline
\end{tabular}
\end{center}

When you have finished getting the images just the way you want them,
you may write them out as \AIPS\ image files.  Select the {\tt SET
  DOOUTPUT} option until its value, shown at the top of the screen, is
what you want.  In {\tt AGAUS}, values 1 and 3 cause a residual image
cube to be written, while values 2 and 3 cause images of the parameter
values and their uncertainties to be written.  The baseline and slope
images and their uncertainties are given class {\tt CONST}, {\tt
  SLOPE}, {\tt DCONST}, and {\tt DSLOPE}, while the parameter images
and their uncertainties get class {\tt AMPL$n$}, {\tt CENTR$n$}, {\tt
  WIDTH$n$}, {\tt DAMPL$n$}, {\tt DCENT$n$}, and {\tt DWIDT$n$}, and
the flux and its uncertainty get class {\tt FLUX$n$}, and {\tt
  DFLUX$n$}, where $n$ is the component number.

%\vfill\eject
\section{Zeeman splitting: {\tt ZAMAN}}

When an intrinsically unpolarized spectral line is emitted in the
presence of a magnetic field, the right and left circular
polarizations have their frequencies shifted in opposite directions by
an amount proportional to the magnetic field (at least for modest
magnetic fields).  The traditional analysis of data to measure this
splitting works only for those cases in which the separation of
polarizations is a small fraction of the line width.  In that case,
the function that has been traditional is
\begin{equation}
     V(x,y,z) = A(y,z) I(x,y,z) + 0.5 B(y,z)\,\, \frac{dI(x,y,z)}{dx}
\label{eqn:Zeman}
\end{equation}
where $V(x,y,z)$ is the V Stokes polarization component. $I(x,y,z)$ is
the unpolarized I Stokes component, $x$ is the spectral axis value,
$(y,z)$ is the celestial coordinate value, and $A(y,z)$ and $B(y,z)$
are the parameters to be found by a linear least-squares method.
\footnote{Sault, R.J., Killeen, N. E. B.., Zmuidzinas, J., Loushin, R.
  1990, {\it Ap.~J.}, {\bf 74}, 437-461.}  The \AIPS\ task {\tt ZAMAN}
offers this model, with two choices for the method by which the
derivative of $I$ with $x$ is computed.

If the total intensity spectrum has had {\tt AGAUS} applied, another
function may be used instead.  I present below a derivation of the
formula for absorption, neglecting the leakage term until the end.
Let us assume that the optical depth is due to a sum of Gaussians each
having its own depth, center channel, and width.  Thus
\begin{eqnarray}
   I(x,y,z) & = & C(y,z) exp (-\sum_1^{\tt NGAUSS} \tau_i(x,y,z) ) \\
            & = & C(y,z) \prod_1^{\tt NGAUSS} exp (-\tau_i(x,y,z))
\end{eqnarray}
When shifted by magnetic fields, assumed different for each Gaussian,
the formula for the V Stokes becomes
\begin{eqnarray}
  V(x,y,z) & \equiv & C(y,z) \left( \prod_1^{\tt NGAUSS} exp
            (-\tau_{iR}(x,y,z)) - \prod_1^{\tt NGAUSS} exp
            (-\tau_{iL}(x,y,z)) \right) \\
           & = & C(y,z) \prod_1^{\tt NGAUSS} exp (-\tau_{iA})
            exp (-(\tau_{iR}-\tau_{iA})) - \prod_1^{\tt NGAUSS}
             exp (-\tau_{iA}) exp (-(\tau_{iL}-\tau_{iA}))
             \label{eqn:ZGdiff}
\end{eqnarray}
where we assume that the difference between $\tau_{iL}$ and
$\tau_{iA}$ is small.  Then
\begin{eqnarray}
exp (-(\tau_{iL}-\tau_{iA})) & \approx & 1 + \tau_{iA} - \tau_{iL} \\
  & = & 1 + \tau_{i0} \left( \exp(-\alpha (x-x_i)^2/\sigma_i^2) -
        \exp (-\alpha (x - (x_i + \delta_i/2))^2 / \sigma_i^2)
        \right) \\
  & \approx & 1 + \tau_{i0} \exp (-\alpha (x - x_i)^2/\sigma_i^2)
             \left[ 1 - exp (\alpha (x-x_i)\delta_i/\sigma_i^2 -
               \alpha\delta_i^2/(4\sigma_i^2)) \right] \\
  & \approx & 1 + \tau_{i0} \exp(-\alpha (x - x_i)^2/\sigma_i^2)
                \left[ -\alpha (x-x_i)\delta_i/\sigma_i^2 \right] \\
exp (-(\tau_{iR}-\tau_{iA})) & \approx & 1 + \tau_{i0}
          \exp(-\alpha (x - x_i)^2/\sigma_i^2)
          \left[ \alpha (x-x_i)\delta_i/\sigma_i^2 \right]
\end{eqnarray}
Thus each of the terms in Equation~\ref{eqn:ZGdiff} is of the form
\begin{eqnarray}
  & = & C(y,z) \prod_1^{\tt NGAUSS} exp (-\tau_{iA}) (1 \pm a_i) \\
  & = & C(y,z) exp (-\sum\tau_i) \prod_1^{\tt NGAUSS} (1 \pm a_i) \\
  & \approx & C(y,z) exp (-\tau) \left( 1 \pm \sum a_i \right)
\end{eqnarray}
We note that the product of the $\exp (-\tau_{iA})$ is just the total
absorption factor $\exp (-\tau)$ and the product of a number of
$(1 \pm a_i)$, where $a_i$ is small, is approximately $1 \pm \sum
a_i$.  We end up with
\begin{eqnarray}
  \tau(x,y,z) & = & \sum_1^{\tt NGAUSS} \tau_{i0} \exp (-\alpha
              (x-x_i)^2 / \sigma_1^2) \\
  V(x,y,z) & \approx & C(y,z) exp (-\tau(x,y,z)) \sum_1^{\tt NGAUSS}
              (2 \tau_i \alpha (x - x_i) \delta_i / \sigma_i^2)
              exp (-\tau_i(x,y,z)) \label{eqn:ZemanGaus}\\
  V(x,y,z) & = & A(y,z) I(x,y,z) + 0.5 \sum_1^{\tt NGAUSS} B_i(y,z)\,
        \,\frac{-d\tau_i(x,y,z)}{dx}
\end{eqnarray}
where $\tau_i(x,y,z)$ is the spectrum of the $i$'th Gaussian component
in the absorption and one solves for the I polarization leakage
$A(y,z)$ and {\tt NGAUSS} component frequency separations $B_i(y,z)$.
Any association of $B_i(y,z)$ with magnetic field values is left to
the user.

\subsection{Inputs}

The inputs for {\tt ZAMAN} are very similar to those for {\tt
  AGAUS}\@.  {\tt INNAME} {\it et al.}~specify the V polarization
cube which must be in transposed form with frequency/velocity as the
first axis.  {\tt IN2NAME} {\it et al.}~specify the corresponding I
polarization cube, similarly transposed.  The axes of the two images
must match.  {\tt BLC} and {\tt TRC} define the spectral and celestial
coordinate regions of interest for the fit and {\tt FLUX} gives
the lower limit in the I image for the average of three consecutive
channels for the spectra to be fit.  {\tt INVERS} specifies the input
version of the {\tt ZE} table in which the results are stored.  Zero
means to make a new one, other values mean to re-visit an existing
solution.  Adverb {\tt OPTYPE} specifies which of the above formul\ae\
are solved.  {\tt OPTYPE = 'GAUS'} says to use the {\tt XG} file {\tt
  IN2VERS} attached to the input I polarization cube to solve using
Equation~\ref{eqn:ZemanGaus}.  {\tt OPTYPE = '2SID'} says to solve
Equation~\ref{eqn:Zeman}, evaluating the derivative by
$$
\frac{dI(x,y,z)}{dx} = 0.5 \left[\, I(x+1,y,z) - I(x-1,y,z) \,\right]
$$
while any other {\tt OPTYPE} value says to evaluate the derivative
with
$$
\frac{dI(x,y,z)}{dx} = I(x,y,z) - I(x-1,y,z)
$$
while solving Equation~\ref{eqn:Zeman}.  {\tt DOOUTPUT} controls what
files are written --- this may be changed interactively so leave it
zero at this point.  Set {\tt DOTV = 2} to use TV menus to prompt you.
Even though the fitting operation is linear and so needs little or no
guidance, it is best to watch what is happening so you should never
set this adverb false.  If all seems well, you can turn off the TV
after watching a few of the solutions.  $A(y,z)$ is always fit, but
should be vanishingly small if {\tt ZAMAN} is applied to a cube
written by {\tt ZAMAN} with the leakage removed.  {\tt RMSLIMIT} is an
upper limit for the rms of a fit before the fit is viewed as
``failed'' which causes the TV and interaction to be turned back on
after you have turned it off.  You should get a good idea of an
appropriate value from your initial uses of {\tt ZAMAN} or from your
knowledge of the noise in your data cube.

\begin{figure}
\begin{center}
\resizebox{6.0in}{!}{\putfig{ZAMAN.init}}
\caption{First OH spectrum to fit, initial ``guess'' has Gaussians but
   the three $B_i(y,z)$ are zero.}
\label{fig:ZAMAN.init}
\end{center}
\end{figure}

{\tt ZAMAN} has one additional adverb, {\tt DOCONT}, for which the
special value $-2$ causes changed behavior in the fitting.  (A special
value was chosen to avoid accidental use of this option since {\tt
  DOCONT} is used by a number of other tasks.)  If the Vpol image cube
was produced from continuum-subtracted visibilities (\ie\ the output
of {\tt UVLSF} or {\tt UVLIN}), then the actual Ipol leakage will be
proportional to the continuum subtracted Ipol, $I(x,y,z) - C(y,z)$
rather than $I(x,y,z)$\@.  {\tt ZAMAN} does the fitting differently
when this option is selected and records the fact in the {\tt ZE}
table by setting the keyword {\tt ABSORPTN} to 2, rather than the
usual 1 for normal absorption modeling.  The post-fitting plot task
{\tt XG2PL} uses this keyword value to honor the changed meaning of
the leakage gain.

%\vfill\eject

\subsection{Fitting}

\begin{figure}
\begin{center}
\resizebox{5.5in}{!}{\putfig{ZAMAN.good}}
\caption{First OH spectrum to fit, plot after fitting.}
\label{fig:ZAMAN.good}
%\end{center}
%\end{figure}

%\begin{figure}[b]
%\begin{center}
\resizebox{5.5in}{!}{\putfig{ZAMAN.init2}}
\caption{Second OH spectrum to fit, initial ``guess'' has Gaussians and
   values for $B_i(y,z)$.}
\label{fig:ZAMAN.init2}
\end{center}
\end{figure}

The fitting process starts with a plot of the I polarization spectrum
across the top and the V polarization across the bottom.  The data
and axis labels are plotted in graphics channel one (usually yellow)
and the initial guess as lines in graphics channel two (usually
green).  The Gaussians, when {\tt OPTYPE='GAUS'}, are shown as a
smooth line on the I polarization plot, while the initial guess,
plotted in the V polarization, is plotted at the locations of the data
samples.  Then you are offered a menu of options, either in your
\AIPS\ terminal window ({\tt DOTV = 1}) or, as shown in the present
figures, on the TV ({\tt DOTV = 2}).  The first spectrum to be fit
(from a model of the absorption shown for {\tt AGAUS}) is illustrated
in Figure~\ref{fig:ZAMAN.init} showing that the first guess has zero
for the three $B_i(y,z)$, but a reasonable guess at the $A(y,z)$.
Thus, the green line mimics the shape of the I spectrum, not the V
spectrum.  The menu that appears at this point is\\
\begin{center}
\begin{tabular}{|l|l|}\hline
   {\tt DO FIT}   & {\tt \hphantom{A}} \\
   {\tt BAD}      & {\tt B} \\
   {\tt QUIT}     & {\tt Q} \\ \hline
\end{tabular}
\end{center}
Note that there is no {\tt RE-GUESS} since there is no need to enter
them with linear least squares.  You select menu options in the same
way as {\tt AGAUS}, with ``button'' {\tt D} giving appropriate
real-time help information and buttons {\tt A}, {\tt B}, and {\tt C}
selecting the highlighted option.  The option to {\tt QUIT} (or {\tt
  Q} on the terminal) causes the task to quit at this point.  You may
re-start later.  The option {\tt BAD} ({\tt B} on the terminal) will
mark this position as failed and go on to the next position.  The
option {\tt DO FIT}, currently highlighted, will cause the task to
attempt the linear fit with the current initial guess.  The result of
that fit is shown in Figure~\ref{fig:ZAMAN.good}.  The second pixel
starts with the $B_i(y,z)$ of the last fit
(Figure~\ref{fig:ZAMAN.init2}) and so is a much better initial guess.

\subsection{Editing and output}

Eventually all positions selected by {\tt FLUX} and {\tt BLC} and {\tt
  TRC} will have been fit.  At this point, the task computes images of
the fit parameters $A(y,z)$ and $B_i(y,z)$ and their uncertainties.
It then offers a menu of options which will allow you to view these
images and revisit positions that seem to have produced incorrect
fits.  This ``edit'' menu is illustrated in
Figure~\ref{fig:ZAMAN.edit}.  Note that the size of the OH region
of interest is quite small as illustrated here.  If {\tt SET
  WINDOW} is used to select a small sub-image, then {\tt ZAMAN} will
blow it up by pixel replication to a reasonable size as in the
  following figure.

There are two kinds of editing implemented here.  In the first, the
user establishes the parameter extrema which should be viewed as
acceptable.  The extrema currently set are shown in the title lines
and include the maximum rms, the range of allowed values for ``gain''
($A(y,z)$), the range of allowed values for ``field'' ($B(y,z)$ in
pixels), and the maximum uncertainty in the field.  Then, {\tt ZAMAN}
may be told to flag all solutions not meeting these criteria, or, more
profitably perhaps, to revisit those positions to see why a poor fit
was obtained.  Note that, unlike the other tasks in this memo, the
linear nature of the fit in {\tt ZAMAN} means that only one solution
is possible at each celestial coordinate.  The other editing method is
similar, but acts on a list of pixel positions.  These may be entered
by typing in values or by clicking on suspect pixel positions in the
{\tt CURVALUE} function described below.  The contents of the list may
be viewed, the solutions at the positions may be flagged, and they may
be revisited to see why they are suspect.  Note that there are no
swapping of solutions in this task; {\tt AGAUS} establishes which
component is which.  After the flagging or revisiting, the list is
cleared.  The first column of the menu includes:

\begin{figure}
\begin{center}
\resizebox{5.5in}{!}{\putfig{ZAMAN.edit}}
\caption{{\tt ZAMAN} fit and image editing screen.}
\label{fig:ZAMAN.edit}
\end{center}
\end{figure}

\begin{center}
\begin{tabular}{|l|l|}\hline
 {\tt EXIT           } & Exit {\tt AGAUS}, writing output images if
                         {\tt DOOUTPUT} is now $> 0.$\\
 {\tt SET MAX RES    } & Set maximum residual for okay solutions\\
 {\tt SET GAIN RANGE } & Set gain value range(s) for okay solutions\\
 {\tt SET FIELD RANGE } & Set field range(s) for okay solutions\\
 {\tt SET MAX ERR FLD} & Set maximum error(s) in field for okay
                         solutions\\
 {\tt REDO ALL       } & Re-do all solutions which are not okay\\
 {\tt FLAG ALL       } & Mark bad all solutions which are not okay\\
 {\tt OFF ZOOM       } & Turn of TV zoom\\
 {\tt OFF TRANSFER   } & Turn off black \&\ white and color TV
                         enhancements\\
 {\tt SET DOOUTPUT   } & Increment {\tt DOOUTPUT} in loop 0-3 --- with
                         1 and 3 causing residual\\
                       & images and 2 and 3 causing parameter images
                         to be written on {\tt EXIT}\\
 {\tt ADD TO LIST    } & Type in output pixel coordinates to add to
                         list\\
 {\tt SHOW LIST      } & Display coordinates in list\\
 {\tt REDO LIST      } & Re-do solutions for all pixels in list\\
 {\tt FLAG LIST      } & Flag solutions for all pixels in list\\ \hline
\end{tabular}
\end{center}

The second menu column contains
\begin{center}
\begin{tabular}{|l|l|}\hline
 {\tt SHOW IMAGE G } & Enter image interaction with gain\\
 {\tt SHOW IMAGE EG } & Enter image interaction with uncertainty in
                     the gain\\
 {\tt SHOW IMAGE F1 } & Enter image interaction with field of
                     component 1\\
 {\tt SHOW IMAGE EF1} & Enter image interaction with uncertainty in
                     field of component 1\\
 {\tt SHOW IMAGE F2 } & Enter image interaction with field of
                     component 2\\
 {\tt SHOW IMAGE EF2} & Enter image interaction with uncertainty in
                     field of component 2\\
 {\tt SHOW IMAGE F3} & Enter image interaction with field of component
                     3\\
 {\tt SHOW IMAGE EF3} & Enter image interaction with uncertainty in
                     field of component 3\\ \hline
\end{tabular}
\end{center}
There is 1 or, if {\tt OPTYPE='GAUS'} as in the illustrated cases,
the maximum of {\tt NGAUSS} of the {\tt F}$n$ and {\tt EF}$n$ options.
When you select one of the {\tt SHOW} options, the options in yet
another menu appear along with a display of the selected image.  As
illustrated in Figure~\ref{fig:ZAMAN.field1}, these options are
\begin{center}
\begin{tabular}{|l|l|}\hline
 {\tt RETURN     } & Return to the above menus, image stays displayed\\
 {\tt LOAD AS SQ } & Re-load image with square root transfer function\\
 {\tt LOAD AS LG } & Re-load image with log transfer function\\
 {\tt LOAD AS L2 } & Re-load image with extreme log transfer function\\
 {\tt LOAD AS LN } & Re-load image with linear transfer function\\
 {\tt SET WINDOW } & Set a sub-image to view\\
 {\tt RESET WINDOW} & Return too viewing the full image\\
 {\tt OFF TRANSF } & Turn off enhancement done with {\tt TVTRANSF}\\
 {\tt OFF COLOR  } & Turn off color enhancements\\
 {\tt TVTRANSF   } & Black \&\ white image enhancement\\
 {\tt TVPSEUDO   } & Color enhancement of numerous sorts\\
 {\tt TVPHLAME   } & Color enhancement of flame type, multiple colors\\
 {\tt TVZOOM     } & Interactive zooming and centering of image\\
 {\tt CURVALUE   } & Display value under cursor, mark pixels for list\\
 {\tt NEXT WINDOW} & Move to next window into large images\\ \hline
\end{tabular}
\end{center}
Only one of the {\tt LOAD AS} options will appear, namely the next
after the current transfer function from the list of linear, square
root, log, and more extreme log transfer functions.  {\tt CURVALUE}
provides the capability of selecting positions for the edit ``list.''
During the {\tt CURVALUE} operation position the cursor over the
desired pixel and press buttons {\tt A}, {\tt B}, or {\tt C} to add
that pixel to the list.  The other options are familiar as {\tt AIPS}
verbs.  Instructions for interaction will appear on the terminal and
button {\tt D} in the menu may be used to obtain a helpful display on
the terminal.

\begin{figure}
\begin{center}
\resizebox{6.0in}{!}{\putfig{ZAMAN.field1}}
\caption{{\tt ZAMAN} image of $B_1(y,z)$ with display options.}
\label{fig:ZAMAN.field1}
\end{center}
\end{figure}

When you have finished getting the images just the way you want them,
you may write them out as \AIPS\ image files.  Select the {\tt SET
  DOOUTPUT} option until its value, shown at the top of the screen, is
what you want.  In {\tt ZAMAN}, values 4, 5, 6, and 7 cause a residual
(data-model) image cube to be written, while values 2, 3, 6, and 7
cause images of the gain and field values and their uncertainties to
be written and values 1, 3, 5, and 7 cause an image of the V cube to
be written with the gain times the I-polarization cube subtracted.
The corrected V-polarization image gets the class specified by {\tt
  OUTCLASS}, the residual image gets class {\tt VRESID}, the gain and
its uncertainty get classes {\tt GAIN} and {\tt DGAIN}, and the
field(s) and their uncertainties get classes {\tt FIELD}$n$ and {\tt
  DFELD}$n$, with $n = 1$ to the maximum {\tt NGAUSS}\@.
%\vfill\eject

\section{Post-fit plotting}

The images produced by {\tt XGAUS}, {\tt AGAUS}, {\tt ZEMAN} and {\tt
  ZAMAN} may be displayed using all the usual tools such as {\tt
  KNTR}, {\tt PROFL}, and numerous other tasks.  However, the display
of the spectral data and the various fits to them required a new task.

\begin{figure}
\begin{center}
\resizebox{6.0in}{!}{\putfig{XG2PL.agaus}}
\caption{Example spectrum from a single pixel for fits done by {\tt
    AGAUS} and {\tt ZAMAN}.  Data are in black, models in green, and
  model components in cyan.  The model line at left had a width of 8
  channels with a splitting of $-1$ channel, the center line had a
  width of 15 channels with no splitting, and the line at right had a
  width of 10 channels and a splitting of $+1.5$ channels.  A leakage
  gain of 0.01 was applied to the full Ipol signal.}
\label{fig:XG2PL.agaus}
\end{center}
\end{figure}

{\tt XG2PL} plots a spectrum for a single pixel or for a rectangular
or circular region about a single pixel.  For each pixel included in
the average, the task reads the I polarization image to obtain the
data and the appropriate line on the {\tt XG} table to obtain the {\tt
  AGAUS} solution for that pixel.  It then computes the spectrum of
each component in the model, plus the sum of the components, and the
residual (data-model).  Each of these then enter into the average of
that parameter.  Finally, the task plots a user-selected number of the
parameters and, optionally, prints all of them to a text file.  The
plot may appear on the TV or be placed in a standard plot file
attached to the I polarization image.

Optionally, {\tt XG2PL} will also add the spectrum of the V
polarization data and the results of the fitting done by {\tt ZEMAN}
or {\tt ZAMAN} at the same pixels as the I polarization Gaussians.
The task reads the V polarization image for the data spectrum and
the {\tt ZE} table for the Zeeman-splitting model (either using the
Gaussians or the simpler ones using the I polarization data).  It
computes the spectrum of each component (including the gain term in
each), the net model (sum of the components but including the gain
term only once), and the residual.  Finally, the task plots a
user-selected number of the parameters and, optionally, prints all of
them to a text file.  In general, the I polarization spectrum appears
in the upper part of the plot and the V polarization spectrum appears
in the lower portion. Either portion may be omitted under control of
the adverbs.  The output of {\tt XG2PL} is illustrated in
Figure~\ref{fig:XG2PL.agaus} for the absorption model.

The inputs for {\tt XG2PL} begin with the I polarization image which
is required and then the V polarization image which may be omitted.
{\tt INVERS} and {\tt IN2VERS} give the version numbers of the {\tt
  XG} and {\tt ZE} tables, respectively.  {\tt APARM} provides the
central pixel coordinates, the plot intensity ranges for I and for V,
the size of the rectangle (or circle) over which to average, and a
flag limiting the average to those pixels having a model fit,  {\tt
  BPARM} is a set of flags selecting which parameters are plotted.
{\tt CPARM} selects the channel range to be plotted, the relative size
of the V and I plots, the type of the horizontal axis (channels,
frequency, velocity), whether the channels are plotted in reverse
order, and whether the data are plotted with stepped or directly
connected lines.  For absorption models, {\tt CPARM} also offers the
option of plotting the I spectra as optical depth rather than as
observed.  {\tt OUTTEXT} specifies the output text file, if any.  The
usual {\tt XYRATIO}, {\tt LTYPE}, {\tt DOTV}, and {\tt GRCHAN} adverbs
control the scale, labeling, choice of TV versus plot file, and, if
TV, which graphic channel(s) are used.  If {\tt GRCHAN} is zero,
graphics channel 1 is used for data and labeling (usually yellow),
channel 2 is used for the full model (usually green), channel is used
for the residual (usually pink), and channel 4 is used for the model
component(s) (usually cyan).

\section{Image model creation}

\AIPS\ Memo 118 describes a wide variety of modeling tasks which will
not be discussed here.  A new, simplified task was written to provide
direct studies of the fitting in {\tt XGAUS}, {\tt AGAUS}, {\tt
  ZEMAN}, and {\tt ZAMAN}\@.

The new task is called {\tt MODAB} and has a small number of adverbs.
{\tt COORDINA}, {\tt CELLSIZE}, {\tt APARM(1)}, and {\tt APARM(2)} set
image header parameters with values that are not otherwise used.  {\tt
  IMSIZE} and {\tt CPARM(3)} set the size of the image cube in
pixels.  {\tt APARM(4)} through {\tt APARM(9)} define a Gaussian which
is the continuum ``source'' when imaging in absorption or is used to
scale the emission as a function of position when making an emission
cube.  {\tt APARM(10)} defines the fractional leakage of the I
polarization into V and {\tt FLUX} sets the noise level in Jy/beam.
{\tt OPTYPE = 'EMIT'} requests an emission cube, otherwise an
absorption cube is produced.

A text file, pointed at with {\tt INLIST} contains up to 99 spectral
features to be spread over the cube.  Each line must contain
\begin{center}
\begin{tabular}{|r|l|}\hline
 1. & Central optical depth (unitless) or peak brightness (Jy/beam)\\
 2. & Central spectral channel (pixels)\\
 3. & Full width of line at half-maximum (pixels)\\
 4. & Separation of R and L polarization in channels (pixels)\\\hline
\end{tabular}
\end{center}

\section{Systematic errors}

\begin{figure}
\begin{center}
\resizebox{6.0in}{!}{\putfig{PLOTR.model10}}
\caption{Measured relative widths ($s/\sigma$) in green and splittings
  ($s/\sigma$) in red plotted against the model splitting
  $\delta/\sigma$.}
\label{fig:PLOTR.model}
\end{center}
\end{figure}

The image model task described above was used in both emission and
absorption with high signal-to-noise to check for ``systematic
errors'' in the answers returned by {\tt XGAUS}, {\tt AGAUS}, {\tt
  ZEMAN} and {\tt ZAMAN}\@.  The I polarization Gaussian width, not
surprisingly, is measured to be wider than $\sigma_i$ since the
emission profile is widened by the splitting due to the magnetic
field.  The quality of the fit to a Gaussian also deteriorates as the
emission line becomes less Gaussian in shape when the splitting
becomes significant with respect to the unsplit line width.  The
measured splitting in V polarization is computed via
Equation~\ref{eqn:ZemanGaus} and so would appear to suffer from the
bias in the measured $\sigma_i$ squared.  Experimentation with the
modeling tasks shows, however, that the bias in the splitting is
almost the same (fractionally) as in the width.  A variety of model
separations ($\delta$) were tried at several fixed component widths
($\sigma$) with very good signal-to-noise.  The apparent Gaussian
full-widths at half maximum ($s$) were measured with {\tt XGAUS} and
the apparent Zeeman splittings ($d$) were measured with {\tt ZEMAN}\@.
When plotted as relative widths ($s/\sigma$) and relative separations
($d/\sigma$) against $\delta/\sigma$, as in
Figure~\ref{fig:PLOTR.model}, the curves for the tested values of
$\sigma$ were approximately the same.  The plotted fits to the curves
in Figure~\ref{fig:PLOTR.model} are given by
\begin{eqnarray}
  s/\sigma & = & 1.0 - 0.010 (\delta/\sigma) + 0.7485
                 (\delta/\sigma)^2 \\
  d/\sigma & = & 1.0 - 0.002 (\delta/\sigma) + 0.6827
                 (\delta/\sigma)^2
\end{eqnarray}
with an rms of the fit of about 0.0002.

Note a certain difficulty with these formul\ae.  Users of {\tt XGAUS},
{\tt AGAUS}, {\tt ZEMAN} and {\tt ZAMAN} will obtain values of $s$ and
$d$ and will have to find $\delta$ and $\sigma$ by some inverse
method.  Note too that, while doing these tests, some rather abnormal
values were found, especially for $d$, despite the rather good
signal-to-noise.  It may be advisable to use the {\tt MODAB} modeling
task to simulate your data in order to determine the real
uncertainties and systematic errors which apply.

\end{document}
