%-----------------------------------------------------------------------
%;  Copyright (C) 2012
%;  Associated Universities, Inc. Washington DC, USA.
%;
%;  This program is free software; you can redistribute it and/or
%;  modify it under the terms of the GNU General Public License as
%;  published by the Free Software Foundation; either version 2 of
%;  the License, or (at your option) any later version.
%;
%;  This program is distributed in the hope that it will be useful,
%;  but WITHOUT ANY WARRANTY; without even the implied warranty of
%;  MERCHANTABILITY or FITNESS FOR A PARTICULAR PURPOSE.  See the
%;  GNU General Public License for more details.
%;
%;  You should have received a copy of the GNU General Public
%;  License along with this program; if not, write to the Free
%;  Software Foundation, Inc., 675 Massachusetts Ave, Cambridge,
%;  MA 02139, USA.
%;
%;  Correspondence concerning AIPS should be addressed as follows:
%;          Internet email: aipsmail@nrao.edu.
%;          Postal address: AIPS Project Office
%;                          National Radio Astronomy Observatory
%;                          520 Edgemont Road
%;                          Charlottesville, VA 22903-2475 USA
%-----------------------------------------------------------------------
%Body of intermediate AIPSletter for 31 December 2012 version

\documentclass[twoside]{article}
\usepackage{graphics}

\newcommand{\AIPRELEASE}{June 30, 2012}
\newcommand{\AIPVOLUME}{Volume XXXII}
\newcommand{\AIPNUMBER}{Number 1}
\newcommand{\RELEASENAME}{{\tt 31DEC12}}
\newcommand{\NEWNAME}{{\tt 31DEC12}}
\newcommand{\OLDNAME}{{\tt 31DEC11}}

%macros and title page format for the \AIPS\ letter.
\input LET98.MAC
%\input psfig

\newcommand{\MYSpace}{-11pt}

\normalstyle

\section{General developments in \AIPS}

\subsection{Reduction of EVLA and ALMA data in \AIPS}

This \Aipsletter\ and those from 2010 and 2011 contain numerous
improvements to \AIPS\ that enable full calibration of EVLA data and
most imaging operations as well.  The one exception is the wide-band
(bandwidth synthesis) deconvolution algorithm (``MSMFS'') being
developed in \CASA\ by Urvashi Rao Venkata, for which there is no
comparable function in \AIPS\@.  Calibrated $uv$ data may be ported
from \AIPS\ in ``UVFITS'' format for use in that program.  ALMA data
may also be reduced in \AIPS, although the package is not fully
qualified to calibrate data from linearly-polarized feeds.  See
Appendix E of the \AIPS\ Cookbook, available via the \AIPS\ web site,
for details.

\subsection{\Aipsletter\ publication}

We have discontinued paper copies of the \Aipsletter\ other than for
libraries and NRAO staff.  The \Aipsletter\ will be available in
PostScript and pdf forms as always from the web site listed above.  It
will be announced in the NRAO e-News mailing and on the bananas list
server.

\subsection{Current and future releases}

We have formal \AIPS\ releases on an annual basis.  While all
architectures can do a full installation from the source files,
Linux (32- and 64-bit), Solaris, and MacIntosh OS/X (PPC and Intel)
systems may install binary versions of recent releases.  The last,
frozen release is called \OLDNAME\ while \RELEASENAME\ remains under
active development.  You may fetch and install a copy of these
versions at any time using {\it anonymous} {\tt ftp} for source-only
copies and {\tt rsync} for binary copies.  This \Aipsletter\ is
intended to advise you of improvements to date in \RELEASENAME\@.
Having fetched \RELEASENAME, you may update your installation whenever
you want by running the so-called ``Midnight Job'' (MNJ) which copies
and compiles the code selectively based on the changes and
compilations we have done.  The MNJ will also update sites that have
done a binary installation.  There is a guide to the install script
and an \AIPS\ Manager FAQ page on the \AIPS\ web site.

The MNJ serves up \AIPS\ incrementally using the Unix tool {\tt cvs}
running with anonymous ftp.  The binary MNJ also uses the tool {\tt
rsync} as does the binary installation.  Linux sites will almost
certainly have {\tt cvs} installed; other sites may have installed it
along with other GNU tools.  Secondary MNJs will still be possible
using {\tt ssh} or {\tt rcp} or NFS as with previous releases.  We
have found that {\tt cvs} works very well, although it has one quirk.
If a site modifies a file locally, but in an \AIPS-standard directory,
{\tt cvs} will detect the modification and attempt to reconcile the
local version with the NRAO-supplied version.  This usually produces a
file that will not compile or run as intended.  Use a copy of the task
and its help file in a private disk area instead.

\AIPS\ is now copyright \copyright\ 1995 through 2012 by Associated
Universities, Inc., NRAO's parent corporation, but may be made freely
available under the terms of the Free Software Foundation's General
Public License (GPL)\@.  This means that User Agreements are no longer
required, that \AIPS\ may be obtained via anonymous ftp without
contacting NRAO, and that the software may be redistributed (and/or
modified), under certain conditions.  The full text of the GPL can be
found in the \texttt{15JUL95} \Aipsletter, in each copy of \AIPS\
releases, and on the web at {\tt http://www.aips.nrao.edu/COPYING}.


\section{Improvements of interest in \RELEASENAME}

We expect to continue publishing the \Aipsletter\ approximately every
six months along with the annual releases.  Henceforth, this
publication will be primarily electronic.  There have been several
significant changes in \RELEASENAME\ in the last six months.  Some of
these were in the nature of bug fixes which were applied to \OLDNAME\
before and after it was frozen.  If you are running \OLDNAME, be sure
that it is up to date to sometime this year (a MNJ after March 8 would
be best). New tasks in \RELEASENAME\ include {\tt MORIF} to break up a
data set into a greater number of spectral windows, {\tt SNREF} to
determine which reference antenna would minimize the right minus left
phase difference, {\tt PRTSY} to print statistics of the values in
SysPower ({\tt SY}) tables from the EVLA, {\tt FIXAN} to correct
errors or change the array center in antenna tables, {\tt SPCOR} to
correct an image cube for spectral index and/or primary beam, {\tt
  FIXRL} to fix data for mislabeled polarizations in some antennas,
and {\tt TARS} to check the Faraday Rotation Synthesis algorithm (task
{\tt FARS}) with simple user input data.  New verbs include {\tt
  NAMEGET} to fill in the file naming adverbs completely to aid in
writing procedures and {\tt IM2HEAD}, {\tt IM3HEAD}, {\tt IM4HEAD},
{\tt IMOHEAD}, {\tt Q2HEADER}, {\tt Q3HEADER}, {\tt Q4HEADER}, and
{\tt QOHEADER} to display headers selected by the second, third, and
fourth input and the output name adverbs, respectively.  New {\tt RUN}
files include {\tt OOCAL} to enable full self-calibration with the
spectral-index and other options of the {\tt OOSUB} task and {\tt
  LINIMAGE} to build a {\tt FLATN}ed spectral cube while separating
spectral windows to improve performance in the {\tt IMAGR} portion.

{\tt 31DEC09} contains a significant change in the format of the
antenna files, which will cause older releases to do wrong things to
data touched by {\tt 31DEC09} and later releases.  {\tt 31DEC08}
contains major changes to the display software.  Older versions may
use the {\tt 31DEC08} display ({\tt XAS}), but {\tt 31DEC08} code may
not use older versions of {\tt XAS}\@.   Magnetic tape logical unit
numbers changed with {\tt 31DEC04}\@.  You are encouraged to use a
relatively recent version of \AIPS, whilst those with EVLA data to
reduce should get the latest release.

\subsection{Mountain lion}

Apple has again announced a new revision of OS X, this time called
Mountain Lion (10.8).  The VAO in Socorro has made a Mac laptop
available to us for testing.  It was found to install and run the
existing {\tt MACINT} load modules.  Several {\tt \$SYSUNIX}
procedures required minor modification since the new version of {\tt
  ps} issues a warning message when it is used by a process with a
library path set.  On lion last year we tried using the latest Intel
compiler (12.0.4 dated 20110503) with the x86\_64 architecture.  That
version ran a little bit faster than the 32-bit version, but not
enough faster to justify making a new {\tt MAC64} ``architecture''
with a new NRAO computer to support it.

\subsection{UV-data calibration and handling}

\subsubsection{Editing}

{\tt RFLAG} has become the standard editing task to remove RFI from
most data sets.  It has received significant attention in the last six
months.  The spectral flagging mode was changed to use a sliding
median window filter of user-set width.  Previously it used,
effectively, a full-width window applying to all channels.  This
caused the disappearance of options to flag full spectra because the
rms was too high or too low.  An option to expand any flags by $\pm$
{\tt FPARM(6)} channels replaced them.  In the spectral mode, {\tt
  RFLAG} automatically deleted all channels not included in {\tt
  ICHANSEL}\@.  Now such channels are retained, but the default {\tt
  ICHANSEL} is now all channels.  A number of errors were corrected
which caused the task to write many more flags than it should,
effectively flagging already flagged data.  Other errors had to be
corrected for the mode in which {\tt RFLAG} is run to write a new flag
table and then immediately plot the consequences of that
table on the data set.  The problems involved pre-clip flux levels as
well as file status.

Following detailed self-calibration, users often delete those data
associated with ``bad'' amplitude and phase solutions.  An option {\tt
  A\&P} has been added to {\tt SNFLG} to perform this operation,
removing the need for a very repetitive use of task {\tt EDITA}\@.
{\tt TVFLG} and {\tt SPFLG} were corrected to retain and use the
source number correctly when flagging a single source extracted
from a multi-source data set.

\subsubsection{Calibration with models}

The task {\tt OOSUB} will subtract a model from, or divide a model
into, a $uv$-data set.  It has options to apply the model in a
frequency-dependent fashion using a model of the primary beam and
images of spectral index and curvature.  It also has options to omit
some of the model Clean components from the computation depending on
their position within the primary beam.  A new procedure has been
written to use {\tt OOSUB} to do self-calibration.  The procedure
divides the model into the data with {\tt OOSUB} and then invokes {\tt
  CALIB} to determine the gains.  The resulting {\tt SN} table is
copied back to the input data set and the temporary, divided data set
is deleted (if requested).

A number of detailed aspects of applying models in tasks like {\tt
  UVSUB} and {\tt CALIB} were also addressed in the last six months.
A user may wish to apply some algorithm to the {\tt CC} tables of some
of the facets of an image, creating {\tt CC} tables of version numbers
greater than one.  The software will now allow the use of the highest
{\tt CC} table version in each facet if {\tt INVERS = 0} or to use a
specific version number, ignoring those facets lacking that version
number, when {\tt INVERS =} {\it n}.  Previously, the software could
get confused and either die or change the modeling to use the Cleaned
images rather than Clean components.  Neither was desirable.  When
using edited {\tt CC} tables, gridded model would get stuck in the
table and only use those components that preceded the first flagged
component.  Models can now come from small images or ones which are
not a power of two on a side.  The former require DFT rather than
gridded modeling, but a bug kept the code from knowing about image
size when it had to make that decision.  The latter often perform well
with gridded modeling, but the sizes of the grids had to be increased
to the next power of two for correct function.

\subsubsection{Miscellaneous $uv$-data matters}

\begin{description}
\myitem{BPASS} no longer supports (\ie\ messes up) the {\tt BIF} and
           {\tt EIF} options.
\myitem{SETJY} now can determine the source velocity at the first data
           sample and write that into the source table with {\tt
           OPTYPE='VCAL'}\@.
\myitem{VLANT} now has a {\tt DOINVERS} option to allow corrections to
           the {\tt CL} and {\tt AN} tables to be undone.  While a new
           {\tt CL} table produced by {\tt VLANT} can simply be
           discarded, the changes to the antenna coordinates are made
           in place so this option is required to undo them.  Errors
           for the EVLA in the sign of the Y correction and the zero
           point of hour angle were found and fixed.
\myitem{FRING} and {\tt RLDLY} can now solve for delays dividing the
           IFs into quarters to support the new 3-bit EVLA system.
\myitem{CLCOR} neglected the fact that the GMRT uses the opposite
           phase sign convention from other telescopes.  This was
           corrected for a number of {\tt OPCODE}s, most importantly
           the one to apply corrections for changes in antenna and
           source positions.
\myitem{FITLD} had the order of the adverbs revised to clarify which
           of the many adverbs apply only to VLBI IDI-format data sets
           and which are more general.
\myitem{FIXRL} is a new task to rearrange data on the polarization
           axis when some antennas have been connected incorrectly
           (swapping R and L usually).
\myitem{INDXR} now has a frequency-dependent VLA gains file which it
           applies to the {\tt CL} file in a frequency-dependent
           manner.
\myitem{ELINT} now displays the rms of the fit on plots.  Adverbs {\tt
           QUAL} and {\tt CALCODE} were added to help separate out the
           desired calibration sources.  Problems reading over-the-top
           tables and large source tables were corrected and the task
           was forced to ignore flagged solutions.
\myitem{SNREF} is a new task to examine the statistics of the right
           minus left phases in an {\tt SN} table as a function of the
           choice of reference antenna.  It should help to select the
           antenna having the most stable R-L and hence providing the
           best polarization calibration.
\myitem{MORIF} is new task which can divide a data set into a larger
           number of spectral windows (IFs) which may help deal with
           spectral index issues at very low frequency and may help
           performance in data sets with a large number of spectral
           channels (see {\tt LINIMAGE} below).
\myitem{PRTSY} is a new task to print the extrema and statistics of
           the EVLA SysPower ({\tt SY}) table as a function of antenna
           and IF\@.  It should help provide parameters for use in
           {\tt TYSMO} and other tasks.
\myitem{FIXAN} is a new and dangerous task that re-writes the
           coordinates of antenna tables.  It has {\tt OPTYPE}s to
           switch between Earth-centric and site-centric systems
           correctly and to correct an apparent site-centric system
           computed by simply differencing Earth-centric coordinates.
           This last is wrong since site-centric systems must have the
           antenna $(X,Y)$ based at the local longitude.
\end{description}

\subsection{Imaging}

\AIPS\ imaging task {\tt IMAGR} received only a little attention in
the last six months.  The biggest change was to a double-precision
pseudo array processor; see below.  This will result in more accurate
images, especially near the corners where round-off error was
magnified by the large correction for the Fourier transform of the
gridding function.  The uniform weighting routine was also worked on
to insure that all summing variables are double precision floats or
integers.  Large EVLA bandwidth synthesis data sets were overflowing
the counters in uniform weighting.  An option was added to average
spectral channels as the data are read in the on-the-fly,
baseline-length dependent time averaging done in writing the work
file.  This will reduce the size of the work file and, in many cases,
will not compromise the accuracy of the images.

When {\tt IMAGR} is used to make a spectral-line cube, the calibration
routines are started up several times per channel both for the input
data which uses a flag table and for the work file which does not.
These calibration start-up routines used to comment on the use of the
flag table on every twentieth call.  This led to confusion since it
would say sometimes ``using flag version $n$'' and other times ``using
no flag table.''  The low level routines were changed so that flag
table use is reported once after each change of the file being read
unless that file is a scratch or work file for which {\tt FLAGVER} was
never set.  For cubes, this means that the flag file usage is
mentioned at the start of each channel.  If the flag table version
changes, as in {\tt RFLAG}, then that change is also reported.

Tasks {\tt SCMAP} and {\tt SCIMG} loop over self-calibration and
imaging (and even data editing) within a single task.  The were
improved by adding {\tt BOXFILE} to define Clean boxes, {\tt IM2PARM}
to allow auto-boxing, and a parameter to allow negative components to
be used in the model and restart of the Clean.  Flag tables were
ignored for input data which were not compressed, while they were used
for compressed data.  This oversight was corrected.

A new {\tt RUN} file and procedure called {\tt LINIMAGE} was written
to help speed the creation of spectral image cubes.  It loops over
spectral window (IF) copying the IF to a scratch file and then invokes
{\tt IMAGR} on that rather smaller file.  This is much faster than
extracting each channel one at a time from the full file.  After {\tt
  IMAGR}, the facets are {\tt FLATN}ed.  When all IFs have been
imaged, the headers are examined and either {\tt MCUBE} or {\tt FQUBE}
is run to put all the IFs back together.  A data set with a very large
number of spectral channels may be run through {\tt MORIF} to prepare
it for {\tt LINIMAGE}\@.

\vfill\eject
\subsection{Display}

\begin{description}
\myitem{LISTR} was changed to do system temperature and gain as
           determined from a SysPower table in the {\tt 'GAIN'}
           listing.  Adverb {\tt XINC} was added to the {\tt 'LIST'}
           and {\tt 'GAIN'} listings as well as an adverb to change
           all times to local sidereal time rather than IAT\@.
\myitem{POSSM} was changed to read the {\tt BP} table to find a list
           of times present in the file to allow it to use {\tt
             SOLINT} properly when plotting {\tt BP} tables.
\myitem{IM2HEAD}\hspace{2em} is a new verb to display the header like
           {\tt IMHEAD}, but for the file pointed to by the adverbs
           {\tt IN2NAME}, {\tt IN2CLASS}, {\tt IN2SEQ}, and {\tt
           IN2DISK}\@.  {\tt IM3HEAD}, {\tt IM4HEAD}, and {\tt
           IMOHEAD} are the same but for the third and fourth input
           name and the output name adverbs.
\myitem{Q2HEADER}\hspace{2em} is a new verb to display the header like
           {\tt QHEADER}, but for the file pointed to by the adverbs
           {\tt IN2NAME}, {\tt IN2CLASS}, {\tt IN2SEQ}, and {\tt
           IN2DISK}\@.  {\tt Q3HEADER}, {\tt Q4HEADER}, and {\tt
           QOHEADER} are the same but for the third and fourth input
           name and the output name adverbs.
\myitem{Slice} plots along a {\tt FQID} axis can usually be plotted on
           a linear frequency axis by placing the plot points
           appropriately.  {\tt ISPEC} and {\tt RSPEC} were changed to
           do this.  Alternatively, the slice can be interpolated to
           regular frequencies as it is computed.  {\tt SLICE}, {\tt
             ISPEC}, {\tt RSPEC}, and {\tt BLSUM} were changed to do
           this.  The slice file format had to be changed slightly to
           provide the necessary information for subsequent plotting.
\myitem{IMEAN} was given a {\tt DOPRINT} option to control printing
           and particularly to write a compact format listing the
           window, minimum, maximum, mean, rms, and flux.  Using this
           option, looping over a plane at a time with {\tt BLC} and
           {\tt TRC}, would give a useful text file for a spectral
           cube.
\myitem{CCNTR} was changed to allow limits on the flux of plotted
           components, to allow plot symbols to scale with component
           flux, and to ignore flagged rows in the {\tt CC} table
           rather than dying.
\myitem{TV code} was changed to remember if the TV has been opened or
           not.  Previously, failing plot tasks could hang waiting for
           plot finishing on a TV display that was never going to
           respond.
\end{description}

\subsection{Image and $uv$ analysis}

\subsubsection{Faraday rotation synthesis}

\AIPS' Faraday rotation synthesis package has continued to be tested
by committed users and corrected by the \AIPS\ group.  The main {\tt
  FARS} task was changed to convert from real/imaginary to
amplitude/phase correctly, which was a particular problem when
residual images were output.  The Clean code was changed to do no
iterations at some pixels due to the value of {\tt FLUX} and to output
zero in those cases.  An option to shift the phases to the original
$\lambda^2$ set was added.  The handling of {\tt APARM(4)}  was
clarified and its display in the history file corrected.  The {\tt
  DOFARS} procedure was changed to use more appropriate default output
classes.

A new task, named {\tt TARS}, was created to test the algorithms used
by {\tt FARS}\@.  It is nearly identical to {\tt FARS} but takes its
input data from a text file which may be prepared more easily by
people wishing to test this new algorithm.

The output of {\tt FARS} may be compromised if the input Q and U
polarization cubes are affected by frequency-dependent amplitude
changes due to the source spectral index and to the primary beam of
the interferometer.  The new task {\tt SPCOR} was written to correct
the Q and U cubes for these effects.  One may use {\tt SPIXR} on a I
cube to determine images of spectral index that may be appropriate to
the Q and U images.  The I spectral index image output of MSMFS in
\CASA\ may also be used.

{\tt AFARS} is the task which reads the {\tt FARS} output images to
find the position on the rotation measure axis of the maximum
amplitude and then write images of the rotation measure and of the
amplitude or phase at that position.  It was corrected to account
properly for {\tt BLC}, to avoid the parabolic fit estimation when it
did not have all required values due to blanking and edge effects, to
use the header coordinates properly rather than guesses which could be
wrong, and to output the blanking value when the input rotation
measure row was all blank or zero.  The help file was improved to
clarify which files are required as inputs.

\vfill\eject
\subsubsection{$UV$-plane model fitting}

The monster task {\tt OMFIT}, written by Ketan Desai many years ago,
received some much overdue attention.  The task contains some very
high quality mathematical routines which implement a wide variety of
source model types and a joint model-fit and self-calibration.  It is
a bit difficult to use, but once convergent models are found, it can
be very useful.  A few rather simple corrections were required to get
it to work at all.  The input model is read by a {\tt KEYIN}-based
routine which had to be corrected to handle parameters longer than 8
characters.  The task was corrected so that it would work on
single-source input files.  Michael Bietenholz provided suggested code
changes to add a {\tt WEIGHTIT} parameter adjusting the data weights,
to normalize phase-only self-cals correctly, to display ``real''
chi-squared values, and to include a spherical shell model type.
Commentaries on how to use {\tt OMFIT} were also provided by
Michael for inclusion in the explain file.

The other $uv$-plane model fitting task {\tt UVFIT} also received some
attention.  It was changed to allow up to 60 components in the model.
All of the code in the task now supports this parametric upper limit
after the last of the hard-coded 4's were rooted out.  The computation
of phase now uses the proper geometric routines which are slower but
very much more accurate for components well separated from the phase
center.  The input text file describing the model may now include a
comment string which is included in the output summary text file.

\subsubsection{Other analysis changes}
\begin{description}
\myitem{COMB} was corrected to adjust the units of each of the input
           images so that units of Jy/beam become units of Jy per the
           same beam.  Previously two images in Jy per different beam
           were added, subtracted, divided, etc.~producing outputs
           which may have had more to do with the differences in beam
           than in source.
\myitem{DTSIM} is a $uv$-data simulation task which was corrected to
           use right-handed antenna geometries, to use correct and
           current-best antenna locations for known antennas (VLBA and
           EVLA), to use sensible and many more user-controlled
           defaults, to create an NX table, and to write the data in
           the standard axis order.  The help file was overhauled to
           tell the truth about things including whole areas
           previously omitted.  A useful example replaced an overly
           simple one.  This task may actually be useful.
\myitem{SAD} now uses a robust determination of the rms to inform the
           histogram fit of the rms, producing a much more reliable
           estimate of the overall uncertainty.
\end{description}

\subsection{General}

\subsubsection{Double-precision pseudo array processor}

When a four-byte integer is used to count, it will overflow after
$2^{31}-1$ or 2147483647.  When a four-byte, IEEE-format real variable
is used to sum 1.0's, it will stop increasing after 16777216.  With
the EVLA, bandwidth synthesis easily exceeds the latter limit and can
exceed the former.  \AIPS' pseudo-array processor (``AP'') code is a
software emulation of the old hardware array processors we used to
own.  It isolates computationally intensive operations in a modest
collection of ``Q'' routines based on a shared data space.  This
structure allowed us to change the entire pseudo-AP from
single-precision to double-precision real and integer for the shared
data and, with some editing, to complete the change so that all
variables inside the AP are extra precision.  Variables received from
and sent to the non-Q calling routines are normally still
single-precision, although new routines {\tt QDGET} and {\tt QDPUT}
were written to allow full-precision data transfers.

This has a number of consequences.  The most important is that the
number of words available in the pseudo-AP memory is halved, or the
memory in bytes required is doubled if the number of words is kept
constant.  This required some changes in the verb {\tt SETMAXAP}\@.
Tests a couple of years ago suggested that performance of a
double-precision AP was a about 10\%\ worse than a single-precision
version, presumably due to memory cache issues as much as anything.
The computed results, however, are more accurate even when
double-precision was not needed to avoid overflow.  Convolving
functions and their Fourier transforms are computed in the AP code and
so will be much more accurate in gridding $uv$ data and then
correcting in the image plane for that gridding.

\subsubsection{Other general-interest changes}

\begin{description}
\myitem{Lustre} is a high-performance file system used at NRAO and
           elsewhere.  Such systems do not keep information about file
           sizes with the files themselves and so can mis-inform
           \AIPS\ about the file size.  Low-level routines were
           changed to pause and try again twice before dying when it
           appears that a task wants to read or write beyond the end
           of file.
\myitem{Source} table format was updated to include columns for the
           observed (or pointing) right ascension and declination.
           The current RA and Dec columns are used for the phase
           stopping position, which may be quite different.
\myitem{Frequency} table format was changed to include a column for
           {\tt BANDCODE}, an 8-character string meant to identify
           which receiver was used for each IF for each frequency
           identifier.
\myitem{ZEXIT.C} is a new routine to allow a non-zero exit code to be
           returned to the calling procedure.  It is useful in
           stand-alone programs, especially the \AIPS\ pre-processor
           program {\tt PP}\@.  It replaces the non-standard Fortran
           system routine {\tt EXIT}\@.
\myitem{NAMEGET}\hspace{2em} is a new verb which may be useful in
           writing procedures.  It uses the first image name
           parameters ({\tt INNAME}, {\it et al.}) to find the image
           or data set that an \AIPS\ task would find with those name
           parameters and to fill in the remaining ones so that there
           are no defaulted or wild-carded ones.
\end{description}

\section{Patch Distribution for \OLDNAME}

Important bug fixes and selected improvements in \OLDNAME\ can be
downloaded via the Web beginning at:

\begin{center}
\vskip -10pt
{\tt http://www.aoc.nrao.edu/aips/patch.html}
\vskip -10pt
\end{center}

Alternatively one can use {\it anonymous} \ftp\ to the NRAO server
{\tt ftp.aoc.nrao.edu}.  Documentation about patches to a release is
placed on this site at {\tt pub/software/aips/}{\it release-name} and
the code is placed in suitable sub-directories below this.  As bugs in
\NEWNAME\ are found, they are simply corrected since \NEWNAME\ remains
under development.  Corrections and additions are made with a midnight
job rather than with manual patches.  Since we now have many binary
installations, the patch system has changed.  We now actually patch
the master version of \OLDNAME, which means that a MNJ run on
\OLDNAME\ after the patch will fetch the corrected code and/or
binaries rather than failing.  Also, installations of \OLDNAME\ after
the patch date will contain the corrected code.

The \OLDNAME\ release has had a number of important patches:
\begin{enumerate}
  \item\ Imaging code had problems with large gridded modeling.
      {\it 2012-01-16}
  \item\ {\tt UVFIT} had many areas dimensioned for the old limit of 4
      components, rather than the new limit of 20. {\it 2012-01-16}
  \item\ {\tt FARS} did not shift residuals back to the original
      $\lambda^2$ correctly and did not add residuals into
      amplitude/phase outputs correctly. {\it 2012-01-16}
  \item\ {\tt AFARS} did not handle {\tt BLC} correctly, producing
      wrong rotation measure images.  It also stated incorrect
      information in the history. {\it 2012-01-16}
  \item\ {\tt SQASH} had an error in history writing that led to
      aborts.  {\it 2012-01-16}
  \item\ {\tt FITAB} omitted an essential keyword for compressed UV
      data.  {\it 2012-01-16}
  \item\ {\tt CLCOR} ignored the fact that the GMRT uses an opposite
      phase convention. {\it 2012-01-16}
  \item\ {\tt VLANT} did not correct EVLA data properly. {\it
      2012-03-01}
  \item\ {\tt SCMAP} and {\tt SCIMG} did not apply the flag table to
      the data entering the self-cal step {\it 2012-03-07}
\end{enumerate}
\vfill\eject

\section{Recent \AIPS\ and related Memoranda}

All \AIPS\ Memoranda are available from the \AIPS\ home page.  There
is one new memoranda in the last six months.

\begin{tabular}{lp{5.8in}}
{\bf 117} & {\bf \AIPS\ FITS File Format}\\
   &  Eric W. Greisen, NRAO\\
   &  June 28 2012 (latest revision)\\
   &  \AIPS\ has been writing images and $uv$ data in FITS-format
  files for a very long time.  While these files have been used widely
  in the community, there is a perception that a detailed document in
  still required.  This memo is an attempt to meet that perception.
  \AIPS\ FITS files for $uv$ are conventions layered upon the standard
  FITS format to assist in the interchange of data recorded by
  interferometric telescopes, particularly by radio telescopes such as
  the EVLA and VLBA\@.  An Appendix providing a serious introduction
  to the subject of FITS format is also provided.
\end{tabular}

\section{\AIPS\ Distribution}

We are now able to log apparent MNJ accesses and downloads of the tar
balls.  We count these by unique IP address.  Since some systems
assign the same computer different IP addresses at different times,
this will be a bit of an over-estimate of actual sites/computers.
However, a single IP address is often used to provide \AIPS\ to a
number of computers, so these numbers are probably an under-estimate
of the number of computers running current versions of \AIPS\@. In
2012, there have been a total of 837 IP addresses so far that have
accessed the NRAO cvs master.  Each of these has at least installed
\AIPS\ and 281 appear to have run the MNJ on \RELEASENAME\ at
least occasionally.  During 2012 more than 194 IP addresses have
downloaded the frozen form of \OLDNAME, while more than 596 IP
addresses have downloaded \RELEASENAME\@.  The binary version was
accessed for installation or MNJs by 303 sites in \OLDNAME\ and 600
sites in \RELEASENAME\@.  All of these numbers are significantly lower
than those of 2010 at a comparable date.  The attached figure shows
the a comparison of the numbers from all of 2011 and from 2012 for the
first half of the year.  From bottom to top the numbers plotted are
the cumulative unique IP address to download the {\tt NEW} version, to
download the {\tt TST} version, to access the {\tt cvs} web site, and
the total unique addresses.  It is strange that the downloads in 2012
are if anything slightly ahead of 2011, while the cvs access and hence
the total numbers in 2012 are way behind those in 2011.  It makes one
suspect an as yet unidentified change in the way the {\tt cvs} numbers
are counted.

\vfill
\centerline{\resizebox{!}{3.95in}{\includegraphics{FIG/PLOTIT12a.PS}}}

\vfill\eject
\hphantom{.}
\vspace{2em}
\section{A personal note}
The rumor that your editor is about to retire is not true.  While I am
certainly old enough, I am having too much fun and believe that I am
still too useful to retire any time soon.  \AIPS\ will continue to be
supported for the forseeable future.

\section{Warning}
Recent $uv$ FITS files written by \CASA\ and containing ALMA data have
had significant errors in the contents of the antenna and frequency
tables and the frequency information in the main header.  The antenna
error will primarily affect the use of {\tt UVFIX} to recompute
projected baselines.  The frequency error will certainly affect
plotted spectra (reversed axis labeling) and will have a small effect
on computed images.  These errors are known to be in release 3.4.0;
FITS files written by earlier releases should be checked closely.
\CASA's {\tt exportuvfits} program is now being worked upon and these
issues are expected to be missing from the next release.  These errors
do not appear to arise with EVLA data as written, via \CASA, from the
archive.
\vfill
\eject
% mailer page
% \cleardoublepage
\pagestyle{empty}
 \vbox to 4.4in{
  \vspace{12pt}
%  \vfill
\centerline{\resizebox{!}{3.2in}{\includegraphics{FIG/Mandrill.eps}}}
%  \centerline{\rotatebox{-90}{\resizebox{!}{3.5in}{%
%  \includegraphics{FIG/Mandrill.color.plt}}}}
  \vspace{12pt}
  \centerline{{\huge \tt \AIPRELEASE}}
  \vspace{12pt}
  \vfill}
\phantom{...}
\centerline{\resizebox{!}{!}{\includegraphics{FIG/AIPSLETS.PS}}}

\end{document}
