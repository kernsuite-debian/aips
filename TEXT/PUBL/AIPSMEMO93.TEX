%-----------------------------------------------------------------------
%;  Copyright (C) 1997
%;  Associated Universities, Inc. Washington DC, USA.
%;
%;  This program is free software; you can redistribute it and/or
%;  modify it under the terms of the GNU General Public License as
%;  published by the Free Software Foundation; either version 2 of
%;  the License, or (at your option) any later version.
%;
%;  This program is distributed in the hope that it will be useful,
%;  but WITHOUT ANY WARRANTY; without even the implied warranty of
%;  MERCHANTABILITY or FITNESS FOR A PARTICULAR PURPOSE.  See the
%;  GNU General Public License for more details.
%;
%;  You should have received a copy of the GNU General Public
%;  License along with this program; if not, write to the Free
%;  Software Foundation, Inc., 675 Massachusetts Ave, Cambridge,
%;  MA 02139, USA.
%;
%;  Correspondence concerning AIPS should be addressed as follows:
%;         Internet email: aipsmail@nrao.edu.
%;         Postal address: AIPS Project Office
%;                         National Radio Astronomy Observatory
%;                         520 Edgemont Road
%;                         Charlottesville, VA 22903-2475 USA
%-----------------------------------------------------------------------
\documentstyle [twoside]{article}
%
\newcommand{\AIPS}{{$\cal AIPS\/$}}
\newcommand{\AMark}{AIPSMark$^{(93)}$}
\newcommand{\AMarks}{AIPSMarks$^{(93)}$}
\newcommand{\AM}{A_m^{(93)}}
%
\newcommand{\whatmem}{\AIPS\ Memo \memnum}
\newcommand{\boxit}[3]{\vbox{\hrule height#1\hbox{\vrule width#1\kern#2%
\vbox{\kern#2{#3}\kern#2}\kern#2\vrule width#1}\hrule height#1}}
\newcommand{\memnum}{93}
\newcommand{\memtit}{Position Angle of the VSOP Antenna Feed.}
%
\setlength{\topmargin}{-8mm}
\setlength{\textwidth}{160mm}
\setlength{\textheight}{220mm}
\setlength{\oddsidemargin}{0mm}
%\setlength{\evensidemargin}{45mm}
\setlength{\evensidemargin}{0mm}
\setlength{\parindent}{3em}
\setlength{\unitlength}{10mm}
\input{epsf.sty}
\newcommand{\nl}{\newline}

\title{
%   \hphantom{Hello World} \\
   \vskip -35pt
%   \fbox{AIPS Memo \memnum} \\
   \fbox{{\large\whatmem}} \\
   \vskip 28pt
   \memtit \\}
%\title{Position angle of the VSOP antenna feed.}
\vspace{2 mm}

\author{ L.~Kogan
\vspace{2 mm}\\
\small National Radio Astronomy Observatory, Socorro, New Mexico,
USA\\}
%\date{September~ 5,~1993}
\vspace{2mm}

\begin{document}
\maketitle
\vspace{5mm}
\section{Introduction}
Knowledge of the position angle of an antenna feed is very important for polarization observations. The issue has been developed for ground based antennas but has not been considered for a orbiting antennas. It is clear that the variation  of the position angle depends on how the orbiting antenna is pointed at the source i.e. from the construction of the mounting device. In this memo we provide a calculation of the position angle of the Japanese satellite VSOP which will be launched soon in a space VLBI mission.
\section{Position angle analysis for VSOP}
The VSOP satellite construction, including antenna and solar panels, is shown in Fig. \ref{fig:1}.   We define a right hand coordinate system fixed on the satellite body ($X_1, Y_1, Z_1$). Axis $Z_1$ points towards the source being observed and  $Y_1$ is the axis about which the solar panels can be rotated. The solar panels are pointed at the Sun by rotation of the pannels about axis $Y_1$ and rotation of the whole satellite about axis $Z_1$. If $\theta$ is the angle between the source and the Sun, there are two possible orientations of the satellite and the solar panels. They differ in that the panels turn by an angle $2 \cdot \theta$ about axis $Y_1$ and the satellite as a whole by $180^{\circ}$ about axis $Z_1$ (Fig. \ref{fig:1}).
\footnote {Information about the VSOP antenna and solar panels was provided by David Murphy (JPL)} So the difference in the aperture plane is  $180^\circ$, which is fortunately not important for polarization analysis. From the construction of the pointing system it is clear that the projection of the vector directed to the Sun on the antenna aperture is fixed in the coordinate system ($X_1, Y_1, Z_1$).
It is constant for all sources observed and for all times. This projection determines the reference point on the aperture $R_a$ (Fig. \ref{fig:2}). This important statement allows us to determine the feed position angle relative to north by the formula:

\begin{equation}
    PA = \widehat{ \vec{s}_a \cdot \vec{n}_a } + PA_{\circ}
\label{eq:pa}
\end{equation}
\begin{tabbing}
where~ \= $\widehat{ \vec {s}_a \cdot \vec{n}_a }$ is the angle between vectors $\vec {s}_a$ and $\vec{n}_a$;\\
       \> $\vec {s}_a$ is the projection of the vector in the Sun direction on the antenna aperture;\\
       \> $\vec{n}_a$ is the projection of the vector in the north direction on the antenna aperture;\\
       \>  $PA_{\circ}$ \= is the angle in the antenna aperture between the reference vector $\vec {s}_a$ \\
       \> \> and major axis of the feed polarization ellipse;
\end{tabbing}
Vector $\vec {s}_a$ can be determined from the following formula:
\begin{equation}
  \vec {s}_a = \vec{s} - \vec{e}\, (\vec{e} \cdot \vec{s})
\label{eq:sa}
\end{equation}
\begin{tabbing}
where~ \= $\vec{s}$ is the unit vector in the Sun direction;\\
       \> $\vec{e}$ is the unit vector in the source direction;\\
       \> $(\vec{e} \cdot \vec{n})$ is the scalar product of the vectors.\\
\end{tabbing}
Vectors $\vec{s}$ and $\vec{e}$ have the following form in the equatorial coordinate system:
\begin{eqnarray}
\vec{s} & = & \{ \cos(\alpha_s) \cos(\delta_s),\:
                \sin(\alpha_s) \cos(\delta_s),\:
                \sin(\delta_s)\}
\label{eq:s}
\\
\vec{e} & = & \{ \cos(\alpha) \cos(\delta),\:
                \sin(\alpha) \cos(\delta),\:
                \sin(\delta)\}
\label{eq:e}
\end{eqnarray}
\begin{tabbing}
where~ \= $\alpha_s,\:\delta_s$ are right ascension and declination
            of the Sun\\
       \> $\alpha,\:\delta$ are right ascension and declination
            of the source\\
\end{tabbing}
To find the angle $\widehat{ \vec {s}_a \cdot \vec{n}_a }$ we need to determine the direction to the north and east in the aperture. Having found the projection of the reference vector $\vec {s}_a$ in these directions, we will be able to find both the absolute value of the angle $\widehat{ \vec {s}_a \cdot \vec{n}_a }$ and its sign.
The unit vectors $\vec{v}$ and $\vec{u}$ are known (See \cite{kn:mor} for example) and correspond to the north and east direction respectively:
\begin{eqnarray}
\vec{v} & = & \{ -\cos(\alpha) \sin(\delta),\:
                -\sin(\alpha) \sin(\delta), \:
                \cos(\delta)\}
\label{eq:v}
\\
\vec{u} & = & \{ -\sin(\alpha),\:  \cos(\alpha), \:  0\}
\label{eq:u}
\end{eqnarray}
Using equations (\ref{eq:sa}, \ref{eq:s}, \ref{eq:v}, \ref{eq:u}) we can find the expression for the required projections:
\begin{eqnarray}
(\vec {s}_a\cdot\vec{n}_a) =  (\vec {s}_a\cdot\vec{v}) =
(\vec{s}\cdot\vec{v})
& = & -\cos(\delta_s) \, \sin(\delta)\, \cos(\alpha_s-\alpha) +
 \sin(\delta_s) \, \cos(\delta)
\label{eq:sv}
\\
(\vec {s}_a\cdot\vec{u}) = (\vec{s}\cdot\vec{u})
& = & \cos(\delta_s) \, \sin(\alpha_s-\alpha)
\label{eq:su}
\end{eqnarray}
Now finally the position angle measured from north to east can be found using the following expression:
\begin{equation}
    PA = \arctan \frac
{\cos(\delta_s) \, \sin(\alpha_s-\alpha)}
{-\cos(\delta_s) \, \sin(\delta)\, \cos(\alpha_s-\alpha) +
 \sin(\delta_s) \, \cos(\delta)} + PA_{\circ}
\label{eq:paf}
\end{equation}

\section{Discussion}
Using equations (\ref{eq:paf}), we have evaluated the position angle of the VSOP antenna feed for different source positions. For the ground-based antennas everything repeats over 24 hours and for sources with different right ascension  shifted in time by the right ascensions difference. The situation is different for orbiting VLBI antennas. First of all, the typical time period is not 1 day  but 1 year. Secondly nothing is repeatable for different right ascensions. That is why we need to provide the calculation for a range of both right ascension and declination. We can limit the right ascension range to $(0, 180^\circ)$ because for the other half of the range the solution is repeated with opposite sign of the declination.
The time range can be limited to a half year because the Sun's right ascension and declination for the second half of the year differs by $180^\circ$ and sign respectively. Such a difference provides the $180^\circ$ position angle change which is not important as discussed above. Examples of the position angle calculation are shown in Figs. \ref{fig:3} through \ref{fig:11}. There are several special cases:\\
~~\underline {The source is located at the ecliptic plane.}\\
We can put the axis $Y_1$ (axis of rotation of the solar panels) perpendicular to the ecliptic plane. VSOP can point the solar panels towards the Sun by rotation of the panels about the axis $Y_1$ without rotation of the satellite about the antenna axis $Z_1$. So the position angle is constant (the constant is between $66.4^\circ $ and $90^\circ $) throughout the year. This case is illustrated by  (Fig. \ref{fig:3}) for the source with both right ascension and declination equal to zero.\\
~~\underline {The Sun direction is close to the source (or to the opposite) direction.}\\
In this case the pointing conditions to the both Sun and the source can be satisfied for any rotation of the satellite about the antenna axis. So the position angle is ambiguous. Such a case occurs near $\alpha=160^\circ, \delta=10^\circ$ around day equal 57 (Fig. \ref{fig:4}).\\
%\setcounter{figure}{2}


\def\plotone#1{\begin{picture}(15,10)(-5, -0.5)
                  \centering \leavevmode
                  \epsfysize=10\unitlength \epsfbox{#1}
               \end{picture}}

\begin{figure}
 \plotone {vsopfig1.ps}
\caption{The two possible orientations of the satelite(VSOP) and the solar pannels.}
                  \label{fig:1}
\end{figure}

\def\plotone#1{\begin{picture}(15,10)(-2, -0.5)
                  \centering \leavevmode
                  \epsfysize=6\unitlength \epsfbox{#1}
               \end{picture}}

\begin{figure}
 \plotone {vsopfig2.ps}
\caption{Vector analysis of position angle of VSOP antenna feed.}
                  \label{fig:2}
\end{figure}

\def\plotone#1{\begin{picture}(15,10)(-3, 0.5)
                  \centering \leavevmode
                  \epsfysize=8\unitlength \epsfbox{#1}
               \end{picture}}

\begin{figure}
 \plotone {vsopfig3.ps}
\caption{Plots of the position angle vs time for DEC=0 degrees.}
                  \label{fig:3}
\end{figure}

\begin{figure}
 \plotone {vsopfig4.ps}
\caption{Plots of the position angle vs time for DEC=10 degrees.}
                  \label{fig:4}
\end{figure}

\begin{figure}
 \plotone {vsopfig5.ps}
\caption{Plots of the position angle vs time for DEC=30 degrees.}
                  \label{fig:5}
\end{figure}

\begin{figure}
 \plotone {vsopfig6.ps}
\caption{Plots of the position angle vs time for DEC=50 degrees.}
                  \label{fig:6}
\end{figure}

\begin{figure}
 \plotone {vsopfig7.ps}
\caption{Plots of the position angle vs time for DEC=70 degrees.}
                  \label{fig:7}
\end{figure}

\begin{figure}
 \plotone {vsopfig8.ps}
\caption{Plots of the position angle vs time for DEC=-70 degrees.}
                  \label{fig:8}
\end{figure}

\begin{figure}
 \plotone {vsopfig9.ps}
\caption{Plots of the position angle vs time for DEC=-50 degrees.}
                  \label{fig:9}
\end{figure}

\begin{figure}
 \plotone {vsopfig10.ps}
\caption{Plots of the position angle vs time for DEC=-30 degrees.}
                  \label{fig:10}
\end{figure}

\begin{figure}
 \plotone {vsopfig11.ps}
\caption{Plots of the position angle vs time for DEC=-10 degrees.}
                  \label{fig:11}
\end{figure}


\begin{thebibliography}{99}
  \bibitem{kn:mor} A.R. Thompson, J.M. Moran, and G.W. Swenson, Interferometry and Synthesis in Radio Astronomy. A Wiley-Interscience Publication, 1986.
\end{thebibliography}
\end{document}








