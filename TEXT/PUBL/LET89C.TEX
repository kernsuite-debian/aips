%-----------------------------------------------------------------------
%;  Copyright (C) 1995
%;  Associated Universities, Inc. Washington DC, USA.
%;
%;  This program is free software; you can redistribute it and/or
%;  modify it under the terms of the GNU General Public License as
%;  published by the Free Software Foundation; either version 2 of
%;  the License, or (at your option) any later version.
%;
%;  This program is distributed in the hope that it will be useful,
%;  but WITHOUT ANY WARRANTY; without even the implied warranty of
%;  MERCHANTABILITY or FITNESS FOR A PARTICULAR PURPOSE.  See the
%;  GNU General Public License for more details.
%;
%;  You should have received a copy of the GNU General Public
%;  License along with this program; if not, write to the Free
%;  Software Foundation, Inc., 675 Massachusetts Ave, Cambridge,
%;  MA 02139, USA.
%;
%;  Correspondence concerning AIPS should be addressed as follows:
%;          Internet email: aipsmail@nrao.edu.
%;          Postal address: AIPS Project Office
%;                          National Radio Astronomy Observatory
%;                          520 Edgemont Road
%;                          Charlottesville, VA 22903-2475 USA
%-----------------------------------------------------------------------
%\input docpubl:al82.mac
\input al82.mac
\input docpubl:al8pt.mac
\letterbegin {IX} {3} {July 15, 1989}

\subtitf{\AIPS\ Release Cancelled}

    The {\tt 15JUL89} version of \AIPS\ is the first version of the
``overhauled'' code and also contains some major changes to the
calibration and imaging software.  It has been extensively tested at
the \hbox{NRAO}.  However, it has recently been found to contain some
serious impediments to {\it filling} or {\it calibrating} certain
classes of spectral-line data obtained with the \hbox{VLA}. We have,
therefore, decided not to release this version of \AIPS\ outside the
\hbox{NRAO}.

     We now plan that the next ``public'' release of \AIPS, and thus
the first of the overhauled code, will be the {\tt 15OCT89} version.
We regret any inconvenience to users that is caused by not releasing
the {\tt 15JUL89} version.  We hope that the \AIPS\ community will
recognize the benefit of waiting for a release that has fewer known
deficiencies.

    Until spectral-line calibration alternatives are debugged in \AIPS,
VLA spectral-line users from outside the NRAO should plan to visit the
Array Operations Center to calibrate their data using ISIS (which emulates
the functionality of the VLA's ``Pipeline'' at the speed of the Convex C-1).
The NRAO will {\it not} distribute or support the ISIS package outside
the observatory, however.  Please do {\it not} contact Miller Goss for
copies of \hbox{ISIS}.

    A need for an \AIPS\ ``user agreement'' has arisen for several
reasons.  The most important is that we want all \AIPS\ sites to
obtain their copies of \AIPS\ from the NRAO and thereby to be made aware
of the restrictions that apply to their use of the code and to our
support of it.  For sites doing astronomical research, these
restrictions are only to maintain the proprietary nature of the code
and to direct third parties who wish to receive the code to the
\hbox{NRAO}.   We have decided to require such an agreement with
each recipient of the \AIPS\ code; it will be necessary for each
site to complete and return a brief agreement form, signed by an
authorized representative, before receiving its next release of
\AIPS.  There will be no charge for the user agreement to any
astronomical research users.

     If you have any questions about the above, please contact Bob
Burns  at any of the addresses in the letterhead (use {\tt bburns}
rather than {\tt aipsmail}).

\eject

\subtit{Computer Positions Available}

\noindent{\bf Green Bank:}

     The NRAO has an immediate opening for a Scientific Programming
Analyst. The successful applicant will participate in an
Observatory-wide effort to modernize its single-dish spectral line,
reduction and analysis software.  The NRAO is in the process of
porting its existing programs to the SUN workstation environment.
Using these workstations, we will embark on an extensive software
development program in support of the new Green Bank telescope and
other existing instruments.

     Candidates should have experience with the Unix operating system,
C and FORTRAN programming languages and have experience in a scientific
programming environment.  An advanced degree or demonstrated equivalent
experience is also required.  Familiarity with advanced workstation
concepts, networking and radio astronomical data reduction techniques
would be attractive.

     The successful candidate will be based in Green Bank, West
Virginia, where the NRAO presently maintains a major observing
facility and where the world's largest steerable radio telescope
will be constructed.

\vskip\parsdef
\vfill
\centerline{**********************************************}
\vfill

\noindent{\bf Charlottesville:}

     The NRAO is seeking an individual with an M.A. or Ph.D. in
astronomy, preferably in radio astronomical spectroscopy, with interest
and experience in applications programming.  In addition to a knowledge
of C and FORTRAN, the candidate should be proficient in the Unix operating
system and have expertise in one or more of the following areas:

\item{$\bullet$}Algorithm development for calibration/reduction/analysis

\item{$\bullet$}Software system design

\item{$\bullet$}Computer graphics/modeling

     The NRAO will embark on an extensive software development
program in support of the new Green Bank telescope and other existing
instruments. The successful applicant will be involved in this
effort.  In addition, the job will carry responsibility for
maintenance of the software systems in place at the Charlottesville,
VA site; implementation of the new single-dish FITS data interchange
standard; and development of code to support observations done with
a new generation of multi-feed receivers and advanced spectrometers.

     If the successful candidate has an active research program in radio
astronomy or related astronomical areas, he/she will be encouraged to
devote up to 25\%\ of his/her time in self-directed research.  Research
that relates to radio astronomical spectroscopy will be particularly
encouraged.  This is not a tenure-track position and is not subject to
the NRAO tenure ``clock.''  Advancement to the rank of Associate Scientist
or Scientist can be made on the basis of merit and professional
development at any time.  If the successful candidate does not wish to
pursue an ongoing research program, an appointment could be made as a
Scientific Programming Analyst.

     The successful candidate will be based at the NRAO Central Office
on the grounds of the University of Virginia in Charlottesville.  This
facility has a staff of 28 Ph.D. astronomers, engineers, and computer
professionals with responsibility to all observing sites.

\vskip\parsdef
\vfill
\centerline{**********************************************}
\eject

\noindent{\bf Tucson:}

     The NRAO has an immediate opening for a Scientific Programming
Analyst. The successful candidate will assist in the development,
coding, testing and implementation of a real time system to acquire
data and control the operation of radio telescopes.

     The initial appointment will be in Tucson, Arizona, where
he/she will join a team currently implementing a new control system
for the 12-meter millimeter-wave telescope on Kitt Peak.

     After one year, the individual will transfer to the Observatory's
Green Bank, West Virginia site to work on the GBT project.  This is
a project to build the world's largest steerable radio telescope.
The individual will have the opportunity to play a key role in the
design and development of this system.

     The position requires at least a Bachelor's degree in computer
science, astronomy, or physics and either several years of general
programming experience, or at least one year of real-time experience.
A working knowledge of the UNIX, VMS or similar programming
environment, and an expert knowledge of the programming language
'C' is desired.

\vskip\parsdef
\vskip\parsdef

\centerline{**********************************************}

\vskip\parsdef

     The NRAO is an Equal Opportunity/Affirmative Action employer with
an excellent fringe benefit program, including medical, dental, and
long-term disability insurance, and a fully vested retirement plan.

     Candidates for these positions should send curriculum vitae,
bibliography, and a letter outlining research interests, salary history,
and requirements to:

\vskip 8pt
\hbox to \hsize{\hfil\vbox{\halign{\ctr{#}\cr
Dr. W. R. Burns\cr
Head, Computing\cr
National Radio Astronomy Observatory\cr
Edgemont Road\cr
Charlottesville, Virginia 22903-2475\cr}}\hfil}

\vfill\eject
\subtit{Summary of Changes:  15 April 1989 --- 15 July 1989}

     There are 219 entries in the {\tt CHANGE.DOC} file listed
below.  This is not a record, but certainly indicates a great deal
of activity --- both new code and bug fixes --- as the code
overhaul winds to a close.  The collaboration between Bill Cotton,
Phil Diamond, and our \AIPS\ friends in Australia has been
particularly productive.

\smallhead{Changes of Interest to Users: 15OCT89}

     Of all the new tasks in {\tt 15OCT89}, the one that is
likely to have the most impact is {\tt MK3IN}.  It reads raw
correlator MKIII VLBI data from Haystack format ``A'' tapes
and converts them to \AIPS\ format in an astronomically useful
time system.  This makes all \AIPS\ calibration available to
MKIII data for the first time (see entry \#s 5083, 5114).
Not coincidentally, the fringe fitting in {\tt CALIB} was
corrected for multiple-IF data (\#s 5081, 5082).

     The calibration area, as usual, received the lion's share
of the coding effort.  The most extensive change was a redefinition
of the multi-source \uv\ data base to allow more than one frequency
and bandwidth, within some limits.  This requires considerable
generalization in {\tt FILLM} to read the data (entry \# 5057).
All calibration tasks must then read the new frequency ({\tt FQ})
tables which describe the more general format and
must provide three new adverbs, {\tt SELBAND}, {\tt SELFREQ},
and {\tt FREQID}, to allow the user to select a subset of the
data base for the current task execution.  (See entry \#s 5020,
5058--5062, and 5143).  We had hoped to make the transition to
{\tt FQ} tables transparent to users, doing translations of old
{\tt CH} tables automatically, but there have been the inevitable
problems anyway.  Another overall change to the calibration was
the addition of support for linear polarization of the form
observed at the Australia Telescope (not that of the WSRT,
entry \# 5198).

New tasks in the calibration area include {\tt QUACK} which does selective
editing of data at the start and end of observation scans (\# 5166).  {\tt
FARAD} is another new task to apply Faraday rotation measure corrections to the
data (\# 5210).  New service tasks are {\tt SETAN} to translate a text file to
an \AIPS\ antenna file (\# 5230) and {\tt UNCAL} to remove on-line calibrations
(presently only from data taken by the Australia Telescope, \# 5235).  The
adverb {\tt OPTYPE} was added to {\tt SETJY} to allow for additional operations
such as computing standard fluxes (\# 5089) and resetting fluxes and velocities
(\# 5109).  The option to plot the difference between the two circular
polarizations was added to {\tt SNPLT} (\# 5022).  {\tt TVFLG} was
changed to accept free-format input for the numeric parameters which
it requires (\# 5079).

     Spectral-line calibration also received some direct attention.
The new adverb {\tt CHANSEL} is used to select ranges of channel
numbers for an operation (\# 5167).  It was added to {\tt SPLIT}
to control frequency averaging (\#s 5170, 5179, 5190) and to {\tt BPASS}
to control the calculation of ``channel 0'' (\# 5203).  The new task
{\tt AVSPC} uses this adverb to produce pseudo-continuum data sets
by summing selected spectral channels (\#s 5181, 5193).  {\tt BPASS}
was also modified to use the same channels as used by the on-line system
for ``channel 0'' (\# 5089), to recognize polarization data, to print
closure errors,  and to do amp-scalar averaging prior to determining
the bandpasses (\# 5203), as well as to support the new {\tt FQ}
tables.

     The display area also received some interesting improvements.
The new task {\tt UVHGM} makes plots of the statistics of values in
a \uv\ data file (\# 5172).  New task {\tt KNTR} plots contours in an
\AIPS\ plot file, including the option to display more than one image
in a single plot (\# 5171).  It is similar to {\tt CNTR} in some ways,
but uses an algorithm better suited to pen plotters (and fine for laser
printers).  Spectral-line plots for a specified region may be generated
by the new task {\tt ISPEC} as well (\# 5238).  Task {\tt GREYS} was
modified to label the intensity step wedge (\# 5021).  {\tt EXTLIST}
was improved and upgraded in a number of ways, including modifications
to the plotting tasks designed to improve the {\tt EXTLIST} display
(\#s 5147, 5021, 5126, 5131, 5135).  In particular, new plot types are
supported and the displays include the actual antennas or baselines for
the given plot file for those tasks which produce more than one plot
per execution.

     The area of tapes also received some attention.  {\tt BAKLD} and
{\tt BAKTP} were overhauled and made more usable (\# 5154).  A serious
bug which made all tapes of compressed \uv\ data unusable was corrected
(\# 5216),  and a pitfall for unwary users was removed (\# 5227), in
\hbox{{\tt FITTP}}.  A default was provided for the {\tt NPOINTS} adverb
(\# 5055) in \hbox{{\tt UVLOD}}.  A bug affecting the positioning of
tapes while being opened on Convex computers was corrected (\# 5095).

     Other areas of general interest include the new task {\tt FETCH}
(\# 5173).  It provides another, straightforward way to bring images
into the \AIPS\ system.  The use of the ``\AIPS\ number'' was changed
to allow there to be more interactive {\tt AIPS} users while still
allowing for \AIPS\ batch queues (\# 5050).  The help file {\tt WHATSNEW}
has been resurrected and should be of use (\#s 5178, 5182)..

\smallhead{Changes of Interest to Programmers: 15OCT89}

     Programmers interested in the calibration area should study the
wide-reaching changes brought about with the {\tt FQ} tables (\#s 5020,
5058--5062, and 5143).  All calibration tasks require three new
adverbs, {\tt SELBAND}, {\tt SELFREQ}, and {\tt FREQID}.  Several
basic subroutines have had their call sequences changed; these include
{\tt CHNDAT}, {\tt FRQTAB}, {\tt CHNCOP}, {\tt TABCAL}, {\tt TABSN},
{\tt TABBP}, and {\tt TABNDX}.  Several table files have had an extra
column added, namely the {\tt CL}, {\tt SN}, {\tt BP} and {\tt NX}
tables.  Other useful tools in the \uv-calibration area are
{\tt SETSTK}, to translate the user's {\tt STOKES} value (\# 5128, 5129),
{\tt REQBAS}, to translate the user's {\tt ANTENNAS} and {\tt BASELINE}
adverb values (\# 5043), and {\tt WANTCH} and {\tt AVGCHN}, to
handle the new {\tt CHANSEL} adverb (\#s 5168, 5169).

     The format of help files has also changed.  They begin with a
precursor section which gives a one-line description and keyword values
as well as separator lines.  Any line beginning with a semi-colon is now
treated as a comment line in help files and is ignored (\# 5115).  The
maximum number of interactive \AIPS\ allowed is now 15 if there is no
batch and 14 minus the number of batch queues if there is batch (\# 5050).
The system manager can set the limit lower than this.  All service
programs now have free-format inputs (\# 5049).  This makes a serious
change in the format of the main input line to {\tt POPSGN}, for example.
Gripes may be sent to Charlottesville on tape using {\tt GRITP} or
turned into text form for e-mail using {\tt GR2TEX}.  Both of these
were cleaned up and improved over the quarter (\#s 5047, 5048).

     The Z-routine areas of the code were simplified by dropping
obsolete operating systems and distinctions.  The Unix area is
now divided into a Bell area, with Mascomp and Cray sub-areas, and
a Berkeley area, with Alliant, Convex, Sun, and Vax sub-areas (\# 5148,
5150, 5151).  The midnight job was generalized to have separate
``lastgood'' dates for the separate portions of the job (\# 5153).
And, the VMS system translation of logicals was extended (\# 5045) and the
printing of a {\tt NOTICE.TXT} file was added to the {\tt AIPS} procedure
(\# 5228).  The TV is now a logical rather than a symbol for \hbox{VMS}.

\vfill\eject
\subtit{CHANGE.DOC: {\tt 15OCT89} Version as {\tt TST}}
\parindent 0pt\smallpt

\cdpar {5020.} {May 3, 1989} {SELBAND, SELFREQ, FREQID} {Phil}
   {\hp {Added to {\tt POPSDAT.HLP}, 3 new adverbs needed for the
        {\tt FQ} table stuff.  They (respectively) select data by
        bandwidth, frequency and {\tt FQ} id number.  Also help
        files for each.  Also added the new adverbs to {\tt DAPL.INC}
        and {\tt NEWPARMS.001}.}
    \hp {Moved nowhere.}}

\cdpar {5021.} {May 5, 1989} {GREYS} {Eric}
   {\hp {Changed the handling of the step wedge so that it is plotted
        at the top or the right depending on \hbox{{\tt DOWEDGE}}.
        If {\tt LTYPE} is $\ge 3$, the wedge is given ticks with
        intensity labels.  The intensity range line at the bottom
        was retained, but changed to use metric scaling where possible.
        Cleaned up typing a bit and filled in more of the defaults
        before the plot file is created.  This matters for {\tt EXTLIST}.
        Added notes about the new capability to the help file.}
    \hp {Moved to {\tt 15JUL89} this date.}}

\cdpar {5022.} {May 8, 1989} {SNPLT} {Chris}
   {\hp {Added a new {\tt STOKES} option. If {\tt STOKES='DIFF'},
        then {\tt SNPLT} plots the differences between RCP and LCP
        solutions --- a useful diagnostic for polarization work.
        Changed the help file to match.}
    \hp {Moved nowhere.}}

\cdpar {5023.} {May 8, 1989} {PRTCC} {Chris}
   {\hp {Corrected implied do-loop ranges in output statements for
        non-point components. This prevents {\tt PRTCC} from crashing when
        listing {\tt CC} files with Gaussian components. Also tidied up the
        header for $101 \le {\tt DOCRT} < 120$.}
    \hp {Moved to {\tt 15JUL89} this date.}}

\cdpar {5024.} {May 8, 1989} {CONVL} {Chris}
   {\hp {Fixed a logic error in subroutine \hbox{{\tt CONVOL}}.
        Although {\tt CONVOL} was properly checking for fatal error
        returns (1--3) from {\tt CATIO}, it was not resetting
        {\tt IERR} if a non-fatal error occurred; any non-zero
        error status was caught by a later error trap giving rise to
        a misleading error message.  As {\tt CATIO} normally returns
        6 (`Warning on READ, file writing') when {\tt CONVOL} is
        called for the second plane (and all subsequent planes) of a
        cube, this bug prevented {\tt CONVL} from handling data cubes.}
    \hp {Moved to {\tt 15JUL89} this date.}}

\cdpar {5025.} {May 9, 1989} {ZI8IL} {Bill C}
   {\hp {Fixed indexing problem; it was picking up input at an address
        1 byte higher than correct.  This should have only affected
        {\tt IMLOD} reading 8-bit images.}
    \hp {Moved to {\tt 15JUL89} this date.}}

\cdpar {5026.} {May 9, 1989} {METSCA} {Eric}
   {\hp {Changed to capitalize the metric prefixes rather than put them
        all in uppercase.}
    \hp {Moved to {\tt 15JUL89} this date.}}

\cdpar {5027.} {May 9, 1989} {MOMNT} {Chris}
   {\hp {Split two history file entries that were longer than 72
        characters into two lines. The long lines caused {\tt MOMNT}
        to crash on the Convex when it tried to squeeze them into
        a 72-character buffer. Also made the `This is a very
        non-standard program...' message lowercase.}
    \hp {Moved to {\tt 15JUL89} this date.}}

\cdpar {5028.} {May 9, 1989} {DUVH.INC, DSEL.INC, PUVD.INC} {Phil}
   {\hp {Modified to cope with the upcoming {\tt FQ} tables.}
    \hp {Moved nowhere.}}

\cdpar {5029.} {May 9, 1989} {DBCON} {Phil}
   {\hp {Modified so that common {\tt /CATHDR/} matched the necessary common
        in {\tt DUVH.INC}.}
    \hp {Moved nowhere.}}

\cdpar {5030.} {May 9, 1989} {SUMIM} {Chris}
   {\hp {In the loop to sum over images, {\tt H2WAWA} was being called
        with an {\tt INTEGER} sequence number rather than a
        \hbox{{\tt REAL}}.  As small integers look like zero on the
        Convex, when interpreted as floating-point numbers, the WaWa
        I/O system set the input file sequence number to zero in each
        iteration.  This had the result that the file with the highest
        sequence number that had the given input name, class and disk
        number was repeatedly added to itself. The {\tt H2WAWA} call
        was replaced by a call to {\tt A2WAWA}, which also allowed the
        removal of several superfluous {\tt CHARACTER}--{\tt HOLLERITH}
        conversions.}
    \hp {Moved to {\tt 15JUL89} this date.}}

\cdpar {5031.} {May 10, 1989} {MX} {Phil}
   {\hp {The maximum size of the beam was being calculated incorrectly
        in subroutine \hbox{{\tt MXIN}}. This only showed up because
        the AP broke and we changed the maximum size of the AP to
        32K words; the resultant beam calculation produced a number
        that was not a power of two.}
    \hp {Moved to {\tt 15JUL89} this date}}

\cdpar {5032.} {May 10, 1989} {POLISH} {Eric}
   {\hp {Corrected bug which prevented {\tt ENDEDIT} from being executed.}
    \hp {Moved to {\tt 15JUL89} this date.}}

\cdpar {5033.} {May 11, 1989} {Times in text tables} {Bill C}
   {\hp {Added a new keyword, {\tt TIMEOFF}, to the tables read by
        {\tt ANCAL} and \hbox{{\tt UVFLG}}.  This allows the entry of
        times in UT (or other times) and gives a constant value to
        correct to \hbox{IAT}.  This is needed for VLBI and VLBA use where
        the times are usually kept in \hbox{UTC}.  {\tt ANCAL} requires a
        value per antenna; {\tt UVFLG} keeps the value until a new
        one is encountered.  Also {\tt ANCAL.HLP} and {\tt UVFLG.HLP}.}
    \hp {Moved nowhere.}}

\cdpar {5034.} {May 11, 1989} {SL2PL} {Chris}
   {\hp {Added code to check that a model exists before plotting model
        or residuals. Also corrected a bug in subroutine {\tt SLABEL} that
        wiped out part of the plot title (including the image file
        name).}
    \hp {Moved to {\tt 15JUL89} this date.}}

\cdpar {5035.} {May 11, 1989} {UVFND} {Chris}
   {\hp {Corrected the conversion of days into hours, minutes, and seconds.}
    \hp {Moved to {\tt 15JUL89} this date.}}

\cdpar {5036.} {May 11, 1989} {TELL} {Eric}
   {\hp {Corrected {\tt GTTELL} and \hbox{{\tt AU2}}.  Both the display of
        the {\tt SHOW} and {\tt TELL} queue and the {\tt TELL}
        parameters were failing due to errors addressing the changed
        format of the file.  Created missing {\tt OPTELL} help file.}
    \hp {Moved to {\tt 15JUL89} this date.}}

\cdpar {5037.} {May 11, 1989} {Miscellaneous} {Eric}
   {\hp {Changed {\tt ZTQSPY} ({\tt GEN}) and {\tt ZTQSP2} ({\tt VMS})
        to upper-lowercase headers.}
    \hp {Moved to {\tt 15JUL89} this date.}}

\cdpar {5038.} {May 12, 1989} {PRTTP} {Chris}
   {\hp {Fixed the Fortran-77 character handing concerned with printing
        image file names. The file type, class and name had been
        concatenated into a single {\tt CHARACTER} variable, apparently on
        the assumption that a {\tt WRITE} statement could split this variable
        between 3 edit descriptors.  Even if this were possible, the
        end index of the type substring was missing from the call
        which filled it, causing the file name and class to be
        overwritten with blanks.}
    \hp {Moved to {\tt 15JUL89} this date.}}

\cdpar {5039.} {May 12, 1989} {ZMOUN2 (VMS)} {Eric}
   {\hp {Corrected error in {\tt ZMOUN2} which caused it to fail to detect
        previously mounted tapes under VMS.}
    \hp {Moved to {\tt 15JUL89} this date.}}

\cdpar {5040.} {May 15, 1989} {COMB} {Chris}
   {\hp {Fixed a bad call to {\tt H2CHR} that was causing {\tt COMB}
        to bomb with {\tt OPCODE='SPIX'}.  Also added a trap for input
        maps at the same frequency when forming spectral-index maps.}
    \hp {Moved to {\tt 15JUL89} this date.}}

\cdpar {5041.} {May 15, 1989} {POSSM} {Phil}
   {\hp {Was not closing files properly when plotting spectra from
        all IF's, also failed when plotting any IF other than the first.}
    \hp {Moved to {\tt 15JUL89} this date.}}

\cdpar {5042.} {May 15, 1989} {RECAT} {Chris}
   {\hp {Fixed a number of problems:}
    \vskip 1.5pt
    \hp {(1) Revised offsets to the sequence number and file name (and the
        length of the catalog entry)  to correspond with revision {\tt C}
        of the {\tt CA} file structure --- the fact that these had
        changed was apparently missed in the code overhaul;}
     \hp {(2) Corrected a bad Fortran-77 substring expression that was
        overwriting the tail of the ``Put {\it file} in slot {\it
        i\/}'' message;}
    \vskip 1.5pt
     \hp {(3) Increased the size of the character transfer between the
        {\tt CB} file and the corresponding record in the {\tt CA} file
        so that type file is now transferred as well as the name and class;}
    \vskip 1.5pt
     \hp {(4) Changed an explicit offset to the file status to a symbolic
        value (previously set in a {\tt DATA} statement but not used); and}
     \hp {(5) Removed two assignments to {\tt NCHAR} in subroutine
        {\tt REBILD} that merely added noise to the code.}
    \hp {Moved to {\tt 15JUL89} this date.}}

\cdpar {5043.} {May 19, 1989} {REQBAS} {Chris}
   {\hp {A new utility routine to determine whether a given baseline
        is in the set selected by the user with the {\tt ANTENNAS} and
        {\tt BASELINE} adverbs.}
    \hp {Moved to {\tt 15JUL89} this date.}}

\cdpar {5044.} {May 20, 1989} {ZTXOP2} {Eric}
   {\hp {Changed the {\tt GEN} and {\tt VMS} versions to prevent
        writing over an old file except when {\tt APPEND} is specified.}
    \hp {Moved to {\tt 15JUL89} 24-May.}}

\cdpar {5045.} {May 20, 1989} {ZTRLOG} {Eric}
   {\hp {Changed the {\tt VMS} version to test the type of device when the
        translation loop fails.  If it is disk, then no error is
        returned.  If it is non-existent or not a disk, then an
        error is returned.  If at least one translation does work,
        then no error is returned even if the last symbol is not
        a device or another logical.  This allows the higher-level
        routines to find any deeper errors, but will let {\tt ZWHOMI}
        work (translation of logical gives TV number and is not
        meant to be a device).}
    \hp {Moved to {\tt 15JUL89} 24-May.}}

\cdpar {5046.} {May 20, 1989} {INQINT, INQFLT} {Eric}
   {\hp {These free-format integer and floating read routines did not
        reset the buffer pointer when they had to read another
        line from the terminal.  This made for an infinite loop.
        Corrected.}
    \hp {Moved to {\tt 15JUL89} 24-May.}}

\cdpar {5047.} {May 20, 1989} {GRITP} {Eric}
   {\hp {Corrected numerous overhaul mistakes.  Added options to mount,
        position, and dismount the tape.}
    \hp {Moved to {\tt 15JUL89} 24-May.}}

\cdpar {5048.} {May 20, 1989} {GR2TEX} {Eric}
   {\hp {Overhauled {\tt AIPGVMS:GRTOTEX} and renamed it
        {\tt AIPPGM:GR2TEX}.  Changed it to be much more \AIPS-like,
        to use \AIPS\ {\tt ZTX}$\ldots$ routines, to offer the option
        of mounting and dismounting the tape, and corrected several
        minor errors.  It still uses the logical (environment) variable
        \hbox{{\tt TARGET}}.  This routine will now convert an \AIPS\
         {\tt GR}ipe file into a text file suitable for e-mail to
        Charlottesville (see addresses on the masthead).}
    \hp {Moved to {\tt 15JUL89} 24-May.}}

\cdpar {5049.} {May 22, 1989} {Service programs} {Eric}
   {\hp {Changed service programs to use free format throughout.
        This required a new subroutine {\tt INQGEN} to get a sequence
        of variables in a list of types for those few lines of
        input which mix types (\eg\ {\tt FILINI} for file names,
        {\tt POPSGN}).  Changed {\tt AIPLAS}, {\tt AIPMAN}, {\tt DELSG},
        {\tt FILAIP}, {\tt FILINI}, {\tt FIXCAT}, {\tt FIXUSR},
        {\tt POPSGN}, {\tt RDFITS}, {\tt SETPAR}, and \hbox{{\tt SETTVP}}.
        Also corrected some shaky logic at the end of line in
        {\tt INQINT} and {\tt INQFLT} free-format read subroutines.
        Corrected bad comment in {\tt GETSTR}.}
    \hp {Moved to {\tt 15JUL89} 24-May.}}

\cdpar {5050.} {May 22, 1989} {Number interactive AIPS} {Eric}
   {\hp{We have been limiting the number of interactive {\tt AIPS} to 6
        if there are any batch queues.  This was done to avoid
        interference in the Task Data file between {\tt AIPSC} and tasks.
        Since we changed to {\tt QMNGR} to run the batch queues, this has
        not really been necessary, so dropped the requirement.  Now
        there can be up to 12 interactive {\tt AIPS}es with 2 batch queues.}
    \vskip 2pt
    \hp {Changed {\tt AIPPGM:} routines {\tt AIPSB}, {\tt AIPSC},
         {\tt BATER}, {\tt BSTRT1}, {\tt FILAI2}, {\tt FILAIP}, {\tt FILINI},
         {\tt GRIPR}, {\tt POPSGN}, and {\tt QMNGR}; {\tt AIPGUnix:} routine
         {\tt ZSTRTB}; {\tt AIPSUB:} routines {\tt AU1}, {\tt AU1A}, {\tt AU2},
         {\tt AU2A}, {\tt AUA}, {\tt AUB}, {\tt AUC}, {\tt EDITOR},
         {\tt HELPS}, {\tt INIT}, {\tt OERROR}, {\tt PREAD}, {\tt PRTMSG},
         {\tt PSEUDO}, {\tt SCHOLD}, and {\tt VERBS}; {\tt APLGEN:} routine
         {\tt ZSETUP}; {\tt APLVMS:} routine {\tt ZSETUP}; {\tt APLPGM:}
         routines {\tt NOBAT} and {\tt PRTAC}; {\tt APLSUB}: routines
         {\tt BATPRT}, {\tt GTPARM}, {\tt MSGWRT}, {\tt RELPOP}, and
         {\tt WHOAMI}; and {\tt QFPS16:} routine {\tt QINIT}.}
    \hp {I doubt that computers are up to the potential load, however.}
    \hp {Moved to {\tt 15JUL89} 24-May.}}

\cdpar {5051.} {May 22, 1989} {VMS} {Eric}
   {\hp {Changed {\tt AIPS.COM} to make {\tt TVDEV}{\it n} a logical
        parameter rather than a symbol.  Changed VMS version of
        {\tt ZTRLOG} to allow for non-disk translations, if one
        level of translation does work.}
    \hp {Moved to {\tt 15JUL89} this date.}}

\cdpar {5052.} {May 23, 1989} {QUEUES} {Eric}
   {\hp {Changed help file and {\tt AUB} to have {\tt BATQUE=0} mean
        all queues for the listing verb.}
    \hp {Moved nowhere.}}

\cdpar {5053.} {May 23, 1989} {Pseudoverbs} {Eric}
   {\hp {Changed {\tt POLISH} to block some pseudoverbs from compiling
       outside of procedure-building mode.  Changed the help files for
       {\tt IF}, {\tt THEN}, and {\tt WHILE} to explain that they may be
       used only in procedures.  Added a new error message to {\tt OERROR}.}
    \hp {Moved to {\tt 15JUL89} this date.}}

\cdpar {5054.} {May 23, 1989} {*TESS programs} {Tim}
   {\hp {A serious shortcoming of the {\tt *TESS} routines has been fixed.
        Previously, the restoring beam was applied incorrectly for
        rectangular images. This only affected the {\tt IVTC} images.
        The {\tt MAKGAU} routine has now been changed to rectify this.}
    \hp {Moved nowhere}}

\cdpar {5055.} {May 23, 1989} {UVLOD} {Eric}
   {\hp {Changed {\tt UVLOD} and help file to provide a default of 50
        for {\tt NPOINTS}, rather than refuse to start {\tt UVLOD} when
        {\tt NPOINTS = 0} for a FITS file, which does not use the parameter.}
    \hp {Moved to {\tt 15JUL89} this date.}}

\cdpar {5056.} {May 23, 1989} {FITTP} {Eric}
   {\hp {{\tt FITTP} was writing character data for keywords one column over
        from that recommended in the standard for the ``fixed''
        format.  It has been that way for years and no one has
        noticed and does not in fact violate the actual standard.}
    \hp {Moved to {\tt 15JUL89} this date.}}

\cdpar {5057.} {May 24, 1989} {FILLM} {Bill C}
   {\hp {Major revision; changes include:}
    \hp {(1) Now {\tt FILLM} can copy all the data for a given
        project/subarray to disk in a single pass.}
    \vskip 1.5pt
    \hp {(2) Output files are automatically expanded and contracted; the
        user no longer needs to specify the size of the output files.
        If the output disk fills up, {\tt FILLM} should save the data already
        read and shut down gracefully.}
    \hp {(3) Optional appends to existing \AIPS\ files are supported for
        reading multiple tapes.}
    \hp {(4) A reference day ({\tt REFDATE}) adverb was added and can
        optionally be used to specify the desired reference day.}
    \hp {(5) {\tt BAND} was expanded to 4 characters; the first two now
        specify the desired band for the A-C and B-D IFs.}
    \hp {(6) {\tt FILLM} now gives more messages about sources, times and
        frequencies.}
    \vskip 1.5pt
    \hp {(7) Bandwidth is now used to specify a source number; a source number
        corresponds to a unique source name-qualifier-bandwidth.}
    \hp {(8) Several bugs were fixed which caused qualifier not to be
        correctly handled.}
    \hp {(9) {\tt FILLM} now uses the LST times of the Modcomp scans
        to determine scan boundries.}
    \hp {(10) Also changed: {\tt FILLM.HLP}, {\tt REFDATE.HLP},
         {\tt POPSDAT.HLP}, {\tt INC:DAPL.INC}.}
    \hp {Moved nowhere.}}

\cdpar {5058.} {May 24, 1989} {FQ Tables} {Phil}
   {\hp {{\tt FQ} tables introduced. These are similar to the old {\tt CH}
        tables in that they contain information about the IF frequency
        offsets.  However, they are more fundamental in that we have
        modified the \AIPS\ database so that it is now able to hold
        more than one frequency/bandwidth (within limits). The use
        of the {\tt FQ} tables is determined by the presence of a
        {\tt FRQSEL} random parameter in the catalog header.  If this
        is present, the programs assume that an {\tt FQ} table is also
        present and will attempt to use it. Each {\tt FRQSEL} value has
        a corresponding entry in the {\tt FQ} table, so the tasks can
        just look in the table and obtain the frequency offset (from the
        reference frequency), the channel bandwidth, the total bandwidth
        and the sideband of the data.}
    \vskip 1.7pt
    \hp {Many tasks have been modifed (see list below) to deal with
        the new data format; most require some form of data selection
        through the three new adverbs {\tt SELBAND}, {\tt SELFREQ} and
        \hbox{{\tt FREQID}}.  Tasks will complain if they are not able
        to manage with the required adverbs.}
    \vskip 1.7pt
    \hp {Several subroutines have also been changed to deal with
        {\tt FQ} tables.  Most of the changes are invisible to the
        programmer.  However, the following have had their call
        sequences modifed: {\tt CHNDAT}, {\tt FRQTAB}, {\tt CHNCOP},
        {\tt TABCAL}, {\tt TABSN}, {\tt TABBP}, and \hbox{{\tt TABNDX}}.}
    \vskip 1.7pt
    \hp {The new frequency selection parameter has also meant that
        some tables have had to be modified, namely the {\tt CL},
        {\tt SN}, {\tt BP} and {\tt NX} tables.  To faciltate the
        translation between the old type tables and new tables (needs
        a major modification because the number of columns has changed),
        I have written the following routines:  {\tt CLREFM},
        {\tt SNREFM}, and {\tt BPREFM}.  The {\tt NX} modification is
        taken care of within {\tt TABNDX} as it was a much simpler case.
        These routines will read and, if necessary, reformat the various
        tables, in a manner that is invisible to the user.}
    \vskip 1.7pt
    \hp {Other routines have been written to deal with the I/O to the
        {\tt FQ} tables.  {\tt FQINI} and {\tt TABFQ} are analagous
        to the similar routines that exist for all other tables.
        {\tt FQMATC} determines if the selection criteria, \ie\ the new
        adverbs, match the {\tt FQ} entries.  This is used to detect any
        possible ambiguities through the selection of data using
        {\tt SELBAND} and {\tt SELFREQ}.}
    \hp {Moved nowhere.}}

\clpar {5059.} {May 24, 1989} {FQ Tables} {Phil}
   {\nhp{Subroutines modified:}
    \clsi{BPASET}{added call to {\tt BPREFM}, added check of {\tt FRQSEL}.}
    \clsi{BPINI}{changed number of columns, be careful with reads/writes.}
    \clsi{CALADJ}{added call to {\tt SNREFM}.}
    \clsi{CALINI}{changed number of columns, be careful with reads/writes.}
    \clsi{CALREF}{changed number of columns, be careful with reads/writes?}
    \clsi{CHNCOP}{added {\tt FREQID} to call sequence.}
    \dclsi{CHNDAT}{added {\tt FREQID} to call sequence; ensured would
        read old {\tt CH} tables if they exist, but only ever write
        an {\tt FQ} table.}
    \clsi{CLUPDA}{added calls to {\tt SNREFM}, {\tt CLREFM}.}
    \clsi{CSLGET}{modified equivalences,added check of {\tt FRQSEL}.}
    \clsi{DATGET}{check to see if reading correct {\tt FRQSEL} entry from
          \uv\ database.}
    \clsi{DGHEAD}{remove {\tt FRQSEL} random parameter in output
          {\tt CATBLK}.}
    \clsi{FRQTAB}{changed call sequence.}
    \clsi{GACSIN}{changed number columns?}
    \clsi{GAININ}{added calls to {\tt SNREFM}, {\tt CLREFM}.}
    \clsi{INDXIN}{updated {\tt TABNDX} call sequence.}
    \clsi{NDXINI}{changed number of columns, be careful with reads/writes.}
    \clsi{SELINI}{added {\tt SELBND}, {\tt SELFRQ}, {\tt FREQID} defaults.}
    \clsi{SELSMG}{modified equivalences, added calls to {\tt CLREFM},
         added check of {\tt FREQID}.}
    \clsi{SN2CL}{modified equivalences, added calls to {\tt SNREFM},
         {\tt CLREFM},added check of {\tt FREQID}.}
    \clsi{SNAPP}{same as above.}
    \clsi{SNINI}{changed number of columns, be careful with reads/writes.}
    \clsi{SNSMO}{same as {\tt SNAPP}.}
    \clsi{TABBP}{added {\tt FREQID} to call sequence, be careful with
         pointers.}
    \clsi{TABCAL}{same as above.}
    \clsi{TABSN}{same as above.}
    \clsi{UVGET}{checks for valid {\tt FRQSEL} values.}
    \clsi{UVPGET}{checks for location of {\tt FRQSEL} in random
         parameter list.}
    \clsi{UVGRID}{changed call sequence.}
    \clsi{UVMDIV}{changed call sequence.}
    \clsi{UVMSUB}{changed call sequence.}
    \clsi{UVTBGD}{changed call sequence.}
    \clsi{MAKMAP}{changed call sequence.}
    \nhp{Moved nowhere.}}

\clpar {5060.} {May 24, 1989} {FQ Tables} {Phil}
   {\nhp{Tasks and help files modified:}
    \clfi{ANCAL}{added {\tt SELBAND}, {\tt SELFREQ}, {\tt FREQID} adverbs;
        modified so will only correct relevant {\tt CL} rows.}
    \clfi{BPASS}{added {\tt SELBAND}, {\tt SELFREQ}, {\tt FREQID} adverbs;
        changed several call sequences.}
    \clfi{CALIB}{added {\tt SELBAND}, {\tt SELFREQ} and {\tt FREQID} adverbs.}
    \clfi{CLCAL}{added {\tt SELBAND}, {\tt SELFREQ}, {\tt FREQID} adverbs.}
    \clfi{CLCOR}{added {\tt SELBAND}, {\tt SELFREQ}, {\tt FREQID} adverbs;
         modified to deal with new format tables.}
    \clfi{FILLM}{added ability to write {\tt FQ} entries; this is now the
         default action. {\tt CH} tables will no longer be written.}
    \dclfi{GETJY}{added {\tt SELBAND}, {\tt SELFREQ}, {\tt FREQID} adverbs,
         so will only use requested {\tt SN} entries while determining flux
         densities.}
    \clfi{HORUS}{added {\tt SELBAND}, {\tt SELFREQ}, {\tt FREQID} adverbs.}
    \clfi{INDXR}{will index on {\tt FQ} entries as well as source changes,
        time gaps, etc.}
    \dclfi{LISTR}{added {\tt SELBAND}, {\tt SELFREQ} and {\tt FREQID} adverbs
        for data selection. Made scan summary more comprehensive; will list
        frequency id index, along with a summary of the {\tt FQ} table.}
    \clfi{MULTI}{modified so that will write {\tt FRQSEL} random parameter
        as well as the source random parameter.}
    \dclfi{POSSM}{added {\tt SELBAND}, {\tt SELFREQ} and {\tt FREQID} adverbs;
        also modified the {\tt BP} plotting section to read the {\tt FREQID}
        set by the user.}
    \clfi{PCAL}{added {\tt SELBAND}, {\tt SELFREQ}, {\tt FREQID} adverbs.}
    \dclfi{SETJY}{added {\tt FREQID} adverb --- needed for velocity
        calculations; program only complains if is not set but it needs it.}
    \clfi{SNPLT}{added \hbox{{\tt SELBAND}}. {\tt SELFREQ}, {\tt FREQID}
        adverbs, only used if relevant.}
    \dclfi{SPLIT}{added {\tt SELBAND}, {\tt SELFREQ}, {\tt FREQID} adverbs,
        will split data ensuring that output files have one {\tt FREQID}
        random parameter only (actually {\tt DGHEAD} ensures that).}
    \clfi{TVFLG}{added {\tt SELBAND}, {\tt SELFREQ}, {\tt FREQID} adverbs.}
    \nhp{Moved nowhere.}}

\clpar {5061.} {May 24, 1989} {FQ Tables} {Phil}
   {\nhp{Tasks and help files modified:}
    \clfi{UVFIT}{added {\tt SELBAND}, {\tt SELFREQ}, {\tt FREQID} adverbs}
    \clfi{UVIMG}{added {\tt SELBAND}, {\tt SELFREQ}, {\tt FREQID} adverbs.}
    \clfi{UVMOD}{changed {\tt CHNDAT} call sequence.}
    \clfi{UVPLT}{added {\tt SELBAND}, {\tt SELFREQ}, {\tt FREQID} adverbs.}
    \clfi{VBPLT}{changed {\tt CHNDAT} call sequence.}
    \clfi{VLBIN}{changed several call sequences. I wonder if this one
          needs to write {\tt FQ} tables??}
    \clfi{MX}{changed call sequences.}
    \clfi{ASCAL}{changed call sequences.}
    \clfi{UVSUB}{changed call sequences.}
    \clfi{VSCAL}{changed call sequences.}
    \nhp{Moved nowhere.}}

\cdpar {5062.} {May 24, 1989} {FQ Tables} {Phil}
   {\hp {Also changed a number of tasks to initialize the frequency
        selection and other new parameters in the commons.  These
        were {\tt CVEL}, {\tt CSCOR}, {\tt GRIDR}, {\tt SDCAL},
        {\tt SDTUV}, {\tt PRTSD}, and {\tt SELSD}.}
     \hp {Moved nowhere.}}

\cdpar {5063.} {May 24, 1989} {SNPLT} {Phil}
   {\hp {Modifed slightly so that when plotting constant values they
        appear in the middle of the frame, not at the bottom.}
    \hp {Moved nowhere.}}

\cdpar {5064.} {May 25, 1989} {UVPLT} {Phil}
   {\hp {Fixed bug that was causing the filename to be missing from
        the plot.  Also the last two letters of the {\tt STOKES} label
        were missing --- of course, I tested that with {\tt RR}, not
        \hbox{{\tt FULL}}.  Made the Stokes labelling a little more
        intelligent.}
    \hp {Moved to {\tt 15JUL89} this date.}}

\cdpar {5065.} {May 25, 1989} {GREYS, QMSPL, TKPL} {Eric}
   {\hp {Corrected {\tt TKPL} --- the call to initialize the TK image catalog
        was zeroing the catalog header.  Changed the variable name.
        Altered {\tt QMSPL} to force pixels to be square when the axis
        increments and {\tt XYRATIO} parameters allow that.  Corrected
        {\tt GREYS} --- the attempt to avoid opening the same file twice
        caused it to contour and grey-scale the same plane whenever
        the second image's name parameters were fully specified.}
    \hp {Moved to {\tt 15JUL89} this date.}}

\cdpar {5066.} {May 28, 1989} {FILLM} {Phil}
   {\hp {Added the ability to define user's own frequency tolerance in
        the assigning of {\tt FQ} numbers.}
    \hp {Moved nowhere.}}

\cdpar {5067.} {May 28, 1989} {POSSM} {Phil}
   {\hp {Removed an errant {\tt MSGSUP} setting and made handling of
        {\tt STOKES} labelling on the plots a little more intelligent.
        Also increased the size of the crosses for the phase plots.}
    \hp {Moved nowhere.}}

\cdpar {5068.} {May 28, 1989} {CLREFM, SNREFM, BPREFM} {Phil}
   {\hp {The reformatting routines were not dealing properly with
        the simple case of a single-polarization table. Also corrected
        an error in changing the read/write status of the files
        whose tables were being modified.}
    \hp {Moved nowhere.}}

\cdpar {5069.} {May 28, 1989} {BPINI} {Phil}
   {\hp {Had wrong number of columns in single-polarization case.}
    \hp {Moved nowhere.}}

\clpar {5070.} {May 27--29, 1989} {TV} {Eric}
   {\nhp{Changed:}
    \clsi{AU5A}{Corrected real = hollerith statement.}
    \dclsi{ZVTVX3}{{\tt APL4PT2} version: corrected reference to number
           of data words to read and made it more machine independent.}
    \dclsi{ITICS}{Corrected the order of statements: a tick label could
           be placed at an arbitrary location before if the inside
           end of the tick actually fell outside the image (\ie\ in
           the presence of rotation).  Changed it to make tick marks
           of the same length, more or less, on all TV plots.}
    \dclsi{IAXIS1}{Extended blanking of graphics one more pixel in all
           directions.  Wedge labels were not being fully erased.}
    \dclsi{HDRBUF}{Corrected it to handle the tranfer function as the 2
           characters it is.  Routine was doing it as integer and,
           hence, losing all the relevant information.}
    \nhp{Moved to {\tt 15JUL89} this date.}}

\cdpar {5071.} {May 29, 1989} {IVAS} {Eric}
   {\hp {For some reason, the IIS subroutine {\tt FIVASMOUSESTATUS} (yes,
        that's its name) is now returning arbitrary error codes.  Before it
        returned either 0 or the Y position only, but now values in
        the range -21760 to -22016 arise fairly arbitrarily.  Added
        code to ignore these error levels. Changed IVAS versions of
        {\tt YBUTON} and {\tt YCRCTL}.}
    \hp {Moved to {\tt 15JUL89} this date.}}

\cdpar {5072.} {May 29, 1989} {IMHEADER, QHEADER} {Eric}
   {\hp {Corrected both to use {\tt ITRIM} rather than {\tt TRIM} in
        displaying the random parameters.  Before the fix, the
        routine dropped the intended indentation.  Changed
        {\tt LSTHDR} and {\tt KWIKHD}.}
    \hp {Moved to {\tt 15JUL89} this date.}}

\cdpar {5073.} {May 29, 1989} {SPLIT} {Phil}
   {\hp {Ensured that a duplicate copy of the {\tt FQ} table was not made
        by putting it into the list for {\tt ALLTAB} exclusion.}
    \hp {Moved nowhere.}}

\cdpar {5074.} {May 30, 1989} {TABSN, TABCAL, TABBP} {Phil}
   {\hp {An error was being made in determining the number of columns;
        I had forgotten to deal with the simple case of a single
        polarization.}
    \hp {Moved nowhere.}}

\cdpar {5075.} {May 30, 1989} {CHNDAT} {Phil}
   {\hp {Had lost the ability to deal with data with no {\tt IF} axis in
        catalogue header due to a {\tt GO} {\tt TO} with wrong statement
        number.}
    \hp {Moved nowhere.}}

\cdpar {5076.} {May 31, 1989} {ZI8IL} {Eric/Bill C}
   {\hp {Corrected {\tt ZGETCH} and {\tt ZPUTCH} in {\tt APLUnix:} to
        force 8-bit integers to be unsigned.  Under Unix C, {\tt char}
        variables are signed (2's complement) integers.}
    \hp {Moved to {\tt 15JUL89} this date.}}

\cdpar {5077.} {May 31, 1989} {TV LUTs and OFMs} {Eric}
   {\hp {Corrected routines handling the LUT and OFM of the \hbox{TV}.  The
        computations of these were not the same everywhere, especially
        for the IVAS where {\tt MAXINT} is 8 times greater than
        \hbox{{\tt LUTOUT}}.  Changed were {\tt AU6}, {\tt AU6A},
        {\tt GRLUTS}, {\tt TVMOVI}, {\tt YINIT} ({\tt YM75}),
        {\tt YINIT} ({\tt YIVAS}), {\tt YINIT} ({\tt YSSS}),
        {\tt TVFLG}, and \hbox{{\tt TVFIDL}}.  Also corrected
        {\tt TVMOVI} to use variables rather than constants in the
        calls to turn on and move the cursor.  The constants were illegal
        since the arguments are in/out type; apparently the {\tt YIVAS}
        version is different than the {\tt YM70} version in its handling
        of them.}
    \hp {Moved to {\tt 15JUL89} this date.}}

\cdpar {5078.} {May 31, 1989} {FILINI} {Eric}
   {\hp {A branch was missing which caused an {\tt INIT} on the {\tt PW}
        file to init also the {\tt TC} file.  Fixed.}
    \hp {Moved to {\tt 15JUL89} this date.}}

\cdpar {5079.} {May 31, 1989} {TVFLG} {Eric}
   {\hp {Changed to use free format when reading numbers from the user.}
    \hp {Moved changes to {\tt 15JUL89} sometime.}}

\cdpar {5080.} {May 31, 1989} {Convex} {Chris}
   {\hp {Brought option files up-to-date.  Affected were {\tt ASOPTS.SH},
        {\tt CCOPTS.SH}, {\tt FCOPTS.SH}, {\tt LDOPTS.SH}, and {\tt PP.}.}
    \hp {Moved to {\tt 15JUL89} this date.}}

\cdpar {5081.} {June 1, 1989} {DATCAL} {Bill C}
   {\hp {Fixed problem with delay corrections when a subset of channels
        was being calibrated or when multiple IFs were present.  The
        phases were wrong by $-2 \pi \times \hbox{delay-correction}
        \times \hbox{frequency-offset}$.  This should have had little
        effect in the past since, normally, all frequency channels were
        calibrated together and {\tt CALIB} could not correctly fringe
        fit multi-IF data.  This is the routine which applies amplitude,
        phase, delay and rate corrections.}
    \hp {Moved nowhere.}}

\cdpar {5082.} {June 1, 1989} {CALIB} {Bill C}
   {\hp {Fixed numerous bugs causing fringe fitting of multi-IF data to
        fail.  General changes were: the delays are now done with
        respect to the reference frequency rather than the first channel;
        checks for valid data are now made per IF; several call sequences
        were fixed which formerly caused trouble when multiple-IF data
        was being fringe fitted.  None of these should have caused trouble
        for single-IF fringe fitting or have affected amplitude and/or phase
        solutions.}
    \hp {Moved nowhere.}}

\cdpar {5083.} {June 1, 1989} {MK3IN} {Bill C}
   {\hp {This task isn't finished yet, but might possibly work under simple
        circumstances.  It reads raw correlator MKIII VLBI data from
        Haystack format ``A'' tapes and converts them to an astronomically
        useful time system.  The correlator model is recomputed on an
        antenna basis and written in the {\tt CL} table.  Also
        {\tt MK3IN.HLP}.}
    \hp {Moved nowhere.}}

\cdpar {5084.} {June 1, 1989} {APLGEN:ZDHPRL, ZRHPRL} {Bill C}
   {\hp {New routines to convert Hewlett-Packard floating format to local
        double- or single-precision used by {\tt MK3IN} to read HP backup
        tapes.  These versions should work on any machine.}
    \hp {Moved nowhere.}}

\cdpar {5085.} {June 1, 1989} {BPINI} {Phil}
   {\hp {Bug in computing column number for {\tt IF} freqs in table.
        Introduced with {\tt FQ} tables.}
    \hp {Moved nowhere.}}

\cdpar {5086.} {June 1, 1989} {BPASET} {Phil}
   {\hp {Was not closing down {\tt BP} table properly --- wrong call sequence
        to {\tt TABBP}.}
    \hp {Moved nowhere.}}

\cdpar {5087.} {June 1, 1989} {UVGET} {Phil}
   {\hp {Tracked down the persistent problem of tasks complaining that they
        couldn't find weight and scale for compressed data, but running
        the same task immediately afterwards worked. This was because
        {\tt GAININ} sorted {\tt CL} tables and modified the catalog
        header, but on the second attempt, it did not sort the table so
        had no problem. All of this was caused by {\tt AXEFND} being
        passed {\tt CATBLK} instead of {\tt CATUV}.}
    \hp {Moved to {\tt 15JUL89} this date.}}

\cdpar {5088.} {June 1, 1989} {BPASS} {Phil}
   {\hp {Modified divide by channel 0 option so that the channels used
        to generate channel 0 are the same as Ken Sowinski uses on the
        Modcomp.}
    \hp {Moved to {\tt 15JUL89} this date.}}

\cdpar {5089.} {June 1, 1989} {SETJY} {Phil}
   {\hp {Added adverb {\tt OPTYPE} to help file, when {\tt OPTYPE='CALC'}
        and {\tt SOURCE = '3C286'} or {\tt '3C48'} the task will use
        the Baars {\it et al.}~formulae to calculate the flux densities
        and insert them into the {\tt SU} table.}
    \hp {Moved nowhere.}}

\cdpar {5090.} {June 1, 1989} {HORUS} {Phil}
   {\hp {Was copying the {\tt FQ} table to the output map. Fixed.}
    \hp {Moved nowhere.}}

\cdpar {5091.} {June 2, 1989} {MX} {Phil}
   {\hp {Bombed on compressed data due to an incorrect call sequence
        to {\tt CMPARM}.}
    \hp {Moved to {\tt 15JUL89} this date.}}

\cdpar {5092.} {June 2, 1989} {TVFLG, DTVF.INC} {Phil}
   {\hp {Fixed bug in routine {\tt TBTIME} which was causing the stop time
        of the selected data to be set incorrectly when reading {\tt NX}
        tables. Also the byte offset to {\tt UVINIT} was wrong for
        single-source files when writing the flagged data. Also added the
        three new {\tt FQ} adverbs to {\tt DTVF.INC}.}
    \hp {Moved to {\tt 15JUL89} this date.}}

\cdpar {5093.} {June 2, 1989} {CLREFM, SNREFM, BPREFM} {Phil}
   {\hp {These routines need two LUN's when reformatting tables which
        were being passed in the call sequence from upper-level
        routines.  However this was causing problems, since I had to
        guess which were unused at the time. Removed the $2^{\und}$ LUN
        from the call sequence and hard-wired it in as 45; this LUN
        is released by the end of the routine. Had to modify
        {\tt ANCAL}, {\tt CLCOR}, {\tt SNPLT}, {\tt CLUPDA},
        {\tt GAININ}, {\tt SELSMG}, {\tt GETJY}, {\tt CALADJ},
        {\tt SNAPP}, {\tt SNSMO}, {\tt SN2CL} and {\tt BPASET}
        as these all called the reformatting routines.}
    \hp {Moved nowhere.}}

\cdpar {5094.} {June 2, 1989} {Bytes} {Eric}
   {\hp {An ``\AIPS\ byte'' is now 16 bits on most machines, one-half
        of a local integer.  Error messages in {\tt ZCREAT} and
        {\tt ZMIO} (generic) were causing user confusion by referring
        to these entities as bytes.  Changed {\tt ZMIO} to display
        8-bit bytes explicitly and dropped bytes from the
        {\tt ZCREAT} message.}
    \hp {Moved to {\tt 15JUL89} this date.}}

\cdpar {5095.} {June 2, 1989} {Convex tapes} {Eric}
   {\hp {A long-standing difficulty with opening tapes on the Convex
        has turned out to be due in part to an assumption that the
        tape was positioned after an end of file record rather than,
        possibly, being after a data record.  Changed {\tt APLCVEX:ZTPOPN}
        to test on the initial back-record for the latter possibility
        and advance-record, rather than advance-file, when that occurs.}
    \hp {Moved to {\tt 15JUL89} this date.}}

\cdpar {5096.} {June 3, 1989} {LISTR} {Bill C}
   {\hp {Fixed call argument to {\tt NDXINI}, changed literal ``1'' to
        a variable.  This seems to have caused trouble when an index
        table was not present.}
    \hp {Moved to {\tt 15JUL89} this date.}}

\cdpar {5097.} {June 4, 1989} {TV labeling} {Eric}
   {\hp {Changed {\tt ITICS} and {\tt IAXIS1} to label wedges on the
        {\it x}-axis only and to remove errors in the previous
        ``fixes'' to \hbox{{\tt ITICS}}.  {\tt TICINC} must be called
        with correct corners.  Then the tick lengths can be set using
        the TV size.}
    \hp {Moved to the Convex and SUNs ({\tt 15OCT89}) and
        to {\tt 15JUL89} this date.}}

\cdpar {5098.} {June 4, 1989} {IVAS cursor} {Eric}
   {\hp {In an attempt to appease the IIS gods, I changed {\tt YCRCTL} and
        {\tt YBUTON} in {\tt YIVAS} to set an error return when
        {\tt FIVASMOUSESTATUS} returns error codes other than expected,
        but not to call \hbox{{\tt YDOERR}}.  The latter has never
        worked right and most of the cursor status errors are not
        serious enough to cause the program to collapse.}
    \hp {Moved to {\tt 15JUL89} this date.}}

\cdpar {5099.} {June 5, 1989} {CLUPDA} {Phil}
   {\hp {Removed an extraneous message suppression.}
    \hp {Moved nowhere.}}

\cdpar {5100.} {June 5, 1989} {SN2CL} {Phil}
   {\hp {Removed a call to {\tt CLREFM} which was trying to reformat a table
        not yet created, and fixed the logic handling {\tt FREQID} values
        being read from tables.}
    \hp {Moved nowhere.}}

\cdpar {5101.} {June 5, 1989} {DATBND} {Phil}
   {\hp {Inserted a check which flags data when the bandpass is zero, which
        occasionally happens for the last channel.}
    \hp {Moved nowhere.}}

\cdpar {5102.} {June 5, 1989} {IIS Model 75} {Eric}
   {\hp {Overhauled the 19 subroutines in \hbox{{\tt YM75}}.  Moved
        {\tt YOFM} and {\tt YGRAPH} to {\tt YIIS}, dropping the
        {\tt YM70} version.  The two versions of {\tt YOFM} became
        similar when we switched to a proper description of the
        OFM intensities.  The latter was always similar, but IIS
        pretended to have 512 levels with the Model 75, when they did
        not use them.  We can't test these yet since we don't have an
        IIS Model 75.}
    \hp {Moved to {\tt 15JUL89} this date.}}

\cdpar {5103.} {June 5, 1989} {Convex} {Chris}
   {\hp {Moved {\tt SPACE.} and {\tt SPACE.FOR} to CVax, representing the
        programs currently used to show disk hogs on {\tt CHOLLA}.}
    \hp {Moved to {\tt 15JUL89} this date.}}

\cdpar {5104.} {June 6, 1989} {DGHEAD} {Phil}
   {\hp {Ensured that the catalogue header holds the correct frequency
        when a frequency id number is specified.}
    \hp {Moved nowhere.}}

\cdpar {5105.} {June 6, 1989} {SPLIT} {Phil}
   {\hp {Ensured that the catalogue header and IF offsets are correct
        when a frequency id number is present.}
    \hp {Moved nowhere.}}

\cdpar {5106.} {June 6, 1989} {SPLIT} {Phil}
   {\hp {Corrected a problem that occurred when compressed data were
        being written.  The correct number of random parameters was
        not being used when the output file was created, so when the
        \uv\ file was over a certain size, the size of the output file
        was too small.}
    \hp {Moved to {\tt 15JUL89} this date.}}

\cdpar {5107.} {June 6, 1989} {GR2TEX} {Eric}
   {\hp {Added {\tt GR2TEX} to include files {\tt DSAT} and
        \hbox{{\tt VSAT}}.  Then recompiled {\tt ZTTOPN} and relinked
        \hbox{{\tt GR2TEX}}.  Every stand-alone program needs to be
         listed in this file.}
    \hp {Moved to {\tt 15JUL89} this date.}}

\cdpar {5108.} {June 6, 1989} {DeAnza} {Eric}
   {\hp {Overhauled the DeAnza Y-routines.  Changed {\tt ZDEAXF} (generic)
        to call {\tt ZIPACK} to convert the TV buffer to and from 16-bit
        form.  Also dropped the Retrographic versions of {\tt ZTKOPN}
        and {\tt ZTKCLS} from {\tt YDEA} and moved overhauled versions of
        these to {\tt APLGEN} under the names {\tt ZTKOPN.RTG} and
        {\tt ZTKCLS.RTG}.}
    \hp {Moved to {\tt 15JUL89} this date.}}

\cdpar {5109.} {June 7, 1989} {SETJY} {Phil}
   {\hp {Added 3 more {\tt OPTYPE} options for resetting fluxes and
        velocities.  This is less clumsy than having to set the respective
        adverbs to $10^{-10}$.  Also modifed the help file.}
    \hp {Moved nowhere.}}

\cdpar {5110.} {June 7, 1989} {BPASS} {Phil}
   {\hp {{\tt BPASS} was not checking the frequency id correctly in
        the indexing stage inside subroutine {\tt BASOLV}.}
    \hp {Moved nowhere.}}

\cdpar {5111.} {June 8, 1989} {DATGET} {Phil}
   {\hp {Added check on frequency id when reading {\tt NX} table; I had
        missed this one because it used {\tt TABIO} directly instead
        of {\tt TABNDX}.}
    \hp {Moved nowhere.}}

\cdpar {5112.} {June 8, 1989} {VISCNT} {Phil}
   {\hp {Added some more comments; repaired spelling.}
    \hp {Moved nowhere.}}

\cdpar {5113.} {June 8, 1989} {CALIB} {Bill C}
   {\hp {Put a trap in {\tt CLBSRC} to keep it from blowing up if it
        was asked to fringe fit an IF which had no valid data.}
    \hp {Moved nowhere.}}

\cdpar {5114.} {June 8, 1989} {MK3IN} {Bill C}
   {\hp {Removed retarded correlator correction after phone conversation
        with Alan Rogers which indicated that it was already done in the
        correlator.  Also cleaned up the logic dealing with line data
        and multiple (parallel hand) polarizations.  It may work now
        for simple continuum or line data, but has not been throughly
        tested.}
    \hp {Moved nowhere.}}

\cdpar {5115.} {June 8, 1989} {Helps} {Eric}
   {\hp {Added copyright information and lines for documentation
        to all help files (bypassing the chkout system).  Changed
        {\tt AU1A}, {\tt AU2}, {\tt AU2A}, {\tt AIPSC}, {\tt GRIPR},
        and {\tt BATER} to skip the precursor comment lines --- all
        have a semicolon in line 1.  Changed all {\tt A*.HLP} to fill
        in the documentation lines.}
    \hp {Moved to {\tt 15JUL89} this date.}}

\cdpar {5116.} {June 9, 1989} {UVINIT} {Bill C}
   {\hp {Allowed use of single buffering when {\tt NPIO} passed was 0.
        This was causing some routines to declare falsely that the
        buffer was too small.  Also removed the last traces of tape
        I/O with {\tt UVINIT}.}
    \hp {Moved nowhere.}}

\cdpar {5117.} {June 9, 1989} {POSSM} {Phil}
   {\hp {Fixed up the scaling/labelling on the ordinate. The time
        of a bad bandpass scan was not being formatted correctly;
        fixed that.}
    \hp {Moved nowhere.}}

\cdpar {5118.} {June 9, 1989} {UVLOD} {Phil}
   {\hp {Fixed up handling of {\tt BAND} and a typo.}
    \hp {Moved nowhere.}}

\cdpar {5119.} {June 12, 1989} {BPASET} {Phil}
   {\hp {A problem with equivalences which was screwing up the $2^{\und}$
        \hbox{IF}.  Redeclared the {\tt BREAL} and {\tt BIMAG} arrays
        to their full sizes and it went away.}
    \hp {Moved to {\tt 15JUL89} this date.}}

\cdpar {5120.} {June 12, 1989} {VLBIN} {Phil}
   {\hp {When {\tt FQ} tables were introduced, I removed the line defining
        the number of IF's in the data --- put it back.}
    \hp {Moved nowhere.}}

\cdpar {5121.} {June 12, 1989} {CATKEY} {Eric}
   {\hp {Corrected bug causing keyword values to fail to get into the
        extended image header.  Most visible so far with {\tt RESCALE}.}
    \hp {Moved to {\tt 15JUL89} this date.}}

\cdpar {5122.} {June 12, 1989} {Histograms} {Eric}
   {\hp {Changed {\tt YRHIST} for {\tt YM70} and {\tt YM75} to account
        correctly for the fact that 2 16-bit words are returned for
        each value.}
    \hp {Moved to {\tt 15JUL89} this date and from the Convex.}}

\cdpar {5123.} {June 25, 1989} {B*.HLP} {Eric}
   {\hp {Filled in documentation lines in the {\tt B*.HLP} files.}
    \hp {Moved this date to {\tt 15JUL89}.}}

\cdpar {5124.} {June 13, 1989} {GAININ} {Phil}
   {\hp {Made error messages more informative.}
    \hp {Moved nowhere.}}

\cdpar {5125.} {June 13, 1989} {UVPLT} {Phil}
   {\hp {Dies a little more elegantly now.}
    \hp {Moved to {\tt 15JUL89} this date.}}

\cdpar {5126.} {June 13, 1989} {Plot files} {Eric}
   {\hp {In the past, the number of adverb locations for a plot
        program was limited to 246 --- at least as saved in the
        plot file header.  Changed {\tt GINIT} to write a second, third,
        or fourth record as needed to hold all the parameters.
        Changed {\tt CANPL}, {\tt LWPLA}, {\tt PRTPL}, {\tt QMSPL},
        {\tt TKPL}, {\tt TVPL}, and {\tt TXPL} to skip over the
        parameters if they move to additional records.  Changed
        {\tt SNPLT} and {\tt POSSM} to have their own plot
        types (17 and 16, resp.).  Corrected number of parameters
        in {\tt UVPLT}.}
    \hp {Moved to {\tt 15JUL89} this date.}}

\cdpar {5127.} {June 13, 1989} {POPSGN} {Eric}
   {\hp {Improved the message prompting for inputs since they are now
        in free format and the old input line will not work.}
    \hp {Moved to {\tt 15JUL89} this date.}}

\cdpar {5128.} {June 14, 1989} {SETSTK} {Phil}
   {\hp {New subroutine to ensure that the {\tt STOKES} adverb requested
        by the user is a valid one for the plotting routines; if
        it is not, it is changed.}
    \hp {Moved to {\tt 15JUL89} this date.}}

\cdpar {5129.} {June 14, 1989} {UVPLT, POSSM, TVFLG} {Phil}
   {\hp {Modified to call {\tt SETSTK}.  This corrects the problem we were
        encountering due to lax description of {\tt STOKES} and a
        misunderstanding of what {\tt UVGET} returned.}
    \hp {Moved to {\tt 15JUL89} this date.}}

\cdpar {5130.} {June 14, 1989} {BPASS} {Phil}
   {\hp {Modified an informative message to the user which was
        incorrect.}
    \hp {Moved to {\tt 15JUL89} this date.}}

\cdpar {5131.} {June 14, 1989} {Plot programs} {Eric}
   {\hp {Changed {\tt CNTR}, {\tt PCNTR}, {\tt GREYS}, and {\tt PROFL}
        to test and set the star-plotting adverbs before the calls to
        \hbox{{\tt GINIT}}.  This allows {\tt EXTLIST} to tell users
        about their having plotted star positions.  Changed
        {\tt SL2PL} to correct errors in the overhaul in which
        {\tt REAL}s had their values integerized incorrectly in the
        special ``extra-information'' plot record.}
    \hp {Moved to {\tt 15JUL89} this date.}}

\cdpar {5132.} {June 14, 1989} {TV} {Eric}
   {\hp {Changed {\tt YIVAS} versions of {\tt YCRCTL} and {\tt YBUTON}
        yet again.  The IVAS has come up with more meaningless error
        numbers from \hbox{{\tt FIVASMOUSESTATUS}}.  What will happen if
        there is ever a real one?  Also changed an error in the
        {\tt YIIS} version of \hbox{{\tt YGRAPH}}.  The graphics
        images were left as ghosts when the graphics planes were
         turned off.}
    \hp {Moved to {\tt 15JUL89} this date.}}

\cdpar {5133.} {June 15, 1989} {TABNDX} {Phil}
   {\hp {Minor mod to ensure that, if no {\tt FQ} column present in {\tt NX}
        table, then a sensible value is always returned.}
    \hp {Moved nowhere.}}

\cdpar {5134.} {June 16, 1989} {SETSTK} {Phil}
   {\hp {In the case where {\tt DOCALIB=1} and {\tt NCOR=1}, routine was
        setting {\tt STOKES} to {\tt I} instead of {\tt RR} or {\tt LL}.}
    \hp {Moved to {\tt 15JUL89} this date.}}

\cdpar {5135.} {June 16, 1989} {POSSM} {Phil}
   {\hp {When the last channel of a spectrum was flagged, the weights
        accumulated during the averaging were set to zero, so the
        normalization of the averaged spectrum was not done. Fixed.
        Also ensured that the scaling relevant to the IF being plotted
        was passed to {\tt GINIT} for recording by {\tt EXTLIST}.}
    \hp {Moved to {\tt 15JUL89} this date.}}

\cdpar {5136.} {June 16, 1989} {PRTUV} {Phil}
   {\hp {The error message about {\tt PRTUV} not being able to handle
        compressed data was not being printed.}
    \hp {Moved to {\tt 15JUL89} this date.}}

\cdpar {5137.} {June 16, 1989} {ZVLBIN, VLBIN} {Phil}
   {\hp {There was a problem with reading the data from the external
        file involving 2-byte integers. Fixed.}
    \hp {Moved to {\tt 15JUL89} this date.}}

\cdpar {5138.} {June 18, 1989} {BPASS} {Phil}
   {\hp {Fixed a small problem when running task on data without an
        {\tt FQ} entry; it was still checking for one.}
    \hp {Moved nowhere.}}

\cdpar {5139.} {June 19, 1989} {PUVD} {Eric}
   {\hp {Changed maximum baseline number to allow for autocorrelations.}
    \hp {Moved to {\tt 15JUL89} 23-June.}}

\cdpar {5140.} {June 20, 1989} {UVPLT} {Phil}
   {\hp {Corrected spelling of 'Amplitude' in label.}
    \hp {Moved nowhere.}}

\cdpar {5141.} {June 20, 1989} {VLBIN} {Phil}
   {\hp {Was not flushing compressed data with the correct buffer,
        also wrote a corrupted last record when hit visibility
        limit.}
    \hp {Moved to {\tt 15JUL89} this date.}}

\cdpar {5142.} {June 20, 1989} {SNPLT} {Phil}
   {\hp {Was not dealing with the Stokes adverb correctly: the user
        had to specify {\tt 'R   '} exactly to get what was wanted.
        Changed to key on first letter.}
    \hp {Moved to {\tt 15JUL89} this date.}}

\cdpar {5143.} {June 20, 1989} {FITTP} {Phil}
   {\hp {Modified to deal with {\tt FQ} tables.  If task finds file with an
        {\tt FQ} table with $> 1$ row, it writes them to tape with warning
        that only {\tt 15OCT89} \AIPS\ can deal with this data.  If
        there is only one row, the {\tt FQ} table is translated to a
        {\tt CH} table so older \AIPS\ can read it.  Also modified
        {\tt FITTP.HLP}.}
    \hp {Moved nowhere.}}

\cdpar {5144.} {June 20, 1989} {UVMAP} {Chris}
   {\hp {Removed quotes from the {\tt UV} argument in a call to
        \hbox{{\tt QGRIDA}}. This was causing {\tt UVMAP} to crash when
        a zero-spacing flux was given.  The bad call was probably
        introduced in the code overhaul.}
    \hp {Moved to {\tt 15JUL89} this date.}}

\cdpar {5145.} {June 20, 1989} {VTESS and UTESS} {Chris}
   {\hp {Altered {\tt VTESS} and {\tt UTESS} to update the maximum and
        minimum values in the output maps under all circumstances.
        Previously, the maximum was only updated if it increased and
        the minimum was only updated if it decreased.}
    \hp {Moved to {\tt 15JUL89} this date.}}

\cdpar {5146.} {June 21, 1989} {UVPLT} {Chris}
   {\hp {Recoded the handling of the {\tt ANTENNAS} and {\tt BASELINE}
        parameters.  {\tt UVPLT} was treating the {\tt BASELINE} input
        array as an extension of the {\tt ANTENNAS} input array. This
        meant that data from baselines between elements of the
        {\tt BASELINE} array were, incorrectly, being plotted
        (or excluded from the plot if an element of {\tt ANTENNAS}
        was negative).}
    \hp {Moved to {\tt 15JUL89} this date.}}

\cdpar {5147.} {June 21, 1989} {EXTLIST, Plot programs} {Eric}
   {\hp {Changed {\tt AU8A}, cleaning up the displays, bringing {\tt UVPLT}
        and others up to the current sets of inputs, and adding {\tt SNPLT}
        and \hbox{{\tt POSSM}}.  Changed {\tt VBPLT}, {\tt GAPLT}, and
        {\tt SNPLT} to display the antennas or antenna pairs actually
        used on the specific plot in the inputs stored with the plot
        file.  Fixed logic in {\tt AN10RS} and {\tt VBPLT} to select
        the advertised set(s) of antennas or antenna pairs.  Dropped
        {\tt INTYPE} from {\tt EXTLIST} --- changed the help file and
        {\tt AU8A}.}
    \hp {Moved fixes to {\tt 15JUL89} this date.}}

\cdpar {5148.} {June 22, 1989} {Unix Z routines} {Eric}
   {\hp {Changed {\tt SYSAIPS} and {\tt SYSUnix} versions of
        {\tt AREAS.DAT} to delete all references to {\tt COS} and
        ModComp ({\tt MC4}) and to change the Unix Z directory
        structure.  The {\tt BELL} area will now refer to System V
        with subdirectories for Mascomp and Cray.  The {\tt BERK}
        area will now refer to bsd 4.2 and later with subdirectories
        for Alliant, Sun, Vax, and Convex.  The last will have
        subdirectories for {\tt NRAO1} and \hbox{{\tt VLAC1}}.
        Appropriate code movements and deletions were done on all
        NRAO computers.  Note that we have now deleted all
        references to Bell versions 3 and 7 and Berkeley 4.1.}
    \hp {Moved to {\tt 15JUL89} this date and to all NRAO computers.}}

\cdpar {5149.} {June 23, 1989} {GREYS} {Chris}
   {\hp {Added a missing argument, {\tt IERR}, to the call to
        \hbox{{\tt SLICPL}}.  Also added code to handle a non-zero
        value of {\tt IERR} on return from \hbox{{\tt SLICPL}}. The
        missing argument caused a bus error under Unix when
        attempting to overlay a slice.}
    \hp {Moved to {\tt 15JUL89} this date.}}

\cdpar {5150.} {June 23, 1989} {Overhaul Z areas} {Eric}
   {\hp {In {\tt APLVax} (the new Vax-Unix area), deleted obsolete
        {\tt ZDCHIN}, {\tt ZR32RL}, and {\tt ZRLR32}, added {\tt ZDCHI2},
        and revised \hbox{{\tt ZTAP2.C}}.  In {\tt APLALLN} (the renamed
        Alliant area), deleted obsolete {\tt ZDCHIN}, {\tt ZR32RL},
        {\tt ZRLR32}, {\tt ZR64RL}, {\tt ZRLR64}, {\tt ZTTYIO},
        {\tt ZTTOPN}, {\tt ZXCLOG}, {\tt ZXTLOG}, and {\tt ZXTSPY},
        added {\tt ZDCHI2}, revised {\tt ZTPWAD} and {\tt ZTAP2.C}, and
        revised and renamed {\tt ZMOUN2.C} to {\tt ZMOUNT.TMOUNT} (since
        it used {\tt tmount} which may be Jilla-specific).  Changed
        {\tt APLUnix} version of {\tt ZTTOPN} to have a somewhat
        cleaner structure and to do better error recovery.}
    \hp {Moved to {\tt 15JUL89} this date.}}

\cdpar {5151.} {June 25, 1989} {BELL Z routines} {Eric}
   {\hp {Overhauled the Bell routines.  In {\tt APLBELL}, overhauled
        {\tt ZLOCK.C}, {\tt ZTACT2.C}, {\tt ZTKILL.C}, and {\tt ZTQSP2.C}
        and added \hbox{{\tt ZDCHI2}}.  In {\tt APLMASC}, overhauled
        {\tt ZFRE2.C}, {\tt ZMOUN2.C}, and {\tt ZTAP2.C}, and deleted
        {\tt ZR32RL}, {\tt ZRLR32}, {\tt ZR64RL}, {\tt ZRLR64},
        {\tt ZTTYIO}, and \hbox{{\tt ZXTSPY.C}}.  Tried a guess on the
        Masscomp format of {\tt df} which may have to be fixed up in
        {\tt ZFRE2.C}.}
    \hp {Moved to {\tt 15JUL89} this date.}}

\cdpar {5152.} {June 25, 1989} {C*.HLP} {Eric}
   {\hp {Filled in documentation lines in the {\tt C*.HLP} files.  Also
        changed {\tt SCALR3} to {\tt CUTOFF} as was intended long ago in
        {\tt SDCLN.HLP}.}
    \hp {Moved this date to {\tt 15JUL89}.}}

\clpar {5153.} {June 26, 1989} {VMS Midnight Job} {Eric}
   {\nhp{Changed in {\tt UPDVMS:}, the {\tt COM} files:}
    \clse{JOBMAIN}{Changed to call {\tt REPORT} for each version.}
    \clse{REPORT}{Changed to report each version separately.}
    \clse{VERSION}{Changed to append to the same log file through
            as many loops as are required.}
    \clse{CLEANUP}{Added creation of {\tt LASTGOOD.TMP}, clarified
            purging commands.}
    \clse{CONTROL}{Changed calls to {\tt SRTUNQ} to specify
            {\tt LAST{\it xxxx}.DAT} name.}
    \dclse{SRTUNQ}{Changed to use $2^{\und}$ argument to give file of
            last good date.  In this way, we can have separate last
            good dates for remove, copy, comrpl, and comlnk operations.}
    \clse{REMOVE}{Changed to create {\tt LASTREMOVE.UPD} when ends.}
    \clse{COPY}{Changed to create {\tt LASTCOPY.UPD} when ends happily.}
    \dclse{COMUPD}{New name for \hbox{{\tt COMLNK.COM}}.  Changed to
             create {\tt LASTCOMRPL.UPD} when compile-replaces complete
             successfully and to create {\tt LASTCOMLNK.UPD} when the
             compile-link error count is zero.   Also changed to make
             link errors give an error exit after trying all links.}
    \clse{END}{Change to make a {\tt LASTGOOD.DAT} only if argument is
             {\tt 'OKAY'}.}
    \noalign{\vskip 1.7pt}
    \nhp{Also changed {\tt UPDVLA:} versions of {\tt JOBMAIN},
        {\tt REPORT}, and {\tt VERSION} and removed \hbox{{\tt COMLNK}}.
        Changed the list of people to whom to report failures given in
        {\tt PEOPLE.DIS}.}
    \nhp{Moved to {\tt 15JUL89} this date.  Must go by hand to {\tt Vax1}
        before next midnight job.}}

\cdpar {5154.} {June 29, 1989} {BAKTP, BAKLD} {Eric}
   {\hp {Overhauled {\tt BAKTP} and {\tt BAKLD} and moved them from
        {\tt APGVMS:} to {\tt APGNOT:}.  Changed VMS versions
        {\tt ZSHCMD} (correct process name, test for warning, bot okay,
        error codes), {\tt ZBKTP1}, {\tt ZBKTP2}, and {\tt ZBKTP3}
        (drop unused arguments in calls, correct DCL commands, dropping
        excess blanks), and {\tt ZBKLD1} and {\tt ZBKLD2} (drop excess
        ``{\tt ]}'' from DCL command, specify {\tt /NOLOG} where
        appropriate).  Changed {\tt DBKL.INC} (declare some things
        {\tt CHARACTER} instead of {\tt REAL}, correct typing, drop
        {\tt DOCRT}) and {\tt DBTP.INC} (correct handling of
        \hbox{{\tt CHARACTER}}).  Changed {\tt BAKLD.HLP}, dropping
        {\tt DOCRT} pretense.  Also changed {\tt APLGEN:} and
        {\tt APLUnix:} versions of {\tt ZBKTP1}, {\tt ZBKTP2}, and
        {\tt ZBKTP3} for the changed call sequence.}
    \hp {Moved to {\tt 15JUL89} this date.}}

\cdpar {5155.} {June 30, 1989} {ZACTV8} {Eric}
   {\hp {Corrected message giving area from which the load module will
        be executed.  It was using a now-uninitialized variable for
        a not-meaningful message, so dropped the message except for
        pseudo-AP usage.  Changed both {\tt APLVMS:} and
        {\tt APLUnix:} versions.}
    \hp {Moved to {\tt 15JUL89} this date.}}

\cdpar {5156.} {July 3, 1989} {LIBR.DAT} {Eric}
   {\hp {The link of {\tt BAKTP} failed on Convexes due to needing to
        search {\tt APLSUB} followed by {\tt APLCVEX} after the first
        \hbox{{\tt APLCVEX}}.  The files on {\tt NRAO1} and {\tt VLAC1}
        in the {\tt \$SYSLOCAL} areas did not match those on {\tt CVax}
        for these machines.  Corrected the {\tt CVax} version ---
        which was more or less okay for {\tt APGNOT} anyway, but lacked the
	upgrade to IEEE for the Ivas link library.}
    \hp {Moved to {\tt 15JUL89} and forced a midnight job with a corrected
        (older) date.}}

\cdpar {5157.} {July 3, 1989} {Help files} {Eric}
   {\hp {Deleted test help file \hbox{{\tt DUMMY}}.  Added category and
        one-line descriptions to all 53 {\tt D*.HLP} files.}
    \hp {Moved from {\tt 15JUL89} this date.}}

\cdpar {5158.} {July 3, 1989} {XTRAN} {Chris}
   {\hp {Added {\tt INFILE} to common block {\tt /CHRCOM/} in local
        include \hbox{{\tt XTRAN.INC}}.  This prevents {\tt XTRAN}
        forgetting the name of its input file.}
    \hp {Moved to {\tt 15JUL89} this date.}}

\cdpar {5159.} {July 4, 1989} {UVFND} {Chris}
   {\hp {The section of code that prints out data meeting the selection
        criteria has been modified to take the requested IF into account.
        {\tt UVFND} previously printed the amplitudes and phases for IF 1,
        no matter which IF was used in the search. This caused some
        confusion when {\tt OPCODE} was set to {\tt 'CLIP'} since the
        amplitudes printed were not always outside the flux range set
        in the {\tt APARM} array.}
    \hp {Moved to {\tt 15JUL89} this date.}}

\cdpar {5160.} {July 5, 1989} {BAKTP} {Eric}
   {\hp {Corrected {\tt BAKTP} and {\tt APLUnix:} version of
        {\tt ZBKTP1} to remove residual references to an old
        variable name in a new guise.  It took the SUN compiler
        to spot this one.}
    \hp {Moved to {\tt 15JUL89} this date.}}

\cdpar {5161.} {July 5, 1989} {SSS} {Eric}
   {\hp {Corrected {\tt YSSS:} version of {\tt YSCROL} to remove a typo that
        caused it not to send any commands to the SUN Screen Server.
        Corrected {\tt YSVU:} version of {\tt SCRWRT.C} to rewrite the screen
        with something in all cases and to recognize when all grey
        channels are off, but the graphics on.  Also changed addressing
        when screen quadrants are written --- now it works.}
    \hp {Moved from {\tt 15OCT89} on {\tt SAIPS} this date.}}

\cdpar {5162.} {July 5, 1989} {Calls to YSCROL} {Chris}
   {\hp {Fixed a number of calls to {\tt YSCROL} in {\tt AU5A}, {\tt AU5D},
        {\tt AU6}, {\tt GRBOXS} and {\tt GRPOLY} that were using literal
        values for the {\tt SCROLX} and {\tt SCROLY} arguments ({\tt SCROLX}
        and {\tt SCROLY} are assigned to within {\tt YSCROL}). The bad
        calls must have been introduced in the code overhaul.}
    \hp {Moved to {\tt 15JUL89} this date.}}

\cdpar {5163.} {July 6, 1989} {INDXR} {Phil}
   {\hp {Added the ability to create a default {\tt CL} table to
        \hbox{{\tt INDXR}}. This is useful if for some reason the
        multi-source file you have created has no {\tt CL} tables
        associated with it.  Also modified {\tt INDXR.HLP}.}
    \hp {Moved nowhere.}}

\cdpar {5164.} {July 6, 1989} {CHANSEL} {Phil/Eric}
   {\hp {Changed {\tt DAPL.INC} and {\tt POPSDAT.HLP} to add adverb
        \hbox{{\tt CHANSEL}}.  Created new file {\tt CHANSEL.HLP} to
        describe adverb selecting ranges of channel numbers.}
    \hp {Moved nowhere.}}

\cdpar {5165.} {July 6, 1989} {Helps} {Eric}
   {\hp {Added categories and one-liners to 16 {\tt E*.HLP}, 15
        {\tt F*.HLP}, and 35 \hbox{{\tt G*.HLP}}.  Also fixed up
        precursors in helps for {\tt APGS}, {\tt BPASS}, {\tt CALIB},
        {\tt CONVL}, and {\tt FFT} (all AP tasks).}
    \hp {Moved to {\tt 15JUL89} this date.}}

\cdpar {5166.} {July 7, 1989} {QUACK} {Bill C}
   {\hp {New task.  Allows flagging  a specified portion of a
        selected set of scans defined in the {\tt NX} table.  The user
        can specify either the amount of bad data at the start
        of the scan or the amount of good data at the end of the
        scan selecting by source and/or timerange.
           This task is useful for systematically flagging data due
        to the instrument taking a while to settle down or when data
        taken while the antennas are still slewing is not properly
        flagged.  Also added a {\tt QUACK.HLP}.}
    \hp {Moved from Australia, nowhere else.}}

\cdpar {5167.} {July 7, 1989} {CHANSEL.HLP} {Phil}
   {\hp {New adverb to select spectral channels in a more general way.
         It is specifically designed for the averaging together of up to
         10 ranges of channels, each of which is defined by a start,
         stop, and increment of channel number.  Updated
         {\tt POPSDAT.HLP}, {\tt DAPL.INC}, {\tt NEWPARMS.001} also.}
    \hp {Moved from Australia, nowhere else.}}

\cdpar {5168.} {July 7, 1989} {WANTCH} {Phil}
   {\hp {New subroutine to determine if a channel is wanted based on the
        {\tt CHANSEL} adverb.  If it is not wanted, the weight is returned
        as 0.0}
    \hp {Moved from Australia, nowhere else}}

\cdpar {5169.} {July 7, 1989} {AVGCHN} {Phil}
   {\hp {General routine to average a spectrum in frequency --- channels
        are included/excluded based on the {\tt CHANSEL} adverb.}
    \hp {Moved from Australia, nowhere else.}}

\cdpar {5170.} {July 7, 1989} {SPLIT} {Phil}
   {\hp {Modified {\tt SPLIT} so that it now uses the two new routines above
        when averaging in frequency. Also changed the help file.}
    \hp {Moved from Australia, nowhere else.}}

\cdpar {5171.} {July 7, 1989} {KNTR} {M. Calabretta/Bill C}
   {\hp {Task imported from the Australia Telescope.  Similar to {\tt CNTR}
        except it plots multiple spectral channels on the same plot
        in a manner similar to {\tt KONTR} and uses a contouring algorithm
        suitable for pen plotters.  Also {\tt KNTR.HLP}.}
    \hp {Moved from Australia, to {\tt 15JUL89} this date.}}

\cdpar {5172.} {July 10, 1989} {UVHGM} {M. Calabretta/Bill C}
   {\hp {New task imported from the Australia Telescope.  It makes plots
        of the statistics of values in a \uv\ data file.  Also
        {\tt UVHGM.HLP}.}
    \hp {Moved from Australia to {\tt 15JUL89} and {\tt 15OCT89}.}}

\cdpar {5173.} {July 10, 1989} {FETCH} {AT/Bill C}
   {\hp {New task.  Reads a text file containing an image in one
        of a variety of formats.  Information about the image is read
        from a set of keyword-value pairs at the beginning of the
        file. This task was adapted from an Australia Telescope
        routine.  This is a simple way to key image-like data into
        \AIPS.  Also {\tt FETCH.HLP}.}
    \hp {Moved from Australia to {\tt 15JUL89} and {\tt 15OCT89}.}}

\cdpar {5174.} {July 10, 1989} {AVGCHN} {Phil}
   {\hp {Added {\tt BCHAN} and {\tt ECHAN} variables to the call sequence,
        and modified the routine slightly so that it would select on
        these as well.  Also modified the defaults in the {\tt CHNSEL}
        array --- if the start and stop channels are zero these are
        reset to 1 and number of channels. This avoids a problem in
        {\tt WANTCH}.}
    \hp {Moved from Australia to {\tt 15OCT89}.}}

\cdpar {5175.} {July 10, 1989} {SPLIT} {Phil}
   {\hp {The call sequence to {\tt AVGCHN} had changed --- fixed.}
    \hp {Moved from Australia to {\tt 15OCT89}.}}

\cdpar {5176.} {July 10, 1989} {Helps} {Eric}
   {\hp {Added one-liner descriptions and categories to 3 {\tt H*.HLP},
        52 {\tt I*.HLP}, 5 {\tt J*.HLP}, and 4 {\tt K*.HLP} files.
        Dropped {\tt JAFPL.HLP} and {\tt IMOFF.HLP} since they are
        obsolete.}
    \hp {Moved to {\tt 15JUL89} this date.}}

\cdpar {5177.} {July 10, 1989} {TVHUEINT} {Eric}
   {\hp {Corrected bad call to {\tt YSLECT} in {\tt AU6} and a bad
        common name in {\tt HILUT} which prevented {\tt TVHUEINT}
        from working after the overhaul.}
    \hp {Moved to {\tt 15JUL89} this date.}}

\cdpar {5178.} {July 11, 1989} {WHATSNEW} {Eric}
   {\hp {Recreated this listing of changes to \AIPS\ of general user
        interest.  Listed all new things in overhaul plus highlights
        from the previous year or so.}
    \hp {Moved to {\tt 15OCT89} with additions, nowhere else.}}

\cdpar {5179.} {July 11, 1989} {SPLIT} {Eric}
   {\hp {Corrected illegal Fortran constructs in which an {\tt INTEGER}
        variable was used in a context requiring a \hbox{{\tt LOGICAL}}.
        The VMS compiler did not complain about this despite using
        various ``standard'' switches.  Also corrected handling of
        {\tt CHANSEL} inputs --- the numbers must be non-negative.
        Also corrected handling of {\tt CHNSEL} in {\tt AVGCHN} which
        set defaults to cause averaging of all channels even if only
        one of the 10 ranges was zero, and simplified the averaging
        code in {\tt SPLIT}.}
    \hp {Moved to Australia this date, nowhere else.}}

\cdpar {5180.} {July 11, 1989} {GETCTL} {M. Calebretta/Bill C}
   {\hp {Fixed bug in logic which caused it to fail when processing R
        or L polarization data.  This fix should allow {\tt MX} to
        clean R or L polarization data.  Relinked {\tt MX},
        {\tt UVSUB}, {\tt CALIB}, and {\tt ASCAL}.}
    \hp {Moved from Australia to {\tt 15JUL89} and {\tt 15OCT89}.}}

\cdpar {5181.} {July 12, 1989} {AVSPC} {Phil}
   {\hp {New task; will average spectral-line data in the frequency
        domain producing a pseudo-continuum dataset. It will copy (and
        modify if necessary) all associated tables and can handle
        compressed data.  Also created a new help file.}
    \hp {Moved from Australia, nowhere else.}}

\cdpar {5182.} {July 12, 1989} {WHATSNEW} {Eric}
   {\hp {Reordered and reworded some to make better emphasis.  Also
        added {\tt AVSPC}.}
    \hp {Moved appropriate changes to {\tt 15JUL89}, nowhere else.}}

\cdpar {5183.} {July 12, 1989} {UVPLT} {Eric}
   {\hp {Corrected help file to be consistent in saying what
        {\tt SUBARR = 0} means and changed the code to make it
        mean {\it all} subarrays.}
    \hp {Moved change to {\tt 15JUL89}, nowhere else.}}

\cdpar {5184.} {July 12, 1989} {POPSGN} {Eric}
   {\hp {Corrected bug in reading precursor comment lines of
        {\tt POPSDAT.HLP}.}
    \hp {Moved to {\tt 15JUL89} and the VLA Vaxes this date.}}

\cdpar {5185.} {July 13, 1989} {UVCOP} {Phil}
   {\hp {When {\tt UVCOP} was copying a channel range, it attempted to adjust
        the $u,v,w$ to correspond to the new reference frequency. The
        indexing in this step was 1 off.}
    \hp {Moved from Australia.}}

\cdpar {5186.} {July 14, 1989} {SMOSP} {Phil}
   {\hp {Minor problem in that a {\tt DO}-loop was using {\tt NUMPOL}
        to decide on its maximum number of loops and this was not
        declared in \hbox{{\tt BPASS}}.  Changed to use {\tt NCOR}.}
    \hp {Moved from Australia and nowhere else.}}

\cdpar {5187.} {July 14, 1989} {DATBND} {Phil}
   {\hp {Tidied up the way this routine was dealing with flagged bandpass
        entries.  If one channel was flagged, then it assumed the whole
        spectrum was bad.  Also, the total-power correction section was
        not even examining the flags.}
    \hp {Moved from Australia and nowhere else.}}

\cdpar {5188.} {July 14, 1989} {DATGET} {Phil}
   {\hp {Modified so that in addition to checking if autocorrelation data
        are required, it also checks to see if cross-correlation data are
        to be used.  More tasks are now selecting on the basis of this.}
    \hp {Moved from Australia and nowhere else.}}

\cdpar {5189.} {July 14, 1989} {BPASET} {Phil}
   {\hp {The weights used in the averaging of bandpass data were
        incorrect if one bandpass entry was flagged in any way.
        Modified to check on this.}
    \hp {Moved from Australia and nowhere else.}}

\cdpar {5190.} {July 14, 1989} {SPLIT} {Phil}
   {\hp {Modified to select (from user adverbs) whether cross- or
        total-power or both is required.  Also tidied up the use of
        {\tt CHANSEL}.  The user now passes the absolute channel
        numbers as adverbs and the task ensures that these are
        correct internally with respect to the {\tt BCHAN} and
        {\tt ECHAN} adverbs.  Also fixed the help file.}
    \hp {Moved from Australia and nowhere else.}}

\cdpar {5191.} {July 14, 1989} {BPASS} {Phil}
   {\hp {Modified to specify more explicitly whether total- or
        cross-power spectra are to be used.}
    \hp {Moved from Australia and nowhere else.}}

\cdpar {5192.} {July 14, 1989} {POSSM} {Phil}
   {\hp {Modified to specify more explicitly whether total- or
        cross-power spectra are to be used.}
    \hp {Moved from Australia and nowhere else.}}

\cdpar {5193.} {July 14, 1989} {AVSPC} {Phil}
   {\hp {Modified so that the way in which the {\tt CHNSEL} array is dealt
        with is consistent with what {\tt AVGCHN} expects.}
    \hp {Moved from Australia and nowhere else.}}

\cdpar {5194.} {July 14, 1989} {COMB} {Eric}
   {\hp {Corrected test in the {\tt 'SPIX'} operation for two identical
        frequencies.  As it was coded, it failed in all executions
        where the second frequency was larger than the first.
        Cleaned up character-handling a little in {\tt 15OCT89} only.}
    \hp {Moved error test to {\tt 15JUL89} this date.}}

\cdpar {5195.} {July 14, 1989} {UVPLT} {Eric}
   {\hp {Corrected two error branches --- one tried to delete a plot
        file before it was created and the other skipped over all
        \AIPS' close-down and accounting steps.}
    \hp {Moved corrections to {\tt 15JUL89} this date.}}

\cdpar {5196.} {July 14, 1989} {LIBR.DAT} {Eric/Kerry}
   {\hp {The non-existent directory {\tt \$APL4PT2} was left in the
        {\tt LIBR.DAT} files for {\tt SYSNRAO1:}, {\tt SYSVLAC1:},
        {\tt SYSALLN:}, and \hbox{{\tt SYSSUN:}}.  This caused
        {\tt NRAO1} at least to use {\tt APLUnix:} rather than
        {\tt APLCVEX:} versions of Z-routines, thereby messing up the
        interactive use of terminals among other things.}
    \hp {Moved to {\tt 15JUL89} this date.}}

\cdpar {5197.} {July 17, 1989} {FILLM} {Bill J}
   {\hp {{\tt FILLM} had assumed that ``time of geometry'' was equivalent
        to time of start of integration.  The ``time of geometry'' is
        updated no more frequently than every 10 seconds so, for shorter
        integrations, the times were incorrect.  Also fixed a small
        coding error in error-handling in {\tt FLMUV} subroutine.}
    \hp {Moved to {\tt 15JUL89} this date.}}

\clpar {5198.} {July 17, 1989} {Linear polarization} {Bill C}
   {\nhp{The following changes were made in an attempt to support
        observations made using linearly polarized feeds in the manner
        used by the Australia Telescope:}
    \clsi{PCAL}{Modified to solve for X-Y ``leakage'' terms.  Added routine
            {\tt XYCALC}.}
    \clsi{PCAL}{Changed help file: short description of using linear data.}
    \clsi{DATPOL}{Changed to call {\tt LXYPOL} to compute transformation
            matrices for X-Y data.}
    \dclsi{LXYPOL}{Changed to compute matrices to transform observed
            XX, YY, XY and YX data to corrected RR, LL, RL and LR data.}
    \clsi{POLSET}{Changed to recognize {\tt 'X-Y LIN '} type instrumental
            polarization models.}
    \dclsi{DGHEAD}{Changed to relabel XX, YY, XY, YX data as RR, LL, RL,
            LR, if it was having the polarization corrections made.}
    \dclsi{DGINIT}{Changed to treat XX, YY, XY, YX data as RR, LL, RL,
            LR, as it should be in this form when it gets to {\tt DGGET}.}
    \clsi{PRTAN}{Changed to recognize {\tt 'X-Y LIN '} type instrumental
            polarization models.}
    \noalign{\vskip 1.9pt}
    \nhp{None of these modifications have been debugged due to a lack of
        suitable data, but the changes should have no impact on the
        calibration of data obtained with circularly polarized feeds.
        The general method of calibration will be to convert the corrected
        XX, YY, XY, YX data into RR, LL, RL, LR so that the bulk of the
        \AIPS\ software can properly deal with it.}
    \nhp{Moved from Australia, nowhere else.}}

\cdpar {5199.} {July 17, 1989} {DDT} {Eric}
   {\hp {Changed {\tt DDTSAVE.HLP} and {\tt DDTLOAD.001} to build an array of
        results (bits agreement, {\tt MAXFIT} differences) and to display
        them at the end (as well as while the job progresses).
        Changed to be selective in tape loading, to use {\tt CCMRG} in all
        three tests, to be more forgiving in {\tt UVDIF} (the large case
        had 32,000 differences because the \uv\ test was too tight for
        single-precision floating).}
    \hp {Moved to {\tt 15JUL89} as well.}}

\cdpar {5200.} {July 17, 1989} {Helps} {Eric}
   {\hp {Added 1-liners and categories to 11 {\tt L*.HLP} and 20
        \hbox{{\tt M*.HLP}}.  Removed obsolete {\tt MMPAS.HLP}.}
    \hp {Moved to {\tt 15JUL89} nowhere else.}}

\cdpar {5201.} {July 18, 1989} {CLREFM} {Chris}
   {\hp {{\tt CLREFM} no longer reports no errors if it receives an error
        status of 2 (cannot find or open specified {\tt CL} table) from
        \hbox{{\tt CALINI}}. {\tt CLREFM} previously did nothing in these
        circumstances and returned an error status of zero; this caused
        {\tt CLCOR} to create a spurious {\tt CL} file with no entries
        and then process it if {\tt GAINVER} was one greater than the
        number of {\tt CL} tables actually present. Any tasks that relied
        on {\tt CLREFM} returning an error code of zero if the requested
        {\tt CL} table did not exist should be modified to check for and
        ignore error code 2.}
    \hp {Moved nowhere.}}

\cdpar {5202.} {July 18, 1989} {DATBND} {Phil}
   {\hp {Modified to cope with all variations of polarization data
        that may be thrown at it. In the process of this, the routine
        was tidied up considerably and made easier to maintain.}
    \hp {Moved from Australia and nowhere else.}}

\cdpar {5203.} {July 18, 1989} {BPASS} {Phil}
   {\hp {Added several new features: it will now recognize various
        types of polarized line data and knows what to do with them;
        it will now calculate ``channel 0'' under the control of the
        {\tt CHANSEL} adverb; it will, if requested, print out the closure
        errors it finds during the least squares solution stage; and it
        will amp-scalar average data prior to determining the
        bandpasses if requested.  Included an updated {\tt BPASS.HLP}.}
    \hp {Moved from Australia and nowhere else.}}

\cdpar {5204.} {July 18, 1989} {AP 120B} {Eric}
   {\hp {The FPS array processor still requires {\tt INTEGER*2} as its
        arguments.  Unfortunately, the code provided by FPS to
        translate microcode and vector-function-chainer routines
        into ``Fortran'' puts only \hbox{{\tt INTEGER}}.  This is now
        4-byte integers in \AIPS\ and causes problems.  Changed
        {\tt INTEGER} to {\tt INTEGER*2} in {\tt Q120B:} versions
        of {\tt DIDSUB}, {\tt FINGRD}, {\tt GRDCC}, {\tt MAXMGV},
        {\tt MCALC}, {\tt PTSUB}, and {\tt RECT}.}
    \hp {Moved to {\tt 15JUL89} this date.}}

\cdpar {5205.} {July 18, 1989} {Unic Systems} {Kerry}
   {\hp {Deleted a large number of obsolete or otherwise unused
        procedures, {\tt sed} scripts, and the like from all Unix system
        areas.  Put back an improved, faster version of {\tt SEARCH.} in
        \hbox{{\tt SYSUnix:}}.  Moved in a new {\tt AREAS.CSH} matching
        the revised \hbox {{\tt AREAS.DAT}}.  Also in {\tt SYSUnix:},
        changed {\tt PP.} to redirect {\tt stdout} to {\tt stderr} and
        {\tt FCOPTS.SH} and {\tt LDOPTS.SH} to define {\tt NOOPT} as null.}
    \hp {Moved to {\tt 15JUL89} this date.}}

\cdpar {5206.} {July 18--19, 1989} {ZDIR} {Eric}
   {\hp {Changed VMS and Unix versions of {\tt ZDIR} to accept
        {\tt AIPS\char'137VERSION} as a standard area like {\tt TST}.}
    \hp {Moved to {\tt 15JUL89} this date.}}

\clpar {5207.} {July 19, 1989} {VMS installation} {Eric}
   {\nhp{Changed VMS installation procedures:}
    \clei{IBATFINI}{changed form of input line to {\tt POPSGN}.}
    \clei{IBUILD}{changed form of input line to {\tt POPSGN}.}
    \clei{ICREDCL}{added more names for queues to {\tt ASSNLOCAL}.}
    \clei{ILOAD}{changed date to {\tt 15OCT89} ({\tt TST} only).}
    \clei{IPROMPT}{raised upper limit of number of interactive {\tt AIPS}.}
    \clei{IREADTAP}{dropped obsolete parts of ``you can delete $\ldots$''
                   display.}
    \clei{TRANSPRT}{dropped some obsolete areas and files from the
                   lists of those ignored.}
    \nhp{Moved all but {\tt ILOAD} to {\tt 15JUL89} this date.}}

\cdpar {5208.} {July 19, 1989} {GRIDR} {Eric}
   {\hp {Corrected addressing bug in moving the projection type from
        the random parameters to the image coordinates parts of the
        header.}
    \hp {Moved correction only from {\tt 15OCT89} this date.}}

\clpar {5209.} {July 19, 1989} {Unix Systems} {Kerry}
   {\nhp{Changed in the Alliant system area:}
    \clth{0README.}{Updated documentation.}
    \clth{FCOPTS.SH}{Updated Fortran options.}
    \clth{LDOPTS.SH}{Updated linker options.}
    \nhp{Changed in the Convex system area:}
    \clth{FC.}{New file, Fortran compilation.}
    \clth{KC.}{New file, ``kludge'' compilation.}
    \clth{FSC.}{New file, Fortran compilation.}
    \clth{OPT2.LIS}{Updated list of routines to optimize.}
     \nhp{Changed in the {\tt NRAO1} system area:}
    \clth{AREAS.CSH}{Changed to match revised {\tt AREAS.DAT}.}
    \clth{ASOPTS.CSH}{Changed to select IEEE floating-point.}
    \clth{ASSNLOCAL.SH}{Added line printer definition.}
    \clth{ASSNLOCAL.CSH}{Added line printer definition.}
    \clth{CCOPTS.SH}{Changed to select IEEE floating-point.}
    \clth{DACK.}{Changed to handle {\tt B} and {\tt C} \AIPS\ file
            formats.}
    \clth{FC.}{Changed ``at'' file handling.}
    \clth{FCOPTS.SH}{Changed to select IEEE floating-point.}
    \clth{FSC.}{Changed ``at'' file handling.}
    \clth{LDOPTS.SH}{Changed to select IEEE floating-point.}
    \clth{OPT2.LIS}{Updated list of routines to optimize.}
    \clth{SPACE.}{Updated to report all six disks.}
    \clth{VERSATEC.}{Changed to use ``anonymous'' {\tt ftp} acount on
             {\tt CVax}.}
    \clth{AREAS.SH}{New file, same as {\tt SYSUnix:}.}
    \clth{DRUN.}{New file, runs stand-alone programs in debug mode.}
    \clth{LIBR.}{New file, finds appropriate link files.}
    \clth{ZLPCL2.QMS}{New file, command file to use QMS as line printer.}
    \noalign{\vskip 1.7pt}
    \nhp{Also added {\tt CCMRG} to {\tt INSUnix:INSTEP3.} and added
             to {\tt SYSSUN:0README.} a summary of the experiences
             with SUNOS 4.{\it x}.}
    \nhp{Moved to {\tt 15JUL89} this date.}}

\cdpar {5210.} {July 20, 1989} {FARAD} {Chris}
   {\hp {New task. Calculates the ionospheric Faraday rotation measure
        along the line of sight and enters it in a {\tt CL} table. The
        rotation  measure is calculated using the total electron
        content of the F2 layer and an offset dipole model of the
        Earth's magnetic field. The electron content may be read from
        an external file of hourly values in the format used by the
        monitoring station at Boulder, Colorado or calculated using an
        empirical model. The program contains stubbed routines that
        will be used to read f0F2 data in some future version.}
    \hp {Moved nowhere.}}

\cdpar {5211.} {July 20, 1989} {TVFLG} {Eric}
   {\hp {Corrected a format that was too long when asking for two values
        for the clip levels.  Changed the order of the code when handling
        the menu to reduce the number of commands sent to the \hbox{TV}.  They
        are too slow, especially on the IVAS, for things that don't
        change anything.}
    \hp {Moved changes from {\tt 15OCT89} this date and to Convex to test.}}

\cdpar {5212.} {July 21, 1989} {DDT} {Eric}
   {\hp {Corrected message levels in {\tt AU3A} (5 becomes 4 for inquiries),
        {\tt KWICK} (4 becomes 5 for {\tt PRINT}), {\tt ACOUNT} (2 to 5 for
        final accounting as in {\tt GTPARM} at start), {\tt TABCOP} (6 to
        3 okay copy) {\tt APCLN} (5 to 4 on beam), {\tt MX} (6 to 4 on
        informative).}
    \hp {Moved nowhere.}}

\cdpar {5213.} {July 24, 1989} {FILLM} {Chris}
   {\hp {Added a loop to initialize the random parameters to zero to
        the top of subroutine \hbox{{\tt FLMDAT}}. {\tt FILLM}
        allocates space for the random parameters {\tt WEIGHT} and
        {\tt SCALE}, but does not necessarily fill these in. This
        meant that the garbage values that happened to be in these
        spaces were written to the output \uv\ file; in some cases,
        the value happened to have the bit pattern for IEEE NaN,
        with fatal consequences for any program that tried to do
        anything with this number (for example {\tt FITTP},
        which multiplies it by $\pm 1$).}
    \hp {Changed in {\tt 15JUL89} this date.}}

\cdpar {5214.} {July 25, 1989} {UVDIF} {Eric}
   {\hp {Corrected erroneous handling of times and changed message levels,
        raising them if there are values printed.}
    \hp {Moved to {\tt 15JUL89} this date.}}

\cdpar {5215.} {July 25, 1989} {Message levels} {Eric}
   {\hp {Changed message levels in programs to be more consistent: 6 and
        above are errors, 5 is a serious result, 4 a less serious result,
        and 2 and 3 are for progress, conditions, etc.  Changed {\tt MX},
        {\tt GETCTL}, {\tt VISDFT}, {\tt UVGRID}, and {\tt UVTBGD}.}
    \hp {Moved some of the fixes to {\tt 15JUL89} only.}}

\cdpar {5216.} {July 25, 1989} {ZR32RL} {Eric}
   {\hp {Changed {\tt APLGEN:} versions of {\tt ZR32RL} and
        \hbox{{\tt ZR64RL}}.  The Vax stores integers in ``word-swapped''
        order, but not floating values.  Thus, the exponents are located
        in the lowest-order 16 bits on a Vax and the highest-order
        16 bits on reasonable machines.  {\tt CC} tables were being
        read by {\tt IMLOD} with blanked values of the fluxes.}
    \hp {Moved to {\tt 15JUL89} this date.}}

\cdpar {5217.} {July 25, 1989} {PRTMSG} {Eric}
   {\hp {Changed the order of the closing of the line printer and the
        message file.  Unix systems want to call {\tt MSGWRT} while closing
        the line printer and this caused a message from {\tt ZMSGCL}.}
    \hp {Moved to {\tt 15JUL89} this date.}}

\cdpar {5218.} {July 25--26, 1989} {PRTCC} {Eric}
   {\hp {Added error branch tests on {\tt WRITE} statements formatting
        the display.  This avoids tracebacks when there are bad
        components.  Added the display of a character to indicate
        an incomplete sum --- it was forgotten on the shortest
        display.  Corrected string handling of selection string.}
    \hp {Moved to {\tt 15JUL89} these dates.}}

\cdpar {5219.} {July 26, 1989} {TAFLG} {Eric}
   {\hp {Corrected error in handling flag-description string introduced
        in the overhaul.  {\tt CHARACTER} variables cannot be used
        where hollerith are expected.}
    \hp {Moved to {\tt 15JUL89} this date.}}

\cdpar {5220.} {July 27, 1989} {VLAPROCS} {Bill J}
   {\hp {Changed the defaults in {\tt CALIB} and {\tt LISTR} parts of this
        procedure from vector averaging to scalar.  Default {\tt FREQID}
        also should be {\tt FREQID = 0}.}
    \hp {Moved nowhere.}}

\cdpar {5221.} {July 27, 1989} {Addressing} {Eric}
   {\hp {The buffers having their bytes flipped have to be declared
        {\tt INTEGER*2} since that is what they are and sometimes they
        land on that sort of address.  Risc architectures care a lot
        about such things.  Corrected {\tt ZADDR} in generic, VMS, and
        Unix, {\tt ZBYTFL} in generic, and {\tt ZBYTF2} in generic,
        VMS, and Unix.}
    \hp {Moved to {\tt 15JUL89} this date.}}

\cdpar {5222.} {July 27, 1989} {PRTTP} {Eric}
   {\hp {Buffers must have their bytes flipped before calling {\tt ZDM2DL}
        to convert Modcomp double-precision to local.  Put in
        {\tt ZBYTFL} calls in {\tt PRTTP} as needed.}
    \hp {Moved to {\tt 15JUL89} this date.}}

\cdpar {5223.} {July 27, 1989} {ZADDR} {Eric}
   {\hp {Changed generic, VMS, and Unix versions of {\tt ZADDR} to
        declare input variables to be {\tt INTEGER*2}.  Otherwise,
        alignment problems can occur on risc computers when the
        inputs are on {\tt INTEGER*2} addresses.}
    \hp {Moved to {\tt 15JUL89} this date.}}

\cdpar {5224.} {July 28, 1989} {UVCMP} {Chris}
   {\hp {Made {\tt UVCMP} much more intelligent with regard to adding
        the {\tt WEIGHT} and {\tt SCALE} random parameters when
        compressing data.  {\tt UVCMP} previously added the random
        parameters needed for compressed data without checking whether
        they already existed.  This caused problems with data read by
        \hbox{{\tt FILLM}}. {\tt FILLM} generates {\tt WEIGHT} and {\tt SCALE}
        parameters even if the output data set is not compressed
        (these parameters are then zero); {\tt UVCMP} appended a
        second set of {\tt WEIGHT} and {\tt SCALE} parameters containing
        real values; subsequent programs only saw the first set (written
        by {\tt FILLM}) and scaled the compressed data by a factor of 0.0 and
        assigned it a weight of zero. The end result was that {\tt UVCMP}
        deleted all data. {\tt UVCMP} should now deal sensibly with all
        uncompressed data that already have {\tt WEIGHT} and {\tt SCALE}
        random parameters, provided that neither of these is already
        duplicated.}
    \hp {Moved to {\tt 15JUL89} this date.}}

\cdpar {5225.} {July 29, 1989} {INDXR} {Eric}
   {\hp {Changed help file to allow {\tt CPARM(3) < 0} as the instructions
        say to use.  Changed Fortran to call {\tt CL}-table routines only
        when a {\tt CL} table is being written.}
    \hp {Moved nowhere.}}

\cdpar {5226.} {July 29, 1989} {FILLM} {Eric}
   {\hp {Changed Fortran to set the default for the $2^{\und}$
        character of {\tt BAND} to the first character.  Then, the
        ``blank-means-first-found'' default takes over.  Corrected
        Fortran to close and open tables all the time.  The code
        that attempted to avoid unneccessary opens and closes caused
        the task to fail to do required ones.  Changed help file to
        explain new {\tt BAND} usage and to describe what is
        ignored when {\tt DOALL} is true.}
    \hp {Moved nowhere, {\tt 15JUL89} required different changes.}}

\cdpar {5227.} {July 30, 1989} {FITTP} {Eric}
   {\hp {It was possible to set {\tt DONEW} in such a fashion that critical
        tables were not written with \uv\ data sets.  Changed program
        to write critical tables in binary tables format if that is
        the only one that works, despite {\tt DONEW}.}
    \hp {Moved to {\tt 15JUL89} this date.}}

\clpar {5228.} {July 30, 1989} {VMS} {Eric}
   {\nhp{The maintenance of the VLA has been very difficult due to the
        use of non-\AIPS\ area to control \AIPS\ procedures.  Deleted
        {\tt Vax1} and {\tt Vax3} {\tt NRAO\$ROOT1:[AIPS]} files as much
        as possible. Brought in the following better ideas from the
        procedures that were there:}
    \cltw{AIPS.COM}{Changed to type {\tt NOTICE.TXT} if present in
                 {\tt AIPS\char'137PROC:}.}
    \cltw{AIPSUSER.COM}{Added symbol {\tt FIXFITS}.}
    \cltw{MRLOGIN.COM}{Changed name of notice.txt to {\tt MRNOTICE.TXT}.}
    \cltw{MRNOTICE.TXT}{The former {\tt SYSVMS:NOTICE.TXT}.}
    \cltw{NOTICE.TXT}{Sample file placed in {\tt SYSAIPS} (Unix could
                 do this too?).}
    \cltw{AIPSINIT.COM}{Batch job to run {\tt AJAX} and {\tt BSTRT1} at
                 boot time.}
    \cltw{BOOTUP.COM}{Changed to use {\tt AIPSINIT} to do long boot jobs
                 asynchronously.}
    \nhp{Moved to {\tt 15JUL89} this date.}}

\cdpar {5229.} {July 31, 1989} {FITTP} {Chris}
   {\hp {Fixed bugs in {\tt FITTP} associated with compressed \uv\
        data. Firstly, the subroutine that dealt with compressed data
        was not taking the fact that the variable {\tt NRPARM}
        recorded the number of random parameters being written to
        tape and not the number of random parameters in the disk
        file; if {\tt WEIGHT} and {\tt SCALE} occurred at the
        end of the random parameter list, they were not written
        to tape and the number of random parameters in the disk
        file was two more than was assumed. This lead to {\tt ZUVXPN}
        (which expands the compressed data) being called with the
        wrong offset into the data buffer. This was fixed by testing
        whether {\tt WEIGHT} and {\tt SCALE} had been deleted before
        calling \hbox{{\tt ZUVXPN}}. Secondly, the index into the
        scratch buffer used to store the expanded data was being
        offset by {\tt NRPARM} before transferring data to the
        output buffer. This was fixed by removing the offset.
        It is likely that all compressed data written to FITS tape
        before this date is corrupt.}
    \hp {Moved nowhere.}}

\cdpar {5230.} {August 4, 1989} {SETAN} {Bill C}
   {\hp {New task: reads antenna and subarray information from a
        text file and writes it to a specified {\tt AN} table, creating
        the table if necessary.  Also created {\tt SETAN.HLP}.}
    \hp {Moved from Oz to nowhere.}}

\cdpar {5231.} {August 4, 1989} {QUACK} {Bill C}
   {\hp {Modified to correct for a bug (feature?) of {\tt SOURNU} that
        returns 1 source found even when all source names were
        blank.  {\tt QUACK} should now work when no sources are specified.}
    \hp {Moved from Oz to nowhere}}

\cdpar {5232.} {August 4, 1989} {LISTR} {Bill C}
   {\hp {Added option to print source elevations when processing {\tt SN}
        or {\tt CL} tables.  Also changed {\tt LISTR.HLP}.}
    \hp {Moved from Oz to nowhere.}}

\cdpar {5233.} {August 4, 1989} {SOUELV} {Bill C}
   {\hp {New routine to compute source elevations from values in common
        set by {\tt GETANT} and {\tt GETSOU}.}
    \hp {Moved from Oz to nowhere.}}

\cdpar {5234.} {August 4, 1989} {GETJY} {Bill C}
   {\hp {Fixed bug handling the default value of the adverb \hbox{{\tt EIF}}.
        This bug would cause erratic printout of the fitted source flux
        densities. }
    \hp {Moved from Oz to nowhere.}}

\cdpar {5235.} {August 4, 1989} {UNCAL} {Phil}
   {\hp {New task to uncalibrate AT data: it takes the {\tt CU} and
        {\tt BU} tables generated by the on-line system (they contain
        the inverse of the calibration applied) and translates them
        into {\tt CL} and {\tt BP} tables. These can then be applied
        to the data in the normal fashion, so uncalibrating the data.
        Also created a help file.}
    \hp {Moved from Australia and nowhere else.}}

\cdpar {5236.} {August 4, 1989} {MAKMAP} {Phil}
   {\hp {{\tt MAKMAP} was not removing the {\tt CH}/{\tt FQ} table
        after each group of eight channels.  This had no effect for
        multi-source files (apart from an extraneous {\tt ZCREAT}
        message), but was causing single-source files to fail after
        the first eight channels.  Relinked {\tt HORUS}.}
    \hp {Moved from Australia and nowhere else.}}

\cdpar {5237.} {August 4, 1989} {CHNDAT} {Phil}
   {\hp {Minor modification to ensure that if no IF (\ie\ {\tt CH}/{\tt FQ})
        table exists that {\tt BUFFER(5)} (\ie\ the number of rows) is set
        to the correct value.}
    \hp {Moved from Australia and nowhere else.}}

\cdpar {5238.} {August 4, 1989} {ISPEC} {Phil}
   {\hp {New task to plot a spectrum for a specified pixel or region
        of a map. User passes {\tt PIXXY} or {\tt BLC}/{\tt TRC} and
        will get the spectrum for that whole region appearing in a
        {\tt PL} extension.   Also created {\tt ISPEC.HLP}.}
    \hp {Moved from Australia and nowhere else.}}

\tenpoint

\vfill
\eject
%\pgskip
\subtit{\AIPS\ Order Form}

%\count0=53
\vskip 20pt plus 17pt minus 12pt

\halign{\lft{#}\quad&\lft{#}\qquad&#\cr
1.&Name and address of Contact Person:        &\abar\cr
\noalign{\vskip 14.3pt plus 2pt minus 4pt}
\ &\abox\quad Address label on back is correct&\abar\cr
\noalign{\vskip 15pt plus 2pt minus 4pt}
\ &include street address for UPS delivery    &\abar\cr
%\noalign{\vskip 15pt plus 2pt minus 4pt}
%\ &\                                          &\abar\cr
\noalign{\vskip 15pt plus 2pt minus 4pt}
\ &\                                          &\abar\cr}

\vskip 20pt plus 17pt minus 12pt

\hbox to \hsize{\hbox to 2.85in{2.\quad\abox\quad new order\qquad\abox
        \quad reorder\hfil}\vbox{\hsize=3.0in{({\it N.B.\/}:
      If you have received a plastic mailing container from us, we
      insist that you use it for a reorder.  Also return tape(s).)}}\hfil}
\vskip 9pt
\hbox to \hsize{\hbox to 2.85in{\phantom{2.\quad}Version of
      \AIPS\ currently running:\hfil}\vbox{\vfill\hrule width 2.0in}\hfil}

\vskip 20pt plus 17pt minus 12pt

\halign{\hbox to 2.85in{#\hfil}&#\quad&\lft{#}\cr
      3.\quad \AIPS\ version desired:\phantom{p}\leaderfill&\abox
                 &15-Apr-1989\qquad(pre-overhaul)\cr
      \phantom{3.\quad}(Shipped $\approx$ 1 week after release date)
            &\abox &15-Oct-1989\qquad(post-overhaul)\cr
\noalign{\vskip 20pt plus 17pt minus 12pt}
      4.\quad Tape type desired:\cr
     \phantom{6.}\qquad (\AIPS, VMS only)\leaderfill&\abox &Vax/VMS
             {\tt BACKUP}\cr
\noalign{\vskip 10pt plus 7pt minus 5pt}
      \phantom{6.}\qquad (\AIPS, Unix only)\leaderfill&\abox &Unix tar\cr
\noalign{\vskip 5pt plus 3pt minus 2pt}
      \phantom{6.}\qquad\qquad Version of Unix system in use:
            &&\qquad\vbox{\vfill\hrule width 2.14in}\cr
      \phantom{6.}\qquad\qquad {\it N.B.\/} we need to know this.
            &&\qquad {\it e.g., } bsd4.{\it x}, Sys III, Sys V, V7,
                {\it etc.}\cr
\noalign{\vskip 10pt plus 7pt minus 5pt}
      \phantom{6.}\qquad (\AIPS, neither Unix nor VMS)\leaderfill&\abox &FITS
                                 compressed text format\cr
\noalign{\vskip 10pt plus 7pt minus 5pt}
      \phantom{6.}\qquad ({\tt DDT} test package: $\ge${\tt\
            15OCT89})\leaderfill&\abox &FITS binary data (``large'' on
            6250 bpi only)\cr
\noalign{\vskip 20pt plus 17pt minus 12pt}
      5.\quad Tape density desired:\leaderfill&\abox &1600 bpi\qquad\abox
                 \quad 6250 bpi\qquad\abox\quad {\tt QIC 24} (tar only)\cr
\noalign{\vskip 20pt plus 17pt minus 12pt}
      6.\quad There are gripes on (returned) tape:\leaderfill&\abox
                        &Yes\qquad\abox\quad No\cr
\noalign{\vskip 20pt plus 17pt minus 12pt}
      7.\quad Printed documents requested:\leaderfill&\abox
                        &Full {\tt 15OCT86} \Cookbook\ (no binder)\cr
      \          &\abox &{\tt 15OCT86} \Cookbook\ chapters:\quad
                        \vbox{\vfill\hrule width 1.3in}\cr
      \          &\abox &{\tt 15APR87} {\ss GOING}\ \AIPS\ Vol 1 (no
                                                            binder)\cr
      \          &\abox &{\tt 15APR87} {\ss GOING}\ \AIPS\ Vol 2 (no
                                                            binder)\cr
      \          &\abox &\AIPS\ Memo No.~59\phantom{p}\cr
      \          &\abox &\AIPS\ Memo No.~60\phantom{p}\cr
      \          &\abox &\AIPS\ Memo No.~61\phantom{p}\cr
\noalign{\vskip 20pt plus 17pt minus 12pt}
      8.\quad Custom binders requested:\leaderfill&\abox
                        &\Cookbook\cr
      \phantom{8.\quad} (now free) &\abox
                        &{\it GOING AIPS}, Vol.~1 \qquad\abox\quad
                         {\it GOING AIPS}, Vol.~2\cr}

\vskip 20pt plus 17pt minus 14pt
\halign{\lft{#}\qquad&\lft{#}\cr
%{ Send order form to: &\AIPS\ Group\cr
% \                   &National Radio Astronomy Observatory\cr
% \                   &Edgemont Road\cr
% \                   &Charlottesville, VA\quad 22903--2475\qquad USA\cr}
 Send order form to: &\AIPS\ Group, NRAO\cr
 \                   &Edgemont Road\cr
 \                   &Charlottesville, VA\quad 22903--2475\qquad USA\cr}

%\vfill
\eject
\end
