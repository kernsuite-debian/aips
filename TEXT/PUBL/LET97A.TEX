%-----------------------------------------------------------------------
%;  Copyright (C) 1996-1997
%;  Associated Universities, Inc. Washington DC, USA.
%;
%;  This program is free software; you can redistribute it and/or
%;  modify it under the terms of the GNU General Public License as
%;  published by the Free Software Foundation; either version 2 of
%;  the License, or (at your option) any later version.
%;
%;  This program is distributed in the hope that it will be useful,
%;  but WITHOUT ANY WARRANTY; without even the implied warranty of
%;  MERCHANTABILITY or FITNESS FOR A PARTICULAR PURPOSE.  See the
%;  GNU General Public License for more details.
%;
%;  You should have received a copy of the GNU General Public
%;  License along with this program; if not, write to the Free
%;  Software Foundation, Inc., 675 Massachusetts Ave, Cambridge,
%;  MA 02139, USA.
%;
%;  Correspondence concerning AIPS should be addressed as follows:
%;          Internet email: aipsmail@nrao.edu.
%;          Postal address: AIPS Project Office
%;                          National Radio Astronomy Observatory
%;                          520 Edgemont Road
%;                          Charlottesville, VA 22903-2475 USA
%-----------------------------------------------------------------------
%Body of \AIPS\ Letter for 15 April 1997

\documentstyle [twoside]{article}

\newcommand{\AMark}{AIPSMark$^{(93)}$}
\newcommand{\AMarks}{AIPSMarks$^{(93)}$}
\newcommand{\LMark}{AIPSLoopMark$^{(93)}$}
\newcommand{\LMarks}{AIPSLoopMarks$^{(93)}$}
\newcommand{\AM}{A_m^{(93)}}
\newcommand{\ALM}{AL_m^{(93)}}

\newcommand{\AIPRELEASE}{April 15, 1997}
\newcommand{\AIPVOLUME}{Volume XVII}
\newcommand{\AIPNUMBER}{Number 1}
\newcommand{\RELEASENAME}{{\tt 15APR97}}
\newcommand{\OLDNAME}{{\tt 15OCT96}}
\newcommand{\NEXTNAME}{{\tt 15OCT97}}

%macros and title page format for the \AIPS\ letter.
\input LET97.MAC
\input psfig

\newcommand{\MYSpace}{-11pt}

\normalstyle

\section{IMPORTANT}

IF YOU'RE CURRENTLY RECEIVING A PAPER COPY OF THE AIPSLETTER, AND
PREFER TO RECEIVE FUTURE ONES ELECTRONICALLY, SEND A NOTICE TO {\tt
aipsmail@nrao.edu} REQUESTING REMOVAL FROM THE PAPER COPY MAILING
LIST.


\section{General developments in \AIPS}

\subsection{Staff}

In August, 1996 Dave Adler accepted an offer from the Space Telescope
Science Institute and left the \AIPS\ group.  Staff and visitors at
the AOC will remember Dave as a very helpful and knowledgeable first
point of contact for all \AIPS\ related questions.  Dave also was
responsible for keeping in running order what has to be one of the
most complex \AIPS\ installations.  In May, 1997, Athol Kemball moved
to the \AIPTOO\ project.  In the past years, Athol has greatly
contributed to \AIPS, in particular to VLBA and Space VLBI
applications.  We are fortunate that Athol will still be available for
consulting on \AIPS\ matters.  We wish both Dave and Athol good luck
in their new endeavors.

\subsection{Current release}

The \RELEASENAME\ release of Classic \AIPS\ is now available.  It may
be obtained via {\it anonymous} ftp or by contacting Ernie Allen at
the address given in the masthead.  \AIPS\ is now copyright \copyright
1995, 1996, 1997 by Associated Universities, Inc., NRAO's parent
corporation, but may be made freely available under the terms of the
Free Software Foundation's General Public License \hbox{(GPL)}.  This
means that User Agreements are no longer required, that \AIPS\ may be
obtained via anonymous ftp without contacting NRAO, and that the
software may be redistributed (and/or modified), under certain
conditions.  The full text of the GPL can be found in the {\tt
15JUL95} \Aipsletter. Details on how to obtain \AIPS\ under the new
licensing system appear later in this \Aipsletter.

A total of 222 copies of the \OLDNAME\ release were distributed, of
which 114 were in source code form and 108 were distributed as binary
executables.  The table below shows the breakdown of how these copies
were distributed. This includes both source code distributions and
binary distributions. The latter method is gaining popularity quickly:
54 \% of all distributions include binaries.

\begin{center}
\begin{tabular}{|r|r|r|r|r|} \hline\hline
{ftp} & {8mm} & {4mm} & {QIC} & {Floppy} \\ \hline
198   &   16  &    7  &    1  &       0  \\ \hline\hline
\end{tabular}
\end{center}

User feedback suggests that the distribution over operating systems
for installed versions of \OLDNAME\ was as follows:

\parbox{8cm}{\begin{center}
\begin{tabular}{|l|r|r|r|} \hline\hline
{Operating System} & {No.} & \RELEASENAME  &  \OLDNAME\  \\
                   &       & {\%}   &  {\%}  \\ \hline
Solaris/SunOS 5 &    228       & 46 &  50 \\
PC Linux        &     92       & 19 &   7 \\
SunOS 4         &     63       & 13 &  22 \\
Dec Alpha       &     48       & 10 &   8 \\
SGI             &     24       &  5 &   5 \\
IBM /AIX        &     22       &  4 &   3 \\
HP-UX           &     18       &  4 &   3 \\
Ultrix          &      1       &  0 &   1 \\
Total           &    496       &    &     \\ \hline\hline
\end{tabular}
\end{center}} \hfill \parbox{8cm}{\begin{minipage}{8cm}

The percentage for the previous release is given in the last column.
The rapidly rising popularity of PC Linux is obvious.  Solaris/SunOS 5
appears to have made significant gains over SunOS 4 since the last
release. These figures are affected by the percentage of \AIPS\ users
that register with NRAO.  For \OLDNAME, though the number of \AIPS\
shipments remained essentially unchanged, we noticed a pronounced drop
in the number of sites that registered.  We remind serious \AIPS\
users that registration is required in order to receive user support.

The next release of \AIPS\ will be \NEXTNAME.

\end{minipage}}

\subsection{Delay in \RELEASENAME}

The \RELEASENAME\ version of \AIPS\ is focused on space VLBI
applications.  This arises from NRAO's commitment to fully support,
within \AIPS, the reduction of data taken with the HALCA satellite.

In April, 1997 we decided to delay the official release of
\RELEASENAME\ to the summer of 1997, to accommodate the shift in the
launch schedule and the start of in-orbit checkout (IOC) observations.
HALCA was launched on 12 February 1997, and the first scientific data
were taken around early April.  Although by then almost all Space VLBI
related software was already available in \AIPS, we decided to
postpone the release until we had had the chance to use this software
at least once on real data.  This became possible in May and June as
fringes were found, followed by a complete pass of HALCA data through
\AIPS, resulting in a map.  Though many aspects of Space VLBI specific
software remain untested under real circumstances, these first
successful results are very encouraging.  In the next release,
\NEXTNAME, we expect to ship Space VLBI application software with
additional features and enhanced robustness.


\subsection{Future plans}

The two positions in the \AIPS\ group that have been lost will not be
filled again.  This is partly due to a general trend of trying to cope
with leaner budgets, and partly due to the increasing shift of focus
of NRAO application development from \AIPS\ to \AIPTOO.  For the
foreseeable future, however, we intend to continue releasing new
versions of \AIPS\ on a 6 month time scale.

\subsection{System Support}

\begin{description}

\myitem{printers} Users can now change their printer on the fly
from within \AIPS, if you have more than one printer defined in {\tt
PRDEVS.LIST}.  If you have multiple versions of \AIPS, this requires
copying the new {\tt\$SYSUNIX} versions of {\tt ZLPCL2} and {\tt
ZLASCL} to all older {\tt\$SYSUNIX} areas.

\myitem{CAD} There is a system utility called {\tt CAD} (a shell
script in {\tt \$SYSUNIX}) that may come in useful in the event of
data or catalog corruption problems (\eg\ if your system crashes and
the disk is compromised).  It offers what is hopefully good advice on
the shape of a given user's catalog in a specific data area.

\myitem{tapes/Linux} \hskip 0.5cm The tape drive code for Linux has
had a good workout, and now correctly sets block sizes and densities
for Exabytes and DAT drives.

\myitem{\tt AIPS\_ROOT} {\tt\$SYSUNIX/TVDEVS.SH} and {\tt START\_AIPS}
no longer expand the \$AIPS\_ROOT\ variable before performing any
remote shell command.  In a heterogeneous environment with possibly
different \$AIPS\_ROOT\ values, doing so was causing problems.  This
requires users to have \$AIPS\_ROOT\ defined in their environment when
performing a remote shell ({\tt rsh} or {\tt remsh}) command (defined
at the remote system), and thus some startup script changes may be
needed at your site.

\myitem{network names} \hskip 1cm Using network service names in upper
case letters has been found to precipitate bugs in certain operating
systems.  In particular, a NIS/YP server running SunOS 5.5 was found
to reject connects for service names (in getservbyname) in uppercase.
All services are now expected to be in lowercase.

\myitem{SGI drives} Better support for SGI tape drives under Irix 6
has been implemented; the older Z routines that were built for Irix 5
are deprecated but still available.

\myitem{debuggers} A new utility is provided that allows a user to switch
debuggers on the fly within an {\tt aips} session.  The source for
this utility is in {\tt\$SYSUNIX /DBGWRAP.C} and documentation in
{\tt\$SYSUNIX/DBGWRAP.DOC}.  It is not built as part of the normal
installation procedure.

\end{description}

\section{Improvements for users in \RELEASENAME}

\subsection{Model-fitting --- 1 - {\tt SLIME}}

{\tt SLIME} is an interactive model-fitting program for \AIPS\ that
allows the user to construct and edit models by manipulating a
graphical representation of the model.  It is primarily intended for
use with VLBI data.

The current version of {\tt SLIME} is Version 2.0.0.  If you are
running an earlier version then you must switch to the new version for
the \RELEASENAME\ release of \AIPS.  To obtain {\tt SLIME} you should
go to the {\tt SLIME} homepage at {\tt
http://\www/$\sim$cflatter/slime.html} and download the appropriate
distribution package for your workstation ({\tt SLIME} is currently
available only for SPARC-based systems running Solaris 2.4 or later
and ALPHA/AXP-based systems running Digital UNIX 4.0).  Each
distribution package contains a precompiled version of {\tt SLIME} and
a script that will install it into a pre-existing \AIPS\ installation
as an \AIPS\ task.

\subsection{Model-fitting --- 2 - {\tt OMFIT}}

{\tt OMFIT}, one of three UV model fitting tasks in AIPS, has been
improved considerably in \RELEASENAME.  Several new models have been
added including optically thin spheres, rings, and disks with limb
darkening.  Other improvements include improved convergence criteria,
inclusion of the w-term, a rewrite of the HELP file, and better
error-bar analysis.  Due to limitations of the AIX fortran compiler,
{\tt OMFIT} now compiles without optimization on IBM architectures.
{\tt OMFIT} is still under development and details of features
currently being implemented can be found at {\tt http://\www/$\sim$kdesai}
or by contacting {\tt kdesai@nrao.edu}.


\subsection{Elevation interpolation}

The elevation interpolation task {\tt ELINT} underwent a series of
improvements for this release.  New {\tt OPTYPE} options were added to
allow fitting for atmospheric opacity (following a law of the form
exp(-tau/cos z)) as well as for the gain.  This option can be useful
in situations of high optical depth (such as at the upper end of
Q-band, or a soaking wet day at K-band). The task can also now deal
with elevation ranges exceeding 90 degrees, when antennas observe
``over-the-top''.  The fitted polynomial can now take two forms: i)
three coefficients of the polynomial; ii) three coefficients which
determine position and value of minimum correction (maximum gain).

\subsection{GPS ionospheric data}

We are currently experimenting with the use of dual-frequency
measurements to estimate the ionospheric electron content and the
ionospheric Faraday rotation at the VLA.  The new \AIPS\ task {\tt
LDGPS} is used to load GPS data and to correct it for known clock
offsets.  The corrected data is used by the task {\tt GPSDL} (GPS
DeLay) to calculate Faraday rotation corrections for polarization
data.

It should be emphasized that this is still an experimental system and
that GPS data is not automatically available.  Anyone who is
interested in trying this system should consult with Alan Roy ({\tt
aroy@nrao.edu}) to determine the current status of this system and to
arrange for the necessary data to be recorded.  This should be done
before the experiment is observed.


\subsection{VLBI DDT test}

Work on VLBI and Space VLBI testing using simulated data have
continued since the last release, and a new VLBI DDT test script is
now available.  The simulated data are generated using task {\tt
DTSIM}, which can apply pre-fringe fitting calibration errors,
including amplitude, phase, polarization and bandpass errors.  Source
models can also be specified.  Test compliance is checked using {\tt
DTCHK} which reads calibration tables or checks the uv-data directly.
The DDT script is called {\tt VLBDDT}, and further information can be
obtained by typing {\tt EXPLAIN VLBDDT}.



\subsection{Software ported from CVX}

Several pieces of software were ported from CVX, the experimental
version of \AIPS.  They include:

\begin{description}

\myitem{\tt EDITA} This is a new TV task to edit uv data based on
calibration tables.

\myitem{\tt EDITR} This new task allows interactive editing using a
display of up to 10 baselines to a single antenna and that allows
editing based on a secondary dataset (\eg\ residual visibilities).

\myitem{\tt SCMAP} The ``Edit Data'' menu command was implemented, and
{\tt APARM(9)} was added to give the averaging time for data editing.

\end{description}


\subsection{Miscellaneous Developments and Improvements}

\begin{description}

\myitem{\tt UVPLT} {\tt UVPLT} can now be told to generate UVW
coverage plots with uniform limits in the U,V,W directions.  If {\tt
BPARM = x,y,2,0} is selected, where x and y are chosen from 6,7,8
[U,V,W respectively], then the plot is generated with identical limits
along the X and Y axes.

\myitem{\tt SPLIT} {\tt SPLIT} now preserves the values of the rest
frequency {\tt RESTFQ} and the LSR velocity {\tt LSRVEL}.  Previously,
they had to be set by hand after running {\tt SPLIT}.

\myitem{\tt SNSMO} {\tt SNSMO} now allows smoothing of tables attached
to single source files, and the new smoothing option {\tt 'DELA'}
specifies smoothing of delays only.  A feature was added to control
whether existing flagged solution entries are interpolated, and a bug
was fixed that caused phase always to be re-referenced whereas delay
and rate were only re-referenced for VLBI and VLMB smoothing
modes. This allowed the possibility of partial re-referencing.
Another re-referencing bug that was corrected affected multi-IF
calibration tables for the second secondary reference antenna and
above.

\myitem{\tt INDXR} A bug was fixed in the routine that evaluates delay
polynomials in the {\tt IM} table, and shifts them to the {\tt CL}
table. This fix gets rid of the {\tt GETDEL} error messages that were
seen occasionally due to a small number of VLBA records being more
than 2 minutes from an {\tt IM} entry at the end of correlator jobs.

\myitem{bandpasses} \hskip 0.5cm Autocorrelation data can now be
corrected by cross-power bandpasses for polynomial bandpass solutions.
Previously, this case would be trapped, and the task would halt with
an appropriate error message.


\myitem{\tt PARALLEL} A {\tt PARALLEL} verb was added to \AIPS.  This
verb controls the number of processors that will be used for \AIPS\
tasks that have been compiled with multiprocessor support.  In the
near future, we hope to add multiprocessor support to a few selected
tasks.

\myitem{\tt IBLED} {\tt IBLED} was modified to allow an overlay plot
of model amplitude as derived from an external CLEAN component
table. This plot can be selected or de-selected using a new {\tt "PLOT
MODEL"} option of the left-hand menu. The model plotting supports
multi-field data, and is useful for editing gravitational lens
datasets in particular.

\myitem{\tt IRING} Two options were added: a) plotting the cumulative
intensity distribution, and b) specifying a range of azimuthal angles
over which to compute the intensity distributions.

\end{description}
\vskip -10pt %%% formatting, get STALIN entry on this page

\subsection{VLBI and Space VLBI specific developments}


\subsubsection{Data Loading --- {\tt FITLD}}

Weight-based flagging was added. The implementation method is not
totally general in that no {\tt FG} table is written.  Instead, data
are not copied from tape to disc if the visibility weight in any IF
drops below the specified threshold.  {\tt FITLD} now also writes the
new spacecraft orbit ({\tt OB}) table and writes output data using the
new {\tt correlation\_id} random parameter.  {\tt FITLD} now supports
reading FITS tables that are split up in 200 Mbyte chunks, as is
customary for VLBA data.

\subsubsection{ Fringe fitting}

{\tt FRING} now implements the {\tt ORIGIN} keyword for single source
data files.  When {\tt CALIB} and {\tt BPASS} run on single source
files, they add the {\tt ORIGIN} keyword to the output {\tt SN} table.
{\tt FRING} has now been modified to do this as well.

Exhaustive fringe searching and subset solve were added to {\tt
FRING}.  The FFT stage in {\tt FRING} used to give up too easily when
searching for fringes.  Now, by setting $\tt APARM(9)>0$, {\tt FRING}
will exhaustively search all baselines, beginning with those baselines
to the reference antenna specified by {\tt REFANT}.  Optionally a new
control adverb called {\tt SEARCH} can be filled in to specify the
order in which to use the remaining antennas to search for fringes to
antennas when {\tt REFANT} was inadequate.  The least squares stage
now allows specification of a subset of antennas to solve for, \eg\
{\tt ANTENNAS =1,2,3,4} to include data on all baselines between
antennas $1-4$ but also use {\tt DOFIT=1,2} to specify that solutions
only be computed for antennas 1 and 2 [effectively, baselines 1-2,
1-3, 1-4, 2-3, 2-4 are used to find solutions for antennas 1 and 2].
{\tt DOFIT=0} specifies that all antennas are solved for, as before.

Baseline stacking is now supported for multiple integration times, as
required by VSOP.  Sparse baselines are padded with duplicate values
before the initial FFT step. Weights and the SNR calculation are
adjusted accordingly.


\subsubsection{\tt BLING}

A major improvement in this release is that {\tt BLING} now
interpolates using quadratic interpolation in delay, rate and
acceleration.  This has allowed the default padding to be greatly
reduced so that {\tt BLING} should be much faster than before.  The
user still has the option of adding extra padding using negative
values of {\tt APARM(5)} and {\tt APARM(6)} but preliminary tests
indicate that this gives little improvement in precision at a large
cost in run-time.

\subsubsection{{\tt DOFIT} adverb in {\tt CALIB}}

The antenna subset solve feature as described above for {\tt FRING}
was also implemented in {\tt CALIB}.  As in {\tt FRING}, the adverb
{\tt DOFIT} specifies for which antennas solutions should be
determined.  $\tt DOFIT=0$ defaults to determining solutions for {\it
all} antennas.


\subsubsection{Ground-phasing for Space VLBI fringe-fitting}

The task {\tt GPHAS}, which averages selected baselines into a single
effective baseline, and is particularly useful for Space VLBI data,
received further development.  New options were added for {\tt REFANT}
selection, and fixed model division.  Previously the {\tt REFANT} was
created as a new virtual antenna.  Now, the {\tt REFANT} can be
specified explicitly, or taken to be the most commonly occurring {\tt
REFANT} in the {\tt CL} table.  {\tt APARM(1)=1} works as before
creating a new fictitious new antenna.  {\tt APARM(1)=0} chooses the
most commonly occurring reference antenna in the chosen {\tt CL} table
as the new reference antenna. {\tt APARM(1)=2} lets the user choose
via {\tt APARM(2)} the reference antenna number.  {\tt APARM(3)} also
allows the user the option to normalize the visibilities before
stacking them together.

\subsubsection{New polarization modes for VLBA}

The new polarization correction modes {\tt DOPOL=2} and {\tt DOPOL=3}
were introduced.  These modes allow second order polarization
corrections when using the linear D-term approximation within
\AIPS. The mode {\tt DOPOL=3} applies more rigorous flagging, removing
visibility points with any missing polarization correlation pairs. The
mode {\tt DOPOL=2} applies the same second-order correction but makes
several approximations for any missing polarization correlation pairs
when computing the correction matrix. The mode {\tt DOPOL=1} makes the
same first-order correction as before.  The new modes are currently
implemented only for VLBI polarization calibration within \AIPS, and
have no effect on other solution types (\eg\ VLA), though a VLA
implementation may be added later.  The second-order corrections have
recently been shown to be important when correcting EVN data which may
have individual D-term amplitudes as high as $\sim 20\%$ in isolated
cases (as reported by K. Lepp\"anen). Note that 2nd order corrections
have always been applied for the orientation-ellipticity VLBI
polarization model in \AIPS.


\subsubsection{Improved handling of external calibration files}

{\tt VLOG} is a new task to segment and re-format the external
calibration file produced by the VLBA monitor and control system for
each project. The resulting output files can be used directly by {\tt
ANTAB}, {\tt APCAL}, {\tt UVFLG} and {\tt PCLOD} and should help to
significantly automate a priori calibration for a broad range of VLBA
projects. This task is VLBA-specific and makes some assumptions about
the format of the input ASCII files.

{\tt PCLOD}, the task that produces an \AIPS\ {\tt PC} table based on
input data from the VLBA monitor system, was refined in several
respects.  The calculation of the reference frequency was revised in
order to correctly take into account the frequency reference pixel.
The new adverb {\tt FQTOL} sets the tolerance used in the frequency
match.  The robustness when handling multi-frequency data was
enhanced: {\tt PCLOD} no longer aborts if it encounters a frequency
group that does not match an {\tt FQ-ID} in the uv file but prints a
warning message and skips pulse-cal groups until it finds a frequency
group that does correspond to an {\tt FQ-ID} in the data file.


\subsubsection{Orbit tables}

Orbit ({\tt OB}) tables were introduced to support Space VLBI data
reduction, and three new tasks make use of this new table.  {\tt
OBTAB} reads the {\tt OB} table, and fills in empty columns with
additional information such as angular distance to the sun, position
angle of the antenna feed, etc.  Optionally {\tt OBTAB} will add the
six orbital parameters to the relevant column of the {\tt AN} table.
A second new task, {\tt OBEDT}, allows flagging of data depending on
entries in the {\tt OB} table.  Finally, {\tt OBPLT} allows various
columns of the {\tt OB} table to be plotted against each other.

\subsubsection{New random parameter: correlation\_id}

For full VSOP support, it was necessary to introduce a new random
parameter.  The VLBA correlator can change correlation modes with
great flexibility, leading to time-variable rate and delay
decorrelation corrections which depend on the type of frequency and
time filtering performed in the correlator. The most general solution
to this problem was to implement a {\tt correlation\_id} random
parameter for VLBA datasets which points to the recorded correlation
modes stored in the existing {\tt CQ} table.  This change made it
possible to allow time variable correlation mode changes in general,
although only time variable OVLB filtering is activated at present.


\subsubsection{new {\tt ALIAS} adverb}

A new adverb, {\tt ALIAS}, has been introduced to facilitate the
calibration of Space VLBI data.  Antennas specified via the {\tt
ALIAS} adverb will be treated as identical for the purposes of certain
tasks.  This allows HALCA, which appears in \AIPS\ as a conglomerate
of tracking stations, to be calibrated as one individual antenna, and
to be viewed as such for some plotting tasks.


\subsubsection{\tt MSORT}

{\tt MSORT} is a new task that reproduces the functionality of {\tt
UVSRT}, but is optimized for large slightly missorted datasets.  {\tt
MSORT} does an in-memory sort of a UV-data file.  At best, {\tt MSORT}
should be about 3 times faster than {\tt UVSRT} (in case when the data
is only weakly mis-sorted --- as is the case for normal VLA and VLBA
data).  At worst {\tt MSORT} should take no more than 50\% longer.  In
either case, {\tt MSORT} requires no ancillary disk space --- making
this the preferred sorting method for large data files.

\subsubsection{\tt RESEQ}

{\tt RESEQ} is a new task which will, via the {\tt ANTENNAS} input,
renumber antennas in a UV file.  Space VLBI requires the ability to
alias stations together at some points in the data reduction stream.
This is the first step for that requirement.  {\tt RESEQ} will
renumber all antennas specified in the {\tt ANTENNAS} adverb to {\tt
ANTENNAS(1)}, or, using {\tt INFILE}, to some desired order.

\subsubsection{\tt CLCAL}

{\tt CLCAL} now allows ``pass-through'' calibration, as well as
``proper'' {\tt FQID} and {\tt SOURCEID} selection.  By choosing {\tt
'CALP'}, {\tt CLCAL} can now be made to pass {\tt CL} table records
for which no {\tt SN} table record is found.  This is dangerous but
necessary if {\tt SN} tables are constructed piecemeal as may be
necessary for VSOP data.  {\tt SN} tables generated by single source
data files and by OOP tasks may contain {\tt FQID=-1} and/or {\tt
SOURCEID=0}.  Such entries should match ALL {\tt CL} table records.
This is perfectly obvious for {\tt FQID}s but a bit ambiguous for {\tt
SOURCEID}s.  Furthermore, the new adverb {\tt CUTOFF} specifies a
maximum time interval over which interpolation is to be performed.

\subsection{Documentation, on-line help, and user support}

\subsubsection{Designated AIP program}

We continue the designated AIP program essentially unchanged. \AIPS\
user support can be obtained by the following methods:
\begin{description}
\vspace{-10pt}

\item{ 1.} {E-mail to {\tt aipsmail@nrao.edu}. This account is checked
several times a day, and messages are forwarded within the AIPS group
as appropriate.}

\item{ 2.} {Submit a gripe. This is usually done from within
AIPS. Newer versions of \AIPS\ ({\tt 15JAN96} and later) will
automatically send an e-mail message to NRAO. The gripe system should
be used for less urgent matters, such as suggestions for improvement.}

\item{ 3.} {Contact the AIPS group member currently designated to
provide user support. This listing is available on the WWW via}

\end{description}
\begin{center}
\vskip -10pt
{\tt http://\cww/aips/d\_aip.html}
\vskip -10pt
\end{center}

The ``designated AIP'' program covers all aspects of \AIPS\ user
support, including VLBI. Users may wish to contact individual members
of the AIPS group directly if their question is of a specialized
nature, and they know who in the AIPS group is the specialist in that
area.



\section{AIPS Publications and the World-Wide Web}

     The {\it World-Wide Web\/} (WWW) is a method for sending and
receiving hypertext over the Internet network and has been made easy
to use by clients such as {\it NCSA Mosaic, Netscape, Arena,\/} and
{\it Lynx\/}.  NRAO is among the many institutions which now offer
informative Web pages and networks of additional information.  The
NRAO ``home'' page is at the Universal Resource Locator (URL) address
\begin{center} \vskip -10pt {\tt http://\www/} \vskip -10pt
\end{center} The \AIPS\ group home page may be found from the NRAO
home page or addressed directly at URL \begin{center} \vskip -10pt
{\tt http://\cww/aips/} \vskip -10pt \end{center} This page points at
basic information, news items about \AIPS, recent \AIPSLETTER s in
PostScript format, patch information for all releases after {\tt
15JAN91}, the latest \AIPS\ benchmark data from various computer
systems, copies of {\tt CHANGE.DOC} for every release since {\tt
15JAN90}, {\it all} relevant \AIPS\ Memos, {\it every} chapter of the
\Cookbook, and all recent quarterly reports to the \hbox{NSF}.  There
is even a tool to let you browse the \RELEASENAME\ versions of all
help/explain files.  We recommend that you check this URL occasionally
since it changes when new software patches, revised \Cookbook\
chapters, and new \AIPS\ Memos are released.

There are two new \AIPS\ Memos with this release:
\begin{center}
\vspace{-6pt}
\begin{tabular}{ccl}
\hline
Memo  &        Date   & Title and author  \\
\hline\hline
  93 & 97/01/29 & Position Angle of the VSOP Antenna Feed \\
     &          & \qquad L. Kogan, NRAO \\
  94 & 97/01/29 & \AIPS\ Benchmarks for the Silicon Graphics Origin 200 \\
     &          & \qquad A. Kemball \& C. Flatters, NRAO \\
\hline
\end{tabular}
\end{center}
\vspace{-6pt}
These memos are available through the WWW pages.  Since some Memos
are not available electronically and others do not yet have computer
readable figures, you may wish to write for a paper copy of these.  To
do so, use an \AIPS\ order form or e-mail your request to {\tt
aipsmail@nrao.edu}.  If you cannot use the Web, you can still use
\ftp\ to retrieve the Memos, \Cookbook\ chapters, etc.:
\begin{description}
\vspace{-10pt}
\item{ 1.} {\tt ftp aips.nrao.edu}  (currently on {\tt 192.33.115.103})
\item{ 2.} Login under user name {\tt anonymous} and use your e-mail
           address as a password ({\it yourname}{\tt @} will do; ftp
           will fill in the machine you are using).
\item{ 3.} {\tt cd pub/aips/TEXT/PUBL}
\item{ 4.} {\tt get AAAREADME} and read it for lots more information.
\item{ 5.} {\tt get AIPSMEMO.LIST} for a full list of \AIPS\ Memos.
\end{description}

\section{Patch Distribution}

As before, important bug fixes and selected improvements in
\RELEASENAME\ can be downloaded via the Web at:

\begin{center}
\vskip -10pt
{\tt http://\cww/aips/15APR97/patches.html}
\vskip -10pt
\end{center}

Alternatively one can use {\it anonymous} \ftp\ on the NRAO cpu {\tt
aips.nrao.edu} (currently located on {\tt baboon} which is {\tt
192.33.115.103}).  Documentation about patches to a release is placed
in the anonymous-ftp area {\tt pub/aips/}{\it release-name} and the
code is placed in suitable subdirectories below this. Information on
patches and how to fetch and apply them is also available through the
World-Wide Web pages for \hbox{\AIPS}.  As bugs in \RELEASENAME\ are
found, the patches will be placed in the {\tt ftp}/Web area for
\hbox{{\RELEASENAME}}.  No matter when you receive your \RELEASENAME\
``tape,'' you must fetch and install these patches if you require
them.

\vfill\eject

\section{Obtaining \AIPS\ under the GNU General Public License}

We have decided to make \AIPS\ available via anonymous ftp under the
GNU General Public License, the meaning of which was spelled out in
the {\tt 15JUL95} \hbox{\Aipsletter}.  The installation of \AIPS\ will
now proceed something like the following example:

We assume that you have created an account for \AIPS\ with a root
directory called \hbox{{\tt /AIPS}}.  Then do
\vskip -10pt
\begin{verbatim}
home_prompt<601> cd /AIPS
home_prompt<602> ftp aips.nrao.edu
Connected to baboon.cv.nrao.edu.
220 baboon FTP server (Version wu-2.4(1) Fri Apr 15 12:08:14 EDT 1994) ready.
Name (aips.nrao.cv:johndoe): anonymous
331 Guest login ok, send your complete e-mail address as password.
Password: johndoe@nrao.edu
230- This is the National Radio Astronomy Observatory ftp server for the
230- AIPS, AIPS++, and FIRST projects.  Your access from primate.cv.nrao.edu
230- has been logged, and all file transfers will be recorded.  If you do not
230- like this, type "quit" now.  Counting you there are 1 (max 20) ftp users.
230-
230- Current time in Charlottesville, Virginia is Mon Jan 18 10:18:46 1996.
230-
230-
230-Please read the file README
230-  it was last modified on Wed Mar  8 14:01:24 1995 - 316 days ago
230 Guest login ok, access restrictions apply.
ftp> cd aips/15APR97
250 CWD command successful.
ftp> get README
200 PORT command successful.
150 Opening ASCII mode data connection for README (nnnn bytes).
226 Transfer complete.
local: README remote: README
nnnn bytes received in T seconds (5 Kbytes/s)
ftp> get INSTALL.PS
200 PORT command successful.
150 Opening ASCII mode data connection for INSTALL.PS (mmmmm bytes).
226 Transfer complete.
local: INSTALL.PS remote: INSTALL.PS
mmmmm bytes received in TT seconds (5 Kbytes/s)
ftp> binary
200 Type set to I.
ftp> hash
Hash mark printing on (8192 bytes/hash mark).
ftp> get 15APR97.tar.gz
200 PORT command successful.
150 Opening ASCII mode data connection for 15APR97.tar.gz ( bytes).
226 Transfer complete.
local: 15APR97.tar.gz remote: 15APR97.tar.gz
mmmmm bytes received in TTTTT seconds (5 Kbytes/s)
ftp> quit
221 Goodbye.
\end{verbatim}
\vskip -10pt
You should type in your full e-mail address (not {\tt
johndoe@nrao.edu}) at the password prompt.  The {\tt hash} command is
optional and may be inappropriate in some versions of ftp; it does
give a useful indication of progress in the long {\tt get} in most
versions.  If you do not have the GNU file compression code ({\tt
gzip}), you should {\tt get 15APR97.tar}.  Our ftp server will
uncompress the gzipped file automatically.  (It would be around 3
times faster if you had {\tt gzip}.)

At this point you should read the {\tt README} file to review the
latest changes, if any, affecting your installation of \hbox{\AIPS}.
You should print out the {\tt INSTALL.PS} PostScript document and
read at least its overview section.  To create the rest of the {\tt
/AIPS} directory tree, and fill it with the \AIPS\ source code
\vskip -10pt
\begin{center}
\begin{tabular}{l}
   {\tt cd /AIPS} \\
   {\tt zcat 15APR97.tar.gz | tar xvf -} \\
\multicolumn{1}{c}{or} \\
   {\tt tar xvf 15APR97.tar}
\end{tabular}
\end{center}
\vskip -10pt
depending on whether you fetched the source file with compression or
without.

If you want to get the binary version(s) of \AIPS, you should read the
{\tt README} file for further directions.  They will tell you about a
procedure to run from the {\tt INSTEP1} installation procedure and/or
at a later time which will initiate a second ftp session to fetch the
appropriate contents from the {\tt \$LOAD}, {\tt \$LIBR}, {\tt MEMORY},
{\tt BIN}, and {\tt DA00} areas.  You may run this procedure more than
once if you need to fetch binaries for more than one architecture.
You may also have to run portions of this procedure ``by hand'' if you
encounter reliability problems with the network.

You will then have to run the {\tt INSTEP1} procedure, as usual, to
tell your \AIPS\ about your computer environment.  A new part of {\tt
INSTEP1} is its offer to assist you in ``registering'' your copy of
\hbox{\AIPS}.  It will help you complete a registration form and will
even e-mail it to us if you want.  When we get a registration request,
we will enter your information in our user data base and reply with
instructions and registration numeric ``keys'' which you may use to
complete the registration process (using {\tt SETPAR} and \hbox{{\tt
SETSP}}).  This may seem cumbersome and onerous, but we have two
reasons for doing this.  The first reason is to provide us with
information about the use of \hbox{\AIPS}.  This information is useful
to us to justify, to management and funding agencies, our existence
and our need for more employees or computers or disk or whatever.  The
second reason is a concern about excessive demands on our employees'
limited time to provide assistance to sites in installing and running
the software.  If an excessive demand should arise, information from
the registration process will allow us to set priorities among the
different sites.  This registration is entirely optional.  We will use
transaction logging in ftp and, hence, know which sites have fetched
the code.  We will assume that sites which do not register are not
``serious'' in their use of \AIPS\ and we will be unable to provide
any assistance to unregistered sites (except, of course, to help them
register).  This means that unregistered sites will receive no
assistance in installing \AIPS\ and users at those sites will receive
no assistance in using \AIPS, including no printed literature.  All
serious sites are strongly encouraged to register since registration
statistics are used to determine the level of effort that NRAO can
provide for the Classic \AIPS\ project.  The statistics are also used
to obtain assistance from computer vendors.
%  All serious sites are strongly encouraged to register,
%even if they do not need assistance during installation, since
%registration statistics are used to determine the level of effort that
%NRAO can provide for the Classic \AIPS\ project.

As of the {\tt 15JUL95} release, \AIPS\ is available under the GNU
General Public License.  The short statement of this license is in
every \AIPS\ file, is available on-line via {\tt HELP GNU}, and was
given (once) in the {\tt 15JUL95} \hbox{\Aipsletter}.  You should have
received the GNU General Public License from several sources, most
notably GNU themselves with their {\tt emacs}, {\tt gcc}, and numerous
other software products.  Since \AIPS\ now applies that license to
itself --- and intends to import and use other GNU-licensed routines
--- we also include the full license text on-line via {\tt EXPLAIN
GNU} and, once, in the {\tt 15JUL95} \hbox{\Aipsletter}.

\section{\Cookbook\ errata}

If you are reducing polarization data, please note that the phase
offset corrections given on Page 4-32 are in error.  The correct
values for the R-L phase difference (twice the position angle of
the electric vector) are as follows.

{\begin{center}
\begin{tabular}{|l|l|l|} \hline\hline
Calibrator & angle & band \\
\hline
	3C286	&	66 degrees    & all wavelengths \\
\hline
	3C138   &      -18 degrees    & L-band \\
                &      -24 degrees    & all other bands \\
\hline
        3C48    &       do not use    & L-band \\
                &       -150 degrees  & C-band \\
                &       -131 degrees  & shorter wavelengths\\
\hline\hline
\end{tabular}
\end{center}}





\vfill\eject
\centerline{\hss\psfig{figure=AIPSORDER.PS,height=23.3cm}\hss}
\vfill
% The dang mailing label template goes here.  Where is it????
\end{document}
