%-----------------------------------------------------------------------
%;  Copyright (C) 1995-1998
%;  Associated Universities, Inc. Washington DC, USA.
%;
%;  This program is free software; you can redistribute it and/or
%;  modify it under the terms of the GNU General Public License as
%;  published by the Free Software Foundation; either version 2 of
%;  the License, or (at your option) any later version.
%;
%;  This program is distributed in the hope that it will be useful,
%;  but WITHOUT ANY WARRANTY; without even the implied warranty of
%;  MERCHANTABILITY or FITNESS FOR A PARTICULAR PURPOSE.  See the
%;  GNU General Public License for more details.
%;
%;  You should have received a copy of the GNU General Public
%;  License along with this program; if not, write to the Free
%;  Software Foundation, Inc., 675 Massachusetts Ave, Cambridge,
%;  MA 02139, USA.
%;
%;  Correspondence concerning AIPS should be addressed as follows:
%;         Internet email: aipsmail@nrao.edu.
%;         Postal address: AIPS Project Office
%;                         National Radio Astronomy Observatory
%;                         520 Edgemont Road
%;                         Charlottesville, VA 22903-2475 USA
%-----------------------------------------------------------------------
\input COOK82.MAC
\input psfig
\def\doFIG{T}
\def\titlea{15-Oct-1998 and earlier}

\cbegint
\if T\doFIG
   \centerline{\hss\psfig{figure=FIG/MONKEY.PLT,height=14cm}\hss}
\vfill
\else
   \vbox to 14cm{\vfill\omit\vfill}
\fi
\cbeginb
\def\tpage{T}
\vglue100pt
\tsect{ACKNOWLEDGMENTS}
\vskip 20pt

     The \COOKBOOK\ cover design is by Pat Smiley of the NRAO Graphic
Arts Department.  It is based on a design suggested by John Bally of
Bell Labs.

     The image on the title page was converted from the \AIPS\
television-like display to Encapsulated PostScript by the \AIPS\ task
\hbox{{\tt \tndx{TVCPS}}}.  It was then included in this \TEX\
\iodx{TEX} document (by {\tt psfig}) and plotted on a Hewlett Packard
LaserJet 4 PostScript printer.  It is the green portion of the
digitized image of a Mandrill which has become a standard in the
image-processing field.  PostScript is a registered trademark of Adobe
Systems Incorporated.

     The plot in Chapter~4 was generated with {\tt TVCPS} in a similar
manner.  The plots at the ends of Chapters 6, 7, 8, and 9 were
generated by various \AIPS\ plotting tasks with task {\tt
\tndx{LWPLA}} used to convert the device-independent plot files into
PostScript.  The data displayed in Chapters 6 and 7 were provided by
Bill Cotton for use with the \AIPS\ {\tt VLAC} test suite and by Alan
Bridle for use with the \AIPS\ {\tt DDT} test suite, while the data in
Chapter~8 were provided by Don Wells for use in testing spectral-line
software.  The editors thank these people for providing their data for
our use.

     This \iodx{COOKBOOK}\COOKBOOK\ itself is based on an early users
guide written by Alan Bridle.  In 1983, it was typeset and edited by
Eric Greisen using the \TEX\ program, initially developed by Donald
E.~Knuth ({\it The \TEX book}, 1984, Addison-Wesley Publishing
Company, Reading, Massachusetts).  There were two editions of the
\COOKBOOK\ in 1983, one in 1985, and one in 1986.  The last, and most
recent, previous edition was edited in 1990 by Bill Junor.  For the
present edition, Eric Greisen has resumed his r\^ole as editor, while
numerous individuals have contibuted to the text.  In particulare,
Glen Langston and Andrea Cox have submitted outline guides to
continuum and spectral-line data reduction which appear as Appendices
A and \hbox{B}.  The output of \TEX\ is now converted to PostScript by
{\tt dvips} (from Radical Eye Software) and printed on a Hewlett
Packard LaserJet 4 printer.  The editors are grateful to Knuth for
this program and, especially, for his decision to place it in the
public domain.

{\lgpoint
This \Cookbook\ is now available on the Internet via the
\indx{World-Wide Web}.  The current Table of Contents together with a
revision history for the full {\tt 15APR98} \Cookbook\ is available at
\indx{URL}$$ \hbox{{\tt http://www.cv.nrao.edu/aips/cook.html}.}$$ You
should review this Web page occasionally to see if chapters important
to you have been  altered and, if so, why.  You may use your favorite
Web browser to click on any chapter you wish to receive and the
\indx{PostScript} version will (eventually) appear on your
workstation.}

\sect{TABLE OF CONTENTS}
\def\titlea{15-OCT-1998 (revised 23-August-1998)}
\count0=-1

\tcl   {1.}     {INTRODUCTION%                                {1\xt1}
}                     {{\tt 15APR98}\hskip 0.6cm 1\xt1}
\tcls  {1.1.}   {The NRAO \AIPS\ Project --- A Summary}       {1\xt1}
\tcls  {1.2.}   {The \COOKBOOK}                               {1\xt3}
\tcls  {1.3.}   {Organization of the \COOKBOOK}               {1\xt4}
\tclss {1.3.1.} {Contents}                                    {1\xt4}
\tclss {1.3.2.} {Minimum match}                               {1\xt5}
\tclss {1.3.3.} {Fonts and what they signify}                 {1\xt5}
\tcls  {1.4.}   {General structure of \AIPS}                  {1\xt5}

\tcl   {2.}     {STARTING UP \AIPS%                          {2\xt1}
}                    {{\tt 15APR98}\hskip 0.6cm 2\xt1}
\tcls  {2.1.}   {Obtaining access to an \AIPS\ computer}      {2\xt1}
\tcls  {2.2.}   {Using the workstation}                       {2\xt1}
\tclss {2.2.1.} {Logging in to the workstation}               {2\xt1}
\tclss {2.2.2.} {Control characters on the workstation}       {2\xt2}
\tclss {2.2.3.} {Starting the {\tt AIPS} program}             {2\xt3}
\tclss {2.2.3.} {Typing commands to the {\tt AIPS} program}   {2\xt8}
\tcls  {2.3.}   {Managing windows}                            {2\xt9}
\tclss {2.3.1.} {General window management}                   {2\xt9}
\tclss {2.3.2.} {Managing the \AIPS\ TV window called XAS}    {2\xt10}

\tcl   {3.}     {BASIC \AIPS\ UTILITIES%                      {3\xt1}
}                    {{\tt 15APR98}\hskip 0.6cm 3\xt1}
\tcls  {3.1.}   {Talking to \AIPS }                           {3\xt1}
\tclss {3.1.1.} {\POPS\ and \AIPS\ utilities}                 {3\xt1}
\tclss {3.1.2.} {Tasks}                                       {3\xt1}
\tclss {3.1.3.} {Verbs}                                       {3\xt2}
\tclss {3.1.4.} {Adverbs}                                     {3\xt2}
\tcls  {3.2.}   {Your \AIPS\ message file}                    {3\xt3}
\tcls  {3.3.}   {Your \AIPS\ data catalog files}              {3\xt4}
\tclss {3.3.1.} {Speedy data file selection}                  {3\xt5}
\tclss {3.3.2.} {Catalog entry status}                        {3\xt5}
\tclss {3.3.3.} {Renaming data files}                         {3\xt6}
\tclss {3.3.4.} {Header listings}                             {3\xt6}
\tcls  {3.4.}   {Your \AIPS\ history files}                   {3\xt8}
\tcls  {3.5.}   {Saving and restoring inputs}                 {3\xt8}
\tcls  {3.6.}   {Monitoring disk space}                       {3\xt9}
\tcls  {3.7.}   {Moving and compressing files}                {3\xt11}
\tcls  {3.8.}   {Finding helpful information in \AIPS}        {3\xt11}
\tcls  {3.9.}   {Magnetic tapes}                              {3\xt14}
\tclss {3.9.1.} {Hardware tape mount}                         {3\xt14}
\tclss {3.9.2.} {Software mounting local tapes}               {3\xt14}
\tclss {3.9.3.} {Software mounting REMOTE tapes}              {3\xt15}
\tclss {3.9.4.} {Using tapes in \AIPS}                        {3\xt15}
\tcls  {3.10.}  {\AIPS\ external disk files}                  {3\xt16}
\tclss {3.10.1.} {Disk text files}                            {3\xt16}
\tclss {3.10.2.} {{\tt RUN} files}                            {3\xt17}
\tclss {3.10.3.} {FITS disk files}                            {3\xt17}
\tcls  {3.11.}  {Additional recipe}                           {3\xt18}
\eject

\tcl   {4.}     {CALIBRATING INTERFEROMETER DATA%             {4\xt1}
}                    {{\tt 15OCT98}\hskip 0.6cm 4\xt1}
\tcls  {4.1.}   {Copying data into \AIPS\ multi-source disk
                     files}                                   {4\xt2}
\tclss {4.1.1.} {Reading from a VLA archive tape using
                     {\tt FILLM}}                             {4\xt3}
\tclss {4.1.2.} {Reading from a FITS tape with {\tt FITLD}}   {4\xt7}
\tcls  {4.2.}   {Record keeping and data management}          {4\xt7}
\tclss {4.2.1.} {Calibrating data with multiple {\tt FQ}
                     entries}                                 {4\xt7}
\tclss {4.2.2.} {Recommended record keeping}                  {4\xt8}
\tcls  {4.3.}   {Beginning the calibration}                   {4\xt9}
\tclss {4.3.1.} {Initial editing}                             {4\xt9}
\tclss {4.3.2.} {Primary flux density calibrators}            {4\xt11}
\tclss {4.3.3.} {First pass of the gain calibration}          {4\xt12}
\tcls  {4.4.}   {Assessing the data quality and initial
                     editing}                                 {4\xt15}
\tclss {4.4.1.} {Editing with {\tt LISTR} and {\tt UVFLG}}    {4\xt16}
\tclss {4.4.2.} {Editing with {\tt EDITA}}                    {4\xt18}
\tclss {4.4.3.} {Editing with {\tt TVFLG}}                    {4\xt20}
\tclss {4.4.4.} {Baseline corrections}                        {4\xt27}
\tcls  {4.5.}   {Antenna-based complex gain solutions}        {4\xt28}
\tclss {4.5.1.} {Bootstrapping secondary flux-density
                     calibrators}                             {4\xt28}
\tclss {4.5.2.} {Full calibration}                            {4\xt29}
\tclss {4.5.3.} {Final (?) initial global calibration}        {4\xt30}
\tcls  {4.6.}   {Polarization calibration}                    {4\xt32}
\tcls  {4.7.}   {Spectral-line calibration}                   {4\xt36}
\tclss {4.7.1.} {Reading the data}                            {4\xt36}
\tclss {4.7.2.} {Editing the data}                            {4\xt37}
\tclss {4.7.3.} {Bandpass calibration}                        {4\xt38}
\tclss {4.7.4.} {Amplitude and phase calibration}             {4\xt40}
\tcls  {4.8.}   {Solar data calibration}                      {4\xt41}
\tclss {4.8.1.} {Reading solar data from a VLA archive tape}  {4\xt41}
\tclss {4.8.2.} {Assessment of the nominal sensitivities using
                     {\tt SNPLT} and {\tt LISTR}}             {4\xt42}
\tclss {4.8.3.} {Applying the system-temperature correction with
                     {\tt SOLCL}}                             {4\xt43}
\tcls  {4.9.}   {Completing the initial calibration}          {4\xt43}
\tclss {4.9.1.} {Writing multi-source data to FITS tape with {\tt
                     FITTP}}                                  {4\xt44}
\tclss {4.9.2.} {Creating single-source data files with
                     {\tt SPLIT}}                             {4\xt44}
\tclss {4.9.3.} {Making images from multi-source data with
                     {\tt IMAGR}}                             {4\xt45}
\tcls  {4.10.}  {Additional recipes}                          {4\xt46}

\tcl   {5.}     {MAKING IMAGES FROM INTERFEROMETER DATA%      {5\xt1}
}                    {{\tt 15OCT98}\hskip 0.6cm 5\xt1}
\tcls  {5.1.}   {Preparing \uv\ data for imaging}             {5\xt1}
\tclss {5.1.1.} {Indexing the data --- {\tt PRTTP}}           {5\xt1}
\tclss {5.1.2.} {Loading the data --- {\tt FITLD} and
                     {\tt UVLOD}}                             {5\xt2}
\tclss {5.1.3.} {Sorting the data --- {\tt UVSRT}}            {5\xt3}
\tcls  {5.2.}   {Basic image making --- {\tt IMAGR}}          {5\xt4}
\tclss {5.2.1.} {Making a simple image}                       {5\xt4}
\tclss {5.2.2.} {Imaging multiple fields and image
                     coordinates}                             {5\xt6}
\tclss {5.2.3.} {Data weighting}                              {5\xt7}
\tclss {5.2.4.} {Cell and image size, shifting}               {5\xt9}
\tclss {5.2.5.} {Zero spacing issues}                         {5\xt10}
\tcls  {5.3.}   {Deconvolving images}                         {5\xt10}
\tclss {5.3.1.} {Basic Cleaning with {\tt IMAGR}}             {5\xt11}
\tclss {5.3.2.} {Multiple fields in {\tt IMAGR}}              {5\xt13}
\tclss {5.3.3.} {Clean boxes and the TV in {\tt IMAGR}}       {5\xt15}
\tclss {5.3.4.} {Data correction options in {\tt IMAGR}}      {5\xt17}
\tclsss {5.3.4.1.} {Frequency-dependent primary beam
                     corrections}                             {5\xt17}
\tclsss {5.3.4.2.} {Frequency-dependent correction for
                     average spectral index}                  {5\xt17}
\tclsss {5.3.4.3.} {Error in the assumed central frequency}   {5\xt18}
\tclsss {5.3.4.4.} {Array mis-orientation effects}            {5\xt18}
\tclsss {5.3.4.5.} {Non-coplanar effects}                     {5\xt18}
\tclsss {5.3.4.6.} {Units mismatch of residuals and Clean
                    components}                               {5\xt18}
\tclss {5.3.5.} {Manipulating Clean components}               {5\xt18}
\tclss {5.3.6.} {Image-plane deconvolution methods --- {\tt APCLN},
                    {\tt SDCLN}, {\tt VTESS}}                 {5\xt20}
\tcls  {5.4.}   {Self-calibration}                            {5\xt21}
\tclss {5.4.1.} {Self-calibration sequence and {\tt SCMAP}}   {5\xt22}
\tclss {5.4.2.} {Self-calibration with {\tt CALIB}}           {5\xt23}
\tclss {5.4.3.} {Considerations in setting {\tt CALIB}
                    inputs}                                   {5\xt25}
\tcls  {5.5.}   {More editing of \uv\ data}                   {5\xt27}
\tclss {5.5.1.} {General remarks on, and tools for, editing}  {5\xt27}
\tclss {5.5.2.} {Baseline-based \uv-data editing ---
                    {\tt EDITR}}                              {5\xt28}

\tcl   {6.}     {DISPLAYING YOUR DATA%                        {6\xt1}
}                    {{\tt 15APR98}\hskip 0.6cm 6\xt1}
\tcls  {6.1.}   {Getting data into your \AIPS\ catalog}       {6\xt1}
\tclss {6.1.1.} {{\tt IMLOD} and {\tt FITLD} from tape}       {6\xt1}
\tclss {6.1.2.} {{\tt IMLOD} and {\tt FITLD} from FITS disk}  {6\xt2}
\tcls  {6.2.}   {Printer displays of your data}               {6\xt3}
\tclss {6.2.1.} {Printing your visibility data}               {6\xt3}
\tclss {6.2.2.} {Printing your image data}                    {6\xt4}
\tclss {6.2.3.} {Printing your table data}                    {6\xt4}
\tclss {6.2.4.} {Printing miscellaneous information}          {6\xt5}
\tcls  {6.3.}   {Plotting your data}                          {6\xt5}
\tclss {6.3.1.} {Plotting your visibility data}               {6\xt6}
\tclss {6.3.2.} {Plotting your image data}                    {6\xt7}
\tclsss {6.3.2.1.} {Contour and grey-scale plots}             {6\xt8}
\tclsss {6.3.2.2.} {Row tracing plots}                        {6\xt9}
\tclsss {6.3.2.3.} {Miscellaneous image plots}                {6\xt11}
\tclss {6.3.3.} {Plotting your table data}                    {6\xt11}
\tclss {6.3.4.} {Plotting miscellaneous information}          {6\xt13}
\tcls  {6.4.}   {Interactive TV displays of your data}        {6\xt14}
\tclss {6.4.1.} {Loading an image to the TV}                  {6\xt14}
\tclss {6.4.2.} {Manipulating the TV display}                 {6\xt15}
\tclss {6.4.3.} {Intensity and color transfer functions}      {6\xt16}
\tclss {6.4.4.} {Setting parameters with the TV}              {6\xt16}
\tclss {6.4.5.} {Reading image values from the TV}            {6\xt17}
\tclss {6.4.6.} {Labeling images on the TV}                   {6\xt17}
\tclss {6.4.7.} {Comparing images on the TV}                  {6\xt18}
\tclss {6.4.8.} {Other functions using the TV}                {6\xt19}
\tclss {6.4.9.} {Capturing the TV}                            {6\xt20}
\tcls  {6.5.}   {Graphics displays of your data}              {6\xt21}
\tclss {6.5.1.} {Plotting data and setting values with
                     the graphics display}                    {6\xt21}
\tclss {6.5.2.} {Slice files and the graphics display}        {6\xt21}
\tclss {6.5.3.} {Data analysis with the graphics display}     {6\xt22}
\tcls  {6.6.}   {Additional recipe}                           {6\xt22}

\tcl   {7.}     {ANALYZING IMAGES%                            {7\xt1}
}                    {{\tt 15APR98}\hskip 0.6cm 7\xt1}
\tcls  {7.1.}   {Combining two images ({\tt COMB})}           {7\xt1}
\tclss {7.1.1.} {Subtracting a continuum image from an image
                      cube}                                   {7\xt1}
\tclss {7.1.2.} {Polarized intensity and position angle images} {7\xt1}
\tclss {7.1.3.} {Other image combination options}             {7\xt2}
\tclss {7.1.4.} {Considerations in image combination}         {7\xt3}
\tcls  {7.2.}   {Combining more than two images {\tt SUMIM}}  {7\xt3}
\tcls  {7.3.}   {Image statistics and flux integration
                     ({\tt IMEAN}, {\tt IMSTAT}, {\tt TVSTAT},
                     {\tt BLSUM})}                            {7\xt4}
\tcls  {7.4.}   {Blanking of images}                          {7\xt4}
\tcls  {7.5.}   {Fitting of images}                           {7\xt5}
\tclss {7.5.1.} {Parabolic fit to maximum ({\tt MAXFIT})}     {7\xt6}
\tclss {7.5.2.} {Two-dimensional Gaussian fitting
                     ({\tt IMFIT})}                           {7\xt6}
\tclss {7.5.3.} {Gaussian fits to slices ({\tt SLFIT})}       {7\xt7}
\tclss {7.5.4.} {Other one-dimensional Gaussian fits
                     ({\tt XGAUS})}                           {7\xt8}
\tclss {7.5.5.} {Source recognition and fitting {\tt SAD}}    {7\xt8}
\tcls  {7.6.}   {Image analysis}                              {7\xt9}
\tclss {7.6.1.} {Geometric conversions}                       {7\xt9}
\tclss {7.6.2.} {Mathematical operations on a single image}   {7\xt10}
\tclss {7.6.3.} {Primary beam correction}                     {7\xt10}
\tclss {7.6.4.} {Filtering}                                   {7\xt11}
\tclss {7.6.5.} {Modeling}                                    {7\xt11}
\tclss {7.6.6.} {Examples}                                    {7\xt12}

\tcl   {8.}     {SPECTRAL-LINE SOFTWARE%                      {8\xt1}
}                    {{\tt 15APR98}\hskip 0.6cm 8\xt1}
\tcls  {8.1.}   {Data preparation and assessment}             {8\xt1}
\tcls  {8.2.}   {Editing and self-calibration}                {8\xt4}
\tcls  {8.3.}   {Continuum subtraction}                       {8\xt5}
\tcls  {8.4.}   {Imaging}                                     {8\xt6}
\tcls  {8.5.}   {Display and manipulation of data cubes}      {8\xt7}
\tclss {8.5.1.} {Building and dismantling data cubes}         {8\xt7}
\tclss {8.5.2.} {Transposing the cube}                        {8\xt9}
\tclss {8.5.3.} {Modifying the image header}                  {8\xt9}
\tclss {8.5.4.} {Displaying the cube}                         {8\xt10}
\tcls  {8.6.}   {Analysis}                                    {8\xt14}
\tcls  {8.7.}   {Additional recipes}                          {8\xt16}

\tcl   {9.}     {REDUCING VLBI DATA IN \AIPS%                 {9\xt1}
}                    {{\tt 15OCT97}\hskip 0.6cm 9\xt1}
\tcls  {9.1.}   {VLBI data reduction recipe}                  {9\xt2}
\tcls  {9.2.}   {Loading and inspecting data}                 {9\xt4}
\tclss {9.2.1.} {Loading data from the VLBA correlator}       {9\xt4}
\tclsss {9.2.1.1.} {Running {\tt FITLD}}                       {9\xt4}
\tclsss {9.2.1.2.} {Running {\tt FITLD} for SVLBI data}        {9\xt7}
\tclsss {9.2.1.3.} {Sorting VLBA correlator data}              {9\xt8}
\tclsss {9.2.1.4.} {Subarraying VLBA correlator data}          {9\xt8}
\tclsss {9.2.1.5.} {Indexing VLBA correlator data}             {9\xt9}
\tclsss {9.2.1.6.} {Concatenating VLBA correlator data}        {9\xt9}
\tclsss {9.2.1.7.} {Labeling VLBA correlator polarization data} {9\xt10}
\tclsss {9.2.1.8.} {Preparing the {\tt OB} table for SVLBI
                     data}                                    {9\xt11}
\tclsss {9.2.1.9.} {Loading the time corrections file for
                     SVLBI data}                              {9\xt11}
\tclss {9.2.2.} {Loading data from a MkIII/MkIV correlator}   {9\xt13}
\tclsss {9.2.2.1.} {Running {\tt MK3IN}}                      {9\xt13}
\tclsss {9.2.2.2.} {Sorting MkIII/IV data}                    {9\xt14}
\tclsss {9.2.2.3.} {Concatenating MkIII/IV data}              {9\xt15}
\tclsss {9.2.2.4.} {Merging MkIII/IV data}                    {9\xt15}
\tclsss {9.2.2.5.} {Correcting MkIII/IV sideband phase
                     offsets}                                 {9\xt16}
\tclsss {9.2.2.6.} {Indexing MkIII/IV data}                   {9\xt16}
\tclss {9.2.3.} {Loading SVLBI data from the Penticton
                     correlator}                              {9\xt16}
\tcls  {9.3.}   {Tools for data examination}                  {9\xt17}
\tclss {9.3.1.} {Textual displays}                            {9\xt17}
\tclss {9.3.2.} {Spectral displays: {\tt POSSM} and
                     {\tt SHOUV}}                             {9\xt18}
\tclss {9.3.3.} {Time displays: {\tt VPLOT} and
                     {\tt CLPLT}}                             {9\xt19}
\tclss {9.3.4.} {{\tt EDITR}}                                 {9\xt19}
\tclss {9.3.5.} {{\tt SNPLT}}                                 {9\xt20}
\tclss {9.3.6.} {{\tt COHER}}                                 {9\xt20}
\tclss {9.3.7.} {{\tt FRPLT}}                                 {9\xt20}
\tcls  {9.4.}   {Calibration strategy}                        {9\xt21}
\tclss {9.4.1.} {Incremental calibration philosophy}          {9\xt22}
\tclss {9.4.2.} {Processing observing log and calibration
                      information}                            {9\xt22}
\tclsss {9.4.2.1.} {Automatic formatting of log files}        {9\xt23}
\tclsss {9.4.2.2.} {VLA log files}                            {9\xt23}
\tclsss {9.4.2.3.} {SVLBI log files}                          {9\xt23}
\tclsss {9.4.2.4.} {Manual formatting of log files}           {9\xt24}
\tclss {9.4.3.} {Data editing}                                {9\xt24}
\tclss {9.4.4.} {{\it a priori\/} calibration}                {9\xt25}
\tclsss {9.4.4.1.} {Digital sampler bias corrections for VLBA
                       correlator data}                       {9\xt25}
\tclsss {9.4.4.2.} {Smoothing and applying corrections}       {9\xt26}
\tclsss {9.4.4.3.} {Continuum amplitude calibration}          {9\xt27}
\tclsss {9.4.4.4.} {Polarization calibration: parallactic
                       angle corrections}                     {9\xt29}
\tclss {9.4.5.} {Bandpass calibration}                        {9\xt29}
\tclss {9.4.6.} {Spectral-line doppler correction}            {9\xt30}
\tclss {9.4.7.} {Spectral-line amplitude calibration}         {9\xt31}
\tclss {9.4.8.} {Phase calibration}                           {9\xt32}
\tclsss {9.4.8.1.} {Special considerations}                   {9\xt33}
\tclsss {9.4.8.2.} {Instrumental phase corrections}           {9\xt34}
\tclsss {9.4.8.2b.} {`Manual' instrumental phase corrections} {9\xt35}
\tclsss {9.4.8.3.} {Antenna-based fringe-fitting}             {9\xt36}
\tclsss {9.4.8.4.} {Baseline-based fringe-fitting}            {9\xt41}
\tclsss {9.4.8.5.} {SVLBI-specific techniques}                {9\xt44}
\tclsss {9.4.8.6.} {Spectral-line fringe-fitting}             {9\xt45}
\tclsss {9.4.8.7.} {Polarization-specific fringe-fitting}     {9\xt47}
\tclssss {9.4.8.7.1.} {R-L delay calibration}                 {9\xt48}
\tclssss {9.4.8.7.2.} {Feed D-term calibration}               {9\xt48}
\tclss {9.4.9.} {Baseline-based errors}                       {9\xt50}
\tcls  {9.5.}   {After initial calibration}                   {9\xt50}
\tclss {9.5.1.} {Applying calibration}                        {9\xt50}
\tclss {9.5.2.} {Time averaging}                              {9\xt51}
\tclss {9.5.3.} {Verifying calibration}                       {9\xt52}
\tcls  {9.6.}   {Self-calibration, imaging, and model-fitting} {9\xt53}
\tclss {9.6.1.} {{\tt CALIB}}                                 {9\xt54}
\tclss {9.6.2.} {{\tt IMAGR} and {\tt SCMAP}}                 {9\xt55}
\tclss {9.6.3.} {Model-fitting}                               {9\xt56}
\tcls  {9.7.}   {Data simulation within \AIPS}                {9\xt56}
\tcls  {9.8.}   {Summary of VLBI calibration tables}          {9\xt58}

\tcl   {10.}    {SINGLE-DISH DATA IN \AIPS%                   {10\xt1}
}                    {{\tt 15OCT98}\hskip 0.6cm 10\xt1}
\tcls  {10.1.}  {\AIPS\ format for single-dish data}          {10\xt1}
\tclss {10.1.1.} {On-the-fly data from the 12m}               {10\xt2}
\tclsss {10.1.1.1} {Listing OTF input files}                  {10\xt2}
\tclsss {10.1.1.2} {Reading spectral-line OTF files into
                      \AIPS}                                  {10\xt3}
\tclsss {10.1.1.3} {Reading continuum OTF files into \AIPS}   {10\xt4}
\tclss {10.1.2.} {Other input data formats}                   {10\xt4}
\tcls  {10.2.}  {Single-dish data in the ``uv'' domain}       {10\xt5}
\tclss {10.2.1.} {Looking at your data: {\tt PRTSD},
                       {\tt UVPLT}, {\tt POSSM}}              {10\xt5}
\tclss {10.2.2.} {Editing your data: {\tt UVFLG}, {\tt SPFLG},
                       {\tt EDITR}}                           {10\xt7}
\tclss {10.2.3.} {Calibrating your data: {\tt CSCOR},
                       {\tt SDCAL}}                           {10\xt11}
\tclss {10.2.4.} {Correcting your spectral-line data:
                       {\tt SDLSF} and {\tt SDVEL}}           {10\xt12}
\tclss {10.2.5.} {Modeling your data: {\tt SDMOD}}            {10\xt13}
\tcls  {10.3.}  {Imaging single-dish data in \AIPS}           {10\xt14}
\tclss {10.3.1.} {Normal single-dish imaging}                 {10\xt14}
\tclss {10.3.2.} {Beam-switched continuum imaging}            {10\xt16}
\tcls  {10.4.}  {Analysis and display of single-dish data}    {10\xt20}
\tclss {10.4.1.} {Spectral baseline removal}                  {10\xt20}
\tclss {10.4.2.} {Combining images: {\tt WTSUM}, {\tt BSAVG}} {10\xt21}
\tclss {10.4.3.} {Spectral moment analysis}                   {10\xt22}
\tclss {10.4.4.} {Source modeling and fitting}                {10\xt22}
\tclss {10.4.5.} {Image displays}                             {10\xt23}
\tclss {10.4.6.} {Backing up your data}                       {10\xt24}
\tcls  {10.5.}  {Combining single-dish and interferometer
                       data}                                  {10\xt24}
\tcls  {10.6.}  {Additional recipes}                          {10\xt25}

\tcl   {11.}    {EXITING FROM, AND SOLVING PROBLEMS IN, \AIPS% {11\xt1}
}                    {{\tt 15JAN96}\hskip 0.6cm 11\xt1}
\tcls  {11.1.}  {Helping the \AIPS\ programmers}              {11\xt1}
\tcls  {11.2.}  {Exiting from \AIPS}                          {11\xt1}
\tclss {11.2.1.} {Backups}                                    {11\xt2}
\tclss {11.2.2.} {Deleting your data}                         {11\xt2}
\tclss {11.2.3.} {Exiting}                                    {11\xt3}
\tcls  {11.3.}  {Solving problems in using \AIPS}             {11\xt3}
\tclss {11.3.1.} {``Terminal'' problems}                      {11\xt3}
\tclss {11.3.2.} {Disk data problems}                         {11\xt4}
\tclss {11.3.2.} {Printer problems}                           {11\xt5}
\tclss {11.3.2.} {Tape problems}                              {11\xt5}
\tcls  {11.4.}  {Additional recipe}                           {11\xt6}

\tcl   {12.}    {\AIPS\ FOR THE MORE SOPHISTICATED USER%      {12\xt1}
}                    {{\tt 15JAN96}\hskip 0.6cm 12\xt1}
\tcls  {12.1.}   {\AIPS\ conventions}                         {12\xt1}
\tclss {12.1.1.} {\AIPS\ shortcuts}                           {12\xt1}
\tclss {12.1.2.} {Data-file names and formats}                {12\xt2}
\tcls  {12.2.}   {Process control features of \AIPS}          {12\xt3}
\tclss {12.2.1.} {{\tt RUN} files}                            {12\xt3}
\tclss {12.2.2.} {More about {\tt GO}}                        {12\xt4}
\tclss {12.2.3.} {Batch jobs}                                 {12\xt5}
\tcls  {12.3.}   {\AIPS\ language}                            {12\xt6}
\tclss {12.3.1.} {Using \POPS\ outside of procedures}         {12\xt8}
\tclss {12.3.2.} {Procedures}                                 {12\xt9}
\tclss {12.3.3.} {Writing your own programs with \POPS}       {12\xt12}
\tcls  {12.4.}   {Remote use of \AIPS}                        {12\xt13}
\tclss {12.4.1.} {Connections via X-Windows}                  {12\xt13}
\tclss {12.4.2.} {Connections to a terminal}                  {12\xt14}
\tclss {12.4.3.} {Remote data connections}                    {12\xt15}
\tclss {12.4.4.} {File transfer connections}                  {12\xt16}
\tcls  {12.5.}   {Adding your own tasks to \AIPS}             {12\xt17}
\tclss {12.5.1.} {Initial choices to make}                    {12\xt17}
\tclss {12.5.2.} {Getting started}                            {12\xt18}
\tclss {12.5.3.} {Initial check of code and procedures}       {12\xt18}
\tclss {12.5.4.} {Modifying an \AIPS\ task}                   {12\xt19}
\tclss {12.5.5.} {Modifying an \AIPS\ template task}          {12\xt20}
\tclss {12.5.6.} {Further remarks}                            {12\xt21}
\tcls  {12.6.}   {Additional recipes}                         {12\xt22}

\tcl   {13.}    {CURRENT \AIPS\ SOFTWARE%                     {13\xt1}
}                    {{\tt 15APR98}\hskip 0.6cm 13\xt1}
\tcls  {13.1.}  {ADVERB}                                      {13\xt1}
\tcls  {13.2.}  {ANALYSIS}                                    {13\xt8}
\tcls  {13.3.}  {AP}                                          {13\xt10}
\tcls  {13.4.}  {ASTROMET}                                    {13\xt10}
\tcls  {13.5.}  {BATCH}                                       {13\xt11}
\tcls  {13.6.}  {CALIBRAT}                                    {13\xt12}
\tcls  {13.7.}  {CATALOG}                                     {13\xt15}
\tcls  {13.8.}  {COORDINA}                                    {13\xt17}
\tcls  {13.9.}  {EDITING}                                     {13\xt17}
\tcls  {13.10.} {EXT-APPL}                                    {13\xt18}
\tcls  {13.11.} {FITS}                                        {13\xt18}
\tcls  {13.12.} {GENERAL}                                     {13\xt18}
\tcls  {13.13.} {HARDCOPY}                                    {13\xt20}
\tcls  {13.14.} {IMAGE-UT}                                    {13\xt21}
\tcls  {13.15.} {IMAGE}                                       {13\xt21}
\tcls  {13.16.} {IMAGING}                                     {13\xt21}
\tcls  {13.17.} {INFORMAT}                                    {13\xt24}
\tcls  {13.18.} {INTERACT}                                    {13\xt25}
\tcls  {13.19.} {MODELING}                                    {13\xt26}
\tcls  {13.20.} {OBSOLETE}                                    {13\xt26}
\tcls  {13.21.} {ONED}                                        {13\xt27}
\tcls  {13.22.} {OOP}                                         {13\xt27}
\tcls  {13.23.} {OPTICAL}                                     {13\xt28}
\tcls  {13.24.} {PARAFORM}                                    {13\xt28}
\tcls  {13.25.} {PLOT}                                        {13\xt28}
\tcls  {13.26.} {POLARIZA}                                    {13\xt30}
\tcls  {13.27.} {POPS}                                        {13\xt31}
\tcls  {13.28.} {PROCEDUR}                                    {13\xt33}
\tcls  {13.29.} {PSEUDOVE}                                    {13\xt33}
\tcls  {13.30.} {RUN}                                         {13\xt34}
\tcls  {13.31.} {SINGLEDI}                                    {13\xt34}
\tcls  {13.32.} {SPECTRAL}                                    {13\xt35}
\tcls  {13.33.} {TABLE}                                       {13\xt36}
\tcls  {13.34.} {TAPE}                                        {13\xt36}
\tcls  {13.35.} {TASK}                                        {13\xt37}
\tcls  {13.36.} {TV}                                          {13\xt44}
\tcls  {13.37.} {TV-APPL}                                     {13\xt46}
\tcls  {13.38.} {UTILITY}                                     {13\xt47}
\tcls  {13.39.} {UV}                                          {13\xt48}
\tcls  {13.40.} {VERB}                                        {13\xt51}
\tcls  {13.41.} {VLA}                                         {13\xt55}
\tcls  {13.42.} {VLBI}                                        {13\xt56}
\tcls  {13.43.} {Additional recipes}                          {13\xt57}

\tcl    {A.}    {Summary of \AIPS\ Continuum UV-data Calibration%
}                    {{\tt 15JUL94}\hskip 0.6cm A\xt1}
\tcls   {A.1.}  {Basic calibration}                            {A\xt1}
\tcls   {A.2.}  {Polarization calibration}                     {A\xt4}
\tcls   {A.3.}  {Backup and imaging}                           {A\xt4}

\tcl    {B.}    {A Step-by-Step Guide to Spectral-Line Data
                     Analysis in \AIPS%
}                    {{\tt 15JUL94}\hskip 0.6cm B\xt1}
\tcls   {B.1}   {Editing and calibrating spectral-line data}   {B\xt1}
\tclss  {B.1.a} {Loading the data}                             {B\xt1}
\tclss  {B.1.b} {Inspecting and editing the data}              {B\xt2}
\tclss  {B.1.c} {Calibrating the data}                         {B\xt3}
\tcls   {B.2.}  {Making and Cleaning map cubes}                {B\xt4}
\tcls   {B.3.}  {Moment analysis and rotation curves of
                      galaxies}                                {B\xt6}
\tcls   {B.4.}  {Multi-frequency observations}                 {B\xt7}

\tcl    {G.}    {GLOSSARY%                                     {G\xt1}
}                    {{\tt 15OCT90}\hskip 0.6cm G\xt1}

\tcl    {Y.}     {FILE SIZES%                                  {Y\xt1}
}                    {{\tt 15JUL95}\hskip 0.6cm Y\xt1}
\tcls   {Y.1.}   {Visibility (\uv) data sets}                  {Y\xt1}
\tclss  {Y.1.1.} {\uv\ database sizes}                         {Y\xt1}
\tclss  {Y.1.2.} {Compressed format for \uv\ data}             {Y\xt2}
\tcls   {Y.2.}   {Image files}                                 {Y\xt3}
\tcls   {Y.3.}   {Extension files}                             {Y\xt3}
\tcls   {Y.4.}   {Storing data on tape}                        {Y\xt4}
\tclss  {Y.4.1.} {9-track tapes}                               {Y\xt5}
\tclss  {Y.4.2.} {DAT and Exabyte tapes}                       {Y\xt6}
\tcls   {Y.5.}   {Very large data sets}                        {Y\xt6}

\tcl    {Z.}       {SYSTEM-DEPENDENT \AIPS\ TIPS%              {Z\xt1}
}                    {{\tt 15JUL94}\hskip 0.6cm Z\xt1}
\tcls   {Z.1.}     {NRAO workstations --- general information} {Z\xt1}
\tclss  {Z.1.1.}   {The ``midnight'' jobs}                     {Z\xt1}
\tclss  {Z.1.2.}   {Generating color hard copy}                {Z\xt2}
\tclsss {Z.1.2.1.} {Color printers}                            {Z\xt2}
\tclsss {Z.1.2.2.} {Software to copy your screen}              {Z\xt2}
\tclsss {Z.1.2.3.} {Color film recorders}                      {Z\xt3}
\tclss  {Z.1.3.}   {Gripe, gripe, gripe $\ldots$}              {Z\xt4}
\tclss  {Z.1.4.}   {Solving problems at the NRAO}              {Z\xt5}
\tclsss {Z.1.4.1.} {Booting the workstations}                  {Z\xt5}
\tclsss {Z.1.4.2.} {Printout fails to appear}                  {Z\xt5}
\tclsss {Z.1.4.3.} {Stopping excess printout}                  {Z\xt5}
\tclsss {Z.1.4.4.} {{\tt CTRL~Z} problems}                     {Z\xt6}
\tclsss {Z.1.4.5.} {``File system is full'' message}           {Z\xt7}
\tclsss {Z.1.4.6.} {Tapes won't mount}                         {Z\xt7}
\tclsss {Z.1.4.7.} {I can't use my data disk}                  {Z\xt7}
\tcls   {Z.2.}     {\AIPS\ at the NRAO in Charlottesville}     {Z\xt8}
\tclss  {Z.2.1.}   {Using the Charlottesville workstations}    {Z\xt8}
\tclsss {Z.2.1.1.} {Signing up for \AIPS\ time in
                                             Charlottesville}  {Z\xt8}
\tclsss {Z.2.1.2.} {Managing workstation windows in
                                             Charlottesville}  {Z\xt8}
\tclsss {Z.2.1.3.} {Data disk management in Charlottesville}   {Z\xt9}
\tclss  {Z.2.2.}   {Using the tape drives in Charlottesville}  {Z\xt10}
\tclsss {Z.2.2.1.} {Mounting and removing tapes on 9-track
                                                      drives}  {Z\xt10}
\tclsss {Z.2.2.2.} {Mounting tapes on Exabyte and DAT drives}  {Z\xt10}
\tclss  {Z.2.3.}   {Color hard copy in Charlottesville}        {Z\xt11}
\tcls   {Z.3.}     {\AIPS\ at the NRAO AOC in Socorro}         {Z\xt12}
\tclss  {Z.3.1.}   {Using the AOC workstations}                {Z\xt12}
\tclsss {Z.3.1.1.} {Signing up for \AIPS\ time in Socorro}     {Z\xt12}
\tclsss {Z.3.1.2.} {Managing workstation windows at the AOC}   {Z\xt12}
\tclssss {Z.3.1.2.1.} {IBM workstation windows at the AOC}     {Z\xt12}
\tclssss {Z.3.1.2.1.} {SUN workstation windows at the AOC}     {Z\xt13}
\tclsss {Z.3.1.3.} {Data disk management at the AOC}           {Z\xt14}
\tclss  {Z.3.2.}   {Using the tape drives at the AOC}          {Z\xt14}
\tclsss {Z.3.2.1.} {Mounting and removing tapes on 9-track
                                                      drives}  {Z\xt14}
\tclsss {Z.3.2.2.} {Mounting tapes on Exabyte drives at the
                                                      AOC}     {Z\xt15}
\tclsss {Z.3.2.3.} {Mounting tapes on DAT drives at the AOC}   {Z\xt15}
\tclss  {Z.3.3.}   {Color hard copy at the AOC}                {Z\xt16}
\tcls   {Z.4.}     {\AIPS\ at the NRAO Very Large Array site}  {Z\xt16}
\tclss  {Z.4.1.}   {Using the VLA workstation}                 {Z\xt16}
\tclsss {Z.4.1.1.} {Signing up for \AIPS\ time at the VLA}     {Z\xt16}
\tclsss {Z.4.1.2.} {Managing workstation windows at the VLA}   {Z\xt17}
\tclsss {Z.4.1.3.} {Data disk management at the VLA}           {Z\xt17}
\tclss  {Z.4.2.}   {Using the tape drives at the VLA}          {Z\xt17}
\tcls   {Z.5.}     {Additional recipes}                        {Z\xt18}

\tcl    {I.}       {INDEX%                                     {I\xt1}
}                    {{\tt 15JUL95}\hskip 0.6cm I\xt1}
%\tcls  {I.1.}  {Additional recipe}                             {I\xt13}

%\vfill
%\eject
%\end

\alphasect{TABLE OF FIGURES}

\tclss {1.4.}   {General structure of \AIPS}                   {1\xt6}
\tclss {4.4.2.} {{\tt EDITA}}                                  {4\xt20}
\tclss {4.4.3.} {{\tt TVFLG}}                                  {4\xt23}
\tclss {5.2.3.} {Synthesized beam patterns versus {\tt ROBUST}} {5\xt9}
\tclss {6.3.1.} {{\tt UVPLT} $u$ versus $v$}                   {6\xt7}
\tclss {6.3.1.} {{\tt UVPLT} baseline length versus amplitude} {6\xt7}
%\tclss {6.3.1.} {{\tt UVPLT} $u$ versus $v$ {\it and}
%                {\tt UVPLT} baseline length versus amplitude} {6\xt7}
%\tclss {6.3.2.1.} {{\tt CNTR}}                                 {6\xt10}
%\tclss {6.3.2.1.} {{\tt PCNTR}}                                {6\xt10}
%\tclss {6.3.2.1.} {{\tt GREYS}}                                {6\xt10}
%\tclss {6.3.2.1.} {{\tt GREYS} with contours}                  {6\xt10}
\tclss {6.3.2.1.} {{\tt CNTR}, {\tt PCNTR}, {\tt GREYS}, {\it and}
                  {\tt GREYS} with contours}                  {6\xt10}
\tclss {6.3.2.2.} {{\tt PLROW}}                                {6\xt12}
\tclss {6.3.2.2.} {{\tt PROFL}}                                {6\xt12}
%\tclss {6.3.2.2.} {{\tt PLROW} {\it and} {\tt PROFL}}         {6\xt12}
\tclss {6.3.2.3.} {{\tt IMVIM} (binned)}                       {6\xt12}
\tclss {6.3.2.3.} {{\tt IMEAN}}                                {6\xt12}
%\tclss {6.3.2.3.} {{\tt IMVIM} (binned) {\it and} {\tt IMEAN}} {6\xt12}
\tclss {6.3.2.2.} {{\tt SL2PL}}                                {6\xt13}
\tclss {6.3.3.} {{\tt TAPLT}}                                  {6\xt13}
\tclss {7.6.6.} {Input image ({\tt CNTR})}                     {7\xt12}
\tclss {7.6.6.} {{\tt LGEOM} rotates image ({\tt CNTR})}       {7\xt12}
\tclss {7.6.6.} {{\tt PGEOM} on image ({\tt CNTR})}            {7\xt12}
\tclss {7.6.6.} {{\tt NINER} differentiates image
                                               ({\tt GREYS})}  {7\xt12}
%\tclss {7.6.6.} {Input image ({\tt CNTR}) {\it and}
%                {\tt LGEOM} rotates image ({\tt CNTR})}       {7\xt12}
%\tclss {7.6.6.} {{\tt PGEOM} on image ({\tt CNTR}) {\it and}
%                {\tt NINER} differentiates image
%                                              ({\tt GREYS})}  {7\xt12}
\tclss {8.5.4.} {{\tt KNTR}}                                   {8\xt11}
\tclss {8.5.4.} {{\tt TVCUBE}}                                 {8\xt12}
\tclss {8.5.4.} {{\tt PLCUB}}                                  {8\xt13}
\tclss {8.6.}   {{\tt XMOM} ({\tt GREYS} with contours)}       {8\xt15}
\tclss {9.2.1.} {{\tt TAPLT} of Delta-T versus table row}      {9\xt12}
\tclss {9.4.8.} {{\tt POSSM} multi-IF before calibration}      {9\xt37}
\tclss {9.4.8.} {{\tt VPLOT} before calibration}               {9\xt37}
\tclss {9.4.8.} {{\tt POSSM} before and after calibration}     {9\xt42}
\tclss {9.7.}   {Performance of {\tt FRING} versus S/N}        {9\xt57}
\tclss {9.8.}   {{\tt CNTR} VLBA correlator first continuum
                        image}                                 {9\xt59}
\tclss {9.8.}   {{\tt CNTR} VLBA correlator first spectral-line
                        image}                                 {9\xt60}
\tclss {10.2.1.} {{\tt UVPLT} single-dish flux and coordinate
                        plots}                                 {10\xt7}
\tclss {10.2.1.} {{\tt POSSM} single-dish flux spectrum}       {10\xt7}
\tclss {10.2.2.} {{\tt SPFLG} single-dish data editing}        {10\xt9}
\tclss {10.3.1.} {Effect of convolving functions on single-dish
                       imaging}                                {10\xt16}
\tclss {10.3.2.} {Effect of rotation on beam-switched imaging} {10\xt19}
\tclss {10.3.2.} {Effect of throw length on beam-switched
                        imaging}                               {10\xt20}
\tclss {10.6.3} {The Moon via beam-switched imaging}           {10\xt26}
\tclss {A.1.}   {{\tt UVPLT} before and after calibration}     {A\xt3}
\vfill
\eject
\end

\alphasect{TABLE OF RECIPES}


\tclss  {0.5.1.}    {Banana storage}                          {0\xt$x$}
\tclss  {3.11.1.}   {Banana daiquiri}                         {3\xt18}
\tclss  {4.10.1.}   {Sopa de Pl\'atano}                       {4\xt46}
\tclss  {4.10.2.}   {Panecillos de Pl\'atano}                 {4\xt46}
\tclss  {6.6.1.}    {Bananes r\^ oties}                       {6\xt22}
\tclss  {8.7.1.}    {Churros de Pl\'atano}                    {8\xt16}
\tclss  {8.7.2.}    {Banana coffeelate}                       {8\xt16}
\tclss  {8.7.3.}    {Orange gingered bananas}                 {8\xt16}
\tclss  {10.6.1.}   {Orange baked bananas}                    {10\xt25}
\tclss  {10.6.2.}   {Banana colada}                           {10\xt26}
\tclss  {11.4.1.}   {Banana pick-me-up}                       {11\xt6}
\tclss  {12.6.1.}   {Hawaiian banana cream pie}               {12\xt22}
\tclss  {12.6.2.}   {Banana nut bread}                        {12\xt22}
\tclss  {12.6.3.}   {Chewy banana split dessert}              {12\xt22}
\tclss  {13.43.1.}  {Banana storage}                          {13\xt57}
\tclss  {13.43.2.}  {Banana bran muffins}                     {13\xt57}
\tclss  {13.43.3.}  {Little banana cream tarts}               {13\xt58}
\tclss  {13.43.4.}  {Golden mousse}                           {13\xt58}
\tclss  {13.43.5.}  {Dulce Zacateca\~no}                      {13\xt58}
\tclss  {13.43.6.}  {Frozen Push-Ups}                         {13\xt58}
\tclss  {Z.5.1.}    {Delightful banana cheesecake}            {Z\xt18}
\tclss  {Z.5.2.}    {Banana poundcake}                        {Z\xt18}

\vfill
\eject
\end
