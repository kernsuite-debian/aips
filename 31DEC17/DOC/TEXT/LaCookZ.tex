%-----------------------------------------------------------------------
%;  Copyright (C) 1995-1996, 1998, 2000-2001, 2004, 2007, 2013-2014
%;  Associated Universities, Inc. Washington DC, USA.
%;
%;  This program is free software; you can redistribute it and/or
%;  modify it under the terms of the GNU General Public License as
%;  published by the Free Software Foundation; either version 2 of
%;  the License, or (at your option) any later version.
%;
%;  This program is distributed in the hope that it will be useful,
%;  but WITHOUT ANY WARRANTY; without even the implied warranty of
%;  MERCHANTABILITY or FITNESS FOR A PARTICULAR PURPOSE.  See the
%;  GNU General Public License for more details.
%;
%;  You should have received a copy of the GNU General Public
%;  License along with this program; if not, write to the Free
%;  Software Foundation, Inc., 675 Massachusetts Ave, Cambridge,
%;  MA 02139, USA.
%;
%;  Correspondence concerning AIPS should be addressed as follows:
%;          Internet email: aipsmail@nrao.edu.
%;          Postal address: AIPS Project Office
%;                          National Radio Astronomy Observatory
%;                          520 Edgemont Road
%;                          Charlottesville, VA 22903-2475 USA
%-----------------------------------------------------------------------
\setcounter{chapter}{25}
\appen{System-Dependent \AIPS\ Tips}{sys}
\renewcommand{\Chapt}{30}

\renewcommand{\titlea}{31-DEC-2014 (revised 9-April-2014)}
\renewcommand{\Rheading}{\AIPS\ \cookbook:~\titlea\hfill}
\renewcommand{\Lheading}{\hfill \AIPS\ \cookbook:~\titlea}
\markboth{\Lheading}{\Rheading}

{\it This appendix has been updated.  The information contained herein
should be indicative of current conditions, but not correct in
detail.  This is particularly true of the Charlottesville and Green
Bank information.}

     Although \AIPS\ attempts to be system independent, some aspects
of its use depend inevitably on the specific site.  These vary from
procedural matters (\eg\ assignment of workstations and location of
sign-up sheets, tape drives, and workstations or other terminals) to
the hardware (\eg\ names and numbers of workstations and tape and disk
drives) to the peculiar (\eg\ the response of the computer to specific
keys on the terminal, the presence of useful job control procedures).
This appendix contains information specific to the NRAO's individual
\AIPS\ installations.  It is intended that non-NRAO installations
replace this appendix with one describing their own procedures,
perhaps using this version as a template.  The general description of
using \AIPS\ on workstations was given in Chapter 2 and will not be
repeated here.

    Within the NRAO, \AIPS\ is installed on two main architectures ---
Linux PCs and Mac workstations.  All our old SUN, IBM RS/6000, CONVEX
C--1 and DEC VAX 11/750 and 11/780 systems have been decommissioned.
Currently the fastest \AIPS\ machines in the Observatory are the
whichever PC was bought last.

\sects{NRAO workstations  --- general information}

     All NRAO workstations run some version of the Unix operating
system, Linux on PCs and Mac OS/X on Apple products.  Unix systems are
intrinsically sensitive to the difference between upper and lower
case.  Be sure to use the case indicated in the comments and advice
given in the following notes.  \AIPS\ itself is case-insensitive,
however; conversion of lower-case characters to upper-case occurs
automatically.  (Unix systems have a variety of characters for the
prompt at monitor (job-control) level, and allow users to set their
own as well.  We will use {\tt \$} as the prompt in the text below.)

\subsections{The ``midnight'' jobs}

    The versions of \AIPS\ on all NRAO PC systems are kept up to date
continually with the master versions on the Socorro Linux PC called
{\tt dave}.  This is achieved by automated jobs that start running at
very antisocial hours of the early morning.  Any changes formally made
to the {\us TST} version of \AIPS\ are copied to the relevant
computers and recompiled/relinked.  Midnight jobs run in
Charlottesville, Socorro, Green Bank, and many other sites around the
world.

\Subsections{Generating color hard copy}{Zhardcopy}

\subsubsections{Color printers}

     Color printers are, these days, simply printers that understand
the color extensions to the PostScript language used to describe
plots.  The NRAO owns several Tektronix \indx{color printer}s, two
public ones in Charlottesville ({\tt ps1color} in the \AIPS\ Caige and
{\tt ps3color} in the Library) and two at the AOC in Socorro ({\tt
ps213c} on the ground floor and {\tt ps324c} upstairs in the former
library).  You may display your PostScript file on the printer in
Socorro simply by typing
\dispx{{\tt \$}\qs lpr\qs -Pps324c\qs {\it filename\/} \CR}{where
        {\it filename} is the name of your file.}
\dispe{The paper size is $8.5 \times 11$ inches, which is the default
for \AIPS\ tasks {\tt TVCPS} and {\tt LWPLA}\@.   To have the file
printed on transparency paper use queue {\tt ps1tektran} rather than
{\tt aoc324c/trans}.  Full control over this complex printer is
available with the {\tt multiprint} command; type {\tt multiprint
--help \CR} for information.  If you do not wish to save the plot as a
disk file, you may also print it directly from within \AIPS\@.  The
color printer is one of the printer choices when you start up {\tt
AIPS}, but you probably want to select a regular PostScript printer as
your default printer.  You can change your printer selection with the
verb {\tt \tndx{PRINTER}}; use {\us PRINTER 999 \CR} to see what your
choices are and then {\us PRINTER {\it n\/} \CR} to choose the printer
numbered {\it n\/}.  \AIPS\ print routines will re-direct PostScript
files that actually contain color commands to the first {\tt PS-CMYK}
printer in the list, but will not re-direct ordinary print jobs to
some printer other than a color printer.}

\Subsubsections{Software to copy your screen}{copyscreen}

     To obtain a color hard-copy of what is on your screen, there are
several software options you can choose.  These include {\tt TVCPS},
{\tt xv}, and {\tt import}.  Having created a PostScript (or other
format) file, you can print it on color printers at the NRAO or copy
the file via e-mail, {\tt scp}, or {\tt ftp} to some other site.

     The {\tt \tndx{TVCPS}} task in \AIPS\ will create a color
Encapsulated \indx{PostScript} file from whatever is displayed on the
\AIPS\ TV server ({\tt XAS})\@.  If you use the {\tt \tndx{OUTFILE}}
adverb, this file is saved with whatever name you specify (see
\Sec{textfile}).  If you specify a black-and-white output to {\tt
TVCPS}, then the output can be sent to any PostScript printer.  Color
PostScript must be sent to a color printer.  You can, of course, edit
the saved file (if you are a PostScript wizard or use {\us HELP
POSTSCRIPT}) and can insert the file (since it is encapsulated) in
another document.

     The {\tt \tndx{xv}} program is a Unix utility program available
on most systems at the NRAO\@.  It is mainly intended for image
display of GIF, JPG, TIFF, and other format files.  When you start
{\tt xv}, click the right button mouse anywhere in the {\tt xv} window
to bring up the control window.  One of its features is a screen grab
which is controlled by the ``Grab'' button in the lower right corner
of the control window.  {\it Before\/} you press this, arrange your
windows and icons so that you can see exactly what it is you want to
grab (\eg\ the {\tt XAS} server).  Now press the ``grab'' button.  A
window with instructions will appear.  Move the cursor to the top left
of the area you want to grab.  Then press and hold down the center
mouse button, and drag the mouse cursor until it is at the bottom
right of the area you want to grab. As you do this, you will see a box
pattern on the screen outlining the area selected.  Once you are done
selecting the area, release the mouse cursor.  When {\tt xv} has
finished grabbing the screen, whatever you grabbed will appear in the
main {\tt xv} window.  You can now use the ``save'' button of the
control window to save this as any format you want.  One nice feature
of this is the ``save as Postscript'' option.  It allows you to scale,
rotate, and position the image in relation to the page.  Its user
interface is better than most image utilities.

     Finally, the {\tt \tndx{import}} program provides similar
functionality to the ``grab'' feature of {\tt xv}, with many options
about output formats and much more.  Enter {\us import\qs -help \CR}
for a summary of the options.  For example, enter {\us import\qs\
-quality 100\qs\ outfile.jpg \CR}.  The cursor will change to a plus
sign.  Position it at the top left corner of the area you wish to
grab, hold down a mouse button, and drag the cursor to the bottom
right of the area you wish to capture.  When you let up on the button,
the file {\tt outfile.jpg} will be written in {\tt jpg} format. The
file extension determines the format, so {\tt .eps} will produce
encapsulated PostScript.

\subsections{Using the tape drives}

    For a general discussion of magnetic tapes, including the {\it
required\/} software mount, see \Sec{magtape}.  The following
describes how to deal with the individual \indx{tape} drives at the
\hbox{AOC}.\iodx{magnetic tape}

\subsubsections{Mounting tapes on Exabyte and DAT drives}

     There are still tape devices in each of the \AIPS\ Caiges: rooms
209, 211, and 257.  Be aware that they are not much used and are not
maintained on a regular basis.  You may have to try more than one to
find a device that works for your tape.  There are ``toaster'' disk
devices on every public workstation.  These are more reliable and have
much more storage capacity, so tape should never be a medium of choice
now.  The devices are kept to allow old data tapes to be read.

     \indx{Exabyte (8mm)} and \indx{DAT (4mm)} drives have a window or
opening through which a mounted tape may be seen.  Before touching
anything, look in the window or opening to see if there is already a
tape in the drive.  If there is, ask around to make sure that the tape
is no longer in use.  Remember that the user of the drive may be in an
office as much as two floors away and that Unix does not provide much
protection.  If you dismount a remote user's tape and mount your own,
that user may well write on it, thinking that he is writing on his own
tape, without knowing that he is destroying all your data.

     On most drives, there will be a single button on the front panel
of the device somewhere.  When the device becomes available, press
this button to open the door.  If there was already a tape in the
drive, it will be ejected after some whirring and clanking and a few
seconds.  If a tape is ejected, remove it.  Now put your tape in the
drive, label facing upwards.  On Exabytes, push the door closed
gently.  For DAT drives, lightly push the tape into the drive until
the device ``grabs'' the tape and pulls it in the rest of the way.
Exabyte and DAT tapes have a small slide in the edge of the tape which
faces out which takes the place of the write ring of 9-track tapes.
For 8mm (Exabyte) tapes push the slide to the right (color black
shows) for writing and to the left (red or white shows) for reading.
With 4mm DAT tapes, the slide also goes to the right for writing (but
white or red shows) and to the left for reading (black shows).

     It is necessary to wait until the mechanism in the drive has
``settled down'', \ie\ when the noises and flashing lights have stopped,
before you can access the drive.  The first access is, of course, the
software {\tt MOUNT} command from inside \hbox{{\tt AIPS}}.

\subsections{Gripe, gripe, gripe, $\ldots$}

The so-called ``designated AIP'' is now Eric Greisen at all times
(except when he is on vacation).  He assists local and remote users
with their \AIPS\ problems, providing quick advice or simple fixes to
bugs,  More complex problems may require some time and may even require
receipt of the user's data in order to debug the problem.  Contact
the designated AIP (and all members of the group) at the e-mail
address {\tt daip@nrao.edu}.  The ``my.nrao'' web portal lets you into
the ``helpdesk'' which has an \AIPS\ department.  Management prefers
if you use this approach to request assistance.

     Suggestions and complaints entered on all computers with the
{\tt \tndx{GRIPE}} verb (see \Sec{gripe}) are sent immediately by
e-mail to {\tt daip} and thereby to all members of the group.
All traffic via daip from 2000 to the present is archived at {\tt
listmgr/cv.nrao.edu/pipermail/daip} and is well-known to Google.  You
might try Googling your \AIPS\ error message to see if you get
something useful.  We stand willing, and are now able, to respond to
user problems and requests on a timely basis.

The Gripes database described in \AIPS\ Memo No.~88  used to be
maintained, but appears to have disappeared.

\vfill\eject
\Subsections{Solving problems at the NRAO}{Zproblems}

     Below are details specific to NRAO systems for handling some of
the problems which may arise in
\hbox{\AIPS}.

\subsubsections{Booting the workstations}

     Modern workstations, especially the powerful PCs and Macs, are
complex Unix systems which may have remote users within the NRAO and
guests from elsewhere on the Internet.  Users should {\it never\/}
attempt to boot the system on their own.  If the machine appears to be
dead, find or call one of the people listed on the bulletin boards in
the \AIPS\ Caige for this purpose.

\subsubsections{Printout fails to appear}

     Check the \AIPS\ output messages that appeared shortly after you
submitted your \indx{print job}, whether it be from {\tt PRTMSG} or
{\tt LWPLA}, or some other task.  You should see the output of the
Unix command to show the printer queue status.  If anything went wrong
with the print submission, an error message should be obvious.  If
not, check the output of the {\tt \tndx{lpq}} command, see what print
queue was involved, and check it again from the Unix command level
(not from inside {\tt AIPS})\@.

     {\tt AIPS} will delete spooled files about 5 minutes after they
are submitted.  If the print queue is stalled (due, say, to a jammed
printer) or backed up with a lot of jobs, it is possible that the file
was deleted before it was gobbled up by the print spooler.  This time
delay has been made a locally-controlled parameter, so it is possible
to set it to values higher than 5 minutes.

     Finally, check to see if the printout was (a) diverted to the
``big'' printer ({\tt psnet} in room 213 at the AOC or {\tt ps3dup} in
the Charlottesville library) because it was too long for the smaller
printers, (b) you forgot which printer you had selected on {\tt aips}
startup, or, at the AOC, (c) someone has taken the output and filed
it in the ``today'' file bin (at the AOC this is on the left side of
the post directly behind the {\tt psnet} printer).

\Subsubsections{Stopping excess printout}{Zprint}

     To find out what jobs are in the spooling queue for the relevant
printer, type, at the monitor level:
\dispx{{\tt \$\qs} lpq \CR}{to list default print queue}
\dispx{{\tt \$\qs} lpstat \CR}{to list default print queue under
            Solaris}
\dispe{or to display a specific queue}
\dispx{{\tt \$\qs} lpq\qs -P{\it ppp\/} \CR}{to show printer {\it
            ppp\/}}
\dispx{{\tt \$\qs} lpstat\qs {\it ppp\/} \CR}{to show printer {\it
            ppp\/} under Solaris}
\dispe{where {\it ppp\/} {\it might\/} be {\tt psnet} at the AOC or {\tt
ps3dup} in Charlottesville.  If the file is still in the queue as job
number {\it nn\/}, you can type  simply}
\dispx{{\tt \$\qs} \tndx{lprm} -P{\it ppp\/} {\it nn\/} \CR}{to remove
        the job}
\dispx{{\tt \$\qs} cancel {\it nn\/} \CR}{to remove the job under
        Solaris}
\dispe{{\tt lprm} and {\tt cancel} will announce the names of any files
that they remove and are silent if there are no jobs in the queue which
match the request.}

     Unfortunately, it is now very difficult to stop long print jobs.
The large memories of modern printers mean that more than one print
job can already be resident in the printer while your long unwanted
job is being printed.  Therefore, turning off the printer is not an
option.  Try to be more careful and not generate excess printout in
the first place (save a tree).

     A nice option available for most \AIPS\ print tasks or verbs
is adverb {\tt \tndx{OUTPRINT}} which allows you to divert the output
to a text file.  Then you can use an editor like {\tt emacs} to
examine the file in detail before printing.  The Unix command {\tt
\tndx{wc} -l {\tt file}} will count the number of lines in a text file
called {\tt file} for you; note that {\tt -l} is the letter ell, not
the number one.  \AIPS\ provides a ``filter'' program to convert plain
(or Fortran) text files to PostScript for printing on PostScript
printers.  The command
\dispx{{\tt \$\qs} \tndx{F2PS} -{\it nn\/} {\tt <} {\it file\/} {\tt
             |} lpr -P{\it ppp\/}}{ }
\dispe{will print text file {\it file\/} on PostScript printer {\it
ppp\/}.  The parameter {\it nn\/} is the number of lines per page used
inside \AIPS; it is likely to be 97 if direct printing comes out in
``portrait'' form or 61 if the direct print outs come out in
``landscape'' form.}

\subsubsections{{\tt CTRL Z} problems}

     The last process placed in the background via {\tt \tndx{CTRL Z}}
can be brought back to the foreground by typing {\us \tndx{fg} \CR} in
response to the monitor level {\tt \%} or {\tt \$} (or whatever) prompt
Alternatively, the user can type {\us\qs \tndx{jobs} \CR}, which
displays all background processes associated with the current login
and can bring a specific process to the foreground by typing {\us
fg\qs {\it \% m\/} \CR}, where {\it m\/} is the job number as
displayed by the {\tt jobs} command as {\tt [{\it m\/}]}.  For
example, if a user initiated his {\tt AIPS}{\it n\/} by typing {\us
aips\qs new\qs pr=4\CR} and:
\dispx{{\tt \^{ }Z}}{{\tt CTRL Z} typed by accident (or
             intentionally).}
\dispx{{\tt Stopped}}{{\tt aips new} is put in the background as
             ``stopped'' and user is returned to the Unix level.}
\dispx{{\tt \$\qs} jobs \CR}{to display status of background jobs.}
\dispx{{\tt [1]\qs + Stopped\qquad aips new}}{info from Unix, where
              {\tt [1]} means job 1, ``Stopped'' is job 1's state and
              ``aips new'' is the command used to start up job 1.}
\dispx{{\tt \$\qs} fg {\it m\/} \CR}{to return job {\it m\/} to the
              foreground.}
\dispx{{\tt aips new}}{appears on the screen just to tell the user
              to which job he is talking (\ie\ it does {\it not\/}
              re-execute {\tt aips new}).  You should now be talking
              to your {\tt AIPS}{\it n\/} again.}
\dispx{\CR}{to get {\tt AIPS}{\it n\/} {\tt >} prompt.}
\pd

\subsubsections{``File system is full'' message}

     The message {\tt write failed, file system is full} will appear
when the search for scratch space encounters a disk or disks without
enough space.  This is only a problem when none of the disks available
for scratch files has enough space, at which point the task will shut
down.  Use the {\tt BADDISK} adverb to avoid disks with little
available space.

\subsubsections{Tapes won't mount}

   Occasionally, both local and remote \indx{tape mount}s may not work
successfully.  The source of the problem is often your failure to load
the tape physically into the device or to wait until the device is
ready to read the tape.  DATs and Exabytes, in particular, go through
lots of clicking and whirring before they are really ready.  An error
message like
\disps{{\tt AIPS 1: ZMOUN2: Couldn't open tape device /dev/nrst0}}
\dispe{(or some other tape-device name gibberish) is to be expected in
this case.}

     If you attempt to mount a remote tape and get the messages:
\disps{{\tt AIPS 1: ZMOUNR: UNABLE TO MOUNT REMOTE TAPE DEVICE, ERROR
              96}}
\disps{{\tt AIPS 1: AMOUNT: TAPE IS ALREADY MOUNTED BY \tndx{TPMON}}}
\dispe{it means that your {\tt AIPS} and the tape d\ae mon that you
are using disagree on whether the tape is already mounted in software.
The most probable reason for this is that you are attempting to mount
someone else's tape (check your inputs and the labels on the device
closely) --- or that the previous user of the device dismounted the
tape from the hardware but neglected to do it from software.  In this
case, you have two choices: (1) find the culprit and have him do a
software dismount, or (2) find an \AIPS\ Manager to kill the confused
d\ae mon and restart it.  (If you are using tape device {\it n\/} on
computer {\it host\_name\/}, then you need to stop the process called
{\tt TPMON}{\it m\/}, where $m = n + 1$ on computer {\it host\_name\/}
and then start it again by running {\tt /AIPS/START\_TPSERVERS} on that
computer.  This should be done by an \AIPS\ Manager.)}

     If you attempt to mount a remote tape and see, instead, the
messages:
\disps{{\tt ZVTPO2 connect (INET): Connection refused}}
\disps{{\tt AIPS 1: ZMOUNR: UNABLE TO OPEN SOCKET TO REMOTE MACHINE,
              ERROR  1}}
\disps{{\tt AIPS 1: ZMOUNT: ERROR     1 RETURNED BY ZMOUN2/ZMOUNR}}
\dispe{then the tape d\ae mons are not running on the remote machine.
Log into the remote machine and type:}
\disps{{\tt /AIPS/START\_TPSERVERS}}
\dispe{After a minute or two, you should see some messages from
{\tt STARTPMON} about starting {\tt TPMON} d\ae mons.  Alternatively,
you could exit from {\tt AIPS} and start back up again, including {\tt
tp={\it host\_name\/}} on the {\tt aips} command line; see \Sec{stAIPS}.
If the tape still doesn't mount after doing this, see the \AIPS\
Manager.}

\Subsubsections{I can't use my data disk!}{Zdisk}

     If at some point during your work you find you are prevented from
reading or writing files on a data disk, it could be that your \AIPS\
number does not have access to that area.  If you encounter the message:
\disps{{\tt AIPS 2: CATOPN: ACCESS DENIED TO DISK  8 FOR USER 1783}}
\dispe{it means that user 1783 has not been given access to write
(or read) on disk 8.  This can be seen, in the {\tt AIPS} session, by
typing {\tt \tndx{FREESPAC}} to list the mounted disks. If you see a
data disk listed with an access of {\tt Not you}, it means your \AIPS\
number has not been enabled for that disk. If you feel that you should
have access to that particular disk, see the data analysts (at the AOC)
or an \AIPS\ Manager about enabling your user number.}

\sects{\AIPS\ at the NRAO AOC in Socorro}

     A small visitor facility for \AIPS\ users is available in both
Charlottesville and Green Bank.  However, most non-NRAO people wishing
to reduce data at an NRAO facility with \AIPS\ should consider coming
to Socorro.  Not only is the green chili good, but the user facilities
are quite substantial.

    Public workstations at the AOC are Linux computers, many being two
processor, 8 core machines with 24 Gbyte of RAM and a lot (4
Terabytes) of disk.  A complete list of the public-use workstations
can be found on the web at {\tt
http://www.aoc.nrao.edu/computing/workstations.shtml}.  All are
equipped with a toaster disk and each \AIPS\ Caige room has one set of
tape (DAT, Exabyte) devices.

    Of the several printers available at the AOC, most \AIPS\ users
will want to use the high-volume PostScript printer {\tt aoc213} and
the Tektronix color printer {\tt aoc213c}, both of which are located
in Room 213.

\subsections{Reserving public-use workstations at the AOC}

    If you are visiting the AOC for observing or to reduce data then
you should fill out a visitor reservation form at least two weeks
before you expect to arrive. These forms can be found on the web at
{\tt http://www.aoc.nrao.edu/}, select ``Facilities'', then ``VLA'',
then ``Other Info for Observers'', the ``Visiting the DSOC/VLA'', and
follow instructions from there.  If you check ``OBSERVING AND/OR DATA
REDUCTION'' as your purpose of visit at the end of Section A of the
reservation form, then you will automatically be assigned an observer
login account and a workstation.  If you have any special data
reduction needs, you should make a note of them in the comments
section of the form.

    If you are unable to fill out this form using the web or if you
have any questions about reservations, you should contact AOC
reservations at 575--835--7357 or {\tt nmreserv\@ nrao.edu}.  Note
that reservations for VLBA projects will not be accepted unless the
observer has been notified that correlation is complete and that the
data have been scrutinized.  If you have any special data reduction
needs then you should make a note of them in the Section C of the
form.

    If you are an AOC staff member and wish to sign up for a public-use
workstation then you should contact {\tt helpdesk@nrao.edu} or visit
real people in Room 262.  You are expected to use your own login
account on the public workstations.  In general, visitor's
reservations take priority over reservations for AOC staff.

\subsections{Using Linux workstations at the AOC --- introduction}

    In order to start \AIPS\, you need to log in to your assigned
workstation and open a terminal window.  At the AOC, we generally
use the windowing environments supplied by the operating system
vendors (RedHat Linux) rather than imposing a
lowest-common-denominator ``standard'' environment.  This means that
the procedures for logging on and manipulating windows depend on the
current version of Linux, which will progress with time.

    On sitting down at a workstation you will be presented with a
log-in window that has a space for you to type your user name.  Enter
your assigned username and press return and then, when prompted, enter
your password and press return again.

    You will be presented with a rather vanilla X-Windows screen.
Figure out how to start a terminal window of some sort, {\tt xterm}s
are preferred.

\subsections{Starting \AIPS}

    Once you have logged in to your workstation and have a terminal emulator
window open, you may start \AIPS\. If you wish to run \AIPS\ on the machine
at which you are sitting we recommend that you start \AIPS\ with the following
command.
\disps{\tt aips tst pr=3}
\dispe{This will bypass the printer menu and select the high-volume PostScript
printer {\tt aoc213/dup} in Room 213 as the default \AIPS\ printer.}

    Note that \AIPS\ is configured to start a separate message window
at the AOC.  If you do not want the separate message window to appear,
issue one of the following commands before starting \AIPS\@.
\disps{\tt export AIPS\_MSG\_EMULATOR=none}
\dispe{If you are using the KornShell or bash or.}
\disps{\tt setenv AIPS\_MSG\_EMULATOR none}
\dispe{if you are using csh or tcsh.  You can move the cursor into the
message window and hit {\tt CTRL-Z} to cause it to disappear instead.}

    You may replace {\tt xterm} with the name of another terminal emulator
(e.g. {\tt dtterm}) if you wish and you can also turn off the Tektronix
emulator using one of the following commands.
\disps{\tt export AIPS\_TEK\_EMULATOR=none}
\dispe{If you are using the KornShell or bash,}
\disps{\tt setenv AIPS\_TEK\_EMULATOR none}
\dispe{if you are using csh or tcsh.  This emulator is easily ignored.}

    If the \AIPS\ TV is already running, you may use the following to avoid
the messages that are produced as \AIPS\ figures out whether the TV is
already running.
\disps{\tt aips tst pr=4 tvok}
\pd

\subsubsections{Starting \AIPS\ on another machine}

    If you are going to run \AIPS\ on a remote machine and use your local
workstation as a display, we recommend that you start \AIPS\ on your local
workstation first and allow it to start the TV. You can then iconize the
terminal that you are using to run \AIPS\ and open a new terminal window.
You should then use the {\tt slogin} command to log in to the remote
machine. You should then start \AIPS\ using the following command.
\disps{\tt aips tst pr=4 tv=mydisplay tvok}
\dispe{You should replace {\tt mydisplay} with the name of the workstation
at which you are sitting.}

\subsubsections{On-line {\tt FILLM}}

    On-line {\tt FILLM} applied to the old VLA and does not apply now.

\sects{\AIPS\ at the NRAO in Charlottesville and Green Bank}

\AIPS\ is installed in Charlottesville based on the Linux computer
named {\tt valen}.  Similarly, it is installed in Green Bank on a
computer named {\tt bratac}.  The installations are current, with
Midnight Jobs updating the development version each night.  However,
there are no current members of the \AIPS\ group at either site, so
the actual state of \AIPS\ usage, particularly for visitors, is hard
to determine from Socorro.  Both sites have visitor procedures and
welcome outside visitors.  However, Charlottesville is concerned with
ALMA users wishing to use {\tt CASA} and Green Bank is concerned with
GBT users needing to use {\tt GBT-IDL}\@.  Visitor information for
both sites can be found from the main NRAO web site, including forms
to register a visit and much information.  Visitors to Charlottesville
and Green Bank should inquire locally for information concerning
public workstations supporting \AIPS\@.
