%-----------------------------------------------------------------------
%;  Copyright (C) 1995, 1997-1998, 2000-2017
%;  Associated Universities, Inc. Washington DC, USA.
%;
%;  This program is free software; you can redistribute it and/or
%;  modify it under the terms of the GNU General Public License as
%;  published by the Free Software Foundation; either version 2 of
%;  the License, or (at your option) any later version.
%;
%;  This program is distributed in the hope that it will be useful,
%;  but WITHOUT ANY WARRANTY; without even the implied warranty of
%;  MERCHANTABILITY or FITNESS FOR A PARTICULAR PURPOSE.  See the
%;  GNU General Public License for more details.
%;
%;  You should have received a copy of the GNU General Public
%;  License along with this program; if not, write to the Free
%;  Software Foundation, Inc., 675 Massachusetts Ave, Cambridge,
%;  MA 02139, USA.
%;
%;  Correspondence concerning AIPS should be addressed as follows:
%;          Internet email: aipsmail@nrao.edu.
%;          Postal address: AIPS Project Office
%;                          National Radio Astronomy Observatory
%;                          520 Edgemont Road
%;                          Charlottesville, VA 22903-2475 USA
%-----------------------------------------------------------------------
\chapts{Calibrating Interferometer Data}{cal}

\renewcommand{\titlea}{31-December-2017 (revised 22-November-2017)}
\renewcommand{\Rheading}{\AIPS\ \cookbook:~\titlea\hfill}
\renewcommand{\Lheading}{\hfill \AIPS\ \cookbook:~\titlea}
\markboth{\Lheading}{\Rheading}

     This chapter focuses on ways to do the initial \Indx{calibration}
of interferometric fringe-visibility data in \AIPS\@.  The sections
which follow concentrate primarily on continuum calibration for
connected-element interferometers, especially the VLA\@.  However, the
information in these sections is useful to spectral-line, solar, and
VLBI observers as well.  For specific advice on the new EVLA, consult
\Rappen{EVLAdata}\@.  For additional advice on spectral-line
calibration, see \Sec{linecal}; for advice on calibrating observations
of the Sun, see \Sec{suncal}; and for the gory details of VLBI, read
\Rchap{vlbi}.  After the initial calibration has been completed, data
for sources with good signal-to-noise are often taken through a number
of cycles of imaging with self-calibration.  See \Sec{selfcal} for
information on these later stages of the reduction process.  For
accurate calibration, you must have accurate {\it a priori\/}
positions and structural information for all your calibration sources
and accurate flux densities for at least one of them.  It is best if
the calibration sources are unresolved ``point'' sources, but it is
not required.

     For the basic calibrations, visibility (``{\it uv\/}'') data are
kept in ``\indx{multi-source data sets},'' each of which contains, in
time order, visibility data for one or more ``unknown'' sources and
one or more calibration sources.  Associated with these data are
``extension'' files containing tables describing these data.
\iodx{extension files}  When VLA archive data are first read into
\AIPS\ a number of basic \indx{tables} are created and filled with
information describing the data set.  These are \iodx{antenna file}
\xben
\Item {\tt AN} (antennas) for sub-array geometric data, date,
     frequency, polarization information, {\it etc.\/},
\Item {\tt FQ} (frequency) for frequency offsets of the different
     IFs (IF pairs in VLA nomenclature),
\Item {\tt NX} (index) to assist rapid access to the data,
\Item {\tt SU} (source) for source specific information such as
     name, position, velocity, and
\Item {\tt TY} (temperature) for measured system temperatures and
     nominal sensitivities
\Item {\tt CL} (calibration) for calibration and model information,
\xeen
\par\noindent An initial {\tt CL} table contains gains due to known
antenna functions of elevation and measured atmospheric opacities.
VLBI, and especially VLBA, data sets will end up with even more table
files.  Calibration and editing tasks then create, as needed, other
tables including
\xben
\setcounter{enumi}{6}
\Item {\tt BP} (bandpass) for bandpass calibration,
\Item {\tt FG} (flag) for flagging (editing) information, and
\Item {\tt SN} (solution) for gain solutions from the calibration
     routines.
\Item {\tt BL} (baseline) for baseline-, or correlator-,
     dependent corrections,
\xeen
\noindent All of these tables can be written to, and read back from,
FITS files along with the visibility data.  These, and any other,
\AIPS\ tables can be manipulated and examined using the general tasks
{\tt \tndx{PRTAB}}, {\tt \tndx{TACOP}}, {\tt \tndx{TABED}}, {\tt
\tndx{TAMRG}}, {\tt \tndx{TASRT}}, \hbox{{\tt \tndx{TAFLG}} and
{\tt \tndx{TAPPE}}}.

     The visibility data within the multi-source data set are not
normally altered by the calibration tasks.  Instead, these tasks
manipulate the tabular information to describe the calibration
corrections to be applied to the data and any flagging (deletion) of
the data.

    The \AIPS\ programs discussed in this chapter are part of a
package that has been developed to calibrate interferometer data from
a wide range of connected-element and VLB arrays, especially the VLA
and VLBA\@.  These programs therefore support many functions (and
inputs) that are not required when calibrating normal VLA data.  The
examples given below show only the essential parameters for the
operation being described, but, to get the results described, it is
essential that you check {\it all\/} the input parameters before
running any task.  Remember that \AIPS\ adverbs are global and will be
``remembered'' as you proceed.  A list of calibration-related symbols
is given in \Sec{aboutcal}, but a possibly more up-to-date list can be
obtained by typing {\us ABOUT CALIBRAT} in your \AIPS\ session.  More
general information on calibration can be routed to your printer by
typing {\us DOCRT\qs FALSE ; EXPLAIN\qs CALIBRAT \CR}, while deeper
information on a specific task is obtained with {\us EXPLAIN\qs
{\it taskname\/} \hbox{\CR}}.

     When you are satisfied with the \Indx{calibration} and editing
(or are simply exhausted), the task {\tt SPLIT} is used to apply the
calibration and editing tables and to write \uv\ files, each
containing the data for only one source.  These ``single-source'' \uv\
files are used by imaging and deconvolution tasks that work with only
one source at a time.  Many of the tasks described in this chapter
will also work on single-source files.  For VLA calibration, there are
several useful procedures described in this chapter and contained in
the {\tt RUN} file called \hbox{{\tt \tndx{VLAPROCS}}}.  Each of these
procedures has an associated {\tt HELP} file and inputs.  Before any
of these procedures can be used, this {\tt RUN} file must be invoked
with:
\dispt{RUN VLAPROCS \CR}{to compile the procedures.}
\dispe{There is a ``pipeline'' procedure designed to do a preliminary
calibration and imaging of ordinary VLA data sets.  This provides a
good first look at the data.  Nonetheless, the results are still not
likely to be of publishable quality.  To run the pipeline, enter}
\dispt{RUN VLARUN \CR}{to compile the procedures.}
\dispt{INP VLARUN \CR}{to review the input adverbs and, when ready,}
\dispt{VLARUN \CR}{to execute the pipeline.}

\Sects{Copying data into \AIPS\  multi-source disk files}{uvtape}

     There are several ways to write VLA data to \AIPS\ multi-source
\uv\ data sets on disk. They include:
\xben
\Item For VLA data from the archive, use {\tt FILLM} to read one or
    more disk files; see \Sec{fillmdisk}.   The VLA format was changed
    on January 1, 1988, but all older data were translated and
    archived in the modern format.  On July 1, 2007, the ModComps were
    replaced with modern computers and the format had an essential
    change made to it.  Use {\tt FILLM} to read data from the
    post-ModComp era.  Archive data are obtained from
    {\tt https://archive.nrao.edu/archive/e2earchive.jsp}
\Item For an \AIPS\ multi-source data set written to a FITS tape or
    FITS disk during an earlier \AIPS\ session, use {\tt \tndx{UVLOD}}
    or {\tt \tndx{FITLD}} to read the tape.
\Item For single-source data sets that are already on disk and
    are very similar in structure, use {\tt \tndx{UV2MS}} on one of
    them to create a multi-source data set, and then on each of the
    others to append them to that multi-source data set.  Each of the
    input data sets should have the same number of polarizations, IFs,
    spectral channels, and ``random parameters.''  {\tt UV2MS} also
    makes no corrections for differences in observed source positions
    or frequencies.  After all are appended, use {\tt UVSRT} to put
    the data in time-baseline order and {\tt \tndx{INDXR}} to make an
    index and initial (null) calibration file.
\Item For single-source data sets that are already on disk and
    are not sufficiently similar in structure for the method above,
    use {\tt \tndx{MULTI}} on each single-source file to convert to
    multi-source format.  Then use {\tt DBCON} to concatenate the
    individual multi-source files into one big multi-source file.
    Finally use {\tt UVSRT}, if needed, to put the data in
    time-baseline order and {\tt INDXR} to make an index and initial
    (null) calibration file.
\Item Data from the Australia Telescope may be loaded from disk files
    into \AIPS\ using the task {\tt \tndx{ATLOD}} which is now
    included with \AIPS\@.
\xeen

     Data from other telescopes can be read into \AIPS\ only if they
are written in \AIPS-like FITS files already or if you have a special
format-translation program for that telescope.  The VLBA correlator
produces a format which is translated by the standard \AIPS\ task {\tt
FITLD}; see \Sec{FITLDfits}.  A translation task for the Westerbork
Synthesis Telescope ({\tt WSLOD}) is available from the Dutch,
but is not distributed by the NRAO with the normal \AIPS\ system.

\Subsections{Reading from VLA archive files using {\tt FILLM}}{fillm}

     The NRAO Archive makes available, among other things, data from
the EVLA, which began observing in January 2010, and from the old VLA
which ceased observing a few days into 2010.  To load data from the
EVLA into \AIPS\ consult \Rappen{EVLAdata} for information about {\tt
BDF2AIPS} and other options.  The following decribes how to read old
VLA data into \AIPS\@.

     The Archive now serves VLA data in the form of one or more
``MOdComp'' format disk files.  To load these into \AIPS, enter
\dispt{TASK\qs 'FILLM' ; INP \CR}{to review the inputs needed.}
\dispt{DATAIN\qs 'MYDATA:AG230\_' \CR}{to read from the disk area
        pointed at by the logical {\tt MYDATA} and the data from
        program ID {\tt AG230}\@.}
\dispt{NFILES\qs 0 \CR}{to start with file AG230\_1.  If your first
        file is \eg\ {\tt AG230\_4}, set {\tt NFILES = 3}.}
\dispt{NCOUNT\qs 4 \CR}{to read four data files {\tt AG230\_1} through
        {\tt AG230\_4}\@.}
\dispt{OUTNA\qs ' ' \CR}{to take the default output file name.}
\dispt{OUTDI\qs 3 \CR}{to write the data to disk 3 (one with enough
         space).}
\dispt{DOUVCOMP\qs -1 \CR}{to write visibilities in uncompressed
         format.  VLA files are small by modern standards, so saving
         space is not worth the costs.}
\dispt{DOWEIGHT\qs 1 \CR}{Data weights will depend on the ``nominal
        sensitivity'' and should be calibrated along with the
        visibility amplitudes ({\tt DOCALIB = 1})\@.}
\dispt{CPARM\qs 0 \CR}{to do no averaging of the data in {\tt FILLM}\@.}
\dispt{CPARM(6)\qs 1 \CR}{to select VLA sub-array 1.}
\dispt{CPARM(7)\qs 2000 \CR}{to have observations within 2 MHz be
         regarded as being at the same frequency.}
\dispt{CPARM(8)\qs 1 \CR}{to use a 2-minute interval for the {\tt CL}
         table; default is 5 min.}
\dispt{CPARM(9)\qs 0.25 \CR}{to use a 15-second interval for the {\tt
         TY} table; default is the input data interval.}
\dispt{DPARM\qs 0 \CR}{to have no selection by specific frequency.}
\dispt{REFDATE\qs '{\it yyyymmdd\/}' \CR}{to specify the year, month,
         and day of the reference date.  This should be the first date
         in the data set (or earlier).  All times in \AIPS\ will be
         measured with respect to that date and must be positive.  The
         default is the first date included by the data selection
         adverbs, which may not be the desired one. Note that {\tt
         REFDATE} is only a reference point; it does not affect which
         data are loaded from the the files.}
\dispf{TIMERANG\qs {\it db} , {\it hb} , {\it mb} , {\it sb} , {\it
         de},  {\it he} , {\it me} , {\it se} \CR}{to specify the
         beginning day, hour, minute, and second and ending day, hour,
         minute, and second (wrt {\tt REFDATE}) of the data to be
         included.  The default is to include all times.}
\dispt{INP \CR}{to review the inputs.}
\dispt{GO \CR}{to run the program when you're satisfied with inputs.}
\dispe{There are numerous adverbs including {\tt BAND}, {\tt QUAL},
{\tt CALCODE}, {\tt VLAOBS}. and {\tt VLAMODE} to limit what data were
loaded from magnetic tapes which could hold data from multiple
projects.  These adverbs still function, but are of little use today.
Note that the values given above are illustrative and should not be
copied verbatim in most cases.}

      Be careful when choosing the averaging time with {\tt CPARM(1)}.
If you have a large data set, setting this time too {\it low\/} will
make an unnecessarily large output file; this may waste disk space
and slow the execution of subsequent programs.  Setting it too {\it
high\/} can, however, (1) smear bad data into good, limiting the
ability to recognize and precisely remove bad data, (2) smear features
of the image that are far from the phase center, and (3) limit the
dynamic range that can be obtained using self-calibration.  If you
need a different (usually shorter) averaging time for the calibrator
sources than for your program sources, use {\tt CPARM(10)} to specify
the averaging time for calibrators.  See Lectures~12  and 13 in
{\it \jndx{Synthesis Imaging in Radio Astronomy}\/}\footnote{{\it
Synthesis Imaging in Radio Astronomy\/}, Astronomical Society of the
Pacific Conference Series, Volume~6 ``A Collection of Lectures from
the Third NRAO Synthesis Imaging Summer School'' eds.\/ R.\ A.\
Perley, F.\ R.\ Schwab and A.\ H.\ Bridle (1989)} for general guidance
about the choice of averaging time given the size of the required
field of view and the observing bandwidth.

     {\tt CPARM(2)} controls a number of mostly esoteric options.  If
your data include the Sun or planets, you must set {\tt CPARM(2) = 16}
to avoid having each scan on the moving source assigned a different
name.  The adverb {\tt DOWEIGHT = 1} has the same affect as {\tt
CPARM(2) = 8} and both select the use of the nominal sensitivity to
scale the data weights.  When this is done, the weights will be
$1/\sigma^2$ as they should for imaging, with $\sigma$ in ``Jy'' in
the same uncalibrated scale as the fringe visibilities.  Having
selected this option, you should apply any amplitude calibration to
the weights as well as the visibilities.  If you store the data in
compressed form, only one weight may be retained with each sample.
Any differences between polarizations and/or IFs in that sample will
be lost.  Uncompressed data require less cpu, but more real, to read
but 2 to 3 times as much disk space to store.

      {\tt CPARM(2)=2048} allows you to load data as correlation
coefficients, which can be scaled to visibilities later with {\tt
TYAPL} (\Sec{VLATY}).  {\tt CPARM(3)} controls which on-line flags are
applied by {\tt FILLM}, which now always writes an {\tt OF} table
containing information about these flags.  That information can be
viewed with {\tt \Tndx{PRTOF}} and applied selectively to the data at
a later time with {\tt \Tndx{OFLAG}}\@.

      {\tt FILLM} writes a weather ({\tt WX}) table to the output
file.  At the same time, it uses ``canned'' VLA antenna gain curves
and a balance of the current with a seasonal model weather data to
estimate opacity and gain corrections to be written into the first
calibration ({\tt CL}) table.  These functions are controlled by
adverbs {\tt CALIN} and {\tt BPARM} and may be turned off, although
the default is to make the corrections. In subsequent tasks, set {\tt
DOCALIB = 1} to use these initial calibration data.  If, for some
reason, the data weights do not depend on the nominal sensitivity, use
{\tt DOCALIB=100} to apply calibration.

     Where possible, {\tt FILLM} will try to place all data in one
file.  However, in many cases this is not possible.  For instance
so-called ``channel 0'' data from a spectral-line observation will be
placed in a separate file from its associated line data.  Similarly,
scans which have differing numbers of frequency channels will also be
placed into separate files.  Another case is observations made in mode
LP, \ie\ one IF-pair is set to L band, the other to P band. In this
case the two bands will be split into separate files.  Yet another
case arises when there are observations of different bandwidths. All
of this should be relatively transparent to the user.

     {\tt FILLM} and many \AIPS\ tasks are able to handle multiple,
logically different, frequencies within a multi-source data set.  {\tt
FILLM} does this by assigning an {\tt FQ} number to each observation
\todx{FQ number} and associating a line of information about that
frequency in the {\tt FQ} file associated with the data set.  Users
should note that this concept can become quite complicated and that
not all tasks can handle it in full generality.  In fact, most tasks
can only process one {\tt FQ} number at a time.  Polarization
calibration works only on one {\tt FQ} at a time since the antenna
file format allows for only one set of instrumental polarization
parameters.  Therefore, it is {\it strongly\/} advised that you fill
continuum experiments which involve multiple frequencies into separate
data sets.  {\tt FILLM} will separate bands automatically, but you
will have to force any remaining separation.  To do this, (a) use the
{\tt QUAL} adverb in {\tt FILLM}, assuming that you have used separate
qualifiers in {\tt OBSERVE} for each frequency pair; (b) use the {\tt
DPARM} adverb array in {\tt FILLM} to specify the desired frequencies
precisely; or (c) use the {\tt UVCOP} task to separate a multiple {\tt
FQ} data set into its constituent parts.  Note that the first two
options require multiple executions of {\tt FILLM}, while the third
option requires more disk space.

      Spectral-line users and continuum observers using different
frequencies in the same band should be aware of the {\tt FQ} entry
tolerance.  Each frequency in a \uv\ file will be assigned an {\tt FQ}
number as it is read from disk by \hbox{{\tt FILLM}}.  For
spectral-line users, the observing frequency will normally change as a
function of time due to Doppler tracking of the Earth's rotation, or
switching between sources or between spectral lines; in general, this
will cause different scans to have different {\tt FQ} numbers.  {\tt
FILLM} assigns an {\tt FQ} number to a scan based on the {\tt FQ}
tolerance adverb {\tt CPARM(7)} which defines the maximum change of
frequency allowed before a new {\tt FQ} number is allocated.  If
{\tt CPARM(7) < 0}, the the same FQ number is assigned to all data in
spectral-line data sets.  If {\tt CPARM(7)} is positive, a scan's
\todx{FQ number} will be assigned to an existing {\tt FQ} number if
$$ \| \nu_{current} - \nu_{firstFQ} \| < {\tt CPARM(7)} $$
where $\nu_{firstFQ}$ is the frequency of the first sample to which
the particular {\tt FQ} number was assigned.  If no match is found,
then a new {\tt FQ} number is created and assigned and another line
added to the {\tt FQ} table file.  Alternatively, if {\tt CPARM(7)} is
zero, then the {\tt FQ} tolerance is assumed to be half of the
maximum frequency difference caused by observing in directions 180
degrees apart (\ie\ $\Delta\nu = 10^{-4}\times \nu$).

     An example: if an observer observes the 1612, 1665 and 1667 MHz
OH masers in VY CMa and NML Cygnus, then presumably he would like his
data to have 3 {\tt FQ} numbers, one associated with each OH
transition.  However, running {\tt \tndx{FILLM}} with {\tt CPARM(7)}
set to $0$ would produce 6 {\tt FQ} numbers because the frequency
difference between the masers in VY CMa and NML Cygnus is greater than
the calculated tolerance of 160 kHz.  Therefore, in order to ensure
that only 3 {\tt FQ} numbers are assigned, he should set {\tt
CPARM(7)} to 1000 kHz.  Setting {\tt CPARM(7) < 0} would result in all
data having the same {\tt FQ} number, which is clearly undesirable.

     For most continuum experiments the {\tt FQ} number will be
constant throughout the database.  Normally any change in frequency
should be given a new {\tt FQ} number.  To achieve this, {\tt FILLM}
treats {\tt CPARM(7)} differently for continuum.  If {\tt CPARM(7)}
$\leq$ 0.0, then {\tt FILLM} assumes a value of 100 kHz.  A positive
value of {\tt CPARM(7)} is treated as a tolerance in kHz as in the
spectral line case.

      {\bf Note:} {\it If your \uv\ database contains several
frequency identifiers, you should go through the calibration steps for
each {\tt FQ} code separately.}

     {\tt FILLM} can still read from magnetic tape.  Set {\tt DATIN}
to blanks, mount your tape (adverb {\tt INTAPE}), index the tape with
{\tt PRTTP}, and use the adverbs to limit the data loaded to that
portion of your project in which you are interested.

      If {\tt FILLM} is executing correctly, your message terminal
will report the number of your observing program, the VLA archive
format revision number, and then the names of the sources as they are
found in the data files.  Once {\tt FILLM} has completed, you can find
the database on disk using:
\dispt{INDI\qs 0 ; UCAT \CR}{ }
\dispe{This should produce a listing such as:}
\bve
Catalog on disk  3
Cat Usid Mapname      Class  Seq  Pt     Last access      Stat
  1  103 25/11/88    .X BAND.   1 UV 05-FEB-1994 12:34:16
\end{verbatim}\eve

     You might then examine the header information for the disk data set by:
\dispt{INDI\qs 3 ; GETN\qs 1 ; IMH \CR}{}
\dispe{This should produce a listing like:}
\bve
 Image=MULTI     (UV)         Filename=25/11/88    .X BAND.   1
 Telescope=VLA                Receiver=VLA
 Observer=AC238               User #=  103
 Observ. date=25-NOV-1988     Map date=05-FEB-1994
 # visibilities    191317     Sort order  TB
 Rand axes: UU-L-SIN  VV-L-SIN  WW-L-SIN  BASELINE  TIME1
            SOURCE  FREQSEL WEIGHT  SCALE
 --------------------------------------------------------------
 Type    Pixels   Coord value  at Pixel    Coord incr   Rotat
 COMPLEX      1   1.0000000E+00    1.00 1.0000000E+00    0.00
 STOKES       4  -1.0000000E+00    1.00-1.0000000E+00    0.00
 IF           2   1.0000000E+00    1.00 1.0000000E+00    0.00
 FREQ         1   8.4110000E+09    1.00 1.2500000E+07    0.00
 RA           1    00 00 00.000    1.00      3600.000    0.00
 DEC          1    00 00 00.000    1.00      3600.000    0.00
 --------------------------------------------------------------
 Maximum version number of extension files of type HI is   1
 Maximum version number of extension files of type AN is   1
 Maximum version number of extension files of type NX is   1
 Maximum version number of extension files of type SU is   1
 Maximum version number of extension files of type FQ is   1
 Maximum version number of extension files of type WX is   1
 Maximum version number of extension files of type CL is   1
 Keyword = 'CORRMODE'  value = '        '
 Keyword = 'VLAIFS  '  value = 'ABCD    '
\end{verbatim}\eve
\dispe{This header identifies the file as a multi-source file ({\tt
Image=MULTI}) with 191317 floating-point visibilities in time-baseline
({\tt TB}) order.  There are two entries on the {\tt IF} axis.  These
correspond to the old VLA's ``AC'' and ``BD'' IF-pairs respectively.
The description of the frequency ({\tt FREQ}) axis shows that the
first IF (``AC'') is at 8411 MHz and has 12.5 MHz bandwidth.  The
parameters of the second IF-pair (``BD'') are determined from the data
in the {\tt FQ} table file and cannot be read directly from this
header; these values are shown in the {\tt 'SCAN'} listing from
\hbox{{\tt LISTR}}.  The header shown above indicates that the data
are in compressed format since the number of pixels on the {\tt
COMPLEX} axis is 1 and the {\tt WEIGHT} and {\tt SCALE} random
parameters are present.  Uncompressed data does not use these random
parameters and has 3 pixels on the {\tt COMPLEX} axis.\todx{IMHEAD}}

     The term ``IF'' can be confusing.  At the VLA, IFs ``A'' and
``C'' correspond to right-hand and left-hand circularly polarized (RHC
and LHC) signals, respectively, and are normally for the same
frequency in an observing band.  Such pairs, if at the same frequency,
are considered to be one ``IF'' in \hbox{\AIPS}.   An observation
which was made in spectral line mode ``{\tt 2AC}'' is considered at
the VLA to have two ``IFs'' whereas within \AIPS\ this would be filled
as one ``IF'' with two polarizations if they were both observed with
the same frequency, the same number of channels, and the same channel
separation.  If these conditions do not hold, then they are filled
into separate \uv\ files, each with a single IF and a single
polarization.  The term ``sub-array'' is also confusing.  At the VLA
--- and in task {\tt \tndx{FILLM}} --- sub-array means the subset of
the 27 antennas actually used to observe your sources.  (The VLA
allows up to 5 simultaneous sub-arrays in this sense.)  In the rest of
\AIPS, sub-array refers to sets of antennas used together at the same
time. If observations from separate times (\eg\ separate array
configurations) are concatenated into the same file, then \AIPS\ will
regard the separate sets of antennas as different ``sub-arrays''
whether or not the same physical antennas occur within more than one
of these sub-arrays.

     If your experiment contains data from several bands {\tt FILLM}
will place the data from each band in separate data sets.  Also, if
you observed with several sets of frequencies or bandwidths in a given
observing run these will be assigned different {\tt FQ} numbers by
\hbox{{\tt FILLM}}.  You can determine which frequencies correspond to
which {\tt FQ} numbers from the {\tt 'SCAN'} listing provided by
\hbox{{\tt LISTR}}.  Line data are divided into the ``channel 0''
(central $3/4$ of the of the observing band averaged) and the spectra.
Data observed in the ``LP'' mode (or any other two-band mode) will be
broken into separate data sets, one for each band.

\Subsubsections{Editing and applying nominal sensitivities to VLA
    data}{VLATY}

{\tt FILLM} scales the correlation coefficients by the instantaneous
measured ``nominal sensitivities,'' producing data approximately in
deci-Jy.   The VLA nominal sensitivities are stored in the {\tt TY}
table as ``system temperatures'' ($T_{sys}$)\@.  For calibration
purposes, it is best to have the nominal sensitivities applied, but it
may be better to use a clipped and/or time-smoothed version of those
sensitivities.  If you want to do this, load the $T_{sys}$ data into
the {\tt TY} table with the highest time resolution possible by
setting {\tt CPARM(9)=0} in {\tt FILLM}\@.  {\tt FILLM} can also be
told not to apply the nominal sensitivities and therefor produce
correlation coefficients by setting {\tt CPARM(2)=2048}, but this is
not strictly necessary.  In order to smooth and clip the {\tt TY}
table use the task {\tt \tndx{TYSMO}}\@.  If you have done editing
such as {\tt QUACK}, it may help to copy the data with {\tt UVCOP},
applying your flag table not only to the visibilities but also to the
{\tt TY} table ({\tt UVCOPPRM(6)=3}) before running {\tt TYSMO} to
remove questionable values at the start of scans.  Alternatively,
{\tt \tndx{SNEDT}} will apply the data flags to the table allowing you
to write a new, cleaned-up version of the table.  Then a {\tt TY}
table may be applied (and/or removed) from a data set with {\tt
\tndx{TYAPL}}\@:
\dispt{TASK\qs 'TYAPL' ; INP \CR}{to review the inputs needed.}
\dispt{INDI\qs {\it n\/} ; GETN\qs {\it m\/} \CR}{to select the
        correct data  set.}
\dispt{FREQID\qs 1 \CR}{to select {\tt FQ} number 1.}
\dispt{INVERS\qs 1 \CR}{{\tt TY} table to remove from data, will only
        work if data are not already correlation coefficients.}
\dispt{IN2VERS\qs 2 \CR}{smoothed {\tt TY} table to apply to data,
        will only work if data is in correlation coefficient form ---
        either initially or after removal of {\tt INVERS}\@.}
\dispt{INP \CR}{to re-check {\it all\/} the inputs parameters.}
\dispt{GO \CR}{to start the task.}

EVLA users (see \Rappen{EVLAdata}) should have an {\tt SY} table which
contains system gain and temperature data.  These data {\it should} be
applied to the visibility data in order to correct for measured gain
changes and to convert the data weights from a simple count of the
integration time into more meaningful values.  Tasks {\tt TYSMO} and
{\tt SNEDT} may be used to clean up the {\tt SY} data and then {\tt
TYAPL} may be used to apply the gain and system temperature data.

\Subsections{Reading data from FITS files with {\tt FITLD}}{FITLDfits}

     {\tt \tndx{FITLD}} is used to read FITS-format disk files (and
tapes) into \hbox{\AIPS}. It recognizes images, single- and
multi-source \uv\ data sets, and the special FITS \uv-data tables
produced by the VLBA and DiFX correlators (``FITS-IDI'' format).  In
particular, VLA data sets that have been read into \AIPS\ previously
with {\tt FILLM} and then saved to tape (or pseudo-tape disk) files
with {\tt FITTP} and {\tt FITAB} can be recovered for further
processing with task \hbox{{\tt FITLD}}.  (The older task {\tt
\tndx{UVLOD}} will also work with \uv\ data sets in FITS format, but
it cannot handle image or FITS-IDI format files.)

     A multi-source data file with all of its tables can be read from
a FITS tape by:
\dispt{TASK\qs 'FITLD' ; INP \CR}{to review the inputs needed.}
\dispt{INTAPE\qs {\it n\/} \CR}{to specify the tape drive for input
          from tape.}
\dispt{DATAIN\qs '{\it filename\/}' \CR}{if the input is from a FITS
          disk file (see \Sec{fitsdisk}).}
\dispt{DOUVCOMP\qs FALSE \CR}{to write visibilities in uncompressed
          format.}
\dispt{OUTNA\qs ' ' \CR}{take default (previous \AIPS) name.}
\dispt{OUTCL\qs ' ' \CR}{take default (previous \AIPS) class.}
\dispt{OUTSEQ\qs 0 \CR}{take default (previous \AIPS) sequence \#.}
\dispt{OUTDI\qs 3 \CR}{to write the data to disk 3 (one with enough
          space).}
\dispt{INP \CR}{to review the inputs (several apply only to VLBA
          format files).}
\dispt{GO \CR}{to run the program when you're satisfied with inputs.}

     {\tt FITLD} is the equivalent of {\tt FILLM}, but for output from
the VLBA, rather than the VLA, correlator.  The data-selection adverbs
{\tt SOURCES}, {\tt QUAL}, {\tt CALCODE}, and {\tt TIMERANG} and the
table-control adverbs {\tt CLINT} and {\tt FQTOL} are used, for
VLBA-format data only, in {\tt FITLD} in ways similar to the
data-selection and control adverbs of \hbox{{\tt FILLM}}.  See
\Rchap{vlbi} for more specific information.

\sects{Record keeping and data management}

\subsections{Calibrating data with multiple {\tt FQ} entries}

      In general an observing run with the VLA, especially a
spectral-line run, will result in a \uv\ data file containing multiple
{\tt FQ} entries.  \todx{FQ number}In early versions of the \AIPS\
software, the different {\tt FQ} entries would automatically have been
placed in different physical files.  Now, {\tt FILLM} allows you to
place all of them in the same file.  This may be convenient, but it
has a number of costs.  If a file contains multiple, independent
frequencies, then it occupies more disk space and costs time in every
program to skip the currently unwanted data (either a small cost when
the index file is used or a rather larger cost when the file must be
read sequentially).  Since multiple frequencies are still not handled
correctly in all programs (\ie\ polarization calibration) and since it
is not possible to calibrate all of the different {\tt FQ} data in one
pass, you might consider separating the multiple frequencies into
separate files (as described in \Sec{fillm}).  In either case, you
must calibrate each frequency with a separate pass of the scheme
outlined below. There are three adverbs to enable you to differentiate
between the different {\tt FQ} entries: {\tt \tndx{FREQID}} enables
the user to specify the {\tt FQ} number directly (with -1 or 0 meaning
to take the first found); {\tt SELFREQ} and {\tt SELBAND} enable the
user to specify the observing frequency and bandwidth to be calibrated
(the tasks then determine to which {\tt FQ} number these adverbs
correspond). If {\tt SELFREQ} and {\tt SELBAND} are specified they
override the value of \hbox{{\tt FREQID}}. \todx{FQ number}

     There are certain bookkeeping tasks that must be performed
between calibrating each {\tt FQ} set.  First, you must ensure that
you have reset the fluxes of your secondary calibrators by running
{\tt \tndx{SETJY}} with {\us OPTYPE = 'REJY'} --- if not, this will
cause the amplitudes of your data to be incorrect.  Second, it is wise
to remove the {\tt SN} tables associated with any previous calibration
using the verb \hbox{{\tt \tndx{EXTDEST}}}.  Although this is not
strictly necessary, it will simplify your bookkeeping.

    A practical note: it is often useful to have used different
qualifiers for different frequencies.  This gives you another
``handle'' on the data.  Unfortunately, not all programs use the
{\tt QUAL}, or even the {\tt CALCODE}, adverb.

\Subsections{Recommended record keeping}{calrecord}

     It is useful to print a summary of the time stamps and source
names of the scans in your data set.  This reminds you of the
structure of your observing program when you decide on interpolation
and editing strategies, and may help to clarify relationships between
later, more detailed listings of parts of the data set.  It is also
useful to have a printed scan summary and a map of the antenna layout
if you need to return to processing the data months or years later.
Finally, it is also making sure that all \AIPS\ input parameters have
their null (default) values before invoking the parts of the
calibration package, such as {\tt CALIB}, that have many inputs.  The
null settings of most parameters are arranged to be sensible ones so
that basic VLA calibration can be done with a minimum of specific
inputs; but some inputs may lose their default values if you
interleave other \AIPS\ tasks with the calibration pattern recommended
below.  Therefore, you should {\it always\/} review the input
parameters with {\us INP\qs {\it taskname\/} \CR} before running task
{\it taskname\/}.

     We suggest that you begin a calibration session with the
following inputs:
\dispt{\tndx{DEFAULT}\qs LISTR \CR}{to set all {\tt LISTR}'s inputs to
       null (default) values.}
\dispt{TASK\qs '\tndx{LISTR}' ; INP \CR}{to review the inputs needed.}
\dispt{INDI\qs {\it n\/}; GETN\qs {\it m\/} \CR}{to select the data
       set, $n=3$ and $m=1$ in {\tt FILLM} example above.}
\dispt{TPUT\qs CALIB \CR}{to store null values for later use with
       {\tt CALIB}\@.}
\dispt{OPTYP\qs 'SCAN' \CR}{to select scan summary listing.}
\dispt{DOCRT\qs -1 \CR}{to send the output to the printer.}
\dispt{INP \CR}{to review the inputs for \hbox{{\tt LISTR}}.}
\dispt{GO \CR}{to run the program when the inputs are set correctly.}
\dispe{Note that the {\tt DEFAULT\qs LISTR} sets the adverbs to select
all sources and all times and to send printed output to the terminal
rather than the printer.  It is also very useful to have a printed
summary of your antenna locations, especially a list of which ones you
actually ended up using.  To do this, enter}
\dispt{NPRINT\qs 0 \CR}{to do all antennas}
\dispt{INVERS\qs {\it }n \CR}{to do sub-array {\it n\/}}
\dispt{GO\qs \tndx{PRTAN} \CR}{to print the list and a map of antenna
           locations.}

     In looking over the output from {\tt LISTR}, you may notice that
some of the sources you wish to use as calibrators have a blank
``Calcode''.  To mark them as calibrators, use:
\dispt{TASK\qs '\tndx{SETJY}' ; INP \CR}{to select the task and review
           its inputs.}
\dispt{SOURCES '{\it sor1\/}' , '{\it sor2\/}' , '{\it sor3\/}' ,
           $\ldots$ \CR}{to select the unmarked calibrator sources.}
\dispt{OPTYPE 'RESE' \CR}{to reset fluxes and velocities.}
\dispt{CALCODE\qs 'C' \CR}{to mark the sources as ``C'' calibrators.}
\dispt{GO \CR}{to run the task.}
\dispe{This operation will let you select the calibrators by their
Calcodes rather than having to spell out their names over and over
again.  You may wish to consider separate calibrator codes for primary
and secondary gain calibrators to make them easier to separate.
You may reset a calibrator code to blank by specifying {\tt CALCODE =
'----'}.}

\Sects{Beginning the calibration}{calbegin}

     After loading the data to disk, it has been traditional to begin
with a substantial session of data checking and \Indx{editing}.  With
data from the VLA, this is always time consuming and often not
necessary. Nonetheless, it is probably a good idea to check for two
specific kinds of problems before beginning the actual calibration.
These are corrupted data in the first record of most scans and totally
dead antennas.  Many other problems in the data are quickly and easily
diagnosed by carefully inspecting the solution tables produced from
the calibrators on un-edited data.  Missing antennas and erratic
amplitudes due to sampling problems and RF interference can be spotted
from the {\tt SN} tables and the closure-error messages produced by
{\tt CALIB}\@.  If you {\it can't\/} spot errors from these, you
may not need to edit the calibrator data.  If the {\tt SN} tables have
well-behaved phases for most antennas and rapidly rotating phases for
one or two, then you may need to apply baseline corrections rather
than editing.  See \Sec{blcorr} for details of how to make
antenna-position corrections.

     The next section tells how to detect simple problems in the data
and eliminate them to reduce the warnings from the calibration tasks.
The following sections tell you how to enter fluxes for the primary
calibrator sources and do a preliminary calibration for all
calibrators.  In so doing, you should generate one or more solution
({\tt SN}) tables containing the complex gains at the times of the
calibration observations.  These tables may be examined for problems
with the observations.  If you find problems, then you need to edit
the data or apply baseline corrections and should consult
\Sec{caledit}.  If you do not find problems, you may proceed directly
to \Sec{calgain}.  (Of course, you may decide to edit the data from
your program sources at a later stage of the data reduction and have
to return to \Sec{caledit} then.)

\Subsections{Initial editing}{initedit}

     The warning messages from the calibrations described in the next
sections may be reduced by \Indx{flagging} those antennas which were
not actually working, but which were not flagged by the on-line
system.  Another problem that has plagued the VLA (and other
interferometers) persistently is that the first record in scans can be
corrupted; usually its amplitudes are lower than they should be.
These data can be flagged using {\tt TVFLG} or {\tt UVFLG}, but this
can be time consuming.  The task {\tt EDITA} described in
\Sec{edita} is now likely to be the best initial (and perhaps only)
editing tool which you need.  For a more traditional approach, we
recommend that you do the following before beginning your regular data
editing.  Use the task {\tt \tndx{LISTR}} on your terminal (to save
time and paper) to see if you have the problem:
\dispt{TASK\qs 'LISTR' \CR}{to set the data listing task}
\dispt{INDI\qs {\it n\/} ; GETN\qs {\it m\/} \CR}{to select the data
         set, $n=3$ and $m=1$ above.}
\dispt{OPTYPE\qs 'LIST' \CR}{to select column listing format}
\dispt{ANTEN\qs {\it a1\/} , 0 \CR}{to select one reliable antenna to
           display.}
\dispt{BASEL\qs 0 \CR}{to select all baselines to this antenna.}
\dispt{SOURCES\qs ' ' ; CALCODE\qs '*' \CR}{to select all calibrator
           sources only.}
\dispt{TIMER\qs 0 \CR}{to select all times.}
\dispt{STOKES\qs 'RR' \CR}{to examine only one Stokes at a time.}
\dispt{BIF\qs 1 ; EIF\qs BIF\CR}{to specify the ``AC'' IFs only; it
           is quicker to look at only 1 IF at a time although more
           than one can be listed in sequence.}
\dispt{FREQID\qs 1 \CR}{to select {\tt FQ} number (each {\tt FQ}
           number must be done separately).}
\dispt{DOCRT\qs 132 \CR}{to see full width display on the terminal.
           Use your window manager to stretch the window to $ \ge 132$
           characters width.}
\dispt{DOCALIB\qs -1 \CR}{to turn off calibration.}
\dispt{DPARM\qs 0 \CR}{to select amplitudes with no averaging.}
\dispt{INP \CR}{to re-check {\it all\/} the inputs parameters.}
\dispt{GO \CR}{to start the task.}
\dispe{The task will prompt you for a \CR\ after each ``page full'' of
output.  When you have seen enough, enter \hbox{{\us Q}}.  This
display will let you determine whether the start-of-scan problem
infects your data and, if so, how badly.  If it is rare, forget it for
now and use manual flagging methods later if needed.  If it is
widespread, use the \AIPS\ task {\tt \tndx{QUACK}}:\todx{LISTR}
\Iodx{editing}\Iodx{flagging}}
\dispt{TASK\qs 'QUACK' \CR}{}
\dispt{SOURCES\qs ' ' \CR}{to select all sources.}
\dispt{TIMER\qs 0 \CR}{to select all times.}
\dispt{ANTENNAS\qs 0 \CR}{to select all antennas.}
\dispt{FLAGVER\qs 1 \CR}{to insert flagging information in FG table 1.}
\dispt{OPCODE\qs 'BEG' \CR}{flag first {\tt APARM(2)} min of each scan.}
\dispt{REASON\qs 'BAD START OF SCAN' \CR}{reason for the flagging.}
\dispt{APARM\qs 0 , 1/6 , 0 \CR}{flag first 10 seconds of each scan.}
\dispt{GO \CR}{}

     The display generated above will also allow you to determine
quickly which antennas are absent, which antennas are present but
dead, and, with more careful examination, which antennas are flaky and
may need special consideration.  ``Dead'' antennas are visible in this
display as columns with small numbers --- columns that differ by
factors of two or so from the others are generally fine.  To be
thorough, it is probably best to check the other IF:
\dispt{BIF\qs 2 ; EIF\qs 2\CR}{to specify the ``BD'' IFs.}
\dispt{GO \CR}{to run the program again.}
\dispe{as well as {\tt STOKES = 'LL'}.}

     To remove the dead antennas, run \hbox{{\tt \tndx{UVFLG}}}.  For
example, if antennas 6, 9, and 22 were bad for the full run in both
IFs and Stokes, they could be deleted with
\dispt{TASK\qs 'UVFLG' ; INP \CR}{to select the editor and check its
          inputs.}
\dispt{TIMER\qs 0 \CR}{to select all times.}
\dispt{BIF\qs 1 ; EIF\qs 2 \CR}{to specify the ``AC'' and ``BD'' IFs.}
\dispt{BCHAN\qs 0 ; ECHAN\qs 0 \CR}{to flag all channels.}
\dispt{FREQID\qs 1 \CR}{to flag only the present {\tt FQ} number.}
\dispt{ANTEN\qs 6 , 9, 22 \CR}{to select the antennas.}
\dispt{BASEL\qs 0 \CR}{to select all baselines to these antennas.}
\dispt{STOKES\qs ' ' \CR}{to select all Stokes.}
\dispt{REASON = 'ZOMBIE ANTENNA' \CR}{to set a reason.}
\dispt{OUTFGVER\qs 1 \CR}{to select the first (only) flag table.}
\dispt{INP \CR}{be careful with the inputs here!}
\dispt{GO \CR}{to run the task when ready.}

\subsections{Primary flux density calibrators}

      Careful measurements made with the D array of the VLA have shown
that the Baars {\it et al.\/}\footnote{1977, {\it Astr.~\&\ Ap.}, 61,
99} formul\ae\ for ``standard'' calibration sources are in error
slightly, based on the assumption that the Baars' expression for 3C295
is correct.   Revised values of the coefficients have been derived by
Rick Perley and Brian Butler.  Task {\tt SETJY} has these formulae
built into it, giving you the option ({\us OPTYPE 'CALC'}) of letting
it calculate the fluxes for primary calibrator sources 3C48, 3C123,
3C138, 3C147, 3C196, 3C286, 3C295, and 1934-638.  The default setting
of {\us APARM(2) = 0}) will calculate the flux densities based on
Perley-Butler 2017 values which cover the range 50 MHz to 50 GHz for
the primary calibrators:3C48, 3C138, 3C147, 3C286, and 3C295 plus
3C123 and 3C196, all with a number of synonyms.  Other sources which
will be computed, but which may not be good calibrator sources,
are (one name each) J0444-2809, PictorA, 3C144 (Taurus A, Crab
Nebula), 3C218 (Hydra A), 3C274 (Virgo A), 3C348 (Hercules A), 3C353,
3C380, 3C405 (Cygnus A), 3C444, and 3C461 (Cassiopeia A).  The 3C name
is the only one used in {\tt SETJY} for those extra sources with a 3C
name --- the parenthetical remarks are here for clarity only.  Most of
the sources in the extra list have limits on the frequency range over
which the function is valid; {\tt SETJY} will tell you if the
frequencies are out of range.  Higher values of {\tt APARM(2)} select
older systems of coefficients if you need them to match previous data
reductions.  {\tt SETJY} will recognize both the 3C and IAU
designations (B1950 and J2000) for the standard sources.  You may
insert your own favorite values for these sources instead ({\us OPTYPE
= ' '}) and you will have to insert values for any other gain
calibrators you intend to use.  Adverbs {\tt SPECINDX} and {\tt
SPECURVE} allow you to enter spectral index information to help set
the calibrator fluxes.\Iodx{calibration}

      Unfortunately, since all the primary flux calibrators are
resolved by the VLA in most configurations and at most frequencies,
they cannot be used directly to determine the amplitude calibration of
the antennas without a detailed model of the source structure, see
\Rfig{3C48_X} as an example.  Beginning in April 2004, model images
for the calibrators at some frequencies are included with \AIPS\@.
Models of 3C286, 3C48, 3C138, and 3C147 are available for all 6
traditional bands of the VLA {\it except} 90\,cm and even for S band
of the EVLA.  Type {\us \tndx{CALDIR} \CR} to see a list of the
currently available calibrator models.  Sources which are small enough
to be substantially unresolved by the VLA have variable flux densities
which must be determined in each observing session.  A common method
used to determine the flux densities of the secondary calibrators from
the primary calibrator(s) is to compare the amplitudes of the gain
solutions from the procedure described below.

    Use {\tt SETJY} to enter/calculate the flux density of each
primary flux density calibrator.  The ultimate reference for the VLA
is 3C295, but 3C286 (1328+307), which is slightly resolved in most
configurations at most frequencies, is the most useful primary
calibrator.  {\tt CALIB} has an option that will allow you to make use
of Clean component models for calibrator sources.  You are strongly
encouraged to use the existing models.  If you follow past practice
at the VLA, you may have to restrict the \uv\ range over which you
compute antenna gain solutions for 3C286, and may therefore insert a
``phony'' flux density appropriate only for that \uv\ range at this
point.  In both cases, the following step should be done.  {\tt CALIB}
will scale the total flux of the model to match the total flux of the
source recorded by {\tt SETJY} in the source table.  This corrects for
the model being taken at a somewhat different frequency than your
observations and for the model containing most, but not all, of the
total flux.  An example of the inputs for {\tt SETJY}, where you let
it calculate the flux, would be:
\dispt{TASK\qs 'SETJY' ; INP \CR}{}
\dispt{SOURCES\qs '3C286' , '\ ' \CR}{if you used 3C286 as the source
            name.}
\dispt{BIF\qs 1 ; EIF\qs 2 \CR}{will calculate for both ``AC'' and
            ``BD'' IFs.}
\dispt{OPTYPE\qs 'CALC' \CR}{perform the calculation.}
\dispt{APARM(2)\qs = 0 \CR}{to use the VLA ``2013'' coefficients.}
\dispt{INP \CR}{to review inputs.}
\dispt{GO \CR}{when inputs okay.}

Or you can set the flux manually as shown below:
\dispt{TASK\qs 'SETJY' ; INP \CR}{}
\dispt{SOURCES\qs '3C286' , '\ ' \CR}{if you used 3C286 as the source
          name.}
\dispt{ZEROSP\qs 7.41 , 0 \CR}{I flux 7.41 Jy, Q, U, V fluxes 0.}
\dispt{BIF\qs 1 ; EIF\qs 1 \CR}{selects first IF \hbox{IF}.}
\dispt{INP \CR}{to review inputs.}
\dispt{OPTYPE\qs '\ '}{use values given in \hbox{{\tt ZEROSP}}.}
\dispt{GO \CR}{when inputs okay.}
\dispt{BIF\qs 2 ; EIF\qs 2 \CR}{selects second IF \hbox{IF}.}
\dispt{ZEROSP 7.46, 0 \CR}{I flux 7.46 Jy at the $2^{\und}$ IF,
          Q, U, V fluxes 0.}
\dispt{GO \CR}{}
\dispe{Note that, although {\tt SOURCES} can accept a source list,
{\tt ZEROSP} has room for only one set of I, Q, U, V flux densities.
To set the flux densities for several different sources or IFs, you
must therefore rerun {\tt \tndx{SETJY}} for each source and each IF,
changing the {\tt SOURCES}, {\tt BIF}, {\tt EIF}, and {\tt ZEROSP}
inputs each time.  Alternatively, set {\tt ZEROSP} to the flux at 1
GHz and enter {\tt SPECINDX} and {\tt SPECURVE} adverbs to describe
the dependence with frequency.\iodx{spectral index}}

{\tt CALIB} will use the V polarization flux in the source table if
one has been entered.  The RR polarization will be calibrated to I+V
and the LL to I-V.  While this has little practical use with circular
polarizations because V is almost always negligible, it can be used
for linearly polarized data from the WSRT\@.  That telescope has
equatorially mounted dishes, so the XX polarization is I-Q and the
YY is I+Q independent of parallactic angle.  For WSRT data, you should
relabel the polarizations to RR/LL and enter {\tt I, 0, 0, -Q} for
{\tt ZEROSP}, since Q is not negligible in standard calibrators.

\Subsections{First pass of the gain calibration}{1passgain}

\Subsubsections{Using calibrator models}{vlacalmodels}

It is now considered standard practice to use flux calibrator models
and you are strongly encouraged to do so.  As mentioned above, all the
primary flux calibrators are resolved at {\it most} frequencies and
configurations.  \Rfig{3C48_X} shows the visibilities and image of
the commonly used calibrator 3C48 at X-band, it is obvious this source
is far from being point like.  Since April 2004, source models have
been shipped with \AIPS\ as FITS files. Currently, models for 3C48,
3C286 and 3C137 are available for all bands except 90\,cm and 3C147 at
all bands except, X, C and 90\,cm.  Additional models are in the
works, so you should always check to see what is available:
\dispt{\tndx{CALDIR} \CR}{to list the available models by source name
          and band code.}

The primary calibration task in \AIPS\ is {\tt CALIB}\@.  Most of the
complexity of {\tt CALIB} can be hidden using the procedure {\tt
\tndx{VLACALIB}}\@.  Before attempting to use this procedure, you must
first load it by typing:
\dispt{RUN VLAPROCS \CR}{to compile the procedures.}
\dispe{Type {\tt HELP VLAPROCS} for a full list of the procedures
available in {\tt VLAPROCS}\@.}

The procedure {\tt VLACALIB} {\it automatically} downloads and uses
calibrator models, if one is available and the inputs are set correctly.
After you have loaded {\tt VLACALIB} you may invoke the calibrator model
usage by setting:
\dispt{INDI\qs {\it n\/} ; GETN\qs {\it m\/} \CR}{to select the data
          set, $n=3$ and $m=1$ above.}
\dispt{CALSOUR\qs = '{\it Cala\/}' \CR}{to name {\it one} primary
          flux calibrator to invoke automatic calibrator model
          usage.}
\dispt{UVRANGE\qs 0 \CR}{set to zero to invoke automatic calibrator
          model usage.}
\dispt{ANTENNAS\qs 0 \CR}{set to zero to invoke automatic calibrator
          model usage.}
\dispt{REFANT\qs {\it n\/} \CR}{reference antenna number --- use a
          reliable antenna located near the center of the array.}
\dispt{MINAMPER\qs 10 \CR}{display warning if baseline disagrees in
         amplitude by more than {\it 10\%\/}\ from the model.}
\dispt{MINPHSER\qs 10 \CR}{display warning if baseline disagrees by
         more than $10^{\circ}$ of phase from the model.}
\dispt{FREQID\qs 1 \CR}{use {\tt FQ} number 1.}
\dispt{DOPRINT\qs 1 ; OUTPRINT\qs '\qs' \CR}{to generate significant
         printed output on the line printer.}
\dispt{INP\qs VLACALIB \CR}{to review inputs.}
\dispt{VLACALIB \CR}{to make the solution and print results.}
\dispe{This procedure load the will load the calibrator model (using
{\tt CALRD}) and use it when it runs {\tt CALIB}, then print any
messages from {\tt CALIB} about closure errors on the line printer,
and finally run {\tt \tndx{LISTR}} to print the amplitudes and phases
of the derived solutions.  Plots of these values may be obtained using
task \hbox{{\tt \tndx{SNPLT}}}.}

Alternatively, you can load the model in manually using {\tt CALRD}
\Todx{CALRD}:
\dispt{TASK\qs 'CALRD' \CR}{to select the calibrator source reading
          task.}
\dispt{OBJECT\qs '3C286' \CR}{to load a model of 3C286.}
\dispt{BAND\qs 'K' \CR}{to select the available model at K band.}
\dispt{OUTDISK\qs $n$ \CR}{to write the model image and Clean
          components to disk $n$.}
\dispt{GO \CR}{to run the task and load the model.}

Then you may select the model image with {\tt GET2N} for use in
{\tt CALIB}\@.  Example inputs for {\tt CALIB} are:
\dispt{TASK\qs 'CALIB'; INP \CR}{to select task and review inputs.}
\dispt{INDI\qs {\it n\/} ; GETN\qs {\it m\/} \CR}{to select the data
          set, $n=3$ and $m=1$ above.}
\dispt{CALSOUR\qs = '{\it Cala\/}' , '\qs' \CR}{flux calibrator for
          which you have a model.}
\dispt{UVRANGE\qs 0 \CR}{no \uv\ limits needed.}
\dispt{ANTENNAS\qs 0 \CR}{antenna selection not needed.}
\dispt{REFANT\qs {\it n\/} \CR}{reference antenna number --- use a
          reliable antenna located near the center of the array.}
\dispt{WEIGHTIT\qs 1 \CR}{to select $1/\sigma$ weights which may be
          more stable.}
\dispt{IN2DI\qs {\it o\/} ; GET2N\qs {\it p\/} \CR}{to select the
          model.}
\dispt{NCOMP\qs 0 \CR}{to use all components.}
\dispt{SOLMODE\qs 'A\&P' \CR}{to do amplitude and phase solutions.}
\dispt{APARM(6)\qs 2 \CR}{to print closure failures.}
\dispt{MINAMPER\qs 10 \CR}{to display warning if baseline disagrees in
         amplitude by more than {\it 10\%\/}\ from the model.}
\dispt{MINPHSER\qs 10 \CR}{to display warning if baseline disagrees by
         more than $10^{\circ}$ of phase from the model.}
\dispt{CPARM(5)\qs 1 \CR}{to vector average amplitudes over spetral
         channels and then scalar average them over time before
         determining solution.}
\dispt{FREQID\qs 1 \CR}{to use {\tt FQ} number 1.}
\dispt{INP\qs CALIB \CR}{to review inputs.}
\dispt{GO \CR}{to make the solution.}

{\tt CALIB} will use the clean components table attached to the model
to find antenna gain solutions.  It will sum the clean components
within a certain radius of the center of the map (so that confusing
sources that are part of the model do not influence the gain) and
scale them to the flux in the {\tt SU} table. Therefore, you must
still run {\tt SETJY} before running {\tt CALIB}.

After running {\tt CALIB} check the solutions for all antennas with
{\tt SNPLT} or {\tt LISTR (OPTYPE='GAIN')}.  If you have multiple
primary or secondary calibrators you will have to run {\tt CALIB}
separately for each, using models where they are available and
restricting the {\tt UVRANGE} and {\tt ANTENNAS} where they are not.
You can either write into the same {\tt SN} table by setting {\tt
SNVER} to a table number or to different {\tt SN} tables by setting
{\tt SNVER = 0}. Then you you can proceed as normal flagging and
editing your data and proceed to final calibration as described in
\Sec{calgain}.

\begin{figure}
\centering
%\resizebox{\hsize}{!}{\gname{3C48Xuv}\hspace{0.5cm}\gname{3C48Xim}}
\resizebox{\hsize}{!}{\gbb{524,526}{3C48Xuv}\hspace{0.5cm}\gbb{525,539}{3C48Xim}}
\caption[3C48 at X-band]{Displays of the visibilities (left) and
image (right) for the fundamental calibration source 3C48\@.  The
plots were made using {\tt UVPLT}, {\tt KNTR}, and {\tt LWPLA};
see \Sec{plotuv} and \Sec{plotcntr}.  Data from all VLA configurations
including the VLBA antenna in Pie Town were used.  A point source
would have visibilities that have a constant amplitude at all
baselines and an image matching the beam plotted in the lower-left
corner.}
\label{fig:3C48_X}
\end{figure}

\Subsubsections{Flux calibration without calibrator models}{novlacal}

We strongly encourage you to use the available models.  If words alone
do not convince you, we encourage you to look at \Rfig{3C48_X} which
shows you the visibilities and image of 3C48 at X-band.  It is rather
far from a point source.  At lower frequencies there are other sources
in the field which have an effect on phases as well.  This section
used to list recommended {\tt UVRANGE}s to use for the standard
calibration sources.  This is such bad practise that we have deleted
those tables --- use the models.

   The values of {\tt \tndx{UVRANGE}} for each secondary calibrator
may be determined from the VLA Calibrator manual or by using {\tt
\tndx{UVPLT}} to plot the amplitudes as a function of baseline length.
If your secondary calibrators are point sources over most baselines,
then it may save you time to do the full calibration now.  Not only
will it save you, possibly, from re-running {\tt CALIB} at a later
time with a wider {\tt UVRANGE}, but it will provide information on
the data quality from the longer baselines.

     Once you have read in procedure {\tt \tndx{VLACALIB}} (see
\Sec{vlacalmodels}), you may use it to invoke {\tt \tndx{CALIB}}\@.
You will have to do this once for each calibrator, unless you can use
the same {\tt UVRANGE} for more than one of them.  Thus,
\dispt{INDI\qs {\it n\/} ; GETN\qs {\it m\/} \CR}{to select the data
          set, $n=3$ and $m=1$ above.}
\dispt{CALSOUR\qs = '{\it Cala\/}' , '{\it Calx\/}' \CR}{to name two
          calibrators using the same {\tt UVRANGE} and other adverb
          values.}
\dispt{UVRANGE\qs {\it uvmin\/} {\it uvmax\/} \CR}{\uv\ limits, if
          any, in kilo$\lambda$.}
\dispt{ANTENNAS\qs {\it list\ of\ antennas\/} \CR}{antennas to use for
          the solutions, see discussion above.}
\dispt{REFANT\qs {\it n\/} \CR}{reference antenna number --- use a
          reliable antenna located near the center of the array.}
\dispt{MINAMPER\qs 10 \CR}{display warning if baseline disagrees in
         amplitude by more than {\it 10\%\/}\ from the model.}
\dispt{MINPHSER\qs 10 \CR}{display warning if baseline disagrees by
         more than $10^{\circ}$ of phase from the model.}
\dispt{DOPRINT\qs 1 ; OUTPRINT\qs '\qs' \CR}{to generate significant
         printed output on the line printer.}
\dispt{FREQID\qs 1 \CR}{use {\tt FQ} number 1.}
\dispt{INP\qs VLACALIB \CR}{to review inputs.}
\dispt{VLACALIB \CR}{to make the solution and print results.}
\dispe{This procedure will first run {\tt CALIB}, then print any
messages from {\tt CALIB} about closure errors on the line printer,
and finally run {\tt \tndx{LISTR}} to print the amplitudes and phases
of the derived solutions.  Plots of these values may be obtained using
task {\tt \tndx{SNPLT}}\@.}

      If the secondary calibrators require different values of {\tt
UVRANGE}, then {\tt CALIB} must be run until it has run for all
calibration sources.  Attached to your input data set is a solution
{\tt SN} table.  Each run of {\tt CALIB} writes in this table (if {\tt
SNVER = 1}), for the times of the included \Indx{calibration} scans,
the solutions for all IFs using the flux densities you set for your
calibrators with {\tt SETJY} or {\tt \indx{GETJY}}\@.  ({\tt CALIB}
assumes a flux density of 1 Jy if no flux density is given in the {\tt
SU} table.)  If a solution fails, however, the whole {\tt SN} table
can be compromised, forcing you to start over.  It is possible to
write multiple {\tt SN} tables with {\us SNVER = 0}.  Later programs
such as {\tt GETJY} and {\tt CLCAL} will merge all {\tt SN} tables
which they find (if told to do so). Tables with failed solutions must
be deleted.

     The {\tt LISTR} outputs provided by {\tt VLACALIB} should be
examined carefully to check on the calibration; amplitudes should be
consistent (both among antennas and among time stamps) and phases
should vary smoothly.  If you decide that the solutions are not
acceptable (\eg\ there are no valid solutions) {\it and\/} you are
creating a new {\tt SN} table on each run of {\tt CALIB}, then delete
that {\tt SN} table using {\tt EXTDEST} before proceeding.  The later
stages of processing assume that all extant {\tt SN} tables are valid.
Note that re-running {\tt CALIB} on the same {\tt SN} table simply
over-writes the old solutions with new ones.    {\tt CALIB} gives
messages which indicate the number of valid and invalid solutions
which should help you evaluate the results.  If {\tt VLACALIB} is run
using the values of {\tt MINAMPER} and {\tt MINPHSER} shown above, it
will print a list of baselines and times which show substantial
``closure'' errors.  (If you use {\tt CALIB} directly rather than {\tt
VLACALIB}, you may use these adverbs plus {\tt CPARM(2}-{\tt 4)} to
get additional reports and statistics on closure errors and {\tt
CPARM(7)} to limit reporting of individual closure errors.)  It is
important to remember that normal thermal noise and, at longer
wavelengths, background confusion cause closure errors too.  Thus,
some closure error on weaker calibrators is to be expected and may be
ignored.  {\tt CALIB} will ignore errors larger than {\tt MINAMPER}
and {\tt MINPHSER} if the data weights (after application of the
source model) indicate that they have significance less than {\tt
CPARM(7)} times the expected error.   Interpreting closure errors is a
real art, but a couple of generalizations are possible.  If the same
closure error shows up in both polarizations and both IFs, then you
have probably got a resolved object.  If one antenna dominates the
closure list, especially if it is at only one IF and/or one
polarization, then you have got a bad antenna.  If the errors are
uniformly small, distributed amongst all antennas, and not correlated
between IFs or polarizations, then you have simply noise and/or
background confusion.  In this case, do {\it not\/} edit the data ---
the randomness of the ``errors'' nearly always averages out nicely and
the solution is just fine.  Large or systematic errors indicate either
that the calibrator source is resolved or that there are problems with
the data requiring editing.  If a calibrator is being resolved, delete
the bad {\tt SN} table and re-run {\tt VLACALIB} with an appropriate
{\tt UVRANGE}\@.  One can now actually flag data based on closure
errors using the {\tt DOFLAG} option.  This should be used carefully,
if at all and then only if the model of the calibration source
contains all of its flux.  Modern versions of {\tt CALIB} display some
statistics of closure problems automatically and allow the use of
``robust'' solution methods.

\Sects{Assessing the data quality and initial editing}{caledit}

     At each stage in the data calibration process, it is a good idea
to take a look at the data to determine their quality and then to
``flag'' (edit, delete) those that are suspect or clearly bad.  Having
begun the actual calibration, it is important to get an impression
of the overall quality of the data and to edit out any obviously
corrupted data, (\eg\ bad integrations that were not detected and
expunged by the on-line monitoring system, high amplitudes due to
interference, unstable amplitudes due to undetected equipment
problems, {\it etc\/}).  During the initial calibration, you need to
do this only on the observations of calibration sources.  However, at
a later stage, you may also need to apply techniques similar to those
described below to your program sources.  If you do edit any
calibration data at this point, you must re-run {\tt CALIB} following
the instructions given above for the affected sources.

     The philosophy of \Indx{editing} and the choice of methods are
matters of personal taste and the advice given below should,
therefore, be taken with a few grains of salt.  When interferometers
consisted of only a couple of movable antennas, there was very little
data and it was sparsely sampled.  At that time, careful editing to
delete all suspect samples, but to preserve all samples which can be
calibrated, was probably justified.  But modern instruments produce a
flood of data, with the substantial redundancy that allows for
self-calibration on strong sources.  Devoting the same care today to
editing is therefore very expensive in your time, while the loss of
data needlessly flagged is rarely significant.  A couple of guidelines
you might consider are:\Iodx{flagging}
\xbit
\Item Don't flag on the basis of phase.  At least with the VLA,
    most phase fluctuations are due to the atmosphere rather than the
    instrument.  Calibration can deal with these up to a point, and
    self-calibration (if you have enough signal) can refine the phases
    to levels that you would never reach by flagging.  The exceptions
    are (1) IF phase jumps which still happen on rare occasions, and
    (2) RF interference which sometimes is seen as an excursion in
    phase rather than amplitude.
\Item Don't flag on minor amplitude errors, especially if they are
    not common.  Except for very high dynamic range imaging, these
    will not be a problem, and in those cases, self-calibration always
    repairs or sufficiently represses the problem.
\Item Don't flag if {\tt CALIB} reports few closure errors and the
    {\tt SN} tables viewed with {\tt \tndx{EDITA}}, {\tt
    \tndx{SNPLT}}, and {\tt \tndx{LISTR}} and the calibrator data
    viewed with the matrix format of {\tt LISTR} show only a few
    problems.
\xeit

     There are three general methods of editing in \hbox{\AIPS}.  The
``old-fashioned'' route uses {\tt LISTR} to print listings of the data
on the printer or the user's terminal.  The user scans these listings
with his eyes and, upon finding a bad point, enters a specific flag
command for the data set using {\tt \tndx{UVFLG}}\@.  While this
may sound clumsy, it is in fact quite simple and by far the faster
method when there are only a few problems.  In a highly corrupted data
set, it can use a lot of paper and may force you to run {\tt LISTR}
multiple times to pin down the exact problems.  The ``hands-off''
route uses tasks which attempt to determine which data are bad using
only modest guidance from the user.  The most general of these are
{\tt RFLAG} and {\tt FLAGR} mentioned below.  The third and ``modern''
route uses interactive (``TV''-based) tasks to display the data in a
variety of ways and to allow you to delete sections of bad data simply
by pointing at them with the TV cursor.  These tasks are {\tt
\tndx{TVFLG}} (\Sec{tvflg}) for all baselines and times (but only shows
one IF, one Stokes, and one spectral channel at a time), {\tt
\tndx{SPFLG}} (\Sec{spflg}) for all spectral channels, IFs, and times
(but only shows one baseline and one Stokes at a time),  {\tt
\tndx{FTFLG}} for all spectral channels, IFs, and times (but shows all
baselines combined, one Stokes at a time), {\tt \tndx{EDITA}}
(\Sec{edita}) for editing based on {\tt TY} (T$_{ant}$), {\tt SN} or
{\tt CL} table values,  {\tt \tndx{EDITR}} (\Sec{editr}) for all
times (but only shows a single antenna (1--11 baselines) and one channel
average at a time) and {\tt \tndx{WIPER}} for all types of data (but
with the time, polarization, and sometimes antenna of the points not
available while editing).  {\tt TVFLG} is the one used for continuum
and channel-0 data from the VLA, while {\tt FTFLG} is only used to
check for channel-dependent interference.  {\tt SPFLG} is very useful
for spectral-line editing in smaller arrays, such as the Australia
Telescope and the VLBA, but is tedious for arrays like the VLA.
Nonetheless, it may be necessary for the modern EVLA\@.  (The
redundancy in the spectral domain on calibrator sources helps the eyes
to locate bad data.)  {\tt EDITR} is more useful for small arrays such
as those common in VLBI experiments.  {\tt EDITA} has been found to be
remarkably effective using VLA system temperature tables.  All four
tasks have the advantage of being very specific in displaying the bad
data.  Multiple executions should not be required. However, they may
require you to look at each IF, Stokes, channel (or baseline)
separately (unless you make certain broad assumptions); {\tt EDITA}
and {\tt EDITR} do allow you to look at all polarizations and/or IFs
at once if you want.  They all require you to develop special skills
since they offer so many options and operations with the TV cursor
(mouse these days).  A couple of general statements can be
made\Iodx{editing}\Iodx{flagging}
\xbit
\Item For highly corrupted data (say with considerable RF
    interference, significant cross-talk between antennas, or erratic
    antennas) {\tt \tndx{TVFLG}} is definitely preferred.  It gives an
    overall view of the data which is far superior to that given by
    \hbox{{\tt LISTR}}.  RFI and similar problems are more troublesome
    at lower frequencies, so {\tt TVFLG} is probably preferred for L,
    P, and ``4'' bands.
\Item Most VLA data at higher frequencies are of good quality and
    the flexibility of {\tt TVFLG} is not needed.  In such cases, {\tt
    \tndx{LISTR}} with {\tt OPCODE = 'MATX'} can find scans with
    erroneous points efficiently.
\Item The displays given by {\tt TVFLG} and, to a lesser extent,
    {\tt LISTR} in its {\tt MATX} mode are less useful when there are
    only a few baselines.  Thus, for arrays smaller than the VLA,
    users may wish to use {\tt \tndx{SPFLG}} on spectral-line data
    sets and {\tt \tndx{EDITR}} on continuum data sets.
\Item A reasonable strategy to use is to run {\tt LISTR} first.  If
    there are only a few questionable points, use {\tt LISTR} and {\tt
    UVFLG}, otherwise switch to an interactive task, such as {\tt
    EDITA} followed by \hbox{{\tt TVFLG}}.
\Item Task {\tt \tndx{FLAGR}} is a somewhat experimental task to
    measure the rms in the data on either a baseline or an antenna
    basis and then delete seriously discrepant points and times when
    many antennas/correlators are questionable.  It also clips
    amplitudes and weights which are outside specified normal ranges.
    Task {\tt \tndx{FINDR}} reports the rmses and excessive values to
    assist in running {\tt FLAGR}\@.
\Item Task {\tt \tndx{RFLAG}} flags data on the belief that RFI is
    highly variable in time and/or frequency.  It can do plots showing
    the time and/or frequency statistics as a function of spectral
    channel and recommends flagging levels.  These, or user-chosen
    values, may be applied to the data to produce large flag tables;
    see \Sec{EVLAflag}.
\Item Task {\tt \tndx{CLIP}} makes entries in a flag table, applying
    calibration and then testing amplitudes for reasonableness on a
    source-by-source basis.  It can be very useful for large data
    sets, but does not show you the bad data to evaluate yourself.
    {\tt \tndx{ACLIP}} does the same operation on auto-correlation
    data.
\Item Task {\tt \tndx{FGSPW}} performs a matrix scalar average over
    all spectral channels on a per-IF (aka ``spectral window'') per
    polarization per baseline per scan basis.  It then flags those IFs
    with excessive amplitude.  This operation is designed to catch
    those windows where amplitudes have overflowed the hardware due to
    RFI within the spectral window and may catch other problems as
    well.
\Item Task {\tt \tndx{DEFLG}} makes entries in a flag table whenever
    the phases are too variable as measured by too low a ratio of
    vector-averaged to scalar-averaged amplitudes.  This may be useful
    when applied to the calibrator source in phase-referencing
    observations and for other data at the highest and lowest
    frequencies which are affected by atmospheric and ionospheric phase
    variability.
\Item Task {\tt \tndx{SNFLG}} makes entries in a flag table whenever
    the phase solutions in an {\tt SN} or {\tt CL} table change
    excessively between samples on a baseline basis.  It can also flag
    data if the amplitude solutions differ from their mean
    excessively.  In {\tt 31DEC12} it can instead flag data when the
    amplitude and/or phase solutions or solution weights are outside
    user-specified ranges.
\Item Task {\tt \tndx{WIPER}} makes entries in a flag table for all
    data samples wiped from a {\tt UVPLT}-like display of any \uv\
    data set parameter versus any other parameter.  The source,
    Stokes, IF, time, etc.~of the points are not known during the
    interactive editing phase, but some baseline information is
    displayed.  It can plot and edit any choice of {\tt STOKES} in one
    execution and can flag/unflag by baseline.
\Item Task {\tt \tndx{WETHR}} makes entries in a flag table whenever
    various weather parameters exceed specified limits.  {\tt WETHR}
    also plots the weather ({\tt WX}) table contents.
\Item Task {\tt \tndx{VPFLG}} flags all correlators in a sample
    whenever one is flagged.  Observations of sources with circular
    polarization (Stokes V) require this operation to correct the
    flagging done on-line (which flags only known bad correlators).
\Item Task {\tt \tndx{FGPLT}} plots the times of selected flag-table
    entries to provide you information on what these powerful tasks
    have done.
\xeit

Many of the above tasks produce large flag tables.  The task {\tt
\tndx{REFLG}} attempts to compress such tables, sometimes quite
remarkably.  Task {\tt \tndx{FGDIF}} may be used to confirm that the
flag tables before and after flag the same things.  {\tt FGCNT} in
{\tt 31DEC16} may also help confirm this.

\Subsections{Editing with {\tt LISTR} and {\tt UVFLG}}{uvflg}

      Data may be flagged using task {\tt UVFLG} based on
listings from \hbox{{\tt \tndx{LISTR}}}.  To print out the
scalar-averaged raw amplitude data for the calibrators, and their {\it
rms\/} values, once per scan in a matrix format, the following inputs
are suggested:
\dispt{TASK\qs 'LISTR' ; INP \CR}{to review the inputs needed.}
\dispt{INDI\qs {\it n\/}; GETN\qs {\it m\/} \CR}{to select the data
           set, $n=3$ and $m=1$ above.}
\dispt{SOURCES ' ' ; CALCODE\qs '*' \CR} {to select calibrators.}
\dispt{TIMER\qs 0 \CR}{to select all times.}
\dispt{ANTENNAS\qs 0 \CR}{to list data for all antennas.}
\dispt{OPTYPE\qs 'MATX' \CR}{to select matrix listing format.}
\dispt{DOCRT\qs FALSE \CR}{to route the output to printer, not
           terminal.}
\dispt{DPARM\qs  3 , 1 , 0 \CR}{amplitude and {\it rms\/}, scalar scan
           averaging.}
\dispt{BIF\qs 1; EIF\qs 0 \CR}{to select all IFs, {\tt LISTR} will list
           IFs separately.}
\dispt{FREQID\qs 1 \CR}{to select {\tt FQ} number 1 (note that {\tt FQ}
           numbers must also be done separately).}
\dispt{INP \CR}{to review the inputs.}
\dispt{GO \CR}{to run the program when inputs set correctly.}

      For unresolved calibrators, the VLA on-line gain settings
normally produce roughly the same values in all rows and columns
within each matrix.  At L, C, X, and U bands, these values should be
approximately $0.1$ of the expected source flux densities.  At P band,
the factor is about $0.01$.  The factors for other bands are
unspecified.  Any rows or columns with consistently high or low values
in either the amplitude or the {\it rms\/}  matrices should be noted,
as they probably indicate flaky antennas.  In particular, you should
look for
\xbit
\Item In the amp-scalar averages, look for {\it dead\/} antennas,
     which are easily visible as rows or columns with small numbers.
     Rows or columns that differ by factors of two or so from the
     others are generally fine.  Such deviations mean only that the
     on-line gains were not set entirely correctly.
\Item In the {\it rms\/} listings, look for discrepant high values.
     Almost all problems are antenna based and will be seen as a row
     or column.  Factors of 2 too high are normally okay, while
     factors of 5 high are almost certainly indicative of serious
     trouble.
\xeit
\Iodx{editing}\Iodx{flagging}

     The next step is to locate the bad data more precisely.  Suppose
that you have found a bad row for antenna 3 in right circular
polarization in IF 2 between times ({\it d1\/}, {\it h1\/}, {\it
m1\/}, {\it s1\/}) and ({\it d2\/}, {\it h2\/}, {\it m2\/}, {\it
s2\/}).  You might then rerun {\tt \tndx{LISTR}} with the following
new inputs:
\dispt{SOURCES\qs ' ' \CR} {to select all sources.}
\dispt{TIMER\qs {\it d1 h1 m1 s1 d2 h2 m2 s2} \CR}{to select by time
           range.}
\dispt{ANTENNAS\qs 1 , 2 , 3 \CR}{to list data for antenna 3 with two
           ``control'' antennas.}
\dispt{BASEL\qs 1 , 2 , 3 \CR}{to list all baselines with these three
           antennas.}
\dispt{OPTYPE\qs 'LIST' \CR}{to select column listing format.}
\dispt{DOCRT\qs 1 \CR}{to route the output to terminal at its width.}
\dispt{DPARM\qs = 0 \CR}{amplitude only, no averaging.}
\dispt{STOKES\qs 'RR' \CR}{to select right circular.}
\dispt{BIF\qs 2 \CR}{to specify the ``BD'' IFs.}
\dispt{FLAGVER\qs 1 \CR}{to choose flag table 1.}
\dispt{GO \CR}{to run the program.}
\dispe{This produces a column listing on your terminal of the
amplitude for baselines 1--2, 1--3 and 2--3 at every time stamp
between the specified start and stop times.  The `1--2'' column
provides a control for comparison with the two columns containing the
suspicious antenna.}

     Note that ``amp-scalar'' averaging ignores phase entirely and is
therefore not useful on weak sources, nor can it find jumps or other
problems with the phases.  To examine the data in a phase-sensitive
way, repeat the above process, but set {\us DPARM(2) = 0} rather than
1.  Bad phases will show up as reduced amplitudes and increased {\it
rms\/}'s.

     Once bad data have been identified, they can be expunged using
\hbox{{\tt \tndx{UVFLG}}}.  For example, if antenna 3 RR was bad for
the full interval shown above, it could be deleted with
\dispt{TASK\qs 'UVFLG' ; INP \CR}{to select the editor and check its
          inputs.}
\dispt{TIMER\qs {\it d1 h1 m1 s1 d2 h2 m2 s2} \CR}{to select by time
           range.}
\dispt{BIF\qs 2 ; EIF = BIF \CR}{to specify the ``BD'' IFs.}
\dispt{BCHAN\qs 0 ; ECHAN\qs 0 \CR}{to flag all channels.}
\dispt{FREQID\qs 1 \CR}{to flag only the present {\tt FQ} number.}
\dispt{ANTEN\qs 3 , 0 \CR}{to select antenna 3.}
\dispt{BASEL\qs 0 \CR}{to select all baselines to antenna 3.}
\dispt{STOKES\qs 'RR' \CR}{to select only the RR Stokes (LL was found
           to be okay in this example).}
\dispt{REASON = 'BAD RMS WHOLE SCAN' \CR}{to set a reason.}
\dispt{OUTFGVER\qs 1 \CR}{to select the first (only) flag table.}
\dispt{INP \CR}{be careful with the inputs here!}
\dispt{GO \CR}{to run the task when ready.}

     Continue the process until you have looked at all parts of the
data set that seemed anomalous in the first matrix listing, then rerun
that listing to be sure that the flagging has cleaned up the data set
sufficiently.  If there are lots of bad data, you may find that you
have missed a few on the first pass.  If you change your mind about a
flagging entry, you can use {\tt \tndx{UVFLG}} with {\us OPCODE =
'UFLG'} to remove entries from the flag table.  All adverbs of {\tt
UVFLG} are used when removing entries, so you may use {\tt REASON}
along with the channel, IF, source, et al.~adverbs to select the
entries to be removed.  {\tt OPCODE}s {\tt 'REAS'} and {\tt 'WILD'}
may be used to undo an entry solely based on the {\tt REASON}\@.  If
the table becomes hopelessly messed up, use {\tt \tndx{EXTDEST}} to
delete the flag table and start over or use a higher numbered flag
table. The contents of the flag table may be examined at any time with
the general task {\tt \tndx{PRTAB}} and entries in it may also be
removed with {\tt \tndx{TABED}} and/or \hbox{{\tt TAFLG}}.  Two flag
tables can be merged using {\tt TAPPE}\@.
\Iodx{editing}\Iodx{flagging}

\Subsections{Editing with {\tt EDITA}}{edita}

     The task {\tt \Tndx{EDITA}} uses the graphics planes on the
\AIPS\ TV display to plot data from tables and to offer options for
\Indx{editing} (deleting, \Indx{flagging}) the associated \uv\ data.
Only the {\tt TY} (system temperature), {\tt SN} (solution), and {\tt
CL} (calibration) tables may be used.  EVLA users will not have a {\tt
TY} table, but they may have an {\tt SY} (SysPower) table which may be
used instead.  We recommend using {\tt EDITA} with the {\tt TY} or
{\tt SY} tables to do the initial editing of VLA and EVLA data sets,
probably before running the programs described in \Sec{calbegin}.  For
accuracy in evaluating and flagging your data, it is a good idea to
have the {\tt TY} table filled with the same interval as the data
themselves; see \Sec{fillm}.  Try:
\dispt{TASK\qs 'EDITA ; INP \CR}{to review the inputs needed.}
\dispt{INDI\qs {\it n\/} ; GETN\qs {\it m\/} \CR}{to select the data
         set, $n=3$ and $m=1$ above.}
\dispt{INEXT\qs 'TY' \CR}{to use the system temperature table.}
\dispt{INVERS\qs 0 \CR}{to use the highest numbered table, usually 1.}
\dispt{TIMER\qs 0 \CR}{to select all times.}
\dispt{FREQID\qs 3 \CR}{Select {\tt FQ} entry 3.}
\dispt{BIF\qs 1 ; EIF\qs 0 \CR}{to specify all IFs; you can then
         toggle between them interactively and even display all at
         once.}
\dispt{ANTENNAS\qs 0 \CR}{to display data for all antennas.}
\dispt{ANTUSE\qs 1 , 2 , 3 , 4 , 5 , 6 , 7 \CR}{to display initially
         the first 7 antenn\ae, editing antenna 1.  Others may be
         selected interactively.}
\dispt{FLAGVER\qs 1 \CR}{to use flag ({\tt FG}) table 1.}
\dispt{OUTFGVER\qs 0 \CR}{to create a new flag table with the flags
      from {\tt FG} table 1 plus the new flags.}
\dispt{SOLINT\qs 0 \CR}{to avoid averaging any samples.}
\dispt{DOHIST\qs FALSE \CR}{to omit recording the flagging in the
         history file.}
\dispt{DOTWO\qs TRUE \CR}{to view a $2^{\und}$ observable for
         comparison}
\dispt{CROWDED\qs TRUE ; DO3COL\qs TRUE \CR}{to allow plots with all
         polarizations and/or IFs simultaneously, using color to
         diffentiate the polarizations and IFs.}
\dispt{INP \CR}{to review the inputs.}
\dispt{GO \CR}{to run the program when inputs set correctly.}
\dispe{If you make multiple runs of {\tt EDITA}, it is important to
make sure that the flagging table entries are all in one version
of the {\tt FG} table. The easiest way to ensure this is to should set
{\tt FLAGVER} and {\tt OUTFGVER} to 0 and keep it that way for all runs
of {\tt EDITA}\@.  This may create an excessive number of flag tables,
but unwanted ones may be deleted with {\tt EXTDEST}\@.  If you make a
mistake two flag tables may be merged with the task {\tt TAPPE}\@.  A
sample display from {\tt EDITA} is shown on the next page.}

The following discussion assumes that you have read \Sec{xas} and are
familiar with using the \AIPS\ TV display.  An item in a menu such as
that shown in the figure is selected by moving the TV cursor to the
item (holding down or pressing the left mouse button).  At this point,
the menu item will change color.  To obtain information about the
item, press \AIPS\ TV ``button D'' (usually the {\tt D} key and also
the {\tt F6} key on your keyboard).  To tell the program to execute
the menu item, press any of \AIPS\ TV buttons A, B, or \hbox{C}.
Status lines around the display indicate what is plotted and which
data will be flagged by the next flagging command.  In the figure
below, only the displayed antenna (2), and time range will be flagged.
You must display at least a few lines of the message window and your
main {\tt AIPS} window since the former will be used for instructions
and reports and the latter will be needed for data entry (\eg\ antenna
selection).

\begin{figure}
\centering
%\resizebox{\hsize}{!}{\gname{edita}}
\resizebox{\hsize}{!}{\gbb{544,435}{edita}}
\caption[{\tt EDITA} display]{A display of a sample TV screen from
\hbox{{\tt EDITA}}, made using the \AIPS\ task {\tt TVCPS} to produce
a negative black-and-white display.  System temperatures are being
used to edit VLBA data.  The {\tt \Tndx{EDITA}} menu (in the boxes),
the status lines (at the bottom), the editing area (bottom) of a
portion of the data from the selected antenna (1), the subsidiary
plots of data from selected secondary antenn\ae\ (3, 5, 7, 9), the
edit tool (bar or box), and the edit location values are displayed in
different graphics planes which normally appear in different colors.
In this example, with {\tt CROWDED=TRUE}, four IFs but only one
polarization are displayed and may be edited simultaneously.  Both
polarizations can be displayed together along with either one or all
four IFs.  Color may be used to differentiate polariztions and IFs.
\Iodx{editing}\Iodx{flagging}\Todx{EDITA}}
\label{fig:edita}
\end{figure}

The first thing to do with {\tt EDITA} is to look at all of the
polarizations, IFs, and antenn\ae, in order to flag the obviously bad
samples (if any).  Use {\tt SWITCH POLARIZATION} to switch between
polarizations and {\tt ENTER IF} to select the IF to edit.
Alternatively, {\tt NEXT CORRELATOR} will cycle through all
polarizations and IFs.  If {\tt CROWDED} was set to true, {\tt SWITCH
POLARIZATION} will cycle through displaying both polarizations as well
as each separately, and {\tt ENTER IF} will accept 0 as indicating
all.  {\tt NEXT CORRELATOR} shows only one correlator at a time, but
can switch away from a multi-correlator display.  These options appear
only if there is more than one polarization and/or more than one IF in
the loaded data.  Use {\tt ENTER ANTENNA} to select the antenna to be
flagged and {\tt ENTER OTHER ANT} to select secondary antenn\ae\ to
be displayed around the editing area.  If the secondary antenn\ae\
have no obvious problems, then they do not have to be selected for
editing.  {\tt \Tndx{EDITA}} will plot all of the times in the
available area, potentially making a very crowded display.  You may
select interactively a smaller time range or ``frame'' in order to see
the samples more clearly.  It is necessary to select each frame in
order to edit the data in that frame so it helps to make the TV screen
as big as possible with the {\tt F2} button or your window manager.
Note that the vertical scales used by {\tt EDITA} are linear, but that
the horizontal scale is irregular and potentially discontinuous.
Integer hours are indicated by tick marks and the time range of the
frame is indicated.  Use {\tt FLAG TIME} or {\tt FLAG TIME RANGE} to
delete data following instructions which will appear on the message
window.  While you are editing, the source name, sample time and
sample value currently selected will be displayed in the upper left
corner of the TV screen.  This information can also be used to
determine if {\tt QUACK} is needed.

Having flagged all obviously bad points, select {\tt SWITCH ALL IF},
{\tt SWITCH ALL TIME}, {\tt SWITCH ALL ANT}, and {\tt SWITCH ALL POL}
so that the next flag command(s) apply to all of the data.  (Decide
whether the flags should apply only to the source(s) displayed or to
all sources and set {\tt SWITCH ALL SOURC} appropriately.)  Set the
{\tt SCAN LENGTH} long enough to include the shorter of the full scan
and about 12 samples.  Then display the difference between the current
sample and the running mean by selecting \hbox{{\tt SHOW TSYS - <T>}}.
Use {\tt FLAG ABOVE} and {\tt FLAG BELOW} to flag all samples more
than a few sigma away from the local mean.  Finally, apply your
flagging to your \uv\ data set by selecting \hbox{{\tt EXIT}}.
\Iodx{flagging}\Iodx{editing}

At this point, return to \Sec{initedit} to run {\tt \tndx{QUACK}}
followed by the first pass of the gain calibration.  Then run {\tt
TVFLG} below with {\tt DOCAL TRUE} so that the data will be displayed
on the same flux scale for all baselines.

\Subsections{Editing with {\tt TVFLG}}{tvflg}

       If your data are seriously corrupted, contain numerous
baselines, and you like video games, {\tt \Tndx{TVFLG}} is the
visibility editor of choice.  The following discussion assumes that
you have read \Sec{xas} and are familiar with using the \AIPS\ TV
display.  The following inputs are suggested:
\dispt{TASK\qs 'TVFLG' ; INP \CR}{to review the inputs needed.}
\dispt{INDI\qs {\it n\/} ; GETN\qs {\it m\/} \CR}{to select the data
         set, $n=3$ and $m=1$ above.}
\dispt{SOURCES\qs ' ' \CR} {to select all sources.}
\dispt{TIMER\qs 0 \CR}{to select all times.}
\dispt{STOKES\qs 'RRLL' \CR}{to select both right and left circular
         polarizations; you can then toggle between RR and LL
         interactively.}
\dispt{FREQID\qs 3 \CR}{Select {\tt FQ} entry 3.}
\dispt{BIF\qs 1 ; EIF\qs 2 \CR}{to specify both VLA IFs; you can then
         toggle between the two interactively.}
\dispt{ANTENNAS\qs 0 \CR}{to display data for all antennas.}
\dispt{BASELINE\qs 0 \CR}{to display data for all baselines.}
\dispt{DOCALIB\qs 1 \CR}{to apply initial calibration to the data.}
\dispt{FLAGVER\qs 1 \CR}{to use flag ({\tt FG}) table 1.}
\dispt{OUTFGVER\qs 0 \CR}{to create a new flag table with the flags
      from {\tt FG} table 1 plus the new flags.}
\dispt{DPARM\qs = 0 \CR}{to use default initial displays and normal
         baseline ordering.}
\dispt{DPARM(6)\qs = 30 \CR}{to declare that the input data are
         30-second averages, or to have the data averaged to 30
         seconds.  Note that one can interactively increase the time
         averaging, in integer units of {\tt DPARM(6)}, after the
         master grid is created.}
\dispt{DPARM(5)\qs = 10 \CR}{to expand the flagging time ranges
         by 10 seconds in each direction.  The times in the master
         grid are average times and may not encompass the times of the
         samples entering the average without this expansion.}
\dispt{DOCAT\qs 1 \CR}{to save the master grid file.}
\dispt{INP \CR}{to review the inputs.}
\dispt{GO \CR}{to run the program when inputs set correctly.}
\dispe{If you make multiple runs of {\tt TVFLG}, it is important to
make sure that the flagging table entries are all in one version
of the {\tt FG} table.  The easiest way to ensure this is to set
{\tt FLAGVER} and {\tt OUTFGVER} to 0 and keep it that way for all
runs of {\tt TVFLG}\@.  If you make a mistake, two flag tables may be
merged with the task {\tt TAPPE}\@.}

     {\tt TVFLG} begins by constructing a ``master grid'' file of all
included data.  This can be a long process if you include lots of data
at once, although in {\tt 31DEC13} a new, faster, large-memory method
of gridding is usually used.  It is probably better to use the channel
selection (including averaging channels with {\tt NCHAV}), IF
selection, source selection, and time range selection adverbs to build
rather smaller master grid files and then to run {\tt TVFLG} multiple
times.  It will work with all data included, allowing you to select
interactively which data to edit at any one moment and allowing you to
resume the editing as often as you like.  But certain operations (such
as undoing flags) have to read and process the entire grid, and will
be slow if that grid is large.  The master grid file is always
cataloged (on {\tt IN2DISK} with class {\tt TVFLGR}), but is saved at
the end of your session only if you set {\us DOCAT = 1} (actually $ >
0$) before starting the task.  To resume {\tt TVFLG} with a
pre-existing master grid file, set the adverb {\tt IN2SEQ} (and {\tt
IN2DISK}) to point at it.  When resuming in this way, {\tt
\Tndx{TVFLG}} ignores all of its data selection adverbs since they
might result in a different master grid than the one it is going to
use.  If you wish to change any of the data selection parameters, \eg\
channels, IFs, sources, times, or time averaging, then you must use a
new master grid.\Iodx{editing}\Iodx{flagging}

     Kept with the master grid file is a special file of {\tt TVFLG}
flagging commands.  This file is updated as soon as you enter a new
flagging command, making the master grid and your long editing time
virtually proof from power failures and other abrupt program
terminations.  These flagging commands are not entered into your
actual {\it uv\/} data set's flagging ({\tt FG}) table until you exit
from {\tt TVFLG} and tell it to do so.  During editing, {\tt TVFLG}
does not delete data from its master grid; it just marks the flagged
data so that they will not be displayed.  This allows you to undo
editing as needed during your {\tt TVFLG} session(s).  When the flags
are transferred to the main {\it uv\/} data set, however, the flagged
data in the master grid are fully deleted since undoing the flags at
that point has no further meaning.  When you are done with a master
grid file, be sure to delete it (with {\tt ZAP}) since it is likely to
occupy a significant amount of disk.

     {\tt TVFLG} keeps track of the source name associated with each
row of data.  When averaging to build the master grid and to build the
displayed grids, {\tt TVFLG} will not average data from different
sources and will inform you that it has omitted data if it has had to
do so for this reason.  For multi-source files, the source name is
displayed during the {\tt CURVALUE}-like sections.  However, the
flagging table is prepared to flag {\it all\/} sources for the
specified antennas, times, {\it etc.}~or just the displayed source.
If you are flagging two calibrator scans, you may wish to do all
sources in between as well.  Use the {\tt SWITCH SOURCE FLAG}
interactive option to make your selection before you create flagging
commands.  Similarly, you will need to decide whether flagging
commands that you are about to prepare apply only to the displayed
channel and/or IF, or to all possible channels and/or IFs.  In
particular, spectral-line observers often use {\tt TVFLG} on the
pseudo-continuum ``channel-0'' data set, but want the resulting flags
to apply to all spectral channels when copied to the spectral-line
data set.  They should be careful to select all channels before
generating any flagging commands.  Each flagging command generated is
applied to a list of Stokes parameters, which {\it does not have to
include\/} the Stokes currently being displayed.  When you begin {\tt
TVFLG} and whenever you switch displayed Stokes, you should use the
{\tt ENTER STOKES FLAG} option to select which Stokes are to be
flagged by subsequent flagging commands.

     If you get some of this wrong, you can use the {\tt UNDO FLAGS}
option in {\tt TVFLG} if the flags have not yet been applied to the
\uv\ data set.  Or you can use tasks {\tt UVFLG}, {\tt TABED} or {\tt
TAFLG} to correct errors written into the {\tt FG} table of your
multi-source \uv\ data set.  Flag tables are now used with both
single- and multi-source data sets.

     {\tt TVFLG} displays the data, for a single IF, channel
(average), and Stokes, as a grey-scale display with time increasing up
the screen and baseline number increasing to the right.  Thus
baselines for the VLA run from left to right as 1--1, 1--2, 1--3,
$\ldots$, 2--2, 2--3, $\ldots$, 27--27, 27--28, and 28--28.  An input
parameter ({\tt DPARM(3) = 1} allows you to create a master grid and
display baselines both as, say 1--2 and 2--1.  An interactive
(switchable) option allows you to order the baselines from shortest to
longest (ignoring projection effects) along the horizontal axis.

     The interactive session is driven by a menu which is displayed on
a graphics overlay of the TV display.  An example of this full display
is shown on the next page.  Move the cursor to the desired operation
(noting that the currently selected one is highlighted in a different
color on many TVs) and press button A, B, or C to select the desired
operation; pressing button D produces on-line help for the selected
operation.  The first (left-most column) of choices is:
\dispx{OFFZOOM  }{turn off any zoom magnification}
\dispx{OFFTRANS }{turn off any black \&\ white enhancement}
\dispx{OFFCOLOR }{turn off any pseudo-coloring}
\dispx{TVFIDDLE }{interactive zoom, black \&\ and white enhancement,
         and pseudo-color contours as in {\tt AIPS}}
\dispx{TVTRANSF }{black \&\ white enhancement as in {\tt AIPS}}
\dispx{TVPSEUDO }{many pseudo-colorings as in {\tt AIPS}}
\dispx{DO WEDGE ? }{switches choice of displaying a step wedge}
\dispx{LOAD xxxx  }{switch TV load transfer function to xxxx}
\dispx{LIST FLAGS }{list selected range of flag commands}
\dispx{UNDO FLAGS }{remove flags by number from the {\tt FC} table
         master grid}
\dispx{REDO FLAGS }{re-apply all remaining flags to master grid}
\dispx{SET REASON }{set reason to be attached to flagging commands}
\dispx{DO LABEL ? }{turn axis labeling on and off}
\dispe{Note: when a flag is undone, all cells in the master grid which
were first flagged by that command are restored to use.  Flag commands
done after the one that was undone may also, however, have applied to
some of those cells.  To check this and correct any improperly
un-flagged pixels, use the {\tt REDO FLAGS} option.  This option even
re-does {\tt CLIP} operations!  After an {\tt UNDO} or {\tt REDO FLAGS}
operation, the TV is automatically re-loaded if needed.  Note that the
{\tt UNDO} operation is one that reads and writes the full master
grid. \Iodx{editing}\Iodx{flagging}\Todx{TVFLG}}

The load to the TV for all non-phase displays may be done with all
standard transfer functions: LINear, LOG, SQRT, LOG2 (more extreme
log).  The menu shows the next one in the list (xxxx above) through
which you may cycle.  The TV is reloaded immediately when a new
transfer function is selected.

\begin{figure}
\centering
%\resizebox{!}{6.35in}{\gname{tvflgnew}}
\resizebox{!}{6.35in}{\gbb{530,730}{tvflg14}}
\caption[{\tt TVFLG} display]{A display of a sample TV screen from
\hbox{{\tt TVFLG}}, made using the \AIPS\ task {\tt TVCPS} to produce
a negative black-and-white display.  The {\tt TVFLG} menu (in the
boxes) is displayed in a graphics plane which is normally colored
light green.  The status line(s) at the bottom and optional axis
labels are displayed in a graphics plane that is normally cyan in
color.  The data are grey scales in a TV memory and may be enhanced in
black-and-white or pseudo-colored. The particular display chosen is
the amplitude of the vector difference between the sample and a
running vector average of samples surrounding it.  This particular
parameter is sensitive to both phase and amplitude problems and may
save you the extra time of looking at phase and amplitude separately.
It requires that there be data to average, but does not blur the
flagging by the averaging interval (as the RMS method does).  The
visibility data are from the JVLA\@.  All baselines are shown once
only in baseline number order.  Antenna 8 and 27 are missing for all
times, while antennas 11 and 22 are missing for some times in the
piece displayed here.  The displayed data are the RR Stokes samples
and have been windowed to exclude some times.  Flag commands generated
at the moment illustrated will flag all source names, one spectral
channel (actually channels 9-24 averaged here), one IF, and Stokes RR
only (from a 2-Stokes data set).  The step wedge and labeling options
have been selected.
\Iodx{editing} \Iodx{flagging}\Todx{TVFLG}}
\label{fig:tvflg}
\end{figure}

     Column 2 offers type-in controls of the TV display and controls
of which data are to be flagged.  In general, the master grid will be
too large to display on the TV screen in its entirety.  The program
begins by loading every $n^{\uth}$ baseline and time smoothing by {\it
m\/} time intervals in order to fit the full image on the screen.
However, you may select a sub-window in order to see the data in more
detail. You may also control the range of intensities displayed (like
the adverb {\tt PIXRANGE} in {\tt TVLOD} inside \hbox{{\tt AIPS}}).
The averaging time to smooth the data for the TV display may be
chosen, as may the averaging time for the ``scan average'' used in
some of the displays.  Which correlators are to be flagged by the next
flagging command may be typed in.  All of the standard Stokes values,
plus any 4-bit mask may be entered.  The spectral channel and IF may
be typed in.  Flagging may be done only for the current channel and IF
and source, or it may be done for all channels and/or IFs and/or
sources.  Note that these controls affect the next {\tt LOAD}s to the
TV or the flagging commands prepared after the parameter is changed.
When the menu of options is displayed at the top of the TV, the
current selections are shown along the bottom.  If some will change on
the next load, they are shown with a trailing asterisk.  Column 2
contains
\dispx{ENTER BLC          }{Type in a bottom left corner pixel number
          on the terminal}
\dispx{ENTER TRC          }{Type in a top right corner pixel number on
          the terminal}
\dispx{ENTER AMP PIXRANGE }{Type in the intensity range to be used
         for loading amplitude images to the TV}
\dispx{ENTER PHS PIXRANGE }{Type in the phase range to be used for
        loading phase images to the TV}
\dispx{ENTER RMS PIXRANGE }{Type in the intensity range to be used
        for loading images of the rms to the TV}
\dispx{ENTER R/M PIXRANGE }{Type in the value range to be used for
        loading rms/mean images to the TV}
\dispx{ENTER SMOOTH TIME  }{Type in the time smoothing (averaging)
        length in units of the master grid cell size}
\dispx{ENTER SCAN TIME    }{Type in the time averaging length for
        the ``scan average'' in units of the master grid cell size}
\dispx{ENTER CHANNEL     }{Type in the desired spectral channel
        number using the terminal}
\dispx{ENTER IF          }{Type in, on the terminal, the desired IF
        number}
\dispx{ENTER STOKES FLAG  }{To type in the 4-character string which
        will control which correlators (polarizations) are flagged.
        Note: this will apply only to subsequent flagging commands. It
        should be changed whenever a different Stokes is displayed.}
\dispx{SWITCH SOURCE FLAG }{To switch between having all sources
        flagged by the current flag commands and having only those
        sources included in this execution of {\tt TVFLG} flagged. The
        former is desirable when a time range encompasses all of 2
        calibrator scans.}
\dispx{SWITCH ALL-CH FLAG }{To reverse the flag all channel status;
        applies to subsequent flag commands}
\dispx{SWITCH ALL-IF FLAG }{To cycle the flag all IFs status;
        applies to subsequent flag commands}
\dispe{The all-channel flag remains true if the input data set has
only one channel and the all-IF flag remains true if the input data
set has no more than one \hbox{IF}.  If there are more than 2 IFs, the
{\tt SWITCH ALL-IF FLAG} cycles between flagging one IF, flagging a
range of IFs, and flagging all IFs.  When going to the range of IFs,
it will ask you to enter the desired range.
\Iodx{editing}\Iodx{flagging}\Todx{TVFLG}}

     An extra word should be said about the ``scan average'' to which
reference was made above.  This is used solely for displaying the
difference of the data at time {\it T\/} and the average of the data
at times near {\it T\/}.  This average is computed with a ``rolling
buffer.''  Thus, for a scan average time of 30 seconds and data at
10-second intervals, the average for a set of 7 points is as follows:
\bve
      time        average of times
       00           00   10   20
       10           00   10   20
       20           10   20   30
       30           20   30   40
       40           30   40   50
       50           40   50   60
       60           40   50   60
\end{verbatim}\eve

     The third column of options is used to control which data are
displayed and to cause the TV display to be updated.  The master grid
must be converted from complex to amplitude, phase, the rms of the
amplitude, or the rms divided by the mean of the amplitude for
display.  It may also be converted to the amplitude of the vector
difference between the current observation and the ``scan average'' as
defined above or the absolute value of the difference in amplitude
with the scalar-average amplitude or the absolute value of the
difference in phase with the vector scan average.  Furthermore, the
baselines may be reordered in the TV display by their length rather
than their numerical position. This column has the options:
\dispx{DISPLAY AMPLITUDE  }{To display amplitudes on the TV}
\dispx{DISPLAY PHASE      }{To display phases on the TV}
\dispx{DISPLAY RMS        }{To display amplitude rms on the TV}
\dispx{DISPLAY RMS/MEAN   }{To display amplitude rms/mean on the TV}
\dispx{DISPLAY VECT RMS   }{To display vector amplitude rms on the TV}
\dispx{DISPLAY VRMS/VAVG  }{To display vector amplitude rms/mean on
            the TV}
\dispx{DISPLAY AMP V DIFF }{To display the amplitude of the difference
            between the data and a running (vector) ``scan average''}
\dispx{DISPLAY AMPL DIFF  }{To display the abs(difference) of the
            amplitude of the data and a running scalar average of the
            amplitudes in the ``scan''}
\dispx{DISPLAY PHASE DIFF }{To display the abs(difference) of the
            phase of the data and the phase of a running (vector)
            ``scan average''}
\dispx{DISPLAY STOKES {\it xx\/}}{To switch to Stokes type {\it xx\/}
            (where {\it xx\/} can be RR, LL, RL, LR, etc.~as chosen by
            the {\tt STOKES} adverb).}
\dispx{SORT BY {\it xxxxxxxx\/}}{To switch to a display with the {\it
            x\/} axis (baseline) sorted by ordered by {\tt LENGTH} or
            by {\tt BASELINE} number}
\dispx{OFF WINDOW + LOAD  }{Reset the window to the full image and
            reload the TV}
\dispx{SET WINDOW + LOAD  }{Interactive window setting (like {\tt
            TVWINDOW}) followed by reloading the TV}
\dispx{LOAD LAST PIECE    }{Reload the TV with the previous piece of
            the total time range.}
\dispx{LOAD NEXT PIECE    }{Reload the TV with the next piece of the
            total time range.}
\dispx{LOAD               }{Reload TV with the current parameters}
\dispe{{\tt SET WINDOW + LOAD} is ``smarter'' than {\tt TVWINDOW} and
will not let you set a window larger than the basic image.  Therefore,
if you wish to include all pixels on some axis, move the TV cursor
outside the image in that direction.  The selected window will be
shown.  When there are more times than will fit on the TV screen at
the current smoothing (averaging) time, the task divides the data up
into overlapping time-range ``pieces.''  When it has done so, the
{\tt LOAD LAST PIECE} and {\tt LOAD NEXT PIECE} menu items will
appear.  This lets you view one piece after the next, rotating through
all pieces, to edit each time interval at full resolution.  Note that
a {\tt FLAG BASELINE} will flag that baseline through all pieces.
\Iodx{editing}\Iodx{flagging}\Todx{TVFLG}}

     The fourth column is used to select the type of flagging to be
done.  During flagging, a TV graphics plane is used to display the
current pixel much like {\tt CURVALUE} in {\tt AIPS}\@.  Buttons
A and B  do the flagging (except A switches corners for the area and
time-range modes).  Button C also does the flagging, but the program
then returns to the main menu rather than prompting for more flagging
selections.  Button D exits back to the menu without doing any
additional flagging.  Another graphics plane is used to show the
current area/time/baseline being flagged.  All flagging commands can
create zero, one, two, or more entries in the flagging list; hit
button D at any time.  There are also two clipping modes, an
interactive one and one in which the user enters the clip limits from
the terminal.  In both, the current image computed for the TV (with
user-set windows and data type, but not any other windows or alternate
pixels etc.~required to fit the image on the TV) is examined for
pixels which fall outside the allowed intensity range.  Flagging
commands are prepared and the master file blanked for all such pixels.
In the interactive mode, buttons A and B switch between setting the
lower and upper clip limits, button C causes the clipping to occur
followed by a return to the main menu, and button D exits to the menu
with no flagging.  The options are
\dispx{FLAG PIXEL      }{To flag single pixels}
\dispx{FLAG/CONFIRM    }{To flag single pixels, but request a yes
          or no on the terminal before proceeding}
\dispx{FLAG AREA       }{To flag a rectangular area in baseline-time}
\dispx{FLAG TIME RANGE }{To flag all baselines for a range of times}
\dispx{FLAG ANTENNA-DT }{To flag all baselines to a specific antenna
          for a range of times}
\dispx{FLAG A TIME     }{To flag all baselines for a specific time}
\dispx{FLAG BASELINE   }{To flag all times for a specific baseline}
\dispx{FLAG BASELINE-DT}{To flag a time range for a specific baseline}
\dispx{CLIP BY SET \#S  }{To enter from the terminal a clipping range
          for the current mode and then clip high and low samples}
\dispx{CLIP INTERACTIV }{To enter with the cursor and LUTs a clipping
          range for the current mode and then clip data outside the
          range.}
\dispx{CLIP BY FORM    }{To clip selected channels/IFs using the
         ``method'' and clipping range of some previous clip
         operation}
\dispx{LOAD NEXT IF/ST }{Load TV with the next IF or polarization.}
\dispx{LOAD NEXT CHAN  }{To load the next spectral channel to the TV
         with current parameters}
\dispx{LOAD PREV CHAN  }{To load the previous spectral channel to the
         TV with current parameters}
\dispe{The {\tt CLIP BY FORM} operation allows you to apply a clipping
method already used on one channel/IF to other channels and/or IFs.
It asks for a command number (use {\tt LIST FLAGS} to find it) and
applies its display type (amp, phase, rms, rms/mean, differences),
averaging and scan intervals and clip levels to a range of channels,
IFs and Stokes (as entered from the terminal).  To terminate the
operation, doing nothing, enter a letter instead of one of the
requested channel or IF numbers.  To omit a Stokes, reply, if
requested for a flag pattern, with a blank line.  You may watch the
operation being carried out on the TV as it proceeds.}

     The right-most column has only the option:
\dispx{EXIT }{Resume {\tt AIPS} and, optionally, enter the flags
          in the data}
\dispe{Before the flags are entered in the data, {\tt TVFLG} asks you
whether or not you actually wish to do this.  You must respond yes or
no.  Note that, if the master grid is to remain cataloged, there is no
need to enter the flagging commands every time you decide to exit the
program for a while.  In fact, if you do not enter the commands, you
can still undo them later, giving you a reason not to enter them in
the main \uv\ data set too hastily.\Iodx{editing}\Iodx{flagging}
\Todx{TVFLG}}

     The two most useful data modes for editing are probably amplitude
and amplitude of the vector difference.  The former is useful for
spotting bad data over longer time intervals, such as whole scans.
The latter is excellent for detecting short excursions from the norm.
For editing uncalibrated data, rms of two time intervals is useful,
but the rms modes require data to be averaged (inside {\tt TVFLG}) and
therefore reduce the time resolution accuracy of the flagging.
If you edit by phase, consider using the pseudo-coloration scheme that
is circular in color (option {\tt TVPSEUDO} followed by button B)
since your phases are also circular.

     Using {\tt TVFLG} on a workstation requires you to plan the real
estate of your screen.  We suggest that you place your message server
window and your input window side-by-side at the bottom of the screen.
Then put the TV window above them, occupying the upper 70--90\%\ of
the screen area.  (Use your window manager's tools to move and stretch
the TV window to fill this area.)  Instructions and informative,
warning and error messages will appear in the message server window.
Prompts for data entry (and your data entry) appear in the input
window.  Remember to move the workstation cursor into the input window
to enter data (such as IF, channel, antenna numbers, and the like) and
then to move the cursor back into the TV area to select options, mark
regions to be flagged, adjust enhancements, and so on.

\Subsections{Baseline corrections}{blcorr}

     Sometimes, \eg\ during a VLA array re-configuration, your
observations may have been made when one or more of the antennas had
their positions poorly determined.  The positional error is usually
less than a centimeter at the VLA, but even this may affect your data
significantly.  The most important effect is a slow and erroneous
phase wind which is a function of source position and time.  Since
this error is a function of source position, it cannot be removed
exactly using observations of a nearby calibrator, although the error
will be small if the target source is close to the calibrator.  In
many observations, the target sources and calibrators are sufficiently
close to allow this phase error to be ignored.  Self-calibration will
remove this error completely {\it if\/} you have enough
signal-to-noise to determine the correction during each integration.

    The maximum phase error introduced into the calibrated visibility
data by incorrect \indx{antenna coordinates} $\Delta\phi_{B}$, in
radians, by a \indx{baseline error} of $\Delta B$ meters is given by
    $$\Delta\phi_{B} \approx 2\pi\Delta\theta\Delta B / \lambda$$
where $\Delta\theta$ is the angular separation between the calibrator
and the target source in radians and $\lambda$ is the wavelength in
meters.

    Note, however, that the error due to the phase-wind is not the
only error introduced by incorrect antenna positions.  A further, but
much smaller effect, will be incorrect gridding of the data due to the
erroneous calculation of the baseline spatial frequency components
{\it u\/}, {\it v\/} and {\it w\/}.  This effect is important only for
full primary beam observations in which the antenna position error is
of the order of a meter.  It is highly unlikely that such a condition
will occur.  Note too, that this error {\it cannot\/} be corrected by
the use of self-calibration.  However, after correcting the antenna
position with {\tt CLCOR}, you may run {\tt UVFIX} to compute
corrected values of $u, v,$ and $w$.  The maximum phase error in
degrees, $\Delta\phi_{G}$, caused by incorrect gridding of the {\it
u,v,w\/} data is
       $$\Delta\phi_{G} \approx 360 \Delta\epsilon \Delta\Theta$$
where $\Delta\epsilon$ is the antenna position error in antenna
diameters and $\Delta\Theta$ is the angular offset in primary beams.

    If \indx{baseline error}s are significant they need to be removed
from your data before calibration.  It is important to do this to
{\tt CL} table 1, right after running {\tt FILLM}.  For the VLA, use the
task {\tt \Tndx{VLANT}}.  This task determines and applies the antenna
position corrections found by the VLA operations staff after your
observation was complete.  To run {\tt VLANT}:
\dispt{TASK\qs 'VLANT' \CR}{}
\dispt{INDISK\qs {\it m\/} ; GETN\qs {\it n\/} \CR}{to get the correct
          data set.  Note that you don't have to keep doing this
          unless you switch between different input data files.}
\dispt{FREQID {\it 1\/} \CR}{to choose {\tt FQ} 1.}
\dispt{SUBARRAY {\it x\/} \CR}{to choose the antenna table to correct.}
\dispt{GAINVER\qs 1 \CR}{to choose the correct version of the {\tt CL}
          table to read. A new one will produced.}
\dispt{GO \CR}{to run \hbox{{\tt VLANT}}.}

For arrays other than the VLA, use {\tt \tndx{CLCOR}} to enter the
antenna position corrections (in meters) in a new {\tt CL} table and
the old {\tt AN} table.  This must be done for each affected antenna
in turn.  {\tt CLCOR} puts the corrections into the {\tt AN} table as
well as the {\tt CL} table, so it is wise to save the {\tt AN} table
before running {\tt CLCOR} by running {\tt TASAV}.
\dispt{TASK\qs 'CLCOR' \CR}{}
\dispt{INDISK\qs {\it m\/} ; GETN\qs {\it n\/} \CR}{to get the correct
          data set.  Note that you don't have to keep doing this
          unless you switch between different input data files.}
\dispt{SOURCES\qs '\ ' ;STOKES\qs '\ ' \CR}{to do all sources, all
          Stokes,}
\dispt{BIF\qs 0 ; EIF\qs 0 \CR}{and all IFs.}
\dispt{SUBARRAY {\it x\/} \CR}{to choose the correct sub-array.}
\dispt{OPCODE\qs 'ANTP' \CR}{to select the antenna position correction
          mode.}
\dispt{GAINVER\qs 1 \CR}{to choose the correct version of the {\tt CL}
          table to read.}
\dispt{GAINUSE\qs 0 \CR}{to have {\tt CLCOR} create a new table.}
\dispt{ANTENNA\qs {\it k\/} \CR}{to select antenna.}
\dispt{CLCORPRM\qs $\Delta b_{x} , \Delta b_{y} , \Delta b_{z} , 0, 0,
          0, 1$ \CR}{to add the appropriate antenna corrections in
          meters; the l in {\tt CLCORPRM(7)} indicates VLA phase
          conventions rather than VLB conventions.}
\dispt{GO \CR}{to run \hbox{{\tt CLCOR}}.}
\dispe{The program will need to be run as many times as there are
antennas for which positional corrections must be made.  Set {\tt
GAINUSE} and {\tt GAINVER} both to 2 after the first correction.
Otherwise, with the above adverbs, {\tt CLCOR} will make multiple {\tt
CL} table versions each with only one correction in them.  Note that
subsequent calibration must be applied to {\tt CL} table 2 to create
higher versions of the calibration table.  This new {\tt CL} table
(version 2) will replace version 1 in all of the subsequent sections
on calibration.  Thus, in subsequent executions of {\tt CALIB}, you
must apply these corrections by specifying {\us DOCALIB\qs TRUE ;
GAINUSE\qs 0} (FOR higheST VERSION).  Note too that {\tt CLCOR} and
{\tt VLANT} change the antenna file for the changed antenna
location(s).  Therefore, it is wise to save the {\tt AN} table before
running {\tt CLCOR} or {\tt VLANT} by running {\tt TASAV}.}

Note that NRAO's data analysts use the \AIPS\ task {\tt \tndx{LOCIT}}
and procedure {\tt \tndx{BASFIT}} to determine the antenna position
corrections.  These are available to the general user, but a data set
designed to determine antenna corrections is normally required.  Such
data sets consist of about 100 observations of a wide range of phase
calibrators taken as rapidly as possible.

\Sects{Antenna-based complex gain solutions}{calgain}

     At this point, we assume that you have removed the worst of the
bad calibrator data (if any) and have run {\tt \tndx{CALIB}} over as
large a {\tt UVRANGE} as possible for each calibrator.  The resulting
gain tables can be brought to a consistent amplitude scale,
bootstrapping the unknown fluxes of the secondary calibrators.  Final
pass(es) of {\tt CALIB} are done if needed and then the solution
tables are merged into a full \Indx{calibration} ({\tt CL}) table.

\subsections{Bootstrapping secondary flux-density calibrators}

      Task {\tt \tndx{GETJY}} can be used to determine the flux
density of the secondary flux calibrators from the primary flux
calibrator based on the flux densities set in the {\tt SU} table and
the antenna gain solutions in the {\tt SN} tables.  The {\tt SU} and
{\tt SN} tables will be updated by {\tt GETJY} to reflect the
calculated values of the secondary calibrators' flux densities.  This
procedure should also work if (incorrect) values of the secondary
calibrators' flux densities were present in the {\tt SU} table when
{\tt CALIB} was run. Bad or redundant {\tt SN} tables should be
deleted using {\tt EXTDEST} before running {\tt GETJY}, or avoided by
selecting tables one at a time with adverb \hbox{{\tt SNVER}}.

      To use {\tt GETJY}:
\dispt{TASK\qs 'GETJY' ; INP \CR}{}
\dispt{SOURCES\qs '{\it cal1\/}' , '{\it cal2\/}' , '{\it cal3\/}'
           $\ldots$ \CR}{to select secondary flux calibrators.}
\dispt{CALSOU\qs '3C286' , '\ ' \CR}{to specify primary flux
           calibrator(s).}
\dispt{CALCODE\qs '\ ' \CR}{to use all calibrator codes.}
\dispt{BIF\qs 1 ; EIF\qs 2 \CR}{to do both IFs.}
\dispt{FREQID\qs 1 \CR}{to use {\tt FQ} number 1.}
\dispt{ANTENNAS\qs 0 \CR}{to include solutions for all antennas.}
\dispt{TIMERANG\qs 0 \CR}{to include all times.}
\dispt{SNVER\qs 0 \CR}{to use all {\tt SN} tables.}
\dispt{INP \CR}{to review inputs.}
\dispt{GO \CR}{to run the task when the inputs are okay.}

      {\tt GETJY} will give a list of the derived flux densities and
estimates of their uncertainties.  These are now found by ``robust''
methods and additional information about numbers of aberrant solutions
are given.  If any of the uncertainties are large, then reexamine the
{\tt SN} tables as described above and re-run {\tt CALIB} and/or {\tt
GETJY} as necessary.  Multiple executions of {\tt GETJY} will not
cause problems as previous solutions for the unknown flux densities
are simply overwritten.  You may wish to run the task {\tt
\tndx{SOUSP}} to determine the spectral indices of your calibrators
from their fluxes in the {\tt SU} table.  You can even replace the
values in the {\tt SU} table with the curve fit by {\tt SOUSP} and, in
{\tt 31DEC15} correct the gains in one or more {\tt SN} tables with
the newly determined fluxes.  These \indx{spectral index} parameters
may be useful in running {\tt BPASS} and {\tt PCAL}\@.  However, {\tt
BPASS} knows the flux coefficients for all standard calibration
sources and can fit the spectral index of other calibration sources
from the {\tt SU} table.

\subsections{Full calibration}

     Once you have determined the flux densities of all your gain
calibrators, you are ready to complete the first pass of the
\Indx{calibration}.  At this point, many observers take a conservative
viewpoint and delete their existing {\tt SN} table(s) with
\dispt{INEXT\qs 'SN' \CR}{to specify the {\tt SN} table.}
\dispt{INVERS\qs -1 \CR}{to delete all versions.}
\dispt{EXTDEST \CR}{to do the deletion.}
\dispe{This step forces you to re-run {\tt CALIB} for all your gain
calibration sources and is not required if the previous bootstrapping
calibrations included all antennas and most correlators, for these
calibrators.}

     Procedure {\tt \tndx{VLACALIB}} may be used for your gain
calibration sources as you did previously.
\dispt{INDI\qs {\it n\/} ; GETN\qs {\it m\/} \CR}{to select the data
          set, $n=3$ and $m=1$ above.}
\dispt{CALSOUR\qs = '{\it aaaa\/}' , '{\it xxxx\/}' \CR}{to name two
          calibration sources using the same {\tt UVRANGE}\@.}
\dispt{UVRANGE\qs {\it uvmin\/} {\it uvmax\/} \CR}{\uv\ limits, if
          any, in kilo$\lambda$.}
\dispt{ANTENNAS\qs {\it list\ of\ antennas\/} \CR}{antennas to use for
          the solutions, see discussion above.}
\dispt{REFANT\qs {\it n\/} \CR}{reference antenna number.}
\dispt{MINAMPER\qs 10 \CR}{display warning if baseline disagrees in
         amplitude by more than {\it 10\%\/}\ from the model.}
\dispt{MINPHSER\qs 10 \CR}{display warning if baseline disagrees by
         more than $10^{\circ}$ of phase from the model.}
\dispt{DOPRINT\qs -1 \CR}{to dispense with all the print out this
         time.}
\dispt{FREQID\qs 1 \CR}{use {\tt FQ} number 1.}
\dispt{INP\qs VLACALIB \CR}{to review inputs.}
\dispt{VLACALIB \CR}{to make the solution and print results.}
\dispe{If there are different \uv\ ranges for different sources, then
re-run the procedure with changed parameters, such as:}
\dispt{CALSOUR\qs = '{\it cal1\/}' , '{\it cal2\/}' , '{\it cal3\/}'
       \CR}{to name secondary flux calibrator(s).}
\dispt{ANTENNAS\qs 0 \CR}{solutions for all antennas.}
\dispt{UVRANGE\qs 0 \CR}{no \uv\ limits, or range if any, in
        kilo$\lambda$.}
\dispt{INP\qs VLACALIB \CR}{to review inputs.}
\dispt{VLACALIB \CR}{to process the secondary calibrators.}
\dispe{At this time, you should use as many antennas and as large a
{\tt UVRANGE} as you can for each calibrator, consistent with its
spatial structure.}

\subsections{Final (?) initial global calibration}

     At this point you should have gain and phase solutions for the
times of all \Indx{calibration} scans, including the correct flux
densities for the secondary calibrators.  The next step is to
interpolate the solutions derived from the calibrators into the {\tt
CL} table for all the sources.  {\tt CLCAL} may be run multiple times
if subsets of the sources are to be calibrated by corresponding
subsets of the calibrators, unless you limit it to one or more tables
with {\tt SNVER} and {\tt INVERS}, {\tt CLCAL} assumes that all {\tt
SN} tables contain only valid solutions and concatenates all of the
{\tt SN} tables with the highest numbered one.  Therefore, any bad
{\tt SN} tables should be removed before using {\tt \tndx{CLCAL}}\@.
For \indx{polarization} calibration, it is essential that you
calibrate the primary flux calibrator (3C48 or 3C286) also so that you
can solve for the left minus right phase offsets and apply {\tt
PCAL}\@.

     {\tt CLCAL} has caused considerable confusion and user error
because it implements to somewhat contrary views of its process.  The
older view, represented by previous versions of this \Cookbook, had
the user gradually building a final {\tt CL} table from multiple runs
of {\tt CLCAL}, each with a selected set of calibration sources,
target sources, antennas, time ranges, and so forth.  In this scheme,
the user had to take great care that the final {\tt CL} table actually
contained information for all antennas, sources, and times for which
it would be needed.  It was easy to get this wrong!  The second and
now prevailing view is that every execution of {\tt CLCAL} should
write a new {\tt CL} table containing all sources, antennas, and
times, but with a selected subset modified by the current execution.
This leads to there being a potentially large number of {\tt CL}
tables, but no data will be flagged due to the absence of data in the
{\tt CL} table.  The user will still have to be careful to insure that
all {\tt CL} records have received the needed calibration information.

      To use {\tt CLCAL}:
\dispt{TASK\qs CLCAL ; INP \CR}{to review the inputs.}
\dispt{SOURCES\qs '{\it sou1\/}' , '{\it sou2\/}' , '{\it sou3\/}' ,
           $\ldots$ \CR}{sources to calibrate, '\ ' means all.}
\dispt{CALSOUR\qs '{\it cal1\/}' , '{\it cal2\/}' , '{\it cal3\/}' ,
           $\ldots$ \CR}{calibrators to use for \hbox{{\tt SOURCES}}.}
\dispt{FREQID\qs {\it n\/} \CR}{use {\tt FQ} number {\it n\/}.}
\dispt{OPCODE\qs 'CALP' \CR}{to combine {\tt SN} tables into a {\tt
           CL} table, passing any records not altered this time.}
\dispt{GAINVER\qs 0 \CR}{to select the latest {\tt CL} table as
           input.}
\dispt{GAINUSE\qs 0 \CR}{to select a new output {\tt CL} table.}
\dispt{REFANT\qs {\it m\/} \CR}{to select the reference antenna; needed
           only if {\tt REFANT} reset since {\tt CALIB} was run.}
\dispt{INTERP\qs '2PT' \CR}{to use linear interpolation of the
           possibly smoothed calibrations..}
\dispt{SAMPTYPE\qs ' ' \CR}{to do no time-smoothing before the
           interpolation.}
\dispt{SAMPTYPE\qs 'BOX' \CR}{to use boxcar smoothing, followed by
           interpolation.}
\dispt{BPARM\qs {\it n\/} , {\it n\/} \CR}{to smooth, if {\tt BOX}
           selected, with an {\it n\/}-hr long boxcar in amplitude and
           phase.}
\dispt{DOBLANK\qs 1 \CR}{to replace failed solutions with smoothed
           ones but to use all previously good solutions without
           smoothing.}
\dispt{INP \CR}{to check inputs.}
\dispt{GO \CR}{to run {\tt CLCAL}\@.}
\dispe{Calibrator sources may also be selected with the {\tt QUAL} and
{\tt CALCODE} adverbs; {\tt QUAL} also applies to the sources to be
calibrated.  Note that {\tt REFANT} appears in the inputs because
\AIPS\ references all phases to those of the reference antenna.  If
none is given, it defaults to the one used in the most solutions.}

     The smoothing and interpolation functions in {\tt CLCAL} have
been separated into two adverbs and the smoothing parameters are
now conveyed with {\tt BPARM} and {\tt ICUT}\@.  In smoothing, the
{\tt DOBLANK} adverb is particularly important; it controls whether
good solutions are replaced with smoothed ones and whether previously
failed solutions are replaced with smoothed ones.  One can select
either or both.

     Note that {\tt \tndx{CLCAL}} uses both the {\tt GAINUSE} and {\tt
GAINVER} adverbs.  This is to specify the input and output {\tt CL}
table versions, which should be different.  If you are building a
single {\tt CL} table piece by piece, then these must be set carefully
and normally held fixed.  In the more modern view, they are set to
zero and the task takes the latest {\tt CL} table as input and makes a
new one.  {\tt CL} table version 1 is intended to be a ``virgin''
table, free of all injury from any calibration you do using the \AIPS\
package.  It may not always be devoid of information, as ``on-line''
corrections may be made and recorded here by some telescope systems,
\eg\ the VLBA\@.  The VLA, through tasks {\tt FILLM} or {\tt INDXR},
can now put opacity and antenna gain information in this file.  {\tt
CLCAL} and most other \AIPS\ tasks are forbidden to over-write version
1 of the {\tt CL} table.  This protects it from modification, and
keeps it around so that you may {\it reset\/} your \Indx{calibration}
to the raw state by using {\tt EXTDEST} to destroy all {\tt CL} table
extensions with versions higher than 1.  Be careful doing this, since
you rarely want to delete {\tt CL} version 1.  Should you destroy {\tt
CL} table version 1 accidentally, you may generate a {\it new\/} {\tt
CL} table version 1 with the task {\tt INDXR}\@.  This new {\tt CL}
table may contain the calibration generated from the weather and
antenna gain files.

     If you have any reason to suspect that the calibration has gone
wrong --- or if you are calibrating data for the first time --- you
should examine the contents of the output {\tt CL} table.  {\tt
\tndx{LISTR}} with {\us OPTYPE = 'GAIN'} will print out the amplitudes
and phases in the specified {\tt CL} or {\tt SN} table.  Note that
these tables can be very large.  Use the {\tt SOURCES} and {\tt
TIMERANG} adverbs to limit the output, or look at it on your terminal
({\us \tndx{DOCRT} = 1}) so that you can stop the display whenever
you have had enough.  Task {\tt \tndx{SNPLT}} will provide you with a
graphical display which may be easier on the eye.  EVLA users may wish
to examine the {\tt SY} table with {\tt LISTR} or {\tt SNPLT}; bad
Psum and Pdif values may point to areas of bad data.

     The task {\tt \tndx{EVASN}} may help you determine the degree of
phase and amplitude coherence in your calibration table.  A lack of
coherence suggests that the calibration is rather uncertain.

     The most important step in the calibration is your verification
that everything has gone according to plan.  To check this, you should
produce matrix listings for all your calibrator sources.  For
simplicity in interpretation, limit each listing to the {\tt UVRANGE}
to which you limited the calibrator during calibration.  Thus:
\dispt{TASK\qs 'LISTR' \CR}{}
\dispt{DOCRT\qs -1 \CR}{to direct output to the printer.}
\dispt{SOURCES\qs '{\it cal1\/}' , '{\it cal2\/}' , '{\it cal3\/}' ,
           $\ldots$ \CR}{to list all selected calibrators by name.}
\dispt{UVRANGE\qs {\it uvmin\/} {\it uvmax\/} \CR}{\uv\ limits, if
           any, in kilo$\lambda$.}
\dispt{OPTYP\qs 'MATX' \CR}{to get the matrix form of listing.}
\dispt{DOCALIB\qs TRUE \CR}{to list with calibration applied.}
\dispt{GAINUSE\qs 0 \CR}{TO point to the {\it latest\/} gain table.}
\dispt{FREQID \qs {\it n\/} \CR}{list data for {\tt FQ} {\it n\/}.}
\dispt{DPARM\qs = 5 , 1 , 0 \CR}{to have amplitude and phase using
           scalar scan averaging.}
\dispt{BIF\qs 1; EIF 0 \CR}{to select all IFs, {\tt LISTR} will loop
      over IFs.}
\dispt{INP \CR}{to review the inputs.}
\dispt{GO\qs \CR}{to run the program when inputs set correctly.}
\dispe{The matrix average amplitudes for the calibrators in this
listing should be very close to the values that you entered with {\tt
SETJY} (or which were derived by {\tt GETJY}) and the phases in all
rows and columns for these sources should be very close to zero.}

     If some rows and columns of the amplitude matrices are
systematically different from the mean, the amplitude calibration for
the associated antennas is imperfect.  The reasons for this should be
investigated.  More flagging of visibilities, scans, or antennas, may
be indicated.  If the phase matrices have all elements near zero, then
the phase \Indx{calibration} is in good shape.  If some calibrators
have discrepant phases and others do not, the discrepant calibrators
are probably resolved.  Note that you will not be able to detect
errors in the assumed positions of your calibrators at this stage if
you have used the usual 2-point interpolation of the calibration.
Position errors in the calibrators have now become phase and position
errors in the target sources.

{\tt \Tndx{ANBPL}} converts baseline-based data before or after
calibration into antenna-based quantities.  In particular, the
calibrated weights are very sensitive to problems with amplitude
calibration.
\dispt{DEFAULT\qs ANBPL \CR}{to select task and initialize all
            its parameters.}
\dispt{IND {\it m\/} ; GETN\qs {\it n\/} \CR}{to specify the
            multi-source data set.}
\dispt{STOKES\qs 'HALF' ; TIMERANG\qs 0 \CR}{to DISPLAY both
            parallel-hand polarizations.}
\dispt{FREQID\qs 1 \CR}{to select {\tt FQ} value to image.}
\dispt{BIF\qs 1 ; EIF 0 \CR}{to select all IFs.}
\dispt{BCHAN\qs {\it n\/} ; ECHAN\qs {\it m\/} \CR}{to combine a range
            of channels.}
\dispt{DOCALIB\qs 1 \CR}{to apply calibration.}
\dispt{GAINUSE\qs 0 \CR}{to use highest numbered {\tt CL} table.}
\dispt{FLAGVER\qs 1 \CR}{to edit data.}
\dispt{DOBAND\qs 3 ; BPVER\qs 1 \CR}{to correct bandpass with time
            smoothing using table 1.}
\dispt{BPARM\qs 2, 17 \CR}{to plot weight versus time.}
\dispt{NPLOTS\qs 3 ; DOTV\qs 1 \CR}{To plot 4 antennas per page on the
            TV.}
\dispt{DOCRT\qs 0 \CR}{to suppress printed versions of the
            antenna-based values.}
\dispt{INP \CR}{to review the inputs.}
\dispt{GO\qs \CR}{to run {\tt \tndx{ANBPL}\@.}}

     If the previous steps indicate serious problems and/or you are
seriously confused about what you have done and you want to start the
calibration again, you can use the procedure {\tt VLARESET} from the
{\tt RUN} file {\tt VLAPROCS} to reset the {\tt SN} and {\tt CL} tables.
\dispt{INP\qs VLARESET \CR}{to verify the data set to be reset.}
\dispt{\tndx{VLARESET}\qs \CR}{to reset {\tt SN} and {\tt CL} tables.}

\Sects{Polarization calibration}{polcal}

     The \Indx{calibration} of visibility data sensitive to linear
\Indx{polarization} involves two distinct operations: (1) determining
and correcting the data for the effects of imperfect telescope feeds
and (2) removing any systematic phase offsets between the two systems
of orthogonal polarization.  These two components of polarization
calibration will be considered separately.

     The effective feed response is parametrized most generally by
its polarization ellipticity and the orientation of the major axis of
that ellipse.  For the VLA, it appears to be adequate to make the
simpler assumption that each polarization is corrupted by a small
complex gain times the orthogonal polarization.

     In general, the polarization of the calibrator(s) to be used to
determine the feed parameters will not be known {\it a priori\/} and
must be determined along with the feed parameters.  Observations of a
given source (or sources) over a wide range ($\ge 90^{\circ}$) of
parallactic angles is necessary to separate calibrator polarization
from the feed parameters.  Task {\tt \indx{LISTR}} may be used to
determine the \indx{parallactic angle}s at which data have been taken:
\dispt{TASK\qs 'LISTR' \CR}{}
\dispt{SOURCES\qs '{\it cal1\/}' , '{\it cal2\/}' , '{\it cal3\/}' ,
           $\ldots$ \CR}{list all calibrators to be used.}
\dispt{INEXT\qs 'CL' \CR}{to determine parallactic angle at times in
           {\tt CL} table.}
\dispt{INVER\qs 1 \CR}{{\tt CL} version 1.}
\dispt{FREQID\qs {\it n\/} \CR}{to use {\tt FQ} number {\it n\/}.}
\dispt{OPTYPE\qs 'GAIN' \CR}{to use gain table rather than visibility
           data.}
\dispt{DPARM\qs = 9 , 0 \CR}{to display parallactic angle.}
\dispt{INP \CR}{to review the inputs.}
\dispt{GO \CR}{to run the program when inputs set correctly.}

     Multiple calibrators may be used in determining the feed
polarization, but the data from them must be accurately calibrated.
In particular, the phase calibration of any calibrator used to
determine antenna polarizations should be determined from that
calibrator itself (\ie\ the source should be self-calibrated).  Note
that this will normally have occurred for all gain calibrators if the
procedure described in the previous sections was followed.

     The normal phase calibration technique treats parallel-hand
visibilities in the two orthogonal polarizations independently.  Thus,
there will be a systematic phase difference between the two
polarizations systems.  This difference may be due to differences in
instrumental phase offset for the two systems or due to the
propagation medium (\ie\ Faraday rotation) or both.  Faraday rotation
effects are particularly bothersome as they may be time variable and
increase rapidly with wavelength.  For data at L band or longer
wavelengths, \AIPS\ should be given an estimate of the ionospheric
Faraday rotation measure using task \hbox{{\tt FARAD}}.  This task
computes the ionospheric rotation measure using either total electron
content from a nearby ionospheric monitoring station (Boulder Colorado
for the VLA) or an empirical model that uses the monthly mean Zurich
sunspot number (R1) as a measure of solar activity.  If monitoring
data are available, they should be used in preference to the model.
{\tt FARAD} enters the ionospheric Faraday rotation measure into the
{\tt CL} table.  This is used by {\tt PCAL}  when determining antenna
polarization parameters and is used by other \Indx{calibration} tasks
to de-rotate the data when \Indx{polarization} corrections are
applied.  {\tt FARAD} may be run any number of times with different
parameters before {\tt \tndx{PCAL}} is run; each run of {\tt
\tndx{FARAD}} over-writes the values written previously.

    Data from Boulder for the total electron content for year {\it
mm\/} is contained in the file {\tt TECB}{\it .mm\/} in the directory
with logical name \hbox{{\tt AIPSIONS}}.  Currently, data are
available for 1980 onwards through part of 1992.  Some gaps in the TEC
data occur, particularly for years 1980 and 1988 and for times when
solar activity has been high and no reliable estimate of the total
electron content could be made.  Unfortunately, we are no longer able
to get the TEC data from Boulder.  If your VLA data are more recent
than early 1992, you can consider using the ionospheric model in {\tt
FARAD}, but you should be aware that the model is crude and should
check that it improves matters before using it in your final
calibration.

     The phase offsets between the right-hand and left-hand
polarizations at a given time may be determined from a source with a
known angle of linear polarization USING data which have had the
effects of imperfect feeds removed.  The phase of the right-left
correlations or the conjugate of the left-right correlations indicates
the phase difference between the two polarizations.

     The (initial) need for ionospheric corrections can be bypassed if
either (1) you use unpolarized sources in {\tt PCAL}, and/or (2) the
ionosphere was well behaved during your observations.  Typical
rotation measures are only a few and therefore affect only L and
longer-wavelength bands.  The ionosphere is almost always well enough
behaved to be ignored at shorter wavelengths and is usually able to be
ignored even at L band.  Changes of around $10^{\circ}$ in the
relative phases of R and L polarizations are not enough to disrupt a
{\tt PCAL} solution seriously.  And, fortunately, calibrators at long
wavelengths, such as P band, tend to be unpolarized.  In general, if
the ionosphere is well behaved, it can be ignored.  If it is bad, no
simple model is able to correct it and you may simply have to forget
about polarization for that observing run.  Note that an apparent
position angle variation of 3C286 with time probably indicates that
ionospheric rotation is significant.  But, if 3C286 shows large
rotations, it does not follow that {\it its\/} rotation can be applied
to other directions in the sky.  All it implies reliably is that a
{\it model\/} is needed.

     Polarization calibration of EVLA data is discussed extensively in
\Rappen{EVLAdata}.  These data are fundamentally multi-channel and a
spectral-dependent polarization solution is required.  {\tt PCAL} and
{\tt RLDIF} now can determine and apply the calibration in a
channel-dependent fashion.  For data not requiring the spectral
options, polarization calibration may be performed on amplitude- and
phase-calibrated VLA data using the following five-step procedure:

{\bf Step 1:} Run {\tt \tndx{PCAL}} on one or more phase calibrator
sources observed with a wide range of parallactic angles:
\dispt{TASK\qs 'PCAL' \CR}{}
\dispt{CALSOUR\qs '{\it cal1\/}' , '{\it cal2\/}' , '{\it cal3\/}' ,
          $\ldots$ \CR}{list all calibrators to be used.}
\dispt{TIMERANG\qs 0 \CR}{to use all times.}
\dispt{ANTENNAS\qs 0 \CR}{to solve for all antennas.}
\dispt{UVRANGE\qs {\it uvmin\/} {\it uvmax\/} \CR}{to set \uv\ limits,
          if any, in kilo$\lambda$.}
\dispt{BIF\qs 1 ; EIF\qs 2 \CR}{to do both IFs.}
\dispt{DOCALIB\qs 1 \CR}{to apply the calibration to the sources
          (very important!).}
\dispt{GAINUSE\qs 0 \CR}{to use the latest {\tt CL} table.}
\dispt{CLR2N \CR}{to clear {\tt IN2NAME} {\it etc.}~since there is no
           Clean-image model.}
\dispt{FREQID\qs {\it n\/} \CR}{to use {\tt FQ} value {\it n\/}; only
           one polarization solution can be stored.}
\dispt{DOMODEL\qs 0 \CR}{to solve for the polarization parameters of
           the calibration sources.  {\tt PCAL} can use a model only
           if that model has $Q = U = 0$ since it cannot solve for the
           right-left phase difference.}
\dispt{SOLINT\qs 2 \CR}{to use a 2-minute solution interval; scan
           averages are usually sufficient.}
\dispt{SOLTYPE\qs 'APPR' \CR}{to use linear approximation model.}
\dispt{PRTLEV 1 \CR}{to display the results and some diagnostic
           information.}
\dispt{REFANT\qs n \CR}{only if {\tt REFANT} reset since {\tt CALIB}
           run.}
\dispt{INP \CR}{to review the inputs.}
\dispt{GO \CR}{to run the program when inputs set correctly.}
\dispe{{\tt PCAL} will list the fitted values of the antenna
polarization parameters and the source polarizations with estimates of
the uncertainties.  If these results do not appear reasonable (\eg\
large errors or large corrections or inconsistent solutions for the
calibrator polarizations at neighboring frequencies), more editing and
a rerun of {\tt PCAL} may be necessary.  {\tt PCAL} puts the derived
source polarizations in the {\tt SU} table and the antenna feed values
in the {\tt AN} table.  These values may be examined later with {\tt
\tndx{PRTAN}} and \hbox{{\tt \tndx{PRTAB}}}.
\Iodx{polarization}\Iodx{calibration}}

     The right minus left phase difference may not in fact be
independent of time making the choise of reference antenna more
important.  In {\tt 31DEC12}, the task {\tt \Tndx{SNREF}} examines the
effect of the choice of reference antenna on the apparent stability of
right minus left phases in {\tt SN} or {\tt CL} tables and can create
tables with different choices.

     {\bf Step 2:}.  Use {\tt \tndx{RLDIF}} to determine the apparent
right minus left phase angle of the polarization calibrator source,
\eg\ \indx{3C286} or \indx{3C138}:
\dispt{TASK\qs 'RLDIF' \CR}{}
\dispt{SOURCE\qs '3C286' , '\ ' \CR}{to view only the polarization
            angle calibrator.}
\dispt{TIMERANG\qs 0 \CR}{to check all times.}
\dispt{ANTENNAS\qs {\it list\ of\ antennas\/} \CR}{antennas to use;
            the list used for \hbox{{\tt CALIB}}.}
\dispt{UVRANGE\qs {\it uvmin\/} {\it uvmax\/} \CR}{to limit \uv, if
            appropriate.}
\dispt{BIF\qs 1 ; EIF\qs 0 \CR}{to view all IFs.}
\dispt{FREQID\qs {\it n\/} \CR}{to view the current {\tt FQ} value
            ({\it n\/}).}
\dispt{DOCALIB\qs TRUE \CR}{to list with calibration applied.}
\dispt{DOPOL\qs TRUE \CR}{to correct for feed polarization and Faraday
            rotation.}
\dispt{GAINUSE\qs 0 \CR}{to use the latest {\tt CL} table.}
\dispt{DOPRINT\qs -1 \CR}{to print the results on the line printer,
       {\tt DOPRINT} $> 0$ prints on your terminal screen, and {\tt
       DOPRINT = 0} does no printing.}
\dispt{SPECTRAL\qs -1 \CR}{to do continuum polarization solutions.}
\dispt{DOAPPLY\qs -1 \CR}{to examine the solutions without applying
       them to the tables.}
\dispt{INP \CR}{to review the inputs.}
\dispt{GO \CR}{to run the program when inputs set correctly.}
\dispe{The matrix of scan-averaged right minus left phase angles
(actually RL and conjugate of LR polarizations) will be printed.
Check that none of the phases differ from the mean by more than a few
degrees.  If any do, then use {\tt UVFLG} to edit these data and go
back to step 1.  After the matrix of phases, the average over the
matrix of the right minus left phases is displayed.  This is the
number to be used in step 4.  {\tt RLDIF} returns these, one for each
IF, in the {\tt CLCORPRM} adverb array.  It even averages over
multiple calibrator scans, getting a reliable estimate of the average
by iteratively discarding outliers.  To see the results,
type\Iodx{polarization}\Iodx{calibration}\todx{OUTPUTS}}
\dispt{OUTPUTS\qs \CR}{to examine the output adverb values.}
\dispe{{\tt \tndx{LISTR}} may also be used with {\tt OPTYPE\qs 'MATX' ;
STOKES\qs 'POLC'} to make the printer display, one IF at a time.  But
you will have to do any averaging and placing of the results in {\tt
CLCORPRM} yourself.}

     This method will fail if the calibrator source (3C286 or 3C138,
usually) is heavily resolved and the atmospheric phase stability is
poor.  (These two are frequently coupled!)  Under these conditions,
the self-calibration of the calibrator will have failed and will have
to be done especially for the polarization calibration.  In the steps
below, you may safely relax the \uv\ limits by about 20\%, but should
solve only for phases using {\tt SOLMODE = 'P'}.  The process consists
of:
\xben
\vspace{-7pt}
\ITEMx{2.1} Apply {\tt \tndx{CALIB}} to the inner (short-baseline)
     antennas on the calibrator source using the rules in the table
     found in \Sec{1passgain} but relaxed a bit.  Set {\us DOCALIB = 1
     ; GAINUSE = 0 ; SOLMODE = 'P'}.
\ITEMx{2.2} Use {\tt \tndx{CLCAL}} to apply these solutions to the
     calibrator source.
\ITEMx{2.3} Run {\tt \tndx{LISTR}} for cross-hand phases using {\it
     only\/} the antennas used with \hbox{{\tt CALIB}}.
\ITEMx{2.4} Use {\tt \tndx{EXTDEST}} to delete {\tt CL} table 3, a most
     important step.
\xeen
\vspace{-14pt}
\noindent After correcting the calibration, repeat steps 2.1 and 2.2 and
the special calibration until satisfactory results are obtained.

     {\bf Step 3:} Use {\tt \tndx{TASAV}} to copy all your table files
to a dummy \uv\ data set, saving in particular the {\tt CL} table with
the results of the amplitude and phase \Indx{calibration}.  This step
is not essential, but it reduces the magnitude of the disaster if the
the next step is done incorrectly.  (Note - this may be a good idea at
several stages of the calibration process!)
\dispt{TASK\qs 'TASAV' \CR}{}
\dispt{CLRO\qs \CR}{Use default output file file name.}
\dispt{INP \CR}{to review the (few) inputs.}
\dispt{GO\qs \CR}{to run the program.}
\dispe{The task {\tt TACOP} may be used to recover any tables that get
trashed during later steps.  {\tt CLCOR} will make a new {\tt CL}
table now, so a {\tt TACOP} step is not needed.}

     {\bf Step 4:} The right minus left phase offset corrections
should be made using task {\tt \tndx{RLDIF}} although the old
mechanism using {\tt \tndx{CLCOR}} will also work.  Use:
\dispt{TGET RLDIF}{to get the parameters used in the last and most
       successful run of {\tt RLDIF}\@.}
\dispt{DOPRINT\qs 0 \CR}{to turn off all printing.}
\dispt{DOAPPLY\qs 1 \CR}{to apply the corrections to the {\tt SU},
       {\tt CL}, and {\tt AN} tables.}
\dispt{INP \CR}{to review the inputs.}
\dispt{GO \CR}{To change the tables applying the correction to the
       source polarizations, the antenna D terms, and the calibration
       phases.}

     The old method of correction is no longer recommended.  It goes
as follows.  The phase offset correction is the expected value ({\it
twice\/} the source \Indx{polarization} angle) minus the observed
phases from step 2.  The expected value is 66 degrees for
\indx{3C286}, -18 for \indx{3C138} (at L band, perhaps -24 at higher
frequencies), and -140 for \indx{3C48} (at 6-cm or shorter
wavelengths).  Thus, having used {\tt RLDIF} and  3C286 in step 2
above
\displ{FOR I = 1 : $n$ ; CLCORP(I) = 66 - CLCORP(I) ; END \CR}{to
       convert the returned phases into corrections for {\tt CLCOR},
       where $n$ is the number of IFs.}
\dispe{Then}
\dispt{TASK\qs 'CLCOR' \CR}{}
\dispt{SOURCE\qs ' ' ; ANTENNAS\qs 0 \CR}{to correct all sources and
          all antennas.}
\dispt{TIMERANG\qs 0 \CR}{to correct all times.}
\dispt{BIF\qs 1 ; EIF 2 \CR}{to correct both IFs.}
\dispt{FREQID\qs {\it n\/} \CR}{to correct only the current {\tt FQ}
          value.}
\dispt{GAINVER\qs 0 \CR}{to modify the latest {\tt CL} table produced
          by {\tt CLCAL}.}
\dispt{GAINUSE\qs 0 \CR}{to make a new {\tt CL} table containing the
           phase corrections as well as all previous calibrations.}
\dispt{OPCODE\qs 'POLR' \CR}{to do right minus left phase offset
           correction.}
\dispt{STOKES\qs 'L' \CR}{correction applied to left circular
           polarization.}
\dispt{INP \CR}{to review the inputs.}
\dispt{GO\qs \CR}{to run the program when inputs set correctly.}
\dispe{Task {\tt \tndx{RLCOR}} may be used to apply this correction
directly to a $uv$ data set which is especially useful for
single-source files to which {\tt CLCOR} does not apply.}

      This will cause {\tt \tndx{CLCOR}} to apply appropriate
corrections to the {\tt CL}, {\tt SU}, and {\tt AN} tables.  If the
{\tt CL} table becomes hopelessly corrupted, delete it and return to
Step 3. If the {\tt AN} table is corrupted, then {\tt PCAL} must be
re-run. If more than one {\tt CL} table needs to be corrected, use the
{\us OPCODE='POLR'} option only once; other {\tt CL} tables must be
corrected using {\us OPCODE='PHAS'} and correcting 1 IF at a time.
{\tt CLCOR} (with {\tt OPCODE = 'POLR'}) may be applied multiple times
to the {\it same\/} {\tt CL} table, in order to get the R-L phases
``right.''  But you must {\it not\/} apply {\tt CLCOR} in succession to
different {\tt CL} tables of the same database.  If there is any
doubt, rerun \hbox{{\tt PCAL}}.  The best way to judge if all is well
in the final polarization solution is to look at the spread in the
cross-hand phases for 3C286 or 3C138 (step 5 below).  If the spread
(``eyeball rms'') is less than 3 degrees, then all is well.  If more
than ten, then there is definitely something wrong.
\Iodx{polarization}\Iodx{calibration}

     {\bf Step 5:} Use {\tt \tndx{RLDIF}} to verify the polarization
corrections:
\dispt{TASK\qs 'RLDIF' \CR}{}
\dispt{SOURCE\qs '{\it cal1\/}' , '{\it cal2\/}' , $\ldots$ \CR}{to
           list the calibrators to be checked.}
\dispt{TIMERANG\qs 0 \CR}{to display all times.}
\dispt{ANTENNAS\qs {\it list\ of\ antennas\/} \CR}{to list the
           antennas to use.}
\dispt{UVRANGE\qs {\it uvmin\/} {\it uvmax\/} \CR}{to set \uv\ limits,
           if appropriate.}
\dispt{BIF\qs 1 ; EIF\qs 0 \CR}{to list all IFs.}
\dispt{FREQID\qs {\it n\/} \CR}{to use the current {\tt FQ} value
           ({\it n\/}).}
\dispt{DOCALIB\qs 1 \CR}{to list with calibration applied.}
\dispt{DOAPPLY\qs -1 \CR}{to leave the table values unchanged.}
\dispt{DOPOL\qs TRUE \CR}{to correct for feed polarization and Faraday
           rotation.}
\dispt{GAINUSE\qs 0 \CR}{to use {\tt CL} table written by \hbox{{\tt
           CLCOR}}.}
\dispt{DOCRT\qs 1}{to display on your terminal.}
\dispt{INP \CR}{to review the inputs.}
\dispt{GO\qs \CR}{to run the program when inputs set correctly.}
\dispe{Note well: all of this calibration process must be done with
only one {\tt FQ} at a time.  {\tt PCAL} with {\us FQID = 2} will
over-write solutions done for any other \hbox{{\tt FQID}}.}

      The phases produced should be consistent.  Significant
deviations of the phase may indicate that further editing is needed or
that residual atmospheric phase errors are still present.  If this
display appears okay, then the polarization corrections may be applied
in {\tt \tndx{SPLIT}} (see below) by specifying {\us DOPOL = 1} when
applying the calibration to produce single-source files.

     An important consideration needs to be mentioned at this point.
The polarization corrections applied to the RL and LR data multiply
the RR and LL values.  If those values are not well known, then the
``calibrated'' cross-hand data will be anything but calibrated.
Normally, you should self-calibrate the parallel hand data of the
target source and apply the phase and amplitude calibrations so
derived to the cross-hand data while applying the polarization D terms
(with {\tt DOPOL 1})\@.  Note that, when polarization is measurable,
there is almost always enough signal to self-calibrate the
parallel-hand data.

\Sects{Spectral-line calibration}{linecal}

     The \Indx{calibration} of \indx{spectral-line} data is very
similar to that of continuum data with the exception that the antenna
gains have to be determined and corrected as a function of frequency
as well as time. The model used by \AIPS\ is to determine the antenna
gains as a function of time using a pseudo-continuum (``channel-0'')
form of the data.  Then the complex spectral response function
(``bandpass'') is determined from observations of one or more strong
continuum sources at or near the same frequency as the line
observation.  In general, the channel-0 data are calibrated using the
recipes in the previous sections of this chapter.  The sub-sections
below are designed to bring out the few areas in which spectral-line
calibration differs from continuum.

\Subsections{Reading the data}{lineread}

     If your data are on a VLA archive tape then they should be read
into \AIPS\ using {\tt \tndx{FILLM}}, as described in \Sec{fillm}.
{\tt FILLM} will fill a typical line observation into two files, a
large one containing the line data only, and a smaller file containing
the ``channel-0'' data.  (Note that {\tt FILLM} computes channel-0
from the line data rather than using the channel-0 provided by the
on-line system.)  The standard \Indx{calibration} and editing steps
are performed on channel 0 and the results copied over to the line
data set.  {\it You must be careful with the tolerance you allow {\tt
FILLM} to use in determining the {\tt FQ} numbers.  If you desire all
of your data to have the same {\tt FQ} number, so that you can
calibrate it all in one pass, then set {\tt CPARM(7)} in {\tt FILLM}
to an appropriately large value.}  If you wish to retain spectral-line
autocorrelation data, you must set {\tt DOACOR} to true.

     By default for the VLA, the channel-0 data are generated by the
vector average of the central {\it 3/4\/} of the observing band.  If
this algorithm is not appropriate for your data, you may generate your
own channel-0 data set by averaging only selected channels.  You may
now select different spectral channels in different IFs.  To do this,
use the task {\tt AVSPC}:
\dispt{TASK\qs 'AVSPC' \CR}{}
\dispt{INDI\qs {\it n\/} ; GETN {\it m\/} \CR}{to specify line data
           set.}
\dispt{OUTDI\qs {\it i\/} ; OUTCL\qs 'CH 0' \CR}{to specify output
           ``channel-0'' data set disk and class.}
\dispt{ICHANSEL\qs 10, 30, 1, 0, 31, 55, 2, 1 \CR}{for example, to
           average every channel between 10 and 30 in all IFs and
           also every other channel between 31 and 55, but only in IF
           1.}
\dispt{GO \CR}{to create a new channel-0 data set.}
\dispe{You might find this necessary when observing neutral hydrogen
at galactic velocities.  Most calibrator sources have some absorption
features at these frequencies.}

\subsections{Spectral-line aspects of {\tt SETJY}}

     The {\tt LISTR} output with {\tt OPTYPE = 'SCAN'} will show
information from the source table including spectral-line parameters.
VLA data from {\tt FILLM} are normally supplied with adequate
information regarding the source velocity, line rest frequency, and
velocity reference (radio versus optical, LSR versus barycentric).
However, data from the EVLA and other telescopes may be missing these
parameters.  {\tt \Tndx{SETJY}} must then be used to fill in the
needed values.  In {\tt 31DEC12}, a new {\tt OPTYPE='VCAL'} option
computes the velocity of the reference channel from first principles.
It is recommended over inaccurate guesses of adverb values in other
{\tt SETJY OPTYPE}s.  It may be run over all sources after the rest
frequencies and velocity reference information has been filled in.

\subsections{Editing the data}

     You should follow the steps outlined in \Sec{caledit} to edit the
calibrator data using the channel-0 data set.  Even though channel-0
data is continuum, be careful to have {\tt TVFLG} and {\tt UVFLG}
generate the flagging commands for all channels, not just channel 1.
Then, copy the resulting {\tt FG} table to the line file. Use {\tt
\tndx{TACOP}}\iodx{flagging}\iodx{editing}:
\dispt{TASK\qs 'TACOP' \CR}{}
\dispt{INDI\qs {\it n\/} ; GETN\qs {\it m\/} \CR}{to specify channel-0
           data set.}
\dispt{OUTDI\qs {\it i\/} ; GETO\qs {\it j\/} \CR}{to specify the line
           data set.}
\dispt{INEXT\qs 'FG' \CR}{to copy the {\tt FG} table.}
\dispt{INVER\qs 1 \CR}{to copy table 1.}
\dispt{NCOUNT\qs 1 \CR}{to copy only one table.}
\dispt{OUTVER\qs 1 \CR}{to copy it to output table 1}
\dispt{INP \CR}{to review the inputs.}
\dispt{GO\qs \CR}{to run the program when inputs set correctly.}
\dispe{Specifying the ``{\tt ALL-CH}'' setting in {\tt TVFLG} and
specifying {\us BCHAN 1 ; ECHAN 0 \CR} in {\tt UVFLG} cause all
channels to be flagged when the {\tt FG} table is copied to the line
data set.}

     Spectral-line observers should also use {\tt \tndx{SPFLG}}
(\Sec{spflg}) to examine and, perhaps, to edit their data.  This task
is very similar to {\tt TVFLG} described in \Sec{tvflg}, but {\tt
SPFLG} displays spectral channels for all IFs on the horizontal axis,
one baseline at a time.  If you have a large number of baselines, as
with the VLA, then you should examine a few of the baselines to check
for interference, absorption (or emission) in your calibrator sources,
and other frequency-dependent effects.  Use the {\tt ANTENNAS} and
{\tt BASELINE} adverbs to limit the displays to a few short spacings
and one or two longer ones as well.  If there are serious
frequency-dependent effects in your calibrators, use {\tt SPFLG} and
{\tt UVFLG} to delete them.  (You might wish to delete the {\tt FG}
table with {\tt EXTDEST} to begin all over again.)  Then use {\tt
\tndx{AVSPC}} to build a new channel-0 data set and repeat the
continuum editing.  Note that you should not copy the {\tt FG} table
from the spectral-line data set to the new continuum one.  The reason
for this is the confusion over the term ``channel.''  If you have
flagged channel 1, but not all channels, in the spectral-line data set
--- a very common occurrence --- then a copied {\tt FG} table would
flag all of the continuum data since it has only one ``channel.''
When you have flagged the channel-0 data set, you can merge the new
flags back into the spectral-line {\tt FG} table with task \hbox{{\tt
\tndx{TABED}}}\iodx{flagging}\iodx{editing}\iodx{spectral-line}.
\dispt{TASK\qs 'TABED' \CR}{}
\dispt{INDI\qs {\it n\/} ; GETN\qs {\it m\/} \CR}{to specify channel-0
              data set.}
\dispt{OUTDI\qs {\it i\/} ; GETO\qs {\it j\/} \CR}{to specify the line
              data set.}
\dispt{INEXT\qs 'FG' \CR}{to copy the {\tt FG} table.}
\dispt{INVER\qs 1 \CR}{to copy table 1.}
\dispt{OUTVER\qs 1 \CR}{to copy it to output table 1.}
\dispt{BCOUNT\qs 1 ; ECOUNT\qs 0 \CR}{to copy from the beginning to
            the end.}
\dispt{OPTYPE\qs 'COPY' \CR}{to do a simple copy appending the input
            table to the output table.}
\dispt{TIMER\qs 0 \CR}{to copy all times.}
\dispt{INP \CR}{to review the inputs.}
\dispt{GO\qs \CR}{to run the program when inputs set correctly.}

     If the channel-0 data set is meaningful for your program sources,
you might consider doing a first-pass editing of them along with your
calibrators before copying the {\tt FG} table back to the line data
set.  If your program sources contain significant continuum emission,
then this is a reasonable operation to perform.  If they do not, then
the standard channel-0 data set is not useful for editing program
sources.  You can use {\tt SPFLG} to edit all channels, or if the
signal is strong in a few channels, you could run {\tt TVFLG} on those
channels from the \indx{spectral-line} data set or average those
with the {\tt BCHAN}, {\tt ECHAN}, and {\tt NCHAV} adverbs.

\Subsections{Bandpass calibration}{BPcal}

     The task {\tt \tndx{BPASS}} is designed to take visibility data
from specified calibrator(s) to determine the antenna-based complex
\Indx{bandpass} functions.  It does this in a manner analogous to
self-calibration in that the data are divided by a source model or the
so-called ``channel 0'' before the antenna gains are determined as a
function of frequency.  These are written to a BandPass ({\tt BP})
table.  The bandpass \Indx{calibration} is the first operation that
should be performed on the line data.  So long as one uses the mode in
which the data are divided by the so-called ``channel 0,'' it is not
necessary to calibrate the data before estimating the bandpasses.
\dispt{TASK\qs 'BPASS' \CR}{}
\dispt{INDI\qs {\it i\/} ; GETN\qs {\it j\/} \CR}{to specify the line
            data set.}
\dispt{CALSOUR\qs '{\it cal1\/}' , '{\it cal2\/}' , $\ldots$ \CR}{to
            specify bandpass calibrators.}
\dispt{FREQID\qs 1 \CR}{to select which {\tt FQ} value to use.}
\dispt{ANTENNAS\qs 0 \CR}{to solve for all antennas.}
\dispt{REFANT\qs {\it n\/} \CR}{to set the reference antenna number.}
\dispt{DOCALIB\qs FALSE \CR}{to avoid applying calibration.}
\dispt{BPASSPRM\qs 0 \CR}{to turn off all ``parameters.''}
\dispt{BPASSPRM(5)\qs 0 \CR}{to divide by channel 0 on a
             record-by-record basis before determining antenna-based
             bandpasses.  Other normalization options are available
             and may be preferred.}
\dispt{IN3DI\qs {\it a\/} ; GET3N\qs {\it b\/} \CR}{to specify the
             channel~0 data file, or}
\dispt{CLR3NAME \CR}{to have channel 0 found from the input data
             themselves.}
\dispt{ICHANSEL\qs 20 50 1}{to use the average, in each IF, of all
             channels from 20 through 50, for example, to determine
             channel 0, when the third input file name is empty.}
\dispt{FLAGVER\qs 1 \CR}{to apply flag table 1.}
\dispt{SOLINT\qs 0 \CR}{to use scan averages.}
\dispt{BPVER\qs 1 \CR}{to select the output {\tt BP} table number.}
\dispt{INP \CR}{to review the inputs.}
\dispt{GO\qs \CR}{to run the program when inputs set correctly.}
\dispe{Be careful with the adverb {\tt SMOOTH}\@.  If you smooth, or
do not smooth, the data while finding a bandpass solution, then you
must apply the same {\tt SMOOTH} adverb values whenever you apply that
bandpass solution to the data.  The only exception is that you may
smooth the data after applying the bandpass solution with {\tt
SMOOTH(1)} values 5 through 8 when you did no smoothing in {\tt
BPASS}\@.}

     The stored bandpass correction table should be corrected for the
spectral index of the calibration source.  Use adverbs {\tt SPECINDX}
and {\tt SPECURVE} to describe the \indx{spectral index} to {\tt
BPASS} and new task {\tt SOUSP} to find the spectral index values from
the fluxes in the source table.  {\tt BPASS} knows the flux
coefficients for all standard calibration sources and, in {\tt
31DEC13}, can fit the spectral index of other calibration sources from
the {\tt SU} table.  It can handle spectral inices for multiple
calibration sources at one time, although the adverbs apply of
necessity only to the first such source.

     The divide by channel 0 option is very convenient in that it
allows one to ignore both source structure (when the bandwidth is
narrow enough) and continuum calibration.  However, the average of
some channels on a record-by-record basis can be rather noisy and the
``division'' operation is actually a subtraction of the average phase
and a division by the average amplitude.  The latter suffers from a
``Ricean'' bias --- the average amplitude will always be larger than
the correct amplitude, averaging one rms larger.  Therefore, if the
continuum calibration is stable (or already known and able to be
applied) and the source structure is negligible, then it would be
better to defer the normalization (on a baseline by baseline basis)
until the data are averaged over {\tt SOLINT} or, better still, to
defer the normalization (on an antenna basis) until the unnormalized
solutions are determined.  {\tt BPASSPRM(5)} and {\tt BPASSPRM(10)}
control the normalization options.  Do note also that, with no
normalization, {\tt BPASS} is capable of replacing any use of {\tt
CALIB} including calibration of the data weights.

     The spectral quality of the final images has been found to be
determined in part by the quality of the bandpass solutions.  In
particular, for reasons which are not yet known, the bandpasses are
not exactly antenna dependent especially in the edge channels.  This
``closure error'' may be measured in individual and statistical ways
by {\tt BPASS} and reported to you.  To check on this problem for your
data set, set
\dispt{MINAMPER\qs {\it a\/} \CR}{to count and, if {\tt BPASSPRM(2)}$
           > 1$, to report amplitude closure failures $> a$ per cent.
           Note that closure errors are accumulated as logarithms so
           that 0.5 and 2.0 are both errors of 100\%.}
\dispt{MINPHSER\qs {\it p\/} \CR}{to count and, if {\tt BPASSPRM(2)}$
           > 1$, to report phase closure failures $> p$ degrees.}
\dispt{BPASSPRM(2)\qs 1 \CR}{to report statistics of amplitude and
           phase closure failures without reporting individual
           failures.}
\dispt{BPASSPRM(6)\qs {\it a\/} \CR}{to report all channels in which
           the average amplitude closure error $> a$ per cent.}
\dispt{BPASSPRM(7)\qs {\it p\/} \CR}{to report all channels in which
           the average phase closure error $> p$ degrees.}
\dispt{SOLTYPE\qs 'R' \CR}{to select robust solutions which discard
           data with serious closure problems.  Try other types if
           there are solution failures.}
\dispe{It is probably a good idea to set {\tt MINAMPER} and {\tt
MINPHSER} fairly high (\ie\ 20 and 12) to make a big deal only about
major excursions, but to set {\tt BPASSPRM(6)} and {\tt (7)} fairly
low (\ie\ 0.5 and 0.5) to view the spectrum of closure errors (which
will look a lot like the spectrum of noise on your final Clean
images).  There is even a task called {\tt \Tndx{BPERR}} which will
summarize and plot the error reports generated by {\tt PBASS} and
written to text files by \hbox{{\tt PRTMSG}}.}

     The bandpass solutions are calculated at each bandpass calibrator
scan.  As a consequence, they are likely to be unevenly spaced in time
and may even have times (due to on-line or later editing) at which
there are solutions for some IFs and polarizations but not all.  When
the latter happens, program source data will be lost unless the
missing solutions are filled in.  The task {\tt \Tndx{BPSMO}} may be
used for this purpose or to create a new {\tt BP} table at regular
time intervals using one of a number of time-smoothing functions.  Set
{\tt APARM(4) = -1} for the ``repair'' mode or set {\tt APARM(4)} to
the desired {\tt BP} interval.

     After the bandpasses have been generated, you can examine them
using tasks {\tt BPLOT} and \hbox{{\tt \tndx{POSSM}}}.  You can obtain
an average from all antennas with\Iodx{calibration}\Iodx{bandpass}
\todx{BPASS} \iodx{spectral-line}
\dispt{TASK\qs 'POSSM' \CR}{}
\dispt{INDI\qs {\it i\/} ; GETN\qs {\it j\/} \CR}{to specify the line
             data set.}
\dispt{SOURCES\qs '{\it cal1\/}' , '{\it cal2\/}' , $\ldots$ \CR}{to
             specify the bandpass calibrators.}
\dispt{ANTENNAS\qs 0 \CR}{to include all antennas.}
\dispt{TIMER\qs 0 \CR}{to average over all times.}
\dispt{BCHAN\qs 1 ; ECHAN 0 \CR}{to display all channels.}
\dispt{BPVER\qs 1 \CR}{to select the {\tt BP} table.}
\dispt{FREQID\qs 1 \CR}{to set the {\tt FQ} value to use.}
\dispt{APARM\qs= -1, 0 \CR}{to do a scalar average and have the plot
             self-scaled and labeled in channels.}
\dispt{APARM(8)\qs 2 \CR}{to plot {\tt BP} table data.}
\dispt{NPLOTS\qs 0 \CR}{to make one plot only, averaging all included
             data.}
\dispt{INP \CR}{to review the inputs --- check closely.}
\dispt{GO\qs \CR}{to run the program when inputs set correctly.}
\dispt{GO\qs \tndx{LWPLA} \CR}{to send the plot to the (PostScript)
             printer/plotter.}
\dispe{To view each antenna individually, using the TV to save paper}
\dispt{DOTV\qs TRUE \CR}{to use the \hbox{TV}.}
\dispt{NPLOTS\qs 1 \CR}{to plot one antenna per page/screen.}
\dispt{GO \CR}{to display the bandpasses, averaged over time, on the
            TV with one antenna per screen.}
\dispe{{\tt POSSM} shows each screen for 30 seconds before going
ahead.  You can cause it wait indefinitely by hitting button A, speed
it up by hitting TV buttons B or C, or tell it to quit by hitting
button D\@.  If {\us DOTV = -1}, then {\tt \tndx{POSSM}} makes
multiple plot extension files, which can be sent to the printer
(individually or collectively) by {\tt LWPLA}\@.  You might want to
use a larger value of {\tt NPLOTS} to reduce the number of pieces of
paper.}

     {\tt \Tndx{BPLOT}} is used to create one or more plots (on the TV
or in plot files) of the selected bandpass table.  The plots will be a
set of profiles separated on the vertical axis by an increment in time
or antenna number (depending on the sort selected).  More than one
plot for more than one antenna or more than one time may be generated.
Multiple IFs and polarizations will be plotted along the horizontal
axis if they are present in the {\tt BP} table and selected by the
adverbs.  Thus, {\tt \Tndx{BPLOT}} is useful for plotting the change in
bandpass shape as a function either of time or of antenna.

     In {\tt 31DEC16}, task {\tt \Tndx{BPEDT}} offers a TV-inteactive
graphical way to check your {\tt BP} table.  It displays the amplitude
and phase of one antenna to be edited and up to 10 more antennas for
comparison.  The task is like {\tt EDITA} except that the horizontal
axis is spectral channel rather than time.  You may walk through the
entire bandpass table selecting the editable antenna and time in
sequence.  If some of the solutions show residual effects from RFI,
then you may generate flags in a flag table to delete certain spectral
channels for certain antennas and times.  If you do this, of course,
you should re-run {\tt BPASS} applying the new flags.

     The {\tt BP} tables are applied to the data by setting the adverb
{\tt DOBAND} $> 0$ and selecting the relevant {\tt BP} table with the
adverb \hbox{{\tt BPVER}}.  There are three modes of \Indx{bandpass}
application.  The first ({\us DOBAND 1}) will average all bandpasses
for each antenna within the time range requested, generating a global
solution for each antenna.  The second mode ({\us DOBAND 2}) will use
the antenna bandpasses nearest in time to the data point being
calibrated.  The third mode ({\us DOBAND 3}) interpolates in time
between the antenna bandpasses and generates the correction from the
interpolated data.  This mode has been found to be required for VLA
data.  If {\tt BPSMO} was used to make a fairly finely sampled {\tt
BP} table, then {\tt \Tndx{DOBAND} 2} may be used.  Modes {\us DOBAND
4} and {\us DOBAND 5} are the same as modes 2 and 3, respectively,
except that data weights are ignored.\Iodx{calibration}

     It is often not possible to observe a strong bandpass calibrator
many times during a run.  In this case, one can run {\tt \tndx{BPASS}}
on the single scan on the strong calibrator and then remove the main
bandpass shape with {\tt DOBAND 1} in task {\tt SPLAT}\@.  Corrections
to this basic bandpass shape as a function of time may then be
determined with adequate signal-to-noise using task {\tt
\tndx{CPASS}}\@.  This task can be used to fit the residual bandpass
with a small number of parameters ($<<$ the number of spectral
channels) at each calibrator scan.  The results may then be applied
with {\tt DOBAND 2}\@.  Check the output of {\tt CPASS} carefully ---
it is capable of making bandpass shapes with large ripples that are
not present in the data.  {\tt CPASS} also knows about the flux
coefficients of all standard sources and can fit the spectral index of
other sources from the flux values in the {\tt SU} table.

\subsections{Amplitude and phase calibration}

The channel-0 data set should be calibrated as described above for
continuum data (\Sec{caledit} and \Sec{calgain}).  When you are
satisfied with your results, you should copy the relevant {\tt CL}
table over to the line data set with {\tt \tndx{TACOP}}:
\iodx{spectral-line}
\dispt{TASK\qs 'TACOP' \CR}{}
\dispt{INDI\qs {\it n\/} ; GETN\qs {\it m\/} \CR}{to specify the
              channel-0 data set.}
\dispt{OUTDI\qs {\it i\/} ; GETO\qs {\it j\/} \CR}{to specify the line
              data set.}
\dispt{INEXT\qs 'CL' \CR}{to copy a {\tt CL} table.}
\dispt{INVER\qs 0 \CR}{to copy highest numbered table from {\tt CLCAL} step.}
\dispt{NCOUNT\qs 1 \CR}{to copy only one table.}
\dispt{OUTVER\qs 0 \CR}{to create new output table.}
\dispt{INP \CR}{to review the inputs.}
\dispt{GO\qs \CR}{to run the program when inputs set correctly.}
\dispe{If you copy {\tt SN}, {\tt TY}, or {\tt SY} tables, you may
apply a flagging table to the table values.}

     At this point it is often useful to examine your fully calibrated
data using {\tt \tndx{POSSM}}:
\dispt{TASK\qs 'POSSM' \CR}{}
\dispt{INDI\qs {\it i\/} ; GETN\qs {\it j\/} \CR}{specify line data.}
\dispt{SOURCES\qs '{\it source1\/}' , '\ ' \CR}{to specify the source
            of interest.}
\dispt{ANTENNAS\qs 0 \CR}{to plot all antennas.}
\dispt{BCHAN\qs 10 ; ECHAN\qs 55 \CR}{to plot spectrum for this
            channel range only.}
\dispt{DOCALIB\qs 1 \CR}{to apply the antenna gain to both
            visibilities and weights (if appropriate).
            \Indx{calibration}.}
\dispt{GAINUSE\qs 0 \CR}{to use most recent {\tt CL} table.}
\dispt{DOBAND\qs 3 \CR}{to apply the bandpass calibration time smoothed.}
\dispt{BPVER\qs 1 \CR}{to use {\tt BP} table 1.}
\dispt{FREQID\qs 1 \CR}{to use only one {\tt FQ} value.}
\dispt{APARM\qs 0 \CR}{to do vector averaging of amplitudes and
             self-scale the plots.}
\dispt{SMOOTH\qs 5 , 0 \CR}{to apply Hanning smoothing in the spectral
             domain after bandpass calibration is applied.  Use 13,0
             to smooth after the data are averaged, which is faster
             and less prone to oddities due to channel-dependent
             flagging.  Use 1,0 only if the data were Hanning smoothed
             when {\tt BPASS} was run.}
\dispt{INP \CR}{to review the inputs.}
\dispt{GO\qs \CR}{to run the program when inputs set correctly.}
\dispt{GO \tndx{LWPLA} \CR}{to send the plot to the (PostScript)
             printer/plotter.}

     If you have multiple {\tt FQ} entries in your data set, you
should repeat the calibration for each additional {\tt FQ} entry.
Bookkeeping is simplified if you eliminate all extant {\tt SN} tables
before calibrating the data associated with each frequency identifier.
However, it is not essential to do this.

\vfill\eject
\Sects{Solar data calibration}{suncal}

     The calibration of solar \uv\ data differs from normal continuum
and spectral-line calibration in one critical respect: the system
temperature correction to the visibility data is applied by the
observer in \hbox{\AIPS}.  See Lecture 21 in {\it \jndx{Synthesis
Imaging in Radio Astronomy}\/} for a discussion of the system
temperature correction as it applies to VLA solar visibility data.
The system temperature correction is embodied in a quantity referred
to as the ``nominal sensitivity,'' an antenna-based numerical factor
normally applied in real time to the scaled correlation coefficients
before they are written to the VLA archive.  With the exception
of X and L band, only a handful of VLA antennas are equipped with
so-called ``solar CALs.''  The nominal sensitivity is only computed
for those antennas so-equipped, namely (for the old VLA) antennas 5,
11, 12, and 18 (at K, U,and C bands) and antennas 7, 12, 21, and 27
(at P band).  The system-temperature correction for those antennas
without solar CALs must, therefore, be bootstrapped from those
antennas which do have solar CALS\@.  This is accomplished through two
tasks for the old VLA and two other tasks for the new vLA\@..  Fpr the
old VLA, {\tt FILLM} fills the uncalibrated visibility data to disk
and places the nominal sensitivities in a {\tt TY} extension table.
Then, {\tt \tndx{SOLCL}} applies the nominal sensitivities to
calibration parameters in the {\tt CL} table.  For the new VLA, {\tt
\tndx{BDF2AIPS}} reads the data in, writing a SysPower ({\tt SY}) and
CalDevice ({\tt CD}) table.  Then {\tt SYSOL} applies the gain and
weight corrections.  See \Sec{bdf2aips} and \Sec{SysPower}.
\Iodx{Solar data}\Iodx{calibration}

\Subsections{Reading solar data from the VLA archive}{sunread}

     New VLA data are loaded from the archive following the
instructions in \Sec{bdf2aips} with no special considerations for
Solar observing except that the versions of OBIT and \AIPS\ must be
after mid-June 2017.

  To load a solar \uv-data file to disk from an
old VLA archive data set follow the general instructions given above
(\Sec{fillm} and \Sec{lineread}) with the following additions:
\dispt{VLAMODE\qs 'S ' \CR}{to indicate solar mode observing.}
\dispt{CPARM(2)\qs 16 \CR}{to indicate that moving sources are allowed
   without renaming.}

If your experiment involved observing active solar phenomena,
(\eg\ flares), you may wish to update the system-temperature
correction every integration time.  For example, if you observed a
flare with an integration time $\tau = 1.67$ seconds, choose
\dispt{CPARM(8)\qs 1.67 / 60 \CR}{for 1.67 sec {\tt CL} and {\tt TY}
   table intervals.}

     Loading an entire solar \uv-data set to disk with the minimum
integration time results in very large disk files which make all
subsequent programs take a long time to run.  A useful strategy is to
load the data with relatively low time resolution (20--30 seconds for
observations of active solar phenomena) and to proceed with the usual
continuum data calibration, deferring the system temperature
correction.  When a satisfactory calibration is obtained, the relevant
{\tt SN} table may be saved using \hbox{{\tt \tndx{TASAV}}}.  (Note
that you must save the {\tt SN} table, before running {\tt
\tndx{CLCAL}} rather than the final {\tt CL} table.)  Then run {\tt
CLCAL} and inspect the data for interesting periods of activity ---
try {\tt \tndx{UVPLT}} with {\us BPARM = 11, 1} for plots of amplitude
versus time or {\tt \tndx{TVFLG}}, displaying amplitudes as a function
of baseline length and time.  Use {\tt \tndx{FILLM}} to load the
relevant time ranges of solar \uv\ data to disk with no averaging.
The saved {\tt SN} table is then copied to each high-time resolution
data set.  Assess, and possibly edit, the nominal sensitivities
(\Sec{sunsens}) and then apply the system-temperature corrections
(\Sec{suntsys}).  Finally, apply the saved/copied {\tt SN} table to
the {\tt CL} table of each using \hbox{{\tt CLCAL}}.

\Subsections{Using {\tt SNPLT} and {\tt LISTR} to assess the nominal
sensitivities}{sunsens}

     For new VLA data sets, use {\tt SNPLT}, {\tt LISTR}, and {\tt
EDITA} or {\tt SNEDT} on the {\tt SY} table (\Sec{SysPower}).  The
following applies to old VLA data primarily.

     When solar \uv\ data are written to disk, {\tt FILLM} writes the
nominal sensitivities of those antennas equipped with solar CALs into
the {\tt TY} table.  Before bootstrapping the system temperature
correction for antennas without solar CALs from those which do, it is
always wise to examine the nominal sensitivity for each of the solar
CAL antennas for each of the IFs.  The tools available for this
purpose include: {\tt \tndx{SNPLT}}, which plots the nominal
sensitivities in graphical form, {\tt \tndx{LISTR}} or {\tt PRTAB},
which allow one to inspect the values directly, and {\tt EDITA}, which
provides an interactive display of the {\tt TY} data and allows you to
edit the data.  To make plots:
\dispt{TASK\qs 'SNPLT' ; INP \CR}{to review the inputs needed.}
\dispt{IND\qs {\it m\/} ; GETN {\it n\/} \CR}{to specify the input
             \uv\ file.}
\dispt{INEXT\qs 'TY' \CR}{to plot data from {\tt TY} extension table.}
\dispt{INVERS\qs 0 \CR}{to use the highest version number.}
\dispt{SOURCES\qs 'SUN' , ' ' \CR}{to plot solar source only.}
\dispt{TIMERANG\qs 0 \CR}{to select all times.}
\dispt{ANTENNAS\qs 5 11 12 18 \CR}{to select only CAL-equipped
              antennas; this sample list for K, U, or C band.}
\dispt{PIXRANGE\qs 0 \CR}{to self-scale each plot.}
\dispt{NPLOTS\qs 4 \CR}{to do 4 plots on a page.}
\dispt{FACTOR\qs 2 ; SYMBOL\qs 5 \CR}{to use triangles to mark the
              data and enlarge them by a factor of 2.  The symbols may
              even be connected by lines.}
\dispt{XINC\qs 1 \CR}{to plot every {\tt XINC}$^{\uth}$ point.}
\dispt{OPTYPE\qs 'TSYS' \CR}{to plot nominal sensitivities.}
\dispt{INP \CR}{to review the inputs.}
\dispt{GO \CR}{to run the program when you're satisfied with inputs.}
\dispe{{\tt \tndx{SNPLT}} produces a {\tt PL} extension file which may
be plotted using {\tt LWPLA}, {\tt TKPL}, or \hbox{{\tt TVPL}} --- or
you could set {\us DOTV TRUE} in {\tt SNPLT} and get the display
directly (and temporarily) on the \hbox{TV}.  Then to inspect the
values over some limited time range in detail, run {\tt \tndx{LISTR}}
(assuming the adverbs set above and):\Iodx{Solar data}\Iodx{calibration}}
\dispt{TASK\qs 'LISTR' ; INP \CR}{to review the inputs needed.}
\dispt{OPTYPE\qs 'GAIN' \CR}{to list quantities in a calibration
              file.}
\dispt{INEXT\qs 'TY' \CR}{to select the sensitivities.}
\dispt{TIMER\qs {\it d1 h1 m1 s1 d2 h2 m2 s2} \CR}{to select by
              suspect time range.}
\dispt{DOCRT\qs -1 \CR}{to route output to the printer.}
\dispt{DPARM\qs 10 0 \CR}{to list nominal sensitivities.}
\dispt{INP \CR}{to review the inputs.}
\dispt{GO \CR}{to run the program when you're satisfied with inputs.}
\dispe{Task {\tt SNIFS} is similar to {\tt SNPLT} except that it
plots IF on the $x$ axis to compare solutions across them.  It has
numerous binning options to control the otherwise excessive plotting.}

     The use of {\tt \tndx{EDITA}} with {\tt TY} tables is described
extensively in \Sec{edita} and need not be described further here.

\Subsections{Using {\tt SOLCL} or {\tt SYSOL} to apply the
system-temperature correction}{suntsys}
\Iodx{Solar data}\Iodx{calibration}

     For the new VLA, task {\tt SYSOL} is provided with both non-Solar
and Solar Tcals in the {\tt CD} table and with measured switched
power values in the {\tt SY} table.  The latter also tells \AIPS\
tasks which Tcals apply to which times.  Therefore, the inputs to {\tt
SYSOL} are simpler than those for the old VLA\@.

     For the old VLA, once you have identified the appropriate subset
of reference solar CAL antennas for each source and IF, you are ready
to bootstrap the system-temperature correction of the remaining
antennas.  It is recommended that you run {\tt \tndx{SOLCL}} before
applying any other calibration to the {\tt CL} table.  In this way,
you can easily verify that the appropriate corrections have been made
to each antenna.  Then you apply the system-temperature correction to
version 2 and correct mistakes by deleting and recreating version 2.
To run {\tt SOLCL}:
\dispt{TASK\qs 'SOLCL' ; INP \CR}{to review the inputs needed.}
\dispt{SOURCES\qs '*' \CR}{to correct all sources.}
\dispt{STOKES\qs '\ ' \CR}{to correct both polarizations.}
\dispt{TIMERANG\qs 0 \CR}{to correct all times.}
\dispt{ANTENNAS\qs 5 11 12 18 \CR}{to use the listed antennas as
             references.}
\dispt{SUBARRAY\qs 1 \CR}{to modify sub-array 1.}
\dispt{GAINVER\qs 2 \CR}{to write corrected entries to {\tt CL} table
             version 2.}
\dispt{INP \CR}{to review the inputs.}
\dispt{GO \CR}{to run the program when you're satisfied with inputs.}

     After applying the system temperature correction, you may proceed
with the usual \AIPS\ data calibration procedures outlined in previous
sections, including the special solar tactics described in
\Sec{sunread}.

\sects{Completing the initial calibration}

     When you are satisfied with the initial calibration (pre
self-calibration) of your data set, you should back up your full
multi-source data set on magnetic tape.  Then you can apply the
calibration to the data for each program source, creating a separate
single-source \uv\ data set for each.  These data sets are used with
the imaging and self-calibration tasks to be described in the
following chapters.  For the impatient, the recommended imaging task
reads the multi-source data set directly, applying any calibration,

\subsections{Using {\tt FITTP} and {\tt FITAB} to write multi-source
data to tape}

     The recommended way out of \AIPS\ for multi-source \uv\ data is
to use {\tt \tndx{FITTP}} to write a FITS-format tape.  This will
preserve the data and all associated calibration and editing tables in
a machine-independent form.  {\tt \Tndx{FITAB}} also writes a
FITS-format tape or disk file using tables rather than random groups.
This has the advantages of allowing a compressed format and of allowing
\uv\ files to be broken into ``pieces'' for increased reliability and
control of space.  {\tt FITAB} output from earlier than 15-Oct-2007
can be read by versions of \AIPS\ between {\tt 15APR99} and 15-Oct-2007,
after 15-Oct-2007 it can only be read by versions of {\tt UVLOD} or
{\tt FITLD} later than 15-Oct-2007.  Also note that {\tt FITAB} output
cannot be read by other \uv-data software packages, except {\it obit}.
{\tt FITTP} output can be read by some other packages.  Consult
\Sec{magtape} about magnetic tapes in \hbox{\AIPS}.  That section
tells you to mount your tape on the hardware device and then to do a
software mount in \hbox{{\tt AIPS}}.  For example,
\dispt{INTAP\qs {\it n\/} \CR}{to specify which tape drive to use.}
\dispt{DENSITY\qs 6250 \CR}{to set the density to 6250-bpi, if needed.}
\dispt{MOUNT \CR} {to mount the tape in software.}
\dispe{This step used to be optional for some operating systems.
However, in recent versions of \AIPS, it is required on all operating
systems.}

      To write the data to tape:
\dispt{TASK\qs 'FITTP' \CR}{}
\dispt{IND {\it m\/} ; GETN\qs {\it n\/} \CR}{to specify the
          multi-source data set.}
\dispt{DOEOT\qs TRUE \CR}{to write at the end of tape --- if there are
          other data files on the tape you wish to preserve.}
\dispt{OUTTA\qs INTAP \CR}{to write to tape just mounted.}
\dispt{DOTABLE\qs TRUE \CR}{to write associated tables.}
\dispt{FORMAT\qs 3 \CR}{to use IEEE floating format for data.}
\dispt{BLOCKING\qs 10 \CR}{to use blocked FITS for tape efficiency.}
\dispt{INP \CR}{to review the inputs.}
\dispt{GO\qs \CR}{to run the program when inputs set correctly.}

     Most people use 8mm Exabyte or 4mm DAT tapes today.  These have
very large capacities.  However, if you must still use half-inch reel
tapes, you will find that many data sets (particularly spectral line)
may be too large to fit on one 6250 bpi tape even with {\tt BLOCKING =
10}.  Since it is not possible to write multi-volume FITS tapes, it is
recommended that you back up the single-source data sets formed after
applying the calibration tables in {\tt SPLIT} (see \Sec{split}).
{\tt FITAB} allows you to break up the data set into pieces which can
fit on your tape.  Multiple executions will be needed for multiple
tapes.  Alternatively, since all of the calibration information is
contained in the extension tables, you may copy these to a dummy \uv\
file with task {\tt TASAV} and write this new file to tape with
\hbox{{\tt FITTP}}.

     Be sure to run task {\tt \tndx{PRTTP}} to make sure that the data
were written successfully on your tape {\it before\/} you delete your
multi-source \uv\ data set!

\Subsections{Creating single-source data files with {\tt SPLIT}}{split}

      When you are happy with the calibration and editing represented
by the current set of calibration and flag tables, you can convert the
multi-source file into single-source files, applying your calibration
and editing tables.  Remember that only one {\tt FREQID} can be {\tt
\tndx{SPLIT}} at a time.
\dispt{TASK\qs 'SPLIT' \CR}{}
\dispt{SOURCE\qs '{\it sou1\/}' , '{\it sou2\/}' , $\ldots$ \CR}{to
             select sources, '\ ' means all.}
\dispt{TIMERANG\qs 0 \CR}{to keep all times.}
\dispt{BIF\qs 1 ; EIF\qs 2 \CR}{to keep both IFs}
\dispt{FREQID\qs 1 \CR}{to set the one {\tt FQ} value to use.}
\dispt{DOCALIB\qs 1 \CR}{to apply calibration to the data and the
             weights.}
\dispt{GAINUSE\qs 0 \CR}{to use the highest numbered {\tt CL} table.}
\dispt{DOPOL\qs TRUE \CR}{to correct for feed polarization.}
\dispt{DOBAND\qs 3 \CR}{to correct bandpass with time smoothing.}
\dispt{BPVER\qs 1 \CR}{to select {\tt BP} table to apply.}
\dispt{STOKES\qs ' ' \CR}{to write the input Stokes type.}
\dispt{DOUVCOMP\qs FALSE \CR}{to write visibilities in uncompressed
            format.}
\dispt{APARM\qs 0 \CR}{to avoid channel averaging and autocorrelation
            data.}
\dispt{INP \CR}{to review the inputs.}
\dispt{GO\qs \CR}{to run the program when inputs set correctly.}
\dispe{The files produced by this process should be completely
calibrated and edited and ready to be imaged or further processed as
described in later chapters.  Note that one may wish to defer the {\tt
DOPOL 1} part of this until after self-calibration of the
parallel-hand visibilities.}

     It is not necessary to run {\tt SPLIT} to make images with
{\tt \tndx{IMAGR}} and it is probably a good idea to make a couple of
quick images to make sure that the calibration is okay.  However, for
serious imaging, it is probably best to run {\tt SPLIT} and then use
the single-source output files.  See \Sec{imagr} for details of the
imaging process.

\subsections{Making images from multi-source data with {\tt IMAGR}}

     {\tt IMAGR} can be used to make images from multi-source data
files.  It is probably a good idea to make a couple of quick images to
make sure that the calibration is okay.  An example set of inputs to
{\tt \tndx{IMAGR}} is:
\dispt{TASK\qs 'IMAGR' ; DEFAULT \CR}{to select task and initialize
            all its parameters.  This selects the usual convolution
            and weighting functions among other things.}
\dispt{IND {\it m\/} ; GETN\qs {\it n\/} \CR}{to specify the
            multi-source data set.}
\dispt{SOURCE\qs '{\it sou1\/}' , ' ' \CR}{to choose one source to
            image.}
\dispt{STOKES\qs 'I' ; TIMERANG\qs 0 \CR}{to image total intensity
            from all times.}
\dispt{FREQID\qs 1 \CR}{to select {\tt FQ} value to image.}
\dispt{BIF\qs 1 ; EIF 0 \CR}{to image all IFs --- multi-channel mode
            images only one \hbox{IF}.}
\dispt{BCHAN\qs {\it n\/} ; ECHAN\qs {\it m\/} \CR}{to combine a range
            of channels.}
\dispt{NCHAV\qs {\it N\/} \CR}{to include {\it N\/} spectral channels
            in each image where $N \leq (m - n + 1)$; for each
            spectral channel, IFs {\it bif\/} through {\it eif\/} are
            also included.  Note that each channel and IF included in
            the ``average'' image is handled individually at its
            correct frequency.}
\dispt{DOCALIB\qs 1 \CR}{to apply calibration.  Use {\us DOCAL 100
            \CR} if the weights should {\it not} be calibrated.}
\dispt{GAINUSE\qs 0 \CR}{to use highest numbered {\tt CL} table.}
\dispt{FLAGVER\qs 1 \CR}{to edit data.}
\dispt{DOPOL\qs TRUE \CR}{to correct for feed polarization.}
\dispt{DOBAND\qs 3 \CR}{to correct bandpass with time smoothing.}
\dispt{BPVER\qs 1 \CR}{to select {\tt BP} table to apply.}
\dispt{OUTNAME\qs '{\it sou1\/}' \CR}{to set the output file name to
             the source name.}
\dispt{OUTDISK\qs 0 \CR}{to use any output disk with enough space.}
\dispt{IMSIZE\qs 512 512  \CR}{to set the size in cells of image.}
\dispt{CELLSIZE\qs 0.25 , 0.25 \CR}{to set the size of each image cell
             in arc-seconds.}
\dispt{RASHIFT\qs 0 ; DECSHIFT\qs 0 \CR}{to (not) shift image center.}
\dispt{NFIELD\qs 1 ; NGAUSS\qs 0\CR}{to make only one image at high
             resolution.}
\dispt{UVWTFN\qs ' ' \CR}{to use uniform weighting.}
\dispt{ZEROSP\qs 0 \CR}{to introduce no zero-spacing flux.}
\dispt{NITER\qs 0 \CR}{to do no Cleaning.}
\dispt{INP \CR}{to review the inputs.}
\dispt{GO\qs \CR}{to run {\tt \tndx{IMAGR}} when the inputs are set
             correctly.}

%\sects{Additional recipes}
