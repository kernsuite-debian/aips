%-----------------------------------------------------------------------
%;  Copyright (C) 1995, 1998, 2004, 2013-2014
%;  Associated Universities, Inc. Washington DC, USA.
%;
%;  This program is free software; you can redistribute it and/or
%;  modify it under the terms of the GNU General Public License as
%;  published by the Free Software Foundation; either version 2 of
%;  the License, or (at your option) any later version.
%;
%;  This program is distributed in the hope that it will be useful,
%;  but WITHOUT ANY WARRANTY; without even the implied warranty of
%;  MERCHANTABILITY or FITNESS FOR A PARTICULAR PURPOSE.  See the
%;  GNU General Public License for more details.
%;
%;  You should have received a copy of the GNU General Public
%;  License along with this program; if not, write to the Free
%;  Software Foundation, Inc., 675 Massachusetts Ave, Cambridge,
%;  MA 02139, USA.
%;
%;  Correspondence concerning AIPS should be addressed as follows:
%;          Internet email: aipsmail@nrao.edu.
%;          Postal address: AIPS Project Office
%;                          National Radio Astronomy Observatory
%;                          520 Edgemont Road
%;                          Charlottesville, VA 22903-2475 USA
%-----------------------------------------------------------------------

\setcounter{chapter}{6}
\Appen{Glossary}{glos}
\renewcommand{\Chapt}{40}

\renewcommand{\titlea}{15-APR-99 (revised 18-November-1998)}
\renewcommand{\Rheading}{\AIPS\ \cookbook:~\titlea\hfill}
\renewcommand{\Lheading}{\hfill \AIPS\ \cookbook:~\titlea}
\markboth{\Lheading}{\Rheading}
\setcounter{subsection}{0}
\setcounter{section}{1}


% was 8 point

\def\IIS{\hbox{I\/\raise1.0ex\hbox{$\scriptstyle{2}$}S}}
\let\iis=\IIS
\def\text#1{\hbox{\rm#1}}
%\def\arg{\text{arg}}

\def\offspace@AmS{\nulldelimiterspace0pt \mathsurround0pt }
\def\leftset#1\mid#2\rightset{\hbox{$\displaystyle
\left\{\,#1\vphantom{#1#2}\;\right|\;\left.
    #2\vphantom{#1#2}\,\right\}\offspace@AmS$}}

\def\charfn#1{\raise.465ex\hbox{$\chi$}_{#1}}
\def\formhat{\hat {\phantom{l}}}
\def\formcheck{\check {\phantom{l}}}
\def\bold#1{{\bf#1}}
\def\parsep{\par\penalty -50\vskip 3pt plus 1.5pt minus 1pt}
\def\aa#1{\parsep{{\small\bf{#1} --- }}\stepcounter{section}\label{\Myfolio}\iodx{#1}}
\def\xspace{\unskip\hskip 7pt plus 3pt minus 4pt}
\def\uv{$u$-$v$ }
\def\qv{({\it q.v.})}
\def\s{\tilde}
%\tolerance=5000
\pretolerance 5000


\newcommand{\intii}{\int\!\!\int}
\def\operatorname#1{\mathop{\mathcode`'="7027 \mathcode`-="70
       \rm #1}\nolimits}                                              % NOTE 31
\def\arg{\operatorname{arg}}
\def\fti{\text{FT}^{-1}}
\def\hca{{\it H\"ogbom Clean algorithm}}
\def\wsp{{\it 1982 Summer Workshop Proceedings}}
\def\ssp{{\it 1985 Summer School Proceedings}}
\def\sira{{\it Third NRAO Synthesis Imaging Summer School}}
\def\sydp{{\it 1983 Sydney Conference Proceedings}}
\def\gronp{{\it 1978 Groningen Conference Proceedings}}
\def\CRn{{\footnotesize \hbox{\raise0.4ex\hbox{C}\hskip-.17em\lower0.4ex\hbox
      {R}}}}
\def\dotsc{\mathinner{\ldotp\ldotp\ldotp}}
\def\hatsymbol{{\mathchoice{\null}{\null}{\,\,\hbox{\lower 10pt\hbox
    {$\widehat{\null}$}}}{\,\hbox{\lower 20pt\hbox
       {$\hat{\null}$}}}}}

%\small
\footnotesize
\renewcommand{\parsdef}{0.5mm plus 0.5mm minus 0.5mm}
\pd
\parindent 13pt

\aa{adverb} See {\it POPS symbols.}

\aa{AIPS monitor}
a computer terminal (perhaps lacking a keyboard) whose CRT
screen is used in AIPS solely for the display of information
related to the progress of the execution of the AIPS tasks.
(Except, at those AIPS sites without a terminal dedicated to this use,
the AIPS user's interactive terminal is used for dual purposes---%
i.e., to serve as the AIPS monitor as well.)
Many of the messages which the AIPS tasks write to the monitor
also are recorded in the {\it message file} \qv.

\aa{aliased response}
in a radio interferometer map, a spurious feature due to
a source---or to a sidelobe---that lies outside of the field of view.
Consider the sampling of a visibility function $V$
at the lattice points of a rectangular grid
as multiplication of $V$ by the comb-like distribution
$R(u,v)=\sum_k\sum_l\delta(u-k\Delta u,v-l\Delta v)$.
The Fourier transform $\widehat {RV}$ of $RV$ is given
by the convolution $\hat R\ast\hat V$.
Since $\hat R$ is again a comb-like distribution, with
peaks, or teeth, separated by ${1\over\Delta u}$ in one direction
and by ${1\over\Delta v}$ in the perpendicular direction,
$\widehat {RV}$ is periodic,
and, about the position of each tooth in the comb,
it looks like an infinite summation of rectangular pieces of $\hat V$,
each of size ${1\over\Delta u}\times{1\over\Delta v}$,
taken from all over the plane.
Aliased responses can be suppressed very effectively, by judicious
choice of the {\it gridding convolution function} \qv.
\par
For a more complete discussion, see Dick Sramek and Fred Schwab's Lecture
No.~6 in the \sira.
Also see VLA Scientific Memoranda Nos.~129 and 131.

\aa{aliasing}
in spectral analysis, error which is due to undersampling:
one may wish to sample a signal that is known to be bandlimited,
but whose bandwidth may not be known a priori.
The Fourier transform of {\it Shannon's series} is periodic;
aliasing error is of the form of an overlapping, or superposition,
of these ``replicated'' spectra.
See {\it Nyquist sampling rate} and {\it aliased response}.

\aa{ALU}
(Arithmetic Logic Unit) an (optional) micro-computer CPU unit
within the \iis\ TV display device which allows simple
arithmetic operations, such as sums, products, and convolutions,
to be performed on the data recorded in the \iis\ image planes.
At present, AIPS makes little use of the ALU, since
many of its features are unique to the \iis\ display unit.
See {\it \iis}.

\aa{antenna file}
in AIPS, an {\it extension file}, associated with a \uv {\it data file},
in which a list of the interferometer antenna positions is stored.

\aa{antenna/i.f.\ gain}
Many of the systematic errors affecting radio interferometer measurements
are multiplicative in the visibility amplitude and
additive in the visibility phase,
and are ascribable to individual antenna elements and their
associated i.f./l.o.\ chains.
For each antenna/i.f.\
these sources of error may be lumped together into a
complex-valued function of time, $g(t)$, called the
{\it antenna/i.f.\ gain}.
Then, the visibility measurement obtained on the $i$--$j$
baseline at time $t$ is given by
$\s V(u_{ij}(t),v_{ij}(t))=g_i(t)\overline g_j(t)V(u_{ij}(t),v_{ij}(t))
+\epsilon_{ij}(t)\,,$
where $V$ is the true source visibility and where the spatial
frequency coordinates $(u,v)$ have been parametrized by time.
$g_i\overline g_j$ is the systematic ``calibration error'',
and $\epsilon_{ij}$, an additive error component, is
assumed to be random and well-behaved.
(Another type of systematic error, the {\it instrumental
polarization} \qv, is not included in the $g_k$,
and always must be corrected, by proper calibration,
in order to interpret polarization data.)
\par
Some of the most serious sources of error---%
including atmospheric attenuation, error arising from variations
in the atmospheric path length,
clock error, and error in the baseline
determination---conform fairly well to this multiplicative model.
This model relation is exploited heavily
by the {\it self-calibration algorithm} \qv.
Compare {\it antenna/i.f.\ phase},
and see {\it isoplanaticity assumption} and {\it correlator offset}.

\aa{antenna/i.f.\ phase}
The {\it antenna/i.f.\ phase} for antenna $k$ of an
interferometer array is given by the argument (or phase)
of the {\it antenna/i.f. gain} $g_k$: $\psi_k(t)=\arg g_k(t)$.
Often in {\it self-calibration} one assumes that no amplitude
errors are present and solves only for the $\psi_k$.

\aa{antenna residual delay}
See {\it residual delay} and {\it global fringe fitting algorithm}.

\aa{antenna residual fringe rate}
See {\it residual fringe rate} and {\it global fringe fitting algorithm}.

\aa{AP} See {\it array processor}.

\aa{AP--120B array processor}
an {\it array processor} manufactured by Floating Point Systems, Inc.,
and used at a number of AIPS sites.
Its floating-point word length is 38-bits.
Typically it is equipped with a main data memory of 32--64
kilowords and a program source memory of 2048 words.
With both a pipeline multiplier and a pipeline adder,
and a memory cycle time of 167\,ns.,
when programmed at top efficiency it can perform
at an arithmetic rate of 12 million floating-point operations per second.
\par
The AP--120B is no longer in production; this product has been superseded
by the 5000 series product line.
Though the AP--5000's are used at some AIPS sites, their advanced
features are not used by AIPS---only those features which are shared by the
older model are fully exploited by AIPS tasks.

\aa{array processor}
a computer peripheral attachment which is capable of performing
certain floating-point computations, especially vector and matrix
operations, at high speed, and independently of the {\it host computer}
central processing unit.
Usually the high-speed performance is achieved by a technique known
as pipelining.
The basic arithmetic operations of addition and multiplication
are performed in stages, by a so-called pipeline adder and a pipeline
multiplier.
These units operate just like an assembly line in a manufacturing plant.
Some array processors (AP's) are constructed with multiple pipelines.
Address computations are performed concurrently with the arithmetic
operations, by a unit which is separate from the pipelines.
The algorithms best-suited to an array processor
implementation are those which can be structured so as to keep
the pipelines filled a fair fraction of the time.
Most AP's have their own high-speed data memory,
but some are parasitic on the memory of the host computer.
Portions of many AIPS tasks have been programmed for
the Floating Systems, Inc. model {\it AP-120B array processor}, \qv.
Also see {\it array processor microcode}, {\it Q-routine},
and {\it pseudo-array processor}.

\aa{array processor microcode}
program source code written in the assembly language of an
{\it array processor}, \qv.
Array processor (AP) manufacturers usually provide an extensive
library of utility subroutines that may be called from a high-level
programming language, such as Fortran;
however, some computationally-intensive algorithms cannot be
easily or efficiently implemented using only these libraries.
Portions of these algorithms must be written in microcode---%
a painstaking process.
The assembly languages of different models of AP's differ considerably
(as do the subroutine libraries, too, in fact)
because of differences in the hardware architectures.
Thus the AIPS programming group tries to avoid writing microcode.
But portions of the AIPS {\it tasks} for mapmaking, deconvolution,
and self-calibration are written in AP microcode.
Also see {\it Q-routine}.

\aa{associated file}
In AIPS, any two or more files among a collection
consisting of a {\it primary data file} and all of
its {\it extension files} are termed {\it associated}.

\aa{auto re-boot}
a boot initiated by the computer itself, of its own volition.
See {\it boot.}

\aa{back-up}
The act of copying the contents of a computer file
to some permanent storage medium such as magnetic tape or punched cards,
for the purpose of protecting against accidental loss
or in order to liberate storage space (e.g., disk space),
is termed {\it backing-up.}
The new copy of the file is termed a {\it back-up} copy,
or simply a {\it back-up.}
See {\it scratch.}

\aa{bandwidth smearing}
in a radio interferometer map, space-variance of the {\it point spread
function} which is attributable to non-monochromaticity, or finite bandwidth.
The point spread function---at a particular point in a map---%
taking into account bandwidth smearing, but
ignoring other instrumental effects, is termed a {\it delay beam}.
Bandwidth smearing is a radial effect: the delay beams become more
elongated, in the radial direction from the interferometer {\it phase tracking
center}, as their distance from the phase tracking center increases.
The delay beams are easily calculable when all of the receivers
in an array have identical, and known, i.f. passbands.
E.g., with rectangular passbands of width $\Delta\nu$, and observations
centered at a frequency $\nu_0$,
the measured visibility amplitude of a point source
is proportional to $\sin\gamma\over\gamma$ ---%
where $\gamma\equiv\pi(ux+vy+wz){\Delta\nu\over\nu_0}$,
$(u,v,w)$ denotes the spatial frequency coordinates, measured in wavelengths
at $\nu_0$, and $(x,y,z)$ denotes the direction cosines of the location of the
point source, with respect to the phase tracking center.
For more details, see Alan Bridle and Fred Schwab's Lecture No.~13
and Bill Cotton's Lecture No.~12 in the \sira\ and
see VLA Scientific Memo.\ No.~137.
\par
Bandwidth smearing can, in principle, be eliminated
(assuming that the bandpasses are known) by applying an
image reconstruction algorithm which has a knowledge of the smearing
mechanism; that is, by an algorithm which is more general than the usual
deconvolution algorithms---see {\it image reconstruction}.
The most common method for reducing bandwidth smearing
is the technique of {\it bandwidth synthesis}, \qv.

\aa{bandwidth synthesis}
a technique of radio interferometry which is intended to diminish
the effect of {\it bandwidth smearing}.
Bandwidth synthesis observing is very similar to spectral-line mode
observing: the i.f. bandpasses are split up into a number of pieces,
or channels, and the data in each channel are treated separately
up until the mapping\slash deconvolution stage of processing.
At that stage, the problem can be formulated as a system of simultaneous
convolution equations: one has the system
$g_1=b_1\ast f+\epsilon_1,\dots,g_n=b_n\ast f+\epsilon_n$,
where $n$ is the number of frequency channels,
$g_k$ is the {\it dirty map} for channel $k$,
$b_k$ the {\it dirty beam} for that channel,
$f$ the unknown radio source brightness distribution (here assuming
that $f$ is not a function of frequency),
and $\epsilon_k$ is noise (were it not for the noise, and for the fact
that each deconvolution problem is ill-posed---in its own right---,
there would be no reason to treat the equations simultaneously,
or even to consider more than a single one of them).
(For a description of a refinement to the bandwidth synthesis technique,
for sources with spatially-varying spectral indices,
see {\it broadband mapping technique}.)
Note that all the $b_k$ are identical, apart from a
dilation factor; i.e., as the {\it u-v coverage} ``shrinks'', toward
the low end of the observing band, the $b_k$ dilate by the
reciprocal of the {\it u-v} shrinkage factor.
\par
The present state of software development does not allow solving the
problem in quite the way it is formulated above.
Rather, some mapping\slash deconvolution algorithm
is applied separately to each
of the channels, and the resulting maps are averaged.

\aa{baseline-time order}
An ordered set of visibility measurements
$\leftset V_{ij}(t_k)\mid 1\le i<j\le n,{\rm\ }k=1,\dots,l \rightset$
re\-cord\-ed with an $n$ element interferometer at times $t_k$
is said to be in {\it baseline-time order}
if the ordering is such that
all of the data for the 1--2 baseline, sorted by time,
occur first, followed by the data for the 1--3 baseline, again
sorted by time, etc., etc.
(This canonical ordering by baseline is the order
$V_{12},V_{13},\dots,V_{1n},V_{23},\dots,\dots,V_{n-1,n}$\,.)
Compare {\it time-baseline order}.
\par
Baseline-time ordering of a {\it u-v data file} is convenient for
purposes of data display.

\aa{batch editor}
a {\it text editor} within the AIPS program
which allows the user to prepare {\it batch jobs} \qv,
to be run non-interactively.

\aa{batch job}
AIPS may be run either interactively---%
allowing the user to make `split-second' decisions---%
or in batch mode.
In batch mode, the user first decides on a set strategy for
reducing the data, and then, using the special AIPS {\it batch editor},
the user prepares a {\it text file},
containing those AIPS commands which are appropriate to the anticipated
data reduction needs.
The batch job is placed in a {\it batch queue},
and the job steps are executed by the {\it batch processor},
in a non-interactive mode.

\aa{batch processor}
the server, or scheduler, for {\it batch jobs} \qv.
The AIPS batch processor follows certain rules in scheduling:
batch jobs requiring the use of an array processor (AP)
often are scheduled to run only during nighttime hours;
the processor serving one of the {\it batch queues} might
refuse service, altogether, to a job requiring an AP;
and batch jobs may be given lower priority than those AIPS
tasks which are run interactively.

\aa{batch queue}
a waiting line for {\it batch jobs}.
The AIPS batch queue is a single-server queue---i.e.,
the server (the {\it batch processor}\/)
initiates the execution of the jobs one after the other,
rather than in parallel.
However, AIPS can be configured with more than one batch queue,
each with its own batch processor; this number varies according to site.

\aa{``battery-powered'' Clean algorithm}
a modified version of the Clark Clean algorithm, devised
by Fred Schwab and Bill Cotton.
At each major cycle of the algorithm, or perhaps less frequently,
the residual map is computed not by convolving the current
iterate with the dirty beam map,
but rather by computing the visibility residuals,
and then re-gridding and re-mapping.
By this means, the edge effects are compensated,
and hence one can search the full dirty map field of view
for Clean components.
Simultaneously, instrumental effects (finite bandwidth and
finite integration time) and sky curvature (the $wz$ term)
can be compensated for (i.e., the algorithm solves a more general
equation than a convolution equation).
See {\it Clark Clean algorithm.}
\par
A ``mosaicing'' version of this algorithm is implemented in the
AIPS task MX.
The deconvolved image is defined over some number $1\le n\le16$ of
rectangular patches.
Within each patch, the data are corrected for sky curvature,
by the correction appropriate to the center of the patch.
Instrumental corrections are not included, at present.

\aa{beam}
1. in radio interferometry,
the inverse Fourier transform ($\fti$) of the {\it u-v sampling distribution},
or $\fti$ of a weighted \uv sampling distribution,
possibly convolved with a gridding convolution function---%
the idealized response to a point, or unresolved, radio
source.
\xspace2. a numerical approximation to 1.
\xspace3. a digitized version of 2, sampled on a
regular grid (usually regarded as a map or image).
\xspace 4. $\approx$  {\it point spread function}, q.v.
\xspace 5. (occasionally) as above, but taking into account
instrumental effects, so that the beam depends on position
in the sky.
See {\it dirty map.}
\par
Occasionally, any one of the above, other than 5, is termed the
{\it synthesized beam.}

\aa{beam patch}
in the Clark Clean algorithm, that portion of the central part
of the beam which is used in the inner iterations, or the minor cycles.
In the AIPS implementation, the beam patch size typically
is set at $101\,\text{pixels}\times101\,\text{pixels}$.
See {\it Clark Clean algorithm.}

\aa{beam squint}
In radio interferometry, direction dependent, or space-variant
{\it instrumental polarization}, which is difficult to calibrate,
can arise from {\it beam squint}.
The beam squint effect, for the usual case of a pair of (nominally)
orthogonally polarized feeds on each array element,
is due to differences in their power patterns---%
in particular, to differences in the directions of their peak response.

\aa{blanked pixel}
in a digital image, a pixel whose value is undefined.
In computer storage of quantized digital images, some
special numeric value is assigned to the blanked pixels,
so that they may be recognized as undefined and given whatever special
treatment is required.
See {\it pixel.}

\aa{BLC} {\it bottom left corner} (of an image).
See $m\times n$ {\it map.}

\aa{blink} See {\it TV blink.}

\aa{boot}
A computer is restarted by means of a
{\it boot}\/strapping procedure, whereby the operating system
and the data management facilities are re-initialized in a
succession of steps.
This ritual, through which the computer gathers it wits,
is termed the {\it boot}.
A boot ($\approx$ {\it re-boot}\/) is required after any
system crash (e.g., after a power failure).
Usually the sequence of steps required to accomplish the
boot is posted in a notice located close to the system operating
console, or on the CPU panel.
On modern computers, such as the Vax, the boot procedure is highly automated.
In fact, there may be an abbreviated boot procedure, termed
a {\it quick boot}, to follow after a ``soft'' system crash.
(On such systems, a quick boot should be attempted before
resorting to a full boot.)
Indeed, some systems (the Vax included) re-boot on their
own initiative following a soft system crash---this is
termed an {\it auto re-boot}.

\aa{BOT marker} (Beginning-Of-Tape marker)
a short strip of metal foil attached near the front, or
beginning, end of a computer magnetic tape.
The tape drive uses the BOT marker in order to position
the tape at its starting position.

\aa{bpi} (bits per inch)
the basic unit of measurement used to specify
the density at which information is recorded on a computer magnetic tape:
the effective number of bits per inch per track.
The standard recording densities are 800, 1600, and 6250\,bpi.
Modern computer tapes are nine-track tapes:
eight recording tracks are used for the data, and the ninth track
is used to record ``parity bits'' for error-checking.
See {\it tape blocking efficiency}.

\aa{broadband mapping technique}
a refinement of the radio interferometric method of
{\it bandwidth synthesis} \qv, in which one solves simultaneously
for the radio brightness distribution $f_{\nu_r}(x,y)$
at some reference frequency $\nu_r$, and for the (spatially varying)
spectral index $\alpha(x,y)$ across the observing band.
Assuming that the observing band is split into frequency
channels centered at $\nu_1,\dots,\nu_n$, one solves the simultaneous
system of convolution equations $g_1=b_1\ast f_1,\dots,g_n=b_n\ast f_n$,
where $g_k$ is the {\it dirty map} from channel $k$, $b_k$ the
{\it dirty beam} from that channel, and where $f_k$ is given by
$$f_k(x,y)=\left(\nu_k\over\nu_r\right)^{\alpha(x,y)}f_{\nu_r}(x,y)\,.$$
All of the $b_k$ are identical, apart from a dilation factor.
Assuming that the frequency channels are narrow enough, one can
expand the {\it u-v coverage} considerably,
with immunity to the {\it bandwidth smearing} effect.
Fractional bandwidths as large as 20--30\%\ can be used, depending
on the linearity of the spectral index variations.
\par
This mapping technique is described by Tim Cornwell [Broadband mapping
of sources with spatially varying spectral index,
VLB Array Memo.\ No.~324, Feb.~1984].
Extensive modification of one of the standard deconvolution algorithms
is required.
The requisite modification of the \hca\ is in progress.

\aa{b-t order} See {\it baseline-time order}.

\aa{bug}
an actual or a perceived programming error or program deficiency.
The bug may be in the eye of the beholder since the
program user may fancy an application similar to, but
differing from, the one for which the program is intended.
In AIPS there is a formal mechanism for reporting program bugs;
see {\it gripe file} for a description.

\aa{byte}
a unit of eight bits of computer storage.

\aa{carriage-return key}
One of the most used keys on any computer terminal keyboard is the
carriage-return key (\CRn).
This is the button which ordinarily must be depressed when one has
finished typing a command to the computer, in order for the
computer to accept or acknowledge the command.

\aa{catalog entry}
an entry within an AIPS {\it catalog file} (``CA'' file) pertaining
to a particular {\it primary data file}.

\aa{catalog file}
In AIPS, each user has, for each disk on which he has data stored,
his own {\it catalog file}, or ``CA'' file---%
a directory of all of his primary data files which reside on that disk.
The AIPS {\it verb} CATALOG (as do its variants MCAT and UCAT)
allows the user to see a summary listing of the contents of his
catalog files.
See {\it header record}.

\aa{catalog slot}
in AIPS, a numbered space reserved in a {\it catalog file} for the
insertion of a {\it catalog entry}.

\aa{cell-averaging}
in radio interferometer mapping, gridding convolution which is achieved
simply by averaging the visibility data which lie in each \uv grid cell.
This is equivalent to use of a {\it gridding convolution function}
equal to the {\it characteristic function} of the rectangle
$\{|u|<\Delta u/2,\ |v|<\Delta v/2\}$, where $\Delta u$ and $\Delta v$
denote the grid spacing---i.e., it is equivalent to the use of a
so-called {\it pillbox} function.
The Fourier transform of the pillbox gridding convolution function
is proportional to a separable product of two $\sin x\over x$ functions;
this function does not decay rapidly enough to yield very effective
{\it aliasing} suppression.
The zero-order {\it spheroidal functions} offer much better aliasing
suppression, at somewhat increased computational expense (equivalent
to averaging the data over a region 36 times larger, in the case of
the default gridding convolution function used by the AIPS mapping
tasks).

\aa{cellsize}
in radio interferometer mapping, the size $\Delta u \times \Delta v$
of the \uv grid cells.
Ordinarily, the visibility data are smoothed by an appropriate
{\it gridding convolution function} and this convolution then is sampled
at the coordinate locations of the centers of the grid cells.
After appropriate weighting, the {\it discrete Fourier transform}
yields the {\it dirty map}.
$\Delta u$ and $\Delta v$ are chosen according to {\it Shannon's
sampling theorem}\/:  if the size of the dirty
map is $x$ radians by $y$ radians, then $\Delta u={1\over x}$
wavelengths and $\Delta v={1\over y}$ wavelengths.

\aa{cereal bowl map defect}
same as {\it negative bowl artifact}.
See {\it zero-spacing flux}.

\aa{characteristic function}
The characteristic function $\charfn A$ of a set $A\subset X$
is defined for all $x\in X$ by the formula
$$\charfn A(x)= \left\{ \begin{array}{ll}
             1\,,&\text{if\ } x\in A\,,\\
             0\,,&\text{if\ } x\notin A\,
             \end{array} \right.
$$
($\charfn A$ is also called the {\sl indicator function} of $A$,
and the notations $c_A$ and $1_A$ commonly are used in lieu of $\charfn A$.)
Note that this usage of the term, which is standard in mathematical
analysis, differs from its usage in probability and statistics,
where it refers to the Fourier transform of a probability measure
(i.e., to the FT of the distribution function of a random variable).

\aa{chromaticity}
in visual perception, essentially the dominant wavelength
and the purity of the spectral distribution of light, as perceived.
{\it Hue} and {\it saturation} determine the chromaticity,
which is independent of {\it intensity}.
See {\it C.I.E. chromaticity diagram}.

\aa{C.I.E. chromaticity diagram}
a two-dimensional diagram devised in 1931
by the Commission Internationale de l'Eclairage
(International Commission on Illumination)
to show the range of perceivable colors
as a function of normalized chromaticity coordinates $(x,y)$,
under standardized viewing conditions.
The color, for an additive mixture of monochromatic
red, green, and blue ($R,G,B$ denoting the intensities
at 650, 520, and 380\,nm.\ wavelengths)
as perceived by a `standard observer',
is displayed in this diagram as a function of
the normalized {\it chromaticity coordinates}
$x=R/(R+G+B)$ and $y=G/(R+G+B)$.
\par
Other chromaticity diagrams can be drawn for different choices
of primary hues, for mixtures of nonmonochromatic light,
or for `nonstandard observers'.
In digital imagery, such a diagram may be tailored
to a particular color image display unit.
See [G.~S.~Shostak, Color basics---a tutorial.
In R.~Albrecht and M.~Capaccioli, I.A.U.\ Astronomical
Image Processing Circular No.~9, Space Telescope Science
Institute, Jan.~1983] and
[G.~Wyszecki and W.~S.~Stiles, {\it Color Science}, Wiley, New York, 1967],
a comprehensive textbook on colorimetry.

\aa{Clark Clean algorithm}
a modified version of the H\"ogbom Clean algorithm, devised
by Barry Clark in order to accomplish an efficient {\it array processor}
implementation of Clean (see [B.~G.~Clark, An efficient implementation of
the algorithm Clean, {\it Astron.\ Astrophys.}, \bold{89}
(1980) 377--378]).
To operate on, say, an $n\times n$ map, the original Clean algorithm
requires on the order of $n^2$ arithmetic operations at each iteration,
and typically there may be hundreds or thousands of iterations.
The Clark algorithm proceeds by operating not on the full residual
map, but rather by picking out only the largest residual points,
iterating on these for a while (during its {\it minor cycles} or
inner iterations) and only occasionally (at the {\it major cycles}\/)
computing the full $n\times n$ residual map, by means of the
FFT algorithm.
After each major cycle, it again picks out the largest residuals
and goes into more minor cycles.
And, for further economy, during these inner iterations
the dirty beam is assumed to be identically zero outside of a
relatively small box (termed the {\it beam patch}\/) which is centered
about the origin.
See \hca.

\aa{Clean} See \hca.

\aa{Clean beam}
in the H\"ogbom Clean algorithm, an elliptical Gaussian function
$h$ with which the final iterate is convolved, in order to
diminish any spurious high spatial frequency features---%
also termed {\it restoring beam.}
$h$ is specified by its major axis (usually the FWHM), its minor axis,
and the position angle on the plane of the sky of its major axis.
Usually these parameters are set by fitting to the central lobe
of the dirty beam.
See \hca\ and {\it super-resolution.}

\aa{Clean box}
a rectangular subregion of a {\it Clean window} \qv.

\aa{Clean component}
in the H\"ogbom Clean algorithm, a $\delta$-function component which
is added to the $(n-1)$st iterate in order to obtain
the $n$th iterate.
Its location is the location of the peak residual after the
$(n-1)$st iteration, and its amplitude is a fraction
$\mu$ (the {\it loop gain}\/) of the largest residual.
See \hca.
\par
The AIPS task implementing the (Clark) Clean algorithm stores a list
of the Clean components in an extension file which is termed
a {\it components file.}

\aa{Clean map}
an approximate deconvolution of the {\it dirty beam}
from the {\it dirty map,} derived by an application of
the H\"ogbom Clean algorithm or one of its derivatives.
See \hca.

\aa{Clean speed-up factor}
in the {\it Clark Clean algorithm}, a number $\alpha$ in the
range $[-1,1]$ used in determining when to end a major cycle.
Smaller $\alpha$ causes a larger number of major cycles to occur
(at greater computational expense)
but yields a result closer to that of the classical \hca.

\aa{Clean window}
in the H\"ogbom Clean algorithm, the region $A$ of the residual map
which is searched in order to locate the {\it Clean components}
comprising the successive approximants to the radio
source brightness distribution.
In the AIPS implementation, $A$ is a union of
rectangles, called {\it Clean boxes,} which may be specified
by the user.
When $A$ is not explicitly specified, the algorithm
searches over the central rectangular one-quarter area of the residual map.
See {\it window Clean} and \hca.

\aa{clipping}
the discarding (i.e., the {\it flagging}\/) of visibility data whose
amplitudes exceed some threshold value, or the discarding of visibility
data whose differences from some tentative source model are too large
in amplitude.
The AIPS task CLIP is used for clipping.
See \uv {\it data flag}.

\aa{closure amplitude}
Assume that the visibility observation on the $i$--$j$ baseline
($i<j$) is given by
$\s V_{ij}=g_i\overline g_jV_{ij}$, where $V_{ij}$ is the true
visibility and where $g_i$ and $g_j$ are the {\it antenna/i.f.\ gains}
(ignore any additive error).
Then, for certain combinations of (at least four) baselines,
one may form ratios of observed visibilities (and their conjugates)---%
including each visibility only once---%
in such a manner that the $g$'s cancel one another.
For example, if $i<j<k<l$, then
$${\s V_{ij}\s V_{kl}\over\s V_{il}{\overline{\s V}}_{jk}} =
{V_{ij}V_{kl}\over V_{il}\overline{V}_{jk}}\,.$$
The modulus of such a ratio is termed a {\it closure amplitude}
(and its argument, a {\it closure phase}).
\par
Closure amplitude is called a ``good observable'',
since, under the above assumptions, it is not sensitive to
measurement error.
The closure amplitude and closure phase relations are
exploited in the {\it hybrid mapping algorithm} \qv.
Also see {\it self-calibration algorithm}.

\aa{closure phase}
Assume that the visibility observation on the $i$--$j$ baseline
($i<j$) is given by
$\s V_{ij}=g_i\overline g_jV_{ij}$, where $V_{ij}$ is the true
visibility and where $g_i$ and $g_j$ are the {\it antenna/i.f.\ gains}
(ignore any additive error).
Then, for a combination of any three or more baselines
forming a closed loop, one may sum the visibility phases
in such a manner that the {\it antenna/i.f.\ phases} $\psi_k$ drop out.
For example, if $i<j<k$, then
$\arg \s V_{ij}+\arg \s V_{jk}-\arg \s V_{ik}=
\arg V_{ij}+\psi_i-\psi_j+\arg V_{jk}+\psi_j-
\psi_k-\arg V_{ik}-\psi_i+\psi_k$.
Such a linear combination of observed visibility phases
is termed a {\it closure phase}.
\par
Closure phase is called a ``good observable'', since,
under the above assumptions, it is not sensitive to measurement error.
The closure phase relations are exploited in the
{\it hybrid mapping algorithm} \qv.
Also see {\it closure amplitude} and {\it self-calibration algorithm}.

\aa{color contour display}
a color digital image display of a real-valued function $f$
of two real variables $(x,y)$,
in which the color assignment (the {\it hue}\/)
is a coarsely quantized function of $f(x,y)$.
The visual effect of this type of {\it pseudo-color display},
in the case when $f$ is continuous,
is similar to the traditional sort of contour display.
One sees curves along which $f$ is constant, separated by
swathes of constant hue---each hue corresponding to a distinct quantization
level.

\aa{color triangle}
Any three non-collinear points plotted on a chromaticity diagram
determine a color triangle.
Since the points are non-collinear, they correspond to
basic, or {\it primary} hues.
All of those colors on the chromaticity diagram
which fall within the triangle determined by the three points
may be produced by addition of the three hues.
See {\it C.I.E. chromaticity diagram}.

\aa{compact support} See {\it support.}

\aa{components file}
in AIPS, an extension file, associated with an image file
containing a {\it Clean map,} whose content is a list of
the positions and amplitudes
of the {\it Clean components} included in that Clean map,
as determined by the Clean algorithm.
The source model specified by this list of components
often is used in {\it self-calibration.}

%\aa{Comtal display unit}
%a computer-controlled TV display device with capabilities
%similar to those of the \iis.
%A Comtal unit has been in use at the VLA for several years,
%as part of the IMPS image analysis system.
%At present this model of display unit is not
%used in any of the AIPS systems at the NRAO.
%See {\it IMPS} and {\it \iis}.

\aa{conjugate symmetry}
that property which characterizes a {\it Hermitian function} \qv.
Generally an assumption of conjugate symmetry is implicit
whenever one speaks of the \uv {\it coverage}
corresponding to some radio interferometric observation.

\aa{Conrac monitor}
the CRT unit of the \iis\ TV display device, in use at a number
of AIPS installations.
See {\it \iis}.

%\aa{contrast resolution}

\aa{convolution theorem}
This theorem is well-known, but seldom is quoted in its distributional
form:  for two distributions, $f$ and $g$, the Fourier transform
of the convolution of $f$ and $g$ is given by
$\widehat{f\ast g} {\phantom{l}}=\hat f\hat g$, whenever one distribution
is of {\it compact support} and the other is a ``tempered'' distribution.
(Loosely speaking, a tempered distribution is one which does not
increase too rapidly at infinity.)
See [Y.~Choquet-Bruhat, C.~Dewitt-Morette, and M.~Dillard-Bleick,
{\it Analysis, Manifolds, and Physics}, North--Holland, New York,
1977, ch.~VI].
\par
One ought to be aware of this
form of the theorem, since often one must deal with convolution
of functions that are not of compact support---%
{\it dirty beams, principal solutions, invisible distributions,} etc.---%
whose Fourier transforms do not exist as ordinary functions,
but only as distributions or generalized functions.
\par
Convolution of distributions, itself, is defined, in general, whenever
the support of either distribution is compact, or (in one
dimension) when the supports of both distributions are limited
on the same side.
For distributions which are absolutely integrable ordinary functions,
and whose Fourier transforms possess the same property,
the compact support assumption is not required here, or above.
Related fact: convolution is not always associative
(i.e., $f\ast(g\ast h) \noteq (f\ast g)\ast h)$, in general),
but it is associative provided that all the distributions, with
the possible exception of one, are of compact support.
See the above-cited reference.

\aa{convolving function} See {\it gridding convolution function}.

\aa{coordinate reference pixel}
in an AIPS {\it image file}, a {\it ``pixel''} whose coordinates are
recorded in the image {\it header} together with the coordinate increments
(i.e., the pixel coordinate separations) that allow the physical
coordinates of all other pixels in the image to be computed.
This ``coordinate reference pixel'' may not actually be present in
the image: all that matters are its physical coordinates and its
pixel coordinates (which too are recorded in the header---%
and which may, in fact, be fractional).
\par
Often, in a radio map (and by default, when the standard AIPS mapmaking
tasks are executed), the position of the coordinate reference pixel
coincides with the map center and with the {\it visibility phase
tracking center}.
See $m\times n$ {\it map} and {\it pixel coordinates}.

\aa{correlator offset}
One of the basic assumptions of much of the VLA calibration software
(e.g., the {\it self-calibration algorithm}\/)
is that the systematic errors in the visibility measurements
are multiplicative errors that are ascribable to
individual array elements and their associated i.f./l.o.\ chains,
and that---at a given instant---each such antenna-based error
has an identical effect on each visibility observation involving
that antenna/i.f.\ combination.
Systematic measurement errors which do not conform to this model
are called {\it correlator offsets} or {\it non-closing errors}.
See {\it antenna/i.f.\ gain}.
\par
Correlator offsets can be the limiting factor in obtaining
high dynamic range VLA maps.
Some observers have reported fairly large multiplicative
correlator offsets which vary slowly with time
and which do not appear to vary with the {\it phase tracking center}
or with source structure.
From observations of an external calibrator,
one may estimate, and compensate for, such offsets.
This mechanism is provided in the AIPS tasks BCAL1 and BCAL2.
See [R.~C.~Walker, Non-closing offsets on the VLA,
VLA Scientific Memo.\ No.~152].

\aa{crash}
the abrupt failure of a computer system or program.
More specifically,
a {\it system crash} is the abrupt failure of a computer---%
or of a computer's operating system---causing the computer to halt
the execution of programs;
and a {\it program crash} is the abrupt failure of a computer
program resulting either from a flaw in the logic of the
program itself, or from some peculiar interaction
with the operating system, the storage management facility,
another program, or the user---or from an act of God.
A {\it hardware crash} (e.g., a {\it disk crash}\/)
is a crash which results from the failure
of the computer electronics or electro-mechanics,
and a {\it software crash} is one
which results from a flaw or an inadequacy in program logic,
or in operating system program logic.
A {\it soft crash} is a crash from which it is easy to recover---%
i.e., easy to restart the computer and resume work---,
and a {\it hard crash} is the opposite.

\aa{crosshair}
1. a marker on the {\it TEK screen}, or {\it green screen},
which may be moved about through the use of thumbwheel knobs which are
located on the terminal keyboard panel.
The position of the crosshair may be sensed by the computer
program, and thus the user may point out to the program
features that are of interest in the graphical display on the CRT
screen.
\xspace2. a marker with the same function as just described,
but on a TV display device, and more likely controlled by a {\it trackball}
than by thumbwheels.
Same as {\it TV cursor}\/; and see {\it trackball}.

\aa{cube} See {\it data cube}.

\aa{cursor}
1. a marker on an interactive computer terminal indicating
the position on the CRT screen where the next character is to be typed.
\xspace2. {\it TV cursor}---on a TV display device, a marker
whose manually controlled position may be sensed by the computer.
See {\it crosshair}.

\aa{data cube}
1. in VLA spectral line data analysis, a three-dimensional map or ``image''
representing a function of three real variables---%
two spatial variables representative of position in the sky,
and one variable related to frequency or velocity.
\xspace2. any $n$-dimensional {\it image,} $n\ge3$.
\par
Computer access of a multi-dimensional data array,
residing in any standard type of storage medium such as
disk or magnetic tape,
is sequential, as if the data were one-dimensional.
Spectral line data cubes are stored plane-by-plane,
row-by-row, column-by-column.
Permutation of the correspondence between plane, row, and column,
and the coordinate axis numbering, is referred to as {\it transposition}
of the data cube.

\aa{database}
a computer filing system, or file structure system.
For example, the AIPS database consists not only of the
data themselves, but also of the directories and the cross-reference
lists of all the AIPS data files (including extension files),
the data format definitions, etc.,
as well as the rules and principles governing the use thereof.

\aa{data file}
on a computer storage medium, such as disk or magnetic tape,
the concrete, or physically present representation
of a logically distinct grouping of data
in a manner permitting repeated access by
computer programs.

\aa{data flag}
See \uv {\it data flag}.

\aa{deconvolution}
the numerical inversion of a convolution equation, either continuous
or discrete, in one or several variables;
i.e., the numerical solution (for $f$) of an equation of the form
$f\ast g=h+\text{noise}$, given $g$ and given the right-hand side
of the equation.
Except in trivial cases, deconvolution is an ill-posed problem:
In the absence of constraints or extra side-conditions,
and in the case of noiseless data---assuming that some solution exists---%
there usually will exist many solutions.
In the case of noisy data, there usually will exist no exact solution,
but a multitude of approximate solutions.
In the latter case, if one is not careful in the choice of a numerical method,
the computed approximate solution is likely not to have
a continuous dependence on the given data.
The so-called {\it regularization method} \qv\ (of which the {\it maximum
entropy method} is a special case) is an effective tool for the
deconvolution problem.
\par
Discrete two-dimensional deconvolution is an everyday problem in
radio interferometry, owing to the fact that---under certain simplifying
assumptions---the so-called {\it dirty map} is the convolution of
the {\it dirty beam} with the true celestial radio image.
In addition to the maximum entropy method,
the {\it H\"ogbom Clean algorithm} is commonly applied to this problem.
See Tim Cornwell and Robert Braun's Lecture No.~8 in the \sira.

\aa{delay}
See {\it residual delay}.

\aa{delay beam}
in radio interferometry, the {\it point spread function} or {\it beam},
taking into account {\it bandwidth smearing}, but ignoring other
instrumental effects.
See {\it bandwidth smearing}.

\aa{DFT}
an abbreviation for {\it discrete Fourier transform}
and {\it direct Fourier transform} \qv.
When used in disciplines other than radio astronomy,
it usually signifies the former.

\aa{Dicomed Image Recorder (Model D47)}
a computer-controlled image display device intended for
photographic reproduction of digital images.
The film is exposed by a cathode ray tube.
The device is capable of $4096\,\text{pixel}\times4096\,\text{pixel}$
resolution and of both black-and-white and color reproduction.
The digital exposure control and eight-bit pixel input
allow 256 discrete exposure levels.
The CRT has a single electron gun and a screen with a
white phosphor; color reproduction is accomplished by means of multiple
exposures, with the insertion of red, green, and blue filters.
There is a Dicomed recorder at the NRAO in Charlottesville,
and another at the VLA.


\aa{direct Fourier transform}
a term used imprecisely in radio astronomy to mean either:
1) a finite trigonometric sum, of the form
$$\sum_{j=0}^{n-1}a_j e^{2\pi iu_jx}\,,$$
with $a_j$ complex, where the (real) $u_j$ are irregularly-spaced;
2) the brute-force {\sl evaluation} of such a sum;
or 3), the na\"\i ve, or brute-force evaluation (using $O(n^2)$
arithmetic operations) of the ($n$-point) {\it discrete Fourier transform}.
\par
The direct Fourier transform, in senses 1) and 2) of the definition,
arises in synthesis mapping applications because of the irregular
distribution of the visibility measurements.
Common practice is to use a {\it gridding convolution
function} to interpolate the data onto a regularly-spaced lattice,
so that, for computational economy,
the {\it fast Fourier transform algorithm} may be used.

\aa{dirty beam}
in radio interferometry, simply a {\it beam,}
but computed with precisely the same operations as
those used to compute some companion {\it dirty map}
(i.e., with the same \uv coverage, the same manner of gridding
convolution, the same \uv weight function and taper, etc.).
In Cleaning a dirty map, only the companion dirty beam should be used.

\aa{dirty map}
1. ignoring instrumental effects, the inverse Fourier transform
($\fti$) of the product of the visibility function $V$
of the radio source and the (possibly {\it weighted}
and\slash or {\it tapered}\/) \uv {\it sampling distribution} $S$;
i.e., $\fti$ of the \uv {\it measurement distribution.}
\xspace2.
a discrete approximation to 1;
in this case, the product $SV$ is convolved with some function
$C$, of {\it compact support,} and an inverse discrete Fourier transform
of samples of $C\ast(SV)$ taken over a regular grid yields the
{\it dirty map}.
\xspace3.
as in 2, but corrected for the taper ($\check C$, the $\fti$ of $C$)
induced by the convolution.
\xspace4.
any of the above, but now taking into account various instrumental
effects (receiver noise, non-monochromaticity or finite bandwidth, finite
integration time, sky curvature, etc.).
\par
If it is assumed that $V\equiv1$, then the map,
or point source response, so obtained is termed the {\it beam} \qv.
Also see {\it gridding convolution function, u-v taper function,
u-v weight function, dirty beam,} and {\it principal solution.}

\aa{discrete Fourier transform}
The (one-dimensional) discrete Fourier transform (DFT) $y_0,\dots,y_{n-1}$
of a sequence of complex numbers $x_0,\dots,x_{n-1}$
is given by the summation
$$y_k=\sum_{j=0}^{n-1}x_j e^{2\pi ijk/n}\,.$$
(The multi-dimensional generalization is straightforward).
The $x_j$ are given by the {\it inverse DFT} of the $y_k$:
$$x_j={1\over n}\sum_{k=0}^{n-1}y_k e^{-2\pi ijk/n}\,.$$
(Frequently the forward and inverse transforms are defined in the manner
opposite to that given here, and the $1\over n$ normalization factor
sometimes is moved about.)
The DFT arises most naturally in numerically approximating the Fourier
coefficients $c_m={1\over2\pi}\int_0^{2\pi}f(x)e^{-imx}\,dx$
of a $2\pi$-periodic function $f$ which is representable
by the trigonometric series $\sum_{m=-\infty}^\infty c_me^{imx}$.
The {\it fast Fourier transform algorithm} \qv\ can be used for efficient
numerical evaluation of the DFT.

\aa{disk hog}
a derogatory term, used to connote a computer user whose disk data files
are excessively voluminous or numerous, therefore putting
other computer users at a relative disadvantage.
Unneeded data files should be {\it scratched}, or destroyed,
in order to free up disk space.
Large disk files which will not be needed for a time
should be {\it backed-up} on magnetic tape and then deleted
from disk.

\aa{dynamic range}
a summary measure of image quality indicative
of the ability to discern dim features when relatively stronger
features are present---i.e., a measure of the ability to distinguish
the dim features
from artifacts of the {\it image reconstruction} procedure
(in a radio map, from remnants of the sidelobes of stronger features)
and from noise.
The dynamic range achievable in a radio interferometer map is determined
primarily by the uniformity of the {\it u-v coverage},
the density and extent of the coverage, the sensitivity of the array,
and the quality of the calibration.
\par
If the true radio source brightness distribution $f$ is known,
one can define the dynamic range of a reconstruction $\s f$
as, say, the ratio of the maximum value of $|f|$ to the r.m.s.\
difference between $f$ and $\s f$.
When $f$ is unknown, as is usually the case, an empirical measure
of the dynamic range is used---perhaps the ratio of the maximum
value of $|\s f|$ to the r.m.s.\ level in an apparently empty
region of the map, or the ratio of the strongest feature to the
weakest ``believable'' feature---,
but there is no widely-accepted definition.
\par
What one might wish to call the ``true'' dynamic range of a radio map
is a spatially-variant quantity.
The ability to discern a dim feature depends on
its proximity to brighter features, because there are relatively
stronger sidelobe remnants near the bright features.
The quality of a map
(and perhaps the dynamic range---depending on how it is defined)
deteriorates away from the {\it phase tracking center},
because of the inability of the
image reconstruction algorithms to compensate for various instrumental
effects (e.g., bad pointing, {\it bandwidth smearing}, etc.).

\aa{EDT}
a sophisticated text editor (a {\it screen editor}) used on the
Vaxes.
It makes use of the ``keypad'' feature of the fancier terminals.
EDT can be run only on certain model terminals:
on the DEC (Digital Equipment Corp.) Models VT--52
and VT--100, and on terminals such as the Visual--50's and the
Visual--100's which are capable of emulating the DEC terminals.
See {\it text editor}.

\aa{EMACS}
a sophisticated text editor used on the Vaxes, as
well as on many computers which run under the UNIX operating system.
(There is also a version for the IBM-PC.)
EMACS is a {\it screen editor}, and the one which is favored by
most among those in the AIPS programming group.
On terminals with the ``keypad'' feature, the keypad keys
can be programmed by the user to perform many useful editing tasks;
however, EMACS can be run from other models of terminals, as well.
EMACS provides two powerful and convenient features which most
other text editors do not offer:
the ability to temporarily exit from the editor and ``return to
monitor level,''
and the ability to initiate an interactive ``job control session,''
or initiate {\it sub-tasks}, in an EMACS buffer.
See {\it text editor}.

\aa{explain file}
in AIPS, a {\it text file} containing a detailed explanation of a
particular AIPS {\it task} or {\it verb}, often including
hints, suggested applications, algorithmic details,
and bibliographical references.
Issuing the AIPS verb EXPLAIN causes the contents of an
explain file to be printed on the terminal screen or on a line printer.
Compare {\it help file}.

\aa{EXPORT format}
a visibility data magnetic tape format for transport of
VLA data from the DEC--10 computer or the on-line computer at the VLA.

\aa{EXPORT tape}
a magnetic tape containing data recorded in the {\it EXPORT format}.

\aa{exp $\times$ sinc function}
a useful gridding convolution function:
same as the {\it Gaussian-tapered sinc function} \qv,
except that the exponent of the argument to the exponential function
may be other than two.

\aa{extension file}
in AIPS, a data file containing data supplemental to those
contained in a {\it primary data file} (either a {\it u-v data file}
or an {\it image file}).
Whenever a primary data file is deleted by the standard
mechanism within AIPS for file destruction, all extension
files associated with that primary data file also are destroyed.
Extension files, however, may be deleted without deleting the
the associated primary data file.
\par
Extension files are grouped into categories of named types.
Examples: {\it plot files, history files, slice files, gain files,}
etc.
\par
When an AIPS task creates a new primary data file from an old one,
generally it attaches, to the new file, clones of any extension
files associated with the old file that remain relevant to the
new one.

\aa{false color display}
In digital imagery, a {\it false color display} is one which is generated by
using a number $n>1$ of real-valued functions
$f_1(x,y),\dots,f_n(x,y)$ to control the
proportions, at each {\it pixel} coordinate $(x,y)$,
of an additive mixture of three primary hues.
In practical terms, the user of a digital display system supplies
$f_1,\dots,f_n$, and twists knobs that control the mapping
$\bold R^n\to\bold R^3$ that sends the $n$ pixel values
at each $(x,y)$ into the proper image {\it chromaticity} and {\it intensity}.
Compare {\it pseudo-color display}.
\par
A so-called {\it true color display} is obtained with $n=3$
and with {\it transfer functions} chosen such that the color assignment
corresponds in an approximate way to the actual coloration of a scene
(as in a color photograph).

\aa{fast Fourier transform algorithm}
a fast algorithm for the computation of the {\it discrete Fourier transform}
(DFT) $y_0,\dots,y_{n-1}$ of a sequence
of $n$ complex numbers $x_0,\dots,x_{n-1}$,
$$y_k=\sum_{j=0}^{n-1}x_j e^{2\pi ijk/n}\,,$$
typically requiring only $O(n\log n)$ arithmetic operations ---
or a multi-dimensional generalization thereof.
By contrast, straightforward, or na\"\i ve evaluation of the DFT
requires $O(n^2)$ operations.
The fast Fourier transform algorithms (FFT's) which currently are the
most popular are the Cooley--Tukey (1965) algorithms, for the case of
$n$ highly composite.
For $n$ a power of two, the (radix-2) Cooley--Tukey FFT requires
about $2n\log_2n$ real multiplications and $3n\log_2n$ real additions.
More generally, the Cooley--Tukey algorithms require
a few times $n\sigma(n)$ complex arithmetic operations,
where $\sigma(n)$ is the sum of the prime factors of $n$,
counting their multiplicities.
S.~Winograd has produced FFT algorithms which are more efficient than
those of Cooley and Tukey, typically requiring about the same number
of additions, but only about 20\%\ the number of multiplications.
(Computation of the required complex exponentials---or sines and
cosines---is not counted, since these generally are either pre-computed and
stored in compact tables, or generated recursively.)
\par
A further advantage of the FFT algorithms is their avoidance of
round-off error, which can build up severely when the DFT is
evaluated by brute-force.
There are related, fast algorithms for the convolution of sequences of
real numbers, for the discrete cosine transform, etc.
Algorithmic details may be found in [H.~J.~Nussbaumer,
{\it Fast Fourier Transform and Convolution Algorithms}, Springer--Verlag,
Berlin, 1982].
The computational complexity of the DFT is discussed by
L.~Auslander and R.~Tolimieri [Is computing with the finite
Fourier transform pure or applied mathematics, {\it Bull.\ (New Series)
Amer.\ Math.\ Soc.}, \bold{1} (1979) 847--897].
\par
AIPS programs which use the FFT make use of the Cooley--Tukey algorithm.
When an {\it array processor} is used to compute the large two-dimensional
DFT's of data which reside on disk, as typically is required in
synthesis mapping,
the input\slash output time greatly exceeds the actual computation time.
%Those AIPS installations which use the {\it pseudo-array processor}
%might benefit if the Winograd FFT's were implemented
%in the pseudo-AP routines.
%Choosing among the Winograd algorithms is a bit tricky,
%because they are different for different $n$,
%and for any given $n$ there may be a number of algorithms---%
%which algorithm is best depends on the coding style, the compiler,
%and the hardware architecture of the computer.

\aa{FFT} See {\it fast Fourier transform algorithm.}

\aa{FITS format}
(Flexible Image Transport System) a magnetic tape data format
well-tailored for the transport of image data among observatories.
The FITS format is recommended for bringing data into and out of AIPS.
See [D.~C.~Wells, E.~W.~Greisen, and R.~H.~Harten,
FITS: A flexible image transport system,
{\it Astron.\ Astrophys.\ Suppl.\ Ser.}, \bold{44} (1981) 363--370].
Also see {\it u-v FITS format} and {\it FITS tape}.

\aa{FITS tape}
a magnetic tape containing data recorded in the {\it FITS format}.
FITS format data blocks are 2880 bytes in length.
The resultant {\it tape blocking efficiency} is
83\%, 75\%, and 61\%\ at recording densities of
800, 1600, and 6250\,bpi, respectively.

\aa{flagging}
in AIPS, the act of discarding one or more visibility data points
by setting a \uv {\it data flag} \qv.
Compare {\it clipping}.

\aa{fringe rotator}
in a correlating-type radio interferometer,
a mechanism to introduce a time-varying phase shift
into the local oscillator signal of a receiver, in order
to reduce the frequency of the oscillations of the correlator output.
Fringe rotation allows the correlator output (whose amplitude
is proportional to visibility amplitude) to be sampled at a lower rate.
The natural fringe frequency can be as high as 200 Hz on the VLA.
The fringe rotation is chosen so that the fringe frequency for
a point source located at the so-called {\it fringe stopping center}
would be reduced to zero, or at least close to zero.
Usually the fringe stopping center and the {\it delay tracking center}
coincide;
both then are called the {\it visibility phase tracking center}.
For further details, see A.~R.~Thompson's Lecture No.~2
and L.~R.~D'Addario's Lecture No.~4 in the \sira,
and see R.~M.~Hjellming and J.~Basart's Ch.~2 of the {\it Green Book}.

\aa{full-synthesis map}
in earth-rotation aperture synthesis, with stationary interferometer
elements, a {\it map} derived from an observation which is of such
lengthy duration that the fullest possible \uv {\it coverage} is
obtained (i.e., from an observation extending from ``horizon to
horizon'').
Compare {\it snapshot}.

\aa{gain file}
in AIPS, an {\it extension file}, associated with a {\it u-v data
file}, in which a table of approximate {\it antenna/i.f.\ gains}
(typically obtained by {\it self-calibration}) is stored.

\aa{Gaussian-tapered sinc function}
A useful {\it gridding convolution function} \qv, of {\it support width}
equal to
the width $m\Delta u$ of $m$ \uv grid cells, is given by the separable product
of two Gaussian-tapered sinc functions, each of the form
\def\ee{{\pi u\over b\Delta u}} \def\ff{{\pi u\over a\Delta u}}
%$$C(u)= \cases
%{e^{-\left(\ff\right)^2}\sin\ee\over\ee}\,,&
%    |u|<{m\Delta u\over2}\,,\\
%0\,,&\text{otherwise}\,.\endcases$$
$$C(u)=\left\{ \begin{array}{ll}
  {\left(\ee\right)^{-1}e^{-\left(\ff\right)^2}\sin\ee}\,,&
      |u|<{m\Delta u\over2}\,,\\
   0\,,&\text{otherwise}\,.
  \end{array} \right.
$$
The choice $m=6$, $a\simeq2.52$, and $b\simeq1.55$, yields what is,
in a certain natural sense, an optimal gridding convolution
function of this particular parametric form
(see [F.~R.~Schwab, Optimal gridding, VLA Scientific Memo.\ No.~132]).
%(see the paper by F.~R.~Schwab in the \sydp).
Also see {\it spheroidal function}.

\aa{Gerchberg--Saxton algorithm}
a simple iterative algorithm which, in the field of signal
processing, is used for the extrapolation of band-limited signals---%
and, in image processing, for deconvolution.
Assume that the Fourier transform $\hat f$ of an image $f$
has been measured over a region $B$, and that $f$ is known
to be confined to a region $A$.
Let $\charfn A$ denote the {\it characteristic function} of $A$
and $\charfn B$ that of $B$.
\def\ghap{\hat g_{\operatorname{approx}}}
Denote the measured data by $\ghap$---i.e.,
$\ghap=\charfn B\hat f+\operatorname{error}$.
From the initial approximant $f_0$
($f_0\equiv0$ may be used)
a sequence $f_n$ of successive approximants to $f$
is obtained, via the formula
$$f_{n+1}=f_n+\mu\charfn A\cdot(\ghap-\charfn B\hat f_n)\formcheck\,.$$
Here, $\ \check{}\ $ denotes inverse Fourier transform,
and $\mu$ is a fixed scalar, analogous to the {\it loop gain}
parameter of the H\"ogbom Clean algorithm.
\par
To apply the algorithm in radio interferometry, one may
identify $\charfn B$ with the $u$-$v$ {\it sampling distribution}
and think of $A$ to be analogous to a {\it Clean window}.
Denoting the {\it dirty map} by $g$ and the {\it dirty beam}
by $b$,
the iteration can be written as
$$f_{n+1}=f_n+\mu\charfn A\cdot(\hat g-\hat b\hat f_n)\formcheck
\quad\left(=f_n+\mu\charfn A\cdot(g-b\ast f_n)\right)\,.$$
The Gerchberg--Saxton algorithm has been implemented
by Tim Cornwell in an AIPS program named APGS.
APGS includes an {\it ad hoc} nonnegativity constraint---%
at each iteration,
any pixel value which would be driven negative
is modified to become nonnegative.
Convergence usually is sluggish.
\par
Some algorithms which are very similar to the Gerchberg--Saxton algorithm
are the Lent--Tuy algorithm, which is used in medical imaging,
the Papoulis, or Papoulis--Youla algorithm, used in signal processing,
and the so-called method of alternating orthogonal projections,
used in image reconstruction.
See [J.~L.~C.~Sanz and T.~S.~Huang,
Unified Hilbert space approach to iterative least-squares linear
signal restoration, {\it J.~Opt.\ Soc.~Am.}, \bold{73} (1983) 1455--1465]
and references cited therein.

\aa{Gibbs' phenomenon}
in the neighborhood of a discontinuity of a periodic function $f$,
the overshoot and oscillation (or ringing)
of the partial sums $S_n$ of the Fourier series for $f$.
In the vicinity of a simple jump discontinuity,
$S_n$ always overshoots the mark by about 9\%, regardless how large $n$.
See [H.~S.~Carslaw, {\it Introduction to the Theory of Fourier's
Series and Integrals}, Dover, New York, 1930, ch.~IX].
\par
In harmonic analysis,
often the Fourier coefficients are multiplied by a weight function
tending smoothly to zero at the boundaries of its {\it support,}
in order to smooth out the discontinuities
and thereby reduce the ringing in the synthesized spectrum.
(This degrades the spectral resolution, however.)
See {\it Hanning smoothing}.
For a discussion of Gibbs' phenomenon in the context
of VLA cross correlation analysis, see Larry D'Addario's
Lecture No.~4 in the \sira.

\aa{GIPSY}
(Groningen Image Processing System)
a data reduction system, similar in scope to AIPS, used in the Netherlands for
analysis of Westerbork Synthesis Radio Telescope (WSRT) data.

\aa{global fringe fitting algorithm}
an antenna-based algorithm
(in the spirit of the {\it self-calibration algorithm}\/)
for VLBI fringe search.
For an $n$ element array, the classical VLBI fringe fitting technique,
a correlator-based method, requires the estimation of $n^2-n$ parameters.
The global fringe fitting method reduces this number to $3n-3$.
Expressing the {\it antenna/i.f.\ gain} for antenna $k$ of the
array as $g_k(t,\nu)=a_ke^{i\psi_k(t,\nu)}$
(here we include a frequency dependence)
one has that the observed visibility on the $i$--$j$ baseline,
to first-order, is given by
$$
V_{ij}(t,\nu)=a_ia_jV_{ij}(t_0,\nu_0) \\
\hfil\times
e^{\sqrt{-1}((\psi_i-\psi_j)(t_0,\nu_0)+(r_i-r_j)(t-t_0)+(\tau_i-\tau_j)
(\nu-\nu_0))} \,,$$
where $V_{ij}$ is the true visibility,
and where the $r_k$ are the {\it antenna residual fringe rates}
and the $\tau_k$ the {\it antenna residual delays}.
\par
Given a source model, one may solve for the $\psi_k(t_0,\nu_0)$,
the $r_k$, and the $\tau_k$,
using either a least-squares method or a Fourier transform method.
Because of the overdeterminacy provided by a simultaneous
solution for the parameters,
this method allows proper delay and fringe rate compensation
of data on baselines of too low signal-to-noise
for the correlator-based method to work effectively.
A full description of the method is given in
[F.~R.~Schwab and W.~D.~Cotton, Global fringe search techniques
for VLBI, {\it Astron.~J.}, \bold{88} (1983) 688--694].
This algorithm is implemented in the AIPS program CALIB.

\aa{graphics overlay plane} same as {\it graphics plane.}

\aa{graphics plane}
a storage area within a TV display device, such as the \iis,
in which a full screen load of one-bit graphics information (labeling,
plotting, axis lines, etc.) is stored.
A typical \iis\ unit is equipped with four graphics planes,
each $512\,\text{pixels}\times512\,\text{pixels}$ in area.
Compare {\it image plane}.

\aa{gray-scale display}
a black-and-white display of a digitized {\it image}---%
typically either a photographic or a video display.

\aa{gray-scale memory plane} same as {\it image plane.}

\aa{Green Book}
{\it An Introduction to the NRAO Very Large Array},
edited by R.~M.~Hjellming, NRAO, Socorro, NM---a useful
reference on many of the technical aspects of the VLA.

\aa{green screen} same as {\it TEK screen}.

\aa{gridding convolution function}
in radio interferometer mapmaking, a function $C$---%
usually supported on a square the width of, say,
six \uv grid cells---with which the \uv measurement distribution
is convolved.
The purpose is twofold: 1) to interpolate and smooth the data, so that
samples may be taken over the lattice points of a rectangular grid
(in order that the fast Fourier transform algorithm may be applied)
and 2) to reduce aliasing (the convolution in the \uv
plane induces a taper in the map plane).
See {\it aliased response, gridding correction function, cell-averaging,
dirty map,} and {\it uniform weighting}.
\par
With judicious choice of $C$, a high degree of aliasing
suppression is possible.
A high degree of suppression is desirable, even when there
are no ``confusing'' radio sources very near the field of interest,
because the effect is not only to reduce the spurious responses due to
sources lying outside of the field of view, but also to
reduce the response to sidelobes of the source of
interest, which too are aliased into the map from outside the
field of view.
See {\it spheroidal function.}

\aa{gridding correction function}
in radio interferometry,
the reciprocal $1/\hat C$ of the Fourier transform (FT)
of the {\it gridding convolution function} $C$.
Since the map plane taper induced by the gridding convolution
usually is very severe, the dirty map normally is corrected
by pointwise division by the FT of the convolution function.
Obviously $C$ should be chosen such that $\hat C$ has no zeros
within the region that is mapped.
See {\it dirty map}.

\aa{gripe}
in AIPS, an entry in the {\it gripe file} \qv.

\aa{gripe file}
in AIPS, a disk file repository for formal reports
of program {\it bugs}, and for formal complaints and
suggestions of a more general nature.
A mechanism by which the user may enter gripes into the gripe file
is activated by the issuance of the AIPS verb {\sl GRIPE}.
The AIPS group provides prompt, written responses to all {\it gripes}.

\aa{Hanning smoothing function}
in the analysis of power spectra, a weight function $w$ by
which the measured correlation function is multiplied,
in order to reduce that oscillation ({\it Gibbs' phenomenon}\/)
in the computed spectrum which is due to having sampled at only a
finite number of lags.
$w$, as a function of lag, is given by
$$w(\tau)=\left\{ \begin{array}{ll}
{1\over2}\left( 1+\cos{\pi\tau\over\tau_{\operatorname{max}}} \right)\,,
&|\tau|<\tau_{\operatorname{max}}\,,\\0\,,&\text{otherwise}\,.
\end{array}\right.
$$
This is equivalent to convolving the discrete spectrum with the sequence
$\{\frac14,\frac12,\frac14\}$.
\par
Hanning smoothing sometimes is applied to the cross correlation
measurements obtained in VLA spectral line observing,
in order to reduce the effect of sharp bandpass filter cutoffs.
It also is used frequently in radio astronomical autocorrelation
spectroscopy.
See {\it Gibbs' phenomenon}, and for more on smoothing see
[R.~B.~Blackman and J.~W.~Tukey, {\it The Measurement of Power
Spectra}, Dover, New York, 1958].

\aa{hard copy}
computer output printed on paper (rather than, say, written on magnetic tape);
e.g., a printed contour plot or gray scale display,
or a listing of a catalog file.

\aa{hardware mount}
the combined acts of installing a computer external storage module,
such as a disk pack or a reel of magnetic tape,
in some electro-mechanical unit
(e.g., a disk drive or a tape drive) that provides computer access
to this data storage medium,
and placing that unit in readiness to be operated under
computer control (e.g., positioning a magnetic tape at
the {\it BOT marker}\/).
Compare {\it software mount.}

\aa{header record}
a distinguished record within a {\it data file}---%
generally the first record---%
which serves to define the contents of the other records in the file
by supplying relevant parameters, units of measurement, etc.;
also termed simply {\it header}.
\par
In AIPS, however, the header record of each {\it primary data file}
is stored apart from that file, in a file which is termed a ``CB'' file.
And a directory, termed a {\it catalog file} \qv, or ``CA'' file,
of all of each user's primary data files on a given disk is stored on
that disk.
AIPS {\it extension file} headers are stored within the extension
files themselves.

\aa{help file}
in AIPS, a {\it text file}, whose contents may be displayed
on the terminal screen of the interactive user,
giving a brief explanation of a particular AIPS verb, adverb,
pseudoverb, task, or miscellaneous general feature.
Compare {\it explain file}.

\aa{Hermitian function}
a complex-valued function, of one or more real variables,
whose real part is an even function and whose imaginary part is odd.
The Fourier transform (FT) of a real-valued function is Hermitian,
and the inverse FT of a Hermitian function is real.
\par
Since each of the radio brightness distributions
$I(x,y)$, $Q(x,y)$, $U(x,y)$, and $V(x,y)$
representing {\it Stokes' parameters}
is real-valued,
{\it Stokes' visibility functions} have the property
of {\it conjugate symmetry}\/:
$V_I(-u,-v)=\overline V_I(u,v)$,
$V_Q(-u,-v)=\overline V_Q(u,v)$,
$V_U(-u,-v)=\overline V_U(u,v)$,
and $V_V(-u,-v)=\overline V_V(u,v)$.
(Here, $V_I=\hat I$, $V_Q=\hat Q$, etc., where $\ \hat {}\ $ denotes FT.)

\aa{history file}
in AIPS, an {\it extension file} containing a summary
of all, or most of the processing, by AIPS tasks, of the data recorded
in all associated files.

\aa{Hogbom Clean algorithm}
a deconvolution algorithm
devised by Jan H\"ogbom for use in radio interferometry
[J.~A.~H\"ogbom, Aperture synthesis with a non-regular
distribution of interferometer baselines, {\it Astron.\ Astrophys.\
Suppl.\ Ser.,} \bold{15} (1974) 417--426].
Denote (the discrete representations of) the dirty map by $g$
and the dirty beam by $b$.
The algorithm iteratively constructs discrete approximants $f_n$
to a solution $f$ of the equation $b\ast f=g$,
starting with an initial approximant $f_0\equiv0$.
At the $n$th iteration, one searches for the peak in the
residual map $g-b\ast f_{n-1}$.
A $\delta$-function component, centered at the location of the largest
residual, and of amplitude $\mu$ (the {\it loop gain}\/)
times the largest residual, is added to $f_{n-1}$ to yield $f_n$.
The search over the residual map is restricted to a region $A$
termed the {\it Clean window.}
The iteration terminates with an approximate solution $f_N$
either when $N$ equals some iteration limit $N_{\operatorname{max}}$,
or when the peak residual (in absolute value) or the r.m.s.\
residual decreases to some given level.
\par
To diminish any spurious high spatial frequency features in
the solution, $f_N$ is convolved with a narrow elliptical Gaussian
function $h$, termed the {\it Clean beam.}
Generally $h$ is chosen by fitting to the central lobe of the
dirty beam.
Also, one generally adds the final residual map $g-b\ast f_N$
to the approximate solution $f_N\ast h$, in order to produce
a final result, termed the {\it Clean map,} with a
realistic-appearing level of noise.
See {\it super-resolution.}

\aa{host computer}
In the parasitic relationship of a computer program or program package,
such as AIPS, to the computer on which it runs, the latter is
termed the {\it host computer.}
Also, in the master--slave relationship of a computer to
one of its peripheral devices, such as an array processor,
the master may be termed the {\it host.}

\aa{hue}
one of the three basic parameters ({\it hue}, {\it intensity},
and {\it saturation}\/) which may be used to describe the
physical perception of the light that reaches one's eye.
Hue, which is also termed {\it tint}, or simply {\it color},
refers to the dominant wavelength of the coloration,
at a given location in an image or scene.
The term also may be used to describe
a multimodal color spectrum---e.g., one speaks of a purple hue.
Different spectral distributions of light,
of identical intensity and saturation,
are capable of producing identical retinal responses;
these unique responses comprise the set of perceptible hues.
\par
Color matching tests have established that
there are three basic types of human retinal receptors,
whose peak responses are to red, green, and blue light.
These are the three {\it primary hues} used in additive color
mixing---e.g., in digital image display.
They may be used to produce all, or virtually all, of the
perceptible hues.
\par
See {\it C.I.E. chromaticity diagram}.

\aa{hybrid mapping algorithm}
an algorithm for calibration of radio interferometer data
which is essentially equivalent to the {\it self-calibration
algorithm} \qv\ (used in VLA data reduction), except in that it
makes explicit use of the {\it closure phase} and
{\it closure amplitude} relations,
rather than explicit use of the relation
$\s V_{ij}=g_i\overline g_jV_{ij}$
relating observed visibility to the product of the true
visibility and a pair of {\it antenna/i.f.\ gains}.
Hybrid mapping, which is used extensively in VLBI data reduction,
is described in
[A.~C.~S.~Readhead {\it et al.}, Mapping radio sources with
uncalibrated visibility data, {\it Nature}, \bold{285}
(1980) 137--140].
\par
Either algorithm (assuming that one cares to make some distinction)
can be applied to data obtained with connected-
(e.g., the VLA) and non-connected-element
interferometers (e.g., VLBI arrays).
Any differences in the results produced by the two algorithms
would be attributable primarily to differences in the effective weighting
of the data (in particular, early implementations of both
algorithms discarded data which could have been used to obtain overdetermined
solutions for the calibration parameters).

\aa{IIS} See {\it \IIS}.

\aa{image}
in the context of AIPS, any finite-volume, linear, rectangular,
or hyper-rectangular array of pixels;
e.g., a digitized photograph, or a radio map.
The term also is used (less technically) to refer to the {\sl display}
of data---e.g., a television picture of a radio map.

\aa{image catalog}
in AIPS, a disk file containing data records
describing the data stored on the TV display device {\it image planes.}
These records are essentially identical in structure
to the {\it header records} stored in the {\it catalog file.}
The data in the image catalog furnish the information that
is required for proper axis labeling, pixel value retrieval, etc.

\aa{image file}
in AIPS, a {\it primary data file} whose content is an {\it image.}

\aa{image plane}
a storage area within a TV display device, such as the {\it \IIS},
in which a full screen load of single word pixels is stored.
A typical \iis\ unit is equipped with four image planes,
each $512\,\text{pixels}\times512\,\text{pixels}$ in area
(each pixel is represented by eight bits).
Often several image planes are used at one time---%
either for black-and-white or {\it pseudo-color} display
of a large image, sections of which may occupy different image planes---%
or for {\it false color} or {\it true color} display of a
smaller image, now using, say, three image planes---one to control
each of the three electron guns (for red, green, or blue phosphor)
in the TV display.
Compare {\it graphics plane}.

\aa{image reconstruction}
the attempted recovery of an {\it image} after it has undergone
the distorting effects, the blurring, etc., produced by some
physical measurement and recording device, such as a camera,
a radio interferometer, or a tomography machine.
The operation of many measurement devices can be adequately modeled by
a linear Fredholm integral equation of the first kind.
In the two-dimensional case, e.g., one assumes
that the measurement $g(x,y)$ is related to the undistorted image $f(x,y)$
by the equation
$$g(x,y)=\int_{-\infty}^\infty\int_{-\infty}^{\infty}
          K(x,y,x^\prime,y^\prime)f(x^\prime,y^\prime)\,
          dx^\prime\,dy^\prime+\epsilon(x,y)\,.$$
(Often it is convenient to use the more compact, operator notation,
$g=\bold Kf+\epsilon$.)
The kernel $K$ of the equation is called the {\it point spread function}, \qv.
Measurement error and the error arising from any simplifying
assumptions are lumped together into the $\epsilon(x,y)$ term.
Some particularly well-behaved measurement systems can be adequately modeled
by a simple convolution equation, in which case $K$ is given by
$K(x,y,x^\prime,y^\prime)=h(x-x^\prime,y-y^\prime)$.
This is the case, e.g., when the VLA is used to observe
a small `unconfused' radio source; then $g$ may be identified with
the {\it dirty map} and $h$ with the {\it dirty beam}.
Or when $K$, considered as a function of $(x,y)$,
is given at each $(x^\prime,y^\prime)$ by the {\it delay beam}
for that position, the equation models the {\it bandwidth smearing
effect} \qv; as the bandwidth $\to0$, the convolution model
again becomes valid.
\par
Except in trivial cases, solution of the Fredholm equation always
is an ill-posed problem.
Mild conditions on $K$ and $f$
(the classical `Picard conditions'---see F.~Smithies
[{\it Integral Equations}, Cambridge Univ.\ Pr., London, 1958])
ensure the existence of (non-unique) solutions when $\epsilon\equiv0$.
But, because of the effect of measurement noise,
one usually does not seek an exact solution,
but rather an approximate solution---one which fits the data
to within the measurement errors.
Uniqueness and regularity of the computed approximate solution are obtained
by imposing such constraints as known {\it support}\/,
nonnegativity, and smoothness conditions.
See {\it regularization method}.
Also see H.~C.~Andrews and B.~R.~Hunt [{\it Digital Image
Restoration}, Prentice--Hall, Englewood Cliffs, NJ, 1977]
and {\it phaseless reconstruction}.

%\aa{IMPS}
%(Interactive Map Processing System) an image analysis system in use at the
%VLA site, configured around a DEC PDP--11/44 computer and
%a Comtal TV display device.
%IMPS' capabilities are similar to the image display and
%image manipulation capabilities of AIPS.

\aa{inputs file}
in AIPS, a {\it text file}, whose contents may be displayed
on the terminal screen of the interactive user,
giving a summary of the {\it adverbs} relevant to a given
{\it verb} or a given AIPS {\it task}.

\aa{instrumental polarization}
any contamination of a polarization measurement
by an instrument's response to an undesired polarization state.
In radio interferometry, the instrumental polarization arises mainly from
feed imperfections and from plumbing leaks between the feeds
and the receiver front-ends.
One tries to remove the instrumental polarization by applying
corrections derived from observations of calibration sources
whose polarization properties are known.
Within AIPS, there is, at present, no facility for polarization calibration.
The polarization calibration of VLA data normally takes place on the
DEC--10 computer at the VLA.
For more details, see Carl Bignell's Lecture No.~4 in the \ssp.
See {\it beam squint}.

\aa{intensity}
one of the three basic parameters ({\it hue}, {\it intensity},
and {\it saturation}\/)
which may be used to describe the physical perception of color.
Intensity is a measure of the energy of the spectral distribution,
at a given point in an image or scene,
weighted by the spectral response of the visual system.
{\it Luminance} is the energy of the physical spectrum,
but not weighted by the visual response.
{\it Brightness} sometimes is used synonymously with either term.
\par
See {\it C.I.E. chromaticity diagram}.

\aa{invisible distribution}
in the context of radio interferometry,
a function $f$ (or a generalized function---or distribution)
whose Fourier transform $\hat f$ vanishes everywhere that the
interferometer pairs have sampled.
This term was introduced by R.~N.~Bracewell and J.~A.~Roberts
[Aerial smoothing in radio astronomy, {\it Austr.\ J.~Phys.},
\bold{7} (1954) 615--640].
Also see {\it principal solution}.
\par
For an actual interferometer,
there exist fewer physically plausible invisible distributions
than for an idealized interferometer.
This is because each visibility sample is not a point sample
of $\hat f$, but rather some kind of local average.
By the {\it Paley--Wiener theorem}\/, if $\hat f$ is nontrivial and vanishes
in some open neighborhood, then $f$ cannot be of {\it compact support},
and hence it may be considered implausible.

\aa{IPL}
(Initial Program Load) same as {\it boot}.

\aa{isoplanaticity assumption}
in the context
of radio interferometry (the term is used too in optics),
the assumption that over each element of an array
all wavefronts arriving from different parts of the sky
to which the interferometer pairs are sensitive
are subject to identical atmospheric phase perturbations.
A patch of sky over which the assumption is valid
is referred to as an {\it isoplanatic patch}.
\par
Approximate validity of the isoplanaticity assumption
is a necessary condition for the success of calibration
({\it self-calibration}, in particular) of radio interferometer
data (from an earth-based array)
if one is to rely on a model incorporating
time-varying {\it antenna/i.f.\ gains}, one per antenna,
whose arguments (or phases) are to include the atmospheric
phase corruption.
However, see F.~R.~Schwab [Relaxing the isoplanatism assumption
in self-calibration; applications to low-frequency radio
interferometry, {\it Astron.~J.}, \bold{89} (1984) 1076--1081].

\aa{I$^2$S}
(International Imaging Systems Models 70 and 75) a TV display device,
capable of both black-and-white and color display,
manufactured by the Stanford Technology Corporation.
At an AIPS site typically it is equipped with four
$512\,\text{pixel}\times512\,\text{pixel}$ eight-bit {\it image planes},
four one-bit
{\it graphics planes}, a {\it trackball,} and sometimes an {\it ALU}.
The eight-bit pixel representation (in the image planes)
allows the intensity of
each of the three electron gun beams to be set at any of 256
discrete levels.
(Actually, 1024 levels can be used, because of an
extra two bits of capability provided in the
{\it transfer function} tables and the internal arithmetic unit.)
An \iis\ is attached to three of the NRAO's computers
on which the AIPS system runs (the VLA and Charlottesville Vaxes).

\aa{line editor}
a {\it text editor} \qv\ which allows the modification
of single lines or records within a text file,
but one which does not allow the simultaneous modification
of more than one line.
{\it SOS} and {\it SEDIT} are both line editors.
{\it Screen editors} \qv\ are more versatile than line editors.

\aa{lobe rotator}
same as {\it fringe rotator}, \qv.

\aa{loop gain}
in the H\"ogbom Clean algorithm, the fraction $\mu$ of the
largest residual which is used in determining the amplitude, or flux,
of a {\it Clean component}.
Convergence can be achieved for $\mu$ in the range $(0,2)$,
but generally a small value, say $\mu={1\over10}$,
is recommended, especially in dealing with extended sources.
See \hca.

\aa{luminance} See {\it intensity}.

\aa{$l_1$ solution algorithm} See
{\it self-calibration gain solution algorithm.}

\aa{$l_2$ solution algorithm} See
{\it self-calibration gain solution algorithm.}

\aa{major cycle}
In the {\it Clark Clean algorithm} \qv, a number of minor cycles,
or inner iterations, followed by the computation by the FFT
algorithm of the full residual map, comprise a major cycle.

\aa{map}
an {\it image,} one or more of whose coordinate axes
represents some spatial coordinate.

\aa{maximum entropy method}
a {\it regularization method} \qv\ for the numerical solution
of ill-posed problems, given noisy data, in which the regularizing
(or smoothing) term---which measures the roughness of the
computed approximate solution $\s f$---%
is given by the negative of the Shannon entropy of $\s f$, $-H(\s f)$\,:
in the continuous case, letting $A$ denote the domain of
definition of $\s f$,
$H(\s f)\equiv -\int_A \s f(x)\log \s f(x)\,dx$,
where $\s f$ has been normalized so that $\int_A \s f(x)\,dx=1$
(and $0\log0\equiv0$);
and in the discrete case, $H(\s f)\equiv -\sum \s f(x_i)\log \s f(x_i)$,
where $\s f$ has been normalized so that $\sum \s f(x_i)=1$.
The underlying philosophy of the method,
espoused early on by Jaynes
(``Jaynes' method of prior estimation'', [E.~T.~Jaynes,
Prior probabilities, {\it IEEE Trans.\ Syst.\ Sci.\ Cyb.},
\bold{SSC-4} (1968) 227--241]) and by J.~P.~Burg at a 1967
meeting of the Society of Exploration Geophysicists,
is that one is being ``maximally noncommittal'' in regard to
the insufficiency of the data if one maximizes the entropy,
and thus minimizes the ``information content'', of $\s f$,
subject to the constraint that $\s f$ should agree with the
given data.
\par
For one-dimensional discrete convolution equations, with
noiseless, regularly-spaced data, there exists a closed-form
solution---for other cases, iterative methods are used, as
with other forms of the regularization method.
\par
Use of the method in radio astronomy was encouraged
by J.~G.~Ables in 1972 in public lectures,
and it now is in common use in radio interferometry
(cf. [S.~F.~Gull and G.~J.~Daniell, Image reconstruction from
incomplete and noisy data, {\it Nature}, \bold{272} (1978)
686--690]).
Nonnegativity of the computed solution is a natural by-product
of the method.
For reconstruction of polarized brightness distributions
in interferometry (Stokes' $Q$, $U$, and $V$),
which, unlike the total intensity, may assume negative values,
Ponsonby has derived an appropriate generalization of the method
[J.~E.~B.~Ponsonby, An entropy measure for partially polarized
radiation\dots, {\it Mon.\ Not.\ R.\ Astr.\ Soc.}, \bold{163} (1973)
369--380].
See {\it Variational Method}.

\aa{memory page} See {\it virtual memory page}.

\aa{memory paging} same as {\it virtual memory page swapping}.

\aa{memory thrashing}
an excessive amount of {\it virtual memory page swapping}
\qv\ on a computer (such as the Vax) with a virtual memory operating system.
A condition of memory thrashing is likely to occur
whenever too many programs with large memory requirements are active
(a single program with excessive memory requirements
also can cause memory thrashing).

\aa{message file}
in AIPS, a {\it text file} containing progress report messages
generated during the execution of AIPS {\it tasks}
and also containing a chronicle of the user's interaction (via
{\it verb} commands) with AIPS.
Each AIPS user is assigned a message file, the contents of
which may be printed out, typed upon a terminal display screen,
or emptied---at will---by invoking the appropriate verb command.
See {\it AIPS monitor}.

\aa{message terminal}
same as {\it AIPS monitor}.

\aa{minor cycle}
in the {\it Clark Clean algorithm} \qv, an inner iteration,
in which the peak residual over a subregion (the {\it Clean
window}\/) of the full residual map is found and is used to
obtain the next successive iterate.
Compare {\it major cycle.}

\aa{microcode} See {\it array processor microcode}.

\aa{monitor} See {\it AIPS monitor} or {\it Conrac monitor.}

\aa{$m\times n$ map}
The convention adopted for AIPS is opposite the standard
matrix algebra terminology:
whereas an $m\times n$ matrix is comprised of $m$ rows
and $n$ columns, an $m\times n$ {\it map} or {\it image}
in AIPS has, in the usual display format, $m$ pixels along
the horizontal axis (usually termed the $x$-axis) and $n$
pixels along the vertical axis (usually termed the $y$-axis).
Moreover, pixels of a two-dimensional map in the usual display format
are numbered from the bottom left-hand corner:  the pixel
location specified by the ordered pair $(i,j)$ is in column number
$i$ and row number $j$, counting from the bottom left.
In other than two-dimensional ``images'', the $(1,\dots,1)$
pixel is also said to be at the ``bottom left corner'' (BLC),
just as in the two-dimensional case.
See {\it data cube}, {\it pixel coordinates}, and {\it coordinate
reference pixel}.

\aa{MX} See {\it ``\/battery-powered'' Clean algorithm}.

\aa{natural weighting} See {\it uniform weighting.}

\aa{negative bowl artifact} See {\it zero-spacing flux}.

\aa{non-closing offset} See {\it correlator offset}.

\aa{Nyquist sampling rate}
the slowest rate of sampling which, according to the {\it Shannon
sampling theorem} \qv, would allow a band-limited function $f(t)$ to
be recovered via the {\it Shannon series}.
If the smallest symmetric interval which contains the {\it support} of
the Fourier transform of $f$ is the interval $[-a,a]$, then
the Nyquist sampling rate for $f$ is $2a$; i.e., the
interval between samples (the {\sl sampling period}\/)
must be less than the {\sl reciprocal bandwidth} $1/2a$.
The terms {\sl oversampling} and {\sl undersampling}
refer to sampling at rates faster or slower than the Nyquist rate.
The difference between $f$ and the Shannon series formed
from too coarsely spaced samples is called {\it aliasing}.

\aa{operating system}

\aa{page} See {\it virtual memory page} and {\it terminal page}.

\aa{page swapping} See {\it virtual memory page swapping}.

\aa{Paley--Wiener theorem}
The classical Paley--Wiener theorem says that a square-integrable
complex-valued function $\hat f$, defined over the real line,
can be extended off the real line as an entire function of
exponential type $\le 2\pi a$ if and only if $f(x)\equiv0$ for $|x|>a\,$---%
i.e., iff $\hat f$ is band-limited to $[-a,a]$
(here $\ \hat{}\ $ denotes Fourier transform).
(An everywhere-analytic function $g(z)$ is said to be of exponential type
$\le A$ if $\exists c$ such that, for all $z$, $|g(z)|\le ce^{A|z|}$.)
For a derivation, see H.~Dym and H.~P.~McKean
[{\it Fourier Series and Integrals}\/, Academic Press, 1972].
The {\it Shannon series}\/ is a means of extending $\hat f$ to $\bold C$.
The extension of the Paley--Wiener theorem to the case of generalized
functions (to tempered distributions) is called the Paley--Wiener--Schwartz
theorem.
\par
The Fourier transform $\hat f:\bold R^n\to\bold C$ of a function $f$
with support in a given $n$-dimensional convex compact set $K$
can be analytically extended to all of $\bold C^n$.
Growth properties on $\hat f$ which are sufficient in order for the
converse to hold are given by K.~T.~Smith, D.~C.~Solomon, and S.~L.~Wagner
[Practical and mathematical aspects of the problem of reconstructing
objects from radiographs, {\it Bull.\ Amer.\ Math.\ Soc.}, \bold{83}
(1977) 1227--1270] (in addition to the classical version of the
multi-dimensional Paley--Wiener theorem, for rectangular $K$,
they give versions with tighter growth bounds, and for arbitrary convex $K$).
Smith {\it et al.}\/ use the Paley--Wiener theorems to establish
indeterminacy theorems for tomographic reconstruction.
Their results are also relevant to Fourier synthesis, because of the
connection between the two-dimensional Fourier transform and the
one-dimensional Radon transform.
The Paley--Wiener theorems have also been used in establishing results
on the problem of {\it phaseless reconstruction} \qv\
and in proving the convergence of constrained {\it Gerchberg--Saxton}\/-type
algorithms (see A.~Lent and H.~Tuy [An iterative method for the
extrapolation of band-limited functions, J.~Math.\ Anal.\ Appl.,
\bold{83} (1981) 554--565]).

\aa{phaseless reconstruction}
the reconstruction of an image $f$ (see {\it image reconstruction}\/)
from knowledge of (only) the magnitude $|\hat f|$ of the Fourier transform
of $f$ (and usually from only partial knowledge of $|\hat f|$).
Phaseless reconstruction has been considered for the NRAO's proposed
millimeter wave interferometer array [T.~J.~Cornwell, Imaging of weak
sources with compact arrays, NRAO Millimeter Array Memo.\ No.~12].
Recent results on phaseless reconstruction appear in the JOSA
Feature Issue on Signal Recovery
[{\it J.~Opt.\ Soc.~Am.}, \bold{73} No.~11 (Nov.\ 1983)].
Also see the papers by J.~R.~Fienup and by R.~H.~T.~Bates
{\it et al.}\ in the \sydp.

\aa{phase tracking center}
same as {\it visibility phase tracking center}, \qv.

\aa{physical memory}
core or semiconductor memory within a computer (as opposed
to slower memory---virtual memory, disk storage, magnetic tape footage, etc.).
A typical Vax is equipped with a physical memory 3--4 megabytes in size.

\aa{pillbox}
See {\it cell-averaging}.

\aa{pixel}
({\it pic}\/ture {\it el}\/ement) an element of a digitized image (or of
a map).
A pixel is characterized by its position in the image
and by its numerical value.
See $m\times n$ {\it map}, {\it coordinate reference pixel},
and {\it pixel coordinates}.

\aa{pixel coordinates}
in an AIPS {\it image file}, the {\it pixels} are numbered consecutively,
beginning with $(1,\dots,1)$ at the bottom left corner ({\it BLC}\/)
of the image.
See {\it coordinate reference pixel} and $m\times n$ {\it map}.

\aa{plot file}
an AIPS extension file containing plotting information, in the
form of the commands which are necessary in order for
a line drawing peripheral device, such as a Calcomp or other pen plotter,
a green screen, or an electrostatic printer\slash plotter, to
generate a plot.

\aa{point source response} same as {\it point spread function}.

\aa{points per beam}
in a digitized radio map, the characteristic width, somehow defined,
of the major lobe of the {\it beam} pattern, or {\it point spread function},
divided by the {\it pixel} separation.
Ordinarily the number of points per beam is calculated
by measuring the narrowest diameter of the 50\%\ contour level
of the major lobe of the beam.
To avoid excessively severe discretization error,
deconvolution algorithms such as the
\hca\ and the {\it maximum entropy method}
require, as a rule-of-thumb, at least three (and preferably 4--5)
points per beam.

\aa{point spread function} (PSF)\xspace
1. the response of a system or an instrument to an impulsive,
or point source, input.
\xspace2. in radio interferometry, the response of the instrument to
a point, or unresolved, radio source---%
a fancy term for {\it beam}.
Ignoring instrumental effects, such as finite bandwidth and
finite integration time, the response does not depend upon
the displacement of the source away from the {\it visibility phase
tracking center}\/---hence the term {\it space-invariant PSF
(SIVPSF)}, and the contrary term {\it space-variant PSF (SVPSF)}.
\par
A so-called linear space invariant measurement system (i.e.,
a linear system with an SIVPSF) is equivalently described
as a system which can be modeled by a convolution equation;
a linear space-variant measurement system is modeled
by a more general linear Fredholm integral equation of the first kind.
See {\it image reconstruction}.

\aa{POPS}
(People-Oriented Parsing System) the parser, or command
interpreter, embedded within the AIPS program; that part
of the AIPS program which attempts to interpret the user's
commands ({\it POPS symbols}\/) and then initiate the appropriate reaction.
POPS is used in other astronomical data reduction
programs at the NRAO: in Condare, TPOWER/SPOWER, and the
Tucson 12\,m single-dish packages.

\aa{POPS procedure} See {\it POPS symbols.}

\aa{POPS symbols}
The AIPS user's primary means of communicating his wishes
to AIPS is by typing commands,
termed {\it POPS symbols,} at the keyboard of a computer terminal.
There are four classes of POPS symbols: {\it adverb, verb,
pseudoverb,} and {\it procedure.}
An {\it adverb} is a {\it symbol} representing the storage area
for a datum or for data that are used to control the action of
verbs, tasks, and procedures; that is to say that the adverb symbols
are used to set {\sl control parameters.}
A {\it verb} is a symbol which causes POPS (or AIPS) to
initiate some action after POPS has finished interpreting,
or compiling, the command line typed at the computer terminal.
A {\it pseudoverb}
is a symbol which suspends, temporarily, the normal parsing
of an input line and which causes some action to take place
while the line is being compiled, and, possibly, after compilation.
A {\it procedure}
is a symbol representing a pre-compiled sequence of POPS symbols.
Also see {\it task.}

\aa{primary beam correction}
in radio interferometry, the multiplicative correction of a radio {\it map}
by the reciprocal of an average of the power patterns of the array elements.
Measurements of the primary beam parameters of the 25\,m VLA elements
are given by Peter Napier and Arnold Rots in the memorandum
[VLA primary beam parameters, VLA Test Memo.\ No.~134, Feb.~1982].
There an average power pattern and its reciprocal are approximated by radial
functions, polynomials in the distance from the pointing position.
The AIPS task PBCOR is used to apply this correction to VLA maps.
The appropriate correction at large distances from the pointing position
is not well-determined, thus
PBCOR ``blanks'' the map pixel values beyond a certain radius (see
{\it blanked pixel}\/).

\aa{primary data file}
in AIPS, either a \uv {\it data file}, containing measurements of
the visibility function of a radio source, or an {\it image
file}, containing a digitized image or a radio map.
Compare {\it extension file}.

\aa{principal solution}
in the context of radio interferometry,
the inverse Fourier transform of the {\it u-v measurement distribution};
i.e., the {\it dirty map} \qv\ in sense 1 of the definition.
This term was introduced by R.~N.~Bracewell and J.~A.~Roberts
[Aerial smoothing in radio astronomy, {\it Austr.\ J.~Phys.},
\bold{7} (1954) 615--640].
Except in the trivial case, the principal solution to the mapping
problem in interferometry is a physically implausible solution,
because the principal solution has not the property of {\it compact support}.
\par
An {\it invisible distribution} \qv\ added to the principal solution
yields another solution---i.e., another
brightness distribution which is consistent with the observations.

\aa{procedure} See {\it POPS symbols}.

\aa{prolate spheroidal wave function}
an eigenfunction of the finite, or truncated, Fourier transform---%
more precisely, for given $c$, one of the countably many solutions
of the integral equation
$$\nu f(\eta)=\int_{-1}^1e^{ic\eta t}f(t)\,dt\,;$$
equivalently, a solution of the differential equation
$(1-\eta^2)f^{\prime\prime}-2\eta f^\prime+(b-c^2\eta^2)f=0$;
or, equivalently, a solution of the wave equation in a system
of prolate spheroidal coordinates.
The eigenfunction of the above equation associated with the largest
eigenvalue $\nu$ is termed the 0-order solution.
\par
If we want a gridding convolution function $C$,
of {\it support width} equal to the width of $m$ grid cells,
that is optimal in the sense that its Fourier transform $\hat C$
has the property that
the concentration ratio
$$\intii_{\operatorname{map}} |\hat C(x,y)|^2\,dx\,dy\over
\int_{-\infty}^\infty\int_{-\infty}^\infty|\hat C(x,y)|^2\,dx\,dy$$
is maximized, then $C$ is the separable product of two
0-order prolate spheroidal wave functions, with $c=\pi m/2$.
See {\it gridding convolution function} and {\it spheroidal function}.

\aa{prompt character}
a character (often the dollar sign ``\$'' or the greater-than
sign ``$>$'') which the computer program or the operating system
prints on the terminal screen of the interactive user in
order to prompt, or invite, a typed response from the user.
The AIPS program's standard prompt character is the greater-than
sign, and on the Vaxes at the NRAO the operating
system's prompt character is the dollar sign.  On most UNIX systems,
the prompt character is the percent sign.  Thus, most commands (or
{\it POPS symbols}\/) peculiar to AIPS must be typed on a line
beginning with the $>$-character, and any command to the
operating system, such as the command to mount a tape,
must be typed on a line beginning with the \$- or \%-character.
\par
When operating in some lesser-used, special modes,
AIPS employs other prompt characters:
``:'' for procedure building, ``;'' for procedure editing,
``!'' for entry of gripes, ``$<$'' for batch file preparation,
and ``\#'' for parameter reading.

\aa{Prussian helmet Clean algorithm}
a modified version of the H\"ogbom Clean algorithm, devised by
Tim Cornwell.
The idea is to drive the Clean algorithm toward an approximate solution
$f$ of minimal Euclidean norm---%
i.e., to find an $f$ consistent with the data, confined to
the Clean window, comprised of a small number of point components,
and such that $\intii_{\operatorname{Clean}\atop\operatorname{window}}
[f(x,y)]^2\,dx\,dy$ is minimized.
This is accomplished by adding a $\delta$-function of amplitude
$\omega$, centered at the origin, to the dirty beam, and then just
proceeding as normal with the Clean algorithm.
Proper choice of $\omega$ depends on the
distribution of measurement errors.
See [T.~J.~Cornwell,
A method of stabilizing the Clean algorithm,
{\it Astron.\ Astrophys.}, \bold{121} (1983) 281--285].
A provision for this modification is incorporated in the AIPS tasks
APCLN and MX.
See {\it regularization method}.

\aa{pseudo-AP} See {\it pseudo-array processor}.

\aa{pseudo-array processor}
in AIPS, the term which is applied to a collection of
Fortran subroutines which may be used to emulate the operation
of an FPS Model AP--120B array processor.
At those AIPS sites which do not have an array processor,
the AIPS tasks which normally would make use of an
array processor use the pseudo-array processor subroutines instead.
See {\it array processor}.

\aa{pseudo-color display}
In digital imagery, a {\it pseudo-color} display
is one which is derived from a single real-valued function $f(x,y)$
and a mapping $\bold R^1\to\bold R^3$ that controls
the {\it hue}, {\it intensity}, and {\it saturation}---or,
equivalently, the proportions in an additive mixture of three
primary hues---of the coloration at each {\it pixel} coordinate $(x,y)$
of the display, according to the value of $f(x,y)$.
A pseudo-color display might be used, for example, to represent
measurements of the intensity of the radio continuum flux density
of a source.
\par
Compare {\it false color display} and see {\it color contour display}.

\aa{pseudo-continuum u-v data file}
in VLA spectral line data reduction, a {\it u-v data file}
containing the visibility measurements from a small
number of spectral line channels, recorded in the same format
as continuum visibility data.
The purpose is to enable the use, for spectral line data analysis,
of programs originally intended only to handle continuum
data reduction.

\aa{pseudoverb} See {\it POPS symbols.}

\aa{PSF} See {\it point spread function}.

\aa{Q-routine}
in AIPS, a primitive level subroutine designed to function on a
particular manufacturer's production model of an {\it array processor}.
A goal of the AIPS project is to construct libraries of Q-routines---%
one library appropriate to each model of array processor which might
be used in conjunction with AIPS---%
with identical names, argument lists, and functionality.
Existing Q-routines emulate the standard library of Floating Point
Systems, Inc.'s, model {\it AP--120B array processor}.

\aa{quick boot} an abbreviated boot procedure.  See {\it boot}.

\aa{RANCID}
(Real (or Radio) Astronomical Numerical Computation and Imaging Device)
the name by which the AIPS data reduction system formerly was known.

\aa{re-boot}
Having booted once already, one {\it re-boots}.  See {\it boot}.

\aa{regularization method}
in the numerical solution of ill-posed problems, given noisy data,
a method in which the original problem is converted into
a well-posed problem by requiring of the solution
to the modified problem (which now is an approximate solution to
the original problem)
that it satisfy some smoothness constraint.
The prototypical ill-posed problem has the form
$Kf=g+\epsilon$, where $K$ is a known linear integral operator (e.g., a
convolution operator), where $g+\epsilon$,
which is given, represents some noisy measurement,
and where $f$ is unknown.
In the context of radio interferometry, one may take $g+\epsilon$
to be the {\it dirty map} and $K$ to be the operator which
convolves the ``true'' radio source brightness distribution $f$
with the {\it dirty beam}.
Now, denoting our approximate solution to the ill-posed
problem by $\s f$, $\s f$ is found by minimizing the expression
$$(1-\lambda)\|g-K\s f\|^2+\lambda S(\s f)\,,$$
for some given choice of the {\it regularization parameter}
$\lambda$, $0<\lambda<1$.
$\|g-K\s f\|^2$ is the mean squared residual (occasionally
some other measurement of the error is used), and $S(\s f)$ is
a measure of the roughness of the computed solution---%
say, some power of a norm or seminorm of $\s f$, or a similar quantity,
such as the negative of the (Shannon) entropy of $\s f$.
\par
Proper choice of $\lambda$ must be based on statistical
considerations which depend on the distribution of measurement
errors; often, one chooses $\lambda$ in order
achieve an {\it a priori} reasonable value of the mean squared residual.
The {\it maximum entropy method, Tikhonov regularization}, and
the {\it Prussian helmet Clean algorithm}
are special cases of the regularization method.
Appropriate choice of $S$ is discussed by J.~Cullum
[The effective choice of the smoothing norm in
regularization, {\it Math.\ Comp.}, \bold{33} (1979) 149--170],
and the choice of $S$ and $\lambda$, by a statistical method known
as ``cross validation'', is described by G.~Wahba
[Practical approximate solutions to linear operator
equations when the data are noisy, {\it SIAM J.\ Numer.\ Anal.},
\bold{14} (1977) 651--677].
Often, some Sobolev norm is chosen for $S$.
\par
Usually, in addition to the smoothness constraint, $f$ is assumed
to be of known, {\it compact support}.
Other constraints, such as nonnegativity, may be included as well.
In the case in which the data are exact---%
i.e., when $\epsilon=0$, so that $g=Kf$---%
one may obtain the regularized solution corresponding to $\lambda=0$
as the limit of regularized solutions $\s f_\lambda$ as $\lambda\to0$.
See {\it Variational Method}.
Also see D.~M.~Titterington [General structure of regularization
procedures in image reconstruction, {\it Astron.\ Astrophys.},
\bold{144} (1985) 381--387].

\aa{regularization parameter}
in the {\it regularization method} \qv\ for the solution of ill-posed
problems, a smoothing parameter $\lambda$, $0<\lambda<1$, which controls
the trade-off between an error term, measuring agreement of the
computed solution $\s f$ with the given data,
and a term $S(\s f)$, which measures the roughness of $\s f$.
I.e., $\lambda$ controls the amount of ``regularization''.
See {\it super-resolution.}

\aa{re-IPL}
same as {\it re-boot}.

\aa{residual delay}
Expressing the {\it antenna/i.f. phase}, $\psi_k$,
for antenna $k$ of a VLBI array as a function of frequency
as well as of time,
the residual delay on the $i$--$j$ baseline at $(t_0,\nu_0)$
is given by
$\tau_{ij}\equiv\left.{\partial(\psi_i-\psi_j+\phi_{ij})\over
\partial\nu}\right|_{(t_0,\nu_0)},$
where $\phi_{ij}$ denotes the visibility phase on the $i$--$j$
baseline.
(The partial w.r.t.\ $t$ is called the {\it residual fringe rate}.)
Usually the major contributor to residual delay is
the difference in the station clock errors.
The residual delay is a group delay, rather than a phase delay.
It is termed residual because it is assumed that
geometric effects have already been compensated for.
\par
The ``antenna components'' of $\tau_{ij}$, namely
$\tau_k\equiv\left.{\partial\psi_k\over\partial \nu}\right|_{(t_0,\nu_0)}$,
are called the {\it antenna residual delays}.
They are among the solution parameters of the global fringe fitting
algorithm for VLBI.
See {\it residual fringe rate} and {\it global fringe fitting algorithm}.

\aa{residual fringe rate}
Expressing the {\it antenna/i.f. phase}, $\psi_k$,
for antenna $k$ of a VLBI array as a function of frequency
as well as of time,
the residual fringe rate on the $i$--$j$ baseline at $(t_0,\nu_0)$
is given by
$r_{ij}\equiv\left.{\partial(\psi_i-\psi_j+\phi_{ij})\over
\partial t}\right|_{(t_0,\nu_0)},$
where $\phi_{ij}$ denotes the visibility phase on the $i$--$j$
baseline.
(The partial w.r.t.\ $\nu$ is called the {\it residual delay}.)
Usually the major contributor to residual fringe rate is
the drift of the station clocks.
\par
The ``antenna components'' of $r_{ij}$,
namely $r_k\equiv\left.{\partial\psi_k\over\partial t}\right|_{(t_0,\nu_0)}$,
are called the {\it antenna residual fringe rates}.
They are among the solution parameters of the global fringe fitting
algorithm for VLBI.
See {\it residual delay} and {\it global fringe fitting algorithm}.

\aa{resolution}
See {\it spatial resolution}. % and {\it contrast resolution}.

\aa{restoring beam} same as {\it Clean beam}.

\aa{roam} See {\it TV roam}.

\aa{run file}
in AIPS, a {\it text file} written by an AIPS user
and containing a sequence of AIPS commands ({\it POPS symbols}\/).
Run files are useful for the storage of strings of commands which
one might wish to execute repeatedly (in particular, for the
storage of lengthy {\it procedures}\/).
The run files for all users at a particular AIPS installation
are stored in a common area.
These files ordinarily are created through use of
one of the standard {\it text editors} of AIPS' host computer.

\aa{sampling theorem} See {\it Shannon sampling theorem}.

\aa{saturation}
one of the three basic parameters ({\it hue}, {\it intensity},
and {\it saturation}\/)
which may be used to describe the physical perception of color.
Saturation is a measure of the (perceived) narrowness of the
color spectrum, or the difference of the hue from a gray
of the same intensity.
Neutral gray---or a ``white'' spectrum---is termed 0\%\ saturated,
and a monochromatic spectrum is termed 100\%\ saturated.
\par
See {\it C.I.E. chromaticity diagram}.

\aa{scratch}
1. The act of deleting a data file---i.e., surrendering
the storage medium space which that
file occupies---is termed {\it scratching} the data file.
Use of the term {\it delete} may be preferable,
but {\it scratch} is more common among AIPS users.
One who is about to delete a data file
may wish first to create a back-up copy.
See {\it back-up}.
\xspace2. an adjective meaning {\it temporary},
as in {\it scratch file.}
\par
In AIPS a primary data file and all of its associated
extension files can be deleted by means of the
verb {\sl ZAP}.

\aa{scratch file}
a data file intended for temporary storage
(esp., of data which represent intermediate results---%
i.e., {\it scratch}\/work).
Many of the AIPS tasks use scratch files;
the necessary scratch files are created and destroyed automatically
by the tasks.
However, when an AIPS task {\it crashes,} sometimes a scratch file remains.

\aa{screen editor}
a {\it text editor} \qv\ which, unlike a {\it line editor}, allows
the simultaneous modification of more than one line or record
within a text file.  For example, a mechanism to facilitate alignment
of margins often is incorporated by a screen editor.
{\it EDT}, {\it EVE}, {\it vi} and {\it EMACS} are screen editors.

\aa{scroll} See {\it terminal scroll} and {\it TV scroll}.

\aa{self-calibration algorithm}
Many of the systematic errors affecting interferometer visibility
measurements may be assumed to be multiplicative and ascribable
to individual array elements.
That is, in an $n$ element array, the observations on the
$n(n-1)/2$ baselines are afflicted by $n$ sources of
systematic error, the so-called {\it antenna/i.f.\ gains} $g_k(t)$.
Given a rough estimate of the true source visibility, a model
obtained, say, by mapping and Cleaning roughly calibrated data,
one may solve for the unknown gains---and it is not unreasonable
to do so, because there are $(n-1)/2$ times more observations
than antenna gains.
The number of degrees of freedom can be held further in check
by assuming that the $g_k(t)$ are slowly-varying or that
they are of unit modulus (i.e., that no amplitude errors are
present), or by designing an array with redundant spacings.
\par
Having once solved for the unknown $g_k$, one may correct the
data, make another map, and repeat the process.
This iterative scheme, which yields successive approximations
to the true radio source brightness distribution, is known as
{\it self-calibration}.
Self-calibration is essentially identical to the technique of
{\it hybrid mapping}, which is widely used in VLBI.
See {\it self-calibration gain solution algorithm}; also see Tim Cornwell
and Ed Fomalont's
Lecture No.~9 in the \sira\ and the review paper by T.~J.~Pearson
and A.~C.~S.~Readhead [Image formation by self-calibration in radio
astronomy, {\it Ann.\ Rev.\ Astron.\ Astrophys.}, \bold{22} (1984)
97--130].

\aa{self-calibration gain solution algorithm}
In self-calibration,
the unknown {\it antenna/i.f.\ gains} $g_k(t)$ may be approximated
by minimizing a functional $S(g_1,\dots,g_n)$
given by a weighted discrete $l^p$ norm of the residuals:
$$S(\bold g)=
\left(\sum_{1\le i<j\le n}w_{ij}\left|\s V_{ij}-g_i\overline g_j
V_{ij}\right|^p\right)^{1/p}\,.$$
Here $\s V_{ij}$ is the visibility measurement obtained on the $i$--$j$
baseline (at a given instant),
$V_{ij}$ is the corresponding {\sl model} visibility,
and $w_{ij}$ is a suitably chosen weight.
Usually the $g_k$ may be assumed not to vary too rapidly with
time, so that one may minimize, instead, the functional
$$S(\bold g)=
\left(\sum_{1\le i<j\le n}w_{ij}\left|\left<\s V_{ij}/V_{ij}\right>
-g_i\overline g_j\right|^p\right)^{1/p}\,,$$
where $\left<\s V_{ij}/V_{ij}\right>$ is the time-average of the
ratio of observed visibility to model visibility, over a time
period during which the $g_k$ may be assumed constant.
\par
The AIPS implementation allows the choices $p=1$ and $p=2$.
Choosing $p=2$ yields the least-squares solution for $\bold g$.
When one chooses $p=1$, so that a weighted sum of the moduli
of the residuals is minimized, the computed gain solutions
are less influenced by wild data points, but there is some loss
of statistical efficiency---i.e., the least-squares solutions
are superior when the distribution of measurement errors is
well-behaved.
(Probably the choice $p\simeq1.2$ would offer a better compromise
between efficiency and robustness).
See [F.~R.~Schwab, Robust solution for antenna gains,
VLA Scientific Memo.\ No.~136] for further details.
\par
One may wish to solve only for the {\it antenna/i.f.\ phases}
$\psi_k(t)$ rather than for the $g_k$
if, for example, atmospheric phase corruption is believed to
be the dominant source of systematic error.
In this case, one minimizes
$$S(\Psi)=
\left(\sum_{1\le j<k\le n}w_{jk}\left|\s V_{jk}-
e^{i(\psi_j-\psi_k)}
V_{jk}\right|^p\right)^{1/p}\,,$$
or the version thereof incorporating time-averages.
\par
Cornwell and Wilkinson
[A new method for making maps with unstable radio interferometers,
{\it Mon.\ Not.\ R.\ Astr.\ Soc.}, \bold{196} (1981) 1067--1086]
suggest adding to $S$ terms which arise by assuming prior
distributions for the $g_k$;
these ``penalty terms'' would be chosen so as to increase in magnitude
as the solution parameter deviates from a prior mean
which one might take, say, as the running mean of previous gain solutions.
The widths of the prior distributions could be based on empirical
knowledge of the behavior of the array elements.
Such a modification can be useful when the
array is composed of antenna elements of differing collecting area.
This modification is used in order to constrain the moduli
of the computed gains in one version of the AIPS task for
self-calibration which is used primarily for VLBI data reduction
(VSCAL).

\aa{Shannon sampling theorem}
Suppose the complex-valued function $f$ of the real variable $t$ to be
square-integrable, and assume that $f$ is band-limited; i.e., that its
Fourier transform
$\hat f(x)=\int_{-\infty}^\infty f(t)e^{2\pi ixt}\,dt\equiv0$ for $|x|>a$.
Then $f$ is completely determined by its values at the discrete
set of sampling points $n/2a$, $n=0,\pm1,\pm2,\ldots\,$, and $f$ can be
recovered via the {\it Shannon series} (also called the cardinal series)
$$f(t)=\sum_{n=-\infty}^{\infty}f\left(n\over2a\right)
{\sin2\pi a(t-n/2a)\over2\pi a(t-n/2a)}\,.$$
The series converges both uniformly and in the mean-square sense.
\par
The Shannon series can be derived by expanding $\hat f$ in a
Fourier series, and then applying Fourier inversion---%
or it can be derived from the classical Poisson summation formula.
It is sometimes referred to as Whittaker's cardinal interpolation formula
or the Whittaker--Shannon sampling series, having first been studied in
detail by E.~T.~Whittaker in 1915 and later introduced into the
literature of communications engineering by Shannon in 1949.
By the {\it Paley--Wiener theorem}\/, since $f$ is band-limited, it can be
analytically extended from the real line to the full complex plane,
as an entire function of slow growth.
The Shannon series, which converges for complex as well as real $t$,
is one means of doing so.
Whittaker referred to the series as ``a function of royal blood in the
family of entire functions, whose distinguished properties separate it
from its bourgeois brethren.''
\par
Suppose that $f(t)$ is ``small'' for $|t|>b$
(no nontrivial signal is both band-limited and time-limited).
Then, assuming that $b$ is integral, the number of terms in
the Shannon series that really matter is $4ab$.
This suggests that the space of ``essentially band-limited''
and ``essentially time-limited'' signals has dimension
equal to the {\sl time-bandwidth product} $4ab$.
The precise sense in which this is so, together with a discussion
of the {\it prolate spheroidal wave functions} \qv, which are relevant
to the problem, is described by H.~Dym and H.~P.~McKean
[{\it Fourier Series and Integrals}, Academic Press, New York, 1972]
and by David Slepian [Some comments on Fourier analysis, uncertainty
and modeling, {\it SIAM Rev.}, \bold{25} (1983) 379--393].
\par
The multi-dimensional extension of the sampling theorem to rectangles
implies that if an ``unconfused'' radio source $f(x,y)$ is confined to a
small region of sky $|x|<x_0$, $|y|<y_0$ (radians), then it can be
reconstructed unambiguously from a discrete set of visibility samples
$\hat f(m\Delta u,n\Delta v),\ {m,n}=0,\pm1,\pm2,\ldots\,,$
with $\Delta u=1/2x_0$ and $\Delta v=1/2y_0$ wavelengths.
See {\it cellsize} and {\it Nyquist sampling rate}.
Other useful extensions of the sampling theorem---for example,
to various multi-dimensional sampling configurations (e.g., 2--D
hexagonal sampling lattices), to the case of stochastically jittered sampling,
to derivative sampling (e.g., in 1--D, $f$ can be recovered from samples of $f$
and its derivatives through order $r$ taken at intervals $(r+1){n\over2a}$),
etc.---and sampling theorems for functions whose transforms of other
than Fourier type are of {\it compact support}\/---%
are described in survey articles by A.~J.~Jerri
[The Shannon sampling theorem---its various extensions and applications:
a tutorial review, {\it Proc.~IEEE}, \bold{65} (1977) 1565--1596]
and J.~R.~Higgins [Five short stories about the cardinal series,
{\it Bull.\ (New Ser.) Amer.\ Math.\ Soc.}, \bold{12} (1985) 45--89].

\aa{Shannon series}
See {\it Shannon sampling theorem}.

\aa{shed} See {\it sub-task}.

\aa{SIVPSF} See {\it point spread function}.

\aa{slice}
a one-dimensional cut across an {\it image}.
E.g., the slice of a two-dimensional image $f$
which passes through $(x_0,y_0)$ and has orientation angle $\phi$
is the {\it subimage} $h$ given by $h(t)=f(x_0+t\cos\phi,y_0+t\sin\phi)$.
In AIPS, a slice may be excised from an image
by issuing the {\it verb} command SLICE.
Since AIPS deals only with digitized images, the program must
interpolate to obtain data along the cut,
except when the slice is taken along a row or column of the image.

\aa{slice file}
in AIPS, an {\it extension file}, associated with an {\it image file},
in which a digitized {\it slice} \qv, or one-dimensional subimage, of the
primary image is stored.
In order to display a slice, one may issue the {\it verb}
command SL2PL, which causes AIPS to read the contents of a slice file
and generate a {\it plot file}.

\aa{snapshot}
in earth-rotation aperture synthesis interferometry,
an observation which is of such short duration that Earth's motion
does not significantly enhance the \uv {\it coverage},
or a {\it map} derived from such a brief observation.
Compare {\it full-synthesis map}.
\par
For a thorough discussion of the use of the VLA in snapshot mode,
see \S 5 of A.~H.~Bridle's Lecture No.~16 in the \ssp.

%\aa{Sobolev norm}

\aa{software mount}
a computer's reaction to
the issuing of a command to it informing it that
the {\it hardware mount} of some external storage module,
such as a disk pack or a reel of magnetic tape,
has occurred, and that the computer should open the channel
of access to this module.
See {\it hardware mount.}

\aa{sort order}
the ordering of visibility measurements within a {\it u-v data file}.
{\it Time-baseline order} is convenient for purposes of calibration,
{\it baseline-time order} for data display,
and so-called {\it x-y order} for gridding and subsequent mapping.

\aa{source editor} same as {\it text editor}.
(Formerly, computers were used mainly for numerical
computations and text editors primarily for the editing
of program source code---hence the name {\it source editor}\/).

\aa{spatial resolution}
In digital image analysis, this term refers rather imprecisely
to the minimum size of details which can be discerned.
The spatial resolution is determined by three factors:
the inherent indeterminacy of whatever {\it image reconstruction}
problem underlies the method by which the image was produced
(and the properties of the image reconstruction
{\sl algorithm} which produced the image); the measurement noise;
and the {\it pixel} size---i.e., the size of the squares or
the rectangles comprising the reconstruction matrix.
\par
In radio interferometry, the inherent spatial resolution
goes roughly in inverse proportion to the physical size scale $D$ of the
array (measured in wavelengths).
For observations at a wavelength $\lambda$, the inherent
spatial resolution, with a filled aperture, is essentially
$\lambda/D$ radians.
However, with a synthesis array with large gaps
in the {\it u-v coverage}, the effective resolution is somewhat coarser.
Often, some measure of the spread of the central lobe of the {\it dirty beam}
(say, the FWHM) is quoted as the spatial resolution.
However, some reconstruction methods (e.g., the {\it regularization
methods}) produce images in which the resolution of bright features
may be much finer than that of dim features.
This property of regularization methods may be viewed as either
good or bad: $S/N$ dependent spatial resolution complicates
the interpretation of an image, but, on the other hand,
one may gain additional contrast resolution---%
i.e., low surface-brightness features may become more readily discernible.
An honest statement concerning the spatial resolution of an image must be based
upon empirical knowledge of the reconstruction method that was used.
See {\it super-resolution}.

\aa{spawn} See {\it sub-task}.

\aa{spheroidal function}
an eigenfunction $\psi_{\alpha n}$ of a finite,
weighted-kernel Fourier transform---%
more precisely, for given $c$ and given $\alpha>-1$,
one of the countably many solutions of the integral equation
$$\nu f(\eta)=\int_{-1}^1e^{ic\eta t}(1-t^2)^\alpha f(t)\,dt\,;$$
equivalently, a solution of the differential equation
$(1-\eta^2)f^{\prime\prime}-2(\alpha+1)\eta f^\prime+(b-c^2\eta^2)f=0$.
The eigenfunction $\psi_{\alpha 0}$ of the equation above associated
with the largest eigenvalue $\nu$ is termed the 0-order solution.
The choice $\alpha=0$ of weighting exponent
yields the family $\leftset\psi_{0n}\mid n=0,1,2,\dotsc\rightset$
of prolate spheroidal wave functions.
\par
Weighted 0-order spheroidal functions
$(1-\eta^2)^\alpha\psi_{\alpha 0}$ are optimal
gridding convolution functions in the same sense that
the {\it prolate spheroidal wave functions} \qv\ are optimal,
except that now the weighted concentration ratio
$$\intii_{\operatorname{map}}|\hat C(x,y)|^2
(1-(2x\Delta u)^2)^\alpha (1-(2y\Delta v)^2)^\alpha\,dx\,dy\over
\int_{-\infty}^\infty\int_{-\infty}^\infty|\hat C(x,y)|^2
|1-(2x\Delta u)^2|^\alpha |1-(2y\Delta v)^2|^\alpha\,dx\,dy$$
is maximized
%(see [F.~R.~Schwab, Optimal gridding, VLA Scientific Memo.\ No.~132]).
(see the paper by F.~R.~Schwab in the \sydp).
The weighting exponent $\alpha$ is used to trade off
the effectiveness of the aliasing suppression at the edge
of the field of view, against that in the central region of the map.
The choice $\alpha=1$, with a {\it support width} of six \uv grid cells,
yields an effective gridding convolution function,
emphasizing aliasing suppression in the central region of the map;
this function, $\psi_{10}$, with $c=3\pi$, is the default function used in the
AIPS mapping program.
See {\it gridding convolution function}.

\aa{Stokes' parameters}
the four coordinates relative to a particular basis
for the representation of the polarization state
of an electromagnetic wave propagating through space.
Consider a wave propagating along the
$z$-direction in a right-handed $(x,y,z)$ Cartesian coordinate system.
At a fixed point in space, let the instantaneous components of the
electric field vector, in the $x$-  and $y$-directions, be denoted
by $E_x(t)$ and $E_y(t)$, respectively; and assume them to be
stationary (in the weak sense, and square-integrable)
stochastic processes.
Form the matrix
$$S=\left( \begin{array}{ll}
             \langle E_x(t)\overline{E}_x(t+\tau)\rangle^\hatsymbol
           & \langle E_x(t)\overline{E}_y(t+\tau)\rangle^\hatsymbol\\
             \langle E_y(t)\overline{E}_x(t+\tau)\rangle^\hatsymbol
           & \langle E_y(t)\overline{E}_y(t+\tau)\rangle^\hatsymbol
            \end{array}\right)\,.$$
Here, the bracketed expressions are expectation values, or correlation
functions, in the lag variable $\tau$,
and $\ {}^\hatsymbol\ $ denotes Fourier transform with respect to $\tau$.
Thus each element of $S$ is a function of frequency $\nu$.
$S$ is Hermitian (conjugate symmetric), owing to the stochasticity assumptions.
The three Pauli spin matrices, together with the $2\times2$ identity
matrix, form a basis for the algebra of $2\times2$ Hermitian matrices;
i.e., each such matrix $S$ can be represented in the form
\begin{eqnarray*}
  S(\nu) & = & \sigma_1(\nu)\left( \begin{array}{cc}
                   1&0\\0&1\end{array}\right)
        +\sigma_2(\nu)\left(\begin{array}{cc} 1&0\\0&-1\end{array}\right)\\
       & &+\sigma_3(\nu)\left(\begin{array}{cc} 0&1\\1&0\end{array}\right)
        +\sigma_4(\nu)\left(\begin{array}{cc}
       0&i\\-i&0\end{array}\right)\,.
\end{eqnarray*}
The four (real) coefficients, $\sigma_1,\dots,\sigma_4$, of the representation
of $S$ in this basis are called Stokes' parameters.
They commonly are denoted by $I(\nu)$, $Q(\nu)$, $U(\nu)$, and $V(\nu)$,
respectively.
In other words,
$$S(\nu)=\left(\begin{array}{cc} I(\nu)+Q(\nu) & U(\nu)+iV(\nu)\\
                  U(\nu)-iV(\nu)& I(\nu)-Q(\nu) \end{array}\right)\,,$$
with $I$, $Q$, $U$, and $V$ real.
\par
Stokes' parameter $I$ measures the total intensity of the radiation
field, $Q$ and $U$ the linearly polarized intensity, and $V$ the
circularly polarized intensity.
$I$ always is nonnegative.
For a totally unpolarized wave, $Q=U=V=0$; for a partially polarized
wave, the ratio $\sqrt{Q^2+U^2+V^2}/I$ measures the total
degree of polarization,
$\sqrt{Q^2+U^2}/I$ the degree of linear polarization,
and ${1\over2}\arctan{U\over Q}$ the orientation angle
of the linearly polarized component.
$Q+iU$ is called the complex linear polarization.
The IAU and IEEE orientation/sign conventions have the $z$-axis
directed toward the observer, the $x$-axis directed north,
and a $+i$ in the argument of the exponential kernel of the FT.
Positive $V$ corresponds to right circular polarization, and conversely.
The polarization response of an interferometer can be described
by forming the so-called cross-spectral density matrix, which is like
the $S$ above but is formed from measurements of the electric field
taken at two points in space.
For further details, including a description
of the polarization response of an interferometer,
for various feed configurations, see Carl Bignell's
Lecture No.~6 in the \wsp.

\aa{Stokes' visibility functions}
Stokes' visibility functions, $V_I$, $V_Q$, $V_U$, and $V_V$,
are the Fourier transforms (FT's) of the radio brightness (spatial)
distributions of {\it Stokes' parameters},
$I(x,y)$, $Q(x,y)$, $U(x,y)$, and $V(x,y)$.
(Here, $V_I=\hat I$, $V_Q=\hat Q$, etc., where $\ \hat {}\ $ denotes FT.)
\par
For a radio interferometer with ideal circularly polarized feeds,
the relations between Stokes' visibility functions
and the visibilities, $V_{RR}$, $V_{LL}$, $V_{RL}$, and $V_{LR}$,
obtained by correlating right circular response with right,
left with left, etc., are
$V_I={1\over2}(V_{RR}+V_{LL})$, $V_Q={1\over2}(V_{LR}+V_{RL})$,
$V_U={i\over2}(V_{LR}-V_{RL})$, $V_V={1\over2}(V_{RR}-V_{LL})$.
Note that each of Stokes' visibility functions is {\it Hermitian}.
On the assumption that circular polarization is absent
(i.e., that $V(x,y)\equiv0$), $V_{RR}$ is equal to $V_{LL}$,
and both are Hermitian.
\par
Components of the systematic errors affecting visibility
measurements are i.f.-dependent;
hence VLA \uv data files usually do not contain Stokes'
visibilities, but rather $V_{RR}$, $V_{LL}$, $V_{RL}$, and $V_{LR}$---%
as these are what is required for calibration purposes.
Stokes' visibility functions generally are constructed only within
the mapping programs.
(But the AIPS visibility data format is designed to accommodate
either type of visibility function, and the mapmaking tasks are
able to recognize the form of their input data and deal with them
appropriately.)

\aa{subimage}
in AIPS parlance, any linear, rectangular, or hyper-rectangular
section of an {\it image}.

\aa{sub-task}
a task, or computer program, whose execution is initiated by
the action of another program.
The act of initiating the execution of the sub-task is called
{\it task shedding} or {\it task spawning}.
See {\it task}.

\aa{super-resolution}
The problem of image reconstruction in radio interferometry
is one of finding an approximation to
an unknown function $f$ (generally assumed to be of {\it compact
support}\/) from {\sl partial knowledge} of its
Fourier transform $\hat f$ --- i.e., from a finite number of
measurements of the visibility.
Any of the techniques which are applied to the problem---%
the \hca\/, the {\it regularization method}\/, etc.---%
may be thought of as methods of smoothing, interpolating, and
extrapolating the noisy measurements.
{\it Super-resolution} is a term which refers to the extrapolation
aspect:
Cautious extrapolation yields an image whose {\it spatial resolution}
is $\approx\lambda/D$, where $D$ is the diameter of the largest
centered region in the \uv plane which has been reasonably well sampled.
Less cautious extrapolation yields super-resolution;
spurious detail appears as caution is abandoned.
\par
Super-resolution in a {\it Clean map} is effected by choosing
an artificially narrow {\it Clean beam}.
With regularization methods (in image reconstruction, and more generally),
super-resolution comes about
by choosing a small value of the {\it regularization parameter}.
The spatial resolution achieved by a regularization method
may be signal-to-noise dependent---bright features may be
super-resolved, and dim ones not.

\aa{support}
The closure of that subset of the domain of definition of a function $f$
(or of a generalized function, or distribution)
on which the function assumes a nonzero value is called the
{\it support} of the function, and is denoted by $\operatorname{supp}(f)$.
I.e., $\operatorname{supp}(f)=\overline{\leftset x\mid f(x)\noteq0\rightset}$.
\par
For example, the support of the function $f(x)=x$ is the
whole real line, even though $f(0)=0$.
And the support of
$$f(x,y)=\left\{ \begin{array}{ll}
1\,,&x^2+y^2<1\,,\\0\,,&\text{otherwise}\,,
\end{array}\right.$$
is the {\it closed} unit disk, $\leftset(x,y)\mid x^2+y^2\le1\rightset$.
\par
In Euclidean space, a function $f$ whose support is bounded---%
i.e., such that $f\equiv0$ ``far-out''---%
is said to be of {\it compact support}.
The Fourier transform of a nontrivial function of compact support
(such as a \uv {\it measurement distribution}
or a {\it gridding convolution function}\/)
cannot itself be of compact support;
i.e., it has ``sidelobes'' extending to infinity.

\aa{support width}
of a function whose {\it support} is a rectangle or
a hyper-rectangle (e.g., the Fourier transform of a band-limited function),
the linear measure of one of the edges of its {\it support.}

\aa{SVPSF} See {\it point spread function}.

\aa{Synthesis Imaging in Radio Astronomy}
A collection of lectures from the 1988 (Third) NRAO Synthesis Imaging
Summer School edited by R.\ A.\ Perley, F.\ R.\ Schwab and A.\ H.\ Bridle.
(Astronomical Society of the Pacific Conference Series, Volume~6 (1989)).
A very useful reference book for the reduction of radio interferometric data.
This volume supersedes the proceedings from the earlier workshops.

\aa{synthesized beam}
in radio interferometry, the {\it beam}---but always ignoring
instrumental effects.
Hence, the synthesized beam is fully determined by the
{\it u-v sampling distribution},
the {\it u-v weight function}, the {\it u-v taper function},
and the {\it gridding convolution function}.
See {\it beam}.

\aa{tape blocking efficiency}
Data are stored on magnetic tape in units of {\it blocks}.
An {\it inter-record gap}\/---essentially wasted space---%
separates one block from the next.
The {\it tape blocking efficiency}, or the fraction of unwasted space,
is the ratio
\def\bl{\operatorname{block\ length}}
\def\rd{\operatorname{recording\ density}}
$${ {\bl\over\rd}\over{\bl\over\rd}+
\text{length of an inter-record gap} }\,.$$
The length of an inter-record gap is
about ${3\over4}$, ${3\over5}$, and ${3\over10}$ inch
at recording densities of 800, 1600, and 6250 bpi, respectively.

%\aa{tape label}

\aa{taper} See {\it u-v taper function}.

\aa{task}
used in two senses: 1) the execution of a computer program
and 2) the program itself.
Thus, if two computer users are (independently) running the
same program at the same time, it may be said either that
two tasks are running, or that two incarnations of the same
task are in existence.
A {\it sub-task} \qv\ is a task whose execution is initiated
by the action of another program.
Many of the
more complicated and the more specialized functions of AIPS are
accomplished by the action of sub-tasks shed
by the AIPS program.
(Simpler functions are invoked by the issuance
of {\it verb} commands---see {\it POPS symbols.})

\aa{t-b order} See {\it time-baseline order}.

\aa{TEK screen}
a cathode ray tube (CRT) terminal and display device appropriate for
pictorial display of data, in the form of contour plots, graphs, etc.,
as well as for display of textual data.
The Tektronix company's Model 4012 terminal
(with a green P4 phosphor, hence the synonymous term {\it green screen}\/)
is the canonical device of this type.
The ``make copy'' button on this device can be used to produce
a copy, on paper, of the image shown on the CRT screen.
Each of the NRAO's AIPS data reduction computers is outfitted
with a {\it TEK screen}.

\aa{TEK4012} same as {\it TEK screen}.

\aa{Telex 6250 tape drive}
a model of tape drive used on the VLA Vaxes,
capable of operation at 1600 and 6250\,bpi.

\aa{terminal page}
Many modern computer terminals contain a semiconductor memory
with a capacity of several CRT screen loads ($\approx24$ lines)
of character data.
A {\it terminal page} is a unit of one screen load of such data.
Certain terminal keys allow one to cause data which previously
appeared on the CRT screen to reappear---this feature is
called {\it terminal scroll} \qv.
A typical terminal at the NRAO has three terminal pages of memory.

\aa{terminal scroll}
that feature present on certain models of computer terminals
which allows data which previously appeared on the CRT
screen to be made to reappear.
Often, depressing one key on the terminal will cause earlier
information to reappear line-by-line (this is termed {\it line
scroll}\/), while the action of another key
will cause a whole earlier screen load to reappear
(this is termed {\it page scroll}\/).

\aa{text editor}
a computer program designed for the creation, manipulation,
and modification of computer files containing textual data
such as reports, documentation, alphanumeric command lines,
and program source code.
Generally, one or more text editors are supplied by the computer
manufacturer.
Three text editors are in widespread use on the Vax---%
{\it SOS}, {\it EMACS} and {\it EDT}.  {\it vi}, {\it edt} and
{\it emacs} are used on NRAO's Convex computers.
See {\it line editor} and {\it screen editor}.

\aa{text file}
a computer data file containing only textual data, as might
be written by a {\it text editor} \qv.
Programs such as the AIPS tasks sometimes write messages, especially
progress report messages, into a text file---see {\it message file.}

\aa{Third NRAO Synthesis Imaging Summer School}
The 1988 Summer School
on Synthesis Imaging which was held in Socorro, New Mexico in June 1988.
The lectures were formally published in {\it Synthesis Imaging in Radio
Astronomy}.

\aa{thrashing} See {\it memory thrashing}.

%\aa{Tikhonov regularization method}

\aa{time-baseline order}
An ordered set of visibility measurements
${\leftset V_{ij}(t_k)\mid\ 1\le i<j\le n,\ k=1,\dots,l\rightset}$
re\-cord\-ed with an $n$ element interferometer at times $t_1<t_2<\dots<t_l$
is said to be in {\it time-baseline order}
if the ordering is such that
all of the data obtained at time $t_1$, sorted
into the canonical ordering by baseline, occur first,
followed by the data obtained at time $t_2$, again ordered
canonically, etc., etc.
(The canonical ordering by baseline is the order
$V_{12},V_{13},\dots,V_{1n},V_{23},\dots,\dots,V_{n-1,n}$\,.)
Compare {\it baseline-time order}.
\par
Time-baseline ordering of a {\it u-v data file} is convenient for
calibration purposes.
The AIPS task for self-calibration requires that its
input \uv data file be time-baseline ordered.

\aa{time smearing}
in a radio interferometer map, the space-variant broadening of
the {\it point spread function} (or {\it beam}\/) which is due to
time averaging of the data.
When, for example, the visibility data along a {\it u-v} track are averaged,
with equal weight, over time intervals of width $\Delta t$ sec.,
the visibility amplitude of a point source
is reduced by a factor $\approx{\sin\gamma\over\gamma}$ ---%
where $\gamma\equiv\pi(u^\prime x+v^\prime y+w^\prime z)\Delta t$, where
the primes denote the time rate of change of the spatial frequency
coordinates $(u,v,w)$ along the track (wavelengths/sec.),
and where $(x,y,z)$ denotes the direction cosines of the location
of the point source with respect to the {\it phase tracking center}.
For further details, see A.~R.~Thompson's Lecture No.~2
and Alan Bridle and Fred Schwab's Lecture No.~13 in the \sira.
Compare {\it bandwidth smearing}.

\aa{trackball}
a spherical ball mechanism, about the size
(10\,cm., or so, in diameter) of a tennis ball,
which may be oriented manually by the interactive user of
a television display device such as the \iis.
The ball can be rotated about any axis, and its orientation, which is
sensed by the computer,
typically is used to control the enhancement or the coloration of
the displayed data (i.e., to control the TV {\it transfer function(s)}\/),
or to position the {\it TV cursor}, in order
to point out to a program features in the displayed image
which are of particular interest.

\aa{trackball button}
On the unit which houses the trackball for the \iis\ Model 70
TV display device are the four {\it trackball buttons},
labeled A, B, C, and D.
These are switches that are used, in conjunction with the display
routines, to exert additional control over the TV display.
Occasionally these buttons are put to other use in AIPS, such as
stopping the Clean deconvolution program.

\aa{transfer function}
a transform which can be used to describe the output of a device (say, an
electrical transducer) as a function of the input to the device.
See {\it TV look-up table}.

\aa{TRC}
{\it top right corner}, the corner of an image diagonally opposite the BLC.
See $m\times n$ {\it map}.

\aa{true color display}
a type of {\it false color display}, \qv.

\aa{TU77 tape drive}
a model of tape drive used on the NRAO's Vaxes,
capable of operation at 800 and 1600\,bpi.

\aa{TU78 tape drive}
a model of tape drive used on the VLA Vaxes,
capable of operation at 1600 and 6250\,bpi.

\aa{TV blink}
a feature of a computer-controlled TV display device,
such as the {\it\iis}\/, intended to facilitate the comparison
of a pair of images stored on two different {\it image planes}.
The TV display is made to alternate between the two images.
The AIPS implementation of blinking allows the user,
by manipulating the {\it trackball}\/, to control
the rate of alternation and the fraction of time that
each image is displayed.

\aa{TV cursor} See {\it crosshair}.

\aa{TV image catalog} See {\it image catalog.}

\aa{TV look-up table}
a memory within the control unit of a TV display device which is used
for storage of the {\it transfer functions} controlling the intensity of
the display, as a function of pixel value.
Within AIPS, the transfer functions may be altered through
the use of interactive verbs
and manipulation of the {\it trackball}.

\aa{TV roam}
a feature of a computer-controlled TV display device such as the {\it\iis}\ %
which allows contiguous parts of a single large image,
stored on more than one {\it image plane}, to be displayed
as if the image were stored on a single, larger image plane.
On the \iis\ unit, the portion of the image to be displayed
on the TV screen is selected by manipulation of the {\it trackball}.
See {\it image plane}.

\aa{TV scroll}
a feature of a computer-controlled TV display device
such as the {\it\iis}\ %
which allows the display of an image stored on a single {\it image plane}
to be moved about the display screen.
This feature, which also is called panning,
commonly is used in combination with the {\it TV zoom} capability.
On the \iis\ unit, the scroll ordinarily is controlled by manipulation
of the {\it trackball}.
Compare {\it TV roam}.

\aa{TV zoom}
a magnification feature
of a computer-controlled TV display device such as the {\it\iis}.
On the \iis, the three available magnification factors
(which multiply the linear dimensions of the original display of the image
by a factor of 2, 4, or 8)
generally are selected by depressing one of the {\it trackball buttons}.
Since the magnification is achieved by pixel replication (i.e.,
by piecewise linear interpolation)---rather than by a smooth interpolation---%
the visual impression may be somewhat displeasing.
The entire magnified image may not fit on the TV screen,
so zoom usually is used in combination with the {\it TV scroll}
feature.

\aa{uniform weighting}
A {\it dirty map} obtained by computing the inverse Fourier
transform (FT) of a weighted {\it u-v measurement distribution}
in which each visibility sample has been weighted in inverse
proportion to the local density of the {\it u-v coverage}
is said to have been computed using {\it uniform weighting}.
When a radio map is computed via the fast Fourier transform algorithm,
uniform weighting may be achieved
by computing normalized discrete convolution summations
$\sum_{i=1}^N C(u-u_i,v-v_i)\s V_i/N$,
where $(u,v)$ denotes the spatial frequency coordinates of a given
\uv grid cell, where
$C$ is an appropriately chosen {\it gridding convolution function},
and where
the $\s V_i$ are the $N$ visibility measurements obtained
at positions $(u_i,v_i)$ in some neighborhood
of $(u,v)$, the size of which is determined by the {\it support} of $C$.
The uniform weighted map is given by the inverse discrete FT
of data interpolated and smoothed in this manner, onto the lattice
points of a rectangular grid.
So-called {\it natural weighting} is achieved by using unnormalized
convolution sums, rather than by dividing by $N$.
The AIPS mapmaking tasks use a weighting scheme which is slightly
more complicated than that described here.
\par
Since the density of \uv coverage typically is greater in
the inner regions of the \uv plane,
a map computed using uniform weighting has finer {\it spatial
resolution} than one computed with natural weighting.
With natural weighting, low surface-brightness extended features
may be more easily discernible than with uniform weighting.
Essentially the same effect can be achieved with uniform weighting,
when accompanied by use of a {\it u-v taper function}.

\aa{UNIX}
a ``universal'' computer operating system developed at the Bell
Telephone Laboratories.  Its virtue is that program packages
such as AIPS---once having been made to run under one UNIX-based
operating system---ought to run on any other such system, even
on a computer of different manufacture, with no alterations.
Many Vaxes operate under UNIX, though not the NRAO's.
The Convexes C-1 in Charlottesville and at the AOC operates under
UNIX.
See {\it operating system}.

\aa{user-coded task}
an AIPS {\it task} written by a user, rather than by a professional
programmer or a member of the AIPS programming group.
One of the design goals for AIPS, not yet fully realized, is that
it should be relatively easy for a user
who is not an experienced programmer to write an AIPS task
suited to his own needs---%
i.e., that it should be fairly simple for him to make some
sense of the AIPS database, and to get at his data and manipulate
it as he sees fit.
The AIPS task named FUDGE is intended to serve as a paradigm
for user-coded tasks for manipulation of {\it u-v data files};
two other tasks, TAFFY and CANDY, are paradigms for
{\it image file} manipulation.
A useful reference is the manual by W.~D.~Cotton and a `cast of AIPS'
[{\it Going AIPS! A Programmers Guide to the NRAO Astronomical
Image Processing System}\/, NRAO, Charlottesville, VA, 1990].
\par
The addition to AIPS of new {\it verbs}\/,
and modification of the functioning of existing verbs,
requires modifying the AIPS program itself;
this is best left to the AIPS programming group.

\aa{u-v coverage}
the {\it support} of the \uv {\it sampling distribution} \qv.
Also see {\it conjugate symmetry.}

\aa{u-v data file}
in AIPS, a {\it primary data file} designed to accommodate  the measurements
of the visibility function of a radio source.

\aa{u-v data flag}
In an AIPS \uv {\it data file}, each visibility measurement is accompanied
by a real-valued weight, which ordinarily is (positive and)
proportional to the length
of the integration period over which the measurement was obtained.
A non-positive weight represents a \uv data flag,
which signifies that the visibility measurement ought to be ignored.
See {\it flagging} and {\it clipping}.

\aa{u-v FITS format}
an extension of the {\it FITS format} (originally designed for the
interchange of image data)
to accommodate radio interferometer visibility data
[E.~W.~Greisen and R.~H.~Harten,
An extension of FITS for groups of small arrays of data,
{\it Astron.\ Astrophys.\ Suppl.\ Ser.}, \bold{44} (1981) 371--374].
See {\it FITS format}.

\aa{u-v measurement distribution}
in radio interferometry,
a linear combination of shifted Dirac $\delta$-functions,
one located at the position in the \uv plane of each
visibility measurement,
and each weighted by the visibility measurement obtained
at that location.
Denoting the \uv coverage by $\{(u_i,v_i)\}_{i=1}^n$,
the visibility function by $V$,
and the measured visibility by $\s V$,
the (two-dimensional) \uv measurement distribution $S$
is given by
$S(u,v)=\sum_{i=1}^n \s V(u_i,v_i)\delta(u-u_i,v-v_i)$.
Compare {\it u-v sampling distribution}.
\par
This definition may be modified to incorporate two
types of weight function, yielding a {\it weighted}
and\slash or {\it tapered} measurement distribution---%
see {\it u-v taper function} and {\it u-v weight function}\/.
\par
The visibility measurements $\{\,\s V(u_i,v_i)\,\}$
are not actual samples of $V$,
but rather are error-corrupted samples of a function
which represents some sort of {\sl local average} of the visibility---%
this is a distinction which it is worthwhile to note, and then to ignore.
Various systematic errors affecting the measurements may be
corrected by proper calibration---see {\it antenna/i.f.\ gain}
and {\it instrumental polarization}.

\aa{u-v sampling distribution}
in radio interferometry, a linear combination of shifted Dirac
$\delta$-functions,
one located at the position in the \uv plane of each
visibility measurement.
Sometimes termed \uv {\it transfer function}.
See {\it beam}.
\par
If $\{(u_i,v_i)\}_{i=1}^n$ (the {\it u-v coverage})
is the set of spatial frequency coordinates at which
the source visibility has been sampled,
then the (two-dimensional) \uv sampling distribution $S$
is given by
$S(u,v)=\sum_{i=1}^n\delta(u-u_i,v-v_i)$.
\par
Occasionally the term {\it u-v sampling distribution} is
used in the same sense as the term
{\it u-v measurement distribution} \qv.

\aa{u-v taper function}
an even, real-valued weight function
(typically, an elliptical Gaussian), smooth and peaked at the origin,
which may be
incorporated into the definition of {\it u-v measurement
distribution} or {\it u-v sampling distribution}\/, above,
serving to control the spatial
resolution of the radio map or the beam; i.e., to enhance the response
to extended features in the radio source brightness distribution
by giving relatively higher weight to the measurements
at short \uv spacings.
Compare {\it u-v weight function.}

\aa{u-v transfer function}
same as {\it u-v sampling distribution}\/, but always
explicitly incorporating any \uv weight function or
\uv taper function.

\aa{u-v weight function}
a real-valued function which may be incorporated in the definition, above,
of {\it u-v measurement distribution} or {\it u-v sampling distribution}\/,
serving to
weight each measurement either according to an estimate of the statistical
measurement error,
or according to the local density of sampling, or both.
Compare {\it u-v taper function} and see {\it uniform weighting.}

\aa{Varian printer}
an electrostatic printer\slash plotter manufactured by the Varian Corp.

\aa{Variational Method}
the name which applies to Tim Cornwell's AIPS implementation
(in the program VM) of the {\it maximum entropy method},
to solve the image deconvolution problem $g=b\ast f$,
where $g$ and $b$ are given, and $f$ is unknown.
The regularizing term $S(\s f)$ (see {\it regularization method}\/),
a function of the computed approximate solution $\s f$,
is given by the negative of an entropy expression, of the form
$$H(\s f)=-\int_A\s f(x)\log{\s f(x)\over h(x)}\,dx\,.$$
Here $A$ denotes the (assumed known) support of $f$,
and $h$ is a prior estimate of $f$;
when $h\equiv\text{constant}$, this agrees with the standard formulation
of the maximum entropy method.
A weighted sum $\chi^2(\s f)+\lambda S(\s f)$
of a $\chi^2$ error term and $S$ is minimized,
and the regularization parameter $\lambda$ is chosen so that the
r.m.s.\ residual corresponding to the final iterate
is approximately equal to an input value.
For optical data the $\chi^2$ term is taken as $\|g-b\ast\s f\|^2$,
whereas for radio data the $\chi^2$ term is evaluated
in the visibility domain, where the measurement errors
may more properly be assumed to be statistically independent.
Also, $\int_A\s f$ is constrained to be near
an estimate of the {\it zero-spacing flux} which is supplied by the user.
The minimization is done using a Newton-type method, with
a diagonal approximation to the Hessian of the objective function
and intricate control of the steplength.
In terms of execution speed, this method is competitive
with the {\it Clark Clean algorithm}\/---%
at least in the case of large objects of complex structure observed
with the VLA---and superior results usually are obtained
for this class of objects.
See [T.~J.~Cornwell, Deconvolution with a maximum entropy type
algorithm, VLA Scientific Memo.\ No.~149].

\aa{verb} See {\it POPS symbols.}

\aa{Versatec printer}
an electrostatic printer\slash plotter manufactured by the Versatec Corp.,
and used on the NRAO's AIPS computer systems.

\aa{Very Long Baseline Interferometry --- Techniques and Applications}
Proceedings of the NATO Advanced Study Institute held at Castel S.\ Pietro
Terme, Bologna, Italy in 1988.  Edited by M.\ Felli and R.\ E.\ Spencer.
Kluwer Academic Publishers, Dordrecht (1989).
This volume contains much useful information on the planning and execution
of VLBI observations as well as on the reduction of VLBI data.

\aa{vi}
a moderately sophisticated text editor (a {\it screen editor}) used
on computers which run the UNIX operating system.
See {\it text editor}.

\aa{virtual memory page}
on a computer running under a virtual memory operating system,
one unit of {\it virtual memory storage}.
At a typical Vax installation, the size of a virtual memory page
is 512 bytes.

\aa{virtual memory page swapping}
on a computer running under a virtual memory operating system,
the action (initiated automatically by the operating system) of reading
new virtual memory pages into the physical memory, and
storing on disk (i.e., in the {\it virtual memory}\/)
the data which thus have been displaced.
Each occurrence of the displacement of a memory page
is referred to as a {\it page fault.}
See {\it memory thrashing}.

\aa{virtual memory storage}
computer storage---typically disk storage---in an area apart
from the {\it physical memory} of a computer.
Access to virtual memory storage is controlled by the operating system,
in a way intended to give the programmer the illusion that a large amount
of physical memory is present.
Access to virtual memory may be much slower than access to physical
memory, and the operating system may incur a significant amount of
overhead in managing the virtual memory.
See {\it memory thrashing}.

\aa{visibility phase tracking center}
In a correlating-type radio interferometer
usually the {\it fringe stopping center} and the {\it delay tracking
center} coincide.
When this is the case, both are referred to as the visibility
phase tracking center.

\aa{VM} See {\it Variational Method}.

\aa{VMS}
(Virtual Memory System) the operating system used on the
NRAO's Vax computers.
See {\it virtual memory storage} and {\it operating system}.

\aa{wedge}
a legend, or scale---generally in the form of a bar graph
with gradations in {\it intensity} and {\it chromaticity}\/---%
which may be displayed adjacent to a photographic or video
display of a digitized {\it image}.
The wedge is a visual representation
of the {\it transfer function} that was used in generating the display.
The wedge is either colored or gray, depending on whether the
display is a {\it pseudo-color display} or a {\it gray-scale display}.
\par
Note that a {\it false color display} would require more than one
wedge (or a multi-tiered wedge) to display the several transfer functions,
as well as an additional wedge to display the possible color mixtures.

\aa{window Clean}
an application of the \hca\/, with an
explicit specification, by the user, of the Clean window.
Generally the user should specify a Clean window
whenever it is possible to make a reasonably valid and restrictive
estimate of the {\it support}
of the true radio source brightness distribution.
At the termination of the algorithm,
it is prudent to examine a display of the residual map
for the presence of large residuals outside of the Clean window;
their presence could suggest that an inappropriate window was selected.
See {\it Clean window.}

\aa{working set size}
on a computer running under a virtual memory operating system,
the amount of {\it physical memory} allocated to a task.
Any program memory requirement in excess of the working set size
is relegated to {\it virtual memory storage}.
At a typical Vax installation, the working set size
is set at ${1\over4}$ or $1\over2$ megabyte.

\aa{x-y order}
An ordered set of visibility samples $\{V(u_i,v_i,w_i)\}_{i=1}^n$
arranged according to descending absolute value of the spatial
frequency coordinate $u$ ---
i.e., with $|u_1|\ge|u_2|\ge\dots\ge|u_n|$ ---
is said to be in {\it x-y order}.
\par
$x$-$y$ order is a convenient ordering for the operation of gridding
convolution; hence the AIPS mapping tasks require that their
input \uv {\it data files} be sorted accordingly.
See {\it sort order}.

\aa{Y-routine}
in AIPS, a subroutine designed to aid in the use of a specific
model of TV display device, such as the {\it\iis} Model 70.
AIPS requires a relatively small core of Y-routines
implementing basic TV display functions;
complicated display functions then are accomplished
by combining these basic functions that are supposed to be
common to many models of TV display device.
At present there are approximately 25
Y-routines for use at those AIPS installations equipped with an \iis.
Compare {\it Z-routine.}

\aa{zero-spacing flux}
The visibility $V(u,v)\equiv\hat f(u,v)$ ($\ \hat{\phantom{.}}\ $
denotes Fourier transform)
of a source brightness distribution $f$ in a neighborhood of $u=v=0$
is inaccessible to an interferometer composed of elements of finite
collecting area.
The {\it zero-spacing flux} is equal to the total, or integrated flux
density of the source---i.e., it is given by
$V(0,0)=\int_{-\infty}^\infty\,\int_{-\infty}^\infty f(x,y)\,dx\,dy\,.$
Because the hole in the {\it u-v coverage} in the neighborhood of
the origin may be fairly large, image reconstruction methods,
such as the \hca\/, may do a poor job, within this central region,
of interpolating the measured data.
This frequently is manifested by the appearance of a
{\it negative bowl artifact}\/---a negative `baseline' beneath the
reconstruction of $f$\,---owing to the reconstruction method having
underestimated the zero-spacing flux.
The {\it Variational Method} for maximum entropy reconstruction
requires that the user supply an estimate of $V(0,0)$.
The Clean algorithm, too, may benefit if a datum at $u=v=0$ is included when
the {\it dirty map} is constructed.
\par
A zero-spacing estimate can be derived from single-dish measurements.
Providing a proper estimate is difficult, because of contamination
of single-dish measurements by `confusing sources.'
The estimate ought to correspond to a telescope with the same
primary beam response as the array elements;
and it is not just a single datum $V(0,0)$ which is missing,
but rather a region---so proper weighting of the zero-spacing
information is tricky.
See Tim Cornwell and Robert Braun's Lecture No.~8 in the \sira.

\aa{zoom} See {\it TV zoom.}

\aa{Z-routine}
in AIPS, a subroutine---generally designed to perform some
routine, often needed function---written for a specific model
of {\it host computer} or for a specific host computer operating system.
The implementation of certain basic functions, especially those for
file access and file management, generally is machine dependent
and operating system dependent.
The typical AIPS installation requires 50--100 Z-routines.
Compare {\it Y-routine.}

\aa{1978 Groningen Conference Proceedings}
{\it Image Formation from Coherence Functions in Astronomy.
Proceedings of IAU Colloquium No.~49 held at Groningen, the Netherlands,
August 10--12, 1978},
edited by C.~van~Schooneveld, D.~Reidel, Dordrecht, Holland, 1979---%
contains many papers on aperture synthesis techniques,
including some of the early papers on {\it hybrid mapping}.

\aa{1982 Summer Workshop Proceedings}
{\it Synthesis Mapping. Proceedings of the NRAO--VLA Workshop
held at Socorro, New Mexico, June 21--25, 1982}, edited by
A.~R.~Thompson and L.~R.~D'Addario, NRAO, Green Bank, WV, 1982---%
a collection of the fifteen lectures which comprised
this short course on aperture synthesis techniques---%
a useful introduction to VLA data reduction methods.

\aa{1983 Sydney Conference Proceedings}
{\it Indirect Imaging: Measurement and Processing for Indirect Imaging.
Proceedings of an International Symposium held in Sydney, Australia,
August 30--September 2, 1983},
edited by J.~A.~Roberts, Cambridge Univ.\ Press, Cambridge, 1984---%
contains a number of interesting papers on aperture synthesis techniques.

\aa{1985 Summer School Proceedings}
lecture notes from the second NRAO summer short course on
radiointerferometric imaging (in preparation).
This volume supersedes the \wsp.

\aa{1988 Summer School Proceedings}
lecture notes from the third NRAO Summer School on radio interferometric
imaging.  The lectures have been published as {\it Synthesis Imaging in Radio
Astronomy} ({\it q.v.}).

\aa{4012} See {\it TEK screen.}
\vfill\eject
\hphantom{.}
\vfill\eject
\hphantom{.}
