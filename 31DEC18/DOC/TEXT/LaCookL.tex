%-----------------------------------------------------------------------
%;  Copyright (C) 2013-2014, 2016-2018
%;  Associated Universities, Inc. Washington DC, USA.
%;
%;  This program is free software; you can redistribute it and/or
%;  modify it under the terms of the GNU General Public License as
%;  published by the Free Software Foundation; either version 2 of
%;  the License, or (at your option) any later version.
%;
%;  This program is distributed in the hope that it will be useful,
%;  but WITHOUT ANY WARRANTY; without even the implied warranty of
%;  MERCHANTABILITY or FITNESS FOR A PARTICULAR PURPOSE.  See the
%;  GNU General Public License for more details.
%;
%;  You should have received a copy of the GNU General Public
%;  License along with this program; if not, write to the Free
%;  Software Foundation, Inc., 675 Massachusetts Ave, Cambridge,
%;  MA 02139, USA.
%;
%;  Correspondence concerning AIPS should be addressed as follows:
%;          Internet email: aipsmail@nrao.edu.
%;          Postal address: AIPS Project Office
%;                          National Radio Astronomy Observatory
%;                          520 Edgemont Road
%;                          Charlottesville, VA 22903-2475 USA
%-----------------------------------------------------------------------
\setcounter{chapter}{11}
\APPEN{Handling EVLA P-band Data in \AIPS}{Special Considerations for EVLA
      P-band Data Processing in \AIPS}{EVLAlowband}
\renewcommand{\Chapt}{26}

\renewcommand{\titlea}{31-December-2017 (revised 7-February-2018)}
\renewcommand{\Rheading}{\AIPS\ \cookbook:~\titlea\hfill}
\renewcommand{\Lheading}{\hfill \AIPS\ \cookbook:~\titlea}
\markboth{\Lheading}{\Rheading}

The \Indx{EVLA} has been equipped with new, wide-band receivers in the
range 230 MHz to 470 MHz, known as P-band.  Most of the data reduction
and imaging at this band are similar to those at other bands, so the
following will assume some degree of familiarity with the rest of the
\Cookbook, particularly \Rchap{cal}\@.  This guide was initially
prepared by Minnie Mao from an earlier guide written by Susan Neff.
Enough has changed with the P-band system that a thorough revision of
the Appendix has been done.

{\bf Note that the choice of file names, disk assignments, catalog
numbers, and the like depends on your circumstances and will almost
certainly not be the same as those shown in the examples below.  Note
also that experienced (but still reasonable) EVLA users disagree on
the details of data reduction.  Some compromise in these details has
been attempted below.}

RFI, especially in spectral-line observations, has tended to
accumulate at the phase center for reasons not well understood.  It
may be best to observe your sources somewhat offset from the phase
center.

\sects{P-band calibration and editing in \AIPS}

\begin{enumerate}
\item\ Download your data set from the archive found at\\
\hphantom{MMMMMMM}{\tt https://archive.nrao.edu/archive/ArchiveQuery}\\
making sure that you select SDM-BDF data set (all files) as the
download data format.

\item\ Before starting {\tt AIPS} and while in the terminal window you
  intend to use for {\tt AIPS}, enter
\dispi{cd {\it my\_data\_directory} \CR}{to change to the data
             directory containing the top-level directory of your
             data set.}
\dispi{export MYAREA=`pwd` \CR}{for bash shells, or}
\dispi{setenv MYAREA `pwd` \CR}{for tcsh and c shells.}
\dispe{Note the back ticks around {\tt pwd} which cause the output of
{\tt pwd} to appear in the typed line.}

\item\ Start {\tt AIPS} with, for example,
\dispi{aips tv=local pr=3 \CR}{to use Unix sockets for the TV and
       printer 3}
\dispe{and enter your user number when prompted.}

\item\ Look at the SDM-BDF file contents with {\tt \tndx{BDFLIST}} ---
  make sure you do {\it not} have a quote at the end of the file name,
  or else the lower case characters get changed into upper case and
  the file won't be found.
\dispi{default bdflist \CR}{initialize all adverbs}
\dispi{dowait 1; docrt 1 \CR}{to get output on terminal as well as log
       file.}
\dispi{asdmfile(1) = 'MYAREA: \CR}{two parts are concatenated.}
\dispil{asdmfile(2) = 'TSUB0001.sb23643640.eb23666175.56454.14034020834
       \CR}{}
\dispi{bdflist \CR}{to examine file contents.}
\dispe{Pay particular attention to the configuration numbers and
  number of spectral channels.  These refer to each unique correlator
  {\it and} antenna setup, not just the size of the current VLA\@.
  Each configuration and number of channels that you want will need to
  be loaded individually.}
\vfill\eject

\item\ Load data into \AIPS\ with {\tt \tndx{BDF2AIPS}} --- We put
  everything on disk $d$ because that's where there's more room.
  Note, do not use {\tt DEFAULT} if you wish to retain the adverbs
  from the previous step.  Loading data may take a while depending on
  the size of your data.\Iodx{EVLA}
\dispi{outn 'my-name' \CR}{to select a meaningful file name.}
\dispi{config $c$ \CR}{to select configuration $c$.}
\dispi{outdi $d$ \CR}{to select the disk.}
\dispi{bdf2aips \CR}{to translate the data into \AIPS.}
\dispi{nchan $N$ \CR}{to limit to data with $N$ spectral channels;
  some spectral-line configurations can contain more than one number
  of channels but {\tt AIPS} can only load one number at a time.}
\dispe{Do this as often as needed to load all desired configurations.
Spectral-line set-ups can have spectral windows with different numbers
of channels within the same configuration.  \AIPS\ can only handle one
at a time, so specify {\tt nchan $N$ \CR} to load the data with $N$
channels and repeat as needed.}

\item\ Look at header of the \AIPS\ file written by {\tt BDF2AIPS}
\dispi{indi $d$; getn 1 \CR}{to select disk and catalog number,
       assuming you had no files on disk $d$ previously.}
\dispi{imh \CR}{to see the details of the $uv$-data set header.}

\item\ look at where the telescopes were during your observation with
     {\tt \tndx{PRTAN}}
\dispi{default prtan \CR}{ }
\dispi{indi $d$; getn 1 \CR}{ }
\dispi{docrt = -1 \CR}{to make a printer listing to keep at your
       terminal.}
\dispi{go prtan \CR}{ }

\item\ Look at the observation summary with {\tt \tndx{LISTR}}
\dispi{default listr \CR}{ }
\dispi{indi $d$; getn 1 \CR}{ }
\dispi{opty 'scan' \CR}{to list scans and sources.}
\dispi{docrt = -1 \CR}{to make a printer listing to keep at your
       terminal.}
\dispi{go \CR}{ }

At this stage, Minnie's summary recommends separating P-band from
4-band data.  At present, only P-band is recorded so that step is not
needed.  Then, Minnie recommends using {\tt POSSM} to find dead
antennas plus those with crossed polarization.  Then she recommends
{\tt UVFLG} to flag the dead ones and {\tt FIXRL} to uncross the
mis-wired ones.  These problems have been essentially eliminated.
In addition, P-band data are now delivered by default in a number of
relatively narrow spectral windows (called IFs in \AIPS) which helps
minimize data lost to RFI and eliminates the {\tt NOIFS} and {\tt
  MORIF} steps from Minnie's guide.

\item\ However, the requantizer gains are now allowed to vary and must
  be corrected.\todx{TYAPL}
\dispi{default tyapl \CR}{to select the needed task}
\dispi{indi $d$; getn 1 \CR}{to select the input file.}
\dispi{optype 'PGN' \CR}{to select requantizer gain
       correction.}
\dispi{inext 'SY' ; in2ver = 1 \CR}{to apply the SysPower table, the
       version number is not defaulted.}
\dispi{outdis $d$ ; go \CR}{to stay on one disk.}

\vfill\eject
\item\ Narrow-band RFI will normally cause ringing in P-band data.
  Use {\tt POSSM} to examine some baselines from a calibration source
  to check this if you want to be careful.  The default setup delivers
  data from outer IFs for which the sensitivity of the P-band receiver
  is poor.  Use \tndx{SPLAT} to drop the insensitive IFs (those
  centered on 232, 248, 264, and 472 MHz) and to do Hanning smoothing
  to reduce the ringing. \todx{SPLAT}
\dispi{default splat \CR}{to select the needed task}
\dispi{indi $d$; getn 2 \CR}{to select the (new) input file.}
\dispi{flagver 1 \CR}{to apply the on-line flags.}
\dispi{smooth = 1, 0 \CR}{to do Hanning smoothing once and for all.}
\dispi{bif = 4; eif = 15 \CR}{to select the more sensitive IFs.}
\dispi{outdis $d$ ; go \CR}{to put the output on the same disk.}

\item\ There is also a Navy satellite that makes a mess of data from
  360 MHz to 380 MHz.  Delete this with {\tt \tndx{UVFLG}} in one of 2
  ways.  For the standard setup, you can specify the channels as
\dispi{default uvflg \CR}{ }
\dispi{indi $d$; getn 3 \CR}{ }
\dispi{bif 6 ; eif 6; bchan 65; echan 128; \CR}{ }
\dispi{go \CR}{ }
\dispi{bif 7 ; eif 7; bchan 1; echan 97; \CR}{ }
\dispi{go \CR}{ }
\dispe{For any setup, you may make a text file containing the line\\
\hphantom{AAAAAAA}{\tt BFREQ = 360   EFREQ = 380  /}\\
and then use it with {\tt UVFLG} as}
\dispi{default uvflg \CR}{ }
\dispi{indi $d$; getn 3 \CR}{ }
\dispi{intext = '{\it MYWORK:pband.flg} \CR}{to give file name}
\dispi{go \CR}{ }

\item\ Edit UV data with {\tt \tndx{EDITA}} and the {\tt SY} table.
  {\it The {\tt SY} table $P_{\rm sum}$ parameter is sensitive to the
    occasional bursts of RFI which render the data useless.  Flag all
    seriously discrepant points but do not flag really tightly since
    noise and even abnormalities in this parameter are ``normal.''}
\dispi{default edita \CR}{ }
\dispi{indi $d$; getn 3 \CR}{ }
\dispi{inext 'SY' \CR}{ }
\dispi{go \CR}{ }

\item\ At this time, it is best to identify those spectral channels
  affected by RFI throughout the run and to flag them once and for
  all.  {\tt \tndx{FTFLG}} will display the data combining all
  included baselines in a single display (per polarization), while the
  very similar {\tt SPFLG} allows much better discrimination (and more
  work) by displaying the baselines separately.  Be sure to flag all
  sources if you flag channels over all times.\Iodx{EVLA}
\dispi{default ftflg \CR}{ }
\dispi{indi $d$; getn 3 \CR}{ }
\dispi{calcode '*' \CR}{to examine calibrators only}
\dispi{dparm(6) = $x$ \CR}{to specify the basic integration time in
       the data in seconds.}
\dispi{go \CR}{ }
\dispe{If you use {\tt SPFLG}, you should limit the times, sources,
       and/or baselines examined.  The flagging then is best done with
       {\tt \tndx{UVFLG}} from notes made while you study the data
       with {\tt SPFLG}\@.}
\dispi{default uvflg \CR}{}
\dispi{indi $d$; getn 3 \CR}{ }
\dispi{bif = $bi$ ; eif = $ei$ \CR}{to set range of spectral windows.}
\dispi{bchan = $bc$ ; echan = $ec$ \CR}{to set range of spectral
       channels.}
\dispi{antennas = $a1, a2, a3$ \CR}{to select antennas to flag.}
\dispi{baseline = $b1, b2, b3, b4$ \CR}{to limit flags to the
       baselines from the pairs $a1$ through $a3$ with $b1$ through
       $b4$.}
 \dispi{reason 'bad channels' \CR}{to set a reason in the flag
       table.}
\dispi{go \CR}{and repeat as needed with different IFs, channels,
       antennas, etc.}

\item\ This step is optional and should be considered only if you have
  several P-band data sets to reduce.  There is a special capability
  in \AIPS\ developed for lowband data.  The task {\tt \tndx{FGTAB}}
  was written to read a flag table and write to a text file those
  flags that apply to all times, sources, antennas, subarrays, and
  frequency IDs.  The channel specifications are written in frequency
  units which enables {\tt \tndx{UVFLG}} to read the text file and
  apply the same flags to the channels to which they pertain in the
  second data set.  Of course, if the two data sets have identical
  frequency structure, then one could simply copy the {\tt FG} table
  from one data set to the other.  Task {\tt \tndx{TACOP}} would be
  used if you want the flag file by itself.  {\tt \tndx{TAPPE}} would
  be used if you want to append the flag table of the first file to
  one of the second.  Note that these two tasks transfer all flags in
  the input table, not just those that flag the general RFI
  environment.

\item\ Calibrate for the delay errors with {\tt \tndx{FRING}}\@.
  Choose a time range in the middle of the observation where you know
  the antennas have settled down.  A short time range is all you need
  --- look at {\tt LISTR} --- because you don't want time variations
  to come into play.  This is also why you should choose something
  bright --- like 3C286, which is used here as the name for the
  generic fringe- and bandpass-fitting calibration source.  It would
  be wise to check these data for RFI with {\tt POSSM} before running
  {\tt FRING}\@.\Iodx{EVLA}
\dispi{default fring \CR}{ }
\dispi{indi $d$; getn 3 \CR}{ }
\dispi{calso '3C286' ' \CR}{to select the primary calibration source.}
\dispi{timerang 0 3 40 0 0 3 40 55 \CR}{to select $< 1$ minute of good
       data from the calibration source.}
\dispi{solint 1 \CR}{to average over all included times.}
\dispi{aparm(5) 1 ; aparm(6) 1 \CR}{to fit a single delay over all IFs
       and to print some extra detail in the fitting.}
\dispi{dparm(9) 1 \CR}{to avoid any fitting for rates --- important.}
\dispi{refant 22 \CR}{to set a reference antenna, if one is known to
       be better than others.}
\dispi{go \CR}{ }
\dispe{This print level will display the delays found, which should
       all be a few nano-seconds although larger (correct) delays
       sometimes arise.}

\item\ Apply these delay solutions to the calibration ({\tt CL}) table
with {\tt \tndx{CLCAL}}
\dispi{default clcal \CR}{ }
\dispi{indi $d$; getn 3 \CR}{ }
\dispi{calso '3C286','' \CR}{ }
\dispi{snver 1 \CR}{to specify the {\tt SN} table version written by
       {\tt FRING}\@.}
\dispi{go \CR}{to copy {\tt CL} table 1 to version 2 adding the delay
       correction, taken to be constant in time.}

\item\ You should flag out any bad data on your bandpass calibrator.
  Some users feel that a quick-and-dirty flagging is enough at this
  stage.  Use {\tt CLIP} with a carefully chosen, perhaps
  IF-dependent, cutoff level.  A more detailed editing is done by
  other users.  Here are options to use {\tt RFLAG} and {\tt SPFLG}.
  Maybe you'd like to {\tt \tndx{RFLAG}} first and then eyeball with
  {\tt SPFLG}\@.  The options are endless!  The first execution of
  {\tt RFLAG} determines flagging level parameters and makes plots on
  the TV for you to be sure that they are reasonable,  The second
  execution applies these levels and determines new levels in case you
  want to run {\tt RFLAG} even more
  times.\label{it:beginbpass}\Iodx{EVLA}
\dispi{default rflag \CR}{ }
\dispi{indi $d$; getn 3 \CR}{ }
\dispi{sources(1) '3C286' \CR}{ }
\dispi{docal 1 \CR}{to correct delays and a-priori gains.}
\dispi{avgchan 11 \CR}{to set the width of the median-window filter
       run across the spectral channels in each IF\@}
\dispi{fparm = 3, $x$, -1, -1 \CR}{to do a small rolling time buffer,
       set the sample interval, and to have {\tt NOISE} and {\tt
         SCUTOFF} set IF-dependent cutoff levels.}
\dispi{fparm(13) = 1000 \CR}{to clip amplitudes at 1000 Jy.}
\dispi{doplot 12 ; dotv 1 \CR}{to do the 2 most useful plots on the
       TV.}
\dispi{go \CR}{ }
\dispe{{\tt RFLAG} will return cutoff levels to the appropriate
adverbs.  Therefore, to apply the flags using the flagging levels
found above, do {\em not} do a {\tt tget rflag} which will reset the
levels to zero.  Instead, simply do}
\dispi{doplot = -12 \CR}{to apply the cutoff levels and determine new
       ones, plotting on the TV.}
\dispi{go \CR}{ }
\dispe{{\tt RFLAG} will make a new flag table each time it determines
flags (when {\tt DOPLOT} $\leq 0$).  {\tt RFLAG} has numerous {\tt
  FPARM} options you may choose.  {\tt FPARM(9)} and {\tt FPARM(10)}
scale the rms for {\tt NOISE} and {\tt SCUTOFF} that {\tt RFLAG}
computes.  If the fraction of a channel that is flagged exceed {\tt
  FPARM(7)}, the entire channel is flagged. {\tt FPARM(6)} is
especially useful at P-band as it allows channels adjacent to an RFI
peak to be flagged. Similarly, {\tt FPARM(8)} allows channels between
strong RFI peaks to be flagged.  Read the help file for detailed
information.  Flagging calibrator scans aggressively is recommended
especially for spectral-line observations.}

Check the results with
\dispi{default spflg \CR}{ }
\dispi{indi $d$; getn 3 \CR}{ }
\dispi{sources(1)  '3C286' \CR}{to examine only the bandpass calibrator.}
\dispi{dparm(6) = $x$ \CR}{to set sampling time in seconds}
\dispi{docal 1 \CR}{to correct delays and a-priori gains.}
\dispi{go \CR}{ }

It is at this point that experienced users differ significantly in
their approach.  If you have only one bandpass-calibration scan and the
ionosphere was reasonably behaved (phases nearly constant through that
scan), then you should run {\tt BPASS} on the calibration source in
place.  However, when one has multiple bandpass scans and/or a badly
behaved ionosphere, it may be best to split the calibration scans out
and self-calibrate them on a short time interval.  The addition of
multiple scans where the source has been phased up, but the remaining
RFI has not, will reduce the effect of that RFI substantially.  This
scheme is described next.

\vfill\eject
\item\ Split off your calibrator for bandpass calibration --- this is
  only necessary for self-calibrating the BP calibrator.
\dispi{default split \CR}{ }
\dispi{indi $d$; getn 3 \CR}{ }
\dispi{outdi indi ; source '3C286' ' \CR}{to select calibrator and stay
       on the same disk.}
\dispi{outcl 'bpspl' \CR}{to indicate the use for this file.}
\dispi{docalib 1 \CR}{to apply delay and a-priori calibration.}
\dispi{go \CR}{To make a single-source file.}

\item\ Self-cal on your bandpass cal with {\tt \tndx{CALIB}} to solve
  for wiggles in phase as a function of time.\Iodx{EVLA}
\dispi{default calib \CR}{ }
\dispi{indi $d$; getn 4 \CR}{ }
\dispi{refant 22 \CR}{to stay with the chosen reference antenna.}
\dispi{solint $y$ \CR}{to average $n$ records in each solution
       interval, set $y = n x$. }
\dispi{solty 'L1R' ; solmode 'P' \CR}{to do phase-only solutions.}
\dispi{aparm(6) 1 \CR}{to get some diagnostic messages.}
\dispi{go \CR}{to write a new {\tt SN} table.}

Eyeball this new table with {\tt \tndx{SNPLT}} --- should be flat, not
all over the place.  But {\tt \tndx{EDITA}} lets you look at more
things at once.
\dispi{default edita \CR}{ }
\dispi{indi $d$; getn 4 \CR}{ }
\dispi{inext 'sn' ; dotwo 1 \CR}{to plot the phases in the new table.}
\dispi{crowded 1 ; do3col 1 \CR}{to combine all IFs in each colorful
       plot.}
\dispi{go \CR}{ }
\dispe{If these plots are too crowded, you can interactively select
  individual IFs and polarizations.  This task will let you flag data
  if desired.  If you do, then you should re-run {\tt CALIB}\@.}

\item\ Use {\tt \tndx{BPASS}} to calibrate the bandpass shape ---
  corrects for wiggles in phase and amplitude as a function of
  frequency \label{it:endbpass}
\dispi{default bpass \CR}{ }
\dispi{indi $d$; getn 4 \CR}{ }
\dispi{docalib 1 \CR}{to apply the new {\tt SN} table phases.}
\dispi{refant 22 \CR}{ }
\dispi{solty 'l1r' ; solint = -1 \CR}{to average the entire time range.}
\dispi{bpassprm(5) = 1; bpassprm(10) = 3 \CR}{to normalize the
       solutions at the end rather than the data record-by-record.}
\dispi{go \CR}{To generate a new {\tt BP} table containing one record
per antenna.}

\item\ Eyeball the resulting bandpass ({\tt BP}) table --- {\tt
    APARM(8) = 2} plots the {\tt BP} table in {\tt \tndx{POSSM}}\@.
    If you see odd bandpass shapes or spikes in the bandpass, you may
    need to edit your data more extensively and repeat steps
    \ref{it:beginbpass} through \ref{it:endbpass}.
\dispi{tget possm \CR}{ }
\dispi{indi $d$; getn 4 \CR}{ }
\dispi{solint = -1 \CR}{ }
\dispi{aparm(8) 2 ; aparm(9) 1 \CR}{to plot all IFs together for each
       bandpass.}
\dispi{nplots 2 \CR}{2 per page is enough here.}
\dispi{go \CR}{ }

Alternatively, {\tt \tndx{BPEDT}} lets you use the bandpass table to
guide additional flagging of the bandpass calibrator(s).  It is
similar to {\tt EDITA} in that it lets you look at multiple antennas
at the same time for comparison.
\dispi{default bpedt \CR}{ }
\dispi{indi $d$; getn 4 \CR}{ }
\dispi{inext 'bp' ; GO \CR}{ }

\item\ Copy the bandpass table back to your main data with {\tt
    \tndx{TACOP}}\@.
\dispi{default tacop \CR}{ }
\dispi{indi $d$; getn 4 \CR}{ }
\dispi{inext 'bp' \CR}{ }
\dispi{outdi indi ; geton 3 \CR}{to select output name parameters}
\dispi{go \CR}{ }

\item\ Since {\tt RFLAG} probably made a large number of flags, let us
  copy the data file with {\tt \tndx{UVCOP}} and apply the
  flags.\Iodx{EVLA}
\dispi{default uvcop \CR}{ }
\dispi{indi $d$; getn 3 \CR}{ }
\dispi{outdi indi ; outn inna \CR}{to stay on $d$.}
\dispi{outcl 'fixcp' \CR}{with a meaningful class name}
\dispi{flagver $n$ \CR}{to apply the highest flag table ($n$); use
       {\tt IMHEAD} if unsure.  Version 0 means no flagging here.}
\dispi{go \CR}{ }

\item\ Use {\tt \tndx{SETJY}} to compute the flux for the primary flux
calibrator (3C286), storing it in the {\tt SU} table.
\dispi{default setjy \CR}{ }
\dispi{indi $d$; getn 5 \CR}{ }
\dispi{source '3C286' ' \CR}{specify all ``known'' calibrator sources.}
\dispi{opty 'calc' \CR}{to calculate and display fluxes from the
       frequencies plus tables of known sources.}
\dispi{go \CR}{}

\item\ Go through all of your calibration sources, one at a time, to
  check for RFI with {\tt \tndx{POSSM}} and, if present, to address it
  with {\tt SPFLG} and/or {\tt RFLAG}
\dispi{tget possm \CR}{ }
\dispi{source '${\rm calsrc}_i$' '  \CR}{to examine the $i^{\rm th}$
       calibration source.}
\dispi{indi $d$; getn 5 \CR}{ }
\dispi{docal 1 ; doband 1 \CR}{apply our hard-won calibration.}
\dispi{solint 0 \CR}{to look at the average over all time.}
\dispi{aparm(9) 1 \CR}{to plot all IFs together.}
\dispi{nplots 2 \CR}{2 per page is enough here.}
\dispi{go \CR}{ }
\dispe{If RFI is present (it almost certainly will be), two rounds or
  more of {\tt \tndx{RFLAG}} may be needed.}
\vfill\eject

\dispi{default rflag \CR}{ }
\dispi{indi $d$; getn 5 \CR}{ }
\dispi{source '${\rm calsrc}_i$' '  \CR}{to examine the $i^{\rm th}$
       calibration source.}
\dispi{docal 1 ; doband 1 \CR}{apply the calibration.}
\dispi{avgchan 11 \CR}{to set the width of the median-window filter
       run across the spectral channels in each IF\@}
\dispi{fparm = 3, $x$, -1, -1 \CR}{to do a small rolling time buffer,
       set the sample interval, and to have {\tt NOISE} and {\tt
         SCUTOFF} set IF-dependent cutoff levels.}
\dispi{fparm(13) = 1000 \CR}{to clip amplitudes at 1000 Jy.  There are
       many more {\tt FPARM} options you may choose also.}
\dispi{doplot 12 ; dotv 1 \CR}{to do the 2 most useful plots on the
       TV.}
\dispi{go \CR}{ }
\dispe{{\tt \tndx{RFLAG}} will return cutoff levels to the appropriate
adverbs.  Therefore, to apply the flags using the flagging levels
found above, do {\em not} do a {\tt tget rflag} which will reset the
levels to zero.  Instead, simply do}
\dispi{doplot = -12 \CR}{to apply the cutoff levels and determine new
       ones, plotting on the TV.}
\dispi{go \CR}{ }
\dispe{{\tt RFLAG} will make a new flag table each time it determines
flags (when {\tt DOPLOT} $\leq 0$).  It may be good to check what {\tt
RFLAG} has done with {\tt POSSM}\@.  If more edits are needed, use
{\tt \tndx{SPFLG}}}
\dispi{default spflg \CR}{ }
\dispi{indi $d$; getn 5 \CR}{ }
\dispi{source '${\rm calsrc}_i$' '  \CR}{to examine the $i^{\rm th}$
       calibration source.}
\dispi{dparm(6) $x$ \CR}{ }
\dispi{docal 1 ; doband 1 \CR}{apply the calibration.}
\dispi{go \CR}{ }

\item\ Since {\tt RFLAG} probably made a large number of flags, let us
  copy the data file once again with {\tt \tndx{UVCOP}} and apply the
  flags. \Iodx{EVLA}
\dispi{default uvcop \CR}{ }
\dispi{indi $d$; getn 5 \CR}{ }
\dispi{outdi indi ; outn inna \CR}{to stay on $d$.}
\dispi{outcl 'flaged' \CR}{with a meaningful class name}
\dispi{flagver $n$ \CR}{to apply the highest flag table ($n$); use
       {\tt IMHEAD} if unsure.  Version 0 means no flagging here.}
\dispi{go \CR}{ }

\item\ Since all flag tables $< n$ were already merged into table $n$
  which we have applied, delete all flag tables in the output file.
  \todx{EXTDEST}
\dispi{default extdest \CR}{}
\dispi{indi $d$; getn 6 \CR}{}
\dispi{inext 'fg' ; invers = -1 \CR}{to delete all remaining versions
       of flag table.}
\dispi{extdest \CR}{}

\vfill\eject
\item\ Find the amplitude and phase calibration as a function of {\it
    time} using {\tt \tndx{CALIB}}
\dispi{default calib \CR}{ }
\dispi{indi $d$; getn 6 \CR}{ }
\dispi{calcode '*' \CR}{to select all calibration sources}
\dispi{docalib 1 ; doband 1 \CR}{to apply all existing calibration.}
\dispi{refant 22 \CR}{to keep our reference antenna.}
\dispi{solint 0 \CR}{One solution per calibrator scan is okay with
       good short-term phase stability.  Otherwise, set to the longest
       time in minutes over which phase is stable.}
\dispi{solty 'L1R' ; solmode 'a\&p' \CR}{to solve for amplitude and
       phase with an L1 method iterated robustly.}
\dispi{aparm(6) 1 \CR}{to print closure error statistics.}
\dispi{go \CR}{ }
\dispe{Pay attention to the fraction of failed solutions.  If the
  fraction is large, try adjusting parameters such as {\tt SOLINT},
  etc.}

\item\ The solution table ({\tt SN}) produced by the previous step
  will contain amplitude gains at multiple levels, correct ones for
  those sources with fluxes in the source table (3C286) and incorrect
  ones for those which {\tt CALIB} was forced to call 1 Jy.  To solve
  for the unknown fluxes of the secondary calibrators \todx{GETJY}
\dispi{default getjy \CR}{ }
\dispi{indi $d$; getn 6 \CR}{ }
\dispi{source '${\rm calsrc}_1$', '${\rm calsrc}_2$', '${\rm
    calsrc}_3$', $\ldots$ \CR}{to list the secondary calibrators.}
\dispi{calsour = '3c286' ' \CR}{to list the known calibrator(s).}
\dispi{snver = $n$ \CR}{to select only the {\tt SN} table just written
       by {\tt CALIB}\@.  Use {\tt IMHEADER} to find the maximum
       version number $n$.}
\dispi{go \CR}{to solve for the secondary fluxes as functions of IF.}
\dispe{Check the values displayed carefully to make sure that they are
sensible.  {\tt SOUSP} offers tools to examine and adjust the fluxes
and {\tt SN} table $n$.}

\item\ Look at both amplitude and phase solutions with {\tt
    \tndx{EDITA}} or {\tt \tndx{SNPLT}}.\Iodx{EVLA}
\dispi{tvini \CR}{ }
\dispi{default snplt \CR}{ }
\dispi{indi $d$; getn 6 \CR}{ }
\dispi{inext 'sn' ; opty 'amp' \CR}{to plot amplitudes.}
\dispi{nplots 4 ; dotv 1 \CR}{to plot 4 panels on each TV page.}
\dispi{opco 'alif' ; do3col 1 \CR}{to plot all IFs in each panel using
       color to distinguish them.}
\dispi{go \CR}{ }
\dispe{One can also plot IFs in separate panels or one at a time if
  these plots are too crowded.  Then}
\dispi{opty 'phas'; go \CR}{To check the phases.}
\dispe{To use {\tt \tndx{EDITA}} to look at both amplitude and phase
  at the same time for multiple antennas}
\dispi{default edita \CR}{ }
\dispi{indi $d$; getn 6 \CR}{ }
\dispi{inext 'sn' ; dotwo 1 \CR}{to plot the phases in the new table.}
\dispi{crowded 1 ; do3col 1 \CR}{to combine all IFs in each colorful
       plot.}
\dispi{go \CR}{ }
\dispe{If these plots are too crowded, you can interactively select
  individual IFs and polarizations.  This task will let you flag data
  if desired.  If you do, you should then re-run {\tt CALIB}\@.}

\item\ Apply the solutions to the calibration table with {\tt
    \tndx{CLCAL}}
\dispi{default clcal \CR}{ }
\dispi{indi $d$; getn 6 \CR}{ }
\dispi{calcode '*' \CR}{to select all calibration sources}
\dispi{sampty 'box' ; bparm 0.3, 0.3 \CR}{To smooth the {\tt SN} table
       if you used {\tt SOLINT} less than full scans.  Leave this out
       if you used scan averages.}
\dispi{refant 22 \CR}{ }
\dispi{go \CR}{ }

\item\ Use {\tt \tndx{SNPLT}} to check that the solutions have been
   interpolated between calibrators.
\dispi{tget snplt \CR}{ }
\dispi{inext 'cl' \CR}{ }
\dispi{go \CR}{ }

\item\ Make an image of 3C286 as a sanity check using {\tt \tndx{IMAGR}}
\dispi{default imagr \CR}{ }
\dispi{indi $d$; getn 6 ; outdi indi \CR}{ }
\dispi{docal 1 ; doband 1 \CR}{to apply the calibration.}
\dispi{source = '3c286' ' \CR}{to select only the primary calibrator.}
\dispi{bchan 5; echan = $n-4$ \CR}{to omit edge channels in each IF,
       where $n$ is the number of channels in an IF, now usually 128.}
\dispi{nchav = echan-bchan+1; chinc nchav \CR}{to average all spectral
       channels into the image.}
\dispi{cellsize 1.\CR}{to set the image cell spacing --- depends on
       configuration.}
\dispi{imsi  1024 \CR}{to make a largish image to see sources around
       3C286.}
\dispi{outn '3C286\_quick' \CR}{ }
\dispi{niter 100 ; dotv 1 \CR}{to do some Cleaning and guide the
       progress with the TV}
\dispi{go \CR}{ }

\item\ If you wish to image Stokes Q and U, it is now necessary to
  calibrate the polarization.  This is still somewhat of a research
  question in \AIPS, but the advice in \Sec{polcal} may work.  You
  will need to run procedure {\tt \tndx{VLATECR}} to correct for
  Faraday rotation and dispersion, task {\tt \tndx{RLDLY}} to correct
  for the X minus Y delay offset, task {\tt \tndx{PCAL}} with {\tt
    SPECTRAL 1} to calibrate the polarization D terms, and then task
  {\tt \tndx{XYDIF}} to calibrate the X minus Y phase offset which is
  also a function of spectral channel.  {\tt PCAL} is best run with an
  unpolarized calibration source which is common at P band.  However,
  {\tt XYDIF} requires a calibration source with a useful flux in
  Stokes U\@.  3C345 has this, but, due to the rotation measure, not
  at all frequencies.  Significant frequency smoothing may help.

\item\ At this point, it may be simpler to {\tt \tndx{SPLIT}} the data
  set and apply calibration to the target sources which we then can
  work with one at a time.\Iodx{EVLA}
\dispi{defsult split \CR}{ }
\dispi{indi $d$; getn 6 \CR}{ }
\dispi{outdi indi \CR}{ }
\dispi{docal 1 ; doband 1 \CR}{ }
\dispi{dopol 1 \CR}{if you calibrated polarization}
\dispi{sources = '${\rm target}_1$', '${\rm target}_2$', $\ldots$
      \CR}{to select only target sources.}
\dispi{go \CR}{ }
\dispe{To create a number of separate, single-source $uv$ data sets
  each with separate target source.}

\item\ Flag RFI and bad data out of the target data set with two or
  more rounds of {\tt \tndx{RFLAG}}
\dispi{default rflag \CR}{ }
\dispi{indi $d$; getn 9 \CR}{ }
\dispi{avgchan 11 \CR}{Median window width on spectra.}
\dispi{fparm = 3, $x$, -1, -1 \CR}{to do a small rolling time buffer,
       set the sample interval, and to have {\tt NOISE} and {\tt
         SCUTOFF} set IF-dependent cutoff levels.}
\dispi{doplot 12 ; dotv 1 \CR}{To examine the two important plots on
       the TV.}
\dispi{go \CR}{ }
\dispe{and then apply the flags levels and determine new ones}
\dispi{doplot -12  ; go \CR}{ }
\dispe{and perhaps repeat this line for more flagging.  {\it NOTE:
  for flagging spectral-line target sources, you should set {\tt
    FPARM(4)=0} since you should not flag in the spectral domain.}
  Check what {\tt RFLAG} has done with {\tt \tndx{SPFLG}}, which can
  be tedious.}
\dispi{default spflg \CR}{ }
\dispi{indi $d$; getn 9 \CR}{to select the first of the split target
       data sets.}
\dispi{dparm(6) $x$ \CR}{to set integration time.}
\dispi{go \CR}{ }

\item\ Because we used {\tt TYAPL} only to apply the re-quantizer
  gains, your data weights simply reflect integration time.  For
  better imaging, weights should reflect the actual data
  uncertainties.  So re-weight your data with {\tt \tndx{REWAY}}
\dispi{default reway \CR}{\Iodx{EVLA}}
\dispi{indi $d$; getn 9 \CR}{ }
\dispi{outdi indi \CR}{ }
\dispi{aparm 31,0,500 \CR}{to find rmses from rolling buffers 31 time
       intervals long on a baseline basis and then smooth them over
       500 seconds.}
\dispi{go \CR}{ }
\dispe{Note that this applies the final {\tt SPFLG}+{\tt RFLAG} flag
       table.  Delete the other flag table versions with}
\dispi{default extdest \CR}{}
\dispi{indi $d$; getn 10 \CR}{}
\dispi{inext 'fg' ; invers = -1 \CR}{to delete all remaining versions
       of flag table.}
\dispi{extdest \CR}{}

\vfill\eject
\sects{P-band imaging in \AIPS}

\item\ The following steps relate to imaging your calibrated data. As
  the aim of this appendix is calibration, the imaging will be
  discussed only briefly here.  The following items refer to continuum
  imaging.  Make a dirty image of the center with {\tt
    \tndx{IMAGR}}\Iodx{EVLA}
\dispi{default imagr \CR}{ }
\dispi{indi $d$; getn 10; outdi indi \CR}{ }
\dispi{bchan 5; echan = $n-4$ \CR}{to omit edge channels in each IF,
       where $n$ is the number of channels in an IF.}
\dispi{nchav = echan-bchan+1; chinc nchav \CR}{to average all spectral
       channels into the image.}
\dispi{cellsize 10 \CR}{to set a cell size which will depend on the
       VLA configuration.}
\dispi{imsi 1024,1024 \CR}{ }
\dispi{niter 0 \CR}{to do no Cleaning}
\dispi{go \CR}{ }

\item\ Clean image with faceting, auto-boxing, and TV display; we use
{\tt DEFAULT IMAGR} here to initialize all of {\tt IMAGR}'s many
parameters  \label{it:imagr}\todx{SETFC}
\dispi{default imagr \CR}{ }
\dispi{indi $d$; getn 10; outdi indi \CR}{ }
\dispi{task 'setfc' \CR}{ }
\dispi{cellsize  0 ; imsize  0 \CR}{to have {\tt SETFC} determine the
       correct cell and image size for these data.}
\dispi{bparm  2.2, 5, 2, 8, 0.3 \CR}{to image fully out to a radius of
       2 degrees and to include 0.3 Jy NVSS sources out to a radius
       of 8 degrees.}
\dispi{flux 1 \CR}{to include exterior sources above 1 Jy only.}
\dispi{boxfile 'MYAREA:Target1.box \CR}{to list the facets found.}
\dispi{go \CR}{ }
\dispe{The following does not use default since we need adverbs set
  for, and by, {\tt SETFC}, such as inname, imsize, and cellsize
  \todx{IMAGR}}
\dispi{task 'IMAGR' \CR}{ }
\dispi{flux  0 \CR}{ }
\dispi{bchan 5; echan = $n-4$ \CR}{to omit edge channels in each IF,
       where $n$ is the number of channels in an IF.}
\dispi{nchav = echan-bchan+1; chinc nchav \CR}{to average all spectral
       channels into the image.}
\dispi{niter 5000 \CR}{to Clean 5000 iterations.}
\dispi{do3dim true ; overlap 2 \CR}{to use true 3-D geometry, Cleaning
       one facet at a time in each cycle (recommended).}
\dispi{dotv true \CR}{to control the imaging with the TV.}
\dispi{oboxfile boxfile \CR}{to save all changes to Clean boxes.}
\dispi{im2parm=2,0 \CR}{to allow up to 2 Clean boxes to be created
       automatically at each cycle.}
\dispi{go \CR}{ }
\dispe{Note, the field may have enough spectral index variation to
  require imaging (and self-calibration) of each spectral window (IF)
  individually.}

\vfill\eject
\item\ There is normally enough flux in a P-band field to do
  phase self-calibration and we recommend that you now do that.  Use
  {\tt CALIB} with the output of step~\ref{it:imagr} as the model
  including all {\tt NFIELD} facets as the {\tt NMAPS} images.
  \label{it:selfcal}\todx{CALIB}\Iodx{EVLA}
\dispi{default calib \CR}{}
\dispi{indi $d$; getn 10; outdi indi \CR}{ }
\dispi{nmaps = nfield \CR}{to include all facets}
\dispi{in2d = indisk; get2n $M$ \CR}{to set the $2^{\rm nd}$ file name
       to the output image of class {\tt ICL001} (catalog number
       $M$).}
\dispi{soltyp 'l1r' ; solmode 'p' \CR}{to start with phase-only
       solutions.}
\dispi{solint = $n x$ \CR}{to set a short integration time as $n$
       times the sampling interval.}
\dispi{go \CR}{}

\item\ Examine the output {\tt SN} table with {\tt EDITA} to see if
  there are times with bad solutions.  If not, repeat steps
  \ref{it:imagr} and \ref{it:selfcal} using
\dispi{tget imagr; docal = 1; go \CR}{to make new images.}
\dispi{tget calib; in2seq = 2 \CR}{to point at the second set of
       images.}
\dispi{solmode 'a\&p' \CR}{to do amplitude as well {\it if there is
       enough flux in the field}.}
\dispi{go \CR}{}
\dispe{Re-examine the new {\tt SN} table with {\tt EDITA}.  In
  particular, watch for periods of time when the amplitude gain of an
  antenna gets rather larger than normal.  It has been found that the
  ionosphere can defocus sources on scales as small as the antenna
  diameter.  It is best to delete such data.}

\item\ Spectral-line imaging is both simpler and more complicated.
  One may image each spectral channel individually.  However, even for
  absorption experiments, it may be best to image all channels free of
  the spectral line together to make a continuum image and model.
  This will almost certainly require multiple facets and extensive
  cleaning but it will produce a single continuum model that can be
  applied to all spectral channels with {\tt \tndx{UVSUB}}\@.
  Spectral index in the continuum may force you to do this one
  spectral window at a time.  You can add the continuum image back to
  the cube with {\tt COMB}\@.

  At this point, run {\tt IMAGR} on the output of {\tt UVSUB} one
  spectral window at a time over all spectral channels of interest
  with {\tt NCHAV=1; CHINC=1}\@.  You may be able to limit the field
  of view imaged to one facet surrounding the target source.

\item\ The task {\tt \tndx{TVSPC}} is a lovely way to explore the
  output image cube.  To run this task, we must first transpose the
  cube
\dispi{default trans \CR}{ }
\dispi{indi $d$; getn $C$; outdi indi \CR}{where $C$ is the catalog
       number of the output cube from {\tt IMAGR}\@.}
\dispi{transcod = '312' \CR}{to put frequency first.}
\dispi{go \CR}{to make a transposed cube.}
\dispe{One also needs a ``meaningful'' single-plane image, perhaps
  the central facet of the continuum image made above.  Or we can
  compute the first moment image.}
\dispi{default xmom \CR}{ }
\dispi{indi $d$; getn $T$; outdi indi \CR}{where $T$ is the catalog
       number of the output cube from {\tt TRANS}\@.}
\dispi{flux = $f$ \CR}{to compute only with brightness $> f$ Jy/beam.}
\dispi{go\CR}{to amke a first moment image.}
\dispe{Finally}
\dispi{default tvspc \CR}{ }
\dispi{indi $d$; getn $M$; outdi indi \CR}{where $M$ is the catalog
       number of the meaningful image, perhaps the output plane from
       {\tt XMOM}\@.}
\dispi{in2di $d$; get2n $T$ \CR}{where $T$ is the catalog number of
      the output cube from {\tt TRANS}\@.}
\dispi{in4di $d$; get4n $C$ \CR}{where $C$ is the catalog
       number of the output cube from {\tt IMAGR}\@.}
\dispi{go \CR}{Have fun!}
\dispe{This allows you to look at spectra from the cube at individual
  spatial pixels as well as the cube as an image at individual
  frequencies.  Enter {\tt EXPLAIN \tndx{TVSPC}} or see \AIPS Memo 120
  for details.}

\item\ Make image(s) and final calibrated data sets into FITS files
  with {\tt \tndx{FITTP}}\@.  Exit {\tt AIPS} and
\dispi{setenv FIT /lustre/mmao/Pband \CR}{for tcsh and c shells}
\dispi{export FIT=/lustre/mmao/Pband \CR}{for bash shell}
\dispe{Then start {\tt AIPS} again}
\dispi{default fittp \CR}{ }
\dispi{indi $d$; getn $M$ \CR}{to select an image or data set}
\dispi{dataout 'FIT:VirA.FITS \CR}{to make an appropriate output name}
\dispi{go \CR}{ }
\dispe{And repeat with new $M$ and {\tt DATAOUT} values.}

\end{enumerate}

\sects{Additional Recipes}

% chapter   *************************************************
\recipe{Banana storage}

      Bananas ripen after harvesting.  They do it best at room
temperature.  Because of this there are three stages to banana
storage.
\par\bre
\Item {{\bf On the counter:} When you buy a bunch of
   bananas that are not exactly at the ripeness you want, you
   can keep them at room temperature until they are just right for
   you.  Be sure to keep them out of any plastic bags or containers.}
\Item {{\bf In the refrigerator:} If there are any bananas left,
   and they are at the ripeness you like, you can put them in the
   refrigerator.  The peel will get dusty brown and speckled, but the
   fruit inside will stay clear and fresh and at that stage of
   ripeness for 3 to 6 days.}
\Item {{\bf In the freezer:} If you want to keep your bananas even
   longer, you can freeze them.  Mash the bananas with a little lemon
   juice, put them in an air tight freezer container and freeze.  Once
   they're defrosted, you'll go bananas baking bread, muffins and a
   world of other banana yummies.  Or, you can freeze a whole banana
   on a Popsicle stick.  When it is frozen, dip it in chocolate sauce,
   maybe even roll it in nuts, then wrap it in aluminum foil and put
   it back in the freezer.  Talk about a scrumptious snack.}
\ere
\vfill\eject
