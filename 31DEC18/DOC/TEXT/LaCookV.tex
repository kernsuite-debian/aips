%-----------------------------------------------------------------------
%;  Copyright (C) 2018
%;  Associated Universities, Inc. Washington DC, USA.
%;
%;  This program is free software; you can redistribute it and/or
%;  modify it under the terms of the GNU General Public License as
%;  published by the Free Software Foundation; either version 2 of
%;  the License, or (at your option) any later version.
%;
%;  This program is distributed in the hope that it will be useful,
%;  but WITHOUT ANY WARRANTY; without even the implied warranty of
%;  MERCHANTABILITY or FITNESS FOR A PARTICULAR PURPOSE.  See the
%;  GNU General Public License for more details.
%;
%;  You should have received a copy of the GNU General Public
%;  License along with this program; if not, write to the Free
%;  Software Foundation, Inc., 675 Massachusetts Ave, Cambridge,
%;  MA 02139, USA.
%;
%;  Correspondence concerning AIPS should be addressed as follows:
%;          Internet email: aipsmail@nrao.edu.
%;          Postal address: AIPS Project Office
%;                          National Radio Astronomy Observatory
%;                          520 Edgemont Road
%;                          Charlottesville, VA 22903-2475 USA
%-----------------------------------------------------------------------
\setcounter{chapter}{21}
\APPEN{VLA-specific programs in \AIPS}{VLA Maintenance Tasks in \AIPS}{VLAtasks}
\renewcommand{\Chapt}{35}

\renewcommand{\titlea}{31-December-2018 (revised 24-August-2018)}
\renewcommand{\Rheading}{\AIPS\ \cookbook:~\titlea\hfill}
\renewcommand{\Lheading}{\hfill \AIPS\ \cookbook:~\titlea}
\markboth{\Lheading}{\Rheading}

Most tasks in \AIPS\ are designed to work for a generalized
interferometer or single-dish.  However, a number of tasks have been
written specifically at the behest of scientists supporting the
calibration and maintenance of the VLA and \indx{EVLA}\@.  These include
a task to read the ``telcal'' text file to prepare an initial solution
({\tt SN}) table, a task to solve for antenna position corrections
from appropriate solutions in an {\tt SN} table, a task to fit
corrections to the antenna pointing model from a file produced by the
on-line system in pointing mode, and several tasks to process
holography data up to and including a map of the corrections to be
applied to the antenna panels.  This appendix is an attempt to
document such tasks.  Note that {\tt TLCAL} is a task which may be of
interest to general VLA users.  The antenna location task {\tt LOCIT}
and the holography tasks may well be of interest to scientists
maintaining interferometers other than the VLA\@.  However, the
pointing task {\tt PEEK} will probably be of use only to people tasked
with maintaining the VLA\@.

\Sects{Real-time calibration tables: {\tt \Tndx{TLCAL}}}{TLCAL}

The VLA on-line system determines antenna-based complex gain and delay
values for the average of every calibrator scan.  These are written
into a text file which is stored at the VLA site on a computer named
{\tt mchammer}.  The actual file name is\\
\centerline{{\tt /home/mchammer/evladata/telcal/{\it year}/{\it
      month}/{\it asdmfile}.GN}}\\
where {\it year} is the four-digit year, {\it month} is the two-digit
month, and {\tt asdmfile} is the long complicated name assigned to the
particular scheduling block.  It is possible to read such files
directly from computers in Socorro and {\tt 31DEC18} task {\tt TLCAL}
supports such access.  The text file may also be copied by, for
example, the analysts and made available to users in Socorro and
elsewhere.  The gains in the telcal file all assume that every source
is 1.0 Jy in all spectral windows.  If you have run {\tt SETJY} on
your data set before using {\tt TLCAL}, then you can adjust the gains
of the known flux calibrators using the fluxes stored in the source
({\tt SU}) table.  In this case, the {\tt SN} table produced will be
ready for use with {\tt GETJY}\@.

To run {\tt TLCAL} on a file in your data area, enter
\dispt{TASK\qs 'TLCAL' \CR}{}
\dispt{INDI\qs {\it n\/} ; GETN {\it m\/} \CR}{to specify the data
       set.}
\dispt{DOAPPLY\qs 1 \CR}{to use the fluxes in the source table.}
\dispi{ASDMFILE(1) = 'MYAREA: \CR}{two parts are concatenated.}
\dispil{ASDMFILE(2) = 'TSUB0001.sb23643640.eb23666175.56454.14034020834
       \CR}{}
\dispt{INP \CR}{to check that you remembered to leave off the close
       quotes above.}
\dispt{GO \CR}{to run the task.}
\dispe{Substitute your file name parameters for those used above and
  leave off the {\tt .GN} part of the name and the close quotes (to
  preserve your desired case).  It would be wise to check the {\tt SN}
  table produced with, \eg\ {\tt SNPLT}\@ .  Bad RFI may render some
  telcal values bogus.}

The Telcal file is text file with very long lines beginning with 3
header lines.  Then there are columns for MJD (to high accuracy), UTC
(sexagesimal), LST (days), LST (sexagesimal), IFID (5-character
string), Sky frequency (MHz), Antenna ID (\eg\ {\tt ea03}), gain
amplitude, phase(degrees), residual, delay (ns), flagged (true/false),
zeroed (true/false). hour angle (radians), azimuth (radians),
elevation (radians), source name, and flag reason.  {\tt TLCAL} uses
the MJD, antenna, amplitude, phase, delay, flagged, and source
columns.  A combination of the IFID (peculiar to the EVLA) and the sky
frequency is used to identify the IFs to which these apply.

\vfill\eject
\Sects{Correcting antenna positions: {\tt \Tndx{LOCIT}}}{LOCIT}

     The data taken for pointing analysis after antennas have been
moved is now also used to determine corrections to the assumed antenna
locations.  The data are loaded into \AIPS\ using {\tt BDF2AIPS} in
the usual way, but {\tt DOKEEP = 1} is required in order to load
pointing data.  Then {\tt FRING} is run on one scan from a strong
source with {\tt SOLINT = 5} minutes and with no rates ({\tt DPARM(9)
= 1})\@.  Then {\tt CLCAL} is used to apply the {\tt SN} table
to all times in the {\tt CL} table.  Then the data set is reduced in
size by running {\tt SPLAT}, averaging in frequency and time:\iodx{EVLA}
\dispt{DEFAULT SPLAT \CR}{to initialize the adverbs}
\dispt{INDISK\qs {\it n\/} ; GETN {\it m\/} \CR}{to specify the data
       set.}
\dispt{OUTDISK\qs INDISK \CR}{to keep on same disk.}
\dispt{APARM\qs 3, 1, 0, 0, 1 \CR}{to average data in frequency, but
       not over IF\@.  The input data are taken at 1-second
       interval.}
\dispt{DOCAL\qs 1 ; GAINU\qs 0 \CR}{to apply the calibration.}
\dispt{CHANNEL\qs 64 ; CHINC\qs 0 \CR}{to average all spectral
       channels together.}
\dispt{SOLINT\qs 5/60 \CR}{to average over time to 5-second
       intervals.}
\dispt{INP \CR}{to check the inputs.}
\dispt{GO \CR}{to make a rather smaller data set.}
\dispt{INDISK\qs {\it n\/} ; GETN {\it j\/} \CR}{to specify the averaged
       data set.}
\dispt{INEXT\qs 'NX' ; INVERS\qs 0 ; EXTDEST \CR}{to delete the index
       table which has each direction of the pointing as a separate
       scan.}
\dispt{DEFAULT 'INDXR'}{to initialize everything.}
\dispt{INDISK\qs {\it n\/} ; GETN {\it j\/} \CR}{to specify the averaged
       data set.}
\dispt{CPARM\qs 0 , 10 \CR}{to set maximum scan length to 10 minutes.}
\dispt{GO \CR}{to make the new index table.}
\dispe{It might be wise to make a backup copy of the data set at this
point in case later machinations go wrong.  Use {\tt UVCOP}\@.}

     A solution table is now needed containing the phases of every
scan in the data set.  Use
\dispt{DEFAULT CALIB \CR}{to initialize the task and adverbs.}
\dispt{INDISK\qs {\it n\/} ; GETN {\it j\/} \CR}{to specify the averaged
       data set.}
\dispt{SOLINT\qs 10 \CR}{to average full scans.}
\dispt{APARM\qs 3, 0 \CR}{to require at least 3 antennas, but do {\it
       not} average polarizations or IFs.}
\dispt{SOLTYPE\qs 'L1R' ; SOLMODE\qs 'P' \CR}{to solve for phases only
       with recursive L1 methods.}
\dispt{DOCALIB\qs 1 ; GAINUS\qs 0 \CR}{to apply calibration, a no-op
       at this stage.}
\dispt{SNVER\qs 0 ; REFANT\qs {\it nn} \CR}{to create a new solution
      table and to select a reference antenna ({\it nn})}
\dispt{INP \CR}{to double check.}
\dispt{GO \CR}{to make the solution table needed next.}

Finally, fit the antenna positions.  It is best to do each IF
separately and also try the mode that uses the phase difference
between IFs.
\dispt{DEFAULT LOCIT \CR}{to initialize things including {\tt DOTV}
       false, and {\tt REFANT 0}.}
\dispt{INDISK\qs {\it n\/} ; GETN {\it j\/} \CR}{to specify the averaged
       data set.}
\dispt{BIF\qs 1 ; EIF\qs BIF \CR}{to examine one IF at a time.}
\dispt{STOKES\qs 'RR' \CR}{to examine one polarization at a time.}
\dispt{DPARM\qs 0, 1, 10, 120 \CR}{to use each scan by itself (rather
       than differencing consecutive scans), do not fit the {\tt K}
       term, and limit the fit to elevations between 10 and 120
       degrees.  All hour angles and declinations are used.  The total
       phase is used.}
\dispt{BPARM\qs 3, 0, -180, 180, 3, 4, 1, 3 \CR}{to plot 3 antennas
       per page between -180 and 180 degrees as boxes with connecting
       lines, and, if doing more than 1 IF or polarization, plot them
       in the same plot.}
\dispt{DOOUTPUT\qs -1 \CR}{to ignore any values contained in
       pre-existing files named {\tt OUTPRINT.}{\it arrayname}.
       If this file exists before running {\tt LOCIT} and {\tt
       DOOUTPUT} $> 0$, it will be read and the corrections contained
       within it added to those found before they are written to a new
       {\tt OUTPRINT.}{\it arrayname} file and also to the other
       output files.  See {\us HELP LOCIT \CR} for details.}
\dispt{OUTPRINT\qs 'FITS:{\it date}' \CR}{to set the output file base
      {\it name} used for output text files.}
\dispt{INP \CR}{to check that all is well.}
\dispt{GO \CR}{to run the task.}
\dispe{Then do}
\dispt{STOKES\qs 'LL' ; GO \CR}{to redo with the other Stokes.}
\dispt{BIF\qs 2; EIF\qs BIF \CR}{to redo with the second IF.}
\dispt{STOKES\qs 'RR' ; GO \CR}{to redo with the other Stokes.}
\dispe{Now compare the answers and uncertainties found in each of the
four runs.  Also it may be good to do}
\dispt{BIF\qs 1; EIF\qs 2 \CR}{to include both IFs.}
\dispt{STOKES\qs 'RR' \CR}{to do one Stokes at a time.}
\dispt{BPARM(9)\qs 1 \CR}{to select the mode which uses the phase
      difference between the 2 IFs.}
\dispt{GO \CR}{to get a fifth estimate of the antenna position
      corrections.}
\dispe{Repeat for {\tt STOKES 'LL'}\@.}

Now comes the hard part.  {\tt \Tndx{LOCIT}} will have created a great
many plot files, an example of which is shown in
Figure~\ref{fig:locit}.  It also writes text files named {\tt
OUTPRINT} with extensions:\iodx{EVLA}\\
\centerline{\begin{tabular}{l@{\extracolsep{2em}}p{4.5in}}
{\tt FIT} & {This file contains all solutions with their uncertainties
in a form that is easy to read.  It may be used to accumulate
corrections depending on the setting of {\tt DOOUTPUT}.}\\
{\tt EVLA} & {This file contains only significant corrections in a
form that can be hand edited to put in the dates when the antennas
move and when the database is fixed.  The edited result may be edited
to a system file on the web to enable {\tt VLANT} to apply the
appropriate corrections.}\\
{\tt PAR} &{This file, containing only significant corrections, is
read into Parminator to update the database.}\\
{\tt 001} & {This file contains only significant corrections in a
script form to read into {\tt CLCOR}.  Do this, to make a new {\tt CL}
table and then rerun {\tt CALIB} to see if the resulting phases really
end up near zero.}
\end{tabular}}
You will need to review the plots ({\tt TVPL} is helpful) and study
the {\tt .FIT} file to decide which antennas have significant
corrections that need to be made.  You will almost certainly need to
edit the {\tt .001} file and then apply it to your averaged data set.
Repeat {\tt CALIB} and then examine the resulting {\tt SN} table with
{\tt EDITA} or {\tt SNPLT}\@.  If things worked well, then edit the
{\tt .PAR} and {\tt .EVLA} files to match.

\begin{figure}
\centering
\resizebox{\hsize}{!}{\gbb{540,405}{locit}}
\caption[Example {\tt LOCIT} plot]{Example of the residual phases
after a {\tt \Tndx{LOCIT}} solution is applied to the input {\tt
SN}-table phases.  The solutions for antennas 14 and 15 are fine, but
antenna 16 did not converge to a correct solution.  The data from both
IFs and both polarizations were used in this run of {\tt LOCIT} and
are represented by different colors in the plot.\iodx{EVLA}}
\label{fig:locit}
\end{figure}

\vfill\eject
\Sects{Improving antenna pointing: {\tt \Tndx{PEEK}}}{PEEK}

When the EVLA on-line system conducts a pointing observation, be it
for a long series of such observations after antenna moves or a
single observation to be applied to the next target source, the
software updates a text file with extension {\tt .PNT}\@.  These files
may be copied from the computers at the VLA and then analyzed to
determine increments to the parameters of the pointing model.  This
analysis was done with a program called {\tt peek} available from Ken
Sowinski's disk area.  In {\tt 31DEC18}, that program has been adapted
to run as an \AIPS\ task.  The method of setting the input parameters
was changed to be the familiar \AIPS\ methods, the program was given
the option to make \AIPS-style plots in addition to the {\tt gnuplot}
files, and has the option to suppress some of the output text files.

An example usage would be
\dispt{DEFAULT PEEK \CR}{to initialize the task name and all adverbs.}
\dispt{INDISK\qs {\it n\/} ; GETN {\it m\/} \CR}{to specify an arbitrary
       catalog entry to which plot files are attached.  It may be
       omitted if {\tt DOPLOT}$ \leq 0$ or {\tt DOTV = 1}.}
\dispi{ASDMFILE(1) = 'MYAREA:peek/ \CR}{two parts are concatenated.}
\dispi{ASDMFILE(2) = 'PX8262 \CR}{file renamed to useful name size}
\dispt{DOPLOT\qs 1 ; DOTV\qs 1 \CR}{to make \AIPS\ plots on the TV
       display.}
\dispt{INP \CR}{to check the inputs.}
\dispt{GO \CR}{to run the task.}
\dispe{Do not include the {\tt .PNT} extension in {\tt ASDMFILE}; the
task will add it.  Adverb {\tt APARM} is used to set data and output
selection parameters.  The defaults are to include all dates, to
include elevations between 7 and 122 degrees, to include all samples
with amplitude $> 0.01$, to omit samples with wind speed $> 5$ m/s, to
include normalized beamwidths between 0.95 and 1.15, and to write the
filtered change file only when the change is $> 0.1$ arc minutes and
$> 3$ times the uncertainty.\iodx{EVLA}}

Adverb {\tt FPARM} is used to determine which parameters the task
determines, with the default being the usual set (Tilt, Azzero, Elzero,
Colima, and Refrac).  {\tt FPARM} can be, but is normally not, used to
attach the tilt to the antenna rather than the pad and to suppress the
writing of various output text files.  These text files are written to
the disk area which is the ``root'' directory of the {\tt ASDMFILE}
adverb --- {\tt MYAREA:peek/} in the example above.  The output files
are named with {\tt ASDMFILE} and extensions {\tt .PTR} (large overall
summary), {\tt .gp} ({\tt gnuplot} plotting instructions), {\tt
.changes} (all of the adjustments found), and {\tt .filtchg} (those
adjustments that are deemed significant).  In addition, files are
written named {\tt ant$nn$.dat} that include all relevant parameters
for all antennas which have been fit and which are used by {\tt
gnuplot}.

You can use the {\tt .gp} file by typing {\tt gnuplot} followed by the
name of the {\tt .gp} file.  It will make a file named {\tt plot.ps}.
Adverb {\tt DOPLOT} may be used to request \AIPS-style plots with
substantially better control, using {\tt DPARM}, over symbol types,
sizes, and colors.  In both styles of plotting, six plots appear on
each plot page showing measured and residual azimuth and elevation
corrections plotted against azimuth, elevation, and relative time.
You may plot these on the TV or make plot files, one for each antenna
fit.  If using the TV, button {\tt D} tells the task to stop plotting
but continue doing all the fits and text file outputs.  A new
procedure to set colors for {\tt LWPLA} has been written called {\tt
OKCOLORS}\@.  It sets the colors to useful values for plotting on a
white background.  Black backgrounds use a lot of printer ink and so
are less desirable when printing many plots.

\vfill\eject
\Sects{Improving antenna surfaces: \Indx{holography}}{holography}

     In ``holography'' data from the \indx{EVLA}, reference antennas
are pointed towards the nominal source position, as normal.  All
other antennas are pointed to an offset position controlled by
the observer.  The angular offset from the nominal position, for
each antenna, is written to data sets in the same variables that are
normally used for $u,v,w$.  In this mode, only data on a baseline
between a reference antenna and an offset (non-reference) is
considered useful.  The correlations observed on these baselines
are stored in the multi-source file in the normal way.  Data are
calibrated as usual and then extracted with {\tt \Tndx{UVHOL}} for
holographic analysis.  This analysis includes determination of the
shapes of the primary beams with {\tt \Tndx{PBEAM}}, Fourier analysis
with {\tt \Tndx{HOLOG}} to make images of the dish surface including
the deviations from the desired shape, and finally {\tt \Tndx{PANEL}}
to prepare detailed prescriptions for adjusting the antenna panels.
Verb {\tt \Tndx{TVLAYOUT}} plots an antenna panel layout on top of a
TV holography image and has the layouts for the VLA and VLBA built in.

     Holography data are taken by driving the moving antennas to a
desired offset and then dwelling for a period of time, accumulating
some number of samples.  Then the moving antennas are driven to the
next desired offset, usually horizontally or vertically, and the
process is repeated.  Holography samples are recorded with the $u$
random parameter as the azimuth offset direction cosine, the $v$
random parameter as the elevation direction cosine, and the $w$ random
parameter as a counter of the dwell interval.  In the old VLA, the $w$
had an additional 40000.0 added as a holography indicator.  With the
EVLA, the source name is the sole method of discriminating between
regular visibility data taken for calibration and the holography data.

     {\tt \Tndx{UVHOL}} has three main modes of operation: printing
the data much like {\tt UVPRT}, averaging and writing the data to text
files for analysis by {\tt HOLOG}, and plotting the data in \AIPS\
plot files or on the TV display.  {\tt UVHOL} has all of the usual
data selection and calibration adverbs.  In addition it has a number
of basic holography-specific adverbs.  {\tt ANTENNAS} is used to list
the reference antennas and {\tt BASELINE} is used to list the moving
antennas.  Since it takes some time for the antennas to settle down
after each move, adverbs {\tt NPOINTS} or {\tt APARM(1)} through {\tt
APARM(3)} control which samples within each pointing are retained.
{\tt APARM} also controls whether the data are averaged over all
moving antennas and/or all reference antennas and/or over time.
Normally, holography data should not have the parallactic angle
correction applied on {\tt DOPOL}, but it can be done if desired.
{\tt DPARM} is used to advise the print routines on scaling of data
and weights and also to invoke special plotting options (use db for
amplitudes, plot phase or imaginary rather than amplitude/real).
There are a number of the usual plot options, such as {\tt SYMBOL},
{\tt DO3COL}, {\tt LTYPE}, and the like.  {\tt DOCRT} controls where
the print out is sent, either to the terminal or to {\tt OUTPRINT}\@.

     On {\tt OPTYPE='HOLG'} mode, the holography data are written to
text files, a number of files for each pair of antennas.  Each line of
these files, after a number of header lines, contains l, m, amplitude
or real, phase or imaginary, error in amplitude or real, and error in
phase or imaginary.  In this mode, {\tt OUTPRINT} specifies the
logical directory name of the output files which are then named with
an additional {\tt HOLO{\it nn}-{\it mmppii}} where {\it nn} is the
reference antenna ({\tt BASELINE({\it j})} or 0 if multiple), {\it
mm} is the antenna (or 0 if multiple), {\it pp} are 2 letters giving
the polarization, and {\it ii} is the IF\@.  These files are then
given to {\tt HOLOG}\@.

An example of plotting one short raster at 4.8 GHz has inputs
\dispt{DEFAULT UVHOL \CR}{to initialize the task name and all adverbs.}
\dispt{INDISK\qs {\it n\/} ; GETN {\it m\/} \CR}{to specify the
       holography data set.}
\dispt{SOURCES\qs 'HOLORASTER' ' \CR}{to specify the standard
       holography source name.  Required with EVLA data.  Note the
       extra quote mark at the end which makes all the other values of
       {\tt SOURCES} blank.}
\dispt{STOKES\qs 'HALF'' \CR}{to plot RR and LL.}
\dispt{ANTENNAS\qs 9,0 \CR}{to specify one reference antenna.}
\dispt{BASELINE\qs 8,0 \CR}{to specify one moving antenna.}
\dispt{ICHANSEL 5, 60 \CR}{to omit outer channels from the average.}
\dispt{DOCALIB\qs 1 ; DOBAND\qs 1 \CR}{to apply the calibration.}
\dispt{APARM\qs 1, 3, 2 \CR}{to include all but the first 3 samples
       and last 2 in each pointing with no averaging.}
\dispt{DPARM\qs 0 0 0 0 0 1 0 \CR}{to plot amplitudes in db.}
\dispt{DO3COLOR\qs 1 \CR}{to plot the two polarizations in different
       colors.}
\dispt{OPTYPE\qs 'PLOT' \CR}{to make plots rather than output text.}
\dispt{INP \CR}{to check the inputs.}
\dispt{GO \CR}{to make the plot file shown in Figure~\ref{fig:uvhol}.}

\begin{figure}
\centering
\resizebox{\hsize}{!}{\gbb{525,403}{uvhol}}
\caption[Example holography raster]{Example of the amplitudes measured
in a single holography scan as plotted by {\tt \Tndx{UVHOL}}\@.
Antenna 8 moved in the {\tt L} coordinate while antenna 9 remained
pointed directly at the source.  Both polarizations are plotted with
multiple samples at each pointing (not averaged).}
\label{fig:uvhol}
\end{figure}

Now, to continue the holography analysis, we write out the data in
appropriate text files.  An example, would be\iodx{EVLA}
\dispt{TGET\qs UVHOL \CR}{to get the previous parameters for {\tt
       UVHOL}\@.}
\dispt{OPTYPE\qs 'HOLG' \CR}{to set the holography mode.}
\dispt{OUTPRINT\qs 'MYAREA:' \CR}{to set the logical directory for the
       output text files.}
\dispt{ANTENNAS\qs 1, 5, 9, 17 \CR}{to specify more of the reference
       antennas.}
\dispt{APARM\qs 1, 3, 2, 0, 1, 1 \CR}{to include all but the first 3
       samples and last 2 in each pointing averaging over time within
       the pointing and over all reference antennas.}
\dispt{INP \CR}{to double check the inputs.}
\dispt{GO \CR}{to write out the data for {\tt HOLOG}\@.}
\dispe{This will write a file named {\tt MYAREA:HOLO00-08RR01} and one
named {\tt MYAREA:HOLO00-08LL01}.  Note that we did not specify the
one IF to be used, so IF number one was what we got. \Todx{UVHOL}}

     {\tt \Tndx{HOLOG}} reads the visibility data from the text file
produced by {\tt UVHOL}\@.  It then Fourier transforms the data to
produce the grading in the aperture plane of the antenna.  Images of
the re-gridded observed amplitude and phase ({\tt A\_AMP}, {\tt
A\_PHA}), of the weights in the re-gridding ({\tt WGT}), of the derived
amplitude and phase of the illumination in the antenna aperture ({\tt
V\_AMP}, {\tt V\_PHA}), of the amplitude and phase of the point-spread
function ({\tt P\_AMP}, {\tt P\_PHA}), of the phase corrections removed
by the focus model ({\tt MODEL}), of the surface deviations of the
antenna ({\tt V\_DEV}) in meters, and of the interpolated antenna
power pattern ({\tt A\_PWR}) may be produced under control of adverb
{\tt DPARM}\@.  Task {\tt PANEL} requires {\tt V\_AMP} ({\tt
DPARM(4)}) for the mask and {\tt V\_DEV} ({\tt DPARM(9)} for the
deviations.  An example of the images produced by {\tt HOLOG} is shown
in Figure~\ref{fig:holog}.

\begin{figure}
\centering
\resizebox{\hsize}{!}{\gbb{540,515}{holog}\hspace{0.5cm}\gbb{540,509}{hologdev}}
\caption[Example {\tt HOLOG} images]{Example of the mask ({\tt V\_AMP})
and deviation ({\tt V\_DEV}) images produced by {\tt HOLOG}\@.  Task
{\tt GREYS} can do both the pseudo-coloring with {\tt OFMFILE =
'RYPHLAME'} and the panel layout with {\tt INFILE = 'VLA'}\@.}
\label{fig:holog}
\end{figure}

Many of the control adverbs must be set to appropriate values.  A
sample set of inputs is\iodx{EVLA}
\dispt{DEFAULT HOLOG \CR}{to initialize everything.}
\dispt{INFILE\qs 'MYAREA:HOLO00-28RR01 \CR}{to point at the text file,
       28 is the moving antenna with several reference antennas.}
\dispt{OUTNAME \qs '28-Q2-40935E' \CR}{to name the output images.}
\dispt{OPTYPE\qs 'SUBR' \CR}{to specify the type of antenna,
       sub-reflector, sub-reflector with reference pointing, or primary
       focus.}
\dispt{FACTOR\qs 8.8 \CR}{to set magnification factor for sub-reflector
       antennas; default 10.0.}
\dispt{APARM\qs 0, -1, 25, 3.4, 9 \CR}{to set antenna outer diameter,
       inner diameter, and focal length in meters.}
\displ{BPARM\qs 30, 256, 0, 1, 1, -2, -2.5, -12, 100, -1 \CR}{to set
       data reduction parameters.}
\dispt{DPARM\qs 0 0 0 1 0 1 0 1 1 0 \CR}{to request {\tt V\_AMP}, {\tt
       P\_AMP} , {\tt MODEL}, and {\tt V\_DEV} images.}
\dispt{INP \CR}{to double check everything.}
\dispt{GO \CR}{to make the images.}
\dispe{{\tt BPARM} specifies the required image size in meters, the
image size in pixels, minimum and maximum antenna scan angle,
amplitude scaling factor, Fourier transform control (-2 says use DFT
with VLA sign convention), minimum and maximum coordinate used in
correcting for pointing, focus, and feed offset (negative says in the
radial coordinate), a control parameter (100 inhibits local phase
ambiguity correction in the {\tt V\_PHA} plane), and type of data (-1
for VLA).  {\tt CPARM} all zero says to use the default spheroidal
function in gridding on both {\tt L} and {\tt M} axes; highly
recommended, but not relevant when using DFT imaging.  Note that {\tt
\Tndx{HOLOG}} can also generate images from a feed model using {\tt
VPARM}\@.}

The last task in the process of measuring corrections to the antenna
surface is {\tt \Tndx{PANEL}}
\dispt{DEFAULT PANEL \CR}{to initialize everything.}
\dispt{INDISK\qs {\it n\/} ; GETN {\it m\/} \CR}{to specify the {\tt
       V\_DEV} image.}
\dispt{IN2DI\qs {\it i\/} ; GET2N {\it j\/} \CR}{to specify the {\tt
       V\_AMP} image used as a mask.}
\dispt{OUTFILE\qs 'MYAREA:PANEL.out \CR}{to specify the text file to
       receive the panel adjustment instructions.}
\dispt{DOPLOT\qs 7 \CR}{to request plots of the input deviations, the
       panel adjustments, and the residuals.}
\dispt{APARM\qs 2, 1, 0, 0, 42 \CR}{To set the panel model, the clip
       level in the mask image, and the number of pointings.  The map
       size and frequency should be available from the image header.}
\dispt{INLIST\qs '\qs' \CR}{to use the default VLA panel layout.}
\dispt{INP \CR}{to review the inputs.}
\dispt{GO \CR}{to fit the data, write the adjustment instructions, and
       make plots.}
\dispe{Example plots of the panel adjustments and residual are shown
in Figure~\ref{fig:panel}.\iodx{EVLA}}

\begin{figure}
\centering
\resizebox{\hsize}{!}{\gbb{540,566}{panel}\hspace{0.5cm}\gbb{540,566}{panelres}}
\caption[Example {\tt PANEL} images]{Example of the panel adjustment
image and the residual image computed and plotted by {\tt
\Tndx{PANEL}}\@.}
\label{fig:panel}
\end{figure}

\vfill\eject

     With appropriate holography data, the output of {\tt
\Tndx{UVHOL}} may also be used to fit an analytic function to the
primary beam.  The task that does this is called {\tt PBEAM} and it
fits a polynomial function and makes contour plots of the input data,
the fit model, the residual, and {\tt IRING}-like plots comparing data
and model in two colors, and then {\tt IRING}-like plots of the
residual image and residual image divided by  the model.

Sample inputs could contain
\dispt{DEFAULT PBEAM \CR}{to initialize everything.}
\dispt{OUTNAME\qs 'PBEAM Cband  \CR}{to make output images of the
       data, model, and residual.  Required if {\tt DOTV} is false and
       {\tt DOPLOT} is $> 0$.}
\dispt{INFILE\qs 'MYWORK:C-6052I\_1' \CR}{to provide the text file
       from {\tt UVHOL}.}
\dispt{OUTFILE\qs 'MYWORK:CbandTxt  \CR}{to save the messages
       including answers to a text file.}
\dispt{VPARM\qs 0 \CR}{to use all defaults in the fitting.}
\dispt{DOPLOT\qs 7 \CR}{to get all possible plots.}
\dispt{OPTYPE\qs 'ELLI' \CR}{to fit eccentricity and position angle.}
\dispt{INP \CR}{to double check everything.}
\dispt{GO \CR}{to do the fit and write out images and plots.}
\dispe{Sample plots are shown in Figure~\ref{fig:pbeam}.\iodx{EVLA}}

\begin{figure}
\centering
\resizebox{\hsize}{!}{\gbb{540,559}{pbeamresid}\hspace{0.5cm}\gbb{540,410}{pbeamiring}}
\caption[Example {\tt PBEAM} plots]{Example of the plots from {\tt
\Tndx{PBEAM}}: contour plot of the residual image and {\tt IRING}-like
plot of the residual image divided by the model.}
\label{fig:pbeam}
\end{figure}
