%-----------------------------------------------------------------------
%;  Copyright (C) 1995, 1997-1998, 2000-2018
%;  Associated Universities, Inc. Washington DC, USA.
%;
%;  This program is free software; you can redistribute it and/or
%;  modify it under the terms of the GNU General Public License as
%;  published by the Free Software Foundation; either version 2 of
%;  the License, or (at your option) any later version.
%;
%;  This program is distributed in the hope that it will be useful,
%;  but WITHOUT ANY WARRANTY; without even the implied warranty of
%;  MERCHANTABILITY or FITNESS FOR A PARTICULAR PURPOSE.  See the
%;  GNU General Public License for more details.
%;
%;  You should have received a copy of the GNU General Public
%;  License along with this program; if not, write to the Free
%;  Software Foundation, Inc., 675 Massachusetts Ave, Cambridge,
%;  MA 02139, USA.
%;
%;  Correspondence concerning AIPS should be addressed as follows:
%;          Internet email: aipsmail@nrao.edu.
%;          Postal address: AIPS Project Office
%;                          National Radio Astronomy Observatory
%;                          520 Edgemont Road
%;                          Charlottesville, VA 22903-2475 USA
%-----------------------------------------------------------------------
\chapts{Calibrating Interferometer Data}{cal}

\renewcommand{\titlea}{31-December-2018 (revised 24-August-2018)}
\renewcommand{\Rheading}{\AIPS\ \cookbook:~\titlea\hfill}
\renewcommand{\Lheading}{\hfill \AIPS\ \cookbook:~\titlea}
\markboth{\Lheading}{\Rheading}

     This chapter focuses on ways to do the initial \Indx{calibration}
of interferometric fringe-visibility data in \AIPS\@.  The sections
which follow concentrate primarily on calibration for
connected-element interferometers, especially the Karl G. Jansky very
Large Array, called the EVLA hereafter.  However, the information in
these sections is useful for data from the historic VLA and other
interferometers and to spectral-line, solar, and VLBI observers as
well.  For specific advice on the historic VLA, consult
\Rappen{Olderdata}. while the original appendix on EVLA data reduction
has been deleted.  For the gory details of VLBI, read \Rchap{vlbi}.
After the initial calibration has been completed, data for sources
with good signal-to-noise are often taken through a number of cycles
of imaging with self-calibration.  See \Sec{selfcal} for information
on these later stages of the reduction process.  For accurate
calibration, you must have accurate {\it a priori\/} positions and
structural information for all your calibration sources and accurate
flux densities for at least one of them.  It is best if the
calibration sources are unresolved ``point'' sources, but it is not
required.

     For the basic calibrations, visibility (``{\it uv\/}'') data are
kept in ``\indx{multi-source data sets},'' each of which contains, in
time order, visibility data for one or more ``unknown'' sources and
one or more calibration sources.  Associated with these data are
``extension'' files containing tables describing these data.
\iodx{extension files}  When VLA archive data are first read into
\AIPS\ a number of basic \indx{tables} are created and filled with
information describing the data set.  A complete set of these would
include \iodx{antenna file}\iodx{EVLA}
\xben
\Item {\tt AN} (antennas) for sub-array geometric data, date,
     frequency, polarization information, {\it etc.\/},
\Item {\tt CD} (CalDevice) for noise tube values,
\Item {\tt CL} (calibration) for calibration and model information,
\Item {\tt CT} (Calc) for astrometric data used in the correlation,
\Item {\tt FG} (flag) for flagging (editing) information,
\Item {\tt FQ} (frequency) for frequency offsets of the different
     IFs (IF pairs in VLA nomenclature),
\Item {\tt GC} (gain curve) for nominal sensitivity and antenna gain
     functions,
\Item {\tt HI} (history) for history records,
\Item {\tt NX} (index) to assist rapid access to the data,
\Item {\tt SU} (source) for source specific information such as
     name, position, velocity,
\Item {\tt SY} (SysPower) for system gains and measured total power
     with the noise tubes on and with them off, and
\Item {\tt WX} (Weather) for weather data.
\xeen
\par\noindent An initial {\tt CL} table contains gains due to known
antenna functions of elevation and measured atmospheric opacities.
VLBI, and especially VLBA, data sets will end up with even more table
files.  Calibration and editing tasks then create, as needed, other
tables including
\xben
\setcounter{enumi}{12}
\Item {\tt BL} (baseline) for baseline-, or correlator-,
     dependent corrections,
\Item {\tt BP} (bandpass) for bandpass calibration,
\Item {\tt PD} (Polarization) for spectral dependent polarization D
     terms,
\Item {\tt SN} (solution) for gain solutions from the calibration
     routines.
\xeen
\noindent All of these tables can be written to FITS files along with
the visibility data using {\tt FITTP} or {\tt FITAB}\@.  They can be
read back in with {\tt FITLD}\@.  These, and any other, \AIPS\ tables
can be manipulated and examined using the general tasks {\tt
\tndx{PRTAB}}, {\tt \tndx{TACOP}}, {\tt \tndx{TABED}}, {\tt
\tndx{TAMRG}}, {\tt \tndx{TASRT}}, \hbox{{\tt \tndx{TAFLG}} and
{\tt \tndx{TAPPE}}}.

     The visibility data within the multi-source data set are not
normally altered by the calibration tasks.  Instead, these tasks
manipulate the tabular information to describe the calibration
corrections to be applied to the data and any flagging (deletion) of
the data.

    The \AIPS\ programs discussed in this chapter are part of a
package that has been developed to calibrate interferometer data from
a wide range of connected-element and VLB arrays, especially the VLA
and VLBA\@.  These programs therefore support many functions (and
inputs) that are not required when calibrating normal VLA data.  The
examples given below show only the essential parameters for the
operation being described, but, to get the results described, it is
essential that you check {\it all\/} the input parameters before
running any task.  Remember that \AIPS\ adverbs are global and will be
``remembered'' as you proceed.  A list of calibration-related symbols
is given in \Sec{aboutcal}, but a possibly more up-to-date list can be
obtained by typing {\us ABOUT CALIBRAT} in your \AIPS\ session.  More
general information on calibration can be routed to your printer by
typing {\us DOCRT\qs FALSE ; EXPLAIN\qs CALIBRAT \CR}, while deeper
information on a specific task is obtained with {\us EXPLAIN\qs
{\it taskname\/} \hbox{\CR}}.

     When you are satisfied with the \Indx{calibration} and editing
(or are simply exhausted), the task {\tt SPLIT} is used to apply the
calibration and editing tables and to write \uv\ files, each
containing the data for only one source.  These ``single-source'' \uv\
files are used by imaging and deconvolution tasks that work with only
one source at a time.  Many of the tasks described in this chapter
will also work on single-source files.  For VLA calibration, there are
several useful procedures described in this chapter and contained in
the {\tt RUN} file called \hbox{{\tt \tndx{VLAPROCS}}}.  Each of these
procedures has an associated {\tt HELP} file and inputs.  Before any
of these procedures can be used, this {\tt RUN} file must be invoked
with:
\dispt{RUN VLAPROCS \CR}{to compile the procedures.}
\dispe{There is a ``pipeline'' procedure designed to do a preliminary
calibration and imaging of ordinary VLA data sets.  This provides a
good first look at the data.  Nonetheless, the results are still not
likely to be of publishable quality.  To run the pipeline, enter}
\dispt{RUN PIPEAIPS \CR}{to compile the procedures.}
\dispt{INP PIPEAIPS \CR}{to review the input adverbs and, when ready,}
\dispt{\tndx{PIPEAIPS} \CR}{to execute the pipeline.}


\Sects{Copying data into \AIPS\  multi-source disk files}{uvtape}

     There are several ways to write VLA data to \AIPS\ multi-source
\uv\ data sets on disk. They include:
\xben
\Item For data from the historic VLA, download the files from the
    archive and run {\tt FILLM} to load the data.  See \Sec{oldVLA}
    for the details.
\Item For data from the new \Indx{EVLA}, download the files from the
    archive and run {\tt BDF2AIPS} to load the data.  See
    \Sec{BDF2aips} below for the details.
\Item For an \AIPS\ multi-source data set written to a FITS tape or
    disk file during an earlier \AIPS\ session, use {\tt \tndx{UVLOD}}
    or {\tt \tndx{FITLD}} to read the tape.
\Item Single-source data sets that are already on disk may be combined
    with multiple executions of {\tt \tndx{UV2MS}} or, if the files
    are dissimilar, with multiple executions of {\tt \tndx{MULTI}}
    followed by {\tt DBCON}\@.
\Item Data from the Australia Telescope may be loaded from disk files
    into \AIPS\ using the task {\tt \tndx{ATLOD}} which is now
    included with \AIPS\@.
\xeen

     Data from other telescopes can be read into \AIPS\ only if they
are written in \AIPS-like FITS files already or if you have a special
format-translation program for that telescope.  The VLBA correlator
produces a format which is translated by the standard \AIPS\ task {\tt
FITLD}; see \Sec{FITLDfits}.  A translation task for the Westerbork
Synthesis Telescope ({\tt WSLOD}) is available from the Dutch,
but is not distributed by the NRAO with the normal \AIPS\ system.

\Subsections{Reading EVLA archive files into \AIPS}{BDF2aips}

     The NRAO Archive makes available, among other things, data from
the EVLA, which began observing in January 2010.  Your \Indx{EVLA}
data are stored as an ``Science Data Model'' (SDM) format file in
``SDMBDF'' (Science Data Model Binary Data Format) in the NRAO
archive.  They may be read out of the archive in that format or a CASA
measurement set format.  Go to the web page\\
      \centerline{{\tt http://archive.cv.nrao.edu/}}\\
and select the Advanced Query Tool.  Fill out enough of the form to
describe your data and submit the query.  If the data are not yet
public, you will need the Locked Project Access Key which may be
obtained from the NRAO data analysts.  To avoid the need for this key,
you may log in to {\tt my.nrao.edu} after which it will know if you
are entitled to access particular locked projects.  The query will
return a list of the data sets which meet your specifications.  On
this form, enter your e-mail address.  If you have disk access at NRAO
Socorro and have a world-writable directory, enter that directory as
the download destination.  Otherwise let the download go to {\tt
/home/e2earchive} and be sure to check the tar file button.  Select
the SDM-BDF dataset format if you wish to reduce the data fully in
\AIPS\@.  If you select the CASA MS format, you will be required to
use CASA either for all your work or at least to convert the MS format
to a UVFITS format which \AIPS\ can read.  Sadly, such UVFITS files
lack a lot of useful information and so they are not recommended.  The
archive tool will notify you, by e-mail, when your data set is
available and will provide information on how to retrieve the data.
Copy the tar file to a data directory that you control, and {\tt
untar} it.

Unlocked files will be downloaded to the NRAO public ftp site\\
    \centerline{{\tt ftp://ftp.aoc.nrao.edu/e2earchive/}}\\
by default and you may then use ftp to copy the file to your computer.
Locked files will go to a protected ftp site and you must use ftp to
download those, even within NRAO\@.  The instructions for downloading
will be e-mailed to you.  Be sure to specify {\it binary} for the
copy.  If you are located in the AOC in Socorro, you may set an
environment variable to the archive location, \eg\
\displx{export E2E=/home/ftp/pub/e2earchive \CR}{for bash
       shells}
\displx{setenv E2E /home/ftp/pub/e2earchive \CR}{for C shells
       such as tcsh}
\dispe{and simply read unlocked data files directly from the public
download area.  Note that the file will be deleted automatically after
48 hours in both public and protected data areas.\Iodx{EVLA}}

You may now run {\tt AIPS} and load the data via the verbs {\tt
\tndx{BDFLIST}} and {\tt \tndx{BDF2AIPS}}\@.  These verbs run programs
in the {\tt \tndx{OBIT}} software package to load your data directly
into \AIPS\ including flag ({\tt FG}), index ({\tt NX}), calibration
({\tt CL}), over-the-top ({\tt OT}), SysPower ({\tt SY}), and
CalDevice ({\tt CD}) tables which you will not get from CASA\@.  Note
that these verbs require that {\tt OBIT} be installed on your computer
--- as it is in Socorro --- and that {\tt ObitTalk} be in your {\tt
\$PATH}\@.  {\tt OBIT} is relatively easy to install and may be
obtained from {\tt www.cv.nrao.edu/\~bcotton/Obit.html}

SDMBDF files may be read into \AIPS\ using {\tt \Tndx{BDFLIST}} to
learn what is in your data set and then {\tt \Tndx{BDF2AIPS}} to
translate the data.  Thus
\dispt{DEFAULT\qs BDF2AIPS; INP \CR}{to initialize all relevant
        adverbs.}
\dispt{DOWAIT\qs 2 ; DOCRT\qs 1 \CR}{to wait for the verbs to finish
        and to display the log file on the terminal after the {\tt
        OBIT} task finishes.  {\tt DOWAIT\qs 1} displays the messages
        as they are generated but is insensitive to returned error
        conditions from the {\tt OBIT} tasks.  Be sure to set {\tt
        DOWAIT\qs -1} after using {\tt BDF2AIPS}\@.}
\dispt{ASDMF(1)\qs = '{\it path\_to\_asdm\_dir} \CR}{to set the
        path name to your data directory into the adverb.  Note the
        lack of close quote so that case is preserved.}
\dispt{ASDMF(2)\qs = '{\it asdm\_file\_name} \CR}{to put the rest of
        the data file name in the second adverb since the names are
        almost always longer than 64 characters.  Trailing blanks in
        {\tt ASDMF(1)} will be ignored.}
\dispt{BDFLIST \CR}{to list the contents of the SDMBDF\@.  Note
        particularly the ``configuration'' numbers.}
\dispt{OUTNA\qs '{\it myname\/}' \CR}{to set the \AIPS\ name.}
\dispt{OUTCL\qs ' ' \CR}{to take default ({\tt UVEVLA}) class.}
\dispt{OUTDI\qs 3 \CR}{to write the data to disk 3 (one with enough
       space).}
\dispt{DOUVCOMP\qs FALSE \CR}{to write visibilities in uncompressed
       format.  There are no weights at present, so there is no
       loss of information in compressed format, but the conversion
       from compressed format costs more than reading the larger
       data files.}
\dispt{FOR\qs CONFIG = 0:100 ; BDF2AIPS; END \CR}{to load all of the
       configurations in your data, terminating with error messages on
       the first configuration number not present in your data (when
       {\tt DOWAIT} is 2).}
\dispe{There are other adverbs --- {\tt NCHAN}, {\tt NIF}, {\tt BAND},
and {\tt CALCODE} --- available if needed to limit which data are
read.  {\tt CONFIG} is frequently all that is needed to select data,
but these others may be needed if more complicated modes of observing
were used.\Iodx{EVLA}}

      If {\tt \tndx{BDF2AIPS}} is executing correctly, your message
terminal will report information about the data being loaded including
the data selection, the data scans loaded, and information about some
of the tables written.  Once {\tt BDF2AIPS} has completed, you can
find the database on disk using:
\dispt{INDI\qs 0 ; \tndx{UCAT} \CR}{to list all cataloged $uv$ files}
\dispe{This should produce a listing such as:}
\bve
Catalog on disk  3
Cat Usid Mapname      Class  Seq  Pt     Last access      Stat
  1  103 MyName      .UVDATA.   1 UV 05-FEB-2017 12:34:16
\end{verbatim}\eve

     You might then examine the header information for the disk data set by:
\dispt{INDI\qs 3 ; GETN\qs 1 ; \tndx{IMHEAD} \CR}{set name adverbs,
        list header in detail.}
\dispe{This should produce a listing like:}
\bve
AIPS 1: Image=MULTI     (UV)         Filename=MyName      .UVDATA.   1
AIPS 1: Telescope=EVLA               Receiver=EVLA
AIPS 1: Observer=Dr. Crys            User #=  200
AIPS 1: Observ. date=29-OCT-2016     Map date=29-JAN-2018
AIPS 1: # visibilities    876000     Sort order  TB
AIPS 1: Rand axes: UU-L-SIN  VV-L-SIN  WW-L-SIN  BASELINE  TIME1
AIPS 1:            SOURCE  INTTIM  WEIGHT  SCALE
AIPS 1: ----------------------------------------------------------------
AIPS 1: Type    Pixels   Coord value     at Pixel     Coord incr   Rotat
AIPS 1: COMPLEX      1   1.0000000E+00       1.00  1.0000000E+00    0.00
AIPS 1: STOKES       2  -1.0000000E+00       1.00 -1.0000000E+00    0.00
AIPS 1: FREQ       128   4.0393400E+09      65.00  1.0000000E+06    0.00
AIPS 1: IF          31   1.0000000E+00       1.00  1.0000000E+00    0.00
AIPS 1: RA           1    00 00 00.000       1.00       3600.000    0.00
AIPS 1: DEC          1    00 00 00.000       1.00       3600.000    0.00
AIPS 1: ----------------------------------------------------------------
AIPS 1: Coordinate equinox 2000.00
AIPS 1: Maximum version number of extension files of type SU is   1
AIPS 1: Maximum version number of extension files of type AN is   1
AIPS 1: Maximum version number of extension files of type FQ is   1
AIPS 1: Maximum version number of extension files of type NX is   1
AIPS 1: Maximum version number of extension files of type CD is   1
AIPS 1: Maximum version number of extension files of type SY is   1
AIPS 1: Maximum version number of extension files of type GC is   1
AIPS 1: Maximum version number of extension files of type CT is   1
AIPS 1: Maximum version number of extension files of type FG is   1
AIPS 1: Maximum version number of extension files of type WX is   1
AIPS 1: Maximum version number of extension files of type CL is   1
AIPS 1: Maximum version number of extension files of type HI is   1
\end{verbatim}\eve
\dispe{This header identifies the file as a \indx{multi-source data
set} ({\tt Image=MULTI}) with 876000 floating-point visibilities in
time-baseline ({\tt TB}) order.  There are 128 spectral channels in
each of 31 entries on the {\tt IF} axis.  These correspond to
``spectral windows'' in the correlator and have, at best, only a loose
correspondence to the separate electronic channels.  The description
of the frequency ({\tt FREQ}) axis shows that the first IF is at 4039
MHz and has 128 MHz total bandwidth.  The parameters of the other IFs
are determined from the data in the {\tt FQ} table file and cannot be
read directly from this header; these values are shown in the {\tt
'SCAN'} listing from \hbox{{\tt LISTR}}.  The header shown above
indicates that the data are in compressed format since the number of
pixels on the {\tt COMPLEX} axis is 1 and the {\tt WEIGHT} and {\tt
SCALE} random parameters are present.  Uncompressed data does not use
these random parameters and has 3 pixels on the {\tt COMPLEX}
axis.\todx{IMHEAD}\Iodx{EVLA}}

     If your experiment contains data from several bands {\tt
\Tndx{BDF2AIPS}} will place the data from each band in separate data sets.
Also, if you observed with several sets of frequencies or bandwidths
in a given observing run these will be assigned different {\tt FQ}
numbers by \hbox{{\tt BDF2AIPS}}.  You can determine which frequencies
correspond to which {\tt FQ} numbers from the {\tt 'SCAN'} listing
provided by \hbox{{\tt LISTR}}.  If you find multiple {\tt FQ} numbers
in your data set, we strongly advise you to run {\tt UVCOP} to
separate them into different files.  This will greatly simplify your
data reduction.

\Subsections{Reading data from FITS files with {\tt FITLD}}{FITLDfits}

     {\tt \tndx{FITLD}} is used to read FITS-format disk files (and
tapes) into \hbox{\AIPS}. It recognizes images, single- and
multi-source \uv\ data sets, and the special FITS \uv-data tables
produced by the VLBA and DiFX correlators (``FITS-IDI'' format).  In
particular, VLA data sets that have been read into \AIPS\ previously
with {\tt FILLM} and then saved to tape (or pseudo-tape disk) files
with {\tt FITTP} and {\tt FITAB} can be recovered for further
processing with task \hbox{{\tt FITLD}}.  (The older task {\tt
\tndx{UVLOD}} will also work with \uv\ data sets in FITS format, but
it cannot handle image or FITS-IDI format files.)

     A multi-source data file with all of its tables can be read from
a FITS FILE by:
\dispt{TASK\qs 'FITLD' ; INP \CR}{to review the inputs needed.}
\dispt{DATAIN\qs '{\it filename\/}' \CR}{to specify the FITS
          disk file (see \Sec{fitsdisk}).}
\dispt{DOUVCOMP\qs FALSE \CR}{to write visibilities in uncompressed
          format.}
\dispt{OUTNA\qs ' ' \CR}{take default (previous \AIPS) name.}
\dispt{OUTCL\qs ' ' \CR}{take default (previous \AIPS) class.}
\dispt{OUTSEQ\qs 0 \CR}{take default (previous \AIPS) sequence \#.}
\dispt{OUTDI\qs 3 \CR}{to write the data to disk 3 (one with enough
          space).}
\dispt{INP \CR}{to review the inputs (several apply only to VLBA
          format files).}
\dispt{GO \CR}{to run the program when you're satisfied with inputs.}
\dispe{The data-selection adverbs {\tt SOURCES}, {\tt QUAL}, {\tt
CALCODE}, and {\tt TIMERANG} and the table-control adverbs {\tt CLINT}
and {\tt FQTOL} are used for VLBA-format data only..  See
\Rchap{vlbi} for more specific information.\todx{FITLD}}

\sects{Record keeping and data management}

\subsections{Calibrating data with multiple {\tt FQ} entries}

     An observing run with the VLA may result in a \uv\ data file
containing multiple {\tt FQ} entries.  \todx{FQ number}
This may be convenient, but it has a number of costs.  If a file
contains multiple, independent frequencies, then it occupies more disk
space and costs time in every program to skip the currently unwanted
data (either a small cost when the index file is used or a rather
larger cost when the file must be read sequentially).  Since multiple
frequencies are still not handled correctly in all programs (\ie\
polarization calibration) and since it is not possible to calibrate
all of the different {\tt FQ} data in one pass, you might consider
separating the multiple frequencies into separate files (use {\tt
\tndx{UVCOP}})\@.  In either case, you  must calibrate each frequency
with a separate pass of the scheme outlined below. There are three
adverbs to enable you to differentiate between the different {\tt FQ}
entries: {\tt \tndx{FREQID}} enables the user to specify the {\tt FQ}
number directly (with -1 or 0 meaning to take the first found); {\tt
SELFREQ} and {\tt SELBAND} enable the user to specify the observing
frequency and bandwidth to be calibrated (the tasks then determine to
which {\tt FQ} number these adverbs correspond). If {\tt SELFREQ} and
{\tt SELBAND} are specified they override the value of \hbox{{\tt
FREQID}}.\todx{FQ number}\Iodx{EVLA}

     There are certain bookkeeping tasks that must be performed
between calibrating each {\tt FQ} set.  First, you must ensure that
you have reset the fluxes of your secondary calibrators by running
{\tt \tndx{SETJY}} with {\us OPTYPE = 'REJY'} --- if not, this will
cause the amplitudes of your data to be incorrect.  Second, it is wise
to remove the {\tt SN} tables associated with any previous calibration
using the verb \hbox{{\tt \tndx{EXTDEST}}}.  Although this is not
strictly necessary, it will simplify your bookkeeping.

\Subsections{Recommended record keeping}{calrecord}

     It is useful to print a summary of the time stamps and source
names of the scans in your data set.  This reminds you of the
structure of your observing program when you decide on interpolation
and editing strategies, and may help to clarify relationships between
later, more detailed listings of parts of the data set.  It is also
useful to have a printed scan summary and a map of the antenna layout
if you need to return to processing the data months or years later.
Finally, it is also making sure that all \AIPS\ input parameters have
their null (default) values before invoking the parts of the
calibration package, such as {\tt CALIB}, that have many inputs.  The
null settings of most parameters are arranged to be sensible ones so
that basic VLA calibration can be done with a minimum of specific
inputs; but some inputs may lose their default values if you
interleave other \AIPS\ tasks with the calibration pattern recommended
below.  Therefore, you should {\it always\/} review the input
parameters with {\us INP\qs {\it taskname\/} \CR} before running task
{\it taskname\/}.\Iodx{calibration}

     We suggest that you begin a calibration session with the
following inputs:
\dispt{\tndx{DEFAULT}\qs \tndx{LISTR} \CR}{to set all {\tt LISTR}'s
       inputs to null (default) values, also setting the {\tt TASK}
       adverb.}
\dispt{INP \CR}{to review the inputs}
\dispt{INDI\qs {\it n\/}; GETN\qs {\it m\/} \CR}{to select the data
       set, $n=3$ and $m=1$ in {\tt FILLM} example above.}
\dispt{TPUT\qs CALIB \CR}{to store null values for later use with
       {\tt CALIB}\@.}
\dispt{OPTYP\qs 'SCAN' \CR}{to select scan summary listing.}
\dispt{DOCRT\qs -1 \CR}{to send the output to the printer.}
\dispt{INP \CR}{to review the inputs for \hbox{{\tt LISTR}}.}
\dispt{GO \CR}{to run the program when the inputs are set correctly.}
\dispe{Note that the {\tt DEFAULT\qs \tndx{LISTR}} sets the adverbs to
select all sources and all times and to send printed output to the
terminal rather than the printer.  It is also very useful to have a
printed summary of your antenna locations, especially a list of which
ones you actually ended up using.  To do this, enter}
\dispt{NPRINT\qs 0 \CR}{to do all antennas}
\dispt{INVERS\qs {\it n} \CR}{to do sub-array {\it n\/}}
\dispt{GO\qs \tndx{PRTAN} \CR}{to print the list and a map of antenna
           locations.}

     In looking over the output from {\tt LISTR}, you may notice that
some of the sources you wish to use as calibrators have a blank
``Calcode''.  To mark them as calibrators, use:
\dispt{TASK\qs '\tndx{SETJY}' ; INP \CR}{to select the task and review
           its inputs.}
\dispt{SOURCES '{\it sor1\/}' , '{\it sor2\/}' , '{\it sor3\/}' ,
           $\ldots$ \CR}{to select the unmarked calibrator sources.}
\dispt{OPTYPE 'RESE' \CR}{to reset fluxes and velocities.}
\dispt{CALCODE\qs 'C' \CR}{to mark the sources as ``C'' calibrators.}
\dispt{GO \CR}{to run the task.}
\dispe{This operation will let you select the calibrators by their
Calcodes rather than having to spell out their names over and over
again.  You may wish to consider separate calibrator codes for primary
and secondary gain calibrators to make them easier to separate.
You may reset a calibrator code to blank by specifying {\tt CALCODE =
'----'}.}

\Sects{Beginning the calibration}{calbegin}

Handling \Indx{EVLA} data is quite different from the situation with the
historic VLA\@.  The principal difference is the very wide bandwidth
of the EVLA which makes every data set have a large number of spectral
channels in every spectral window.  This requires corrections for
delay errors and bandpass shape which were not needed with historic
VLA continuum data.  It also means that almost all observations are
contaminated by radio frequency interference (RFI) at a level
requiring that some portion of the data be marked as bad
(``flagged'') and omitted from later use.  Writing a general list of
\indx{calibration} steps is rather difficult because the steps needed
depend on the problems found in the data set.  Low frequency data
(bands less than 8 GHz) are likely to be heavily contaminated with
RFI, while the highest frequency bands have very little.  But the high
frequencies suffer greatly from rapidly varying ``instrumental'' (\ie\
atmospheric) phase variation.  Nonetheless, we will make a list of the
general steps that need to be taken following those already
described.

\xben
\Item Correct the data for antenna locations with {\tt VLANT}
      (\Sec{blcorr})
\Item Correct 3-bit data (only) for post-detector gain with {\tt
      TYAPL} (\Sec{TYAPL3bit})
\Item Look at some of your calibrator data with {\tt POSSM} using the
      TV display (\Sec{POSSMcheck})
\Item If there is significant narrow-band RFI causing spectral
      ringing, copy the data applying a Hanning smooth with {\tt
      SPLAT} (\Sec{SPLATringing})
\Item If there is significant RFI on your calibrator sources, flag the
      bad channels and times using {\tt UVFLG}, {\tt FTFLG}, {\tt
      RFLAG}, or some other flagging task. (\Sec{caledit})
\Item If the phases show significant slope with frequency, run {\tt
      FRING} to find delay corrections and apply them with {\tt
      CLCAL} (\Sec{FRINGit})
\Item Run {\tt SETJY} with {\tt OPTYPE 'CALC'} on the primary flux
      calibrators and set {\tt CALCODE}s for other calibrators if
      needed. (\Sec{SETJYcalc})
\Item Run {\tt BPASS} to find corrections for bandpass shape and
      examine them with {\tt BPEDT}, perhaps flagging some calibrator
      data and re-running {\tt BPASS}. (\Sec{BPASSit})
\Item Run {\tt VLACALIB} separately for each calibration source
      (\Sec{vlacalib})
\Item Run {\tt GETJY} to determine the calibration sources' flux from
      that of the primary flux calibrator (\Sec{getjy})
\Item Use {\tt EDITA} to examine the resulting {\tt SN} table for bad
      values and to flag calibrator data as needed. (\Sec{edita})
\Item If significant editing was done, run {\tt SETJY} with {\tt
      OPTYPE 'REJY'} on the secondary calibrators; then re-run {\tt
      VLACALIB} and {\tt GETJY} (\Sec{vlacalib}, \Sec{getjy})
\Item Run {\tt CLCAL} to apply the {\tt SN} table to the {\tt CL}
      table (\Sec{clcal})\Iodx{EVLA}
\Item Examine your data to see if there remains significant RFI.  Use
      {\tt POSSM} or {\tt SPFLG} on a limited set of baselines or even
      {\tt FTFLG} although this last will encourage you to flag too
      much data.
\Item If there is significant RFI, use multiple passes of {\tt RFLAG}
      to address the problem. (\Sec{rflag})
\Item If you have deleted a bunch of calibrator data, delete the
      calibration tables except for {\tt CL} tables 1 and 2.
      (\Sec{cleanup})
\Item For 8-bit data (only), examine the {\tt SY} table with {\tt
      EDITA} or {\tt SNPLT}\@.  Correct it as needed with {\tt TYSMO}
      and then apply it to the data with {\tt TYAPL} or {\tt SYSOL}
      for Solar data. (\Sec{tyapl})
\Item Now return to the {\tt FRING} step and repeat.
\Item If you wish to calibrate polarization, run {\tt RLDLY} to
      correct right - left delay, run {\tt PCAL} to find the antenna D
      terms, and then {\tt RLDIF} to correct the right - left phase
      difference.  Note that this requires a calibration source of
      known linear polarization. (\Sec{polcal})
\Item Now examine your target source data with {\tt POSSM} and edit
      them with {\tt RFLAG}
\Item Calibrate the weights with {\tt REWAY} if {\tt TYAPL} has not
      already done this.  Shift spectral-line data to constant
      velocity with {\tt CVEL}\@. (\Sec{reway})
\Item Try some quick images with {\tt IMAGR} to check the calibration
      (\Sec{calimagr})
\Item Back up your calibrated data with {\tt FITTP} or {\tt FITAB}
      (\Sec{calfittp})
\Item Apply the \indx{calibration} and editing to the target sources,
      writing single-source files for imaging and self-calibration
      with {\tt SPLIT} (\Sec{split})
\xeen

\Subsections{Baseline corrections}{blcorr}

     Sometimes, \eg\ during a VLA array re-configuration, your
observations may have been made when one or more of the antennas had
their positions poorly determined.  The positional error is usually
less than a centimeter at the VLA, but even this may affect your data
significantly.  The most important effect is a slow and erroneous
phase wind which is a function of source position and time.  Since
this error is a function of source position, it cannot be removed
exactly using observations of a nearby calibrator, although the error
will be small if the target source is close to the calibrator.  In
many observations, the target sources and calibrators are sufficiently
close to allow this phase error to be ignored.  Self-calibration will
remove this error completely {\it if\/} you have enough
signal-to-noise to determine the correction during each integration.

    The maximum phase error introduced into the calibrated visibility
data by incorrect \indx{antenna coordinates} $\Delta\phi_{B}$, in
radians, by a \indx{baseline error} of $\Delta B$ meters is given by
    $$\Delta\phi_{B} \approx 2\pi\Delta\theta\Delta B / \lambda$$
where $\Delta\theta$ is the angular separation between the calibrator
and the target source in radians and $\lambda$ is the wavelength in
meters.

    Note, however, that the error due to the phase-wind is not the
only error introduced by incorrect antenna positions.  A further, but
much smaller effect, will be incorrect gridding of the data due to the
erroneous calculation of the baseline spatial frequency components
{\it u\/}, {\it v\/} and {\it w\/}.  This effect is important only for
full primary beam observations in which the antenna position error is
of the order of a meter.  It is highly unlikely that such a condition
will occur.  Note too, that this error {\it cannot\/} be corrected by
the use of self-calibration.  However, after correcting the antenna
position with {\tt CLCOR}, you may run {\tt UVFIX} to compute
corrected values of $u, v,$ and $w$.  The maximum phase error in
degrees, $\Delta\phi_{G}$, caused by incorrect gridding of the {\it
u,v,w\/} data is
       $$\Delta\phi_{G} \approx 360 \Delta\epsilon \Delta\Theta$$
where $\Delta\epsilon$ is the antenna position error in antenna
diameters and $\Delta\Theta$ is the angular offset in primary beams.

If \indx{baseline error}s are significant they need to be removed from
your data before calibration.  It is important to do this to {\tt CL}
table 1, right after running {\tt BDF2AIPS}.  For the \Indx{EVLA}, use
the task {\tt \Tndx{VLANT}}\@.  This task determines and applies the
antenna position corrections found by the VLA operations staff after
your observation was complete.  To run {\tt VLANT}:
\dispt{TASK\qs 'VLANT' \CR}{}
\dispt{INDISK\qs {\it m\/} ; GETN\qs {\it n\/} \CR}{to get the correct
          data set.  Note that you don't have to keep doing this
          unless you switch between different input data files.}
\dispt{FREQID {\it 1\/} \CR}{to choose {\tt FQ} 1.}
\dispt{SUBARRAY {\it x\/} \CR}{to choose the antenna table to correct.}
\dispt{GAINVER\qs 1 \CR}{to choose the correct version of the {\tt CL}
          table to read. A new one will produced.}
\dispt{GO \CR}{to run \hbox{{\tt VLANT}}.\Iodx{calibration}}

For arrays other than the VLA, use {\tt \tndx{CLCOR}} to enter the
antenna position corrections (in meters) in a new {\tt CL} table and
the old {\tt AN} table.  This must be done for each affected antenna
in turn.  {\tt CLCOR} puts the corrections into the {\tt AN} table as
well as the {\tt CL} table, so it is wise to save the {\tt AN} table
before running {\tt CLCOR} by running {\tt TASAV}.
\dispt{TASK\qs 'CLCOR' \CR}{}
\dispt{INDISK\qs {\it m\/} ; GETN\qs {\it n\/} \CR}{to get the correct
          data set.  Note that you don't have to keep doing this
          unless you switch between different input data files.}
\dispt{SOURCES\qs '\ ' ;STOKES\qs '\ ' \CR}{to do all sources, all
          Stokes,}
\dispt{BIF\qs 0 ; EIF\qs 0 \CR}{and all IFs.}
\dispt{SUBARRAY {\it x\/} \CR}{to choose the correct sub-array.}
\dispt{OPCODE\qs 'ANTP' \CR}{to select the antenna position correction
          mode.}
\dispt{GAINVER\qs 1 \CR}{to choose the correct version of the {\tt CL}
          table to read.}
\dispt{GAINUSE\qs 0 \CR}{to have {\tt CLCOR} create a new table.}
\dispt{ANTENNA\qs {\it k\/} \CR}{to select antenna.}
\dispt{CLCORPRM\qs $\Delta b_{x} , \Delta b_{y} , \Delta b_{z} , 0, 0,
          0, 1$ \CR}{to add the appropriate antenna corrections in
          meters; the l in {\tt CLCORPRM(7)} indicates VLA phase
          conventions rather than VLB conventions.}
\dispt{GO \CR}{to run \hbox{{\tt CLCOR}}.}
\dispe{The program will need to be run as many times as there are
antennas for which positional corrections must be made.  Set {\tt
GAINUSE} and {\tt GAINVER} both to 2 after the first correction.
Otherwise, with the above adverbs, {\tt CLCOR} will make multiple {\tt
CL} table versions each with only one correction in them.  Note that
subsequent calibration must be applied to {\tt CL} table 2 to create
higher versions of the calibration table.  This new {\tt CL} table
(version 2) will replace version 1 in all of the subsequent sections
on calibration.  Thus, in subsequent executions of {\tt CALIB}, you
must apply these corrections by specifying {\us DOCALIB\qs TRUE ;
GAINUSE\qs 0} (for highest version).  Note too that {\tt CLCOR} and
{\tt VLANT} change the antenna file for the changed antenna
location(s).  Therefore, it is wise to save the {\tt AN} table before
running {\tt CLCOR} or {\tt VLANT} by running {\tt TASAV}.}

Note that NRAO's data analysts use the \AIPS\ task {\tt \tndx{LOCIT}}
to determine the antenna position corrections; see \Rappen{VLAtasks}..
These are available to the general user, but a data set designed to
determine antenna corrections is normally required.  Such data sets
consist of about 100 observations of a wide range of phase calibrators
taken as rapidly as possible.\Iodx{calibration}

\Subsections{Correcting EVLA 3-bit data}{TYAPL3bit}

With the historic VLA, the system temperature was measured in real
time and a correction for it applied to the data as they were
archived.   With the \indx{EVLA} this is not the case.  Instead the
{\tt SY} table records the total power when the switched noise tubes
are on and the total power when they are off.  These values may be
used to correct data taken with 8-bit correlation (\Sec{tyapl}) to
values which would be fully calibrated in Janskys if the noise tube
levels and antenna efficiencies were accurately known.  For 3-bit
data, frequently used to observe very wide bandwidths at high
frequency, the the function to correct the visibilities from these
measurements is not well known.  As a consequence, for such data, we
can only correct for any post-detector gain changes.  These are also
recorded in the {\tt SY} table, so use
\dispt{DEFAULT\qs \tndx{TYAPL} \CR}{to select the task and init the
          adverbs.}
\dispt{INDISK\qs {\it m\/} ; GETN\qs {\it n\/} \CR}{to get the correct
          data set.}
\dispt{INEXT \qs 'SY' \CR}{to specify an {\tt SY} table.}
\dispt{OPTYPE\qs 'PGN' \CR}{to specify the post-detector gain
          correction.}
\dispt{INP \CR}{to check the inputs.}
\dispt{GO \CR}{to write out a new data set with gain corrections.}
\dispe{Reminder --- do this only for 3-bit correlator data.  Your
8-bit data will be discussed later.}

\Subsections{Check your data with {\tt POSSM}}{POSSMcheck}

The next step is to take a look at your data.  Do not be obsessive
about this.  Choose your main calibration source (that you plan to use
for {\tt FRING}, {\tt BPASS}, and absolute gain calibration), choose
a few baselines (say all baselines to one interior antenna), and use
the \AIPS\ TV display.  At this point you want to look for antennas
that appear out of the ordinary (\ie\ dead) and for frequencies
damaged by RFI\@.  Use
\dispt{DEFAULT\qs \tndx{POSSM} \CR}{to select the task and init the
          adverbs.}
\dispt{INDISK\qs {\it m\/} ; GETN\qs {\it n\/} \CR}{to get the correct
          data set.}
\dispt{SOURCE\qs '{\it bandpass\_cal\/}' \CR}{to select the strong
       bandpass calibrator.}
\dispt{DOTV\qs 1 ; NPLOTS\qs 1 \CR}{to plot only on the TV, one
       baseline at a time.}
\dispt{ANTEN\qs {\it n1} , 0 \CR}{to select an antenna nearest the
       center of the array.}
\dispt{BASELINE\qs 0 \CR}{and all antennas to that antenna.}
\dispt{DOCAL\qs 1 ; APARM\qs 0 \CR}{to apply the initial calibrations
       and display vector averaged spectra.  Scalar averaged spectra
       will turn up at the edges reflecting the decreased signal to
       noise in the outer channels.}
\dispt{APARM(9)=3 \CR}{to plot all IFs and both polarizations in a
       single plot.}
\dispt{GO \CR}{to run the task.  Make notes of what you see.}

\Subsections{Remove spectral ringing with {\tt SPLAT}}{SPLATringing}

If there is very narrow-band RFI, the {\tt POSSM} plots will show
ringing - a $\sin(x) / x$ pattern as a function of spectral channel.
If this occurs in your data, and it will not occur in some bands,
then\Iodx{calibration}
\dispt{DEFAULT\qs \tndx{SPLAT} \CR}{to select the task and init the
          adverbs.}
\dispt{INDISK\qs {\it m\/} ; GETN\qs {\it n\/} \CR}{to get the correct
          data set.}
\dispt{DOCAL\qs 1 \CR}{to apply any initial and {\tt VLANT}
          calibration.}
\dispt{SMOOTH\qs 1,0 \CR}{to use Hanning smoothing.}
\dispt{INP \CR}{always check the many inputs which should be defaults
          here.}
\dispt{GO \CR}{to make a new data set with the ringing reduced.}

\Subsections{General considerations in flagging}{caledit}

At this point, it is essential only to flag the most egregiously bad
data samples.  These are, with the \Indx{EVLA}, usually either dead
antennas or spectral regions affected more or less uniformly by RFI\@.
We will however use this section for a more general discussion of
flagging, most of which will apply more directly to later stages in
the reduction.  Probably {\tt FTFLG} may be used carefully at this
stage to flag only those channels seen to be generally bad in the {\tt
POSSM} plots.

     The philosophy of \Indx{editing} and the choice of methods are
matters of personal taste and the advice given below should,
therefore, be taken with a few grains of salt.  When interferometers
consisted of only a couple of movable antennas, there was very little
data and it was sparsely sampled.  At that time, careful editing to
delete all suspect samples, but to preserve all samples which can be
calibrated, was probably justified.  But modern instruments produce a
flood of data, with the substantial redundancy that allows for
self-calibration on strong sources.  Devoting the same care today to
editing is therefore very expensive in your time, while the loss of
data needlessly flagged is rarely significant.  A couple of guidelines
you might consider are:\Iodx{flagging}
\xbit
\Item Don't flag on the basis of phase.  At least with the VLA,
    most phase fluctuations are due to the atmosphere rather than the
    instrument.  Calibration can deal with these up to a point, and
    self-calibration (if you have enough signal) can refine the phases
    to levels that you would never reach by flagging.  The exceptions
    are (1) IF phase jumps which still happen on rare occasions, and
    (2) RF interference which sometimes is seen as an excursion in
    phase rather than amplitude.
\Item Don't flag on minor amplitude errors, especially if they are
    not common.  Except for very high dynamic range imaging, these
    will not be a problem, and in those cases, self-calibration always
    repairs or sufficiently represses the problem.
\Item Don't flag if {\tt CALIB} reports few closure errors and the
    {\tt SN} tables viewed with {\tt \tndx{EDITA}}, {\tt
    \tndx{SNPLT}}, and {\tt \tndx{LISTR}} and the calibrator data
    viewed with the matrix format of {\tt LISTR} show only a few
    problems.
\xeit

     There are three general methods of editing in \hbox{\AIPS}.  The
``old-fashioned'' route uses {\tt LISTR} to print listings of the data
on the printer or the user's terminal.  The user scans these listings
with his eyes and, upon finding a bad point, enters a specific flag
command for the data set using {\tt \tndx{UVFLG}}\@.  While this
may sound clumsy, it is in fact quite simple and by far the faster
method when there are only a few problems.  In a highly corrupted data
set, it can use a lot of paper and may force you to run {\tt LISTR}
multiple times to pin down the exact problems.  The ``hands-off''
route uses tasks which attempt to determine which data are bad using
only modest guidance from the user.  The most general of these are
{\tt RFLAG} and {\tt FLAGR} mentioned below.  The third and ``modern''
route uses interactive (``TV''-based) tasks to display the data in a
variety of ways and to allow you to delete sections of bad data simply
by pointing at them with the TV cursor.  These tasks are {\tt
\tndx{TVFLG}} (\Sec{tvflg}) for all baselines and times (but only shows
one IF, one Stokes, and one spectral channel at a time), {\tt
\tndx{SPFLG}} (\Sec{spflg}) for all spectral channels, IFs, and times
(but only shows one baseline and one Stokes at a time),  {\tt
\tndx{FTFLG}} for all spectral channels, IFs, and times (but shows all
baselines combined, one Stokes at a time), {\tt \tndx{EDITA}}
(\Sec{edita}) for editing based on {\tt SY} (SysPower), {\tt TY}
(T$_{ant}$), {\tt SN} or {\tt CL} table values,  {\tt \tndx{EDITR}}
(\Sec{editr}) for all times (but only shows a single antenna (1--11
baselines) and one channel average at a time) and {\tt \tndx{WIPER}}
for all types of data (but with the time, polarization, and sometimes
antenna of the points not available while editing).  {\tt TVFLG} is
the one used for continuum and channel-0 data from the VLA, while {\tt
FTFLG} is only used to check for channel-dependent interference.  {\tt
SPFLG} is very useful for spectral-line editing in smaller arrays,
such as the Australia Telescope and the VLBA, but is tedious for
arrays like the VLA.  Nonetheless, it may be necessary for the modern
EVLA\@.  (The redundancy in the spectral domain on calibrator sources
helps the eyes to locate bad data.)  {\tt EDITR} is more useful for
small arrays such as those common in VLBI experiments.  {\tt EDITA}
has been found to be remarkably effective using VLA system temperature
tables.  All four tasks have the advantage of being very specific in
displaying the bad data.  Multiple executions should not be required.
However, they may require you to look at each IF, Stokes, channel (or
baseline) separately (unless you make certain broad assumptions); {\tt
EDITA} and {\tt EDITR} do allow you to look at all polarizations
and/or IFs at once if you want.  They all require you to develop
special skills since they offer so many options and operations with
the TV cursor (mouse these days).  A couple of general statements can
be made\Iodx{editing}\Iodx{flagging}
\xbit
\Item For data corrupted by RFI, try {\tt FTFLG} but be careful.  It
    displays frequency versus time for the selected sources and
    polarization, but averages all baselines together.  A few bad
    baselines can make it look like all are bad and cause you to flag
    too much.  Flagging channels seen to be bad in all {\tt POSSM}
    plots is a good thing and checking channels that appear bad in
    {\tt FTFLG} more carefully elsewhere is also good.
\Item The ultimate tool for \Indx{EVLA} flagging is {\tt SPFLG} but
    that shows a frequency-time display for each baseline
    individually.  That me be reasonable for the ATCA or VLBA, but 351
    baselines in the VLA is a lot of work.  Nonetheless, {\tt SPFLG}
    may help you identify channels you may wish to flag more generally.
\Item For highly corrupted data (say with considerable RF
    interference, significant cross-talk between antennas, or erratic
    antennas) {\tt \tndx{TVFLG}} may be useful.  It gives an overall
    view of the data which is far superior to that given by {\tt
    LISTR}\@.  RFI and similar problems are more troublesome at lower
    frequencies, so {\tt TVFLG} is probably preferred for L, P, and
    ``4'' bands.
\Item Most VLA data at higher frequencies are of good quality and
    the flexibility of {\tt TVFLG} is not needed.  In such cases, {\tt
    \tndx{LISTR}} with {\tt OPCODE = 'MATX'} can find scans with
    erroneous points efficiently.
\Item The displays given by {\tt TVFLG} and, to a lesser extent,
    {\tt LISTR} in its {\tt MATX} mode are less useful when there are
    only a few baselines.  Thus, for arrays smaller than the VLA,
    users may wish to use {\tt \tndx{SPFLG}} on spectral-line data
    sets and {\tt \tndx{EDITR}} on continuum data sets.
\Item A reasonable strategy to use is to run {\tt LISTR} first.  If
    there are only a few questionable points, use {\tt LISTR} and {\tt
    UVFLG}, otherwise switch to an interactive task, such as {\tt
    EDITA} followed by \hbox{{\tt TVFLG}}.
\Item Task {\tt \tndx{FLAGR}} is a somewhat experimental task to
    measure the rms in the data on either a baseline or an antenna
    basis and then delete seriously discrepant points and times when
    many antennas/correlators are questionable.  It also clips
    amplitudes and weights which are outside specified normal ranges.
    Task {\tt \tndx{FINDR}} reports the rmses and excessive values to
    assist in running {\tt FLAGR}\@.
\Item Task {\tt \tndx{RFLAG}} flags data on the belief that RFI is
    highly variable in time and/or frequency.  It can do plots showing
    the time and/or frequency statistics as a function of spectral
    channel and recommends flagging levels.  These, or user-chosen
    values, may be applied to the data to produce large flag tables;
    see \Sec{rflag}.  {\tt RFLAG} is best used after data have some
    initial calibration to remove delay and bandpass issues and
    equalize the flux scale.
\Item Task {\tt \tndx{CLIP}} makes entries in a flag table, applying
    calibration and then testing amplitudes for reasonableness on a
    source-by-source basis.  It can be very useful for large data
    sets, but does not show you the bad data to evaluate yourself.
    {\tt \tndx{ACLIP}} does the same operation on auto-correlation
    data.
\Item Task {\tt \tndx{FGSPW}} performs a matrix scalar average over
    all spectral channels on a per-IF (aka ``spectral window'') per
    polarization per baseline per {\tt SOLINT} or scan  basis.  It
    then flags those IFs with excessive amplitude.  This operation is
    designed to catch those windows where amplitudes have overflowed
    the hardware due to RFI within the spectral window and may catch
    other problems as     well.
\Item Task {\tt \tndx{WIPER}} makes entries in a flag table for all
    data samples wiped from a {\tt UVPLT}-like display of any \uv\
    data set parameter versus any other parameter.  The source,
    Stokes, IF, time, etc.~of the points are not known during the
    interactive editing phase, but some baseline information is
    displayed.  It can plot and edit any choice of {\tt STOKES} in one
    execution and can flag/unflag by baseline.
\Item Task {\tt \tndx{DEFLG}} makes entries in a flag table whenever
    the phases are too variable as measured by too low a ratio of
    vector-averaged to scalar-averaged amplitudes.  This may be useful
    when applied to the calibrator source in phase-referencing
    observations and for other data at the highest and lowest
    frequencies which are affected by atmospheric and ionospheric phase
    variability.
\Item Task {\tt \tndx{SNFLG}} makes entries in a flag table whenever
    the phase solutions in an {\tt SN} or {\tt CL} table change
    excessively between samples on a baseline basis.  It can also flag
    data if the amplitude solutions differ from their mean
    excessively.  In {\tt 31DEC12} it can instead flag data when the
    amplitude and/or phase solutions or solution weights are outside
    user-specified ranges.
\Item Task {\tt \tndx{REWAY}} examines spectral rms to try to
    determine better weights for the data.  It is better to use it
    after delay, bandpass, and complex gain issues have been addressed
    at some level.  Run it with no flagging of the output for bad
    values of the spectral rms.  Then plot the weights with {\tt
    VPLOT} or {\tt ANBPL} to look for weights that are seriously
    abnormal (high or low).  Those data may need to be flagged.  High
    weights mean that the data are of abnormally low amplitude, whilst
    low weights mean that the data are very noisy.  {\tt REWAY} uses
    robust methods to find the rms and so a few channels of RFI may
    not cause very low weights, but lots of RFI or receiver failures
    will make the weights abnormally low.  {\tt REWAY} now displays
    statistical information to help you assess what weights are
    ``high'' and ``low''.
\Item Task {\tt \tndx{WETHR}} makes entries in a flag table whenever
    various weather parameters exceed specified limits.  {\tt WETHR}
    also plots the weather ({\tt WX}) table contents.
\Item Task {\tt \tndx{VPFLG}} flags all correlators in a sample
    whenever one is flagged.  Observations of sources with circular
    polarization (Stokes V) require this operation to correct the
    flagging done on-line (which flags only known bad correlators).
\Item Task {\tt \tndx{FGPLT}} plots the times of selected flag-table
    entries to provide you information on what these powerful tasks
    have done.
\xeit

Many of the above tasks produce large flag tables.  The task {\tt
\tndx{REFLG}} attempts to compress such tables, sometimes quite
remarkably.  Task {\tt \tndx{FGDIF}} may be used to confirm that the
flag tables before and after flag the same things.  {\tt FGCNT} in
{\tt 31DEC16} may also help confirm this.

\Subsections{Correcting delay errors with {\tt FRING}}{FRINGit}

We have had difficulty setting all of the delays in the \Indx{EVLA} to
values which are sufficiently accurate.  If the delay is not set
correctly, the interferometer phase will vary linearly with frequency,
potentially wrapping through several turns of phase within a single
spectral window (``IF band'').  Your {\tt POSSM} displays probably
show such an effect.  Delay errors are problem familiar to VLBI users
and \AIPS\ has a well-tested method to correct the problem.  Using
your {\tt LISTR} output, select a time range of about one minute {\it
toward the end of a scan} on a strong point-source calibrator, usually
your bandpass calibrator.  Then\Iodx{calibration}
\dispt{DEFAULT\qs \tndx{FRING} ; INP \CR}{to initialize the {\tt FRING}
       inputs and review them.}
\dispt{INDI\qs {\it n\/}; GETN\qs {\it m\/} \CR}{to select the data
       set on disk $n$ and catalog number $m$.}
\dispt{SOLINT\qs 1.05 * $x$ \CR}{to set the averaging interval in
       minutes slightly longer than the data interval ($x$)
       selected.}
\dispf{TIMERANG\qs {\it db} , {\it hb} , {\it mb} , {\it sb} , {\it
       de},  {\it he} , {\it me} , {\it se} \CR}{to specify the
       beginning day, hour, minute, and second and ending day, hour,
       minute, and second (wrt {\tt REFDATE}) of the data to be
       included.  Too much data will cause trouble.}
\dispt{DPARM(9) = 1 \CR}{to fit only delay, not rate.  {\it This is
       very important.}}
\dispt{DPARM(4) = $t$ \CR}{to help the task out by telling it the
       integration time $t$ in seconds.  Oddities in data sample times
       may cause {\tt FRING} to get a very wrong integration time
       otherwise.}
\dispt{INP \CR}{to check the voluminous inputs.}
\dispt{GO}{to run the task, writing {\tt SN} table 1 with delays for
       each antenna, IF, and polarization.}
\dispe{The different IFs in current \Indx{EVLA} data sets may come
from different basebands and therefore have different residual
delays.  The option {\tt APARM(5) = 3} to force the first $N_{if}/2$
IFs to have one delay solution while the second half of the IFs has
another is strongly recommended, but only when the first half all come
from one of the ``AC'' or ``BD'' basebands (hardware IFs) and the
second half come from the other.  The 3-bit data path of the EVLA
actually has four hardware IFs, so {\tt APARM(5) = 4} produces four
delay solutions, dividing the IFs in quarters.  The Widar correlator
may now separate the hardware basebands into unequal numbers of
spectral windows.  In {\tt 31DEC18}, {\tt APARM(5) = -1} will cause
{\tt FRING} to use {\tt BPARM} to specify the IFs used in each delay
solution.  Note that, at low frequencies, the phases may also be
affected by dispersion (phase differences proportional to wavelength).
{\tt FRING} now offers {\tt APARM(10)} to enable solving for a single
delay plus dispersion from the fitted single-IF delays.  This {\tt SN}
table will need to be applied to the main {\tt CL} table created by
{\tt INDXR} or {\tt OBIT}\@.\todx{FRING}\Iodx{calibration}}
\dispt{TASK\qs '\tndx{CLCAL}' ; INP \CR}{to look at the necessary
       inputs.}
\dispt{TIMERANG\qs 0 \CR}{to reset the time range.}
\dispt{GAINUSE 0 ; GAINVER 0 \CR}{to select the highest {\tt CL}
       table as input and write one higher as output (probably version
       2 and 3, resp.~in this case).}
\dispt{SNVER 1 ; INVER 1 \CR}{to use only the {\tt SN} table just
       created.}
\dispt{INP \CR}{to review the inputs.}
\dispt{GO \CR}{to make an updated calibration table.}
\dispe{Be sure to apply this (or higher) {\tt CL} table with {\tt
DOCALIB\qs 1} in all later steps.}

\Subsections{Primary flux density calibrators}{SETJYcalc}

      Careful measurements made with the D array of the VLA have shown
that the Baars {\it et al.\/}\footnote{1977, {\it Astr.~\&\ Ap.}, 61,
99} formul\ae\ for ``standard'' calibration sources are in error
slightly, based on the assumption that the Baars' expression for 3C295
is correct.   Revised values of the coefficients have been derived by
Rick Perley and Brian Butler.  Task {\tt \Tndx{SETJY}} has these
formulae built into it, giving you the option ({\us OPTYPE 'CALC'}) of
letting it calculate the fluxes for primary calibrator sources 3C48,
3C123, 3C138, 3C147, 3C196, 3C286, 3C295, and 1934-638.  The default
setting of {\us APARM(2) = 0}) will calculate the flux densities based
on Perley-Butler 2017 values which cover the range 50 MHz to 50 GHz
for the primary calibrators:3C48, 3C138, 3C147, 3C286, and 3C295 plus
3C123 and 3C196, all with a number of synonyms.  Other sources which
will be computed, but which may not be good calibrator sources,
are J0444-2809, PictorA, 3C144 (Taurus A, Crab Nebula), 3C218 (Hydra
A), 3C274 (Virgo A), 3C348 (Hercules A), 3C353, 3C380, 3C405 (Cygnus
A), 3C444, and 3C461 (Cassiopeia A).  multiple names are also allowed
for these sources; see {\us EXPLAIN SETJY}\@.  Most of the sources in
the extra list have limits on the frequency range over which the
function is valid; {\tt SETJY} will tell you if the frequencies are
out of range.  Higher values of {\tt APARM(2)} select older systems of
coefficients if you need them to match previous data reductions.  {\tt
SETJY} will recognize both the 3C and IAU designations (B1950 and
J2000) for the standard sources.  See {\us EXPLAIN SETJY} for details
of all models.  You may also insert your own favorite values for these
sources instead ({\us OPTYPE = ' '}) and you will have to insert
values for any other gain calibrators you intend to use (or use {\tt
GETJY}, see \Sec{getjy}).  Adverbs {\tt SPECINDX} and {\tt SPECURVE}
allow you to enter \indx{spectral index} information to help set the
calibrator fluxes.\Iodx{calibration}\Iodx{EVLA}

      Unfortunately, since all the primary flux calibrators are
resolved by the VLA in most configurations and at most frequencies,
they cannot be used directly to determine the amplitude calibration of
the antennas without a detailed model of the source structure, see
\Rfig{3C48_X} as an example.  Beginning in April 2004, model images
for the calibrators at some frequencies are included with \AIPS\@.
Models of 3C286, 3C48, 3C138, and 3C147 are available for all 6
traditional bands of the VLA {\it except} 90\,cm and even for S band
of the EVLA.  Type {\us \tndx{CALDIR} \CR} to see a list of the
currently available calibrator models.  Sources which are small enough
to be substantially unresolved by the VLA have variable flux densities
which must be determined in each observing session.  A common method
used to determine the flux densities of the secondary calibrators from
the primary calibrator(s) is to compare the amplitudes of the gain
solutions from the procedure described below.

    Use {\tt SETJY} to enter/calculate the flux density of each
primary flux density calibrator.  The ultimate reference for the VLA
is 3C295, but 3C286 (1328+307), which is slightly resolved in most
configurations at most frequencies, is the most useful primary
calibrator.  {\tt CALIB} has an option that will allow you to make use
of Clean component models for calibrator sources.  You are strongly
encouraged to use the existing models.  If you follow past practice
at the VLA, you may have to restrict the \uv\ range over which you
compute antenna gain solutions for 3C286, and may therefore insert a
``phony'' flux density appropriate only for that \uv\ range at this
point.  In both cases, the following step should be done.  {\tt CALIB}
will scale the total flux of the model to match the total flux of the
source recorded by {\tt SETJY} in the source table.  This corrects for
the model being taken at a somewhat different frequency than your
observations and for the model containing most, but not all, of the
total flux.  An example of the inputs for {\tt SETJY}, where you let
it calculate the flux, would be:
\dispt{TASK\qs 'SETJY' ; INP \CR}{}
\dispt{SOURCES\qs '3C286' , '\ ' \CR}{if you used 3C286 as the source
            name; see the {\tt LISTR} scan output.}
\dispt{BIF\qs 1 ; EIF\qs 2 \CR}{will calculate for both ``AC'' and
            ``BD'' IFs.}
\dispt{OPTYPE\qs 'CALC' \CR}{perform the calculation.}
\dispt{APARM(2)\qs = 0 \CR}{to use the VLA ``2013'' coefficients.}
\dispt{INP \CR}{to review inputs.}
\dispt{GO \CR}{when inputs okay.}

Or you can set the flux manually as shown below:
\dispt{TASK\qs 'SETJY' ; INP \CR}{}
\dispt{SOURCES\qs '3C286' , '\ ' \CR}{if you used 3C286 as the source
          name.}
\dispt{ZEROSP\qs 7.41 , 0 \CR}{I flux 7.41 Jy, Q, U, V fluxes 0.}
\dispt{BIF\qs 1 ; EIF\qs 1 \CR}{selects first IF \hbox{IF}.}
\dispt{INP \CR}{to review inputs.}
\dispt{OPTYPE\qs '\ '}{use values given in \hbox{{\tt ZEROSP}}.}
\dispt{GO \CR}{when inputs okay.}
\dispt{BIF\qs 2 ; EIF\qs 2 \CR}{selects second IF \hbox{IF}.}
\dispt{ZEROSP 7.46, 0 \CR}{I flux 7.46 Jy at the $2^{\und}$ IF,
          Q, U, V fluxes 0.}
\dispt{GO \CR}{}
\dispe{Note that, although {\tt SOURCES} can accept a source list,
{\tt ZEROSP} has room for only one set of I, Q, U, V flux densities.
To set the flux densities for several different sources or IFs, you
must therefore rerun {\tt \tndx{SETJY}} for each source and each IF,
changing the {\tt SOURCES}, {\tt BIF}, {\tt EIF}, and {\tt ZEROSP}
inputs each time.  Alternatively, set {\tt ZEROSP} to the flux at 1
GHz and enter {\tt SPECINDX} and {\tt SPECURVE} adverbs to describe
the dependence with frequency.\iodx{spectral index}}\Iodx{calibration}

{\tt CALIB} will use the V polarization flux in the source table if
one has been entered.  The RR polarization will be calibrated to I+V
and the LL to I-V.  While this has little practical use with circular
polarizations because V is almost always negligible, it can be used
for linearly polarized data from the WSRT\@.  That telescope has
equatorially mounted dishes, so the XX polarization is I-Q and the
YY is I+Q independent of parallactic angle.  For WSRT data, you should
relabel the polarizations to RR/LL and enter {\tt I, 0, 0, -Q} for
{\tt ZEROSP}, since Q is not negligible in standard calibrators.

If the {\tt LISTR} scan listing shows calibration sources with a blank
{\tt CALCODE}, you should also use {\tt SETJY} to correct this.  Set
{\tt OPTYPE = ' '}, set the desired {\tt SOURCE} name, and set {\tt
CODETYPE} to the desired {\tt CALCODE}.  Expert users sometimes prefer
{\tt \tndx{TABPUT}} for this function; run {\tt \tndx{PRTAB}} to see
what values must be set in {\tt PIXXY}\@.

\Subsections{Calibrating the bandpass shape with {\tt BPASS}}{BPASSit}

Bandpass calibration is for the modern VLA {\it required} rather than
merely {\it recommended}.  Having chosen those channels which may be
reliably used to normalize the bandpass functions,
\dispt{DEFAULT\qs \tndx{BPASS} ; INP}{to reset all adverbs and choose
       the task.}
\dispt{INDI\qs {\it n\/}; GETN\qs {\it m\/} \CR}{to select the data
       set on disk $n$ and catalog number $m$.}
\dispt{DOCAL\qs 1 \CR}{to apply the delay calibration --- {\it very
       important}.}
\dispt{CALSOUR\qs '{\it bandpass\_cal\/}' \CR}{to select the strong
       bandpass calibrator.}
\dispt{SOLINT\qs 0 \CR}{to find a bandpass solution for each scan
       on the BP calibrator.}
\dispf{ICHANSEL\qs $c11, c12, 1, if1, c21, c22, 1, if2, c31, c32, 1,
       if3, \ldots\/$ \CR}{to select the range(s) of channels which
       are reliable for averaging in each IF\@.  Use the central 30\%\
       of the channels if your calibrators are all very strong or
       more like 90\%\ if they are not.  Remember these values
       --- you will use them again.}
\dispt{BPASSPRM(5)\qs 1 ; BPASSPRM(10)\qs 3 \CR}{to normalize the
       results only after the solution is found using the channels
       selected by {\tt ICHANSEL}\@.  Use {\tt BPASSP(5)=-1} if your
       phases are not stable within each scan.}
\dispt{GO \CR}{to make a bandpass ({\tt BP}) table.}
\dispe{Do not use spectral smoothing at this point unless you want to
use the same smoothing forever after.  Apply the flag table.  A model
for the calibrator may be used; see \Sec{vlacalmodels}.\Iodx{EVLA}}

{\tt BPASS} now contains the adverbs {\tt SPECINDX} and {\tt SPECURVE}
through which the \indx{spectral index} and its curvature (to higher
order than is known for any source) may be entered.  For the standard
amplitude calibrators 3C286, 3C48, 3C147, and 3C138, these parameters
are known and will be provided for you by {\tt BPASS}\@.  For other
sources, you may provide these parameters, but {\tt BPASS} will fit
the fluxes in the {\tt SU} table for a spectral index (including
curvature optionally) if you do not.  Note that, if no spectral index
correction is applied, the spectral index of the calibration source
will be frozen into the target source.  Bandwidths on the EVLA are
wide enough that this is a serious problem.  If you do not know the
spectral index of your calibration source, {\tt BPASS} itself or the
new task {\tt \Tndx{SOUSP}} may be used to determine the spectral
indices from the {\tt SU} table.  Of course, that means that {\tt
GETJY} must already have been run.  Since {\tt BPASS} must usually be
run before {\tt CALIB} and hence {\tt GETJY}, this suggests that one
may have to iterate this whole process at least once.  {\tt SOUSP} now
offers the option of correcting one or more {\tt SN} tables after it
adjusts the source fluxes for the spectral index it determined.  This
may reduce the need for further iterations.

Note that the bandpass parameters shown above assume that the phases
are essentially constant through each scan of the bandpass calibrator.
This may not be true, particularly at higher frequencies.  In this
case, you have two choices.  One is to set {\tt BPASSPRM(5)} to 0
which will determine the vector average of the channels selected by
{\tt ICHANSEL} at every integration and divide that into the data of
that integration.  This will remove all continuum phase fluctuations,
but runs a risk of introducing a bias in the amplitudes since they do
not have Gaussian statistics.  {\tt BPASSPRM(5) = -1} now applies a
phase-only correction on a record-by-record basis.  A better
procedure, which is rather more complicated, is as follows.  Use {\tt
SPLIT} to separate the bandpass calibrator scans into a separate
single-source file applying any flags and delay calibration and the
like.  Then run {\tt CALIB} on this data set with a short {\tt SOLINT}
to determine and apply a phase-only self-calibration.  On the $uv$
data set written out by {\tt CALIB}, run {\tt BPASS} using the
parameters described in the previous paragraph.  Finally, use {\tt
TACOP} to copy the {\tt BP} table back to the initial data set.

     The spectral quality of the final images has been found to be
determined in part by the quality of the bandpass solutions.  In
particular, for reasons which are not yet known, the bandpasses are
not exactly antenna dependent especially in the edge channels.  This
``closure error'' may be measured in individual and statistical ways
by {\tt BPASS} and reported to you.  To check on this problem for your
data set, set\Iodx{EVLA}
\dispt{MINAMPER\qs {\it a\/} \CR}{to count and, if {\tt BPASSPRM(2)}$
           > 1$, to report amplitude closure failures $> a$ per cent.
           Note that closure errors are accumulated as logarithms so
           that 0.5 and 2.0 are both errors of 100\%.}
\dispt{MINPHSER\qs {\it p\/} \CR}{to count and, if {\tt BPASSPRM(2)}$
           > 1$, to report phase closure failures $> p$ degrees.}
\dispt{BPASSPRM(2)\qs 1 \CR}{to report statistics of amplitude and
           phase closure failures without reporting individual
           failures.}
\dispt{BPASSPRM(6)\qs {\it a\/} \CR}{to report all channels in which
           the average amplitude closure error $> a$ per cent.}
\dispt{BPASSPRM(7)\qs {\it p\/} \CR}{to report all channels in which
           the average phase closure error $> p$ degrees.}
\dispt{SOLTYPE\qs 'R' \CR}{to select robust solutions which discard
           data with serious closure problems.  Try other types if
           there are solution failures.}
\dispe{It is probably a good idea to set {\tt MINAMPER} and {\tt
MINPHSER} fairly high (\ie\ 20 and 12) to make a big deal only about
major excursions, but to set {\tt BPASSPRM(6)} and {\tt (7)} fairly
low (\ie\ 0.5 and 0.5) to view the spectrum of closure errors (which
will look a lot like the spectrum of noise on your final Clean
images).  There is even a task called {\tt \Tndx{BPERR}} which will
summarize and plot the error reports generated by {\tt BPASS} and
written to text files by \hbox{{\tt PRTMSG}}.}

     The bandpass solutions are calculated at each bandpass calibrator
scan.  As a consequence, they are likely to be unevenly spaced in time
and may even have times (due to on-line or later editing) at which
there are solutions for some IFs and polarizations but not all.  When
the latter happens, program source data will be lost unless the
missing solutions are filled in.  The task {\tt \Tndx{BPSMO}} may be
used for this purpose or to create a new {\tt BP} table at regular
time intervals using one of a number of time-smoothing functions.  Set
{\tt APARM(4) = -1} for the ``repair'' mode or set {\tt APARM(4)} to
the desired {\tt BP} interval.

     After the bandpasses have been generated, you can examine them
using tasks {\tt \tndx{BPLOT}} and \hbox{{\tt \tndx{POSSM}}}.  You can
obtain an average from all antennas
with\Iodx{calibration}\Iodx{bandpass}
\todx{BPASS} \iodx{spectral-line}
\dispt{TASK\qs 'POSSM' \CR}{}
\dispt{INDI\qs {\it i\/} ; GETN\qs {\it j\/} \CR}{to specify the line
             data set.}
\dispt{SOURCES\qs '{\it cal1\/}' , '{\it cal2\/}' , $\ldots$ \CR}{to
             specify the bandpass calibrators.}
\dispt{ANTENNAS\qs 0 \CR}{to include all antennas.}
\dispt{TIMER\qs 0 \CR}{to average over all times.}
\dispt{BCHAN\qs 1 ; ECHAN 0 \CR}{to display all channels.}
\dispt{BPVER\qs 1 \CR}{to select the {\tt BP} table.}
\dispt{FREQID\qs 1 \CR}{to set the {\tt FQ} value to use.}
\dispt{APARM\qs= -1, 0 \CR}{to do a scalar average and have the plot
             self-scaled and labeled in channels.}
\dispt{APARM(8)\qs 2 \CR}{to plot {\tt BP} table data.}
\dispt{NPLOTS\qs 0 \CR}{to make one plot only, averaging all included
             data.}
\dispt{INP \CR}{to review the inputs --- check closely.}
\dispt{GO\qs \CR}{to run the program when inputs set correctly.}
\dispt{GO\qs \tndx{LWPLA} \CR}{to send the plot to the (PostScript)
             printer/plotter.}
\dispe{To view each antenna individually, using the TV to save paper}
\dispt{DOTV\qs TRUE \CR}{to use the \hbox{TV}.}
\dispt{NPLOTS\qs 1 \CR}{to plot one antenna per page/screen.}
\dispt{GO \CR}{to display the bandpasses, averaged over time, on the
            TV with one antenna per screen.}
\dispe{{\tt POSSM} shows each screen for 30 seconds before going
ahead.  You can cause it wait indefinitely by hitting button A, speed
it up by hitting TV buttons B or C, or tell it to quit by hitting
button D\@.  If {\us DOTV = -1}, then {\tt \tndx{POSSM}} makes
multiple plot extension files, which can be sent to the printer
(individually or collectively) by {\tt LWPLA}\@.  You might want to
use a larger value of {\tt NPLOTS} to reduce the number of pieces of
paper.}

     {\tt \Tndx{BPLOT}} is used to create one or more plots (on the TV
or in plot files) of the selected bandpass table.  The plots will be a
set of profiles separated on the vertical axis by an increment in time
or antenna number (depending on the sort selected).  More than one
plot for more than one antenna or more than one time may be generated.
Multiple IFs and polarizations will be plotted along the horizontal
axis if they are present in the {\tt BP} table and selected by the
adverbs.  Thus, {\tt BPLOT} is useful for plotting the change in
bandpass shape as a function either of time or of antenna.

     In {\tt 31DEC16}, task {\tt \Tndx{BPEDT}} offers a TV-interactive
graphical way to check your {\tt BP} table.  It displays the amplitude
and phase of one antenna to be edited and up to 10 more antennas for
comparison.  The task is like {\tt EDITA} except that the horizontal
axis is spectral channel rather than time.  You may walk through the
entire bandpass table selecting the editable antenna and time in
sequence.  If some of the solutions show residual effects from RFI,
then you may generate flags in a flag table to delete certain spectral
channels for certain antennas and times.  If you do this, of course,
you should re-run {\tt BPASS} applying the new flags.\Iodx{EVLA}

     The {\tt BP} tables are applied to the data by setting the adverb
{\tt DOBAND} $> 0$ and selecting the relevant {\tt BP} table with the
adverb \hbox{{\tt BPVER}}.  There are three modes of \Indx{bandpass}
application.  The first ({\us DOBAND 1}) will average all bandpasses
for each antenna within the time range requested, generating a global
solution for each antenna.  The second mode ({\us DOBAND 2}) will use
the antenna bandpasses nearest in time to the data point being
calibrated.  The third mode ({\us DOBAND 3}) interpolates in time
between the antenna bandpasses and generates the correction from the
interpolated data.  This mode has been found to be required for VLA
data.  If {\tt BPSMO} was used to make a fairly finely sampled {\tt
BP} table, then {\tt \Tndx{DOBAND} 2} may be used.  Modes {\us DOBAND
4} and {\us DOBAND 5} are the same as modes 2 and 3, respectively,
except that data weights are ignored.\Iodx{calibration}

     It is often not possible to observe a strong bandpass calibrator
many times during a run.  In this case, one can run {\tt \tndx{BPASS}}
on the single scan on the strong calibrator and then remove the main
bandpass shape with {\tt DOBAND 1} in task {\tt SPLAT}\@.  Corrections
to this basic bandpass shape as a function of time may then be
determined with adequate signal-to-noise using task {\tt
\tndx{CPASS}}\@.  This task can be used to fit the residual bandpass
with a small number of parameters ($<<$ the number of spectral
channels) at each calibrator scan.  The results may then be applied
with {\tt DOBAND 2}\@.  Check the output of {\tt CPASS} carefully ---
it is capable of making bandpass shapes with large ripples that are
not present in the data.  {\tt CPASS} also knows about the flux
coefficients of all standard sources and can fit the spectral index of
other sources from the flux values in the {\tt SU} table.

\Subsections{Calibrating the complex antenna gains with {\tt VLACALIB}}{vlacalib}

\Subsubsections{Using calibrator models}{vlacalmodels}

It is now considered standard practice to use flux calibrator models
and you are strongly encouraged to do so.  As mentioned above, all the
primary flux calibrators are resolved at {\it most} frequencies and
configurations.  \Rfig{3C48_X} shows the visibilities and image of
the commonly used calibrator 3C48 at X-band, it is obvious this source
is far from being point like.  Since April 2004, source models have
been shipped with \AIPS\ as FITS files. Currently, models for 3C48,
3C286 and 3C138 are available for all bands except 90\,cm and 3C147 at
all bands except, X, C and 90\,cm.  Additional models are in the
works, so you should always check to see what is available:
\dispt{\tndx{CALDIR} \CR}{to list the available models by source name
          and band code.\Iodx{calibration}}

You can load a model with the procedure {\tt VLACALIB} (see below) or
you can load the model in manually using {\tt CALRD}\Todx{CALRD}:
\dispt{TASK\qs 'CALRD' \CR}{to select the calibrator source reading
          task.}
\dispt{OBJECT\qs '3C286' \CR}{to load a model of 3C286.}
\dispt{BAND\qs 'K' \CR}{to select the available model at K band.}
\dispt{OUTDISK\qs $n$ \CR}{to write the model image and Clean
          components to disk $n$.}
\dispt{GO \CR}{to run the task and load the model.}

\subsubsections{Calibrating complex gains}

The primary calibration task in \AIPS\ is {\tt CALIB}\@.  Most of the
complexity of {\tt CALIB} can be hidden using the procedure {\tt
\tndx{VLACALIB}}\@.  Before attempting to use this procedure, you must
first load it by typing:\Iodx{EVLA}
\dispt{RUN \tndx{VLAPROCS} \CR}{to compile the procedures.}
\dispe{Type {\tt HELP VLAPROCS} for a full list of the procedures
available in {\tt VLAPROCS}\@.}

The procedure {\tt VLACALIB} {\it automatically} downloads and uses
calibrator models, if one is available and the inputs are set correctly.
After you have loaded {\tt VLACALIB} you may invoke the calibrator model
usage by setting:
\dispt{INDI\qs {\it n\/} ; GETN\qs {\it m\/} \CR}{to select the data
          set, $n=3$ and $m=1$ above.}
\dispt{CALSOUR\qs = '{\it Cala\/}' \CR}{to name {\it one} primary
          flux calibrator to invoke automatic calibrator model
          usage.}
\dispt{UVRANGE\qs 0 \CR}{set to zero to invoke automatic calibrator
          model usage.}
\dispt{ANTENNAS\qs 0 \CR}{set to zero to invoke automatic calibrator
          model usage.}
\dispt{REFANT\qs {\it n\/} \CR}{reference antenna number --- use a
          reliable antenna located near the center of the array.}
\dispt{MINAMPER\qs 10 \CR}{display warning if baseline disagrees in
         amplitude by more than {\it 10\%\/}\ from the model.}
\dispt{MINPHSER\qs 10 \CR}{display warning if baseline disagrees by
         more than $10^{\circ}$ of phase from the model.}
\dispt{FREQID\qs 1 \CR}{use {\tt FQ} number 1.}
\dispt{DOPRINT\qs 1 ; OUTPRINT\qs '\qs' \CR}{to generate significant
         printed output on the line printer.}
\dispt{INP\qs VLACALIB \CR}{to review inputs.}
\dispt{VLACALIB \CR}{to make the solution and print results.}
\dispe{This procedure will load the calibrator model (using {\tt
CALRD}) and use it when it runs {\tt CALIB}, then print any messages
from {\tt CALIB} about closure errors on the line printer, and finally
run {\tt \tndx{LISTR}} to print the amplitudes and phases of the
derived solutions.  Plots of these values may be obtained using task
\hbox{{\tt \tndx{SNPLT}}}.}

Then you may select the model image with {\tt GET2N} for use directly
in {\tt \tndx{CALIB}}\@.  Example inputs for {\tt CALIB} are:
\dispt{TASK\qs 'CALIB'; INP \CR}{to select task and review inputs.}
\dispt{INDI\qs {\it n\/} ; GETN\qs {\it m\/} \CR}{to select the data
          set, $n=3$ and $m=1$ above.}
\dispt{CALSOUR\qs = '{\it Cala\/}' , '\qs' \CR}{flux calibrator for
          which you have a model.}
\dispt{UVRANGE\qs 0 \CR}{no \uv\ limits needed.}
\dispt{ANTENNAS\qs 0 \CR}{antenna selection not needed.}
\dispt{REFANT\qs {\it n\/} \CR}{reference antenna number --- use a
          reliable antenna located near the center of the array.}
\dispt{WEIGHTIT\qs 1 \CR}{to select $1/\sigma$ weights which may be
          more stable.}
\dispt{IN2DI\qs {\it o\/} ; GET2N\qs {\it p\/} \CR}{to select the
          model.}
\dispt{NCOMP\qs 0 \CR}{to use all components.}
\dispt{SOLMODE\qs 'A\&P' \CR}{to do amplitude and phase solutions.}
\dispt{APARM(6)\qs 2 \CR}{to print closure failures.}
\dispt{MINAMPER\qs 10 \CR}{to display warning if baseline disagrees in
         amplitude by more than {\it 10\%\/}\ from the model.}
\dispt{MINPHSER\qs 10 \CR}{to display warning if baseline disagrees by
         more than $10^{\circ}$ of phase from the model.}
\dispt{CPARM(5)\qs 1 \CR}{to vector average amplitudes over spetral
         channels and then scalar average them over time before
         determining solution.}
\dispt{FREQID\qs 1 \CR}{to use {\tt FQ} number 1.}
\dispt{INP\qs CALIB \CR}{to review inputs.}
\dispt{GO \CR}{to make the solution.\Iodx{calibration}\Iodx{EVLA}}

{\tt CALIB} will use the clean components table attached to the model
to find antenna gain solutions.  It will sum the clean components
within a certain radius of the center of the map (so that confusing
sources that are part of the model do not influence the gain) and
scale them to the flux in the {\tt SU} table. Therefore, you must
still run {\tt SETJY} before running {\tt CALIB}.

After running {\tt CALIB} check the solutions for all antennas with
{\tt SNPLT} or {\tt LISTR (OPTYPE='GAIN')}.  If you have multiple
primary or secondary calibrators you will have to run {\tt CALIB}
separately for each, using models where they are available and
restricting the {\tt UVRANGE} and {\tt ANTENNAS} where they are not.
You can either write into the same {\tt SN} table by setting {\tt
SNVER} to a table number or to different {\tt SN} tables by setting
{\tt SNVER = 0}. Then you you can proceed as normal flagging and
editing your data and proceed to final calibration as described in
\Sec{getjy}.

\begin{figure}
\centering
%\resizebox{\hsize}{!}{\gname{3C48Xuv}\hspace{0.5cm}\gname{3C48Xim}}
\resizebox{\hsize}{!}{\gbb{524,526}{3C48Xuv}\hspace{0.5cm}\gbb{525,539}{3C48Xim}}
\caption[3C48 at X-band]{Displays of the visibilities (left) and
image (right) for the fundamental \Indx{calibration} source 3C48\@.
The plots were made using {\tt UVPLT}, {\tt KNTR}, and {\tt LWPLA};
see \Sec{plotuv} and \Sec{plotcntr}.  Data from all VLA configurations
including the VLBA antenna in Pie Town were used.  A point source
would have visibilities that have a constant amplitude at all
baselines and an image matching the beam plotted in the lower-left
corner.}
\label{fig:3C48_X}
\end{figure}

\Subsubsections{Flux calibration without calibrator models}{novlacal}

Normally, the primary flux calibration sources are located too far
from the target sources to be used to track time variations in phase
and amplitude.  Therefore, secondary calibration sources whose fluxes
and spectral indices are not known in advance.

     Once you have read in procedure {\tt \tndx{VLACALIB}} (see
\Sec{vlacalmodels}), you may use it to invoke {\tt \tndx{CALIB}}\@.
You will have to do this once for each calibrator, unless you can use
the same {\tt UVRANGE} for more than one of them.  Thus,
\dispt{INDI\qs {\it n\/} ; GETN\qs {\it m\/} \CR}{to select the data
          set, $n=3$ and $m=1$ above.}
\dispt{CALSOUR\qs = '{\it Cala\/}' , '{\it Calx\/}' \CR}{to name two
          calibrators using the same {\tt UVRANGE} and other adverb
          values.}
\dispt{UVRANGE\qs {\it uvmin\/} {\it uvmax\/} \CR}{\uv\ limits, if
          any, in kilo$\lambda$.}
\dispt{ANTENNAS\qs {\it list\ of\ antennas\/} \CR}{antennas to use for
          the solutions, see discussion above.}
\dispt{REFANT\qs {\it n\/} \CR}{reference antenna number --- use a
          reliable antenna located near the center of the array.}
\dispt{MINAMPER\qs 10 \CR}{display warning if baseline disagrees in
         amplitude by more than {\it 10\%\/}\ from the model.}
\dispt{MINPHSER\qs 10 \CR}{display warning if baseline disagrees by
         more than $10^{\circ}$ of phase from the model.}
\dispt{DOPRINT\qs 1 ; OUTPRINT\qs '\qs' \CR}{to generate significant
         printed output on the line printer.}
\dispt{FREQID\qs 1 \CR}{use {\tt FQ} number 1.}
\dispt{INP\qs VLACALIB \CR}{to review inputs.}
\dispt{\tndx{VLACALIB} \CR}{to make the solution and print results.}
\dispe{This procedure will first run {\tt CALIB}, then print any
messages from {\tt CALIB} about closure errors on the line printer,
and finally run {\tt \tndx{LISTR}} to print the amplitudes and phases
of the derived solutions.  Plots of these values may be obtained using
task {\tt \tndx{SNPLT}}\@.\Iodx{EVLA}}

      If the secondary calibrators require different values of {\tt
UVRANGE}, then {\tt CALIB} must be run until it has run for all
calibration sources.  Attached to your input data set is a solution
{\tt SN} table.  Each run of {\tt CALIB} writes in this table (if {\tt
SNVER = 1}), for the times of the included \Indx{calibration} scans,
the solutions for all IFs using the flux densities you set for your
calibrators with {\tt SETJY} or {\tt \tndx{GETJY}}\@.  ({\tt CALIB}
assumes a flux density of 1 Jy if no flux density is given in the {\tt
SU} table.)  If a solution fails, however, the whole {\tt SN} table
can be compromised, forcing you to start over.  It is possible to
write multiple {\tt SN} tables with {\us SNVER = 0}.  Later programs
such as {\tt GETJY} and {\tt CLCAL} will merge all {\tt SN} tables
which they find (if told to do so). Tables with failed solutions must
be deleted.\todx{CALIB}

\Subsections{Bootstrapping secondary flux-density calibrators}{getjy}

      Task {\tt \tndx{GETJY}} can be used to determine the flux
density of the secondary flux calibrators from the primary flux
calibrator based on the flux densities set in the {\tt SU} table and
the antenna gain solutions in the {\tt SN} tables.  The {\tt SU} and
{\tt SN} tables will be updated by {\tt GETJY} to reflect the
calculated values of the secondary calibrators' flux densities.  This
procedure should also work if (incorrect) values of the secondary
calibrators' flux densities were present in the {\tt SU} table when
{\tt CALIB} was run. Bad or redundant {\tt SN} tables should be
deleted using {\tt EXTDEST} before running {\tt GETJY}, or avoided by
selecting tables one at a time with adverb \hbox{{\tt SNVER}}.

      To use {\tt GETJY}:
\dispt{TASK\qs 'GETJY' ; INP \CR}{}
\dispt{SOURCES\qs '{\it cal1\/}' , '{\it cal2\/}' , '{\it cal3\/}'
           $\ldots$ \CR}{to select secondary flux calibrators.}
\dispt{CALSOU\qs '3C286' , '\ ' \CR}{to specify primary flux
           calibrator(s).}
\dispt{CALCODE\qs '\ ' \CR}{to use all calibrator codes.}
\dispt{BIF\qs 1 ; EIF\qs 2 \CR}{to do both IFs.}
\dispt{FREQID\qs 1 \CR}{to use {\tt FQ} number 1.}
\dispt{ANTENNAS\qs 0 \CR}{to include solutions for all antennas.}
\dispt{TIMERANG\qs 0 \CR}{to include all times.}
\dispt{SNVER\qs 0 \CR}{to use all {\tt SN} tables.}
\dispt{INP \CR}{to review inputs.}
\dispt{GO \CR}{to run the task when the inputs are okay.}

      {\tt GETJY} will give a list of the derived flux densities and
estimates of their uncertainties.  These are now found by ``robust''
methods and additional information about numbers of aberrant solutions
are given.  If any of the uncertainties are large, then reexamine the
{\tt SN} tables as described above and re-run {\tt CALIB} and/or {\tt
GETJY} as necessary.  Multiple executions of {\tt GETJY} will not
cause problems as previous solutions for the unknown flux densities
are simply overwritten.  You may wish to run the task {\tt
\tndx{SOUSP}} to determine the spectral indices of your calibrators
from their fluxes in the {\tt SU} table.  You can even replace the
values in the {\tt SU} table with the curve fit by {\tt SOUSP} and, in
{\tt 31DEC15} correct the gains in one or more {\tt SN} tables with
the newly determined fluxes.  These \indx{spectral index} parameters
may be useful in running {\tt BPASS} and {\tt PCAL}\@.  However, {\tt
BPASS} knows the flux coefficients for all standard \Indx{calibration}
sources and can fit the spectral index of other calibration sources
from the {\tt SU} table.\Iodx{EVLA}

\Subsections{Editing visibility data with {\tt EDITA}}{edita}

     The task {\tt \Tndx{EDITA}} uses the graphics planes on the
\AIPS\ TV display to plot data from tables and to offer options for
\Indx{editing} (deleting, \Indx{flagging}) the associated \uv\ data.
Only the {\tt TY} (historic VLA system temperature), {\tt SY} (modern
VLA SysPower table), {\tt SN} (solution), and {\tt CL} (calibration)
tables may be used.  We recommend using {\tt EDITA} with the {\tt SN}
table updated by {\tt GETJY} as a quick and easy way to examine the
quality of the solutions.  If a small amount is disturbed, use {\tt
EDITA} to flag the relevant data.  If the solutions for most antennas
are disturbed, make a note of the time and source and use other
flagging tools to find and flag the cause of the disturbance.  If you
flag more than a little, you should delete the {\tt SN} table(s) and
re-run {\tt VLACALIB} and {\tt GETJY}\@.  Try:
\dispt{TASK\qs 'EDITA ; INP \CR}{to review the inputs needed.}
\dispt{INDI\qs {\it n\/} ; GETN\qs {\it m\/} \CR}{to select the data
         set, $n=3$ and $m=1$ above.}
\dispt{INEXT\qs 'SN' \CR}{to use the solution table.}
\dispt{INVERS\qs 0 \CR}{to use the highest numbered table, usually 1.}
\dispt{TIMER\qs 0 \CR}{to select all times.}
\dispt{BIF\qs 1 ; EIF\qs 0 \CR}{to specify all IFs; you can then
         toggle between them interactively and even display all at
         once.}
\dispt{ANTENNAS\qs 0 \CR}{to display data for all antennas.}
\dispt{FLAGVER\qs 1 \CR}{to use flag ({\tt FG}) table 1 on input.}
\dispt{OUTFGVER\qs 0 \CR}{to create a new flag table with the flags
      from {\tt FG} table 1 plus the new flags.}
\dispt{SOLINT\qs 0 \CR}{to avoid averaging any samples.}
\dispt{DOHIST\qs FALSE \CR}{to omit recording the flagging in the
         history file.}
\dispt{DOTWO\qs TRUE \CR}{to view a $2^{\und}$ observable for
         comparison}
\dispt{CROWDED\qs TRUE ; DO3COL\qs TRUE \CR}{to allow plots with all
         polarizations and/or IFs simultaneously, using color to
         differentiate the polarizations and IFs.}
\dispt{INP \CR}{to review the inputs.}
\dispt{GO \CR}{to run the program when inputs set correctly.}
\dispe{If you make multiple runs of {\tt EDITA}, it is important to
make sure that the flagging table entries are all in one version
of the {\tt FG} table. The easiest way to ensure this is to should set
{\tt FLAGVER} and {\tt OUTFGVER} to 0 and keep it that way for all runs
of {\tt EDITA}\@.  This may create an excessive number of flag tables,
but unwanted ones may be deleted with {\tt EXTDEST}\@.  If you make a
mistake two flag tables may be merged with the task {\tt TAPPE}\@.  A
sample display from {\tt EDITA} is shown on the next page.
\Iodx{calibration}\Iodx{EVLA}}

The following discussion assumes that you have read \Sec{xas} and are
familiar with using the \AIPS\ TV display.  An item in a menu such as
that shown in the figure is selected by moving the TV cursor to the
item (holding down or pressing the left mouse button).  At this point,
the menu item will change color.  To obtain information about the
item, press \AIPS\ TV ``button D'' (usually the {\tt D} key and also
the {\tt F6} key on your keyboard).  To tell the program to execute
the menu item, press any of \AIPS\ TV buttons A, B, or \hbox{C}.
Status lines around the display indicate what is plotted and which
data will be flagged by the next flagging command.  In the figure
below, only the displayed antenna (2), and time range will be flagged.
You must display at least a few lines of the message window and your
main {\tt AIPS} window since the former will be used for instructions
and reports and the latter will be needed for data entry (\eg\ antenna
selection).\Iodx{editing}\Iodx{flagging}

\begin{figure}
\centering
\resizebox{\hsize}{!}{\gbb{544,658}{edita}}
\caption[{\tt EDITA} display]{Sample TV screen from \hbox{{\tt
\Tndx{EDITA}}}.  Solution phases are being used to edit the $uv$ data
with amplitude shown for comparison.  The {\tt \Tndx{EDITA}} menu (in
the boxes), the status lines (at the bottom left), the editing area
(bottom) of the data from the selected antenna (1), and the subsidiary
plots of data from selected secondary antenn\ae\ (2, 3) are shown.
When flagging, the edit tool (bar or box), and  the edit location
values are also displayed in different graphics planes which normally
appear in different colors.  In this example, with {\tt CROWDED=TRUE},
2 polarizations and 2 IFs are displayed and may be edited
simultaneously.  Color is used to differentiate polarizations and
IFs.\Iodx{editing}\Iodx{flagging}\Iodx{calibration}\Iodx{EVLA}}
\label{fig:edita}
\end{figure}

The first thing to do with {\tt EDITA} is to look at all of the
polarizations, IFs, and antenn\ae, in order to make notes about
disturbances in the solutions.  Use {\tt SWITCH POLARIZATION} to
switch between polarizations and {\tt ENTER IF} to select the IF to
edit.  Alternatively, {\tt NEXT POL/IF} will cycle through all
polarizations and IFs.  If {\tt CROWDED} was set to true, {\tt SWITCH
POLARIZATION} will cycle through displaying both polarizations as well
as each separately, and {\tt ENTER IF} will accept 0 as indicating
all.  These options appear only if there is more than one polarization
and/or more than one IF in the loaded data.  Use {\tt ENTER ANTENNA}
to select the antenna to be flagged and {\tt ENTER OTHER ANT} to
select secondary antenn\ae\ to be displayed around the editing area.
If the secondary antenn\ae\ have no obvious problems, then they do not
have to be selected for editing.  {\tt \Tndx{EDITA}} will plot all of
the times in the available area, potentially making a very crowded
display.  You may select interactively a smaller time range or
``frame'' in order to see the samples more clearly.  It is necessary
to select each frame in order to edit the data in that frame so it
helps to make the TV screen as big as possible with the {\tt F2}
button or your window manager.  Note that the vertical scales used by
{\tt EDITA} are linear, but that the horizontal scale is irregular and
potentially discontinuous.  Integer hours are indicated by tick marks
and the time range of the frame is indicated.  Use {\tt FLAG TIME} or
{\tt FLAG TIME RANGE} to delete data following instructions which will
appear on the message window.  While you are editing, the source name,
sample time and sample value currently selected will be displayed in
the upper left corner of the TV screen.

Having noted all obviously bad points, you may now selectively flag
visibility data.  Select {\tt SWITCH ALL IF}, {\tt SWITCH ALL TIME},
{\tt SWITCH ALL ANT}, {\tt SWITCH ALL POL}, and {\tt SWITCH ALL SOURC}
so that the next flag command(s) apply to the portion of the data that
should be flagged.  The current setting of these toggles is displayed
at the lower left corner.  The various {\tt FLAG} options allow you to
flag single times, time ranges, all samples above or below a selected
level, and single points either carefully or quickly.  Finally, apply
your flagging to your \uv\ data set by selecting {\tt EXIT} or throw it
all away by selecting {\tt ABORT}\@..
\Iodx{flagging}\Iodx{editing}

Note that it may also be profitable to run {\tt \tndx{EDITA}} using
the {\tt SY} table.  Unlike complex gains which reflect the values of
all antennas, the {\tt SY} values are strictly dependent only on the
specific antenna.  They are however affected by RFI in ways that may
not affect the visibilities, so flagging $uv$ data based on the {\tt
SY} values should be done with caution.  {\tt EDITA}, when run on {\tt
SY} or {\tt TY} tables, has the option to display the difference of
the current parameter with a running mean of that parameter.  Such
displays are a powerful editing tool.

\Subsections{Applying solutions to the calibration table with {\tt CLCAL}}{clcal}

     At this point you should have gain and phase solutions for the
times of all \Indx{calibration} scans, including the correct flux
densities for the secondary calibrators.  The next step is to
interpolate the solutions derived from the calibrators into the {\tt
CL} table for all the sources.  {\tt CLCAL} may be run multiple times
if subsets of the sources are to be calibrated by corresponding
subsets of the calibrators, unless you limit it to one or more tables
with {\tt SNVER} and {\tt INVERS}, {\tt CLCAL} assumes that all {\tt
SN} tables contain only valid solutions and concatenates all of the
{\tt SN} tables with the highest numbered one.  Therefore, any bad
{\tt SN} tables should be removed before using {\tt \tndx{CLCAL}}\@.
For \indx{polarization} calibration, it is essential that you
calibrate the primary flux calibrator (3C48 or 3C286) also so that you
can solve for the left minus right phase offsets and apply {\tt
PCAL}\@.\Iodx{EVLA}

     {\tt CLCAL} has caused considerable confusion and user error
because it implements two somewhat contrary views of its process.  The
older view, represented by previous versions of this \Cookbook, had
the user gradually building a final {\tt CL} table from multiple runs
of {\tt CLCAL}, each with a selected set of calibration sources,
target sources, antennas, time ranges, and so forth.  In this scheme,
the user had to take great care that the final {\tt CL} table actually
contained information for all antennas, sources, and times for which
it would be needed.  It was easy to get this wrong!  The second and
now prevailing view is that every execution of {\tt CLCAL} should
write a new {\tt CL} table containing all sources, antennas, and
times, but with a selected subset modified by the current execution.
This leads to there being a potentially large number of {\tt CL}
tables, but no data will be flagged due to the absence of data in the
{\tt CL} table.  The user will still have to be careful to insure that
all {\tt CL} records have received the needed calibration information.

      To use {\tt CLCAL}:
\dispt{TASK\qs CLCAL ; INP \CR}{to review the inputs.}
\dispt{SOURCES\qs '{\it sou1\/}' , '{\it sou2\/}' , '{\it sou3\/}' ,
           $\ldots$ \CR}{sources to calibrate, '\ ' means all.}
\dispt{CALSOUR\qs '{\it cal1\/}' , '{\it cal2\/}' , '{\it cal3\/}' ,
           $\ldots$ \CR}{calibrators to use for \hbox{{\tt SOURCES}}.}
\dispt{FREQID\qs {\it n\/} \CR}{use {\tt FQ} number {\it n\/}.}
\dispt{OPCODE\qs 'CALP' \CR}{to combine {\tt SN} tables into a {\tt
           CL} table, passing any records not altered this time.}
\dispt{GAINVER\qs 0 \CR}{to select the latest {\tt CL} table as
           input.}
\dispt{GAINUSE\qs 0 \CR}{to select a new output {\tt CL} table.}
\dispt{REFANT\qs {\it m\/} \CR}{to select the reference antenna; needed
           only if {\tt REFANT} reset since {\tt CALIB} was run.}
\dispt{INTERP\qs '2PT' \CR}{to use linear interpolation of the
           possibly smoothed calibrations..}
\dispt{SAMPTYPE\qs ' ' \CR}{to do no time-smoothing before the
           interpolation.}
\dispt{SAMPTYPE\qs 'BOX' \CR}{to use boxcar smoothing, followed by
           interpolation.}
\dispt{BPARM\qs {\it n\/} , {\it n\/} \CR}{to smooth, if {\tt BOX}
           selected, with an {\it n\/}-hr long boxcar in amplitude and
           phase.}
\dispt{DOBLANK\qs 1 \CR}{to replace failed solutions with smoothed
           ones but to use all previously good solutions without
           smoothing.}
\dispt{INP \CR}{to check inputs.}
\dispt{GO \CR}{to run {\tt CLCAL}\@.}
\dispe{Calibrator sources may also be selected with the {\tt QUAL} and
{\tt CALCODE} adverbs; {\tt QUAL} also applies to the sources to be
calibrated.  Note that {\tt REFANT} appears in the inputs because
\AIPS\ references all phases to those of the reference antenna.  If
none is given, it defaults to the one used in the most solutions.}

     The smoothing and interpolation functions in {\tt CLCAL} have
been separated into two adverbs and the smoothing parameters are
now conveyed with {\tt BPARM} and {\tt ICUT}\@.  In smoothing, the
{\tt DOBLANK} adverb is particularly important; it controls whether
good solutions are replaced with smoothed ones and whether previously
failed solutions are replaced with smoothed ones.  One can select
either or both.

     Note that {\tt \tndx{CLCAL}} uses both the {\tt GAINUSE} and {\tt
GAINVER} adverbs.  This is to specify the input and output {\tt CL}
table versions, which should be different.  If you are building a
single {\tt CL} table piece by piece, then these must be set carefully
and normally held fixed.  In the more modern view, they are set to
zero and the task takes the latest {\tt CL} table as input and makes a
new one.  {\tt CL} table version 1 is intended to be a ``virgin''
table, free of all injury from any calibration you do using the \AIPS\
package.  It may not always be devoid of information, as ``on-line''
corrections may be made and recorded here by some telescope systems,
\eg\ the VLBA\@.  The VLA, through tasks {\tt FILLM} or {\tt INDXR},
can now put opacity and antenna gain information in this file.  {\tt
CLCAL} and most other \AIPS\ tasks are forbidden to over-write version
1 of the {\tt CL} table.  This protects it from modification, and
keeps it around so that you may {\it reset\/} your \Indx{calibration}
to the raw state by using {\tt EXTDEST} to destroy all {\tt CL} table
extensions with versions higher than 1.  Be careful doing this, since
you rarely want to delete {\tt CL} version 1.  Should you destroy {\tt
CL} table version 1 accidentally, you may generate a {\it new\/} {\tt
CL} table version 1 with the task {\tt INDXR}\@.  This new {\tt CL}
table may contain the calibration generated from the weather and
antenna gain files.\Iodx{EVLA}

     If you have any reason to suspect that the calibration has gone
wrong --- or if you are calibrating data for the first time --- you
should examine the contents of the output {\tt CL} table.  The task
{\tt \tndx{EVASN}} may help you determine the degree of phase and
amplitude coherence in your calibration table.  A lack of coherence
suggests that the calibration is rather uncertain.  Task {\tt
\tndx{SNPLT}} will provide you with a graphical display:
\dispt{DEFAULT\qs \tndx{SNPLT} ; INP}{to reset all adverbs and choose
       the task.}
\dispt{INDI\qs {\it n\/}; GETN\qs {\it m\/} \CR}{to select the data
       set on disk $n$ and catalog number $m$.}
\dispt{INEXT \qs 'CL' \CR}{to examine the latest calibration table.}
\dispt{OPTYPE\qs 'AMP' \CR}{to examine amplitudes.}
\dispt{OPTYPE\qs 'PHAS' \CR}{to examine phases (do both).}
\dispt{NPLOTS\qs 4 ; DOTV 1 \CR}{to have 4 plots per page on the TV
       display.}
\dispt{OPCODE\qs 'ALIF' ; DO3COL\qs 1\CR}{to display all selected IFs
       in each plot using colors from red to blue.}
\dispt{INP \CR}{to check the inputs.}
\dispt{GO \CR}{to examine your {\tt CL} table.}
\dispe{Note that {\tt SNPLT} may be used to examine {\tt SN}, {\tt
SY}, and {\tt TY} tables as well.  {\tt OPCODE} controls whether 1 IF
or all IFs and 1 polarization or all polarizations are displayed.}

To test that the calibrators are actually properly calibrated by the
new {\tt CL} table, use {\tt UVPLT}\@  Try
\dispt{DEFAULT\qs \tndx{UVPLT} ; INP}{to reset all adverbs and choose
       the task.}
\dispt{INDI\qs {\it n\/}; GETN\qs {\it m\/} \CR}{to select the data
       set on disk $n$ and catalog number $m$.}
\dispt{SOURCE \qs '{\it cala}'\qs ' \CR}{to select calibrator sources
       one at a time.}
\dispt{DOCAL\qs 1 ; DOBAND 1 \CR}{to apply calibration.}
\dispt{BCHAN $N/8$ ; ECHAN=$7N/8$ ; NCHAV=ECHAN-BCHAN+1 \CR}{to
       average the central channels of each IF\@.}
\dispt{BPARM 11 , 2 \CR}{to examine phase.}
\dispt{BPARM 11 , 1 \CR}{to examine amplitude.}
\dispt{DOTV\qs 1 \CR}{to use the TV display.}
\dispt{GO \CR}{}
\dispe{You may need to limit which IFs are displayed or color the IFs
with {\us DO3COL\qs 1}.  The phases should be near zero and the
amplitudes should be nearly constant with baseline length.  A decrease
in the amplitude at the longest spacings indicates that the calibrator
is not strictly a point source.\todx{UVPLT}}

{\tt \Tndx{ANBPL}} converts baseline-based data before or after
calibration into antenna-based quantities.  In particular, the
calibrated weights are very sensitive to problems with amplitude
calibration.\Iodx{calibration}\Iodx{EVLA}
\dispt{DEFAULT\qs ANBPL \CR}{to select task and initialize all
            its parameters.}
\dispt{IND {\it m\/} ; GETN\qs {\it n\/} \CR}{to specify the
            multi-source data set.}
\dispt{STOKES\qs 'HALF' ; TIMERANG\qs 0 \CR}{to DISPLAY both
            parallel-hand polarizations.}
\dispt{FREQID\qs 1 \CR}{to select {\tt FQ} value to image.}
\dispt{BIF\qs 1 ; EIF 0 \CR}{to select all IFs.}
\dispt{BCHAN\qs {\it n\/} ; ECHAN\qs {\it m\/} \CR}{to combine a range
            of channels.}
\dispt{DOCALIB\qs 1 \CR}{to apply calibration.}
\dispt{GAINUSE\qs 0 \CR}{to use highest numbered {\tt CL} table.}
\dispt{FLAGVER\qs 1 \CR}{to edit data.}
\dispt{DOBAND\qs 3 ; BPVER\qs 1 \CR}{to correct bandpass with time
            smoothing using table 1.}
\dispt{BPARM\qs 2, 17 \CR}{to plot weight versus time.}
\dispt{NPLOTS\qs 3 ; DOTV\qs 1 \CR}{To plot 4 antennas per page on the
            TV.}
\dispt{DOCRT\qs 0 \CR}{to suppress printed versions of the
            antenna-based values.}
\dispt{INP \CR}{to review the inputs.}
\dispt{GO\qs \CR}{to run {\tt \tndx{ANBPL}\@.}}

     If the previous steps indicate serious problems and/or you are
seriously confused about what you have done and you want to start the
calibration again, you can use the procedure {\tt VLARESET} from the
{\tt RUN} file {\tt VLAPROCS} to reset the {\tt SN} and {\tt CL} tables.
\dispt{INP\qs VLARESET \CR}{to verify the data set to be reset.}
\dispt{\tndx{VLARESET}\qs \CR}{to reset {\tt SN} and {\tt CL} tables.}

\Subsections{Flagging RFI with {\tt RFLAG}}{rflag}

A very promising relatively new tool flags RFI on the assumption that
it is either quite variable in time or in frequency.  This task,
called {\tt \Tndx{RFLAG}}, computes the rms over short time intervals
in each spectral channel and IF individually and flags the interval
whenever the rms exceeds a user-controlled threshold.  Optionally, it
will also use a sliding median window of user-specified width over the
spectral channels to the real and imaginary parts of the visibility
separately.  Any channel deviating from the median in either part by
more than a user-specified amount will also be flagged.  If {\tt
DOPLOT}$> 0$, {\tt RFLAG} will make plots of normal and cumulative
histograms and of the mean and rms of the time and spectral
computations as a function of channel.  It will also make a flag table
only if requested ({\tt DOFLAG}$ > 0$).  These plots will suggest
threshold parameters and allow you to choose values to use.  A flag
table is made for any value of {\tt DOFLAG} if no plots are requested
({\tt DOPLOT}$\leq 0$).\iodx{flagging}\iodx{editing}

\vfill\eject
In detail, {\tt \Tndx{RFLAG}} is run using
\dispt{DEFAULT\qs \tndx{RFLAG} ; INP \CR}{to clear and review the adverbs.}
\dispt{INDI\qs {\it n\/}; GETN\qs {\it m\/} \CR}{to select the data
       set on disk $n$ and catalog number $m$.}
\dispt{SOURCES\qs '{\it source\_1}', '{\it source\_2}', $\ldots$
       \CR}{to select sources of similar flux level.}
\dispt{DOCALIB\qs 1 ; DOBAND\qs 1 \CR}{to apply continuum and bandpass
       calibration.}
\dispt{STOKES\qs 'FULL' \CR}{to examine all polarizations.}
\dispt{DOPLOT\qs 15 ; DOTV\qs 1 \CR}{to examine all kinds of plots on
       the TV.}
\dispt{FPARM\qs 3 , $x$ , -1, -1 \CR}{to examine spectral rms over 3
       time intervals each a bit longer than $x$ seconds.  The $-1$'s
       cause the program to use other adverbs for the cutoffs and to
       do a spectral solution as well as the time one.}
\dispt{FPARM(9) = 4.0 ; FPARM(10) = 4. \CR}{to set the cutoff values
       as 4 times the median rms plus deviation found in the spectral
       plots as a function of IF\@.  The default is 5.}
\dispt{FUNCTYPE\qs 'LG' \CR}{to plot the histograms on a log scale.}
\dispt{NBOXES\qs 1000 \CR}{to use 1000 boxes in the histograms.}
\dispt{INP\qs \CR}{to re-examine the inputs.  {\tt VPARM} will let you
       control aspects of the plotting.}
\dispt{GO}{to run the program.}
\dispe{This will produce plots and set cutoff levels in adverbs {\tt
NOISE} and {\tt SCUTOFF}\@.  An example of the spectral plot is shown
on the next page.  Another run, with {\tt DOPLOT = 0} will
apply these cutoffs and create a new flag table.  Note that the flux
cutoff levels may depend on the source flux, calling for different
levels for strong calibrators, weak calibrators, and very weak target
sources.  Different cutoff levels for {\tt STOKES='RRLL'} and {\tt
STOKES = 'RLLR'} may also be needed.  A strong, resolved target source
may require different levels for different {\tt UVRANGE}s.  In {\tt
31DEC18}, {\tt UVRANGE} is applied on a per spectral window basis so
you no longer have to break up wide bandwidth data into multiple files
when {\tt UVRANGE} is required.  {\tt RFLAG} is a new task, so
experiment a bit.  Note that, if you set {\tt DOFLAG=1}, the creation
of a new flag table will happen after the plots in the same execution
of {\tt RFLAG}\@.  If a channel is found bad at a time in any one
polarization, all polarizations are flagged.  If you have a significant
spectral line signal in your data, use {\tt DCHANSEL} to have the
affected channels ignored throughout {\tt RFLAG} or set {\tt FPARM(4)
= 0} to that the spectral function is not
performed.\Iodx{EVLA}\iodx{flagging}\iodx{editing}\Iodx{calibration}}

\begin{figure}
\centering
\resizebox{\hsize}{!}{\gbb{536,418}{rflag}}
\caption[{\tt RFLAG} display]{Plot produced by {\tt \tndx{RFLAG}}
showing the mean rms of short time intervals (dark line in middle of
green) and the rms of the rmses (green vertical bars) as a function of
spectral channel.  The recommended flagging level, returned in adverb
{\tt NOISE} is shown by the horizontal bars in each spectral window.
\iodx{flagging}\iodx{editing}\Iodx{calibration}\Iodx{EVLA}}
\label{fig:rflag}
\end{figure}

There are a lot of adverbs to {\tt RFLAG}\@.  {\tt FPARM(5)} allows
you to speed up the spectral part of the flagging by testing more than
just the central channel in the sliding median filter.  {\tt FPARM(6)}
allows you to expand all flags to adjacent channels.  {\tt FPARM(7)},
8, 11, and 12 control the extending of flags to additional channels,
baselines, or antennas if too large a fraction of channels, baselines,
or baselines to an antenna are flagged in the basic time and spectral
operations.  Similar adverbs also occur in the new task {\tt
\Tndx{REFLG}} whose job it is to compress the enormous flag tables
generated generated by {\tt RFLAG}\@.  {\tt REFLG} does not handle
flags generated by {\tt CLIP}, {\tt TVFLG}, and {\tt SPFLG} since they
vary with polarization.  {\tt REFLG} can extend a flag to all times if
too large a fraction of time is flagged for a given channel, baseline,
etc.  {\tt REFLG} may not reduce your flag table enough, although it
is inexpensive to run and so worth the effort.  The application of 10
million flag entries to a data set repetitively is rather expensive.
Copying the data, applying the flags once and for all, is the best
solution.  {\tt UVCOP} has been the traditional method to do this.
However, {\tt TYAPL} which needs to be run next and must make a new
copy of the data has been given the option of applying a large flag
table to avoid having to copy the data set twice.  Task {\tt
\Tndx{FGCNT}} lets you see how much of your data is flagged by any
particular flag table.\iodx{flagging}\iodx{editing}

\vfill\eject
\Subsections{Restart the calibration}{cleanup}

Having done a more careful job with your editing, it is now time to
discard with {\tt \tndx{EXTDEST}} the bandpass ({\tt BP}) tables and
all {\tt CL} tables after the one written by {\tt VLANT}\@.  Discard
all {\tt SN} tables, but keep the highest numbered flag ({\tt FG})
table.
\dispt{INEXT = 'BP' ; INVERS = -1 ; EXTDEST \CR}{to remove all
       bandpass tables.}
\dispt{INEXT = 'SN' ; INVERS = -1 ; EXTDEST \CR}{to remove all
       solution tables.}
\dispt{INEXT = 'CL' ; FOR INVERS=3:100; EXTDEST; END \CR}{to remove all
       calibration tables after {\tt VLANT}\@.}
\dispe{Be very careful with the flag tables to save the last one - use
{\tt IMHEADER} to show you how many there are.}
\Iodx{EVLA}\Iodx{calibration}

\Subsections{Calibration with the SysPower table}{tyapl}

Because of its wide dynamic range, the EVLA does not normalize its
output visibilities.  To calibrate gains it records the total power
when the switched noise tube is on and when it is off.  These data,
taken in synchronism with the visibilities, are recorded in the
\Indx{SysPower} table of the ASDM\@.  The {\tt \tndx{OBIT}} program
{\tt BDFIn}, available to \AIPS\ users in the verb {\tt
\tndx{BDF2AIPS}}, reads this table and creates an \AIPS\ {\tt SY}
table.  The columns of this table contain {\tt POWER DIF} ($Gain
\times (P_{on} - P_{off}$), {\tt POWER SUM} ($Gain \times (P_{on} +
P_{off}$)), and {\tt POST GAIN} ($Gain$) columns for right and left
polarizations with values for each IF\@.

This table is accessible to \AIPS\ users with a number of tasks.
Begin with task {\tt \Tndx{PRTSY}} to view the table statistically
over time on a per IF, per antenna basis or to view scan or source
median averages of one or more of the {\tt SY} table parameters.  Then
to examine its contents in more detail in various ways, use {\tt
\tndx{SNPLT}} with {\tt OPTYPE}s {\tt 'PDIF'}, {\tt 'PSUM'}, {\tt
'PGN'}, {\tt 'PON'}, {\tt 'POFF'}, {\tt 'PSYS'}, {\tt 'PDGN'}, or {\tt
'PSGN'}\@.  You could use {\tt OPTYPE = 'MULT'} to examine more than
one of these at one time, comparing any oddities in \eg\ Psum and
Pdif.  Note that {\tt PSYS} is especially interesting since
$T_{cal}*P_{sum}/(2*P_{dif}) = T_{sys}$, the system temperature.  It
should reflect changes in elevation and strength of the observed
source, but should be immune to adjustments to the gain of the
telescope.  It determines data weights in {\tt TYAPL} while
$\sqrt{P_{dif_i} P_{dif_j}}$ divides into the visibilities.  You may
use {\tt \tndx{EDITA}} (\Sec{edita}) to edit your $uv$ data on the
basis of the contents of the {\tt SY} table.  Editing may be based on
Psum, Pdif, Pgain, Tsys, and on the differences between these
parameters and a running median of these parameters.  One may also
edit the {\tt SY} table itself with {\tt \tndx{SNEDT}}; the same
parameters are available.  {\tt LISTR} can even display the {\tt SY}
table Psum, Pdif,  system temperatures, and gain factors with {\tt
OPTYPE 'GAIN'} and {\tt DPARM(1)} set to 17, 18, 15, or 16,
respectively.

More importantly, the {\tt SY} table can be used to do an initial
calibration of the visibility data.  Use the display programs to
decide if your {\tt SY} table is fine as is or needs editing.  The
following is somewhat controversial; there are users who believe that
the {\tt SY} table is never adequately reliable for all antennas
because of RFI and other issues.  Since the value of Tcal are not
perfectly known, nor are the antenna efficiencies, the corrections
made by {\tt TYAPL} can never be perfect.  Nonetheless, in many cases,
they are a great help.  {\tt CALIB} will still be needed to finish the
amplitude calibration.

The tasks {\tt \tndx{TYSMO}} and {\tt \tndx{TYAPL}} may be used with
\Indx{EVLA} data having an {\tt SY} table.  {\tt TYSMO} flags {\tt SY}
samples on the basis of Pdif, Psum, Pgain, and Tsys and then smooths
Psum, Pdif and Pgain to replace the flagged samples and/or reduce the
noise.  You may want to do this to remove outlying bad points and to
reduce the jitter in these measurements.  {\tt TYSMO} even applies a
flag table to the {\tt SY} before its clipping and smoothing
operations.  Be sure to plot the results to make sure that the task
did what you wanted.  Then use {\tt TYAPL} to remove a previously
applied {\tt SY} table (if any) and to apply the {\tt SY} table you
have prepared.  The result should be data scaled nearly correctly in
Jy and weights in $1/{\rm Jy}^2$ in all IFs.  The {\tt CUTOFF} option
allows you to use obviously good values from the {\tt SY} table while
passing the data from antennas with poor {\tt SY} values along
unchanged.  If the {\tt SY} values from some polarizations or IFs are
bad due to RFI while others are good, you may copy the good values to
replace the bad values using task {\tt \tndx{TYCOP}}\@.  The
wide-band, 3-bit mode of the EVLA has serious non-linearities in the
Pdif measurements which are not yet fully understood.  For such data,
use {\tt OPTYPE='PGN'} to apply only the post-detection gains to the
data.  This will remove abrupt jumps due to changes in those gains
(which will be more common with 3-bit data).  Note that {\tt TYSMO}
and {\tt TYAPL} also require a table of the Tcal values which {\tt
OBIT} provides in an \AIPS\ {\tt CD} table.  Amplitude calibrations
are not applied to EVLA data weights until they have been made
meaningful by {\tt TYAPL} or {\tt REWAY}\@.  Set {\tt FLAGVER} in {\tt
TYAPL} if you want to apply your flag table once and for all.  {\tt
TYAPL} can handle flag tables much larger than those that many other
tasks can handle.  Note that {\tt TYAPL} must copy the data set since
it makes corrections to the antennas on very short intervals (not
suitable to a {\tt CL} table).

If you have observations of the Sun, do not use {\tt \tndx{TYAPL}};
use {\tt \tndx{SYSOL}} instead.  This task does the usual {\tt TYAPL}
operation on non-Solar scans, but, for Solar scans, it must determine
the average gain and weight factors on those antennas having solar
Tcals and then apply those averages to all antennas not equipped with
Solar Tcals.  Use a well-edited {\tt SY} table if possible.  This
Solar capability was made available in {\tt OBIT} (version 567) and
\AIPS\ in mid-June 2017.  The revised format of the {\tt SY} table
includes a column to tell tasks whether to use normal or solar Tcals
for that row of the table.\iodx{Solar data}

\Subsections{Calibrating polarization}{polcal}

You may skip this section unless you have cross-hand
\indx{polarization} data and wish to make use of them.  Although there
have been major improvements in \AIPS\ polarization routines, they
still do not correct parallel hand visibilities for polarization
leakage.  Thus you need to calibrate polarization only if you wish to
make images of target source Q and U Stokes parameters.  Note that we
discuss here the calibration of both linear and circular polarization
with tasks {\tt TECOR}, {\tt RLDLY}, and {\tt PCAL}\@.  At the phase
difference stage, however, the process for linear feeds is different
than for circular feeds.

The \Indx{calibration} of wide-band visibility data sensitive to
\Indx{polarization} involves three distinct operations: (1)
determining and correcting the offset in delay between the two
parallel-hand polarizations, (2) determining and correcting the data
for the effects of imperfect telescope feeds, and (3) removing any
systematic phase offsets between the two systems of orthogonal
polarization.  These three components of polarization calibration will
be considered separately.\Iodx{EVLA}

Faraday rotation in the Earth's ionosphere can add a phase difference
between the two parallel-hand polarizations which is both time and
direction dependent.  These are particularly important at lower
frequencies, but may affect higher frequency-data as well.  One way to
remove at least some of the ionospheric phase offsets is by applying a
global ionospheric model derived from GPS measurements.  The \AIPS\
task {\tt \Tndx{TECOR}} processes such ionospheric models that are in
standard format known as the IONEX format.  These models are available
from the Crustal Dynamics Data Information System (CDDIS) archive.
There is a procedure which is part of {\tt VLAPROCS}, called
{\tt \Tndx{VLATECR}} that automatically downloads the needed IONEX
files from CDDIS and runs {\tt TECOR}\@.  It will examine the header
and the {\tt NX} table and figure out which dates need to be
downloaded, so the observation date in the header must be correct and
an {\tt NX} table must exist.  See {\tt EXPLAIN VLATECR} for other
requirements.
\dispt{RUN \tndx{VLAPROCS} \CR}{to acquire the procedures; this should
           be done only once since they will be remembered.}
\dispt{INDISK\qs{\it n\/} ; GETN\qs {\it ctn\/} \CR}{to specify the
           input file.}
\dispt{INP\qs VLATECR \CR}{to review the inputs.}
\dispt{APARM\qs 1,0 \CR}{to apply dispersive delay corrections which
           may be useful for wide bandwidths.}
\dispt{\Tndx{VLATECR} \CR}{to run the procedure.}
\dispe{{\tt TECOR} applies its corrections to the highest numbered
{\tt CL} table and writes a new one.}

Frequently, the delay difference between right- and left-hand, or X
and Y, polarizations must be determined even if {\tt FRING} was not
required for the parallel-hand data.  Use {\tt POSSM} to plot the RL
and LR (or XY and YX) spectra to see of there are significant slopes
in phase.  If so, use a calibration source with significant
polarization, although the EVLA D terms are often large enough to
provide a usable signal in the absence of a real polarized signal.
Note that 3C286 is significantly polarized and is likely to be the
best source to use for this purpose.  Then
\dispt{TASK\qs \Tndx{RLDLY} ; INP \CR}{to look at the necessary
       inputs.}
\dispt{BCHAN\qs $c_1$ ; ECHAN\qs $c_2$ \CR}{to select channels free of
       edge effects.}
\dispt{DOCAL\qs 1 ;\qs GAINUSE\qs 0 \CR}{to apply the {\tt FRING}
       results and all other current calibrations.}
\dispt{REFANT\qs $n_r$ \CR}{to select a reference antenna - only
       baselines to this antenna are used so select carefully.
       Alternatively, {\tt REFANT 0} will loop over all possible (not
       necessarily good) reference antennas, averaging the result.}
\dispt{DOIFS\qs $j$ \CR}{to set the adverb to the value of {\tt
       APARM(5)} used in {\tt FRING} (\Sec{FRINGit}.  The IFs are
       done independently ($\leq 0$), all together ($= 1$), in halves
       ($= 2$), or more generally in $N$ groups ($= N$).  A value of
       $-1$ causes {\tt RLDLY} to use {\tt BPARM} to define the IF
       groupings, as in {\tt FRING}\@.}
\dispf{TIMERANG\qs {\it db} , {\it hb} , {\it mb} , {\it sb} , {\it
       de},  {\it he} , {\it me} , {\it se} \CR}{to specify the
       beginning day, hour, minute, and second and ending day, hour,
       minute, and second (wrt {\tt REFDATE}) of the data to be
       included.  Use an interval not unlike the one you used in {\tt
       FRING}.}
\dispt{INP \CR}{to check the inputs.}
\dispt{GO \CR}{to produce a new {\tt SN} table with a suitable left
       polarization delay.}
\dispe{Note that {\tt RLDLY} now always creates an {\tt SN} table and
may be run with multiple calibrator scans.  If there is only one
calibrator scan and {\tt APARM(2)}$\leq 0$, it will also copy the {\tt
CL} table which was applied to the input data through {\tt GAINUSE} to
a new {\tt CL} table applying the correction to the L polarization
delay.  For all other cases, you must apply the added L polarization
delays with {\tt CLCAL}\@.
\Iodx{EVLA}\Iodx{calibration}\Iodx{polarization}}

To begin with, it is probably better to determine a continuum solution
for source polarization and antenna D terms before doing the lower
signal-to-noise spectral solutions.  To find an average solution for
each IF:
\dispt{DEFAULT\qs \Tndx{PCAL} ; INP}{to reset all adverbs and choose
       the task.}
\dispt{INDI\qs {\it n\/}; GETN\qs {\it m\/} \CR}{to select the data
       set on disk $n$ and catalog number $m$.}
\dispt{DOCAL\qs 1 ; DOBAND\qs 3 \CR}{to apply the delay, complex gain,
       and bandpass calibration.}
\dispt{CALSOUR\qs '{\it pol\_cal1\/}', '{\it pol\_cal2\/}'  \CR}{to
       select the polarization calibrator(s) by whatever form of their
       names appears in your {\tt LISTR} output.  These sources must
       have I polarization fluxes in the source table.}
\dispf{ICHANSEL\qs $c11, c12, 1, if1, c21, c22, 1, if2, c31, c32, 1,
       if3, \ldots\/$ \CR}{to select the range(s) of channels which
       are reliable for averaging in each IF\@.  These probably should
       be the same values that you used in {\tt BPASS}\@.}
\dispt{DOMODEL\qs -1; SPECTRAL\qs -1 \CR}{to solve for source
       polarization in a continuum manner.}
\dispt{PRTLEV\qs 1 \CR}{to see the answers and uncertainties on an
       antenna and IF basis.}
\dispt{CPARM\qs 0,1 \CR}{to update the source table with the calibrator
       source Q and U found.}
\dispt{INP  \CR}{to review the inputs.}
\dispt{GO \CR}{to find the antenna leakage terms and the source Q and
       U values on an IF-dependent basis.}
\dispe{{\tt PCAL} will write the antenna leakage terms in the antenna
file and the source Q and U terms in the source table (if {\tt
CPARM(2)} $> 0$).  {\tt DOMODEL} may be set to true only if the model
has ${\tt Q} = {\tt U} = 0$ since {\tt PCAL} cannot solve for the
right minus left phase difference.  If {\tt SOLINT=0}, {\tt PCAL} will
break up a single scan into multiple intervals, attempting to get a
solution even without a wide range of parallactic angles.}

Use {\tt \tndx{TASAV}} to copy all your table files to a dummy \uv\
data set, saving in particular the {\tt CL} table with the results of
the amplitude and phase \Indx{calibration}.  This step is not
essential, but it reduces the magnitude of the disaster if the the
next step fails in some way.  (Note - this may be a good idea at
several stages of the calibration process!)
\dispt{TASK\qs 'TASAV' \CR}{}
\dispt{CLRO\qs \CR}{Use default output file file name.}
\dispt{INP \CR}{to review the (few) inputs.}
\dispt{GO\qs \CR}{to run the program.}
\dispe{The task {\tt TACOP} may be used to recover any tables that get
trashed during later steps.  Note that the next step writes a new {\tt
CL} table, but changes the {\tt AN} and {\tt SU} tables in place.}

Having prepared a continuum solution for Q and U, you must also
correct it for the difference in phase between R and L  (or X and Y)
polarizations which normally varies considerably between IFs.  For
circularly polarized feeds only, the task {\tt RLDIF} will correct the
antenna, source, and calibration tables for this difference using
observations of a source with known ratio of Q to U\@.  3C286 is by
far the best calibrator for this purpose.
\dispt{DEFAULT\qs \Tndx{RLDIF} ; INP}{to reset all adverbs and choose
       the task.}
\dispt{INDI\qs {\it n\/}; GETN\qs {\it m\/} \CR}{to select the data
       set on disk $n$ and catalog number $m$.}
\dispt{DOCAL\qs 1 ; DOBAND\qs 3 \CR}{to apply the delay, complex gain,
       and bandpass calibration.}
\dispt{DOPOL\qs 1 \CR}{to apply the polarization calibration.}
\dispt{BCHAN\qs $c_1$; ECHAN\qs $c_2$ \CR}{to average data from
       channels $c_1$ through $c_2$ only.}
\dispt{SOURC\qs '{\it pol\_cal1\/}', '{\it pol\_cal2\/}'  \CR}{to
       select the polarization calibrator(s) by whatever form of their
       names appears in your {\tt LISTR} output.  These sources must
       have known polarization angles.}
\dispt{SPECTRAL\qs 0 \CR}{to do the correction in continuum mode.}
\dispt{DOAPPLY\qs 1 \CR}{to apply the solutions to a {\tt CL} table
       (making a new modified one) and to the {\tt AN} and {SU}
       tables, updating them in place.}
\dispt{DOPRINT\qs 0 \CR}{to omit all the possible printing.}
\dispt{INP\qs \CR}{to review the inputs.}
\dispt{GO \CR}{to determine and apply the corrections.}

The EVLA polarizers appear to be very stable in time, but to have
significant variation with frequency.  See
Figure~\ref{fig:PCALspectral}.  Serious polarimetry with the EVLA will
{\it require} solving for the antenna \indx{polarization} leakage as a
function of frequency.  To compute a spectral solution, assuming you
already did the process in the preceding paragraph:
\todx{PCAL}\Iodx{EVLA}
\dispt{TGET\qs PCAL \CR}{to retrieve the {\tt PCAL} adverbs.}
\dispt{SPECTRAL\qs 1 \CR}{to do the channel-dependent mode.}
\dispt{DOMODEL\qs 0 \CR}{to solve for Q and U as a function of
       frequency.  Because {\tt PCAL} does not solve for a right-left
       phase difference and that difference is a function of spectral
       channel, you must solve for a source polarization.}
\dispt{SPECPARM\qs 0 \CR}{to determine the calibration source I, Q,
       and U spectral indices from fluxes in the source table.  If you
       use {\tt PMODEL} you must provide spectral indices for the
       model that apply in the frequency range of the data (curvature
       cannot be specified).\iodx{spectral index}}
\dispt{INTPARM\qs $p_1, p_2, p_3$\qs \CR}{to smooth the data after
       all calibration has been done while honoring {\tt ICHANSEL}.}
\dispt{INP \CR}{to review the inputs, the task will take a while to
       run.}
\dispt{GO\qs \CR}{to run the task writing a {\tt PD} table of spectral
       leakages (``D terms'') and, if {\tt DOMODEL} $\leq 0$, a {\tt
       CP} table of source Q and U spectra.}
\dispe{If the combination of flagging, {\tt ICHANSEL}, and {\tt
INTPARM} results in no solutions for some channels, the solutions from
nearby channels will be interpolated or extrapolated so that all
channels get solutions.  If the calibration source is known to have no
polarization, then you may set {\tt DOMODEL 1} and set {\tt PMODEL =
{\it flux} , 0}}

\begin{figure}
\centering
%\resizebox{\hsize}{!}{\gname{PDpcal}}
\resizebox{\hsize}{!}{\gbb{703,462}{PDpcal}}
\caption[Example antenna D term spectrum]{Example spectrum showing D
term solutions for one antenna in right and left polarizations
covering about 2 GHz at C band}
\label{fig:PCALspectral}
\end{figure}

After running {\tt \Tndx{PCAL}} in spectral mode, you may examine the
resulting {\tt PD} (\indx{polarization} D terms) table with {\tt
\tndx{POSSM}} using {\tt APARM(8)=6} and {\tt \tndx{BPLOT}} using {\tt
INEXT = 'PD'}\@.  If a {\tt CP} table (calibrator polarization) was
written, you may also use {\tt POSSM} with {\tt APARM(8) = 7} or 8 and
{\tt BPLOT} with {\tt INEXT = 'CP'} to examine the results.
\Iodx{calibration}

You are almost, but not quite done.  The combination of {\tt CALIB}
and {\tt BPASS} has produced a good calibration for everything except
the phase difference between right and left polarizations.  This is
now a function of spectral channel and needs to be corrected.  The
task {\tt \Tndx{RLDIF}} has been modified to determine a continuum or
spectral right minus left phase difference and to modify the {\tt CL}
or {\tt BP} table, respectively, to apply a phase change to the left
polarization on an IF or channel, respectively, basis.  Thus
\Iodx{EVLA}
\dispt{DEFAULT\qs \Tndx{RLDIF} ; INP}{to reset all adverbs and choose
       the task.}
\dispt{INDI\qs {\it n\/}; GETN\qs {\it m\/} \CR}{to select the data
       set on disk $n$ and catalog number $m$.}
\dispt{DOCAL\qs 1 ; DOBAND\qs 3 \CR}{to apply the delay, complex gain,
       and bandpass calibration.}
\dispt{DOPOL\qs 1 \CR}{to apply the polarization calibration, spectral
       if present.}
\dispt{INTPARM\qs $p_1, p_2, p_3$\qs \CR}{to smooth the data after
       all calibration has been done.}
\dispt{BCHAN\qs $c_1$; ECHAN\qs $c_2$ \CR}{to use solutions from
       channels $c_1$ through $c_2$ only, extrapolating solutions to
       channels outside this range.\Iodx{EVLA}}
\dispt{SOURC\qs '{\it pol\_cal1\/}', '{\it pol\_cal2\/}'  \CR}{to
       select the polarization calibrator(s) by whatever form of their
       names appears in your {\tt LISTR} output.  These sources must
       have known polarization angles.}
\dispt{POLANGLE\qs $p_1, p_2$ \CR}{to provide the task with the source
       polarization angle(s) in degrees in source number order.  The
       phase correction will be twice this value minus the observed
       RL phase.  Do not provide values for 3C286, 3C147, 3C48, and
       3C138.  These are known to {\tt RLDIF} including rotation
       measures and other spectral dependence.}
\dispt{SPECTRAL\qs 1 \CR}{to do the correction in spectral mode.}
\dispt{DOAPPLY\qs 1 \CR}{to apply the solutions to a {\tt BP} table
       (making a new modified one) and to the {\tt PD} and {CP}
       tables.}
\dispt{DOPRINT\qs -1 ; OUTPRI\qs '{\it file\_name}' \CR}{to write the
       phase corrections applied to a text file suitable for plotting
       by {\tt PLOTR}.}
\dispt{INP\qs \CR}{to review the inputs.}
\dispt{GO \CR}{to determine and apply the corrections.}

Note that {\tt \Tndx{RLDIF}} applies to circularly polarized feeds
only and requires you to know the linear polarization angle of at
least one calibration source.  The methods for calibrating
linearly-polarized feeds, which are only at P- and 4-bands on the VLA,
are still being developed.  It appears that one requires one or more
calibration sources known to have polarization in Stokes U and to have
no emission in Stokes V (circularly-polarized emission.  If you have
such data, then the new task {\tt \Tndx{XYDIF}} performs the functions
of {\tt RLDIF} for linear feeds.  3C345 is such a calibration source,
but at P-band its rotation measure makes the U polarization very weak
at some frequencies while it is strong enough at
others.\Iodx{calibration}

\Subsections{Additional calibrations for EVLA data}{reway}

A task new to {\tt 31DEC13} will help calibrate data sets which
contain a strong, \eg\ maser, line in one or a few channels with
interesting but weaker signals in other channels and/or the continuum.
After applying the best standard calibration described above, you can
split out the maser channel(s) with {\tt UVCOP}\@.  Make images and
self-calibrate following the processes described in \Rchap{image}
making an {\tt SN} table containing the full calibration needed to
apply to the maser.  Then use task {\tt \Tndx{SNP2D}} to convert this
into delays and phases to apply to the full, multi-IF data set.  This
will work so long as the residual phases found in the self-cal are
small and is actually required to do the best calibration possible
over very wide bandwidths.

For the historic VLA, {\tt FILLM} was able to compute meaningful data
weights which were then improved with the amplitude calibration.  For
the EVLA, however, the data weights begin as solely the integration
time.  For 8-bit data, {\tt TYAPL} (\Sec{tyapl}) is able to convert
these into meaningful weights.  However, if {\tt TYAPL} was used only
with {\tt OPTYPE = 'PGN'}, or not used at all, then it behooves you to
rectify the situation.  There is a task called {\tt \Tndx{REWAY}}
which computes a robust rms over spectral channels within each IF and
polarization.  It can simply base the weights on these on a
record-by-record, baseline-by-baseline basis.  Alternatively, it can
use a scrolling buffer in time so that the robust rms includes data
for a user-specified number of records surrounding the current one.  A
third choice is to average the single-time rmses over a time range and
then convert them to antenna-based rmses.  In all three modes, the
task can then smooth the rmses over time applying clipping based on
user adverbs and the mean and variance found in the rmses.  A flag
table (extension file) may be written to the input data file removing
those data found to have rmses that are either too high or too low.
For these weights to be meaningful, the bandpass and spectral
polarization calibration must be applied and it helps to omit any RFI
or other real spectral-line signal channels from the rms computation.
For the weights to be correctly calibrated, all amplitude calibration
must also be applied.  For these reasons, {\tt REWAY} might well be
used instead of {\tt SPLIT} --- when {\tt TYAPL} was not used ---
running it one source at a time.  Thus,
\dispt{DEFAULT\qs \tndx{REWAY} ; INP}{to reset all adverbs and choose
       the task.}
\dispt{INDI\qs {\it n\/}; GETN\qs {\it m\/} \CR}{to select the data
       set on disk $n$ and catalog number $m$.}
\dispt{DOCAL\qs 1 ; DOBAND\qs 3 \CR}{to apply the delay, complex gain,
       and bandpass calibration.}
\dispt{SOURCE\qs '$target_1$' , ' ' \CR}{to do one target source.}
\dispt{APARM\qs 11, 30, 12, 0, 10, 4 \CR}{to use a rolling buffer of 11
       times separated by no more than 30 seconds and then smoothed
       further with a Gaussian 12 seconds in FWHM\@.  Data are flagged
       if the rms is more than 4 times the variance away from the mean
       averaged over all baselines, IFs, and polarizations.  Flagging
       on the variance of the rms from the mean on a baseline basis is
       essentially turned off by the 10.}
\dispt{GO \CR}{to write out a calibrated, weighted data set for the
       first target source.}
\dispe{Then, when that finishes}
\dispt{SOURCE\qs '$target_2$' , ' ' ; GO \CR}{to do another target
       source.}
\dispe{It is not clear that this algorithm is optimal, but it
certainly should be better than using all weights 1.0 throughout.  It
will be interesting to compare data weights found with {\tt TYAPL} to
those found with {\tt REWAY}\@.  The task now displays statistical
information to assist the user in determining which weights are
excessively high or low.}

With the historic VLA, spectral-line observations were normally done
with frequencies which changed with time in order to track a chosen
source velocity at a specific spectral channel.  The narrow bandwidths
of the historic VLA kept this operation out of non-linear regimes.
The EVLA, however, has very wide bandwidths, making the appropriate
frequency changes different for each spectral window.  For spectral
imaging, however, this frequency shift must be done, at least for
narrow bandwidths, long observing sessions, or combinations of
multiple observing sessions.  It is probably best to separate out the
spectral window (IF) containing the desired spectral line signals and
do the following on that window alone.

At present, the \Indx{EVLA} observing setup allows you to select the
initial frequency of observation based on a desired LSRK velocity in
the central channel.  From there, however, the observations are
conducted at a fixed frequency.  Furthermore, the information about
rest frequencies, source velocities, and even more fundamental
parameters such as reference frame (LSRK or barycentric) and type of
velocity (radio or optical) are lost.  {\tt SETJY} allows you to
correct this. First use {\tt SETJY} to set the desired rest
frequencies (note that they are allowed to be a function of IF) and
the {\tt VELTYP} and {\tt VELDEF}\@.  Then use {\tt OPTYEP='VCAL'}
over all sources and IFs.  This will compute the velocities at which
you observed for the first time you observed each source and enter the
values in the source table.

Having done this, the task {\tt \tndx{CVEL}} may be used to shift
the visibility data to correct for the rotation of the Earth about its
axis as well as the motion of the Earth about the Solar System
barycenter and the motion of the barycenter with respect to the Local
kinetic Standard of Rest.  {\tt CVEL} works on multi-source as well as
single-source data sets.  It applies any flagging and bandpass
calibration to the data before shifting the velocity (which it does by
a carefully correct Fourier transform method).  Note, the use of
Fourier-transforms means that one {\it must not} use {\tt CVEL} on
data with channel separations comparable to the widths of some of the
spectral features.  Furthermore, narrow EVLA bands apparently have
sharp cutoffs at the edges which cause any continuum signal to
generate sine waves in amplitude after the FFT\@.  Therefore, {\tt
UVLSF} {\it must be run before} {\tt CVEL}\@.  The velocity
information used by {\tt CVEL} must be correct.  Use {\tt LISTR} and
{\tt SETJY} to insure this before using {\tt SPLIT} and {\tt CVEL}\@.
A special version of {\tt CVEL} has been written to correct not only
for the Earth's motion but also for planetary motion to observe a line
at rest with respect to a planet; see {\tt \tndx{PCVEL}}\@.

\Subsections{Making images from multi-source data with {\tt IMAGR}}{calimagr}

     {\tt IMAGR} can be used to make images from multi-source data
files.  It is probably a good idea to make a couple of quick images to
make sure that the calibration is okay.  An example set of inputs to
{\tt \tndx{IMAGR}} for a continuum image is:\Iodx{EVLA}
\dispt{DEFAULT\qs IMAGR ; INP \CR}{to select task and initialize
            all its parameters.  This selects the usual convolution
            and weighting functions among other things.}
\dispt{IND {\it m\/} ; GETN\qs {\it n\/} \CR}{to specify the
            multi-source data set.}
\dispt{SOURCE\qs '{\it sou1\/}' , ' ' \CR}{to choose one source to
            image.}
\dispt{STOKES\qs 'I' ; TIMERANG\qs 0 \CR}{to image total intensity
            from all times.}
\dispt{FREQID\qs 1 \CR}{to select {\tt FQ} value to image.}
\dispt{BIF\qs 1 ; EIF 0 \CR}{to image all IFs --- multi-channel mode
            images only one \hbox{IF}.}
\dispt{BCHAN\qs {\it n\/} ; ECHAN\qs {\it m\/} \CR}{to combine a range
            of channels.}
\dispt{NCHAV\qs $(m - n + 1)$ \CR}{to include all spectral channels
            plus IFs {\it bif\/} through {\it eif\/} in each image.
            Note that each channel and IF included in the ``average''
            image is handled individually at its correct frequency.}
\dispt{DOCALIB\qs 1 \CR}{to apply calibration.  Use {\us DOCAL 100
            \CR} if the weights should {\it not} be calibrated.}
\dispt{GAINUSE\qs 0 \CR}{to use highest numbered {\tt CL} table.}
\dispt{FLAGVER\qs 1 \CR}{to edit data.}
\dispt{DOPOL\qs TRUE \CR}{to correct for feed polarization if you
            solved for it.}
\dispt{DOBAND\qs 3 \CR}{to correct bandpass with time smoothing.}
\dispt{BPVER\qs 1 \CR}{to select {\tt BP} table to apply.}
\dispt{OUTNAME\qs '{\it sou1\/}' \CR}{to set the output file name to
             the source name.}
\dispt{OUTDISK\qs 0 \CR}{to use any output disk with enough space.}
\dispt{IMSIZE\qs 512 512  \CR}{to set the size in cells of image.}
\dispt{CELLSIZE\qs 0.25 , 0.25 \CR}{to set the size of each image cell
             in arc-seconds.}
\dispt{RASHIFT\qs 0 ; DECSHIFT\qs 0 \CR}{to (not) shift image center.}
\dispt{NFIELD\qs 1 ; NGAUSS\qs 0\CR}{to make only one image at high
             resolution.}
\dispt{UVWTFN\qs ' ' \CR}{to use uniform weighting.}
\dispt{ZEROSP\qs 0 \CR}{to introduce no zero-spacing flux.}
\dispt{NITER\qs 0 \CR}{to do no Cleaning.}
\dispt{INP \CR}{to review the inputs.}
\dispt{GO\qs \CR}{to run {\tt \tndx{IMAGR}} when the inputs are set
             correctly.}
\dispe{Note that some of the above values are set by {\tt DEFAULT} but
we wish to emphasize the parameters here.  The {\tt CELLSIZE} and {\tt
IMSIZE} depend on your data; use {\tt SETFC} to make recommendations.}

\vfill\eject
\Subsections{Back up your data with {\tt FITTP} or {\tt FITAB}}{calfittp}

After you have edited your target sources with {\tt RFLAG} and checked
the calibration with {\tt IMAGR}, your calibration is complete --- at
least until you do self-calibration.  It is probably time to back up
your data.  You may wish to apply the final large flag table from {\tt
RFLAG} before backing up the data. Use {\tt \tndx{UVCOP}} to apply
especially large flag tables, but be sure to copy all of your data
otherwise.  Then\Iodx{EVLA}
\dispt{DEFAULT FITTP ; INP \CR}{to initialize the adverbs.}
\dispt{IND {\it m\/} ; GETN\qs {\it n\/} \CR}{to specify the
          multi-source data set.}
\dispt{DATAOUT\qs '{\it logical:filename} \CR}{to specify the output
          disk file name in the usual manner; see \Sec{fitsdisk}}
\dispt{INP \CR}{to review the inputs.}
\dispt{GO \CR}{to write a ``uvfits'' file on disk.}
\dispe{Task {\tt FITAB} writes a variant of this format in which the
visibility data are also written in a FITS table format.  The {\tt
FITAB} form is preferred esthetically by some, but most other software
packages cannot read it properly.  The FITS format used by \AIPS\ is
extensively documented in \AIPS\ Memo 117.\footnote{Eric W. Greisen,
2012, ``\AIPS\ FITS File Format,'' \AIPS\ Memo 117, NRAO, with
numerous updates}}

\Subsections{Creating single-source data files with {\tt SPLIT}}{split}

      When you are happy with the \Iodx{calibration} and editing
represented by the current set of calibration and flag tables, you can
convert the multi-source file into single-source files, applying your
calibration and editing tables.  Remember that only one {\tt FREQID}
can be {\tt \tndx{SPLIT}} at a time.
\dispt{TASK\qs 'SPLIT' \CR}{}
\dispt{SOURCE\qs '{\it sou1\/}' , '{\it sou2\/}' , $\ldots$ \CR}{to
             select sources, '\ ' means all.}
\dispt{TIMERANG\qs 0 \CR}{to keep all times.}
\dispt{BIF\qs 1 ; EIF\qs 2 \CR}{to keep both IFs}
\dispt{FREQID\qs 1 \CR}{to set the one {\tt FQ} value to use.}
\dispt{DOCALIB\qs 1 \CR}{to apply calibration to the data and the
             weights.}
\dispt{GAINUSE\qs 0 \CR}{to use the highest numbered {\tt CL} table.}
\dispt{DOPOL\qs TRUE \CR}{to correct for feed polarization, but only
       if you solved for the polarization in \Sec{polcal}.}
\dispt{DOBAND\qs 3 \CR}{to correct bandpass with time smoothing.}
\dispt{BPVER\qs 1 \CR}{to select {\tt BP} table to apply.}
\dispt{STOKES\qs ' ' \CR}{to write the input Stokes type.}
\dispt{DOUVCOMP\qs FALSE \CR}{to write visibilities in uncompressed
            format.}
\dispt{APARM\qs 0 \CR}{to avoid channel averaging and autocorrelation
            data.}
\dispt{INP \CR}{to review the inputs.}
\dispt{GO\qs \CR}{to run the program when inputs set correctly.}
\dispe{The files produced by this process should be completely
calibrated and edited and ready to be imaged or further processed as
described in later chapters.  Note that one may wish to defer the {\tt
DOPOL 1} part of this until after self-calibration of the
parallel-hand visibilities.\todx{SPLIT}\Iodx{calibration}}

\vfill\eject
\sects{Concluding remarks, early science}

\AIPS\ itself, and particularly this appendix, do not begin to cover
all of the issues that will arise with \Indx{EVLA} data.  The
increased sensitivity of the EVLA will mean that imaging will no
longer be able to ignore effects that are difficult to correct such as
pointing errors, beam squint, variable antenna polarization across the
field, leakage of polarized signal into the parallel-hand
visibilities, etc.,~etc.  These are research topics which may have
solutions in \AIPS\ or other software packages such as {\tt OBIT} and
{\tt CASA} eventually.

\sects{Additional recipes}

% chapter   *************************************************
\recipe{Columbian fresh banana cake with sea foam frosting}

\bre
\Item {Open an 18.5 oz {\bf yellow cake mix} into a large mixing
   bowl; do not use a mix that contains pudding or requires oil.
   Combine with 1/8 teaspoon {\bf baking soda}.}
\Item {Stir 3/4 cup {\bf Coca-Cola} briskly until foaming stops.
   Add to cake mix and blend until just moistened.  Then beat at high
   speed for 3 minutes, scraping the bowl often.}
\Item {Combine 2 teaspoons {\bf lemon juice} with 1 cup mashed {\bf
   bananas} and then add to batter.  Add 1/3 cup finely chopped {\bf
   nuts} and beat for 1 minute at medium speed.}
\Item {Turn the batter into a well greased, lightly floured 9x13
   baking dish. Bake in a preheated \dgg{350} oven for about 40
   minutes or until the cake tests done.   Cool on a rack for 15
   minutes, remove cake from pan and turn right side up on a rack to
   finish cooling.}
\Item {In the top of a double boiler, combine 2 large {\bf egg
   whites}, 1 1/2 cups packed {\bf light brown sugar}, 1/8 teaspoon
   {\bf cream of tartar} (or 1 tablespoon corn syrup), and 1/3 cup
   {\bf Coca-Cola}.  Beat at high speed for 1 minute with an electric
   mixer.}
\Item {Place over boiling water --- the water should not touch the
   bottom of the top half of the double boiler.  Beat on high speed
   for about 7 minutes until the frosting forms peaks when the mixer
   is raised. Remove from boiling water.  Empty into a large bowl.}
\Item {Add 1 teaspoon {\bf vanilla extract} and continue beating on
   high speed until thick  enough to spread, about 2 minutes. Spread
   on the sides and top of the cold banana cake.}
\ere

% chapter *************************************************
\recipe{Chewy banana split dessert}

\bre
\Item {Prepare and bake one package (19.8 0z) chewy fudge (or
      other favorite) {\bf brownie mix}.  Allow to cool thoroughly,
      four hours or more.}
\Item {Peel 2 large ripe {\bf bananas} and place very thin
      slices on top of brownie.}
\Item {Cover bananas evenly with one 12-oz.~container of {\bf
       whipped topping} (thawed) and drizzle 1/2 cup {\bf chocolate
       syrup} over that.}
\Item {Refrigerate to chill completely.  Cut into squares to
       serve.}
\ere

\vfill\eject

\begin{figure}[!ht]
\centering
%\resizebox{\hsize}{!}{\gname{OrionA}\hspace{0.5cm}\gname{OrionB}}
%\resizebox{\hsize}{!}{\gbb{272,699}{OrionA}\hspace{0.5cm}\gbb{272,699}{OrionB}}
\resizebox{5.8in}{!}{\gbb{272,699}{OrionA}\hspace{0.5cm}\gbb{272,699}{OrionB}}
\caption[Orion hot core at K band]{{\bf Early EVLA science:} The
spectrum of the hot core of Orion A at K band.  Three separate
observations of 8192 channels each 0.125 MHz wide were made using 12
antennas in the D array.  Two hours total telescope time went into
each of the two lower thirds of the spectrum and 1 hour was used for
the highest third.  The plot was made using {\tt ISPEC} over a 54 by
60 arc second area.  Line identifications provided by Karl
Menten.\Iodx{EVLA}}
\label{fig:OrionKband}
\end{figure}

%\sects{Additional recipes}

%\vfill\eject

%\hphantom{MMMM}
