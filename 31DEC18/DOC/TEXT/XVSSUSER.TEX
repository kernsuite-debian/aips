%-----------------------------------------------------------------------
%;  Copyright (C) 1995
%;  Associated Universities, Inc. Washington DC, USA.
%;
%;  This program is free software; you can redistribute it and/or
%;  modify it under the terms of the GNU General Public License as
%;  published by the Free Software Foundation; either version 2 of
%;  the License, or (at your option) any later version.
%;
%;  This program is distributed in the hope that it will be useful,
%;  but WITHOUT ANY WARRANTY; without even the implied warranty of
%;  MERCHANTABILITY or FITNESS FOR A PARTICULAR PURPOSE.  See the
%;  GNU General Public License for more details.
%;
%;  You should have received a copy of the GNU General Public
%;  License along with this program; if not, write to the Free
%;  Software Foundation, Inc., 675 Massachusetts Ave, Cambridge,
%;  MA 02139, USA.
%;
%;  Correspondence concerning AIPS should be addressed as follows:
%;          Internet email: aipsmail@nrao.edu.
%;          Postal address: AIPS Project Office
%;                          National Radio Astronomy Observatory
%;                          520 Edgemont Road
%;                          Charlottesville, VA 22903-2475 USA
%-----------------------------------------------------------------------
\documentstyle{article}

\title{XVSS User Notes\
       15JUL90 (Beta) Release}

\author{Chris Flatters}

\begin{document}

\maketitle

\section{Introduction}

This is the (preliminary) documentation for the AIPS
XView\footnote{XView is a trademark of Sun Microsystems Inc} Screen
Server, XVSS, running under the X Window System\footnote{X Window
System is a trademark of the Massachussetts Institute of Technology}.
Section 2 is a brief introduction to the OPEN LOOK\footnote{OPEN
LOOK is a trademark of AT\&T} graphical user interface. It also
assumes that the XView programs {\tt cmdtool} and {\tt shelltool} are
available from the ``Programs'' workspace menu. AIPS managers
at sites which use a different GUI or window manager may wish to
substitute their own Section 2.

\section{An Introduction to OPEN LOOK}

Most sites will probably run XVSS under an OPEN LOOK window manager.
This section should give you enough information to get started using
OPEN LOOK. Like most GUIs OPEN LOOK is fairly intuitive so that once
you have a grasp of the basic conventions you can work the rest out
for yourself.

Once you log in and start X Windows (this may be done automatically
when you log in) you should see a display with a number of {\em
windows} with obvious borders and possibly some small {\em icons} (any
icons will be grouped at one edge of the screen).

To get the feel of how OPEN LOOK works, start a terminal emulator
program. Use the mouse to move the cursor (a small black arrow
outlined in white) over the background pattern on the screen and press
the right-hand mouse button\footnote{Some sites may reconfigure the
default mouse button assignments. This is not a recommended practice.}
(called the {\em MENU} button). A menu will pop up containing about
half-a-dozen items. Move the pointer over the ``Programs'' item and
press MENU again and a menu entitled ``Programs'' will appear to the
right of the original menu. Any item in a menu that is marked with a
small triangle has a submenu that may be invoked in this fashion. At
the top left of the new menu is a small {\em glyph} representing a
push-pin. Move the pointer over the glyph and press the left-hand
mouse button (called {\em SELECT}); the push-pin will be pushed in.
Now move the pointer to the item in the second menu labelled ``Command
Tool'' and press SELECT again. After a few seconds a terminal emulator
will appear on the screen. Now move the pointer back over the push-pin
and press SELECT. Both menus disappear. If you had not pushed the pin
in before starting the command tool the menus would have disappeared
as soon as you had pressed the menu item. Push-pins provide a way to
keep otherwise transient objects present on the display.

The terminal emulator you have just created has a large window
representing the terminal. To the right of the is a {\em scroll bar}
and surrounding both is a border. At the top left of the window is a
small square {\em button} containing a triangle. To send input to the
terminal emulator, or to any other program running in a window, you
need to direct the {\em input focus} to the window. Move the pointer
in and out of the terminal emulator window several times. If a bar
over the top of the window changes when you do this you then are in {\em
follow-mouse} mode and the input focus is automatically directed to
the window containing the pointer. If not, you are in {\em
click-to-focus} mode and will have to click on a mouse button after
placing the pointer in the window. Click-to-focus mode is not
recommended for using XVSS; if you find you are in click-to-focus mode
see a local expert to find out how to change it. The next section will
implicitly assume that you are in follow-mouse mode. Type some commands
into the main window and you will see that it behaves like a normal
terminal. When you have typed enough commands that the window is full
you can look at what has scrolled off the screen by using the scroll
bar. There are a number of different ways you can use the scroll bar
by positioning the pointer over various regions of it and pressing
SELECT. The best way to find out how the scroll bar works is to
experiment with it.

You can move the entire window about by placing the pointer on the
window border, pressing SELECT and (keeping it pressed) moving the
mouse. Release SELECT to stop moving the window. This operation is
called {\em dragging}. You may change the size of the window by
dragging one of the four highlighted {\em resize corners}. The final
element on the window is the small square button. Clicking SELECT with
the pointer over this causes the window to shrink to an icon. To get
the window back, place the pointer over the icon and {\em
double-click} (quickly press SELECT twice) on it.

You should now play around with OPEN LOOK for a few minutes to develop
a feel for it.

\section{Using XVSS}

XVSS may be started in a number of ways: it may be started
automatically when you run AIPS or you may have to start it with an
explicit command. You may even be able to start it from the
``Programs'' menu. Consult your local site guide for instructions. If
you are running AIPS on a computer different from the one on which
XVSS is running, you will have to perform some special operations to
connect AIPS to XVSS; again, see your local site guide for details.

XVSS will appear on the screen as a window with two areas, a control
panel with five buttons (``Resize'', ``A'', ``B'', ``C'' and ``D'')
and the display surface. On some machines the colours on the screen
will change when you move the pointer in and out of the display area.
You shouldn't worry about this, other than the fact that the colours
on the AIPS TV will not be displayed correctly unless the pointer is
in the display area.

The four ``letter'' buttons are the buttons referred to in AIPS
prompts. The ``Resize'' button toggles between a full-sized display
(which takes up most of the screen) and a smaller display. You may
change the size of the smaller display using the resize corners. To
press a button, you should move the pointer over it and press select.
The following function keys may be used as short-cuts if present on
your keyboard.

\vspace{16pt}

\begin{tabular}{|c|c|}
\hline
Button & Function Key Equivalents \\
\hline
Resize & F2, F7 \\
A & F3 \\
B & F4 \\
C & F5 \\
D & F6 \\
\hline
\end{tabular}

\vspace{16pt}

The AIPS cursor is not visible on the XVSS screen (for technical
reasons). It is manipulated using the normal window system pointer. If
the pointer is in the display area then pressing SELECT will move the
AIPS cursor to the same location as the pointer. The only way to get
used to this is to practice: load an image and use TVFIDDLE or
TVPSEUDO to modify the colours.

Finally, you should note that it is essential that you initialize XVSS
using TVINIT before using it. If XVSS is not properly initialized it
may cause AIPS to crash.
\end{document}
