%-----------------------------------------------------------------------
%;  Copyright (C) 1995
%;  Associated Universities, Inc. Washington DC, USA.
%;
%;  This program is free software; you can redistribute it and/or
%;  modify it under the terms of the GNU General Public License as
%;  published by the Free Software Foundation; either version 2 of
%;  the License, or (at your option) any later version.
%;
%;  This program is distributed in the hope that it will be useful,
%;  but WITHOUT ANY WARRANTY; without even the implied warranty of
%;  MERCHANTABILITY or FITNESS FOR A PARTICULAR PURPOSE.  See the
%;  GNU General Public License for more details.
%;
%;  You should have received a copy of the GNU General Public
%;  License along with this program; if not, write to the Free
%;  Software Foundation, Inc., 675 Massachusetts Ave, Cambridge,
%;  MA 02139, USA.
%;
%;  Correspondence concerning AIPS should be addressed as follows:
%;          Internet email: aipsmail@nrao.edu.
%;          Postal address: AIPS Project Office
%;                          National Radio Astronomy Observatory
%;                          520 Edgemont Road
%;                          Charlottesville, VA 22903-2475 USA
%-----------------------------------------------------------------------
% NRAOBASIC.TEX   Copied   850812 by Pat Murphy from Howard Bond's STBASIC
%                                 File, taken in turn from STScI, SCIVAX.
%                 Modified 850813 by Pat Murphy -- make 12 point the default
%                                 font by the \magnification=\magstep1 macro.
%				  Also added Fred Schwab's stuff to make it
%				  look better; need to do more than magnify!
%		  Modified 850819 by Pat Murphy: back to 10point default
%				  - let user give MAGSTEP command.
%		  Expanded 860113 by Pat Murphy: Try to get rid of "TRUE" so
%				  that most stuff will work with \magnification
%		  Modified 860213 by Pat Murphy: add \title, \notitle
% NRAO_MACROS	  Modified 860321 by Pat Murphy: new name (as indicated) and
%				  remove interlinepenalty setting (let it go
%				  back to default
% NRAO_MACROS_V2  Modified 860912 by Pat Murphy: test version for TeX V2.0
%                 Modified 861216 by Pat Murphy: Fix up sections, headings
% ============================================================================
% Make old 12 point stuff available as Howard Bond had it....
% I have put in the "cheap substitutes" for the true 12-point fonts here. (PPM)
% Twelvepoint style assuming all 12-pt fonts are downloaded
% (see TeXbook, pp. 351,414)   Required 12-pt fonts are mi12,sy12,ti12,b12
\def\oldtwelvepoint{
  \font\aipsfont=cmsy10 at 12 pt		% Font for writing fancy AIPS
  \textfont0=\twelverm \scriptfont0=\tenrm     
    \scriptscriptfont0=\sevenrm                 %text fonts
  \def\it{\fam0 \twelveit}
  \def\bf{\fam0 \twelvebf}
  \def\sl{\fam0 \twelvesl}
  \def\tt{\fam0 \twelvett}
  \def\rm{\fam0 \twelverm}   
  \textfont1=\twelvei  \scriptfont1=\teni  
    \scriptscriptfont1=\seveni                  %math italic fonts
  \def\mit{\fam1 } \def\oldstyle{\fam1 \twelveit}
  \textfont2=\twelvesy \scriptfont2=\tensy 
    \scriptscriptfont2=\sevensy                 %math symbol fonts
  \parskip=3pt plus 1pt minus 1pt  %it is zero plus 1pt in plain.tex
  \parindent=0.42in       %that's 5/12 inch!
  \baselineskip=12pt\lineskiplimit=0pt
    \lineskip=-0.5mm      %this suppresses extra spacing before lines with
                          % superscripts, Angstroms, etc.
  \everypar={}            %turns off anything left from \beginminutes, etc.
  \def\doublespace{\baselineskip=24pt}
  \nonfrenchspacing
  \rm}
%
%       Abbreviations
%
\def\aa{{\it Astr.~Ap.},\ }
\def\aips{{\aipsfont AIPS}}
\def\AIPS{{\aipsfont AIPS}}
\def\aj{{\it A.~J.},\ }
\def\ang{\AA}                                %Angstrom unit
\def\apj{{\it Ap.~J.},\ }                    %for references
\def\apjl{{\it Ap.~J. (Letters)},\ }
\def\deg{$^\circ$}
\def\eg{{\it e.g.}}
\def\etal{{\it et~al.}}
\def\etc{{\it etc.}}
\def\ie{{\it i.e.}}
\def\iue{{\it IUE\/}}                        %IUE in italics
\def\IUE{{\it IUE\/}}                        %IUE in italics
\def\mn{{\it M.N.R.A.S.},\ }
\def\pasp{{\it Pub.~A.S.P.},\ }
\def\peryr{$\>{\rm yr}^{- 1}$}
\def\sqdeg{${\rm deg}^2$}
\def\squig{$\sim\!\!$\ }
\def\subsun{$_{\odot}$}
\def\sun{$\odot$}                
\def\VLA{{\it VLA\/}}
\def\vla{{\it VLA\/}}
%
%                    Control sequences
%
\def\beginminutes{\everypar={\hangindent=0.3in} \parindent=0pt
  \parskip=5pt plus 1pt minus 1pt}
\def\beginrefs{\parindent=0pt\frenchspacing\parskip=0pt
   \everypar={\hangindent=0.42in}}                  %start bibliography
\def\block{$\;$\vrule height7pt width6pt depth0pt$\>$}
\def\ctr#1{\bigbreak\centerline{#1}\nobreak\bigskip\nobreak}  %center with 
                                                            %skips before&after
\def\distribution{\vskip30pt\bigbreak\parindent=0pt\frenchspacing\parskip=0pt
   \obeylines{\it Distribution:}\par\vskip3pt\parindent=0.15in}

\def\er{\par}                 %put at end of each reference in bibliography
\def\ef{\par\vskip1pt\hangindent=53pt} %at end of each fig caption
\def\pagenumbers{\footline={\hss - \folio\ - \hss}}  %turns on pagenumbering
\def\section#1{\bigbreak\bigbreak\medbreak\centerline{\bf #1}\nobreak\medskip
   \nobreak}
\def\xc{\vskip30pt\bigbreak\parindent=0pt\frenchspacing\parskip=0pt
   \obeylines{\it xc:}\par\vskip3pt\parindent=0.15in}

%Memoheader control sequence from Howard Bond (@STScI)
% - the sequence is \memoheader {to}{subject}
%                or \memoheader to, subject        (If text has no commas)
% Today's date is assumed
% Also, you should say \name {Bilbo Baggins} before using it, or you will
% not get your own name!
% Modified 890811 PPM: set title too.

\def\memoheader#1#2{\title{#2}
  {\noindent {\nraofont National Radio Astronomy Observatory} 
   \hfill {\sl MEMORANDUM}}
  \vskip 3pt
  \hrule height1pt
  \vskip 6pt
  {\parindent=0pt\parskip=0pt
  \line{{\it To:}\hskip2.48em #1
  \hfil {\it Date:\/}\hskip0.5em \today}
  \vskip 4pt
  {\it From:}\hskip1.36em \sender
  \vskip 4pt                         
  {\it Subject:\/}\hskip0.5em  #2}
  \vskip 6pt
  \hrule height1pt
  \vskip 20pt}

%"NASA dots" and "\nnarrower" control sequence to set text in 
%      from the current right and left margins more than \narrower gives.
%      Note that \nnarrower has no effect unless the paragraph ends before the
%      group ends (TeXbook, p. 100).

\def\nnarrower{\advance\leftskip by 50pt\advance\rightskip by 45pt}
\def \ndot{\nnarrower\parindent=0pt\itemitem{$\bullet\,$}}
\def\nndot#1{{\nnarrower\parindent=0pt\itemitem{$\bullet\,$}#1}}
\def\nodot{\nnarrower\parindent=0pt\itemitem{o\ \/}}  % "o-shaped" "NASA dots" 

%some handy stuff for Outlines

\def\beginoutline{
  \frenchspacing
  \parskip=0pt plus 1pt \parindent=0.25in
  \def\in{\par\indent \hangindent2\parindent \textindent}    % like \itemitem
  \def\inin{\par\indent \indent \hangindent3\parindent \textindent}} 
							% "\itemitemitem"

% ==========================================================================
% Copied from NEWFMT.TEX in Fred Schwab's VAX3 area by Pat Murphy 850812 and
% modified to be more VLA-specific (for letterhead...)

% extra fonts needed

\font\eightrm=cmr8
\font\sixrm=cmr6
\font\eighti=cmmi8 \skewchar\eighti='177
\font\sixi=cmmi6 \skewchar\sixi='177
\font\eightsy=cmsy8 \skewchar\eightsy='60
\font\sixsy=cmsy6 \skewchar\sixsy='60
\font\eightbf=cmbx8
\font\sixbf=cmbx6
\font\eightsl=cmsl8
\font\eightit=cmti8
\font\tensmc=cmcsc10

\font\twelverm=cmr10 at 12pt		% Roman
\font\twelveit=cmti10 at 12pt		% Normal Italic
\font\twelvei=cmmi10 at 12pt		% Math Italic
\font\twelvesy=cmsy10 at 12 pt		% Math symbol
\font\twelvesl=cmsl10 at 12pt		% Roman Slanted
\font\twelvebf=cmbx10 at 12pt		% Roman Boldface
\font\twelvett=cmtt10 at 12pt		% Typewriter
\font\twelvesmc=cmcsc10 at 12pt		% Capitals and small capitals
\font\twelveex=cmex10 at 12pt		% Math extended

% THREE DIFFERENT POINT SIZES 
% (NB if magstep1 set, then ten point is like 12 point, 8 like 10, etc)
% All three sizes work at all magsteps now (subject to font availability!)

\def\twelvepoint{\normalbaselineskip=14pt % changed from 12pt 890811 PPM
 \abovedisplayskip 17pt plus 3pt minus 9pt
 \belowdisplayskip 17pt plus 3pt minus 9pt
 \abovedisplayshortskip 0pt plus 3pt
 \belowdisplayshortskip 10pt plus 3pt minus 4pt
  \parskip=3pt plus 1pt minus 1pt  %it is zero plus 1pt in plain.tex
  \parindent=0.42in       %that's 5/12 inch!
 \def\rm{\fam0\twelverm}%
 \def\it{\fam\itfam\twelveit}%
 \def\sl{\fam\slfam\twelvesl}%
 \def\bf{\fam\bffam\twelvebf}%
 \def\tt{\fam\ttfam\twelvett}%
 \def\smc{\twelvesmc}%
 \def\mit{\fam 1}%
 \def\cal{\fam 2}%
 \font\aipsfont=cmsy10 at 12 pt		% Font for writing fancy AIPS
 \textfont0=\twelverm   \scriptfont0=\tenrm   \scriptscriptfont0=\sevenrm
 \textfont1=\twelvei    \scriptfont1=\teni    \scriptscriptfont1=\seveni
 \textfont2=\twelvesy   \scriptfont2=\tensy   \scriptscriptfont2=\sevensy
 \textfont3=\twelveex   \scriptfont3=\twelveex     \scriptscriptfont3=\twelveex
 \textfont\itfam=\twelveit
 \textfont\slfam=\twelvesl
 \textfont\bffam=\twelvebf \scriptfont\bffam=\tenbf
   \scriptscriptfont\bffam=\sevenbf
   \def\doublespace{\baselineskip=24pt}
\normalbaselines\rm}

\def\tenpoint{\normalbaselineskip=12pt          
 \abovedisplayskip 12pt plus 3pt minus 9pt
 \belowdisplayskip 12pt plus 3pt minus 9pt
 \abovedisplayshortskip 0pt plus 3pt
 \belowdisplayshortskip 7pt plus 3pt minus 4pt
  \parskip=2.5pt plus 1pt minus 1pt  %it is zero plus 1pt in plain.tex
  \parindent=0.42in       %that's 5/12 inch!
 \def\rm{\fam0\tenrm}%
 \def\it{\fam\itfam\tenit}%
 \def\sl{\fam\slfam\tensl}%
 \def\bf{\fam\bffam\tenbf}%
 \def\tt{\fam\ttfam\tentt}%
 \def\smc{\tensmc}%
 \def\mit{\fam 1}%
 \def\cal{\fam 2}%
 \font\aipsfont=cmsy10 			% Font for writing fancy AIPS
 \textfont0=\tenrm   \scriptfont0=\sevenrm   \scriptscriptfont0=\fiverm
 \textfont1=\teni    \scriptfont1=\seveni    \scriptscriptfont1=\fivei
 \textfont2=\tensy   \scriptfont2=\sevensy   \scriptscriptfont2=\fivesy
 \textfont3=\tenex   \scriptfont3=\tenex     \scriptscriptfont3=\tenex
 \textfont\itfam=\tenit
 \textfont\slfam=\tensl
 \textfont\bffam=\tenbf \scriptfont\bffam=\sevenbf
   \scriptscriptfont\bffam=\fivebf
   \def\doublespace{\baselineskip=20pt}
\normalbaselines\rm}

\def\eightpoint{\normalbaselineskip=10pt
 \abovedisplayskip 10pt plus 2.4pt minus 7.2pt
 \belowdisplayskip 10pt plus 2.4pt minus 7.2pt
 \abovedisplayshortskip 0pt plus 2.4pt
 \belowdisplayshortskip 5.6pt plus 2.4pt minus 3.2pt
 \def\rm{\fam0\eightrm}%
 \def\it{\fam\itfam\eightit}%
 \def\sl{\fam\slfam\eightsl}%
 \def\bf{\fam\bffam\eightbf}%
 \def\tt{\fam\ttfam\tentt}%
 \def\mit{\fam 1}%
 \def\cal{\fam 2}%
 \font\aipsfont=cmsy8 		% Font for writing fancy AIPS
 \textfont0=\eightrm   \scriptfont0=\sixrm   \scriptscriptfont0=\fiverm
 \textfont1=\eighti    \scriptfont1=\sixi    \scriptscriptfont1=\fivei
 \textfont2=\eightsy   \scriptfont2=\sixsy   \scriptscriptfont2=\fivesy
 \textfont3=\tenex   \scriptfont3=\tenex     \scriptscriptfont3=\tenex
 \textfont\itfam=\eightit
 \textfont\slfam=\eightsl
 \textfont\bffam=\eightbf \scriptfont\bffam=\sixbf
   \scriptscriptfont\bffam=\fivebf
   \def\doublespace{\baselineskip=16pt}
\normalbaselines\rm}

% TODAY macro to give today's date in format Month DD, YYYY.

\def\today{\ifcase\month\or
 January\or February\or March\or April\or May\or June\or
 July\or August\or September\or October\or November\or
 December\fi\space\number\day, \number\year}

\hsize=6.25truein\vsize=9.5truein		% These are good for the
\hoffset=.125truein\voffset=-.2truein		% QMS Lasergrafix 800 printer
\advance\vsize by-24pt

%-----------------------------------------------------------------------
% Stuff from Roger Noble: table of contents, etc.
%
\newcount\secnum	% main section number,   the M in M.N.O
\newcount\subsecnum	% subsection number,     the N in M.N.O
\newcount\subsubsecnum	% sub-subsection number, the O in M.N.O
\newcount\footnum	% What is this?
\newcount\contflag
%
% Initialize these newly declared variables
%
\secnum=0
\subsecnum=0
\subsubsecnum=0
\footnum=0
\contflag=0		% 1 if generating table of contents
%
% Write contents record
%
\def\leaderfill{\leaders\hbox{\kern 0.5em.\kern 0.5em}\hfill}

\def\writecontents#1{\ifnum\contflag=1\write\contfile{\vskip 5pt\par\noindent
 {\noexpand\bf\the\secnum.\enspace#1}\unskip\noexpand\leaderfill
 \rm\the\pageno}\fi}

\def\writesubcontents#1{\ifnum\contflag=1\write\contfile{\vskip 2pt\par
 {\noexpand\sl\quad\the\secnum.\the\subsecnum\enspace#1}
 \unskip\noexpand\leaderfill\rm\the\pageno}\fi}

\def\writesubsubcontents#1{\ifnum\contflag=1\write\contfile{\par
 {\noexpand\rm\quad\quad\the\secnum.\the\subsecnum.\the\subsubsecnum\enspace#1}
 \unskip\noexpand\leaderfill\rm\the\pageno}\fi}
%
% Define the section macro
%
\def\newsection#1\par{\global\advance\secnum by 1\global\subsecnum=0
 \vskip 8pt plus.1\vsize\penalty-250\vskip 0pt plus-.1\vsize\bigskip\vskip
 \parskip\mark{#1}\message{\the\secnum. #1}\immediate\writecontents{#1}
 \leftline{\nraofont \the\secnum.\enspace#1}\nobreak\smallskip\noindent}
%
% and the subsection macro
%
\def\newsubsection#1\par{\global\advance\subsecnum by 1\global\subsubsecnum=0
 \ifnum\subsecnum>1
  \vskip 4pt plus .1\vsize\penalty-250\vskip 0pt plus-.1\vsize
 \else
  \nobreak
 \fi
 \medskip\vskip\parskip\mark{#1}\immediate\writesubcontents{#1}
 \leftline{\twelvebf \the\secnum.\the\subsecnum\enspace#1}\nobreak
 \smallskip\noindent}
%
% and the subsubsection macro
%
\def\newsubsubsection#1\par{\global\advance\subsubsecnum by 1
 \ifnum\subsubsecnum>1
  \vskip 2pt plus .1\vsize\penalty-250\vskip 0pt plus-.1\vsize
 \else
  \nobreak
 \fi
 \medskip\vskip\parskip\mark{#1}\immediate\writesubsubcontents{#1}
 \leftline{\twelveit\the\secnum.\the\subsecnum.\the\subsubsecnum\enspace#1}
 \nobreak\smallskip\noindent}
%
% and the revised (auto-numbering) footnote macro
%
\def\makefoot#1{\global\advance\footnum by 1\footnote{$^{\the\footnum}$}{#1}}
%
%-----------------------------------------------------------------------
% headline stuff
% 
\headline={\ifnum\pageno>1
              \ifodd\pageno {\it\doctitle\hfil Page \folio}
              \else         {\it Page \folio\hfil\doctitle} \fi 
           \else
               \hfil
           \fi}
\footline={\ifnum\pageno>1
              \ifodd\pageno {\it\hfil \today}
              \else         {\it\today \hfil} \fi
           \else
               \hfil
           \fi}
%-----------------------------------------------------------------------
%\def\noheadline{\headline={\hfil}
%                \footline={\hss\folio\hss}}
\def\doctitle{}					% set this to document title
						% (in YOUR file, not mine!)
\def\title#1{\def\doctitle{#1}}			% or better still say \title...
\def\notitle{\def\doctitle{}}
\def\name#1{\def\sender{#1}}			% type \name {My name}
\def\tel#1{\def\telno{#1}}			% type \tel {(800) 555--1212}
\def\example#1{\line{\tt\hskip 2.5cm {#1}\hfil}}% for examples embedded in text

\font\nraofont=cmbx10 at 14.4pt %changed from 12truept - and 13.5truept
\tenpoint
\name {AirHead}
\tel {555--1212}
% ----------------------------------------------------------------------------
