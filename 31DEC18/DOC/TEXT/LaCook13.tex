%-----------------------------------------------------------------------
%;  Copyright (C) 1995-2018
%;  Associated Universities, Inc. Washington DC, USA.
%;
%;  This program is free software; you can redistribute it and/or
%;  modify it under the terms of the GNU General Public License as
%;  published by the Free Software Foundation; either version 2 of
%;  the License, or (at your option) any later version.
%;
%;  This program is distributed in the hope that it will be useful,
%;  but WITHOUT ANY WARRANTY; without even the implied warranty of
%;  MERCHANTABILITY or FITNESS FOR A PARTICULAR PURPOSE.  See the
%;  GNU General Public License for more details.
%;
%;  You should have received a copy of the GNU General Public
%;  License along with this program; if not, write to the Free
%;  Software Foundation, Inc., 675 Massachusetts Ave, Cambridge,
%;  MA 02139, USA.
%;
%;  Correspondence concerning AIPS should be addressed as follows:
%;         Internet email: aipsmail@nrao.edu.
%;         Postal address: AIPS Project Office
%;                         National Radio Astronomy Observatory
%;                         520 Edgemont Road
%;                         Charlottesville, VA 22903-2475 USA
%-----------------------------------------------------------------------
\chapts{Current \AIPS\ Software}{list}

\renewcommand{\titlea}{31-December-2018, revised 16-August-2018)}
\renewcommand{\Rheading}{\AIPS\ \cookbook:~\titlea\hfill}
\renewcommand{\Lheading}{\hfill \AIPS\ \cookbook:~\titlea}
\markboth{\Lheading}{\Rheading}

     The complete lists of software in \AIPS\ are kept up-to-date in
certain special files which may then be accessed with the {\tt AIPS}
verb \hbox{{\tt ABOUT}}.  Semi-automatic software makes these listing
files using the primary and secondary keywords entered by \AIPS\
programmers in the numerous help files.  The explanations given for
each symbol are also those entered by the programmers as the
``one-liner'' descriptions of the symbol for which the help file is
written.  The lists of primary and secondary keywords may be viewed
directly by typing:
\dispt{HELP \qs CATEGORY \CR}{to view primary keywords}
\dispt{HELP \qs SECONDARY \CR}{to view secondary keywords}
\dispe{The help file for {\tt ABOUT} is more explanatory, however.
The general help file (for {\tt HELP} itself) lists a number of
general help files which will be of interest.  These are also
mentioned in the section called {\tt INFORMATION} below.}

     The following sections are verbatim reproductions of the various
listing files used by {\tt ABOUT} and are roughly current to the
current release of \hbox{\AIPS}.  Each section title is the name of
the keyword which is used as the {\tt ABOUT} topic.  Each line within
a section lists a task, verb, pseudoverb, procedure, adverb, or {\tt
RUN} file with a very brief description of its function.  Pseudoverbs
come in two flavors: those that act roughly like verbs and those that
must be treated specially, \ie\ that must appear alone on a line or
only in certain contexts.  The help file for each pseudoverb should
clarify the its grammatical limits.  Typing
\dispt{HELP\qs {\it name} \CR}{ }
\dispe{where {\it name} is one of the entries in the left-hand column,
will give more useful information about that \AIPS\ symbol. }

\small
\sects{ADVERB}

\vskip 0.5pt\todx{ABOUT ADVERB}\iodx{adverb}
\bbve\begin{verbatim}
ADVERB
Type:  General type of POPS symbol
Use:   Adverbs are the symbols used to address values.  They
       may be REAL (single-precision floating point), ARRAY
       (multiply-dimensioned REALs), or STRING (character
       strings with or without subscripts).  The user may
       create new adverbs by defining them while typing or
       editing procedures.
Grammar:   Adverb names may be used either in compile mode
       or in regular execute mode.  In the former, their
       pointers are compiled with the procedure and their
       values, at the time the procedure is invoked, are
       used during the execution of the procedure.
Usage examples:
       ARRAY2 = ARRAY1
       ARRAY3(I,J) = 23.6
       CHAN = 4
       STRARRY = 'STR1','STR2','STR3',STRAR2
       DUM3 (24, I, STRARRY)
    =====>  Character string data must be enclosed in quotes!
    =====>  This allows the compiler to tell data from adverb
            names and allows embedded special characters.
****************************************************************
List of ADVERBs

ALIAS      adverb to alias antenna numbers to one another
ALLOKAY    specifies that initial conditions have been met.
ANTENNAS   Antennas to include/exclude from the task or verb
ANTNAME    A list of antenna (station) names
ANTUSE     Antennas to include/exclude from the task or verb
ANTWT      Antenna Weights for UV data correction in Calibration
APARM      General numeric array adverb used many places
ARRAY1     General scratch array adverb
ARRAY2     General scratch array adverb
ARRAY3     General scratch array adverb
ASDMFILE   Full path to EVLA ASDM/BDF directory
ASPMM      Plot scaling parameter - arc seconds per millimeter on plot
AVGCHAN    Controls averaging of spectral channels
AVGIF      Controls averaging of IF channels
AVOPTION   Controls type or range of averaging done by a task
AX2REF     Second reference pixel number
AXINC      Axis increment - change in coordinate between pixels
AXREF      Reference pixel number
AXTYPE     Type of coordinate axis
AXVAL      Value of axis coordinate at reference pixel
BADDISK    specifies which disks are to be avoided for scratch files
BAND       specifies the approximate frequency of UV data to be selected
BANDPOL    specifies polarizations of individual IFs
BASELINE   specifies which antenna pairs are to be selected/deselected
BATFLINE   specifies starting line in a batch work file
BATNLINE   specifies the number of lines to process in a batch work file
BATQUE     specifies the desired batch queue
BCHAN      sets the beginning channel number
BCOMP      gives beginning component number for multiple fields
BCOUNT     gives beginning location for start of a process
BDROP      gives number of pooints dropped at the beginning
BIF        gives first IF to be included
BITER      gives beginning point for some iterative process
BLC        gives lower-left-corner of selected subimage
BLOCKING   specifies blocking factor to use on e.g. tape records
BLVER      specifies the version of the baseline-calibration table used
BMAJ       gives major axis size of beam or component
BMIN       gives minor axis size of beam or component
BOXFILE    specifies name of Clean box text file
BOX        specifies pixel coordinates of subarrays of an image
BPA        gives position angle of major axis of beam or component
BPARM      general numeric array adverb used too many places
BPASSPRM   Control adverb array for bandpass calibration
BPRINT     gives beginning location for start of a printing process
BPVER      specifies the version of the bandpass table to be applied
BWSMEAR    amount of bandwidth smearing correction to use
CALCODE    specifies the type of calibrator to be selected
CALIN      specifies name of input disk file usually with calibration data
CALSOUR    specifies source names to be included in calibration
CATNO      Specifies AIPS catalog slot number range
CBPLOT     selects a display of a Clean beam full width at half maximum
CCBOX      specifies pixel coordinates of subarrays of an image
CELLSIZE   gives the pixel size in physical coordinates
CHANNEL    sets the spectral channel number
CHANSEL    Array of start, stop, increment channel numbers to average
CHINC      the increment between selected channels
CLBOX      specifies subarrays of an image for Clean to search
CLCORPRM   Parameter adverb array for task CLCOR
CLEV       Contour level multiplier in physical units
CLINT      CL table entry interval
CMETHOD    specifies the method by which the uv model is computed
CMODEL     specifies the method by which the uv model is computed
CODETYPE   specifies the desired operation type
COLORS     specifies the desired TV colors
COMMENT    64-character comment string
CON3COL    Controls use of full 3-color graphics for contouring
CONFIG     Configuration ID number within an EVLA ASDM/BDF data set
COOINC     Celestial axes increment: change in coordinate between pixels
COORDINA   Array to hold coordinate values
COOREF     Reference pixel number for two coordinate axes
COOTYPE    Celestial axes projection type
COPIES     sets the number of copies to be made
CPARM      general numeric array adverb used many places
CROWDED    allows a task to perform its function in a crowded fashion
CTYPE      specifies type of component
CUTOFF     specifies a limit below or above which the operation ends
DARKLINE   The level at which vectors are switched from light to dark
DATA2IN    specifies name of input FITS disk file
DATAIN     specifies name of input FITS disk file
DATAOUT    specifies name of output FITS disk file
DCHANSEL   Array of start, stop, increment channel #S + IF to avoid
DCODE      General string adverb
DDISK      Deterimins where input DDT data is found
DDTSIZE    Deterimins which type of DDT is RUN.
DECIMAL    specifies if something is in decimal format
DECSHIFT   gives Y-coordinate shift of an image center from reference
DEFER      Controls when file creation takes place
DELCORR    specifies whether VLBA delay corrections  are to be used
DELTAX     Increment or size in X direction
DELTAY     Increment or size in Y direction
DENSITY    gives the desired tape density
DENUMB     a scalar decimal number
DETIME     specifies a time interval for an operation (destroy, batch)
DIGICOR    specifies whether VLBA digital corrections are to be applied
DIST       gives a distance - PROFL uses as distance to observer
DO3COL     Controls whether full 3-color graphics are used in a plot
DO3DIMAG   specifies whether uvw's are reprojected to each field center
DOACOR     specifies whether autocorrelation data are included
DOALIGN    specifies how two or more images are aligned in computations
DOALL      specifies if an operation is done once or for all matching
DOALPHA    specifies whether some list is alphabetized
DOAPPLY    Flag to indicate whether an operation is applied to the data
DOARRAY    spcifies if subarrays are ignored or the information used
DOBAND     specifies if/how bandpass calibration is applied
DOBLANK    controls handling of blanking
DOBTWEEN   Controls smoothing between sources in calibration tables
DOCALIB    specifies whether a gain table is to be applied or not
DOCAT      specifies whether the output is saved (cataloged) or not
DOCELL     selects units of cells over angular unit
DOCENTER   selects something related to centering
DOCIRCLE   select a "circular" display (i.e. trace coordinates, ...)
DOCOLOR    specifies whether coloring is done
DOCONCAT   selects concatenated or indivudual output files
DOCONFRM   selects user confirmation modes of repetitive operation
DOCONT     selects a display of contour lines
DOCRT      selects printer display or CRT display (giving width)
DODARK     specifies whether "dark" vectors are plotted dark or light
DODELAY    selects solution for phase/amplitude or delay rate/phase
DOEBAR     Controls display of estimates of the uncertainty in the data
DOEOF      selects end-of-file writing or reading until
DOEOT      selects tape positioning before operation: present or EOI
DOFIT      Controls which antennas are fit by what methods
DOFLAG     Controls closure cutoff in gain solutions and flagging
DOFRACT    Tells whether to compute a fraction or ratio
DOGREY     selects a display of a grey-scale image
DOGRIDCR   selects correction for gridding convolution function
DOHIST     selects a histogram display
DOHMS      selects sexagesimal (hours-mins-secs) display format
DOIFS      controls functions done across IFs
DOINVERS   selects opposite of normal function
DOKEEP     specifies if something is kept or deleted
DOMAX      selects solutions for maxima of models
DOMODEL    selects display of model function
DONEWTAB   do we make new tables, use a new table format, etc.
DOOUTPUT   selects whether output image or whatever is saved / discarded
DOPLOT     Controls plotting of something
DOPOL      selects application of any polarization calibration
DOPOS      selects solutions for positions of model components
DOPRINT    selects printer display or CRT display (giving width)
DORESID    selects display of differences between model and data
DOROBUST   Controls method of averaging - simple mean/rms or robust
DOSCALE    specifies if a scaling operation of some sort is to be performed
DOSCAN     specifies if a scan-related operation is to be done
DOSLICE    selects display of slice data
DOSPIX     selects solutions for spectral index of model components
DOSTOKES   selects options related to polarizations
DOTABLE    selects use of table-format for data
DOTV       selects use of TV display option in operation
DOTWO      do we make two of something
DOUVCOMP   selects use of compression in writing UV data to disk
DOVECT     selects display of polarization vectors
DOWAIT     selects wait-for-completion mode for running tasks
DOWEDGE    selects display of intensity step wedge
DOWEIGHT   selects operations with data weights
DOWIDTH    selects solution for widths of model components
DPARM      General numeric array adverb used many places
ECHAN      define an end for a range of channel numbers
ECOUNT     give the highest count or iteration for some process
EDGSKP     Deterimins border excluded from comparision or use
EDROP      number of points/iterations to be omitted from end of process
EFACTOR    scales some error analysis process
EHNUMB     an extended hexadecimal "number"
EIF        last IF number to be included in operation
EPRINT     gives location for end of a printing process
ERROR      was there an error
EXPERT     specifies an user experience level or mode
FACTOR     scales some display or CLEANing process
FGAUSS     Minimum flux to Clean to by widths of Gaussian models
FITOUT     specifies name of output text file for results of fitting
FLAGVER    selects version of the flagging table to be applied
FLDSIZE    specifies size(s) of images to be processed
FLMCOMM    Comment for film recorder image.
FLUX       gives a total intensity value for image/component or to limit
FMAX       specifies peak values of model components - results of fits
FORMAT     gives a format code number: e.g. FITS accuracy required
FOV        Specifies the field of view
FPARM      General numeric array adverb used in modeling
FPOS       specifies pixel positions of fit model components
FQCENTER   specifies that the frequency axis should be centered
FQTOL      Frequency tolerance with which FQ entries are accepted.
FREQID     Frequency Identifier for frequency, bandwidth combination
FSHIFT     specifies a position shift - output from fitting routines
FSIZE      file size in Megabytes
FUNCTYPE   specifies type of intensity transfer function
FWIDTH     gives widths of model components - results of fitting
GAINERR    gives estimate of gain uncertainty for each antenna
GAIN       specifies loop gain for deconvolutions
GAINUSE    specifies output gain table or gain table applied to data
GAINVER    specifies the input gain table
GCVER      specifies the version of the gain curve table used
GG         spare scalar adverb for use in procedures
GMAX       specifies peak values of model components
GPOS       specifies pixel positions of model components
GRADDRES   specifies user's home address for replies to gripes
GRCHAN     specifies the TV graphics channel to be used
GREMAIL    gives user's e-mail address name for reply to gripe entry
GRNAME     gives user's name for reply to gripe entry
GRPHONE    specifies phone number to call for questions about a gripe
GUARD      portion of UV plane to receive no data in gridding
GWIDTH     gives widths of model components
HIEND      End record number in a history-file operation
HISTART    Start record number in a history-file operation
ICHANSEL   Array of start, stop, increment channel #S + IF to average
ICUT       specifies a cutoff level in units of the image
I          spare scalar adverb for use in procedures
IM2PARM    Specifes enhancement parameters for OOP-based imaging: 2nd set
IMAGRPRM   Specifes enhancement parameters for OOP-based imaging
IMSIZE     specifies number of pixels on X and Y axis of an image
IN2CLASS   specifies the "class" of the 2nd input image or data base
IN2DISK    specifies the disk drive of the 2nd input image or data base
IN2EXT     specifies the type of the 2nd input extension file
IN2FILE    specifies name of a disk file, outside the regular catalog
IN2NAME    specifies the "name" of the 2nd input image or data base
IN2SEQ     specifies the sequence # of the 2nd input image or data base
IN2TYPE    specifies the type of the 2nd input image or data base
IN2VERS    specifies the version number of the 2nd input extension file
IN3CLASS   specifies the "class" of the 3rd input image or data base
IN3DISK    specifies the disk drive of the 3rd input image or data base
IN3EXT     specifies the type of the 3rd input extension file
IN3NAME    specifies the "name" of the 3rd input image or data base
IN3SEQ     specifies the sequence # of the 3rd input image or data base
IN3TYPE    specifies the type of the 3rd input image or data base
IN3VERS    specifies the version number of the 3rd input extension file
IN4CLASS   specifies the "class" of the 4th input image or data base
IN4DISK    specifies the disk drive of the 4th input image or data base
IN4NAME    specifies the "name" of the 4th input image or data base
IN4SEQ     specifies the sequence # of the 4th input image or data base
IN4TYPE    specifies the type of the 4th input image or data base
IN5CLASS   specifies the "class" of the 5th input image or data base
IN5DISK    specifies the disk drive of the 5th input image or data base
IN5NAME    specifies the "name" of the 5th input image or data base
IN5SEQ     specifies the sequence # of the 5th input image or data base
IN5TYPE    specifies the type of the 5th input image or data base
INCLASS    specifies the "class" of the 1st input image or data base
INDISK     specifies the disk drive of the 1st input image or data base
INEXT      specifies the type of the 1st input extension file
INFILE     specifies name of a disk file, outside the regular catalog
INLIST     specifies name of input disk file, usually a source list
INNAME     specifies the "name" of the 1st input image or data base
INSEQ      specifies the sequence # of the 1st input image or data base
INTAPE     specifies the input tape drive number
INTERPOL   specifies the type of averaging done on the complex gains
INTEXT     specifies name of input text file, not in regular catalog
INTPARM    specifies the parameters of the frequency interpolation function
INTYPE     specifies the type of the 1st input image or data base
INVERS     specifies the version number of the 1st input extension file
IOTAPE     Deterimins which tape drive is used during a DDT RUN
ISCALIB    states that the current source is a point-source calibrator
J          spare scalar adverb for use in procedures
JOBNUM     specifies the batch job number
KEYSTRNG   gives contents of character-valued keyword parameter
KEYTYPE    Adverb giving the keyword data type code
KEYVALUE   gives contents of numeric-valued keyword parameter
KEYWORD    gives name of keyword parameter - i.e. name of header field
LABEL      selects a type of extra labeling for a plot
LEVS       list of multiples of the basic level to be contoured
LPEN       specifies the "pen width" code # => width of plotted lines
LTYPE      specifies the type and degree of axis labels on plots
MAPDIF     Records differences between DDT test results and standards
MAXPIXEL   maximum pixels searched for components in Clark CLEAN
MDISK      Deterimins where input DDT data is found
MINAMPER   specifies the minimum amplitude error prior to some action
MINANTEN   states minimum number of antennas for a solution
MINPATCH   specifies the minimum size allowed for the center of the beam
MINPHSER   specifies the minimum phase error prior to some action
NAXIS      Axis number
NBOXES     Number of boxes
NCCBOX     Number of clean component boxes
NCHAN      Number of spectral channels in each spectral window
NCHAV      Number of channels averaged in an operation
NCOMP      Number of CLEAN components
NCOUNT     General adverb, usually a count of something
NDIG       Number of digits to display
NFIELD     The number of fields imaged
NFILES     The number of files to skip, usually on a tape.
NGAUSS     Number of Gaussians to fit
NIF        Number of IFs (spectral windows) in a data set
NITER      The number of iterations of a procedure
NMAPS      Number of maps (images) in an operation
NOISE      estimates the noise in images, noise level cutoff
NORMALIZ   specifies the type of gain normaliztion if any
NPIECE     The number of pieces to make
NPLOTS     gives number of plots per page or per job
NPOINTS    General adverb giving the number of something
NPRINT     gives number of items to be printed
NTHREAD    Controls number of threads used by multi-threaded processes in OBIT
NUMTELL    selects POPS number of task which is the target of a TELL or ABORT
NX         General adverb referring to a number of things in the Y direction
NY         General adverb referring to a number of things in the Y direction
OBJECT     The name of an object
OBOXFILE   specifies name of output Clean box text file
OFFSET     General adverb, the offset of something.
OFMFILE    specifies the name of a text file containing OFM values
ONEBEAM    specifies whether one beam is made for all facets or one for each
ONEFREQ    states that the current CC model was made with one frequency
OPCODE     General adverb, defines an operation
OPTELL     The operation to be passed to a task by TELL
OPTYPE     General adverb, defines a type of operation.
ORDER      Adverb used usually to specify the order of polynomial fit
OUT2CLAS   The class of a secondary output file
OUT2DISK   The disk number of a secondary output file.
OUT2NAME   The name of a secondary output file.
OUT2SEQ    The sequence of a secondary output file.
OUTCLASS   The class of an output file
OUTDISK    The disk number of an output file.
OUTFGVER   selects version of the flagging table to be written
OUTFILE    specifies name of output disk file, not in regular catalog
OUTNAME    The name of an output file.
OUTPRINT   specifies name of disk file to keep the printer output
OUTSEQ     The sequence of an output file.
OUTTAPE    The output tape drive number.
OUTTEXT    specifies name of output text file, not in regular catalog
OUTVERS    The output version number of an table or extension file.
OVERLAP    specifies how overlaps are to be handled
OVRSWTCH   specifies when IMAGR switches from OVERLAP >= 2 to OVERLAP = 1 mode
PBPARM     Primary beam parameters
PBSIZE     estimates the primary beam size in interferometer images
PCUT       Cutoff in polarized intensity
PDVER      specifies the version of the spetral polarization table to use
PHASPRM    Phase data array, by antenna number.
PHAT       Prussian hat size
PHSLIMIT   gives a phase value in degrees
PIX2VAL    An image value in the units specified in the header.
PIX2XY     Specifies a pixel in an image
PIXAVG     Average image value
PIXRANGE   Range of pixel values to display
PIXSTD     RMS pixel deviation
PIXVAL     Value of a pixel
PIXXY      Specifies a pixel location.
PLCOLORS   specifies the colors to be used
PLEV       Percentage of peak to use for contour levels
PLVER      specifies the version number of a PL extension file
PMODEL     Polarization model parameters
POL3COL    Controls use of full 3-color graphics for polarization lines
POLANGLE   Intrinsic polarization angles for up to 30 sources
POLPLOT    specifies the desired polarization ratio before plotting.
PRIORITY   Limits prioroty of messages printed
PRNUMBER   POPS number of messages
PRSTART    First record number in a print operation
PRTASK     Task name selected for printed information
PRTIME     Time limit
PRTLEV     Specified the amount of information requested.
PRTLIMIT   specifies limits to printing functions
QCREATE    adverb controlling the way large files are created
QUAL       Source qualifier
QUANTIZE   Quantization level to use
RADIUS     Specify a radius in an image
RASHIFT    Shift in RA
REASON      The reason for an operation
REFANT     Reference antenna
REFDATE    To specify the initial or reference date of a data set
REMHOST    gives the name of another computer which will provide service
REMQUE     specifies the desired batch queue on a remote computer
REMTAPE    gives the number of another computer's tape device
RESTFREQ   Rest frequency of a transition
REWEIGHT   Reweighting factors for UV data weights.
RGBCOLOR   specifies the desired TV graphics color
RGBGAMMA   specifies the desired color gamma corrections
RGBLEVS    colors to be applied to the contour levels
RMSLIMIT   selects things with RMS above this limit
ROBUST     Uniform weighting "robustness" parameter
ROMODE     Specified roam mode
ROTATE     Specifies a rotation
RPARM      General numeric array adverb used in modeling
SAMPTYPE   Specifies sampling type
SCALR1     General adverb
SCALR2     General adverb
SCALR3     General adverb
SCANLENG   specify length of "scan"
SCUTOFF    noise level cutoff
SEARCH     Ordered list of antennas for fring searches
SELBAND    Specified bandwidth
SELFREQ    Specified frequency
SHIFT      specifies a position shift
SKEW       Specifies a skew angle
SLOT       Specifies AIPS catalog slot number
SMODEL     Source model
SMOOTH     Specifies spectral smoothing
SMOTYPE    Specifies smoothing
SNCORPRM   Task-specific parameters for SNCOR.
SNCUT      Specifies minimum signal-to-noise ratio
SNVER      specifies the output solution table
SOLCON     Gain solution constraint factor
SOLINT     Solution interval
SOLMIN     Minimum number of solution sub-intervals in a solution
SOLMODE    Solution mode
SOLSUB     Solution sub-interval
SOLTYPE    Solution type
SORT       Specified desired sort order
SOUCODE    Calibrator code for source, not calibrator, selection
SOURCES    A list of source names
SPARM      General string array adverb
SPECINDX   Spectral index used to correct calibrations
SPECPARM   Spectral index per polarization per source
SPECTRAL   Flag to indicate whether an operation is spectral or continuum
SPECURVE   Spectral index survature used to correct calibrations
STFACTOR   scales star display or SDI CLEANing process
STOKES     Stokes parameter
STORE      Store current POPS environment
STRA1      General string adverb
STRA2      General string adverb
STRA3      General string adverb
STRB1      General string adverb
STRB2      General string adverb
STRB3      General string adverb
STRC1      General string adverb
STRC2      General string adverb
STRC3      General string adverb
STVERS     star display table version number
SUBARRAY   Subarray number
SYMBOL     General adverb, probably defines a plotting symbol type
SYS2COM    specifies a command to be sent to the operating system
SYSCOM     specifies a command to be sent to the operating system
SYSOUT     specifies the output device used by the system
SYSVEL     Systemic velocity
TASK       Name of a task
TAU0       Opacities by antenna number
TBLC       Gives the bottom left corner of an image to be displayed
TCODE      Deterimins which type of DDT is RUN.
TDISK      Deterimins where output DDT data is placed
THEDATE    contains the date and time in a string form
TIMERANG   Specifies a timerange
TIMSMO     Specified smoothing times
TMASK      Deterimins which tasks are executed when a DDT is RUN.
TMODE      Deterimins which input is used when a DDT is RUN.
TNAMF      Deterimins which files are input to DDT.
TRANSCOD   Specified desired transposition of an image
TRC        Specified the top right corner of a subimage
TRECVR     Receiver temperatures by polarization and antenna
TRIANGLE   specifies closure triangles to be selected/deselected
TTRC       Specifies the top right corner of a subimage to be displayed
TVBUT      Tells which AIPS TV button was pushed
TVCHAN     Specified a TV channel (plane)
TVCORN     Specified the TV pixel for the bottom left corner of an image
TVLEVS     Gives the peak intensity to be displayed in levels
TVXY       Pixel position on the TV screen
TXINC      TV X coordinate increment
TYINC      TV Y coordinate increment
TYVER      specifies the version of the system temperature table used
TZINC      TV Z coordinate increment
USERID     User number
UVBOX      radius of the smoothing box used for uniform weighting
UVBXFN     type of function used when counting for uniform weighting
UVCOPPRM   Parameter adverb array for task UVCOP
UVFIXPRM   Parameter adverb array for task UVFIX
UVRANGE    Specify range of projected baselines
UVSIZE     specifies number of pixels on X and Y axes of a UV image
UVTAPER    Widths in U and V of gaussian weighting taper function
UVWTFN     Specify weighting function, Uniform or Natural
VCODE      General string adverb
VECTOR     selects method of averaging UV data
VELDEF     Specifies velocity definition
VELTYP     Velocity frame of reference
VERSION    Specify AIPS version or local task area
VLAMODE    VLA observing mode
VLAOBS     Observing program or part of observer's name
VLBINPRM   Control parameters to read data from NRAO/MPI MkII correlators
VNUMBER    Specifies the task parameter (VGET/VPUT) save area
VPARM      General numeric array adverb used in modeling
WEIGHTIT   Controls modification of weights before gain/fringe solutions
WGAUSS     Widths of Gaussian models (FWHM)
WTTHRESH   defines the weight threshold for data acceptance
WTUV       Specifies the weight to use for UV data outside UVRANGE
XAXIS      Which parameter is plotted on the horizontal axis.
X          spare scalar adverb for use in procedures
XINC       increment associated with an array of numbers
XPARM      General adverb for up to 10 parameters, may refer to X coord
XTYPE      Specify type of process, often the X axis type of an image
XYRATIO    Ratio of X to Y units per pixel
Y          spare scalar adverb for use in procedures
YINC       Y axis increment
YPARM      Specifies Y axis convolving function
YTYPE      Y axis (V) convolving function type
ZEROSP     Specify how to include zero spacing fluxes in FT of UV data
ZINC       Set the increment of the third axis
ZXRATIO    Ratio between Z axis (pixel value) and X axis
\end{verbatim}\eve

\sects{ANALYSIS}

\vskip 0.5pt\todx{ABOUT ANALYSIS}\iodx{analysis}
\bbve\begin{verbatim}
ACTNOISE   puts estimate of actual image uncertainty and zero in header
AFARS      Is used after FARS to determine Position and Value of the maximum
AGAUS      Fits 1-dimensional Gaussians to absorption-line spectra
AHIST      Task to convert image intensities by adaptive histogram
AVOPTION   Controls type or range of averaging done by a task
BDEPO      computes depolarization due to rotation measure gradients
BLANK      blanks out selected, e.g. non-signal, portions of an image
BLSUM      sums images over irregular sub-images, displays spectra
BSCOR      Combines two beam-switched images
BSTST      Graphical display of solutions to frequency-switched data
BWSMEAR    amount of bandwidth smearing correction to use
CC2IM      Make model image from a CC file
COMB       combines two images by a variety of mathematical methods
CTYPE      specifies type of component
CUBIT      Model a galaxy's density and velocity distribution from full cube
DFTIM      Makes image of DFT at arbitrary point showing time vs frequency
DOALIGN    specifies how two or more images are aligned in computations
DOFARS     Procedure to aid in Faraday rotation synthesis using the FARS task
DOINVERS   selects opposite of normal function
DOMAX      selects solutions for maxima of models
DOOUTPUT   selects whether output image or whatever is saved / discarded
DOPOS      selects solutions for positions of model components
DOSPIX     selects solutions for spectral index of model components
DOWIDTH    selects solution for widths of model components
ECOUNT     give the highest count or iteration for some process
FARS       Faraday rotation synthesis based on the brightness vs wavelength
FLUX       gives a total intensity value for image/component or to limit
FMAX       specifies peak values of model components - results of fits
FPOS       specifies pixel positions of fit model components
FQUBE      collects n-dimensional images into n+1-dimensional FREQID image
FSHIFT     specifies a position shift - output from fitting routines
FWIDTH     gives widths of model components - results of fitting
GAL        Determine parameters from a velocity field
GMAX       specifies peak values of model components
GPOS       specifies pixel positions of model components
GRBLINK    Verb which blinks 2 TV graphics planes
GWIDTH     gives widths of model components
HGEOM      interpolates image to different gridding and/or geometry
HLPTVHLD   Interactive image display with histogram equalization - run-time help
HLPTVSAD   Find & fit Gaussians to an image with interaction - run-time help
HLPTVSPC   Interactive display of spectra from a cube - run-time help
HOLGR      Read & process holography visibility data to telescope images
HOLOG      Read & process holography visibility data to telescope images
IMCENTER   returns pixel position of sub-image centroid
IMDIST     determines spherical distance between two pixels
IMEAN      displays the mean & extrema and plots histogram of an image
IMERG      merges images of different spatial resolutions
IMFIT      Fits Gaussians to portions of an image
IMLIN      Fits and removes continuum emission from cube
IMMOD      adds images of model objects to an image
IMSTAT     returns statistics of a sub-image
IMVAL      returns image intensity and coordinate at specified pixel
IMVIM      plots one image's values against another's
IRING      integrates intensity / flux in rings / ellipses
JMFIT      Fits Gaussians to portions of an image
LAYER      Task to create an RGB image from multiple images
LGEOM      regrids images with rotation, shift using interpolation
MATHS      operates on an image with a choice of mathematical functions
MAXFIT     returns pixel position and image intensity at a maximum
MCUBE      collects n-dimensional images into n+1-dimensional image
MEDI       combines four images by a variety of mathematical methods
MFITSET    gets adverbs for running IMFIT and JMFIT
MFPRT      prints MF tables in a format needed by modelling software
MINPATCH   specifies the minimum size allowed for the center of the beam
MODAB      Makes simple absorption/emission spectral-line image in I/V
MODIM      adds images of model objects to image cubes in IQU polarization
MODSP      adds images of model objects to image cubes in I/V polarization
MOMFT      calculates images of moments of a sub-image
MOMNT      calculates images of moments along x-axis (vel, freq, ch)
MWFLT      applies linear & non-linear filters to images
NGAUSS     Number of Gaussians to fit
NINER      Applies various 3x3 area operaters to an image.
NNLSQ      Non-Negative-Least-Squares decomposition of spectrum
OMFIT      Fits sources and, optionally, a self-cal model to uv data
OUTTEXT    specifies name of output text file, not in regular catalog
PANEL      Convert HOLOG output to panel adjustment table
PBCOR      Task to apply or correct an image for a primary beam
PRTIM      prints image intensities from an MA catalog entry
QIMVAL     returns image intensity and coordinate at specified pixel
QUXTR      extracts text files from Q,U cubes for input to TARS
RFARS      Correct Q/U cubes for Faraday rotation synthesis results
RM2PL      Plots spectrum of a pixel with RMFIT fit
RMFIT      Fits 1-dimensional polarization spectrum to Q/U cube
RM         Task to calculate rotation measure and magnetic field
RMSD       Calculate rms for each pixel using data at the box around the pixel
RMSLIMIT   selects things with RMS above this limit
SAD        Finds and fits Gaussians to portions of an image
SCLIM      operates on an image with a choice of mathematical functions
SERCH      Finds line signals in transposed data cube
SET1DG     Verb to set 1D gaussian fitting initial guesses.
SHADO      Calculate the shadowing of antennas at the array
SLCOL      Task to collate slice data and models.
SLFIT      Task to fit gaussians to slice data.
SLICE      Task to make a slice file from an image
SMOTH      Task to smooth a subimage from upto a 7-dim. image
SPCOR      Task to correct an image for a primary beam and spectral index
SPFIX      Makes cube from input to and output from SPIXR spectral index
SPIXR      Fits spectral indexes to each row of an image incl curvature
SPMOD      Modify UV database by adding a model with spectral lines
STFUN      Task to calculate a structure function image
STVERS     star display table version number
SUMSQ      Task to sum the squared pixel values of overlapping,
TABGET     returns table entry for specified row, column and subscript.
TABPUT     replaces table entry for specified row, column and subscript.
TARPL      Plot output of TARS task
TARS       Simulation of Faraday rotation synthesis (mainly task FARS)
TK1SET     Verb to reset 1D gaussian fitting initial guess.
TKAGUESS   Verb to re-plot slice model guess directly on TEK
TKAMODEL   Verb to add slice model display directly on TEK
TKASLICE   Verb to add a slice display on TEK from slice file
TKGUESS    Verb to display slice model guess directly on TEK
TKMODEL    Verb to display slice model directly on TEK
TKSET      Verb to set 1D gaussian fitting initial guesses.
TKSLICE    Verb to display slice file directly on TEK
TKVAL      Verb to obtain value under cursor from a slice
TKXY       Verb to obtain pixel value under cursor
TV1SET     Verb to reset 1D gaussian fitting initial guess on TV plot.
TVACOMPS   Verb to add slice model components directly on TV graphics
TVAGUESS   Verb to re-plot slice model guess directly on TV graphics
TVAMODEL   Verb to add slice model display directly on TV graphics
TVARESID   Verb to add slice model residuals directly on TV graphics
TVASLICE   Verb to add a slice display on TV graphics from slice file
TVBLINK    Verb which blinks 2 TV planes, can do enhancement also
TVCOMPS    Verb to display slice model components directly on TV graphics
TVCUBE     Verb to load a cube into tv channel(s) & run a movie
TVDIST     determines spherical distance between two pixels on TV screen
TVGUESS    Verb to display slice model guess directly on TV graphics
TVHLD      Task to load an image to the TV with histogram equalization
TVMAXFIT   displays fit pixel positions and intensity at maxima on TV
TVMODEL    Verb to display slice model directly on TV graphics
TVRESID    Verb to display slice model residuals directly on TV graphics
TVSAD      Finds and fits Gaussians to portions of an image with interaction
TVSET      Verb to set slice Gaussian fitting initial guesses from TV plot
TVSLICE    Verb to display slice file directly on TV
TVSPC      Display images and spectra from a cube
UVADC      Fourier transforms and corrects a model and adds to uv data.
UVCON      Generate sample UV coverage given a user defined array layout
UVFIT      Fits source models to uv data.
UVGIT      Fits source models to uv data.
UVHIM      Makes image of the histogram on two user-chosen axes
UVMOD      Modify UV database by adding a model incl spectral index
UVSEN      Determine RMS sidelobe level and brightness sensitivity
UVSIM      Generate sample UV coverage given a user defined array layout
WARP       Model warps in Galaxies
XBASL      Fits and subtracts nth-order baselines from cube (x axis)
XG2PL      Plots spectrum of a pixel with XGAUS/AGAUS and ZEMAN/ZAMAN fits
XGAUS      Fits 1-dimensional Gaussians to images: restartable
XMOM       Fits one-dimensional moments to each row of an image
ZAMAN      Fits 1-dimensional Zeeman model to absorption-line data
ZEMAN      Fits 1-dimensional Zeeman model to data
\end{verbatim}\eve

\vfill\eject
\sects{AP}

\vskip 0.5pt\todx{ABOUT AP}\todx{AP}\iodx{array processor}
\bbve\begin{verbatim}
APCLN      Deconvolves images with CLEAN algorithm
APGS       deconvolves image with Gerchberg-Saxton algorithm
APVC       Deconvolves images with van Cittert algorithm
BLING      find residual rate and delay on individual baselines
BPASS      computes spectral bandpass correction table
BSGRD      Task to image beam-switched single-dish data
CALIB      determines antenna calibration: complex gain
CCRES      Removes or restores a CC file to a map with a gaussian beam.
CONPL      Plots AIPS gridding convolution functions
CONVL      convolves an image with a gaussian or another image
CPASS      computes polynomial spectral bandpass correction table
EDITA      Interactive TV task to edit uv data based on TY/SY/SN/CL tables
EDITR      Interactive baseline-oriented visibility editor using the TV
FFT        takes Fourier Transform of an image or images
FRCAL      Faraday rotation self calibration task
FRING      fringe fit data to determine antenna calibration, delay, rate
GUARD      portion of UV plane to receive no data in gridding
HLPCLEAN   Cleaning tasks  - run-time help
HLPSCIMG   Full-featured image plus self-cal loops, editing - run-time help
HLPSCMAP   Imaging plus self-cal and editing SCMAP - run-time help
IM2UV      converts an image to a visibility data set
IMAGR      Wide-field and/or wide-frequency Cleaning / imaging task.
IMFRING    large image delay fitting with IM2CC and OOFRING
IMSCAL     large image self-cal with IM2CC and OOCAL
KRING      fringe fit data to determine antenna calibration, delay, rate
MAXPIXEL   maximum pixels searched for components in Clark CLEAN
NOBAT      Task to lock lower priority users out of the AP
OOCAL      determines antenna complex gain with frequency-dependent models
OOFRING    fringe fit data to determine antenna calibration, delay, rate
RLDLY      fringe fit data to determine antenna R-L delay difference
RSTOR      Restores a CC file to a map with a gaussian beam.
SCIMG      Full-featured imaging plus self-calibration loop with editing
SCMAP      Imaging plus self-calibration loop with editing
SDGRD      Task to select and image random-position single-dish data
SDIMG      Task to select and image random-position single-dish data
SNEDT      Interactive SN/CL/TY/SY table editor using the TV
TDEPEND    Time-dependent imaging procedure sequence
TD_SCANS   Time-dependent imaging procedure sequence: find intervals
TD_SSCAN   Time-dependent imaging procedure sequence: find intervals
TD_STEP3   Time-dependent imaging procedure "step 3"
TD_STEP5   Time-dependent imaging procedure sequence: later steps
UVADC      Fourier transforms and corrects a model and adds to uv data.
UVMAP      makes images from calibrated UV data.
VLBABPSS   computes spectral bandpass correction table
WFCLN      Wide field and/or widefrequency  CLEANing/imaging task.
\end{verbatim}\eve

\sects{ASTROMET}

\vskip 0.5pt\todx{ABOUT ASTROMETRY}\iodx{astrometry}
\bbve\begin{verbatim}
ASTROMET   Describes the process of astrometric/geodetic reduction in AIPS
FRMAP      Task to build a map using fringe rate spectra
HF2SV      convert HF tables from FRING/MBDLY to form used by Calc/Solve
HFPRT      write HF tables from CL2HF
XTRAN      Create an image with transformed coordinates
\end{verbatim}\eve
\vfill\eject

\sects{BATCH}

\vskip 0.5pt\todx{ABOUT BATCH}\iodx{batch}
\bbve\begin{verbatim}
Type:  Operations to prepare, submit, and monitor batch jobs
Use:   There are two batch streams of AIPS, each capable of
   processing a queue of jobs.  To run a batch job, one must
   first prepare the text of the job in a work file.  This text
   may contain any normal AIPS/POPS statement including RUN,
   except for verbs and tasks related to batch preparation, the
   TV, the TEK4012 green screen, and the tape drives.  When the
   text is ready, it may be submitted to the batch AIPS.  On
   the way, it is tested for errors and is submitted only if
   none are found.  After successful submission, the work file
   and any RUN files involved may be altered without affecting
   the job.  Array processor tasks are allowed only in queue
   #2 and only at night.  They may be submitted at any time,
   however.  Line printer output should be directed to a user
   chosen file (via adverb OUTPRINT).  If OUTPRINT = ' ', all
   tasks and AIPS itself will write to a file named
   PRTFIL:BATCHjjj.nnn, where jjj is the job number in hex and
   nnn is the user number in hex.  Note that all print jobs
   are concatenated into the specified file(s).
Adverbs:
   BATQUE     Number of queue to be used ( 1 or 2 or more )
   JOBNUM     Job number involved (101 - 164, 201 -264, ...)
   BATFLINE   First line number to be editted or listed
   BATNLINE   Number of lines to be listed
Verbs:
   BATCH     Add text to BATQUE work file
   BATCLEAR  Initiate and clear BATQUE work file
   BATLIST   List BATNLINE starting with BATFLINE from BATQUE
             work file
   BATEDIT   Edit text in BATQUE work file starting with line
             BATFLINE (or immediate argument)
   BAMODIFY  Edit text in BATQUE work file in line BATFLINE (or
             immediate argument), character-mode editing.
   SUBMIT    Submit text in BATQUE work file as job for queue
             BATQUE
   JOBLIST   List BATNLINE starting with BATFLINE from text file
             of job JOBNUM
   QUEUES    List jobs submitted, running, and completed in
             queue BATQUE
   UNQUE     Remove JOBNUM from queue, copy text of job to work
             file BATQUE

Batch jobs may also be prepared and submitted outside of AIPS,
using the program BATER.  See HELP BATER.
****************************************************************

AIPSB      AIPS main program for executing batch jobs
AIPSC      AIPS main program for testing and queuing batch jobs
BAMODIFY   edits characters in a line of a batch work file
BATCH      starts entry of commands into batch-job work file
BATCLEAR   removes all text from a batch work file
BATEDIT    starts an edit (replace, insert) session on a batch work file
BATER      stand-alone program to prepare and submit batch jobs
BATFLINE   specifies starting line in a batch work file
BATLIST    lists the contents of a batch work file
BATNLINE   specifies the number of lines to process in a batch work file
BATQUE     specifies the desired batch queue
ENDBATCH   terminates input to batch work file
JOBLIST    lists contents of a submitted and pending batch job
JOBNUM     specifies the batch job number
QUEUES     Verb to list all submitted jobs in the job queue
REMQUE     specifies the desired batch queue on a remote computer
UNQUE      remove a given job from the job queue
\end{verbatim}\eve

\Sects{CALIBRAT}{aboutcal}

\vskip 0.5pt\todx{ABOUT CALIBRATION}\iodx{calibration}
\bbve\begin{verbatim}
For a lengthy description of the calibration of interferometric
data (VLA and VLB line and continuum) enter:
              HELP CALIBRAT
****************************************************************

ACCOR      Corrects cross amplitudes using auto correlation measurements
ACFIT      Determine antenna gains from autocorrelations
ACLIP      edits suto-corr data for amplitudes, phases, and weights out of range
ACSCL      Corrects cross amplitudes using auto correlation measurements
ANCAL      Places antenna-based Tsys and gain corrections in CL table
ANCHECK    Checks By sign in Antenna files
ANTAB      Read amplitude calibration information into AIPS
ANTENNAS   Antennas to include/exclude from the task or verb
ANTUSE     Antennas to include/exclude from the task or verb
ANTWT      Antenna Weights for UV data correction in Calibration
APCAL      Apply TY and GC tables to generate an SN table
APGPS      Apply GPS-derived ionospheric corrections
ATMCA      Determines delay/phase gradient from calibrator observations
BASELINE   specifies which antenna pairs are to be selected/deselected
BASFIT     fits antenna locations from SN-table data
BDAPL      Applies a BD table to another data set
BLAPP      applies baseline-based fringe solutions a la BLAPP
BLAVG      Average cross-polarized UV data over baselines.
BLCAL      Compute closure offset corrections
BLCHN      Compute closure offset corrections on a channel-by-channel basis
BLING      find residual rate and delay on individual baselines
BLVER      specifies the version of the baseline-calibration table used
BPASS      computes spectral bandpass correction table
BPASSPRM   Control adverb array for bandpass calibration
BPCOR      Correct BP table.
BPEDT      Interactive TV task to edit uv data based on BP tables
BPERR      Print and plot BPASS closure outputs
BPLOT      Plots bandpass tables in 2 dimensions as function of time
BPSMO      Smooths or interpolates bandpass tables to regular times
BPVER      specifies the version of the bandpass table to be applied
BPWAY      Determines channel-dependent relative weights
BPWGT      Calibrates data and scales weights by bandpass correction
BSPRT      print BS tables
BSROT      modifies SD beam-switch continuum data for error in throw
CALCODE    specifies the type of calibrator to be selected
CALDIR     lists calibrator source models available as AIPS FITS files
CALIB      determines antenna calibration: complex gain
CALIBRAT   describes the process of data calibration in AIPS
CALIN      specifies name of input disk file usually with calibration data
CALRD      Reads calibrator source model-image FITS file
CALSOUR    specifies source names to be included in calibration
CC2IM      Make model image from a CC file
CENTR      modifies UV data to center the reference channel
CHANSEL    Array of start, stop, increment channel numbers to average
CLCAL      merges and smooths SN tables, applies them to CL tables
CLCOP      copy CL/SN file calibration between polarizations or IFs
CLCOR      applies user-selected corrections to the calibration CL table
CLCORPRM   Parameter adverb array for task CLCOR
CLINT      CL table entry interval
CLINV      copy CL/SN file inverting the calibration
CLIP       edits data based on amplitudes, phases, and weights out of range
CLSMO      smooths a calibration CL table
CLVLB      Corrects CL table gains for pointing offsets in VLBI data
CMETHOD    specifies the method by which the uv model is computed
CMODEL     specifies the method by which the uv model is computed
CONFI      Optimize array configuration by minimum side lobes
CPASS      computes polynomial spectral bandpass correction table
CSCOR      applies specified corrections to CS tables
CVEL       shifts spectral-line UV data to a given velocity
DCHANSEL   Array of start, stop, increment channel #S + IF to avoid
DECOR      Measures the decorrelation between channels and IF of uv data
DEFLG      edits data based on decorrelation over channels and time
DELCORR    specifies whether VLBA delay corrections  are to be used
DELZN      Determines residual atmosphere depth at zenith and clock errors
DFCOR      applies user-selected corrections to the calibration CL table
DIGICOR    specifies whether VLBA digital corrections are to be applied
DOACOR     specifies whether autocorrelation data are included
DOAPPLY    Flag to indicate whether an operation is applied to the data
DOBAND     specifies if/how bandpass calibration is applied
DOBTWEEN   Controls smoothing between sources in calibration tables
DOCALIB    specifies whether a gain table is to be applied or not
DODELAY    selects solution for phase/amplitude or delay rate/phase
DOFIT      Controls which antennas are fit by what methods
DOFLAG     Controls closure cutoff in gain solutions and flagging
DOOSRO     calibrating amplitude and phase, and imaging VLA data
DOPOL      selects application of any polarization calibration
DOVLAMP    Produces amp calibration file for phased-VLA VLBI data
DTSIM      Generate fake UV data
EDITA      Interactive TV task to edit uv data based on TY/SY/SN/CL tables
EDITR      Interactive baseline-oriented visibility editor using the TV
ELFIT      Plots/fits selected contents of SN, TY, SY, PC or CL files
ELINT      Determines and removes gain dependence on elevation
EVASN      Evaluates statistics in SN/CL tables
EVAUV      Subtracts & divides a model into UV data, does statistics on results
FACES      makes images of catalog sources for initial calibration
FARAD      add ionospheric Faraday rotation to CL table
FGPLT      Plots selected contents of FG table
FGTAB      writes frequency-range flags to a text file to be read by UVFLG
FINDR      Find normal values for a uv data set
FIXWT      Modify weights to reflect amplitude scatter of data
FLAGR      Edit data based on internal RMS, amplitudes, weights
FLAGVER    selects version of the flagging table to be applied
FLGIT      flags data based on the rms of the spectrum
FQTOL      Frequency tolerance with which FQ entries are accepted.
FRCAL      Faraday rotation self calibration task
FREQID     Frequency Identifier for frequency, bandwidth combination
FRING      fringe fit data to determine antenna calibration, delay, rate
FTFLG      interactive flagging of UV data in channel-time using the TV
GAINERR    gives estimate of gain uncertainty for each antenna
GAINUSE    specifies output gain table or gain table applied to data
GAINVER    specifies the input gain table
GCPLT      Plots gain curves from text files
GCVER      specifies the version of the gain curve table used
GETJY      determines calibrator flux densities
GPSDL      Calculate ionospheric delay and Faraday rotation corrections
HLPCLEAN   Cleaning tasks  - run-time help
HLPEDIBP   Interactive BP table uv-data editor BPEDT - run-time help
HLPEDICL   Interactive SN/CL table uv-data editor - run-time help
HLPEDIPC   Interactive PC table editor PCEDT - run-time help
HLPEDIPD   Interactive PD table editor PDEDT - run-time help
HLPEDISN   Interactive SN/CL table (not UV) editor - run-time help
HLPEDISS   Interactive SY table (not UV) editor - run-time help
HLPEDISY   Interactive SY table uv-data editor EDITA - run-time help
HLPEDITS   Interactive TY table (not UV) editor - run-time help
HLPEDITY   Interactive TY table uv-data editor EDITA - run-time help
HLPEDIUV   Interactive uv-data editor EDITR - run-time help
HLPFTFLG   Interactive time-channel visibility Editor - run-time help
HLPIBLED   Interactive Baseline based visibility Editor - run-time help
HLPPCFLG   Interactive time-channel PC table Editor PCFLG - run-time help
HLPSCIMG   Full-featured image plus self-cal loops, editing - run-time help
HLPSCMAP   Imaging plus self-cal and editing SCMAP - run-time help
HLPSPFLG   Interactive time-channel visibility Editor SPFLG - run-time help
HLPTVFLG   Interactive time-baseline visibility Editor TVFLG - run-time help
IBLED      Interactive BaseLine based visibility EDitor
ICHANSEL   Array of start, stop, increment channel #S + IF to average
IM2CC      Task to convert an image to multi-facet Clean Components
IMAGR      Wide-field and/or wide-frequency Cleaning / imaging task.
IMFRING    large image delay fitting with IM2CC and OOFRING
IMSCAL     large image self-cal with IM2CC and OOCAL
INDXH      writes index file describing contents of UV data base
INDXR      writes index file describing contents of UV data base
INTERPOL   specifies the type of averaging done on the complex gains
INTPARM    specifies the parameters of the frequency interpolation function
KRING      fringe fit data to determine antenna calibration, delay, rate
LDGPS      load GPS data from an ASCII file
LISTR      prints contents of UV data sets and assoc. calibration tables
LOCIT      fits antenna locations from SN-table data
LPCAL      Determines instrumental polarization for UV data
MAPBM      Map VLA beam polarization
MBDLY      Fits multiband delays from IF phases, updates SN table
MINAMPER   specifies the minimum amplitude error prior to some action
MINANTEN   states minimum number of antennas for a solution
MINPHSER   specifies the minimum phase error prior to some action
MSORT      Sort a UV dataset into a specified order
MULTI      Task to convert single-source to multi-source UV data
NORMALIZ   specifies the type of gain normaliztion if any
OMFIT      Fits sources and, optionally, a self-cal model to uv data
ONEFREQ    states that the current CC model was made with one frequency
OOCAL      determines antenna complex gain with frequency-dependent models
OOFRING    fringe fit data to determine antenna calibration, delay, rate
OOSRT      Sort a UV dataset into a specified order
OOSUB      Subtracts/divides a model from/into a uv data base
PBEAM      Fits the analytic function to the measured values of the beam
PCAL       Determines instrumental polarization for UV data
PCASS      Finds amplitude bandpass shape from pulse-cal table data
PCAVG      Averages pulse-cal (PC) tables over time
PCCOR      Corrects phases using  PCAL tones data from PC table
PCEDT      Interactive TV task to edit pulse-cal (PC) tables
PCFIT      Finds delays and phases using a pulse-cal (PC) table
PCFLG      interactive flagging of Pulse-cal data in channel-TB using the TV
PCLOD      Reads ascii file containing pulse-cal info to PC table.
PCPLT      Plots pulse-cal tables in 2 dimensions as function of time
PCRMS      Finds statistics of a pulse-cal table; flags bad times and channels
PCVEL      shifts spectral-line UV data to a given velocity: planet version
PDEDT      Interactive TV task to edit polarization D-term (PD) tables
PDVER      specifies the version of the spetral polarization table to use
PEEK       fits pointing model function to output from the VLA
PEELR      calibrates interfering sources in multi-facet imges
PHASPRM    Phase data array, by antenna number.
PIPEAIPS   calibrating amplitude and phase, and imaging VLA data
POLANGLE   Intrinsic polarization angles for up to 30 sources
POLSN      Make a SN table from cross polarized fringe fit
PRTSY      Task to print statistics from the SY table
QUFIX      determines Right minus Left phase difference, corrects cal files
QUOUT      writes text file of Q, U versus frequency to be used by RLDIF
REAMP      modifies UV data re-scaling the amplitudes
REFANT     Reference antenna
RESEQ      Renumber antennas
REWGT      modifies UV data re-scaling the weights only
RFI        Look for RFI in uv data
RLCAL      Determines instrumental right-left phase versus time (a self-cal)
RLDIF      determines Right minus Left phase difference, corrects cal files
RLDLY      fringe fit data to determine antenna R-L delay difference
SCIMG      Full-featured imaging plus self-calibration loop with editing
SCMAP      Imaging plus self-calibration loop with editing
SDCAL      Task to apply single dish calibration
SDVEL      shifts spectral-line single-dish data to a given velocity
SEARCH     Ordered list of antennas for fring searches
SELBAND    Specified bandwidth
SELFREQ    Specified frequency
SETJY      Task to enter source info into source (SU) table.
SHOUV      displays uv data in various ways.
SMODEL     Source model
SMOTYPE    Specifies smoothing
SNCOR      applies user-selected corrections to the calibration SN table
SNCORPRM   Task-specific parameters for SNCOR.
SNDUP      copies and duplicates SN table from single pol file to dual pol
SNEDT      Interactive SN/CL/TY/SY table editor using the TV
SNFIT      Fits parabola to SN amplitudes and plots result
SNFLG      Writes flagging info based on the contents of SN files
SNIFS      Plots selected contents of SN, TY, SY, PC or CL files
SNP2D      Task to convert SN table single-channel phase to delay
SNPLT      Plots selected contents of SN, SY, TY, PC or CL files
SNREF      Chooses best reference antenna to minimize R-L differences
SNSMO      smooths and filters a calibration SN table
SNVER      specifies the output solution table
SOLCL      adjust gains for solar data according to nominal sensitivity
SOLCON     Gain solution constraint factor
SOLINT     Solution interval
SOLMIN     Minimum number of solution sub-intervals in a solution
SOLMODE    Solution mode
SOLSUB     Solution sub-interval
SOLTYPE    Solution type
SOUCODE    Calibrator code for source, not calibrator, selection
SOUSP      fits source spectral index from SU table or adverbs
SPCAL      Determines instrumental polzn. for spec. line UV data
SPECINDX   Spectral index used to correct calibrations
SPECPARM   Spectral index per polarization per source
SPECTRAL   Flag to indicate whether an operation is spectral or continuum
SPECURVE   Spectral index survature used to correct calibrations
SPFLG      interactive flagging of UV data in channel-TB using the TV
SPLAT      Applies calibration and splits or assemble selected sources.
SPLIT      converts multi-source to single-source UV files w calibration
SY2TY      Task to generate a TY extension file from an EVLA SY table
SYSOL      undoes and re-does nominal sensitivity application for Solar data
TASAV      Task to copy all extension tables to a dummy uv or map file
TAU0       Opacities by antenna number
TDEPEND    Time-dependent imaging procedure sequence
TECOR      Calculate ionospheric delay and Faraday rotation corrections
TIMSMO     Specified smoothing times
TLCAL      Converts JVLA telcal files to an SN file
TRECVR     Receiver temperatures by polarization and antenna
TRIANGLE   specifies closure triangles to be selected/deselected
TRUEP      determines true antenna polarization from special data sets
TVFLG      interactive flagging of UV data using the TV
TYAPL      undoes and re-does nominal sensitivity application
TYCOP      copy TY or SY table calibration between IFs
TYSMO      smooths and filters a calibration TY or SY table
TYVER      specifies the version of the system temperature table used
UNCAL      sets up tables for uncalibrating Australia Telescope data
USUBA      Assign subarrays within a uv-data file
UVCRS      Finds the crossing points of UV-ellipses.
UVFIT      Fits source models to uv data.
UVFLG      Flags UV-data
UVGIT      Fits source models to uv data.
UVHOL      prints holography data from a UV data base with calibration
UVMLN      edits data based on the rms of the spectrum
UVPRT      prints data from a UV data base with calibration
UVSRT      Sort a UV dataset into a specified order
UVSUB      Subtracts/divides a model from/into a uv data base
VBCAL      Scale visibility amplitudes by antenna based constants
VLABP      VLA antenna beam polarization correction for snapshot images
VLACALIB   Runs CALIB and LISTR for VLA observation
VLACLCAL   Runs CLCAL and prints the results with LISTR
VLALIST    Runs LISTR for VLA observation
VLAMODE    VLA observing mode
VLAMP      Makes ANTAB file for phased VLA used in VLBI observations
VLANT      applies VLA/EVLA antenna position corrections from OPs files
VLAOBS     Observing program or part of observer's name
VLAPROCS   Procedures to simplify the reduction of VLBA data
VLARESET   Reset calibration tables to a virginal state
VLARUN     calibrating amplitude and phase, and imaging VLA data
VLBAAMP    applies a-priori amplitude corrections to VLBA data
VLBABPSS   computes spectral bandpass correction table
VLBACALA   applies a-priori amplitude corrections to VLBA data
VLBACCOR   applies a-priori amplitude corrections to VLBA data
VLBACPOL   Procedure to calibrate cross-polarization delays
VLBAEOPS   Corrects Earth orientation parameters
VLBAFQS    Copies different FQIDS to separate files
VLBAFRGP   Fringe fit phase referenced data and apply calibration
VLBAFRNG   Fringe fit data and apply calibration
VLBAKRGP   Fringe fit phase referenced data and apply calibration
VLBAKRNG   Fringe fit data and apply calibration
VLBAMCAL   Merges redundant calibration data
VLBAMPCL   Calculates and applies manual instrumental phase calibration
VLBAPANG   Corrects for parallactic angle
VLBAPCOR   Calculates and applies instrumental phase calibration
VLBAPIPE   applies amplitude and phase calibration procs to VLBA data
VLBARUN    applies amplitude and phase calibration procs to VLBA data
VLBATECR   Calculate ionospheric delay and Faraday rotation corrections
VLBAUTIL   Procedures to simplify the reduction of VLBA data
VLOG       Pre-process external VLBA calibration files
WEIGHTIT   Controls modification of weights before gain/fringe solutions
WETHR      Plots selected contents of WX tables, flags data based on WX
WRTPROCS   Procedures to simplify the reduction of VLBA data
WTTHRESH   defines the weight threshold for data acceptance
WTUV       Specifies the weight to use for UV data outside UVRANGE
XYDIF      find/apply X minus Y linear polarization phase difference
\end{verbatim}\eve

\sects{CATALOG}

\vskip 0.5pt\todx{ABOUT CATALOG}\iodx{catalog file}
\bbve\begin{verbatim}
ABACKUP    VMS procedure to back up data on tape
ACTNOISE   puts estimate of actual image uncertainty and zero in header
ADDBEAM    Inserts clean beam parameters in image header
ADDDISK    makes a computer's disks available to the current AIPS session
ALLDEST    Delete a group or all of a users data files
ALTDEF     Sets frequency vs velocity relationship into image header
ALTSWTCH   Switches between frequency and velocity in image header
ARESTORE   Restores back up tapes of users data
AX2REF     Second reference pixel number
AXDEFINE   Define or modify an image axis description
AXINC      Axis increment - change in coordinate between pixels
AXREF      Reference pixel number
AXTYPE     Type of coordinate axis
AXVAL      Value of axis coordinate at reference pixel
BAKLD      reads all files of a catalog entry from BAKTP tape
BAKTP      writes all files of a catalog entry to tape in host format
CATALOG    list one or more entries in the user's data directory
CATNO      Specifies AIPS catalog slot number range
CELGAL     switches header between celestial and galactic coordinates
CHKNAME    Checks for existence of the specified image name
CLR2NAME   clears adverbs specifying the second input image
CLR3NAME   clears adverbs specifying the third input image
CLR4NAME   clears adverbs specifying the fourth input image
CLR5NAME   clears adverbs specifying the fourth input image
CLRNAME    clears adverbs specifying the first input image
CLRONAME   clears adverbs specifying the first output image
CLRSTAT    remove any read or write status flags on a directory entry
COODEFIN   Define or modify an image axis coordinate description
COOINC     Celestial axes increment: change in coordinate between pixels
COOREF     Reference pixel number for two coordinate axes
COOTYPE    Celestial axes projection type
DAYNUMBR   finds day nuumber of an image or uv data set
DISKU      shows disk use by one or all users
DOALPHA    specifies whether some list is alphabetized
DOCAT      specifies whether the output is saved (cataloged) or not
DOOUTPUT   selects whether output image or whatever is saved / discarded
EGETHEAD   returns parameter value from image header and error code
EGETNAME   fills in input name adverbs by catalog slot number, w error
EPOSWTCH   Switches between B1950 and J2000 coordinates in header
ERROR      was there an error
EXTDEST    deletes one or more extension files
EXTLIST    lists detailed information about contents of extension files
GET2NAME   fills 2nd input image name parameters by catalog slot number
GET3NAME   fills 3rd input image name parameters by catalog slot number
GET4NAME   fills 4th input image name parameters by catalog slot number
GET5NAME   fills 5th input image name parameters by catalog slot number
GETHEAD    returns parameter value from image header
GETNAME    fills 1st input image name parameters by catalog slot number
GETONAME   fills 1st output image name parameters by catalog slot number
GETVERS    finds maximum version number of an extension file
HGEOM      interpolates image to different gridding and/or geometry
HIEND      End record number in a history-file operation
HINOTE     adds user-generated lines to the history extension file
HISTART    Start record number in a history-file operation
HITEXT     writes lines from history extension file to text file
IM2HEAD    displays the image 2 header contents to terminal, message file
IM3HEAD    displays the image 3 header contents to terminal, message file
IM4HEAD    displays the image 4 header contents to terminal, message file
IM5HEAD    displays the image 5 header contents to terminal, message file
IMDIST     determines spherical distance between two pixels
IMHEADER   displays the image header contents to terminal, message file
IMOHEAD    displays the output image header contents
IMPOS      displays celestial coordinates selected by the TV cursor
IMVAL      returns image intensity and coordinate at specified pixel
IN2CLASS   specifies the "class" of the 2nd input image or data base
IN2DISK    specifies the disk drive of the 2nd input image or data base
IN2EXT     specifies the type of the 2nd input extension file
IN2NAME    specifies the "name" of the 2nd input image or data base
IN2SEQ     specifies the sequence # of the 2nd input image or data base
IN2TYPE    specifies the type of the 2nd input image or data base
IN2VERS    specifies the version number of the 2nd input extension file
IN3CLASS   specifies the "class" of the 3rd input image or data base
IN3DISK    specifies the disk drive of the 3rd input image or data base
IN3EXT     specifies the type of the 3rd input extension file
IN3NAME    specifies the "name" of the 3rd input image or data base
IN3SEQ     specifies the sequence # of the 3rd input image or data base
IN3TYPE    specifies the type of the 3rd input image or data base
IN3VERS    specifies the version number of the 3rd input extension file
IN4CLASS   specifies the "class" of the 4th input image or data base
IN4DISK    specifies the disk drive of the 4th input image or data base
IN4NAME    specifies the "name" of the 4th input image or data base
IN4SEQ     specifies the sequence # of the 4th input image or data base
IN4TYPE    specifies the type of the 4th input image or data base
IN5CLASS   specifies the "class" of the 5th input image or data base
IN5DISK    specifies the disk drive of the 5th input image or data base
IN5NAME    specifies the "name" of the 5th input image or data base
IN5SEQ     specifies the sequence # of the 5th input image or data base
IN5TYPE    specifies the type of the 5th input image or data base
INCLASS    specifies the "class" of the 1st input image or data base
INDISK     specifies the disk drive of the 1st input image or data base
INEXT      specifies the type of the 1st input extension file
INNAME     specifies the "name" of the 1st input image or data base
INSEQ      specifies the sequence # of the 1st input image or data base
INTYPE     specifies the type of the 1st input image or data base
INVERS     specifies the version number of the 1st input extension file
KEYSTRNG   gives contents of character-valued keyword parameter
KEYTYPE    Adverb giving the keyword data type code
KEYVALUE   gives contents of numeric-valued keyword parameter
KEYWORD    gives name of keyword parameter - i.e. name of header field
LGEOM      regrids images with rotation, shift using interpolation
M2CAT      displays images in the user's catalog directory for IN2DISK
M3CAT      displays images in the user's catalog directory for IN3DISK
M4CAT      displays images in the user's catalog directory for IN4DISK
M5CAT      displays images in the user's catalog directory for IN5DISK
MCAT       lists images in the user's catalog directory on disk INDISK
MOCAT      displays images in the user's catalog directory for OUTDISK
MOVE       Task to copy or move data from one user or disk to another
NAMEGET    fills 1st input image name parameters by default matching
OUT2CLAS   The class of a secondary output file
OUT2DISK   The disk number of a secondary output file.
OUT2NAME   The name of a secondary output file.
OUT2SEQ    The sequence of a secondary output file.
OUTCLASS   The class of an output file
OUTDISK    The disk number of an output file.
OUTNAME    The name of an output file.
OUTSEQ     The sequence of an output file.
OUTVERS    The output version number of an table or extension file.
PCAT       Verb to list entries in the user's catalog (no log file).
PLVER      specifies the version number of a PL extension file
PRTHI      prints selected contents of the history extension file
PUTHEAD    Verb to modify image header parameters.
Q2HEADER   Verb to summarize the image 2 header: positions at center
Q3HEADER   Verb to summarize the image 3 header: positions at center
Q4HEADER   Verb to summarize the image 4 header: positions at center
Q5HEADER   Verb to summarize the image 5 header: positions at center
QGETVERS   finds maximum version number of an extension file quietly
QHEADER    Verb to summarize the image header: positions at center
QOHEADER   Verb to summarize the output image header: center positions
QUAL       Source qualifier
REASON      The reason for an operation
RECAT      Verb to compress the entries in a catalog file
REMDISK    removes a computer's disks from the current AIPS session
RENAME     Rename a file (UV or Image)
RENUMBER   Verb to change the catalog number of an image.
RESCALE    Verb to modify image scale factor and offset
REVERSN    checks disk for presence of extension files
SCAT       lists scratch files in the user's catalog directory on all disks
SCRDEST    Verb to destroy scratch files left by bombed tasks.
SLOT       Specifies AIPS catalog slot number
STALIN     revises history by deleting lines from history extension file
TVDIST     determines spherical distance between two pixels on TV screen
U2CAT      list a user's UV and scratch files on disk IN2DISK
U3CAT      list a user's UV and scratch files on disk IN3DISK
U4CAT      list a user's UV and scratch files on disk IN4DISK
U5CAT      list a user's UV and scratch files on disk IN5DISK
UCAT       list a user's UV and scratch files on disk INDISK
UOCAT      list a user's UV and scratch files on disk OUTDISK
USERID     User number
ZAP        Delete a catalog entry and its extension files
\end{verbatim}\eve
\vfill\eject

\sects{COORDINA}

\vskip 0.5pt\todx{ABOUT COORDINATES}\iodx{coordinates}
\bbve\begin{verbatim}
ALTDEF     Sets frequency vs velocity relationship into image header
ALTSWTCH   Switches between frequency and velocity in image header
CODECIML   Convert between decimal and sexagesimal coordinate values
COODEFIN   Define or modify an image axis coordinate description
COORDINA   Array to hold coordinate values
COPIXEL    Convert between physical and pixel coordinate values
COSTAR     Verb to plot a symbol at given position on top of a TV image
COTVLOD    Proc to load an image into a TV channel about a coordinate
COWINDOW   Set a window based on coordinates
EPOCONV    Convert between J2000 and B1950 coordinates
EPOSWTCH   Switches between B1950 and J2000 coordinates in header
FRMAP      Task to build a map using fringe rate spectra
FSHIFT     specifies a position shift - output from fitting routines
NAXIS      Axis number
OBEDT      Task to flag data of orbiting antennas
OBTAB      Recalculate orbit parameters and other spacecraft info
PIX2VAL    An image value in the units specified in the header.
PIX2XY     Specifies a pixel in an image
RASHIFT    Shift in RA
REGRD      Regrids an image from one co-ordinate frame to another
RESTFREQ   Rest frequency of a transition
ROTATE     Specifies a rotation
SHIFT      specifies a position shift
SKYVE      Regrids a DSS image from one co-ordinate frame to another
SYSVEL     Systemic velocity
VELDEF     Specifies velocity definition
VELTYP     Velocity frame of reference
XINC       increment associated with an array of numbers
XPARM      General adverb for up to 10 parameters, may refer to X coord
XTRAN      Create an image with transformed coordinates
XTYPE      Specify type of process, often the X axis type of an image
XYRATIO    Ratio of X to Y units per pixel
ZINC       Set the increment of the third axis
ZXRATIO    Ratio between Z axis (pixel value) and X axis
\end{verbatim}\eve

\sects{EDITING}

\vskip 0.5pt\todx{ABOUT EDITING}\iodx{editing}
\bbve\begin{verbatim}
ACLIP      edits suto-corr data for amplitudes, phases, and weights out of range
BPEDT      Interactive TV task to edit uv data based on BP tables
CLIP       edits data based on amplitudes, phases, and weights out of range
CROWDED    allows a task to perform its function in a crowded fashion
DEFLG      edits data based on decorrelation over channels and time
DOFLAG     Controls closure cutoff in gain solutions and flagging
DOPLOT     Controls plotting of something
DOROBUST   Controls method of averaging - simple mean/rms or robust
EDITA      Interactive TV task to edit uv data based on TY/SY/SN/CL tables
EDITR      Interactive baseline-oriented visibility editor using the TV
EXPERT     specifies an user experience level or mode
FGCNT      Counts samples comparing two flag tables
FGDIF      Compares affect of 2 FG tables
FGSPW      Flags bad spectral windows
FGTAB      writes frequency-range flags to a text file to be read by UVFLG
FINDR      Find normal values for a uv data set
FLAGR      Edit data based on internal RMS, amplitudes, weights
FLGIT      flags data based on the rms of the spectrum
FTFLG      interactive flagging of UV data in channel-time using the TV
HLPEDIBP   Interactive BP table uv-data editor BPEDT - run-time help
HLPEDICL   Interactive SN/CL table uv-data editor - run-time help
HLPEDIPC   Interactive PC table editor PCEDT - run-time help
HLPEDIPD   Interactive PD table editor PDEDT - run-time help
HLPEDISN   Interactive SN/CL table (not UV) editor - run-time help
HLPEDISS   Interactive SY table (not UV) editor - run-time help
HLPEDISY   Interactive SY table uv-data editor EDITA - run-time help
HLPEDITS   Interactive TY table (not UV) editor - run-time help
HLPEDITY   Interactive TY table uv-data editor EDITA - run-time help
HLPEDIUV   Interactive uv-data editor EDITR - run-time help
HLPFTFLG   Interactive time-channel visibility Editor - run-time help
HLPUFLAG   Edit uv-data on a grid UFLAG - run-time help
HLPWIPER   edit uv data from UVPLT-like plot WIPER - run-time help
IBLED      Interactive BaseLine based visibility EDitor
ISCALIB    states that the current source is a point-source calibrator
NOISE      estimates the noise in images, noise level cutoff
OFLAG      uses on-line flag table information to write a flag table
OUTFGVER   selects version of the flagging table to be written
PCEDT      Interactive TV task to edit pulse-cal (PC) tables
PCFLG      interactive flagging of Pulse-cal data in channel-TB using the TV
PDEDT      Interactive TV task to edit polarization D-term (PD) tables
REFLG      Attempts to compress a flag table
RFI        Look for RFI in uv data
RFLAG      Flags data set based on time and freq rms in fringe visibilities
SCANLENG   specify length of "scan"
SCIMG      Full-featured imaging plus self-calibration loop with editing
SCMAP      Imaging plus self-calibration loop with editing
SCUTOFF    noise level cutoff
SDLSF      least squares fit to channels and subtracts from SD uv data
SNEDT      Interactive SN/CL/TY/SY table editor using the TV
SNFLG      Writes flagging info based on the contents of SN files
SPFLG      interactive flagging of UV data in channel-TB using the TV
TABED      Task to edit tables
TAFLG      Flags data in a Table extension file
TVFLG      interactive flagging of UV data using the TV
UFLAG      Plots and edits data using a uv-plane grid and the TV
UVFLG      Flags UV-data
UVFND      prints selected data from UV data set to search for problems
UVLIN      Fits and removes continuum visibility spectrum, also can flag
UVLSD      least squares fit to channels and divides the uv data.
UVLSF      least squares fit to channels and subtracts from uv data.
UVMLN      edits data based on the rms of the spectrum
VPFLG      Resets flagging to all or all corss-hand whenever some are flagged
WETHR      Plots selected contents of WX tables, flags data based on WX
WIPER      plots and edits data from a UV data base using the TV
\end{verbatim}\eve

%\vfill\eject
\sects{\hspace{0.5em}EXT-APPL}

\vskip 0.5pt\todx{ABOUT EXTENSION APPLICATIONS}\iodx{extension files}
\bbve\begin{verbatim}
BSPRT      print BS tables
GETTHEAD   returns keyword and other values value from a table header
MF2ST      Task to generate an ST ext. file from Model Fit ext. file
PRTSY      Task to print statistics from the SY table
PUTTHEAD   inserts a given value into a table keyword/value pair
SY2TY      Task to generate a TY extension file from an EVLA SY table
TABGET     returns table entry for specified row, column and subscript.
TABPUT     replaces table entry for specified row, column and subscript.
\end{verbatim}\eve

\sects{\hspace{0.5em}FITS}

\vskip 0.5pt\todx{ABOUT FITS}\iodx{FITS}
\bbve\begin{verbatim}
CALRD      Reads calibrator source model-image FITS file
CALWR      writes calibrator source model images w CC files to FITS disk files
DATAIN     specifies name of input FITS disk file
DATAOUT    specifies name of output FITS disk file
FILEZAP    Delete an external file
FIT2A      reads the fits input file and records it to the output ascii file
FITAB      writes images / uv data w extensions to tape in FITS format
FITDISK    writes images / uv data w extensions to disk in FITS format
FITLD      Reads FITS files to load images or UV (IDI or UVFITS) data to disk
FITTP      writes images / uv data w extensions to tape in FITS format
IMLOD      reads tape to load images to disk
READISK    writes images / uv data w extensions to tape in FITS format
TCOPY      Tape to tape copy with some disk FITS support
UVLOD      Read export or FITS data from a tape or disk
VLBALOAD   Loads VLBA data
WRTDISK    writes images / uv data w extensions to tape in FITS format
WRTPROCS   Procedures to simplify the reduction of VLBA data
\end{verbatim}\eve

\sects{\hspace{0.5em}GENERAL}

\vskip 0.5pt\todx{ABOUT GENERAL}\iodx{general information}
\bbve\begin{verbatim}
ABORTASK   stops a running task
ABOUT      displays lists and information on tasks, verbs, adverbs
AIPSB      AIPS main program for executing batch jobs
AIPS       AIPS main program for interactive use
ANTNAME    A list of antenna (station) names
APARM      General numeric array adverb used many places
BADDISK    specifies which disks are to be avoided for scratch files
BATCHJOB   Information about BATCH
BCOUNT     gives beginning location for start of a process
BITER      gives beginning point for some iterative process
BLC        gives lower-left-corner of selected subimage
BPARM      general numeric array adverb used too many places
CATEGORY   List of allowed primary keywords in HELP files
CLRMSG     deletes messages from the user's message file
COMMENT    64-character comment string
CPARM      general numeric array adverb used many places
CPUTIME    displays curren tcpu and real time usage of the AIPS task
CROWDED    allows a task to perform its function in a crowded fashion
DCODE      General string adverb
DDISK      Deterimins where input DDT data is found
DDT        verifies correctness and performance using standard problems
DDTSAVE    verifies correctness and performance using standard problems
DDTSIZE    Deterimins which type of DDT is RUN.
DECIMAL    specifies if something is in decimal format
DETIME     specifies a time interval for an operation (destroy, batch)
DISKU      shows disk use by one or all users
DOALL      specifies if an operation is done once or for all matching
DOCONFRM   selects user confirmation modes of repetitive operation
DOKEEP     specifies if something is kept or deleted
DOSCALE    specifies if a scaling operation of some sort is to be performed
DOSCAN     specifies if a scan-related operation is to be done
DOWAIT     selects wait-for-completion mode for running tasks
DOWEIGHT   selects operations with data weights
DPARM      General numeric array adverb used many places
DRCHK      stand-alone program checks system setup files for consistency
ECOUNT     give the highest count or iteration for some process
EDGSKP     Deterimins border excluded from comparision or use
EHEX       converts decimal to extended hex
EXPERT     specifies an user experience level or mode
EXPLAIN    displays help + extended information describing a task/symbol
FILEZAP    Delete an external file
FPARM      General numeric array adverb used in modeling
FREESPAC   displays available disk space for AIPS in local system
GETDATE    Convert the current date and time to a string
GET        restores previously SAVEd full POPS environment
GNUGPL     Information about GNU General Public License for AIPS
GO         starts a task, detaching it from AIPS or AIPSB
GRADDRES   specifies user's home address for replies to gripes
GRDROP     deletes the specified gripe entry
GREMAIL    gives user's e-mail address name for reply to gripe entry
GRINDEX    lists users and time of all gripe entries
GRIPE      enter a suggestion or bug report for the AIPS programmers
GRIPR      standalone program to enter suggestions/complaints to AIPS
GRLIST     lists contents of specified gripe entry
GRNAME     gives user's name for reply to gripe entry
GRPHONE    specifies phone number to call for questions about a gripe
HELP       displays information on tasks, verbs, adverbs
HINOTE     adds user-generated lines to the history extension file
HITEXT     writes lines from history extension file to text file
IN2FILE    specifies name of a disk file, outside the regular catalog
INFILE     specifies name of a disk file, outside the regular catalog
INTEXT     specifies name of input text file, not in regular catalog
IOTAPE     Deterimins which tape drive is used during a DDT RUN
LSAPROPO   Data input to APROPO to find what uses what words
MAPDIF     Records differences between DDT test results and standards
MDISK      Deterimins where input DDT data is found
MSGKILL    turns on/off the recording of messages in the message file
MSGSERVER  Information about the X11-based message server
MSGSRV     Information about the X11-based message server
NBOXES     Number of boxes
NCCBOX     Number of clean component boxes
NCOUNT     General adverb, usually a count of something
NITER      The number of iterations of a procedure
NPOINTS    General adverb giving the number of something
NTHREAD    Controls number of threads used by multi-threaded processes in OBIT
OBJECT     The name of an object
OFFSET     General adverb, the offset of something.
OPCODE     General adverb, defines an operation
OPTELL     The operation to be passed to a task by TELL
OPTYPE     General adverb, defines a type of operation.
ORDER      Adverb used usually to specify the order of polynomial fit
OTFBS      Translates on-the-fly continuum SDD format to AIPS UV file
OTFUV      Translates on-the-fly single-dish SDD format to AIPS UV file
OUTFILE    specifies name of output disk file, not in regular catalog
OUTTEXT    specifies name of output text file, not in regular catalog
OUTVERS    The output version number of an table or extension file.
PANIC      Instructions for what to do when things go wrong
PIX2VAL    An image value in the units specified in the header.
PIXRANGE   Range of pixel values to display
PIXVAL     Value of a pixel
POSTSCRIP  General comments about AIPS use of PostScript incl macros
PRTAC      prints contents and summaries of the accounting file
PRTASK     Task name selected for printed information
PRTHI      prints selected contents of the history extension file
PRTLIMIT   specifies limits to printing functions
PRTMSG     prints selected contents of the user's message file
QUAL       Source qualifier
READLINE   Information about AIPS use of the GNU readline library.
REASON      The reason for an operation
REBYTE     service program to transform byte order of full data sets
REHEX      converts extended hex string to decimal
ROTATE     Specifies a rotation
RPARM      General numeric array adverb used in modeling
RTIME      Task to test compute times
SCALR1     General adverb
SCALR2     General adverb
SCALR3     General adverb
SECONDARY  List of allowed secondary keywords in HELP files
SECONDRY   List of allowed secondary keywords in HELP files
SOURCES    A list of source names
SPARM      General string array adverb
STALIN     revises history by deleting lines from history extension file
STRA1      General string adverb
STRA2      General string adverb
STRA3      General string adverb
STRB1      General string adverb
STRB2      General string adverb
STRB3      General string adverb
STRC1      General string adverb
STRC2      General string adverb
STRC3      General string adverb
SUBARRAY   Subarray number
SYMBOL     General adverb, probably defines a plotting symbol type
SYS2COM    specifies a command to be sent to the operating system
SYSCOM     specifies a command to be sent to the operating system
SYSOUT     specifies the output device used by the system
SYSTEM     Verb to send a command to the operating system
TCODE      Deterimins which type of DDT is RUN.
TDISK      Deterimins where output DDT data is placed
TELL       Send parameters to tasks that know to read them on the fly
THEDATE    contains the date and time in a string form
TIMERANG   Specifies a timerange
TMASK      Deterimins which tasks are executed when a DDT is RUN.
TMODE      Deterimins which input is used when a DDT is RUN.
TNAMF      Deterimins which files are input to DDT.
TPMON      Information about the TPMON "Daemon"
VCODE      General string adverb
VLAC       verifies correctness of continuum calibration software
VLACSAVE   verifies correctness of continuum calibration
VLAL       verifies correctness of spectral line calibration software
VLALSAVE   verifies correctness of continuum calibration
VLBDDT     Verification tests using simulated data
VPARM      General numeric array adverb used in modeling
WHATSNEW   lists changes and new code in the last several AIPS releases
XHELP      Accesses hypertext help system
XPARM      General adverb for up to 10 parameters, may refer to X coord
XTYPE      Specify type of process, often the X axis type of an image
Y2K        verifies correctness and performance using standard problems
Y2KSAVE    verifies correctness and performance using standard problems
YINC       Y axis increment
YTYPE      Y axis (V) convolving function type
\end{verbatim}\eve

\sects{\hspace{0.5em}HARDCOPY}

\vskip 0.5pt\todx{ABOUT HARDCOPY}\iodx{hard copy}
\bbve\begin{verbatim}
BPRINT     gives beginning location for start of a printing process
BSPRT      print BS tables
EPRINT     gives location for end of a printing process
FACTOR     scales some display or CLEANing process
FLMCOMM    Comment for film recorder image.
HIEND      End record number in a history-file operation
HISTART    Start record number in a history-file operation
INTEXT     specifies name of input text file, not in regular catalog
ISPEC      Plots and prints spectrum of region of a cube
NPLOTS     gives number of plots per page or per job
NPRINT     gives number of items to be printed
OTFIN      Lists on-the-fly single-dish SDD format data files
OUTFILE    specifies name of output disk file, not in regular catalog
OUTPRINT   specifies name of disk file to keep the printer output
OUTTEXT    specifies name of output text file, not in regular catalog
POSTSCRIP  General comments about AIPS use of PostScript incl macros
PRINTER    Verb to set or show the printer(s) used
PRIORITY   Limits prioroty of messages printed
PRNUMBER   POPS number of messages
PRSTART    First record number in a print operation
PRTASK     Task name selected for printed information
PRTIME     Time limit
PRTLEV     Specified the amount of information requested.
PRTSD      prints contents of AIPS single-dish data sets
PRTUV      prints contents of a visibility (UV) data set
RGBGAMMA   specifies the desired color gamma corrections
RSPEC      Plots and prints spectrum of rms of a cube
TVCPS      Task to copy a TV screen-image to a PostScript file.
TVDIC      Task to copy a TV screen-image to a Dicomed film recorder.
UVFND      prints selected data from UV data set to search for problems
UVHOL      prints holography data from a UV data base with calibration
UVPRT      prints data from a UV data base with calibration
\end{verbatim}\eve

\sects{\hspace{0.5em}IMAGE-UT}

\vskip 0.5pt\todx{ABOUT IMAGE-UTILITIES}\iodx{image}
\bbve\begin{verbatim}
BDROP      gives number of pooints dropped at the beginning
BSGEO      Beam-switched Az-El image to RA-Dec image translation
CALWR      writes calibrator source model images w CC files to FITS disk files
DOMODEL    selects display of model function
DORESID    selects display of differences between model and data
DOSLICE    selects display of slice data
DSKEW      Geometric interpolation correction for skew
EDROP      number of points/iterations to be omitted from end of process
FIT2A      reads the fits input file and records it to the output ascii file
FITAB      writes images / uv data w extensions to tape in FITS format
FITDISK    writes images / uv data w extensions to disk in FITS format
FITTP      writes images / uv data w extensions to tape in FITS format
HLPTVHUI   Interactive intensity-hue-saturation display - run-time help
HLPTVRGB   Interactive red-green-blue display - run-time help
HLPTVSPC   Interactive display of spectra from a cube - run-time help
IMCLP      Clip an image to a specified range.
IMLOD      reads tape to load images to disk
OGEOM      Simple image rotation, scaling, and translation
OHGEO      Geometric interpolation with correction for 3-D effects
READISK    writes images / uv data w extensions to tape in FITS format
TVSPC      Display images and spectra from a cube
WRTDISK    writes images / uv data w extensions to tape in FITS format
WRTPROCS   Procedures to simplify the reduction of VLBA data
\end{verbatim}\eve

\sects{\hspace{0.5em}IMAGE}

\vskip 0.5pt\todx{ABOUT IMAGE}\iodx{image}
\bbve\begin{verbatim}
CPYRT      replaces history with readme file, inserts copyright
IMRMS      Plot IMEAN rms answers
MAPBM      Map VLA beam polarization
NANS       reads an image or a UV data set and looks for NaNs
PROFL      Generates plot file for a profile display.
STRAN      Task compares ST tables, find image coordinates (e.g. guide star )
VLABP      VLA antenna beam polarization correction for snapshot images
XSMTH      Smooth data along the x axis
XSUM       Sum or average images on the x axis
XTRAN      Create an image with transformed coordinates
\end{verbatim}\eve

\sects{\hspace{0.5em}IMAGING}

\vskip 0.5pt\todx{ABOUT IMAGING}\iodx{imaging}
\bbve\begin{verbatim}
AHIST      Task to convert image intensities by adaptive histogram
ALLOKAY    specifies that initial conditions have been met.
APCLN      Deconvolves images with CLEAN algorithm
APGS       deconvolves image with Gerchberg-Saxton algorithm
APVC       Deconvolves images with van Cittert algorithm
AVOPTION   Controls type or range of averaging done by a task
BCOMP      gives beginning component number for multiple fields
BLC        gives lower-left-corner of selected subimage
BLWUP      Blow up an image by any positive integer factor.
BMAJ       gives major axis size of beam or component
BMIN       gives minor axis size of beam or component
BOX2CC     Converts CLBOX in pixels to CCBOX in arc seconds
BOXES      Adds Clean boxes to BOXFILE around sources from a list
BOXFILE    specifies name of Clean box text file
BOX        specifies pixel coordinates of subarrays of an image
BPA        gives position angle of major axis of beam or component
BSAVG      Task to do an FFT-weighted sum of beam-switched images
BSCLN      Hogbom Clean on beam-switched difference image
BSGRD      Task to image beam-switched single-dish data
BSMAP      images weak sources with closure phases
CANDY      user-definable (paraform) task to create an AIPS image
CCBOX      specifies pixel coordinates of subarrays of an image
CCEDT      Select CC components in BOXes and above mininum flux.
CCFND      prints the contents of a Clean Components extension file.
CCGAU      Converts point CLEAN components to Gaussians
CCMOD      generates clean components to fit specified source model
CCMRG      sums all clean components at the same pixel
CCRES      Removes or restores a CC file to a map with a gaussian beam.
CELLSIZE   gives the pixel size in physical coordinates
CHKFC      makes images of Clean boxes from Boxfile
CLBOX      specifies subarrays of an image for Clean to search
CMETHOD    specifies the method by which the uv model is computed
CMODEL     specifies the method by which the uv model is computed
COHER      Baseline Phase coherence measurement
CONPL      Plots AIPS gridding convolution functions
CONVL      convolves an image with a gaussian or another image
CUTOFF     specifies a limit below or above which the operation ends
CXCLN      Complex Hogbom CLEAN
DCONV      deconvolves a gaussian from an image
DECSHIFT   gives Y-coordinate shift of an image center from reference
DELBOX     Verb to delet boxes with TV cursor & graphics display.
DFILEBOX   Verb to delete Clean boxes with TV cursor & write to file
DO3DIMAG   specifies whether uvw's are reprojected to each field center
DOGRIDCR   selects correction for gridding convolution function
DOOSRO     calibrating amplitude and phase, and imaging VLA data
DRAWBOX    Verb to draw Clean boxes on the display
DTSUM      Task to provide a summary of the contents of a dataset
EVAUV      Subtracts & divides a model into UV data, does statistics on results
FACES      makes images of catalog sources for initial calibration
FACTOR     scales some display or CLEANing process
FETCH      Reads an image from an external text file.
FFT        takes Fourier Transform of an image or images
FGAUSS     Minimum flux to Clean to by widths of Gaussian models
FILEBOX    Verb to reset Clean boxes with TV cursor & write to file
FILIT      Interactive BOXFILE editing with facet images
FIXBX      converts a BOXFILE to another for input to IMAGR
FLATN      Re-grid multiple fields into one image incl sensitivity
FLDSIZE    specifies size(s) of images to be processed
FLUX       gives a total intensity value for image/component or to limit
FOV        Specifies the field of view
FSHIFT     specifies a position shift - output from fitting routines
FSIZE      file size in Megabytes
GAIN       specifies loop gain for deconvolutions
GUARD      portion of UV plane to receive no data in gridding
HA2TI      Converts data processed by TI2HA (STUFFER) back to real times
HISEQ      task to translate image by histogram equalization
HLPCLEAN   Cleaning tasks  - run-time help
HLPSCIMG   Full-featured image plus self-cal loops, editing - run-time help
HLPSCMAP   Imaging plus self-cal and editing SCMAP - run-time help
HOLGR      Read & process holography visibility data to telescope images
HOLOG      Read & process holography visibility data to telescope images
IM2CC      Task to convert an image to multi-facet Clean Components
IM2PARM    Specifes enhancement parameters for OOP-based imaging: 2nd set
IM2UV      converts an image to a visibility data set
IMAGR      Wide-field and/or wide-frequency Cleaning / imaging task.
IMAGRPRM   Specifes enhancement parameters for OOP-based imaging
IMERG      merges images of different spatial resolutions
IMSIZE     specifies number of pixels on X and Y axis of an image
IMTXT      Write an image to an external text file.
INLIST     specifies name of input disk file, usually a source list
LINIMAGE   Build image cube from multi-IF data set
LTESS      makes mosaic images by linear combination
MANDL      creates an image of a subset of the Mandlebrot Set
MAPPR      Simplified access to IMAGR
MAXPIXEL   maximum pixels searched for components in Clark CLEAN
MODVF      task to create a warped velocity field
MWFLT      applies linear & non-linear filters to images
NBOXES     Number of boxes
NCCBOX     Number of clean component boxes
NCOMP      Number of CLEAN components
NDIG       Number of digits to display
NFIELD     The number of fields imaged
NMAPS      Number of maps (images) in an operation
NOIFS      makes all IFs into single spectrum
NOISE      estimates the noise in images, noise level cutoff
OBITIMAG   Access to OBIT task Imager without self-cal or peeling
OBITMAP    Simplified access to OBIT task Imager
OBITPEEL   Access to OBIT task Imager with self-cal and peeling
OBITSCAL   Access to OBIT task Imager with self-cal, NOT peeling
OBOXFILE   specifies name of output Clean box text file
ONEBEAM    specifies whether one beam is made for all facets or one for each
ONEFREQ    states that the current CC model was made with one frequency
OOSUB      Subtracts/divides a model from/into a uv data base
OUT2CLAS   The class of a secondary output file
OUT2DISK   The disk number of a secondary output file.
OUT2NAME   The name of a secondary output file.
OUT2SEQ    The sequence of a secondary output file.
OVERLAP    specifies how overlaps are to be handled
OVRSWTCH   specifies when IMAGR switches from OVERLAP >= 2 to OVERLAP = 1 mode
PADIM      Task to increase image size by padding with some value
PASTE      Pastes a selected subimage of one image into another.
PATGN      Task to create a user specified test or primary-beam pattern
PBCOR      Task to apply or correct an image for a primary beam
PBPARM     Primary beam parameters
PBSIZE     estimates the primary beam size in interferometer images
PGEOM      Task to transform an image into polar coordinates.
PHASE      Baseline Phase coherence measurement
PHAT       Prussian hat size
PIPEAIPS   calibrating amplitude and phase, and imaging VLA data
PIX2VAL    An image value in the units specified in the header.
PIX2XY     Specifies a pixel in an image
PIXAVG     Average image value
PIXRANGE   Range of pixel values to display
PIXSTD     RMS pixel deviation
PIXVAL     Value of a pixel
PIXXY      Specifies a pixel location.
PRTCC      prints the contents of a Clean Components extension file.
PUTVALUE   Verb to store a pixel value at specified position
QCREATE    adverb controlling the way large files are created
QUANTIZE   Quantization level to use
REBOX      Verb to reset boxes with TV cursor & graphics display.
REGRD      Regrids an image from one co-ordinate frame to another
REMAG      Task to replace magic blanks with a user specified value
RMSD       Calculate rms for each pixel using data at the box around the pixel
ROBUST     Uniform weighting "robustness" parameter
RSTOR      Restores a CC file to a map with a gaussian beam.
SABOX      create box file from source islands in facet images
SCIMG      Full-featured imaging plus self-calibration loop with editing
SCMAP      Imaging plus self-calibration loop with editing
SDCLN      deconvolves image by Clark and then "SDI" cleaning methods
SDGRD      Task to select and image random-position single-dish data
SDIMG      Task to select and image random-position single-dish data
SETFC      makes a BOXFILE for input to IMAGR
SHADW      Generates the "shadowed" representation of an image
SHIFT      specifies a position shift
SIZEFILE   return file size plus estimate of IMAGR work file size
SKEW       Specifies a skew angle
SKYVE      Regrids a DSS image from one co-ordinate frame to another
SMODEL     Source model
SNCUT      Specifies minimum signal-to-noise ratio
SPCOR      Task to correct an image for a primary beam and spectral index
SPECR      Spectral regridding task for UV data
SPFIX      Makes cube from input to and output from SPIXR spectral index
SPIXR      Fits spectral indexes to each row of an image incl curvature
STACK      Task to co-add a set of 2-dimensional images with weighting
STEER      Task which deconvolves the David Steer way.
STESS      Task which finds sensitivity in mosaicing
STFACTOR   scales star display or SDI CLEANing process
STUFFR     averages together data sets in hour angle
SUBIM      Task to select a subimage from up to a 7-dim. image
SUMIM      Task to sum overlapping, sequentially-numbered images
TDEPEND    Time-dependent imaging procedure sequence
TD_SCANS   Time-dependent imaging procedure sequence: find intervals
TD_SSCAN   Time-dependent imaging procedure sequence: find intervals
TD_STEP3   Time-dependent imaging procedure "step 3"
TD_STEP5   Time-dependent imaging procedure sequence: later steps
TKBOX      Procedure to set a Clean box with the TK cursor
TKNBOXS    Procedure to set Clean boxes 1 - n with the TK cursor
TRANSCOD   Specified desired transposition of an image
TRANS      Task to transpose a subimage of an up to 7-dim. image
TRC        Specified the top right corner of a subimage
TVBOX      Verb to set boxes with TV cursor & graphics display.
UBAVG      Baseline dependent time averaging of uv data
UTESS      deconvolves images by maximizing emptiness
UVBOX      radius of the smoothing box used for uniform weighting
UVBXFN     type of function used when counting for uniform weighting
UVFRE      Makes one data set have the spectral structure of another
UVIMG      Grid UV data into an "image"
UVMAP      makes images from calibrated UV data.
UVPOL      modifies UV data to make complex image and beam
UVSIZE     specifies number of pixels on X and Y axes of a UV image
UVSUB      Subtracts/divides a model from/into a uv data base
UVWTFN     Specify weighting function, Uniform or Natural
VLARUN     calibrating amplitude and phase, and imaging VLA data
VTESS      Deconvolves sets of images by the Maximum Entropy Method
WFCLN      Wide field and/or widefrequency  CLEANing/imaging task.
WGAUSS     Widths of Gaussian models (FWHM)
WTSUM      Task to do a a sum of images weighted by other images
XMOM       Fits one-dimensional moments to each row of an image
YPARM      Specifies Y axis convolving function
ZEROSP     Specify how to include zero spacing fluxes in FT of UV data
\end{verbatim}\eve

\sects{\hspace{0.5em}INFORMAT}

\vskip 0.5pt\todx{ABOUT INFORMATION}\iodx{information}
\bbve\begin{verbatim}
ASTROMET   Describes the process of astrometric/geodetic reduction in AIPS
BATCHJOB   Information about BATCH
CALIBRAT   describes the process of data calibration in AIPS
CATEGORY   List of allowed primary keywords in HELP files
GNUGPL     Information about GNU General Public License for AIPS
LSAPROPO   Data input to APROPO to find what uses what words
MSGSERVER  Information about the X11-based message server
MSGSRV     Information about the X11-based message server
NEWTASK    Information about installing a new task
NOADVERB   Information about the lack of a defined adverb or verb
PANIC      Instructions for what to do when things go wrong
POPSDAT    lists all POPS symbols, used to create them in MEmory files
POPSYM     Describes the symbols used in POPS
POSTSCRIP  General comments about AIPS use of PostScript incl macros
PSEUDO     Description of POPS pseudoverbs - obsolete list file
READLINE   Information about AIPS use of the GNU readline library.
REBYTE     service program to transform byte order of full data sets
SECONDARY  List of allowed secondary keywords in HELP files
SECONDRY   List of allowed secondary keywords in HELP files
TEKSERVER  Information about the X-11 Tektronix emulation server
TEKSRV     Information about the X-11 Tektronix emulation server
TPMON      Information about the TPMON "Daemon"
USERLIST   Alphabetic and numeric list of VLA users, points to real list
UV1TYPE    Convolving function type 1, pillbox or square wave
UV2TYPE    Convolving function type 2, exponential function
UV3TYPE    Convolving function type 3, sinc function
UV4TYPE    Convolving function type 4, exponent times sinc function
UV5TYPE    Convolving function type 5, spheroidal function
UV6TYPE    Convolving function type 6, exponent times BessJ1(x) / x
WHATSNEW   lists changes and new code in the last several AIPS releases
XAS        Information about TV-Servers
XVSS       Information about older Sun OpenWindows-specific TV-Server
\end{verbatim}\eve

\sects{\hspace{0.5em}INTERACT}

\vskip 0.5pt\todx{ABOUT INTERACTIVE}\iodx{interactive}
\bbve\begin{verbatim}
AIPS       AIPS main program for interactive use
BPEDT      Interactive TV task to edit uv data based on BP tables
DELBOX     Verb to delet boxes with TV cursor & graphics display.
DFILEBOX   Verb to delete Clean boxes with TV cursor & write to file
EDITA      Interactive TV task to edit uv data based on TY/SY/SN/CL tables
EDITR      Interactive baseline-oriented visibility editor using the TV
FILEBOX    Verb to reset Clean boxes with TV cursor & write to file
FTFLG      interactive flagging of UV data in channel-time using the TV
HLPAGAUS   Interactive Gaussian absorption fitting task AGAUS - run-time help
HLPCLEAN   Cleaning tasks  - run-time help
HLPEDIBP   Interactive BP table uv-data editor BPEDT - run-time help
HLPEDICL   Interactive SN/CL table uv-data editor - run-time help
HLPEDIPC   Interactive PC table editor PCEDT - run-time help
HLPEDIPD   Interactive PD table editor PDEDT - run-time help
HLPEDISN   Interactive SN/CL table (not UV) editor - run-time help
HLPEDISS   Interactive SY table (not UV) editor - run-time help
HLPEDISY   Interactive SY table uv-data editor EDITA - run-time help
HLPEDITS   Interactive TY table (not UV) editor - run-time help
HLPEDITY   Interactive TY table uv-data editor EDITA - run-time help
HLPEDIUV   Interactive uv-data editor EDITR - run-time help
HLPFILIT   Interactive Clean box file editing with image display - run-time help
HLPFTFLG   Interactive time-channel visibility Editor - run-time help
HLPIBLED   Interactive Baseline based visibility Editor - run-time help
HLPPCFLG   Interactive time-channel PC table Editor PCFLG - run-time help
HLPPLAYR   OOP TV class demonstration task - run-time help
HLPRMFIT   Polarization fitting task RMFIT - run-time help
HLPSCIMG   Full-featured image plus self-cal loops, editing - run-time help
HLPSCMAP   Imaging plus self-cal and editing SCMAP - run-time help
HLPSPFLG   Interactive time-channel visibility Editor SPFLG - run-time help
HLPTVFLG   Interactive time-baseline visibility Editor TVFLG - run-time help
HLPTVHLD   Interactive image display with histogram equalization - run-time help
HLPTVHUI   Interactive intensity-hue-saturation display - run-time help
HLPTVRGB   Interactive red-green-blue display - run-time help
HLPTVSAD   Find & fit Gaussians to an image with interaction - run-time help
HLPTVSPC   Interactive display of spectra from a cube - run-time help
HLPUFLAG   Edit uv-data on a grid UFLAG - run-time help
HLPWIPER   edit uv data from UVPLT-like plot WIPER - run-time help
HLPXGAUS   Interactive Gaussian fitting task XGAUS - run-time help
HLPZAMAN   Fits 1-dimensional Zeeman model to absorption data - run-time help
HLPZEMAN   Fits 1-dimensional Zeeman model to data - run-time help
IBLED      Interactive BaseLine based visibility EDitor
IMAGR      Wide-field and/or wide-frequency Cleaning / imaging task.
MAPPR      Simplified access to IMAGR
MFITSET    gets adverbs for running IMFIT and JMFIT
OPTELL     The operation to be passed to a task by TELL
PCEDT      Interactive TV task to edit pulse-cal (PC) tables
PCFLG      interactive flagging of Pulse-cal data in channel-TB using the TV
PDEDT      Interactive TV task to edit polarization D-term (PD) tables
PLAYR      Verb to load an image into a TV channel
READ       Read a value from the users terminal
READLINE   Information about AIPS use of the GNU readline library.
REBOX      Verb to reset boxes with TV cursor & graphics display.
SCIMG      Full-featured imaging plus self-calibration loop with editing
SCMAP      Imaging plus self-calibration loop with editing
SETSLICE   Set slice endpoints on the TV interactively
SNEDT      Interactive SN/CL/TY/SY table editor using the TV
SPFLG      interactive flagging of UV data in channel-TB using the TV
TDEPEND    Time-dependent imaging procedure sequence
TD_SCANS   Time-dependent imaging procedure sequence: find intervals
TD_SSCAN   Time-dependent imaging procedure sequence: find intervals
TD_STEP3   Time-dependent imaging procedure "step 3"
TD_STEP5   Time-dependent imaging procedure sequence: later steps
TK1SET     Verb to reset 1D gaussian fitting initial guess.
TKBOX      Procedure to set a Clean box with the TK cursor
TKNBOXS    Procedure to set Clean boxes 1 - n with the TK cursor
TKPOS      Read a position from the graphics screen or window
TKSET      Verb to set 1D gaussian fitting initial guesses.
TKWIN      Procedure to set BLC and TRC with Graphics cursor
TV1SET     Verb to reset 1D gaussian fitting initial guess on TV plot.
TVBOX      Verb to set boxes with TV cursor & graphics display.
TVFLG      interactive flagging of UV data using the TV
TVSAD      Finds and fits Gaussians to portions of an image with interaction
TVSCROL    Shift position of image on the TV screen
TVSET      Verb to set slice Gaussian fitting initial guesses from TV plot
TVSPC      Display images and spectra from a cube
TVSPLIT    Compare two TV image planes, showing halves
TVSTAT     Find the mean and RMS in a blotch region on the TV
TVTRANSF   Interactively alters the TV image plane transfer function
TVWINDOW   Set a window on the TV with the cursor
TVZOOM     Activate the TV zoom
UFLAG      Plots and edits data using a uv-plane grid and the TV
WEDERASE   Load a wedge portion of the TV with zeros
WIPER      plots and edits data from a UV data base using the TV
TVHELIX    Verb to activate a helical hue-intensity TV pseudo-coloring
\end{verbatim}\eve

\sects{\hspace{0.5em}MODELING}

\vskip 0.5pt\todx{ABOUT MODELING}\iodx{model}
\bbve\begin{verbatim}
ACTNOISE   puts estimate of actual image uncertainty and zero in header
AGAUS      Fits 1-dimensional Gaussians to absorption-line spectra
BOXES      Adds Clean boxes to BOXFILE around sources from a list
BSMOD      creates single-dish UV beam-switched data with model sources
BSTST      Graphical display of solutions to frequency-switched data
CMETHOD    specifies the method by which the uv model is computed
CMODEL     specifies the method by which the uv model is computed
CUBIT      Model a galaxy's density and velocity distribution from full cube
DIFUV      Outputs the difference of two matching input uv data sets
DOMODEL    selects display of model function
DORESID    selects display of differences between model and data
DOSPIX     selects solutions for spectral index of model components
EFACTOR    scales some error analysis process
EVAUV      Subtracts & divides a model into UV data, does statistics on results
FACES      makes images of catalog sources for initial calibration
FITOUT     specifies name of output text file for results of fitting
FSHIFT     specifies a position shift - output from fitting routines
GLENS      models galaxy gravitational lens acting on 3 component source
HLPAGAUS   Interactive Gaussian absorption fitting task AGAUS - run-time help
HLPRMFIT   Polarization fitting task RMFIT - run-time help
HLPTVSAD   Find & fit Gaussians to an image with interaction - run-time help
HLPXGAUS   Interactive Gaussian fitting task XGAUS - run-time help
HLPZAMAN   Fits 1-dimensional Zeeman model to absorption data - run-time help
HLPZEMAN   Fits 1-dimensional Zeeman model to data - run-time help
IMFIT      Fits Gaussians to portions of an image
IMMOD      adds images of model objects to an image
INLIST     specifies name of input disk file, usually a source list
JMFIT      Fits Gaussians to portions of an image
MFITSET    gets adverbs for running IMFIT and JMFIT
MFPRT      prints MF tables in a format needed by modelling software
MODAB      Makes simple absorption/emission spectral-line image in I/V
MODIM      adds images of model objects to image cubes in IQU polarization
MODSP      adds images of model objects to image cubes in I/V polarization
MODVF      task to create a warped velocity field
OMFIT      Fits sources and, optionally, a self-cal model to uv data
ONEFREQ    states that the current CC model was made with one frequency
OOSUB      Subtracts/divides a model from/into a uv data base
RADIUS     Specify a radius in an image
RM2PL      Plots spectrum of a pixel with RMFIT fit
RMFIT      Fits 1-dimensional polarization spectrum to Q/U cube
RMSLIMIT   selects things with RMS above this limit
SAD        Finds and fits Gaussians to portions of an image
SDMOD      modifies single-dish UV data with model sources
SLFIT      Task to fit gaussians to slice data.
SPMOD      Modify UV database by adding a model with spectral lines
STVERS     star display table version number
TK1SET     Verb to reset 1D gaussian fitting initial guess.
TKAMODEL   Verb to add slice model display directly on TEK
TKARESID   Verb to add slice model residuals directly on TEK
TKGUESS    Verb to display slice model guess directly on TEK
TKMODEL    Verb to display slice model directly on TEK
TKRESID    Verb to display slice model residuals directly on TEK
TKSET      Verb to set 1D gaussian fitting initial guesses.
TKSLICE    Verb to display slice file directly on TEK
TV1SET     Verb to reset 1D gaussian fitting initial guess on TV plot.
TVACOMPS   Verb to add slice model components directly on TV graphics
TVAGUESS   Verb to re-plot slice model guess directly on TV graphics
TVAMODEL   Verb to add slice model display directly on TV graphics
TVARESID   Verb to add slice model residuals directly on TV graphics
TVASLICE   Verb to add a slice display on TV graphics from slice file
TVCOMPS    Verb to display slice model components directly on TV graphics
TVGUESS    Verb to display slice model guess directly on TV graphics
TVMODEL    Verb to display slice model directly on TV graphics
TVRESID    Verb to display slice model residuals directly on TV graphics
TVSAD      Finds and fits Gaussians to portions of an image with interaction
TVSET      Verb to set slice Gaussian fitting initial guesses from TV plot
TVSLICE    Verb to display slice file directly on TV
UVFIT      Fits source models to uv data.
UVGIT      Fits source models to uv data.
UVMOD      Modify UV database by adding a model incl spectral index
UVSUB      Subtracts/divides a model from/into a uv data base
XG2PL      Plots spectrum of a pixel with XGAUS/AGAUS and ZEMAN/ZAMAN fits
XGAUS      Fits 1-dimensional Gaussians to images: restartable
ZAMAN      Fits 1-dimensional Zeeman model to absorption-line data
ZEMAN      Fits 1-dimensional Zeeman model to data
\end{verbatim}\eve

\sects{\hspace{0.5em}OBSOLETE}

\vskip 0.5pt\todx{ABOUT OBSOLETE}\iodx{obsolete}
\bbve\begin{verbatim}
ABACKUP    VMS procedure to back up data on tape
ARESTORE   Restores back up tapes of users data
CLIPM
CODETYPE   specifies the desired operation type
HORUS      makes images from unsorted UV data, applying any calibration
MX         makes images & deconvolves using UV data directly - replaced
OFFROAM    Procedure to clear the TV from a Roam condition
PFT        The Perley-Feigelson Test; see PFTLOAD.RUN, PFTEXEC.RUN
PHCLN      PHCLN has been removed, use PHAT adverb in APCLN.
PSEUDO     Description of POPS pseudoverbs - obsolete list file
SAMPTYPE   Specifies sampling type
SETROAM    Verb use to set roam image mode, then do roam.  OBSOLETE
SNCUT      Specifies minimum signal-to-noise ratio
XVSS       Information about older Sun OpenWindows-specific TV-Server
\end{verbatim}\eve

\sects{\hspace{0.5em}ONED}

\vskip 0.5pt\todx{ABOUT ONED}\iodx{one-dimensional}
\bbve\begin{verbatim}
AGAUS      Fits 1-dimensional Gaussians to absorption-line spectra
PFPL2      Paraform Task to generate a plot file: (slice intensity)
PLCUB      Task to plot intensity vs x panels on grid of y,z pixels
PLROW      Plot intensity of a series of rows with an offset.
RM2PL      Plots spectrum of a pixel with RMFIT fit
RMFIT      Fits 1-dimensional polarization spectrum to Q/U cube
SETSLICE   Set slice endpoints on the TV interactively
SL2PL      Task to convert a Slice File to a Plot File
SLFIT      Task to fit gaussians to slice data.
SLICE      Task to make a slice file from an image
SLPRT      Task to print a Slice File
TK1SET     Verb to reset 1D gaussian fitting initial guess.
TKAMODEL   Verb to add slice model display directly on TEK
TKARESID   Verb to add slice model residuals directly on TEK
TKASLICE   Verb to add a slice display on TEK from slice file
TKGUESS    Verb to display slice model guess directly on TEK
TKMODEL    Verb to display slice model directly on TEK
TKRESID    Verb to display slice model residuals directly on TEK
TKSET      Verb to set 1D gaussian fitting initial guesses.
TKSLICE    Verb to display slice file directly on TEK
TV1SET     Verb to reset 1D gaussian fitting initial guess on TV plot.
TVACOMPS   Verb to add slice model components directly on TV graphics
TVAGUESS   Verb to re-plot slice model guess directly on TV graphics
TVAMODEL   Verb to add slice model display directly on TV graphics
TVARESID   Verb to add slice model residuals directly on TV graphics
TVASLICE   Verb to add a slice display on TV graphics from slice file
TVCOMPS    Verb to display slice model components directly on TV graphics
TVGUESS    Verb to display slice model guess directly on TV graphics
TVMODEL    Verb to display slice model directly on TV graphics
TVRESID    Verb to display slice model residuals directly on TV graphics
TVSET      Verb to set slice Gaussian fitting initial guesses from TV plot
TVSLICE    Verb to display slice file directly on TV
XG2PL      Plots spectrum of a pixel with XGAUS/AGAUS and ZEMAN/ZAMAN fits
XGAUS      Fits 1-dimensional Gaussians to images: restartable
XPLOT      Plots image rows one at a time on the graphics or TV screen
ZAMAN      Fits 1-dimensional Zeeman model to absorption-line data
ZEMAN      Fits 1-dimensional Zeeman model to data
\end{verbatim}\eve

\sects{\hspace{0.5em}OOP}

\vskip 0.5pt\todx{ABOUT OOP}\iodx{OOP}\iodx{Object-based tasks}
\bbve\begin{verbatim}
BLING      find residual rate and delay on individual baselines
BPEDT      Interactive TV task to edit uv data based on BP tables
BSCOR      Combines two beam-switched images
BSGEO      Beam-switched Az-El image to RA-Dec image translation
BSGRD      Task to image beam-switched single-dish data
CCEDT      Select CC components in BOXes and above mininum flux.
CCSEL      Select signifigant CC components
DSKEW      Geometric interpolation correction for skew
EDITA      Interactive TV task to edit uv data based on TY/SY/SN/CL tables
EDITR      Interactive baseline-oriented visibility editor using the TV
FILIT      Interactive BOXFILE editing with facet images
FINDR      Find normal values for a uv data set
FIXWT      Modify weights to reflect amplitude scatter of data
FLAGR      Edit data based on internal RMS, amplitudes, weights
FLATN      Re-grid multiple fields into one image incl sensitivity
FRCAL      Faraday rotation self calibration task
HLPCLEAN   Cleaning tasks  - run-time help
HLPEDIBP   Interactive BP table uv-data editor BPEDT - run-time help
HLPEDICL   Interactive SN/CL table uv-data editor - run-time help
HLPEDIPC   Interactive PC table editor PCEDT - run-time help
HLPEDIPD   Interactive PD table editor PDEDT - run-time help
HLPEDISN   Interactive SN/CL table (not UV) editor - run-time help
HLPEDISS   Interactive SY table (not UV) editor - run-time help
HLPEDISY   Interactive SY table uv-data editor EDITA - run-time help
HLPEDITS   Interactive TY table (not UV) editor - run-time help
HLPEDITY   Interactive TY table uv-data editor EDITA - run-time help
HLPEDIUV   Interactive uv-data editor EDITR - run-time help
HLPFILIT   Interactive Clean box file editing with image display - run-time help
HLPPLAYR   OOP TV class demonstration task - run-time help
HLPSCIMG   Full-featured image plus self-cal loops, editing - run-time help
HLPSCMAP   Imaging plus self-cal and editing SCMAP - run-time help
IM2PARM    Specifes enhancement parameters for OOP-based imaging: 2nd set
IMAGR      Wide-field and/or wide-frequency Cleaning / imaging task.
IMAGRPRM   Specifes enhancement parameters for OOP-based imaging
IMCLP      Clip an image to a specified range.
IMFRING    large image delay fitting with IM2CC and OOFRING
IMSCAL     large image self-cal with IM2CC and OOCAL
MAPBM      Map VLA beam polarization
MBDLY      Fits multiband delays from IF phases, updates SN table
MULIF      Change number of IFs in output
OGEOM      Simple image rotation, scaling, and translation
OHGEO      Geometric interpolation with correction for 3-D effects
OMFIT      Fits sources and, optionally, a self-cal model to uv data
OOCAL      determines antenna complex gain with frequency-dependent models
OOSRT      Sort a UV dataset into a specified order
OOSUB      Subtracts/divides a model from/into a uv data base
PASTE      Pastes a selected subimage of one image into another.
PCEDT      Interactive TV task to edit pulse-cal (PC) tables
PDEDT      Interactive TV task to edit polarization D-term (PD) tables
PLAYR      Verb to load an image into a TV channel
RFI        Look for RFI in uv data
SABOX      create box file from source islands in facet images
SCIMG      Full-featured imaging plus self-calibration loop with editing
SCMAP      Imaging plus self-calibration loop with editing
SDGRD      Task to select and image random-position single-dish data
SNEDT      Interactive SN/CL/TY/SY table editor using the TV
TDEPEND    Time-dependent imaging procedure sequence
TD_SCANS   Time-dependent imaging procedure sequence: find intervals
TD_SSCAN   Time-dependent imaging procedure sequence: find intervals
TD_STEP3   Time-dependent imaging procedure "step 3"
TD_STEP5   Time-dependent imaging procedure sequence: later steps
UV2MS      Append single-source file to multi-source file.
VLABP      VLA antenna beam polarization correction for snapshot images
WFCLN      Wide field and/or widefrequency  CLEANing/imaging task.
\end{verbatim}\eve

\sects{\hspace{0.5em}OPTICAL}

\vskip 0.5pt\todx{ABOUT OPTICAL}\iodx{optical}
\bbve\begin{verbatim}
COSTAR     Verb to plot a symbol at given position on top of a TV image
GSTAR      Task to read a Guide Star (UK) table and create an ST table.
IMFLT      fits and removes a background intensity plane from an image
STFND      Task to find stars in an image and generate an ST table.
STRAN      Task compares ST tables, find image coordinates (e.g. guide star )
TVSTAR     Verb to plot star positions on top of a TV image
XTRAN      Create an image with transformed coordinates
\end{verbatim}\eve

\sects{\hspace{0.5em}PARAFORM}

\vskip 0.5pt\todx{ABOUT PARAFORM}\iodx{paraform tasks}
\bbve\begin{verbatim}
CANDY      user-definable (paraform) task to create an AIPS image
DTCHK      Task to check results of a test using simulated data.
FUDGE      modifies UV data with user's algorithm: paraform task
NEWTASK    Information about installing a new task
PFPL1      Paraform Task to generate a plot file: (does grey scale)
PFPL2      Paraform Task to generate a plot file: (slice intensity)
PFPL3      Paraform Task to generate a plot file: (does histogram)
TAFFY      User definable task to operate on an image
TBTSK      Paraform OOP task for tables
UVFIL      Create, fill a uv database from user supplied information
\end{verbatim}\eve

\sects{\hspace{0.5em}PLOT}

\vskip 0.5pt\todx{ABOUT PLOT}\iodx{plots}
\bbve\begin{verbatim}
AGAUS      Fits 1-dimensional Gaussians to absorption-line spectra
ALIAS      adverb to alias antenna numbers to one another
ANBPL      plots and prints  uv data converted to antenna based values
ASPMM      Plot scaling parameter - arc seconds per millimeter on plot
AVGCHAN    Controls averaging of spectral channels
AVGIF      Controls averaging of IF channels
BDROP      gives number of pooints dropped at the beginning
BLSUM      sums images over irregular sub-images, displays spectra
BPERR      Print and plot BPASS closure outputs
BPLOT      Plots bandpass tables in 2 dimensions as function of time
CANPL      translates a plot file to a Canon printer/plotter
CAPLT      plots closure amplitude and model from CC file
CBPLOT     selects a display of a Clean beam full width at half maximum
CCNTR      generate a contour plot file from an image
CIRCLEVS   Sets RGBLEVS to fill LEVS with a circular color scheme
CLEV       Contour level multiplier in physical units
CLPLT      plots closure phase and model from CC file
CNTR       generate a contour plot file or TV plot from an image
CON3COL    Controls use of full 3-color graphics for contouring
CONPL      Plots AIPS gridding convolution functions
COPIES     sets the number of copies to be made
COSTAR     Verb to plot a symbol at given position on top of a TV image
DARKLINE   The level at which vectors are switched from light to dark
DEFCOLOR   Sets adverb PLCOLORS to match s default XAS TV
DFTPL      plots DFT of a UV data set at arbitrary point versus time
DIST       gives a distance - PROFL uses as distance to observer
DO3COL     Controls whether full 3-color graphics are used in a plot
DOALIGN    specifies how two or more images are aligned in computations
DOBLANK    controls handling of blanking
DOCELL     selects units of cells over angular unit
DOCENTER   selects something related to centering
DOCIRCLE   select a "circular" display (i.e. trace coordinates, ...)
DOCOLOR    specifies whether coloring is done
DOCONT     selects a display of contour lines
DOCRT      selects printer display or CRT display (giving width)
DODARK     specifies whether "dark" vectors are plotted dark or light
DOEBAR     Controls display of estimates of the uncertainty in the data
DOFRACT    Tells whether to compute a fraction or ratio
DOGREY     selects a display of a grey-scale image
DOHIST     selects a histogram display
DOHMS      selects sexagesimal (hours-mins-secs) display format
DOMODEL    selects display of model function
DOPLOT     Controls plotting of something
DOPRINT    selects printer display or CRT display (giving width)
DORESID    selects display of differences between model and data
DOSLICE    selects display of slice data
DOVECT     selects display of polarization vectors
DOWEDGE    selects display of intensity step wedge
EDROP      number of points/iterations to be omitted from end of process
ELFIT      Plots/fits selected contents of SN, TY, SY, PC or CL files
EXTAB      exports AIPS table data as tab-separated text
EXTLIST    lists detailed information about contents of extension files
FACTOR     scales some display or CLEANing process
FGPLT      Plots selected contents of FG table
FLAMLEVS   Sets RGBLEVS to fill LEVS with a red flame color scheme
FRPLT      Task to plot fringe rate spectra
FUNCTYPE   specifies type of intensity transfer function
GCPLT      Plots gain curves from text files
GREYS      plots images as contours over multi-level grey
GSTAR      Task to read a Guide Star (UK) table and create an ST table.
HLPUFLAG   Edit uv-data on a grid UFLAG - run-time help
HLPWIPER   edit uv data from UVPLT-like plot WIPER - run-time help
ICUT       specifies a cutoff level in units of the image
IMEAN      displays the mean & extrema and plots histogram of an image
IMRMS      Plot IMEAN rms answers
IMVIM      plots one image's values against another's
IRING      integrates intensity / flux in rings / ellipses
ISPEC      Plots and prints spectrum of region of a cube
KNTR       make a contour/grey plot file from an image w multiple panels
LABEL      selects a type of extra labeling for a plot
LAYER      Task to create an RGB image from multiple images
LEVS       list of multiples of the basic level to be contoured
LPEN       specifies the "pen width" code # => width of plotted lines
LTYPE      specifies the type and degree of axis labels on plots
LWPLA      translates plot file(s) to a PostScript printer or file
MF2ST      Task to generate an ST ext. file from Model Fit ext. file
NPLOTS     gives number of plots per page or per job
NX         General adverb referring to a number of things in the Y direction
NY         General adverb referring to a number of things in the Y direction
OBPLT      Plot columns of an OB table.
OFMFILE    specifies the name of a text file containing OFM values
PCFIT      Finds delays and phases using a pulse-cal (PC) table
PCHIS      Generates a histogram plot file from text input, e.g. from PCRMS
PCNTR      Generate plot file with contours plus polarization vectors
PCPLT      Plots pulse-cal tables in 2 dimensions as function of time
PCUT       Cutoff in polarized intensity
PFPL1      Paraform Task to generate a plot file: (does grey scale)
PFPL2      Paraform Task to generate a plot file: (slice intensity)
PFPL3      Paraform Task to generate a plot file: (does histogram)
PLCOLORS   specifies the colors to be used
PLCUB      Task to plot intensity vs x panels on grid of y,z pixels
PLEV       Percentage of peak to use for contour levels
PLGET      gets the adverbs used to make a particular plot file
PLOTC      plots color schems used by 3-color plot tasks
PLOTR      Basic task to generate a plot file from text input
PLROW      Plot intensity of a series of rows with an offset.
PLVER      specifies the version number of a PL extension file
POL3COL    Controls use of full 3-color graphics for polarization lines
POLPLOT    specifies the desired polarization ratio before plotting.
POSTSCRIP  General comments about AIPS use of PostScript incl macros
PRINTER    Verb to set or show the printer(s) used
PROFL      Generates plot file for a profile display.
PRTAB      prints any table-format extension file
PRTIM      prints image intensities from an MA catalog entry
PRTPL      Task to send a plot file to the line printer
QMSPL      Task to send a plot file to the QMS printer/plotter
RAINLEVS   Sets RGBLEVS to fill LEVS with a rainbow color scheme
RGBLEVS    colors to be applied to the contour levels
RM2PL      Plots spectrum of a pixel with RMFIT fit
RMFIT      Fits 1-dimensional polarization spectrum to Q/U cube
RSPEC      Plots and prints spectrum of rms of a cube
SCLIM      operates on an image with a choice of mathematical functions
SL2PL      Task to convert a Slice File to a Plot File
SLPRT      Task to print a Slice File
SNFIT      Fits parabola to SN amplitudes and plots result
SNIFS      Plots selected contents of SN, TY, SY, PC or CL files
SNPLT      Plots selected contents of SN, SY, TY, PC or CL files
STARS      Task to generate an ST ext. file with star positions
STEPLEVS   Sets RGBLEVS to fill LEVS with a repeated sequence of colors
STFND      Task to find stars in an image and generate an ST table.
STVERS     star display table version number
SYMBOL     General adverb, probably defines a plotting symbol type
TAPLT      Plots data from a Table extension file
TARPL      Plot output of TARS task
TEKSERVER  Information about the X-11 Tektronix emulation server
TEKSRV     Information about the X-11 Tektronix emulation server
TKAMODEL   Verb to add slice model display directly on TEK
TKARESID   Verb to add slice model residuals directly on TEK
TKASLICE   Verb to add a slice display on TEK from slice file
TKBOX      Procedure to set a Clean box with the TK cursor
TKERASE    Erase the graphics screen or window
TKGUESS    Verb to display slice model guess directly on TEK
TKMODEL    Verb to display slice model directly on TEK
TKNBOXS    Procedure to set Clean boxes 1 - n with the TK cursor
TKPL       Task to send a plot file to the TEK
TKPOS      Read a position from the graphics screen or window
TKRESID    Verb to display slice model residuals directly on TEK
TKSLICE    Verb to display slice file directly on TEK
TKWIN      Procedure to set BLC and TRC with Graphics cursor
TVACOMPS   Verb to add slice model components directly on TV graphics
TVAGUESS   Verb to re-plot slice model guess directly on TV graphics
TVAMODEL   Verb to add slice model display directly on TV graphics
TVARESID   Verb to add slice model residuals directly on TV graphics
TVASLICE   Verb to add a slice display on TV graphics from slice file
TVCOLORS   Sets adverb PLCOLORS to match the TV (DOTV=1) usage
TVCOMPS    Verb to display slice model components directly on TV graphics
TVGUESS    Verb to display slice model guess directly on TV graphics
TVMODEL    Verb to display slice model directly on TV graphics
TVPL       Display a plot file on the TV
TVRESID    Verb to display slice model residuals directly on TV graphics
TVSLICE    Verb to display slice file directly on TV
TVSTAR     Verb to plot star positions on top of a TV image
TXPL       Displays a plot (PL) file on a terminal or line printer
UFLAG      Plots and edits data using a uv-plane grid and the TV
UVHGM      Plots statistics of uv data files as histogram.
UVPLT      plots data from a UV data base
UVPRM      measures parameters from a UV data base
VLBACRPL   Plots crosscorrelations
VLBASNPL   Plots selected contents of SN or CL files
VPLOT      plots uv data and model from CC file
WETHR      Plots selected contents of WX tables, flags data based on WX
WIPER      plots and edits data from a UV data base using the TV
XAXIS      Which parameter is plotted on the horizontal axis.
XBASL      Fits and subtracts nth-order baselines from cube (x axis)
XG2PL      Plots spectrum of a pixel with XGAUS/AGAUS and ZEMAN/ZAMAN fits
XGAUS      Fits 1-dimensional Gaussians to images: restartable
XPLOT      Plots image rows one at a time on the graphics or TV screen
ZAMAN      Fits 1-dimensional Zeeman model to absorption-line data
ZEMAN      Fits 1-dimensional Zeeman model to data
\end{verbatim}\eve

\vfill\eject
\sects{\hspace{0.5em}POLARIZA}

\vskip 0.5pt\todx{ABOUT POLARIZATION}\iodx{polarization}
\bbve\begin{verbatim}
AFARS      Is used after FARS to determine Position and Value of the maximum
BANDPOL    specifies polarizations of individual IFs
BDEPO      computes depolarization due to rotation measure gradients
COMB       combines two images by a variety of mathematical methods
DOFARS     Procedure to aid in Faraday rotation synthesis using the FARS task
DOPOL      selects application of any polarization calibration
FARAD      add ionospheric Faraday rotation to CL table
FARS       Faraday rotation synthesis based on the brightness vs wavelength
HLPRMFIT   Polarization fitting task RMFIT - run-time help
HLPZAMAN   Fits 1-dimensional Zeeman model to absorption data - run-time help
HLPZEMAN   Fits 1-dimensional Zeeman model to data - run-time help
LPCAL      Determines instrumental polarization for UV data
MAPBM      Map VLA beam polarization
MEDI       combines four images by a variety of mathematical methods
MODAB      Makes simple absorption/emission spectral-line image in I/V
MODIM      adds images of model objects to image cubes in IQU polarization
MODSP      adds images of model objects to image cubes in I/V polarization
PCAL       Determines instrumental polarization for UV data
PCNTR      Generate plot file with contours plus polarization vectors
PCUT       Cutoff in polarized intensity
PDEDT      Interactive TV task to edit polarization D-term (PD) tables
PDVER      specifies the version of the spetral polarization table to use
PMODEL     Polarization model parameters
POLANGLE   Intrinsic polarization angles for up to 30 sources
POLCO      Task to correct polarization maps for Ricean bias
POLPLOT    specifies the desired polarization ratio before plotting.
QUFIX      determines Right minus Left phase difference, corrects cal files
QUOUT      writes text file of Q, U versus frequency to be used by RLDIF
QUXTR      extracts text files from Q,U cubes for input to TARS
RFARS      Correct Q/U cubes for Faraday rotation synthesis results
RLCAL      Determines instrumental right-left phase versus time (a self-cal)
RLCOR      corrects a data set for R-L phase differences
RLDIF      determines Right minus Left phase difference, corrects cal files
RM2PL      Plots spectrum of a pixel with RMFIT fit
RMFIT      Fits 1-dimensional polarization spectrum to Q/U cube
RM         Task to calculate rotation measure and magnetic field
SWPOL      Swap polarizations in a UV data base
TARPL      Plot output of TARS task
TARS       Simulation of Faraday rotation synthesis (mainly task FARS)
TRUEP      determines true antenna polarization from special data sets
VLABP      VLA antenna beam polarization correction for snapshot images
VLBACPOL   Procedure to calibrate cross-polarization delays
XG2PL      Plots spectrum of a pixel with XGAUS/AGAUS and ZEMAN/ZAMAN fits
XYDIF      find/apply X minus Y linear polarization phase difference
ZAMAN      Fits 1-dimensional Zeeman model to absorption-line data
ZEMAN      Fits 1-dimensional Zeeman model to data
\end{verbatim}\eve

\vfill\eject
\sects{\hspace{0.5em}POPS}

\vskip 0.5pt\todx{ABOUT POPS}\iodx{POPS}
\bbve\begin{verbatim}
ABOUT      displays lists and information on tasks, verbs, adverbs
ABS        returns absolute value of argument
ACOS       Returns arc cosine of argument (half-circle)
APROPOS    displays all help 1-line summaries containing specified words
ARRAY1     General scratch array adverb
ARRAY2     General scratch array adverb
ARRAY3     General scratch array adverb
ARRAY      Declares POPS symbol name and dimensions
ASIN       Returns arc sine of argument (half-circle)
ATAN2      Returns arc tangent of two arguments (full circle)
ATAN       Returns arc tangent of argument (half-circle)
BY         gives increment to use in FOR loops in POPS language
CATNO      Specifies AIPS catalog slot number range
CEIL       returns smallest integer greater than or equal the argument
CHAR       converts number to character string
CLRTEMP    clears the temporary literal area during a procedure
COMPRESS   recovers unused POPS address space and new symbols
CORE       displays the used and total space used by parts of POPS table
COS        returns cosine of the argument in degrees
DEBUG      turns on/off the POPS-language's debug messages
DEFAULT    Verb-like sets adverbs for a task or verb to initial values
DELAY      Verb to pause AIPS for DETIME seconds
DELTAX     Increment or size in X direction
DELTAY     Increment or size in Y direction
DENUMB     a scalar decimal number
DOOSRO     calibrating amplitude and phase, and imaging VLA data
DOVLAMP    Produces amp calibration file for phased-VLA VLBI data
DPARM      General numeric array adverb used many places
DUMP       displays portions of the POPS symbol table in all formats
EDIT       enter edit-a-procedure mode in the POPS language
EHEX       converts decimal to extended hex
EHNUMB     an extended hexadecimal "number"
ELSE       starts POPS code done if an IF condition is false (IF-THEN..)
ENDEDIT    terminates procedure edit mode of POPS input
END        marks end of block (FOR, WHILE, IF) of POPS code
ERASE      removes one or more lines from a POPS procedure
ERROR      was there an error
EVLA       puts the list of eVLA antennas in the current file on stack
EXIT       ends an AIPS batch or interactive session
EXP        returns the exponential of the argument
EXPLAIN    displays help + extended information describing a task/symbol
FINISH     terminates the entry and compilation of a procedure
FLOOR      returns largest integer <= argument
FOR        starts an iterative sequence of operations in POPS language
GETDATE    Convert the current date and time to a string
GET        restores previously SAVEd full POPS environment
GETPOPSN   Verb to return the pops number on the stack
GG         spare scalar adverb for use in procedures
GRANDOM    Finds a random number with mean 0 and rms 1
HELP       displays information on tasks, verbs, adverbs
HSA        puts the list of HSA antennas in the current file on stack
IF         causes conditional execution of a set of POPS statements
I          spare scalar adverb for use in procedures
INP        displays adverb values for task, verb, or proc - quick form
INPUTS     displays adverb values for task, verb, or proc - to msg file
ISBATCH    declares current AIPS to be, or not to be, batch-like
J          spare scalar adverb for use in procedures
KLEENEX    ends an AIPS interactive session wiping the slate klean
LENGTH     returns length of string to last non-blank character
LIST       displays the source code text for a POPS procedure
LN         returns the natural logarithm of the argument
LOG        returns the base-10 logarithm of the argument
MAX        returns the maximum of its two arguments
MIN        returns the minimum of its two arguments
MOD        returns remainder after division of 1st argument by 2nd
MODIFY     modifies the text of a line of a procedure and recompiles
MODULUS    returns square root of sum of squares of its two arguments
NOADVERB   Information about the lack of a defined adverb or verb
NUMTELL    selects POPS number of task which is the target of a TELL or ABORT
NX         General adverb referring to a number of things in the Y direction
NY         General adverb referring to a number of things in the Y direction
OUTPUTS    displays adverb values returned from task, verb, or proc
PARALLEL   Verb to set or show degree of parallelism
PASSWORD   Verb to change the current password for the login user
PCAT       Verb to list entries in the user's catalog (no log file).
PIPEAIPS   calibrating amplitude and phase, and imaging VLA data
POPSDAT    lists all POPS symbols, used to create them in MEmory files
POPSYM     Describes the symbols used in POPS
PRINTER    Verb to set or show the printer(s) used
PRINT      Print the value of an expression
PROCEDUR   Define a POPS procedure using procedure editor
PROC       Define a POPS procedure using procedure editor.
PSEUDO     Description of POPS pseudoverbs - obsolete list file
PSEUDOVB   Declares a name to be a symbol of type pseudoverb
QINP       displays adverb values for task, verb, or proc - restart form
RANDOM     Compute a random number from 0 to 1
READ       Read a value from the users terminal
REHEX      converts extended hex string to decimal
RENAME     Rename a file (UV or Image)
RESTART    Verb to trim the message log file and restart AIPS
RESTORE    Read POPS memory file from a common area.
RETURN     Exit a procedure allowing a higher level proc to continue.
RUN        Pseudoverb to read an external RUN files into AIPS.
SAVDEST    Verb to destroy all save files of a user.
SAVE       Pseudoverb to save full POPS environment in named file
SCALAR     Declares a variable to be a scalar in a procedure
SCANLENG   specify length of "scan"
SCRATCH    delete a procedure from the symbol table.
SETDEBUG   Verb to set the debug print and execution level
SG2RUN     Verb copies the K area to a text file suitable for RUN
SGDESTR    Verb-like to destroy named POPS environment save file
SGINDEX    Verb lists SAVE areas by name and time of last SAVE.
SIN        Compute the sine of a value
SLOT       Specifies AIPS catalog slot number
SPY        Verb to determine the execution status of all AIPS tasks
SQRT       Square root function
STORE      Store current POPS environment
STQUEUE    Verb to list pending TELL operations
STRING     Declare a symbol to be a string variable in POPS
SUBMIT     Verb which submits a batch work file to the job queue
SUBSTR     Function verb to specify a portion of a STRING variable
T1VERB     Temporary verb for testing (also T2VERB...T9VERB)
TAN        Tangent function
TAPES      Verb to show the TAPES(s) available
TASK       Name of a task
TGET       Verb-like gets adverbs from last GO of a task
TGINDEX    Verb lists those tasks for which TGET will work.
THEDATE    contains the date and time in a string form
THEN       Specified the action if an IF test is true
TIMDEST    Verb to destroy all files which are too old
TO         Specifies upper limit of a FOR loop
TPUT       Verb-like puts adverbs from a task in file for TGETs
TYPE       Type the value of an expression
USAVE      Pseudoverb to save full POPS environment in named file
VALUE      Convert a string to a numeric value
VERB       Declares a name to be a symbol of type verb
VERSION    Specify AIPS version or local task area
VGET       Verb-like gets adverbs from version task parameter save area
VGINDEX    Verb lists those tasks for which VGET will work.
VLA        puts the list of VLA antennas in the current file on stack
VLARUN     calibrating amplitude and phase, and imaging VLA data
VLBA       puts the list of VLBA antennas in the current file on stack
VLBAPIPE   applies amplitude and phase calibration procs to VLBA data
VLBARUN    applies amplitude and phase calibration procs to VLBA data
VNUMBER    Specifies the task parameter (VGET/VPUT) save area
VPUT       Verb-like puts adverbs from a task in files for VGETs
WAITTASK   halt AIPS until specified task is finished
WHILE      Start a conditional statement
XHELP      Accesses hypertext help system
X          spare scalar adverb for use in procedures
Y          spare scalar adverb for use in procedures
\end{verbatim}\eve

\sects{\hspace{0.5em}PROCEDUR}

\vskip 0.5pt\todx{ABOUT PROCEDURE}\iodx{procedure}
\bbve\begin{verbatim}
ANTNUM     Returns number of a named antenna
BASFIT     fits antenna locations from SN-table data
BOX2CC     Converts CLBOX in pixels to CCBOX in arc seconds
BREAK      procedure to TELL FILLM to break all current uv files, start new
CIRCLEVS   Sets RGBLEVS to fill LEVS with a circular color scheme
COTVLOD    Proc to load an image into a TV channel about a coordinate
CROSSPOL   Procedure to make complex poln. images and beam.
CRSFRING   Procedure to calibrate cross pol. delay and phase offsets
CXPOLN     Procedure to make complex poln. images and beam.
DEFCOLOR   Sets adverb PLCOLORS to match s default XAS TV
DOFARS     Procedure to aid in Faraday rotation synthesis using the FARS task
FEW        procedure to TELL FILLM to append incoming data to existing uv files
FITDISK    writes images / uv data w extensions to disk in FITS format
FLAMLEVS   Sets RGBLEVS to fill LEVS with a red flame color scheme
FXALIAS    least squares fit aliasing function and remove
FXAVG      Procedure to enable VLBA delay de-correlation corrections
GRANDOM    Finds a random number with mean 0 and rms 1
IMFRING    large image delay fitting with IM2CC and OOFRING
IMSCAL     large image self-cal with IM2CC and OOCAL
LINIMAGE   Build image cube from multi-IF data set
MANY       procedure to TELL FILLM to start new uv files on each scan
MAPPR      Simplified access to IMAGR
MAXTAB     Returns maximum version number of named table
MERGECAL   Procedure to merge calibration records after concatenation
OFFROAM    Procedure to clear the TV from a Roam condition
OOCAL      determines antenna complex gain with frequency-dependent models
OOFRING    fringe fit data to determine antenna calibration, delay, rate
PEELR      calibrates interfering sources in multi-facet imges
PFT        The Perley-Feigelson Test; see PFTLOAD.RUN, PFTEXEC.RUN
QUIT       procedure to TELL FILLM to stop at the end of the current scan
RAINLEVS   Sets RGBLEVS to fill LEVS with a rainbow color scheme
READISK    writes images / uv data w extensions to tape in FITS format
RUNWAIT    Runs a task and waits for it to finish
SCANTIME   Returns time range for a given scan number
SETXWIN    Procedure to set BLC and TRC with TV cursor
STEPLEVS   Sets RGBLEVS to fill LEVS with a repeated sequence of colors
STOP       procedure to TELL FILLM to break all current uv files and stop
STUFFR     averages together data sets in hour angle
TDEPEND    Time-dependent imaging procedure sequence
TD_SCANS   Time-dependent imaging procedure sequence: find intervals
TD_SSCAN   Time-dependent imaging procedure sequence: find intervals
TD_STEP3   Time-dependent imaging procedure "step 3"
TD_STEP5   Time-dependent imaging procedure sequence: later steps
TELFLM     procedure to TELL real-time FILLM a new APARM(1) value
TKBOX      Procedure to set a Clean box with the TK cursor
TKNBOXS    Procedure to set Clean boxes 1 - n with the TK cursor
TKWIN      Procedure to set BLC and TRC with Graphics cursor
TVALL      Procedure loads image to TV, shows labeled wedge, enhances
TVCOLORS   Sets adverb PLCOLORS to match the TV (DOTV=1) usage
TVDIST     determines spherical distance between two pixels on TV screen
TVFLUX     displays coordinates and values selected with the TV cursor
TVMAXFIT   displays fit pixel positions and intensity at maxima on TV
TVRESET    Reset the TV without erasing the image planes
VLACALIB   Runs CALIB and LISTR for VLA observation
VLACLCAL   Runs CLCAL and prints the results with LISTR
VLALIST    Runs LISTR for VLA observation
VLARESET   Reset calibration tables to a virginal state
VLASUMM    Plots selected contents of SN or CL files
VLATECR    Calculate ionospheric delay and Faraday rotation corrections
VLBAAMP    applies a-priori amplitude corrections to VLBA data
VLBACALA   applies a-priori amplitude corrections to VLBA data
VLBACCOR   applies a-priori amplitude corrections to VLBA data
VLBACPOL   Procedure to calibrate cross-polarization delays
VLBACRPL   Plots crosscorrelations
VLBAEOPS   Corrects Earth orientation parameters
VLBAFIX    Procedure that fixes VLBA data, if necessary
VLBAFPOL   Checks and corrects polarization labels for VLBA data
VLBAFQS    Copies different FQIDS to separate files
VLBAFRGP   Fringe fit phase referenced data and apply calibration
VLBAFRNG   Fringe fit data and apply calibration
VLBAIT     Procedure to read and process VLBA data (Phil Diamond)
VLBAKRGP   Fringe fit phase referenced data and apply calibration
VLBAKRNG   Fringe fit data and apply calibration
VLBALOAD   Loads VLBA data
VLBAMCAL   Merges redundant calibration data
VLBAMPCL   Calculates and applies manual instrumental phase calibration
VLBAPANG   Corrects for parallactic angle
VLBAPCOR   Calculates and applies instrumental phase calibration
VLBASNPL   Plots selected contents of SN or CL files
VLBASRT    Sorts VLBA data, if necessary
VLBASUBS   looks for subarrays in VLBA data
VLBASUMM   Prints a summary of a VLBI experiment
VLBATECR   Calculate ionospheric delay and Faraday rotation corrections
WRTDISK    writes images / uv data w extensions to tape in FITS format
\end{verbatim}\eve

%\vfill\eject
\sects{\hspace{0.5em}PSEUDOVE}

\vskip 0.5pt\todx{ABOUT PSEUDOVERB}\iodx{pseudo-verb}
\bbve\begin{verbatim}
Type:  General type of POPS symbol
Use:   Pseudo-verbs are magic symbols which cause FORTRAN routines to
       carry out specific actions.  Unlike verbs, pseudo-verbs are
       executed as soon as they are encountered by the compiler even
       in compile mode.  In general, the FORTRAN routines which are
       invoked will parse the remainder of the input line under
       special, non-standard rules.  Any normal code typed on the line
       ahead of the pseudo-verb will not be executed.
Grammar:  See the HELP listings for the specific pseudoverb.
Examples:    HELP  HELP
             ARRAY   JUNK(4, -7 TO 9)
             PROC DUMMY (I,J)
             LIST DUMMY
             DEBUG TRUE
             INPUTS MLOAD
****************************************************************

ABORTASK   stops a running task
ARRAY      Declares POPS symbol name and dimensions
COMPRESS   recovers unused POPS address space and new symbols
CORE       displays the used and total space used by parts of POPS table
DEBUG      turns on/off the POPS-language's debug messages
EDIT       enter edit-a-procedure mode in the POPS language
ELSE       starts POPS code done if an IF condition is false (IF-THEN..)
ENDBATCH   terminates input to batch work file
ENDEDIT    terminates procedure edit mode of POPS input
ERASE      removes one or more lines from a POPS procedure
FINISH     terminates the entry and compilation of a procedure
GET        restores previously SAVEd full POPS environment
IF         causes conditional execution of a set of POPS statements
ISBATCH    declares current AIPS to be, or not to be, batch-like
LIST       displays the source code text for a POPS procedure
MODIFY     modifies the text of a line of a procedure and recompiles
MSGKILL    turns on/off the recording of messages in the message file
TELL       Send parameters to tasks that know to read them on the fly
WHILE      Start a conditional statement
\end{verbatim}\eve

\sects{\hspace{0.5em}RUN}

\vskip 0.5pt\todx{ABOUT RUN}\todx{RUN}
\bbve\begin{verbatim}
ANCHECK    Checks By sign in Antenna files
DDT        verifies correctness and performance using standard problems
DDTSAVE    verifies correctness and performance using standard problems
DOOSRO     calibrating amplitude and phase, and imaging VLA data
DOVLAMP    Produces amp calibration file for phased-VLA VLBI data
PIPEAIPS   calibrating amplitude and phase, and imaging VLA data
VLAC       verifies correctness of continuum calibration software
VLACSAVE   verifies correctness of continuum calibration
VLAL       verifies correctness of spectral line calibration software
VLALSAVE   verifies correctness of continuum calibration
VLAPROCS   Procedures to simplify the reduction of VLBA data
VLARUN     calibrating amplitude and phase, and imaging VLA data
VLBAARCH   Procedure to archive VLBA correlator data
VLBAPIPE   applies amplitude and phase calibration procs to VLBA data
VLBARUN    applies amplitude and phase calibration procs to VLBA data
VLBAUTIL   Procedures to simplify the reduction of VLBA data
VLBDDT     Verification tests using simulated data
WRTPROCS   Procedures to simplify the reduction of VLBA data
Y2K        verifies correctness and performance using standard problems
Y2KSAVE    verifies correctness and performance using standard problems
\end{verbatim}\eve

\sects{\hspace{0.5em}SINGLEDI}

\vskip 0.5pt\todx{ABOUT SINGLEDISH}\iodx{single-dish}
\bbve\begin{verbatim}
BSAVG      Task to do an FFT-weighted sum of beam-switched images
BSCLN      Hogbom Clean on beam-switched difference image
BSCOR      Combines two beam-switched images
BSFIX      Corrects the ra/dec offsets recorded by the 12m
BSGRD      Task to image beam-switched single-dish data
BSMOD      creates single-dish UV beam-switched data with model sources
BSROT      modifies SD beam-switch continuum data for error in throw
BSTST      Graphical display of solutions to frequency-switched data
CSCOR      applies specified corrections to CS tables
DATA2IN    specifies name of input FITS disk file
DATAIN     specifies name of input FITS disk file
DIFUV      Outputs the difference of two matching input uv data sets
INDXH      writes index file describing contents of UV data base
INDXR      writes index file describing contents of UV data base
OTFBS      Translates on-the-fly continuum SDD format to AIPS UV file
OTFIN      Lists on-the-fly single-dish SDD format data files
OTFUV      Translates on-the-fly single-dish SDD format to AIPS UV file
PRTSD      prints contents of AIPS single-dish data sets
SDCAL      Task to apply single dish calibration
SDGRD      Task to select and image random-position single-dish data
SDIMG      Task to select and image random-position single-dish data
SDLSF      least squares fit to channels and subtracts from SD uv data
SDMOD      modifies single-dish UV data with model sources
SDTUV      Task to convert SD table files to UV like data.
SDVEL      shifts spectral-line single-dish data to a given velocity
VTEST      Measures velocity discrepancy across fields
WTSUM      Task to do a a sum of images weighted by other images
\end{verbatim}\eve

\sects{\hspace{0.5em}SPECTRAL}

\vskip 0.5pt\todx{ABOUT SPECTRAL}\iodx{spectral-line}
\bbve\begin{verbatim}
Almost all parts of AIPS are general enough to handle mutiple
dimensions of data including multiple frequency channels in the
uv domain and 3 or more dimensional "cubes" in the image domain.
ACFIT      Determine antenna gains from autocorrelations
AGAUS      Fits 1-dimensional Gaussians to absorption-line spectra
ALTDEF     Sets frequency vs velocity relationship into image header
ALTSWTCH   Switches between frequency and velocity in image header
AVSPC      Averages uv-data in the frequency domain
BASRM      Task to remove a spectral baseline from total power spectra
BCHAN      sets the beginning channel number
BDAPL      Applies a BD table to another data set
BLOAT      converts line data to greater number channels
BLSUM      sums images over irregular sub-images, displays spectra
BPASS      computes spectral bandpass correction table
BPASSPRM   Control adverb array for bandpass calibration
BPERR      Print and plot BPASS closure outputs
BPLOT      Plots bandpass tables in 2 dimensions as function of time
BPSMO      Smooths or interpolates bandpass tables to regular times
BPVER      specifies the version of the bandpass table to be applied
CHANNEL    sets the spectral channel number
CHANSEL    Array of start, stop, increment channel numbers to average
CHINC      the increment between selected channels
CPASS      computes polynomial spectral bandpass correction table
CUBIT      Model a galaxy's density and velocity distribution from full cube
CVEL       shifts spectral-line UV data to a given velocity
DCHANSEL   Array of start, stop, increment channel #S + IF to avoid
DOAPPLY    Flag to indicate whether an operation is applied to the data
ECHAN      define an end for a range of channel numbers
FIXAL      least squares fit aliasing function and remove
FLGIT      flags data based on the rms of the spectrum
FQUBE      collects n-dimensional images into n+1-dimensional FREQID image
FRMAP      Task to build a map using fringe rate spectra
FRPLT      Task to plot fringe rate spectra
FTFLG      interactive flagging of UV data in channel-time using the TV
FXALIAS    least squares fit aliasing function and remove
HLPFTFLG   Interactive time-channel visibility Editor - run-time help
HLPPCFLG   Interactive time-channel PC table Editor PCFLG - run-time help
HLPSPFLG   Interactive time-channel visibility Editor SPFLG - run-time help
HLPTVHUI   Interactive intensity-hue-saturation display - run-time help
HLPTVRGB   Interactive red-green-blue display - run-time help
HLPZAMAN   Fits 1-dimensional Zeeman model to absorption data - run-time help
HLPZEMAN   Fits 1-dimensional Zeeman model to data - run-time help
ICHANSEL   Array of start, stop, increment channel #S + IF to average
IMLIN      Fits and removes continuum emission from cube
IRING      integrates intensity / flux in rings / ellipses
ISPEC      Plots and prints spectrum of region of a cube
MCUBE      collects n-dimensional images into n+1-dimensional image
MODAB      Makes simple absorption/emission spectral-line image in I/V
MORIF      Combines IFs or breaks spectral windows into multiple windows (IFs)
NCHAV      Number of channels averaged in an operation
ORDER      Adverb used usually to specify the order of polynomial fit
PCASS      Finds amplitude bandpass shape from pulse-cal table data
PCFIT      Finds delays and phases using a pulse-cal (PC) table
PCPLT      Plots pulse-cal tables in 2 dimensions as function of time
PCVEL      shifts spectral-line UV data to a given velocity: planet version
PDVER      specifies the version of the spetral polarization table to use
PLCUB      Task to plot intensity vs x panels on grid of y,z pixels
POLANGLE   Intrinsic polarization angles for up to 30 sources
POSSM      Task to plot total and cross-power spectra.
QUFIX      determines Right minus Left phase difference, corrects cal files
REIFS      Breaks spectral windows into multiple spectral windows (IFs)
RM2PL      Plots spectrum of a pixel with RMFIT fit
RMFIT      Fits 1-dimensional polarization spectrum to Q/U cube
RSPEC      Plots and prints spectrum of rms of a cube
SDLSF      least squares fit to channels and subtracts from SD uv data
SDVEL      shifts spectral-line single-dish data to a given velocity
SERCH      Finds line signals in transposed data cube
SMOTH      Task to smooth a subimage from upto a 7-dim. image
SPECINDX   Spectral index used to correct calibrations
SPECPARM   Spectral index per polarization per source
SPECTRAL   Flag to indicate whether an operation is spectral or continuum
SPECURVE   Spectral index survature used to correct calibrations
SPFLG      interactive flagging of UV data in channel-TB using the TV
SPMOD      Modify UV database by adding a model with spectral lines
SQASH      Task to collapse several planes in a cube into one plane or row
SYSVEL     Systemic velocity
UJOIN      modifies UV data converting IFs to spectral channels
UV2TB      Converts UV autocorrelation spectra to tables
UVBAS      averages several channels and subtracts from uv data.
UVDEC      Decrements the number of spectral channels, keeping every nth
UVFRE      Makes one data set have the spectral structure of another
UVGLU      Glues UV data frequency blocks back together
UVLIN      Fits and removes continuum visibility spectrum, also can flag
UVLSD      least squares fit to channels and divides the uv data.
UVLSF      least squares fit to channels and subtracts from uv data.
UVMLN      edits data based on the rms of the spectrum
UVMOD      Modify UV database by adding a model incl spectral index
VBGLU      Glues together data from multiple passes thru the VLBA corr.
VLBABPSS   computes spectral bandpass correction table
VTEST      Measures velocity discrepancy across fields
WTSUM      Task to do a a sum of images weighted by other images
XBASL      Fits and subtracts nth-order baselines from cube (x axis)
XG2PL      Plots spectrum of a pixel with XGAUS/AGAUS and ZEMAN/ZAMAN fits
XGAUS      Fits 1-dimensional Gaussians to images: restartable
XPLOT      Plots image rows one at a time on the graphics or TV screen
ZAMAN      Fits 1-dimensional Zeeman model to absorption-line data
ZEMAN      Fits 1-dimensional Zeeman model to data
\end{verbatim}\eve

\sects{\hspace{0.5em}TABLE}

\vskip 0.5pt\todx{ABOUT TABLE}\iodx{tables}
\bbve\begin{verbatim}
BSPRT      print BS tables
CLCOP      copy CL/SN file calibration between polarizations or IFs
CLINV      copy CL/SN file inverting the calibration
DOTABLE    selects use of table-format for data
EXTAB      exports AIPS table data as tab-separated text
HF2SV      convert HF tables from FRING/MBDLY to form used by Calc/Solve
HFPRT      write HF tables from CL2HF
HLPEDIBP   Interactive BP table uv-data editor BPEDT - run-time help
HLPEDICL   Interactive SN/CL table uv-data editor - run-time help
HLPEDIPC   Interactive PC table editor PCEDT - run-time help
HLPEDIPD   Interactive PD table editor PDEDT - run-time help
HLPEDISN   Interactive SN/CL table (not UV) editor - run-time help
HLPEDISS   Interactive SY table (not UV) editor - run-time help
HLPEDISY   Interactive SY table uv-data editor EDITA - run-time help
HLPEDITS   Interactive TY table (not UV) editor - run-time help
HLPEDITY   Interactive TY table uv-data editor EDITA - run-time help
MFPRT      prints MF tables in a format needed by modelling software
OBEDT      Task to flag data of orbiting antennas
OBTAB      Recalculate orbit parameters and other spacecraft info
OFLAG      uses on-line flag table information to write a flag table
PRTAB      prints any table-format extension file
PRTOF      prints on-line flag table information
SNCOP      Task to copy SN table averaging some input IFs
TABED      Task to edit tables
TACOP      task to copy tables, other extension files
TAFLG      Flags data in a Table extension file
TAMRG      Task to merge table rows under specified conditions
TAPLT      Plots data from a Table extension file
TAPPE      Task to append 2 tables and merge to output table
TASAV      Task to copy all extension tables to a dummy uv or map file
TASRT      Task to sort extension tables.
TBDIF      Compare entries in two tables
TBIN       Reads a text file AIPS table into AIPS
TBOUT      Writes an AIPS table into a text file for user editting.
TYCOP      copy TY or SY table calibration between IFs
VLBAMCAL   Merges redundant calibration data
\end{verbatim}\eve

\sects{\hspace{0.5em}TAPE}

\vskip 0.5pt\todx{ABOUT TAPE}\iodx{tape}\iodx{magnetic tape}
\bbve\begin{verbatim}
ATLOD      Reads ATCA data in RPFITS dormat into AIPS
AVEOT      Advances tape to end-of-information point
AVFILE     Moves tape forward or back to end-of-file marks
AVMAP      Advance tape by one image (IBM-CV = obsolete tape file)
AVTP       Positions tape to desired file
BAKLD      reads all files of a catalog entry from BAKTP tape
BAKTP      writes all files of a catalog entry to tape in host format
BLOCKING   specifies blocking factor to use on e.g. tape records
DENSITY    gives the desired tape density
DISMOUNT   disables a magnetic tape and dismounts it from the tape drive
DOEOF      selects end-of-file writing or reading until
DOEOT      selects tape positioning before operation: present or EOI
DONEWTAB   do we make new tables, use a new table format, etc.
DOTABLE    selects use of table-format for data
DOTWO      do we make two of something
FILLM      reads VLA on-line/archive format uv data tapes (post Jan 88)
FILLR      reads old VLA on-line-system tapes into AIPS
FIT2A      reads the fits input file and records it to the output ascii file
FITAB      writes images / uv data w extensions to tape in FITS format
FITLD      Reads FITS files to load images or UV (IDI or UVFITS) data to disk
FITTP      writes images / uv data w extensions to tape in FITS format
FORMAT     gives a format code number: e.g. FITS accuracy required
GSCAT      reads Fits Guide star catalog file
IMLOD      reads tape to load images to disk
INTAPE     specifies the input tape drive number
MOUNT      makes a tape drive available to user's AIPS and tasks
NFILES     The number of files to skip, usually on a tape.
NPIECE     The number of pieces to make
OUTTAPE    The output tape drive number.
PRTTP      prints contents of tapes, all supported formats
QUANTIZE   Quantization level to use
REMHOST    gives the name of another computer which will provide service
REMTAPE    gives the number of another computer's tape device
REWIND     Verb to rewind a tape
TAPES      Verb to show the TAPES(s) available
TCOPY      Tape to tape copy with some disk FITS support
TPHEAD     Verb to list image header from FITS or IBM-CV tape
TPMON      Information about the TPMON "Daemon"
UVLOD      Read export or FITS data from a tape or disk
VLAMODE    VLA observing mode
VLAOBS     Observing program or part of observer's name
\end{verbatim}\eve

\vfill\eject
\sects{\hspace{0.5em}TASK}

\vskip 0.5pt\todx{ABOUT TASK}\iodx{tasks}
\bbve\begin{verbatim}
TASKS
Type:  General type of POPS symbol (not in symbol table)
Use:   Tasks are separate programs which may be started by
       AIPS and which receive their input parameters from
       AIPS.  In the interactive AIPS, tasks run
       asynchronously from AIPS.  In the batch AIPS,
       the language processor waits for each task to finish
       before starting another one.
Grammar:    TASK = 'name' ;   GO
       will cause the task whose name is assigned to the
       string adverb TASK to be started.  Note: the name
       should have no leading blanks and should be no longer
       than 5 characters.
Alternative grammar:    GO  name ;
       where name is the name of the task to be run.
Related adverbs:
   TASK       Task name
   DOWAIT     On "GO", wait for task completion before returning
                 to AIPS control
   VERSION    Version of task to be executed.
   QCREATE    Create files by fast or more reliable methods
Related verbs:
   GO         Initiate a shed task
   HELP       List information about a task
   INP        List adverb values for a task
   INPUTS     Same as INP but also written to MSG file
   SPY        Inquire which tasks are active
   WAITTASK   Suspend AIPS operation until a specific task
                 is complete
   ABORTASK   Kill a task immediately
   TGET       Get adverb values from last execution of TASK
   TPUT       Save adverb values without execution of TASK
   TGINDEX    List all TGET/SAVE files
****************************************************************

ACCOR      Corrects cross amplitudes using auto correlation measurements
ACFIT      Determine antenna gains from autocorrelations
ACLIP      edits suto-corr data for amplitudes, phases, and weights out of range
ACSCL      Corrects cross amplitudes using auto correlation measurements
ADDIF      Adds an IF axis to a uv data set
AFARS      Is used after FARS to determine Position and Value of the maximum
AFILE      sorts and edits MkIII correlator A-file.
AGAUS      Fits 1-dimensional Gaussians to absorption-line spectra
AHIST      Task to convert image intensities by adaptive histogram
AIPSB      AIPS main program for executing batch jobs
AIPSC      AIPS main program for testing and queuing batch jobs
AIPS       AIPS main program for interactive use
ALVAR      plots the Allan Variance statistic of a UV data set
ALVPR      prints statistics on the Allan Variance of a UV data set
ANBPL      plots and prints  uv data converted to antenna based values
ANCAL      Places antenna-based Tsys and gain corrections in CL table
ANTAB      Read amplitude calibration information into AIPS
APCAL      Apply TY and GC tables to generate an SN table
APCLN      Deconvolves images with CLEAN algorithm
APGPS      Apply GPS-derived ionospheric corrections
APGS       deconvolves image with Gerchberg-Saxton algorithm
APVC       Deconvolves images with van Cittert algorithm
ATLOD      Reads ATCA data in RPFITS dormat into AIPS
ATMCA      Determines delay/phase gradient from calibrator observations
AVER       Averages over time UV data sets in 'BT' order
AVSPC      Averages uv-data in the frequency domain
AVTP       Positions tape to desired file
BAKLD      reads all files of a catalog entry from BAKTP tape
BAKTP      writes all files of a catalog entry to tape in host format
BASRM      Task to remove a spectral baseline from total power spectra
BATER      stand-alone program to prepare and submit batch jobs
BDAPL      Applies a BD table to another data set
BDEPO      computes depolarization due to rotation measure gradients
BDF2AIPS   Read EVLA ASDM/BDF data into AIPS
BLANK      blanks out selected, e.g. non-signal, portions of an image
BLAPP      applies baseline-based fringe solutions a la BLAPP
BLAVG      Average cross-polarized UV data over baselines.
BLCAL      Compute closure offset corrections
BLCHN      Compute closure offset corrections on a channel-by-channel basis
BLING      find residual rate and delay on individual baselines
BLOAT      converts line data to greater number channels
BLSUM      sums images over irregular sub-images, displays spectra
BLWUP      Blow up an image by any positive integer factor.
BOXES      Adds Clean boxes to BOXFILE around sources from a list
BPASS      computes spectral bandpass correction table
BPCOR      Correct BP table.
BPEDT      Interactive TV task to edit uv data based on BP tables
BPERR      Print and plot BPASS closure outputs
BPLOT      Plots bandpass tables in 2 dimensions as function of time
BPSMO      Smooths or interpolates bandpass tables to regular times
BPWAY      Determines channel-dependent relative weights
BPWGT      Calibrates data and scales weights by bandpass correction
BSAVG      Task to do an FFT-weighted sum of beam-switched images
BSCAN      seeks best scan to use for phase cal, fringe search, ..
BSCLN      Hogbom Clean on beam-switched difference image
BSCOR      Combines two beam-switched images
BSFIX      Corrects the ra/dec offsets recorded by the 12m
BSGEO      Beam-switched Az-El image to RA-Dec image translation
BSGRD      Task to image beam-switched single-dish data
BSMAP      images weak sources with closure phases
BSMOD      creates single-dish UV beam-switched data with model sources
BSPRT      print BS tables
BSROT      modifies SD beam-switch continuum data for error in throw
BSTST      Graphical display of solutions to frequency-switched data
CALIB      determines antenna calibration: complex gain
CALRD      Reads calibrator source model-image FITS file
CALWR      writes calibrator source model images w CC files to FITS disk files
CANDY      user-definable (paraform) task to create an AIPS image
CANPL      translates a plot file to a Canon printer/plotter
CAPLT      plots closure amplitude and model from CC file
CC2IM      Make model image from a CC file
CCEDT      Select CC components in BOXes and above mininum flux.
CCFND      prints the contents of a Clean Components extension file.
CCGAU      Converts point CLEAN components to Gaussians
CCMOD      generates clean components to fit specified source model
CCMRG      sums all clean components at the same pixel
CCNTR      generate a contour plot file from an image
CCRES      Removes or restores a CC file to a map with a gaussian beam.
CCSEL      Select signifigant CC components
CENTR      modifies UV data to center the reference channel
CHKFC      makes images of Clean boxes from Boxfile
CL2HF      Convert CL table to HF table
CLCAL      merges and smooths SN tables, applies them to CL tables
CLCOP      copy CL/SN file calibration between polarizations or IFs
CLCOR      applies user-selected corrections to the calibration CL table
CLINV      copy CL/SN file inverting the calibration
CLIP       edits data based on amplitudes, phases, and weights out of range
CLIPM
CLPLT      plots closure phase and model from CC file
CLSMO      smooths a calibration CL table
CLVLB      Corrects CL table gains for pointing offsets in VLBI data
CNTR       generate a contour plot file or TV plot from an image
COHER      Baseline Phase coherence measurement
COMB       combines two images by a variety of mathematical methods
CONFI      Optimize array configuration by minimum side lobes
CONPL      Plots AIPS gridding convolution functions
CONVL      convolves an image with a gaussian or another image
CORER      calculates correlator statistics and flags bad ones
CORFQ      corrects uvw for incorrect observing frequency
CPASS      computes polynomial spectral bandpass correction table
CPYRT      replaces history with readme file, inserts copyright
CSCOR      applies specified corrections to CS tables
CUBIT      Model a galaxy's density and velocity distribution from full cube
CVEL       shifts spectral-line UV data to a given velocity
CXCLN      Complex Hogbom CLEAN
DAYFX      Fixes day number problems left by FILLM
DBAPP      appends one or more data sets to the output data set
DBCON      concatenates two UV data sets
DCONV      deconvolves a gaussian from an image
DECOR      Measures the decorrelation between channels and IF of uv data
DEFLG      edits data based on decorrelation over channels and time
DELZN      Determines residual atmosphere depth at zenith and clock errors
DESCM      copies a portion of a UV data set
DFCOR      applies user-selected corrections to the calibration CL table
DFQID      modifies UV data changing the indicated FQIDs
DFTIM      Makes image of DFT at arbitrary point showing time vs frequency
DFTPL      plots DFT of a UV data set at arbitrary point versus time
DIFRL      divides the RR data by LL data
DIFUV      Outputs the difference of two matching input uv data sets
DISKU      shows disk use by one or all users
DQUAL      Rearranges source list, dropping qualifiers
DRCHK      stand-alone program checks system setup files for consistency
DSKEW      Geometric interpolation correction for skew
DSORC      copies a data set elliminating some source numbers
DSTOK      Drops the cross-hand polarizations
DTCHK      Task to check results of a test using simulated data.
DTSIM      Generate fake UV data
DTSUM      Task to provide a summary of the contents of a dataset
EDITA      Interactive TV task to edit uv data based on TY/SY/SN/CL tables
EDITR      Interactive baseline-oriented visibility editor using the TV
ELFIT      Plots/fits selected contents of SN, TY, SY, PC or CL files
ELINT      Determines and removes gain dependence on elevation
EVASN      Evaluates statistics in SN/CL tables
EVAUV      Subtracts & divides a model into UV data, does statistics on results
EXTAB      exports AIPS table data as tab-separated text
FACES      makes images of catalog sources for initial calibration
FARAD      add ionospheric Faraday rotation to CL table
FARS       Faraday rotation synthesis based on the brightness vs wavelength
FETCH      Reads an image from an external text file.
FFT        takes Fourier Transform of an image or images
FGCNT      Counts samples comparing two flag tables
FGDIF      Compares affect of 2 FG tables
FGPLT      Plots selected contents of FG table
FGSPW      Flags bad spectral windows
FGTAB      writes frequency-range flags to a text file to be read by UVFLG
FILIT      Interactive BOXFILE editing with facet images
FILLM      reads VLA on-line/archive format uv data tapes (post Jan 88)
FILLR      reads old VLA on-line-system tapes into AIPS
FINDR      Find normal values for a uv data set
FIT2A      reads the fits input file and records it to the output ascii file
FITAB      writes images / uv data w extensions to tape in FITS format
FITLD      Reads FITS files to load images or UV (IDI or UVFITS) data to disk
FITTP      writes images / uv data w extensions to tape in FITS format
FIXAL      least squares fit aliasing function and remove
FIXAN      fixes the contents of the ANtenna extension file
FIXBX      converts a BOXFILE to another for input to IMAGR
FIXRL      correctes right vs left polarizations for a list of antennas
FIXWT      Modify weights to reflect amplitude scatter of data
FLAGR      Edit data based on internal RMS, amplitudes, weights
FLATN      Re-grid multiple fields into one image incl sensitivity
FLGIT      flags data based on the rms of the spectrum
FLOPM      reverses the spectral order of UV data, can fix VLA error
FQUBE      collects n-dimensional images into n+1-dimensional FREQID image
FRCAL      Faraday rotation self calibration task
FRING      fringe fit data to determine antenna calibration, delay, rate
FRMAP      Task to build a map using fringe rate spectra
FRPLT      Task to plot fringe rate spectra
FTFLG      interactive flagging of UV data in channel-time using the TV
FUDGE      modifies UV data with user's algorithm: paraform task
FXPOL      Corrects VLBA polarization assignments
FXTIM      fixes start date so all times are positive
FXVLA      Task to correct VLA data for on-line errors in special cases.
FXVLB      Builds a CQ table to enable VLBA correlator loss corrections
GAL        Determine parameters from a velocity field
GCPLT      Plots gain curves from text files
GETJY      determines calibrator flux densities
GLENS      models galaxy gravitational lens acting on 3 component source
GPSDL      Calculate ionospheric delay and Faraday rotation corrections
GREYS      plots images as contours over multi-level grey
GRIPR      standalone program to enter suggestions/complaints to AIPS
GSCAT      reads Fits Guide star catalog file
GSTAR      Task to read a Guide Star (UK) table and create an ST table.
HA2TI      Converts data processed by TI2HA (STUFFER) back to real times
HAFIX      Recomputes u,v,w when time is hour angle (UVdata is output of TI2HA)
HF2SV      convert HF tables from FRING/MBDLY to form used by Calc/Solve
HFPRT      write HF tables from CL2HF
HGEOM      interpolates image to different gridding and/or geometry
HISEQ      task to translate image by histogram equalization
HLPAGAUS   Interactive Gaussian absorption fitting task AGAUS - run-time help
HLPCLEAN   Cleaning tasks  - run-time help
HLPEDIBP   Interactive BP table uv-data editor BPEDT - run-time help
HLPEDICL   Interactive SN/CL table uv-data editor - run-time help
HLPEDIPC   Interactive PC table editor PCEDT - run-time help
HLPEDIPD   Interactive PD table editor PDEDT - run-time help
HLPEDISN   Interactive SN/CL table (not UV) editor - run-time help
HLPEDISS   Interactive SY table (not UV) editor - run-time help
HLPEDISY   Interactive SY table uv-data editor EDITA - run-time help
HLPEDITS   Interactive TY table (not UV) editor - run-time help
HLPEDITY   Interactive TY table uv-data editor EDITA - run-time help
HLPEDIUV   Interactive uv-data editor EDITR - run-time help
HLPFILIT   Interactive Clean box file editing with image display - run-time help
HLPFTFLG   Interactive time-channel visibility Editor - run-time help
HLPIBLED   Interactive Baseline based visibility Editor - run-time help
HLPPCFLG   Interactive time-channel PC table Editor PCFLG - run-time help
HLPPLAYR   OOP TV class demonstration task - run-time help
HLPRMFIT   Polarization fitting task RMFIT - run-time help
HLPSCIMG   Full-featured image plus self-cal loops, editing - run-time help
HLPSCMAP   Imaging plus self-cal and editing SCMAP - run-time help
HLPSPFLG   Interactive time-channel visibility Editor SPFLG - run-time help
HLPTVFLG   Interactive time-baseline visibility Editor TVFLG - run-time help
HLPTVHLD   Interactive image display with histogram equalization - run-time help
HLPTVHUI   Interactive intensity-hue-saturation display - run-time help
HLPTVRGB   Interactive red-green-blue display - run-time help
HLPTVSAD   Find & fit Gaussians to an image with interaction - run-time help
HLPTVSPC   Interactive display of spectra from a cube - run-time help
HLPUFLAG   Edit uv-data on a grid UFLAG - run-time help
HLPWIPER   edit uv data from UVPLT-like plot WIPER - run-time help
HLPXGAUS   Interactive Gaussian fitting task XGAUS - run-time help
HLPZAMAN   Fits 1-dimensional Zeeman model to absorption data - run-time help
HLPZEMAN   Fits 1-dimensional Zeeman model to data - run-time help
HOLGR      Read & process holography visibility data to telescope images
HOLOG      Read & process holography visibility data to telescope images
HORUS      makes images from unsorted UV data, applying any calibration
HUINT      make RGB image from images of intensity & hue, like TVHUEINT
IBLED      Interactive BaseLine based visibility EDitor
IM2CC      Task to convert an image to multi-facet Clean Components
IM2UV      converts an image to a visibility data set
IMAGR      Wide-field and/or wide-frequency Cleaning / imaging task.
IMCLP      Clip an image to a specified range.
IMEAN      displays the mean & extrema and plots histogram of an image
IMERG      merges images of different spatial resolutions
IMFIT      Fits Gaussians to portions of an image
IMFLT      fits and removes a background intensity plane from an image
IMLHS      converts images to luminosity/hue TV display
IMLIN      Fits and removes continuum emission from cube
IMLOD      reads tape to load images to disk
IMMOD      adds images of model objects to an image
IMRMS      Plot IMEAN rms answers
IMTXT      Write an image to an external text file.
IMVIM      plots one image's values against another's
INDXH      writes index file describing contents of UV data base
INDXR      writes index file describing contents of UV data base
IRING      integrates intensity / flux in rings / ellipses
ISPEC      Plots and prints spectrum of region of a cube
JMFIT      Fits Gaussians to portions of an image
KNTR       make a contour/grey plot file from an image w multiple panels
KRING      fringe fit data to determine antenna calibration, delay, rate
LAYER      Task to create an RGB image from multiple images
LDGPS      load GPS data from an ASCII file
LGEOM      regrids images with rotation, shift using interpolation
LISTR      prints contents of UV data sets and assoc. calibration tables
LOCIT      fits antenna locations from SN-table data
LPCAL      Determines instrumental polarization for UV data
LTESS      makes mosaic images by linear combination
LWPLA      translates plot file(s) to a PostScript printer or file
M3TAR      translate Haystack MKIII VLBI format "A" TAR's into AIPS
MANDL      creates an image of a subset of the Mandlebrot Set
MAPBM      Map VLA beam polarization
MATCH      changes antenna, source, FQ numbers to match a data set
MATHS      operates on an image with a choice of mathematical functions
MBDLY      Fits multiband delays from IF phases, updates SN table
MCUBE      collects n-dimensional images into n+1-dimensional image
MEDI       combines four images by a variety of mathematical methods
MF2ST      Task to generate an ST ext. file from Model Fit ext. file
MFPRT      prints MF tables in a format needed by modelling software
MK3IN      translate Haystack MKIII VLBI format "A" tapes into AIPS
MK3TX      extract text files from a MKIII VLBI archive tape
MODAB      Makes simple absorption/emission spectral-line image in I/V
MODIM      adds images of model objects to image cubes in IQU polarization
MODSP      adds images of model objects to image cubes in I/V polarization
MODVF      task to create a warped velocity field
MOMFT      calculates images of moments of a sub-image
MOMNT      calculates images of moments along x-axis (vel, freq, ch)
MORIF      Combines IFs or breaks spectral windows into multiple windows (IFs)
MOVE       Task to copy or move data from one user or disk to another
MSORT      Sort a UV dataset into a specified order
MULIF      Change number of IFs in output
MULTI      Task to convert single-source to multi-source UV data
MWFLT      applies linear & non-linear filters to images
MX         makes images & deconvolves using UV data directly - replaced
NANS       reads an image or a UV data set and looks for NaNs
NINER      Applies various 3x3 area operaters to an image.
NNLSQ      Non-Negative-Least-Squares decomposition of spectrum
NOBAT      Task to lock lower priority users out of the AP
NOIFS      makes all IFs into single spectrum
OBEDT      Task to flag data of orbiting antennas
OBPLT      Plot columns of an OB table.
OBTAB      Recalculate orbit parameters and other spacecraft info
OFLAG      uses on-line flag table information to write a flag table
OGEOM      Simple image rotation, scaling, and translation
OHGEO      Geometric interpolation with correction for 3-D effects
OMFIT      Fits sources and, optionally, a self-cal model to uv data
OOSRT      Sort a UV dataset into a specified order
OOSUB      Subtracts/divides a model from/into a uv data base
OTFBS      Translates on-the-fly continuum SDD format to AIPS UV file
OTFIN      Lists on-the-fly single-dish SDD format data files
OTFUV      Translates on-the-fly single-dish SDD format to AIPS UV file
PADIM      Task to increase image size by padding with some value
PANEL      Convert HOLOG output to panel adjustment table
PASTE      Pastes a selected subimage of one image into another.
PATGN      Task to create a user specified test or primary-beam pattern
PBCOR      Task to apply or correct an image for a primary beam
PBEAM      Fits the analytic function to the measured values of the beam
PCAL       Determines instrumental polarization for UV data
PCASS      Finds amplitude bandpass shape from pulse-cal table data
PCAVG      Averages pulse-cal (PC) tables over time
PCCOR      Corrects phases using  PCAL tones data from PC table
PCEDT      Interactive TV task to edit pulse-cal (PC) tables
PCFIT      Finds delays and phases using a pulse-cal (PC) table
PCFLG      interactive flagging of Pulse-cal data in channel-TB using the TV
PCHIS      Generates a histogram plot file from text input, e.g. from PCRMS
PCLOD      Reads ascii file containing pulse-cal info to PC table.
PCNTR      Generate plot file with contours plus polarization vectors
PCPLT      Plots pulse-cal tables in 2 dimensions as function of time
PCRMS      Finds statistics of a pulse-cal table; flags bad times and channels
PCVEL      shifts spectral-line UV data to a given velocity: planet version
PDEDT      Interactive TV task to edit polarization D-term (PD) tables
PEEK       fits pointing model function to output from the VLA
PFPL1      Paraform Task to generate a plot file: (does grey scale)
PFPL2      Paraform Task to generate a plot file: (slice intensity)
PFPL3      Paraform Task to generate a plot file: (does histogram)
PGEOM      Task to transform an image into polar coordinates.
PHASE      Baseline Phase coherence measurement
PHCLN      PHCLN has been removed, use PHAT adverb in APCLN.
PHNEG      Negates a UV datafile's visibility phase.
PHSRF      Perform phase-referencing within a spectral line database.
PLAYR      Verb to load an image into a TV channel
PLCUB      Task to plot intensity vs x panels on grid of y,z pixels
PLOTC      plots color schems used by 3-color plot tasks
PLOTR      Basic task to generate a plot file from text input
PLROW      Plot intensity of a series of rows with an offset.
POLCO      Task to correct polarization maps for Ricean bias
POLSN      Make a SN table from cross polarized fringe fit
POSSM      Task to plot total and cross-power spectra.
PROFL      Generates plot file for a profile display.
PRTAB      prints any table-format extension file
PRTAC      prints contents and summaries of the accounting file
PRTAN      prints the contents of the ANtenna extension file
PRTCC      prints the contents of a Clean Components extension file.
PRTIM      prints image intensities from an MA catalog entry
PRTOF      prints on-line flag table information
PRTPL      Task to send a plot file to the line printer
PRTSD      prints contents of AIPS single-dish data sets
PRTSY      Task to print statistics from the SY table
PRTTP      prints contents of tapes, all supported formats
PRTUV      prints contents of a visibility (UV) data set
QMSPL      Task to send a plot file to the QMS printer/plotter
QUACK      Flags beginning or end portions of UV-data scans
QUFIX      determines Right minus Left phase difference, corrects cal files
QUOUT      writes text file of Q, U versus frequency to be used by RLDIF
QUXTR      extracts text files from Q,U cubes for input to TARS
REAMP      modifies UV data re-scaling the amplitudes
REFLG      Attempts to compress a flag table
REGRD      Regrids an image from one co-ordinate frame to another
REIFS      Breaks spectral windows into multiple spectral windows (IFs)
REMAG      Task to replace magic blanks with a user specified value
RESEQ      Renumber antennas
REWAY      computes weights based in rms in spectra
REWGT      modifies UV data re-scaling the weights only
RFARS      Correct Q/U cubes for Faraday rotation synthesis results
RFI        Look for RFI in uv data
RFLAG      Flags data set based on time and freq rms in fringe visibilities
RGBMP      Task to create an RGB image from the 3rd dim of an image
RLCAL      Determines instrumental right-left phase versus time (a self-cal)
RLCOR      corrects a data set for R-L phase differences
RLDIF      determines Right minus Left phase difference, corrects cal files
RLDLY      fringe fit data to determine antenna R-L delay difference
RM2PL      Plots spectrum of a pixel with RMFIT fit
RMFIT      Fits 1-dimensional polarization spectrum to Q/U cube
RM         Task to calculate rotation measure and magnetic field
RMSD       Calculate rms for each pixel using data at the box around the pixel
RSPEC      Plots and prints spectrum of rms of a cube
RSTOR      Restores a CC file to a map with a gaussian beam.
RTIME      Task to test compute times
SABOX      create box file from source islands in facet images
SAD        Finds and fits Gaussians to portions of an image
SBCOR      Task to correct VLBA data for phase shift between USB & LSB
SCIMG      Full-featured imaging plus self-calibration loop with editing
SCLIM      operates on an image with a choice of mathematical functions
SCMAP      Imaging plus self-calibration loop with editing
SDCAL      Task to apply single dish calibration
SDCLN      deconvolves image by Clark and then "SDI" cleaning methods
SDGRD      Task to select and image random-position single-dish data
SDIMG      Task to select and image random-position single-dish data
SDLSF      least squares fit to channels and subtracts from SD uv data
SDMOD      modifies single-dish UV data with model sources
SDTUV      Task to convert SD table files to UV like data.
SDVEL      shifts spectral-line single-dish data to a given velocity
SERCH      Finds line signals in transposed data cube
SETAN      Reads an ANtenna file info from a text file
SETFC      makes a BOXFILE for input to IMAGR
SETJY      Task to enter source info into source (SU) table.
SHADO      Calculate the shadowing of antennas at the array
SHADW      Generates the "shadowed" representation of an image
SHOUV      displays uv data in various ways.
SKYVE      Regrids a DSS image from one co-ordinate frame to another
SL2PL      Task to convert a Slice File to a Plot File
SLCOL      Task to collate slice data and models.
SLFIT      Task to fit gaussians to slice data.
SLICE      Task to make a slice file from an image
SLPRT      Task to print a Slice File
SMOTH      Task to smooth a subimage from upto a 7-dim. image
SNCOP      Task to copy SN table averaging some input IFs
SNCOR      applies user-selected corrections to the calibration SN table
SNDUP      copies and duplicates SN table from single pol file to dual pol
SNEDT      Interactive SN/CL/TY/SY table editor using the TV
SNFIT      Fits parabola to SN amplitudes and plots result
SNFLG      Writes flagging info based on the contents of SN files
SNIFS      Plots selected contents of SN, TY, SY, PC or CL files
SNP2D      Task to convert SN table single-channel phase to delay
SNPLT      Plots selected contents of SN, SY, TY, PC or CL files
SNREF      Chooses best reference antenna to minimize R-L differences
SNSMO      smooths and filters a calibration SN table
SOLCL      adjust gains for solar data according to nominal sensitivity
SOUSP      fits source spectral index from SU table or adverbs
SPCAL      Determines instrumental polzn. for spec. line UV data
SPCOR      Task to correct an image for a primary beam and spectral index
SPECR      Spectral regridding task for UV data
SPFIX      Makes cube from input to and output from SPIXR spectral index
SPFLG      interactive flagging of UV data in channel-TB using the TV
SPIXR      Fits spectral indexes to each row of an image incl curvature
SPLAT      Applies calibration and splits or assemble selected sources.
SPLIT      converts multi-source to single-source UV files w calibration
SPMOD      Modify UV database by adding a model with spectral lines
SQASH      Task to collapse several planes in a cube into one plane or row
STACK      Task to co-add a set of 2-dimensional images with weighting
STARS      Task to generate an ST ext. file with star positions
STEER      Task which deconvolves the David Steer way.
STESS      Task which finds sensitivity in mosaicing
STFND      Task to find stars in an image and generate an ST table.
STFUN      Task to calculate a structure function image
STRAN      Task compares ST tables, find image coordinates (e.g. guide star )
SUBIM      Task to select a subimage from up to a 7-dim. image
SUFIX      modifies source numbers on uv data
SUMIM      Task to sum overlapping, sequentially-numbered images
SUMSQ      Task to sum the squared pixel values of overlapping,
SWAPR      modifies UV data by swapping real and imaginary parts
SWPOL      Swap polarizations in a UV data base
SY2TY      Task to generate a TY extension file from an EVLA SY table
SYSOL      undoes and re-does nominal sensitivity application for Solar data
TABED      Task to edit tables
TACOP      task to copy tables, other extension files
TAFFY      User definable task to operate on an image
TAFLG      Flags data in a Table extension file
TAMRG      Task to merge table rows under specified conditions
TAPLT      Plots data from a Table extension file
TAPPE      Task to append 2 tables and merge to output table
TARPL      Plot output of TARS task
TARS       Simulation of Faraday rotation synthesis (mainly task FARS)
TASAV      Task to copy all extension tables to a dummy uv or map file
TASRT      Task to sort extension tables.
TBAVG      Time averages data combining all baselines.
TBDIF      Compare entries in two tables
TBIN       Reads a text file AIPS table into AIPS
TBOUT      Writes an AIPS table into a text file for user editting.
TBSUB      Make a new table from a subset of an old table
TBTSK      Paraform OOP task for tables
TCOPY      Tape to tape copy with some disk FITS support
TECOR      Calculate ionospheric delay and Faraday rotation corrections
TFILE      sorts and edits MkIII correlator UNIX-based A-file.
TI2HA      modifies times in UV data to hour angles
TIORD      checks data for time baseline ordering, displays failures
TKPL       Task to send a plot file to the TEK
TLCAL      Converts JVLA telcal files to an SN file
TRANS      Task to transpose a subimage of an up to 7-dim. image
TRUEP      determines true antenna polarization from special data sets
TVCPS      Task to copy a TV screen-image to a PostScript file.
TVDIC      Task to copy a TV screen-image to a Dicomed film recorder.
TVFLG      interactive flagging of UV data using the TV
TVHLD      Task to load an image to the TV with histogram equalization
TVHUI      make TV image from images of intensity, hue, saturation
TVPL       Display a plot file on the TV
TVRGB      make TV image from images of true color (RGB) images
TVSAD      Finds and fits Gaussians to portions of an image with interaction
TVSPC      Display images and spectra from a cube
TXPL       Displays a plot (PL) file on a terminal or line printer
TYAPL      undoes and re-does nominal sensitivity application
TYCOP      copy TY or SY table calibration between IFs
TYSMO      smooths and filters a calibration TY or SY table
UBAVG      Baseline dependent time averaging of uv data
UFLAG      Plots and edits data using a uv-plane grid and the TV
UJOIN      modifies UV data converting IFs to spectral channels
UNCAL      sets up tables for uncalibrating Australia Telescope data
USUBA      Assign subarrays within a uv-data file
UTESS      deconvolves images by maximizing emptiness
UV2MS      Append single-source file to multi-source file.
UV2TB      Converts UV autocorrelation spectra to tables
UVADC      Fourier transforms and corrects a model and adds to uv data.
UVAVG      Average or merge a sorted (BT, TB) uv database
UVBAS      averages several channels and subtracts from uv data.
UVCMP      Convert a UV database to or from compressed format
UVCON      Generate sample UV coverage given a user defined array layout
UVCOP      Task to copy a subset of a UV data file
UVCRS      Finds the crossing points of UV-ellipses.
UVDEC      Decrements the number of spectral channels, keeping every nth
UVDGP      Copy a UV data file, deleting a portion of it
UVDI1      Subtract UV data(averaged up to one time) from the other UV data
UVDIF      prints differences between two UV data sets
UVFIL      Create, fill a uv database from user supplied information
UVFIT      Fits source models to uv data.
UVFIX      Recomputes u,v,w for a uv database
UVFLG      Flags UV-data
UVFND      prints selected data from UV data set to search for problems
UVFRE      Makes one data set have the spectral structure of another
UVGIT      Fits source models to uv data.
UVGLU      Glues UV data frequency blocks back together
UVHGM      Plots statistics of uv data files as histogram.
UVHIM      Makes image of the histogram on two user-chosen axes
UVHOL      prints holography data from a UV data base with calibration
UVIMG      Grid UV data into an "image"
UVLIN      Fits and removes continuum visibility spectrum, also can flag
UVLOD      Read export or FITS data from a tape or disk
UVLSD      least squares fit to channels and divides the uv data.
UVLSF      least squares fit to channels and subtracts from uv data.
UVMAP      makes images from calibrated UV data.
UVMLN      edits data based on the rms of the spectrum
UVMOD      Modify UV database by adding a model incl spectral index
UVMTH       Averages one data set and applied it to another.
UVNOU      flags uv samples near the U,V  axes to reduce interference
UVPLT      plots data from a UV data base
UVPOL      modifies UV data to make complex image and beam
UVPRM      measures parameters from a UV data base
UVPRT      prints data from a UV data base with calibration
UVRFI      Mitigate RFI by Fourier transform or fitting the circle
UVSEN      Determine RMS sidelobe level and brightness sensitivity
UVSIM      Generate sample UV coverage given a user defined array layout
UVSRT      Sort a UV dataset into a specified order
UVSUB      Subtracts/divides a model from/into a uv data base
UVWAX      flags uv samples near the U,V  axes to reduce interference
VBCAL      Scale visibility amplitudes by antenna based constants
VBGLU      Glues together data from multiple passes thru the VLBA corr.
VBMRG      Merge VLBI data, eliminate duplicate correlations
VLABP      VLA antenna beam polarization correction for snapshot images
VLAMP      Makes ANTAB file for phased VLA used in VLBI observations
VLANT      applies VLA/EVLA antenna position corrections from OPs files
VLBABPSS   computes spectral bandpass correction table
VLBIN      Task to read VLBI data from an NRAO/MPI MkII correlator
VLOG       Pre-process external VLBA calibration files
VPFLG      Resets flagging to all or all corss-hand whenever some are flagged
VPLOT      plots uv data and model from CC file
VTESS      Deconvolves sets of images by the Maximum Entropy Method
VTEST      Measures velocity discrepancy across fields
WARP       Model warps in Galaxies
WETHR      Plots selected contents of WX tables, flags data based on WX
WFCLN      Wide field and/or widefrequency  CLEANing/imaging task.
WIPER      plots and edits data from a UV data base using the TV
WTMOD      modifies weights in a UV data set
WTSUM      Task to do a a sum of images weighted by other images
XBASL      Fits and subtracts nth-order baselines from cube (x axis)
XG2PL      Plots spectrum of a pixel with XGAUS/AGAUS and ZEMAN/ZAMAN fits
XGAUS      Fits 1-dimensional Gaussians to images: restartable
XMOM       Fits one-dimensional moments to each row of an image
XPLOT      Plots image rows one at a time on the graphics or TV screen
XSMTH      Smooth data along the x axis
XSUM       Sum or average images on the x axis
XTRAN      Create an image with transformed coordinates
XYDIF      find/apply X minus Y linear polarization phase difference
ZAMAN      Fits 1-dimensional Zeeman model to absorption-line data
ZEMAN      Fits 1-dimensional Zeeman model to data
\end{verbatim}\eve

\sects{\hspace{0.5em}TV}

\vskip 0.5pt\todx{ABOUT TV}\iodx{TV functions}
\bbve\begin{verbatim}
BLANK      blanks out selected, e.g. non-signal, portions of an image
CHARMULT   Changes the multiplication factor for TV characters
CNTR       generate a contour plot file or TV plot from an image
COLORS     specifies the desired TV colors
COSTAR     Verb to plot a symbol at given position on top of a TV image
COTVLOD    Proc to load an image into a TV channel about a coordinate
CURBLINK   switch TV cursor between steady and blinking displays
CURVALUE   displays image intensities selected via the TV cursor
DEFCOLOR   Sets adverb PLCOLORS to match s default XAS TV
DELBOX     Verb to delet boxes with TV cursor & graphics display.
DELTAX     Increment or size in X direction
DELTAY     Increment or size in Y direction
DFILEBOX   Verb to delete Clean boxes with TV cursor & write to file
DONEWTAB   do we make new tables, use a new table format, etc.
DOTV       selects use of TV display option in operation
DRAWBOX    Verb to draw Clean boxes on the display
FACTOR     scales some display or CLEANing process
FILEBOX    Verb to reset Clean boxes with TV cursor & write to file
FILIT      Interactive BOXFILE editing with facet images
GRBLINK    Verb which blinks 2 TV graphics planes
GRCHAN     specifies the TV graphics channel to be used
GRCLEAR    clears the contents of the specified TV graphics channels
GREAD      reads the colors of the specified TV graphics channel
GROFF      turns off specified TV graphics channelS
GRON       turns on specified TV graphics channelS
GWRITE     reads the colors of the specified TV graphics channel
HLPTVHLD   Interactive image display with histogram equalization - run-time help
HLPTVSAD   Find & fit Gaussians to an image with interaction - run-time help
HLPTVSPC   Interactive display of spectra from a cube - run-time help
HUEWEDGE   Show a wedge on the TV suitable for TVHUEINT displays
HUINT      make RGB image from images of intensity & hue, like TVHUEINT
IM2TV      Verb to convert pixel coordinates to TV pixels
IMERASE    replaces an image portion of the TV screen with zeros
IMLHS      converts images to luminosity/hue TV display
IMPOS      displays celestial coordinates selected by the TV cursor
IMWEDGE    load step wedge of full range of image values to TV
IMXY       returns pixel coordinates selected by the TV cursor
MFITSET    gets adverbs for running IMFIT and JMFIT
NBOXES     Number of boxes
NCCBOX     Number of clean component boxes
OFFHUINT   Proc which restores TV functions to normal after TVHUE
OFFPSEUD   Verb which deactivates all pseudo-color displays
OFFSCROL   Verb which deactivates scroll of an image
OFFTRAN    Verb which restores transfer function to normal
OFFZOOM    Verb which returns the hardware IIS zoom to normal
OFMFILE    specifies the name of a text file containing OFM values
PCNTR      Generate plot file with contours plus polarization vectors
PIX2XY     Specifies a pixel in an image
PIXAVG     Average image value
PIXRANGE   Range of pixel values to display
PIXSTD     RMS pixel deviation
PIXVAL     Value of a pixel
PROFL      Generates plot file for a profile display.
REBOX      Verb to reset boxes with TV cursor & graphics display.
REMOVIE    Verb to rerun a previously loaded (TVMOVIE) movie
REROAM     Verb to use previous roam image mode, then does roam
RGBCOLOR   specifies the desired TV graphics color
RGBGAMMA   specifies the desired color gamma corrections
RGBMP      Task to create an RGB image from the 3rd dim of an image
ROAM       Roam around an image too large for the display.
ROAMOFF    Verb to recover image from roam display in simple display mode
ROMODE     Specified roam mode
SETMAXAP   Examines/alters system parameter limiting dynamic pseudo-AP
SETROAM    Verb use to set roam image mode, then do roam.  OBSOLETE
SETSLICE   Set slice endpoints on the TV interactively
SETXWIN    Procedure to set BLC and TRC with TV cursor
TBLC       Gives the bottom left corner of an image to be displayed
TTRC       Specifies the top right corner of a subimage to be displayed
TV1SET     Verb to reset 1D gaussian fitting initial guess on TV plot.
TV3COLOR   Verb to initiate 3-color display using 3 TV channels
TVACOMPS   Verb to add slice model components directly on TV graphics
TVAGUESS   Verb to re-plot slice model guess directly on TV graphics
TVALL      Procedure loads image to TV, shows labeled wedge, enhances
TVAMODEL   Verb to add slice model display directly on TV graphics
TVANOT     Verb to load anotation to the TV image or graphics
TVARESID   Verb to add slice model residuals directly on TV graphics
TVASLICE   Verb to add a slice display on TV graphics from slice file
TVBLINK    Verb which blinks 2 TV planes, can do enhancement also
TVBOX      Verb to set boxes with TV cursor & graphics display.
TVBUT      Tells which AIPS TV button was pushed
TVCHAN     Specified a TV channel (plane)
TVCLEAR    Verb to clear image from TV channel(s)
TVCOLORS   Sets adverb PLCOLORS to match the TV (DOTV=1) usage
TVCOMPS    Verb to display slice model components directly on TV graphics
TVCORN     Specified the TV pixel for the bottom left corner of an image
TVCPS      Task to copy a TV screen-image to a PostScript file.
TVCUBE     Verb to load a cube into tv channel(s) & run a movie
TVDIC      Task to copy a TV screen-image to a Dicomed film recorder.
TVDIST     determines spherical distance between two pixels on TV screen
TVFIDDLE   Verb enhances B/W or color TV image with zooms
TVFLUX     displays coordinates and values selected with the TV cursor
TVGUESS    Verb to display slice model guess directly on TV graphics
TVHELIX    Verb to activate a helical hue-intensity TV pseudo-coloring
TVHLD      Task to load an image to the TV with histogram equalization
TVHUEINT   Verb to make hue/intensity display from 2 TV channels
TVHUI      make TV image from images of intensity, hue, saturation
TVILINE    Verb to draw a straight line on an image on the TV
TVINIT     Verb to return TV display to a virgin state
TVLABEL    Verb to label the (map) image on the TV
TVLAYOUT   Verb to label the holography image on the TV with panel layout
TVLEVS     Gives the peak intensity to be displayed in levels
TVLINE     Verb to load a straight line to the TV image or graphics
TVLOD      Verb to load an image into a TV channel
TVLUT      Verb which modifies the transfer function of the image
TVMAXFIT   displays fit pixel positions and intensity at maxima on TV
TVMBLINK   Verb which blinks 2 TV planes either auto or manually
TVMLUT     Verb which modifies the transfer function of the image
TVMODEL    Verb to display slice model directly on TV graphics
TVMOVIE    Verb to load a cube into tv channel(s) & run a movie
TVNAME     Verb to fill image name of that under cursor
TVOFF      Verb which turns off TV channel(s).
TVON       Turns on one or all TV image planes
TVPHLAME   Verb to activate "flame-like" pseudo-color displays
TVPL       Display a plot file on the TV
TVPOS      Read a TV screen position using cursor
TVPSEUDO   Verb to activate three types of pseudo-color displays
TVRESET    Reset the TV without erasing the image planes
TVRESID    Verb to display slice model residuals directly on TV graphics
TVRGB      make TV image from images of true color (RGB) images
TVROAM     Load up to 16 TV image planes and roam a subset thereof
TVSAD      Finds and fits Gaussians to portions of an image with interaction
TVSCROL    Shift position of image on the TV screen
TVSET      Verb to set slice Gaussian fitting initial guesses from TV plot
TVSLICE    Verb to display slice file directly on TV
TVSPC      Display images and spectra from a cube
TVSPLIT    Compare two TV image planes, showing halves
TVSTAR     Verb to plot star positions on top of a TV image
TVSTAT     Find the mean and RMS in a blotch region on the TV
TVTRANSF   Interactively alters the TV image plane transfer function
TVWEDGE    Show a linear wedge on the TV
TVWINDOW   Set a window on the TV with the cursor
TVWLABEL   Put a label on the wedge that you just put on the TV
TVXY       Pixel position on the TV screen
TVZOOM     Activate the TV zoom
TXINC      TV X coordinate increment
TYINC      TV Y coordinate increment
TZINC      TV Z coordinate increment
WEDERASE   Load a wedge portion of the TV with zeros
XAS        Information about TV-Servers
XVSS       Information about older Sun OpenWindows-specific TV-Server
\end{verbatim}\eve

%\vfill\eject
\sects{\hspace{0.5em}TV-APPL}

\vskip 0.5pt\todx{ABOUT TV APPLICATIONS}\iodx{TV functions}
\bbve\begin{verbatim}
AGAUS      Fits 1-dimensional Gaussians to absorption-line spectra
BLSUM      sums images over irregular sub-images, displays spectra
BPEDT      Interactive TV task to edit uv data based on BP tables
EDITA      Interactive TV task to edit uv data based on TY/SY/SN/CL tables
EDITR      Interactive baseline-oriented visibility editor using the TV
FTFLG      interactive flagging of UV data in channel-time using the TV
GAMMASET   changes the gamma-correction exponent used in the TV OFM
HLPAGAUS   Interactive Gaussian absorption fitting task AGAUS - run-time help
HLPCLEAN   Cleaning tasks  - run-time help
HLPEDIBP   Interactive BP table uv-data editor BPEDT - run-time help
HLPEDICL   Interactive SN/CL table uv-data editor - run-time help
HLPEDIPC   Interactive PC table editor PCEDT - run-time help
HLPEDIPD   Interactive PD table editor PDEDT - run-time help
HLPEDISN   Interactive SN/CL table (not UV) editor - run-time help
HLPEDISS   Interactive SY table (not UV) editor - run-time help
HLPEDISY   Interactive SY table uv-data editor EDITA - run-time help
HLPEDITS   Interactive TY table (not UV) editor - run-time help
HLPEDITY   Interactive TY table uv-data editor EDITA - run-time help
HLPEDIUV   Interactive uv-data editor EDITR - run-time help
HLPFILIT   Interactive Clean box file editing with image display - run-time help
HLPFTFLG   Interactive time-channel visibility Editor - run-time help
HLPIBLED   Interactive Baseline based visibility Editor - run-time help
HLPPCFLG   Interactive time-channel PC table Editor PCFLG - run-time help
HLPPLAYR   OOP TV class demonstration task - run-time help
HLPRMFIT   Polarization fitting task RMFIT - run-time help
HLPSCIMG   Full-featured image plus self-cal loops, editing - run-time help
HLPSCMAP   Imaging plus self-cal and editing SCMAP - run-time help
HLPSPFLG   Interactive time-channel visibility Editor SPFLG - run-time help
HLPTVFLG   Interactive time-baseline visibility Editor TVFLG - run-time help
HLPTVHUI   Interactive intensity-hue-saturation display - run-time help
HLPTVRGB   Interactive red-green-blue display - run-time help
HLPUFLAG   Edit uv-data on a grid UFLAG - run-time help
HLPWIPER   edit uv data from UVPLT-like plot WIPER - run-time help
HLPXGAUS   Interactive Gaussian fitting task XGAUS - run-time help
HLPZAMAN   Fits 1-dimensional Zeeman model to absorption data - run-time help
HLPZEMAN   Fits 1-dimensional Zeeman model to data - run-time help
IBLED      Interactive BaseLine based visibility EDitor
IMAGR      Wide-field and/or wide-frequency Cleaning / imaging task.
OFMADJUS   interactive linear adjustment of current TV OFM lookup tables
OFMCONT    creates/modifies TV color OFMs with level or wedged contours
OFMDIR     lists names of the user's and system's OFM files from OFMFIL
OFMGET     loads TV OFMS from an OFM save file
OFMLIST    lists the current TV OFM table(s) on the terminal or printer
OFMSAVE    saves the TV's current OFM lookup table in a text file
OFMTWEAK   interactive modification of current TV OFM lookup tables
OFMZAP     deletes an OFM lookup table save file
PCEDT      Interactive TV task to edit pulse-cal (PC) tables
PCFLG      interactive flagging of Pulse-cal data in channel-TB using the TV
PDEDT      Interactive TV task to edit polarization D-term (PD) tables
PLAYR      Verb to load an image into a TV channel
RM2PL      Plots spectrum of a pixel with RMFIT fit
RMFIT      Fits 1-dimensional polarization spectrum to Q/U cube
SNEDT      Interactive SN/CL/TY/SY table editor using the TV
SPFLG      interactive flagging of UV data in channel-TB using the TV
TDEPEND    Time-dependent imaging procedure sequence
TVFLG      interactive flagging of UV data using the TV
UFLAG      Plots and edits data using a uv-plane grid and the TV
WIPER      plots and edits data from a UV data base using the TV
XBASL      Fits and subtracts nth-order baselines from cube (x axis)
XG2PL      Plots spectrum of a pixel with XGAUS/AGAUS and ZEMAN/ZAMAN fits
XGAUS      Fits 1-dimensional Gaussians to images: restartable
XPLOT      Plots image rows one at a time on the graphics or TV screen
ZAMAN      Fits 1-dimensional Zeeman model to absorption-line data
ZEMAN      Fits 1-dimensional Zeeman model to data
\end{verbatim}\eve

\vfill\eject
\sects{\hspace{0.5em}UTILITY}

\vskip 0.5pt\todx{ABOUT UTILITY}\iodx{utilities}
\bbve\begin{verbatim}
ANTNUM     Returns number of a named antenna
CCEDT      Select CC components in BOXes and above mininum flux.
CCSEL      Select signifigant CC components
CL2HF      Convert CL table to HF table
DOOSRO     calibrating amplitude and phase, and imaging VLA data
DOVLAMP    Produces amp calibration file for phased-VLA VLBI data
EPOCONV    Convert between J2000 and B1950 coordinates
MAXTAB     Returns maximum version number of named table
MBDLY      Fits multiband delays from IF phases, updates SN table
MK3TX      extract text files from a MKIII VLBI archive tape
MOVE       Task to copy or move data from one user or disk to another
NANS       reads an image or a UV data set and looks for NaNs
OPCODE     General adverb, defines an operation
OPTELL     The operation to be passed to a task by TELL
PIPEAIPS   calibrating amplitude and phase, and imaging VLA data
PRNUMBER   POPS number of messages
PRTIME     Time limit
RUNWAIT    Runs a task and waits for it to finish
SCANTIME   Returns time range for a given scan number
SHOW       Verblike to display the TELL adverbs of a task.
SORT       Specified desired sort order
SQASH      Task to collapse several planes in a cube into one plane or row
STRAN      Task compares ST tables, find image coordinates (e.g. guide star )
SWAPR      modifies UV data by swapping real and imaginary parts
TBDIF      Compare entries in two tables
TBIN       Reads a text file AIPS table into AIPS
TBOUT      Writes an AIPS table into a text file for user editting.
TBSUB      Make a new table from a subset of an old table
TBTSK      Paraform OOP task for tables
TCOPY      Tape to tape copy with some disk FITS support
UVAVG      Average or merge a sorted (BT, TB) uv database
UVCMP      Convert a UV database to or from compressed format
UVDI1      Subtract UV data(averaged up to one time) from the other UV data
UVNOU      flags uv samples near the U,V  axes to reduce interference
UVRFI      Mitigate RFI by Fourier transform or fitting the circle
UVWAX      flags uv samples near the U,V  axes to reduce interference
VLAPROCS   Procedures to simplify the reduction of VLBA data
VLARUN     calibrating amplitude and phase, and imaging VLA data
VLASUMM    Plots selected contents of SN or CL files
VLBAAMP    applies a-priori amplitude corrections to VLBA data
VLBAARCH   Procedure to archive VLBA correlator data
VLBACALA   applies a-priori amplitude corrections to VLBA data
VLBACCOR   applies a-priori amplitude corrections to VLBA data
VLBACPOL   Procedure to calibrate cross-polarization delays
VLBACRPL   Plots crosscorrelations
VLBAEOPS   Corrects Earth orientation parameters
VLBAFIX    Procedure that fixes VLBA data, if necessary
VLBAFPOL   Checks and corrects polarization labels for VLBA data
VLBAFQS    Copies different FQIDS to separate files
VLBAFRGP   Fringe fit phase referenced data and apply calibration
VLBAFRNG   Fringe fit data and apply calibration
VLBAKRGP   Fringe fit phase referenced data and apply calibration
VLBAKRNG   Fringe fit data and apply calibration
VLBALOAD   Loads VLBA data
VLBAMCAL   Merges redundant calibration data
VLBAMPCL   Calculates and applies manual instrumental phase calibration
VLBAPANG   Corrects for parallactic angle
VLBAPCOR   Calculates and applies instrumental phase calibration
VLBAPIPE   applies amplitude and phase calibration procs to VLBA data
VLBARUN    applies amplitude and phase calibration procs to VLBA data
VLBASNPL   Plots selected contents of SN or CL files
VLBASRT    Sorts VLBA data, if necessary
VLBASUBS   looks for subarrays in VLBA data
VLBASUMM   Prints a summary of a VLBI experiment
VLBATECR   Calculate ionospheric delay and Faraday rotation corrections
VLBAUTIL   Procedures to simplify the reduction of VLBA data
\end{verbatim}\eve

\sects{\hspace{0.5em}UV}

\vskip 0.5pt\todx{ABOUT UV}\iodx{UV}
\bbve\begin{verbatim}
ACCOR      Corrects cross amplitudes using auto correlation measurements
ACLIP      edits suto-corr data for amplitudes, phases, and weights out of range
ACSCL      Corrects cross amplitudes using auto correlation measurements
ADDIF      Adds an IF axis to a uv data set
AFILE      sorts and edits MkIII correlator A-file.
ALIAS      adverb to alias antenna numbers to one another
ALVAR      plots the Allan Variance statistic of a UV data set
ALVPR      prints statistics on the Allan Variance of a UV data set
ANBPL      plots and prints  uv data converted to antenna based values
ASDMFILE   Full path to EVLA ASDM/BDF directory
ATLOD      Reads ATCA data in RPFITS dormat into AIPS
AVER       Averages over time UV data sets in 'BT' order
AVOPTION   Controls type or range of averaging done by a task
AVSPC      Averages uv-data in the frequency domain
BAND       specifies the approximate frequency of UV data to be selected
BASFIT     fits antenna locations from SN-table data
BASRM      Task to remove a spectral baseline from total power spectra
BDF2AIPS   Read EVLA ASDM/BDF data into AIPS
BDFLIST    List contents of EVLA ASD data file
BIF        gives first IF to be included
BLAVG      Average cross-polarized UV data over baselines.
BLOAT      converts line data to greater number channels
BPASSPRM   Control adverb array for bandpass calibration
BPEDT      Interactive TV task to edit uv data based on BP tables
BPLOT      Plots bandpass tables in 2 dimensions as function of time
BPSMO      Smooths or interpolates bandpass tables to regular times
BPWAY      Determines channel-dependent relative weights
BPWGT      Calibrates data and scales weights by bandpass correction
BREAK      procedure to TELL FILLM to break all current uv files, start new
BSCAN      seeks best scan to use for phase cal, fringe search, ..
BSMOD      creates single-dish UV beam-switched data with model sources
BSROT      modifies SD beam-switch continuum data for error in throw
CALIB      determines antenna calibration: complex gain
CAPLT      plots closure amplitude and model from CC file
CENTR      modifies UV data to center the reference channel
CLIP       edits data based on amplitudes, phases, and weights out of range
CLPLT      plots closure phase and model from CC file
CMETHOD    specifies the method by which the uv model is computed
CMODEL     specifies the method by which the uv model is computed
COHER      Baseline Phase coherence measurement
CONFIG     Configuration ID number within an EVLA ASDM/BDF data set
CORER      calculates correlator statistics and flags bad ones
CORFQ      corrects uvw for incorrect observing frequency
CVEL       shifts spectral-line UV data to a given velocity
DAYFX      Fixes day number problems left by FILLM
DBAPP      appends one or more data sets to the output data set
DBCON      concatenates two UV data sets
DECOR      Measures the decorrelation between channels and IF of uv data
DEFER      Controls when file creation takes place
DEFLG      edits data based on decorrelation over channels and time
DESCM      copies a portion of a UV data set
DFQID      modifies UV data changing the indicated FQIDs
DFTIM      Makes image of DFT at arbitrary point showing time vs frequency
DFTPL      plots DFT of a UV data set at arbitrary point versus time
DIFRL      divides the RR data by LL data
DIFUV      Outputs the difference of two matching input uv data sets
DOACOR     specifies whether autocorrelation data are included
DOARRAY    spcifies if subarrays are ignored or the information used
DOBTWEEN   Controls smoothing between sources in calibration tables
DOCONCAT   selects concatenated or indivudual output files
DOEBAR     Controls display of estimates of the uncertainty in the data
DOIFS      controls functions done across IFs
DOROBUST   Controls method of averaging - simple mean/rms or robust
DOSTOKES   selects options related to polarizations
DOUVCOMP   selects use of compression in writing UV data to disk
DQUAL      Rearranges source list, dropping qualifiers
DSORC      copies a data set elliminating some source numbers
DSTOK      Drops the cross-hand polarizations
DTCHK      Task to check results of a test using simulated data.
DTSUM      Task to provide a summary of the contents of a dataset
EDITA      Interactive TV task to edit uv data based on TY/SY/SN/CL tables
EDITR      Interactive baseline-oriented visibility editor using the TV
EIF        last IF number to be included in operation
ELFIT      Plots/fits selected contents of SN, TY, SY, PC or CL files
EVASN      Evaluates statistics in SN/CL tables
EVAUV      Subtracts & divides a model into UV data, does statistics on results
EVLA       puts the list of eVLA antennas in the current file on stack
FEW        procedure to TELL FILLM to append incoming data to existing uv files
FGCNT      Counts samples comparing two flag tables
FGDIF      Compares affect of 2 FG tables
FGPLT      Plots selected contents of FG table
FGSPW      Flags bad spectral windows
FGTAB      writes frequency-range flags to a text file to be read by UVFLG
FILLM      reads VLA on-line/archive format uv data tapes (post Jan 88)
FILLR      reads old VLA on-line-system tapes into AIPS
FINDR      Find normal values for a uv data set
FITAB      writes images / uv data w extensions to tape in FITS format
FITDISK    writes images / uv data w extensions to disk in FITS format
FITTP      writes images / uv data w extensions to tape in FITS format
FIXAL      least squares fit aliasing function and remove
FIXAN      fixes the contents of the ANtenna extension file
FIXRL      correctes right vs left polarizations for a list of antennas
FIXWT      Modify weights to reflect amplitude scatter of data
FLAGR      Edit data based on internal RMS, amplitudes, weights
FLGIT      flags data based on the rms of the spectrum
FLOPM      reverses the spectral order of UV data, can fix VLA error
FOV        Specifies the field of view
FQCENTER   specifies that the frequency axis should be centered
FRMAP      Task to build a map using fringe rate spectra
FRPLT      Task to plot fringe rate spectra
FTFLG      interactive flagging of UV data in channel-time using the TV
FUDGE      modifies UV data with user's algorithm: paraform task
FXALIAS    least squares fit aliasing function and remove
FXPOL      Corrects VLBA polarization assignments
FXTIM      fixes start date so all times are positive
FXVLA      Task to correct VLA data for on-line errors in special cases.
FXVLB      Builds a CQ table to enable VLBA correlator loss corrections
GCPLT      Plots gain curves from text files
HA2TI      Converts data processed by TI2HA (STUFFER) back to real times
HAFIX      Recomputes u,v,w when time is hour angle (UVdata is output of TI2HA)
HLPEDIBP   Interactive BP table uv-data editor BPEDT - run-time help
HLPEDICL   Interactive SN/CL table uv-data editor - run-time help
HLPEDIPC   Interactive PC table editor PCEDT - run-time help
HLPEDIPD   Interactive PD table editor PDEDT - run-time help
HLPEDISN   Interactive SN/CL table (not UV) editor - run-time help
HLPEDISS   Interactive SY table (not UV) editor - run-time help
HLPEDISY   Interactive SY table uv-data editor EDITA - run-time help
HLPEDITS   Interactive TY table (not UV) editor - run-time help
HLPEDITY   Interactive TY table uv-data editor EDITA - run-time help
HLPEDIUV   Interactive uv-data editor EDITR - run-time help
HLPFTFLG   Interactive time-channel visibility Editor - run-time help
HLPIBLED   Interactive Baseline based visibility Editor - run-time help
HLPPCFLG   Interactive time-channel PC table Editor PCFLG - run-time help
HLPSPFLG   Interactive time-channel visibility Editor SPFLG - run-time help
HLPTVFLG   Interactive time-baseline visibility Editor TVFLG - run-time help
HLPUFLAG   Edit uv-data on a grid UFLAG - run-time help
HLPWIPER   edit uv data from UVPLT-like plot WIPER - run-time help
HOLGR      Read & process holography visibility data to telescope images
HOLOG      Read & process holography visibility data to telescope images
HSA        puts the list of HSA antennas in the current file on stack
IBLED      Interactive BaseLine based visibility EDitor
IM2UV      converts an image to a visibility data set
IMFRING    large image delay fitting with IM2CC and OOFRING
IMSCAL     large image self-cal with IM2CC and OOCAL
LISTR      prints contents of UV data sets and assoc. calibration tables
LOCIT      fits antenna locations from SN-table data
LPCAL      Determines instrumental polarization for UV data
M3TAR      translate Haystack MKIII VLBI format "A" TAR's into AIPS
MANY       procedure to TELL FILLM to start new uv files on each scan
MAPBM      Map VLA beam polarization
MATCH      changes antenna, source, FQ numbers to match a data set
MK3IN      translate Haystack MKIII VLBI format "A" tapes into AIPS
MORIF      Combines IFs or breaks spectral windows into multiple windows (IFs)
MSORT      Sort a UV dataset into a specified order
MULIF      Change number of IFs in output
MULTI      Task to convert single-source to multi-source UV data
NANS       reads an image or a UV data set and looks for NaNs
NCHAN      Number of spectral channels in each spectral window
NIF        Number of IFs (spectral windows) in a data set
NOIFS      makes all IFs into single spectrum
NPIECE     The number of pieces to make
OBJECT     The name of an object
OBPLT      Plot columns of an OB table.
OMFIT      Fits sources and, optionally, a self-cal model to uv data
OOCAL      determines antenna complex gain with frequency-dependent models
OOSRT      Sort a UV dataset into a specified order
OOSUB      Subtracts/divides a model from/into a uv data base
PCAL       Determines instrumental polarization for UV data
PCASS      Finds amplitude bandpass shape from pulse-cal table data
PCCOR      Corrects phases using  PCAL tones data from PC table
PCEDT      Interactive TV task to edit pulse-cal (PC) tables
PCFIT      Finds delays and phases using a pulse-cal (PC) table
PCFLG      interactive flagging of Pulse-cal data in channel-TB using the TV
PCPLT      Plots pulse-cal tables in 2 dimensions as function of time
PCVEL      shifts spectral-line UV data to a given velocity: planet version
PDEDT      Interactive TV task to edit polarization D-term (PD) tables
PEELR      calibrates interfering sources in multi-facet imges
PHASE      Baseline Phase coherence measurement
PHNEG      Negates a UV datafile's visibility phase.
PHSLIMIT   gives a phase value in degrees
PHSRF      Perform phase-referencing within a spectral line database.
PLOTC      plots color schems used by 3-color plot tasks
POSSM      Task to plot total and cross-power spectra.
PRTAN      prints the contents of the ANtenna extension file
PRTUV      prints contents of a visibility (UV) data set
QCREATE    adverb controlling the way large files are created
QUACK      Flags beginning or end portions of UV-data scans
QUAL       Source qualifier
QUFIX      determines Right minus Left phase difference, corrects cal files
QUIT       procedure to TELL FILLM to stop at the end of the current scan
QUOUT      writes text file of Q, U versus frequency to be used by RLDIF
READISK    writes images / uv data w extensions to tape in FITS format
REAMP      modifies UV data re-scaling the amplitudes
REFDATE    To specify the initial or reference date of a data set
REFLG      Attempts to compress a flag table
REIFS      Breaks spectral windows into multiple spectral windows (IFs)
RESEQ      Renumber antennas
REWAY      computes weights based in rms in spectra
REWEIGHT   Reweighting factors for UV data weights.
REWGT      modifies UV data re-scaling the weights only
RFLAG      Flags data set based on time and freq rms in fringe visibilities
RLCAL      Determines instrumental right-left phase versus time (a self-cal)
RLCOR      corrects a data set for R-L phase differences
RLDIF      determines Right minus Left phase difference, corrects cal files
ROBUST     Uniform weighting "robustness" parameter
SBCOR      Task to correct VLBA data for phase shift between USB & LSB
SDLSF      least squares fit to channels and subtracts from SD uv data
SDMOD      modifies single-dish UV data with model sources
SDVEL      shifts spectral-line single-dish data to a given velocity
SETAN      Reads an ANtenna file info from a text file
SHADO      Calculate the shadowing of antennas at the array
SHOUV      displays uv data in various ways.
SMOOTH     Specifies spectral smoothing
SNEDT      Interactive SN/CL/TY/SY table editor using the TV
SNFIT      Fits parabola to SN amplitudes and plots result
SNFLG      Writes flagging info based on the contents of SN files
SNIFS      Plots selected contents of SN, TY, SY, PC or CL files
SNPLT      Plots selected contents of SN, SY, TY, PC or CL files
SNREF      Chooses best reference antenna to minimize R-L differences
SORT       Specified desired sort order
SOUSP      fits source spectral index from SU table or adverbs
SPCAL      Determines instrumental polzn. for spec. line UV data
SPECR      Spectral regridding task for UV data
SPFLG      interactive flagging of UV data in channel-TB using the TV
SPLAT      Applies calibration and splits or assemble selected sources.
SPLIT      converts multi-source to single-source UV files w calibration
SPMOD      Modify UV database by adding a model with spectral lines
STOKES     Stokes parameter
STOP       procedure to TELL FILLM to break all current uv files and stop
STUFFR     averages together data sets in hour angle
SUFIX      modifies source numbers on uv data
SWAPR      modifies UV data by swapping real and imaginary parts
SWPOL      Swap polarizations in a UV data base
TBAVG      Time averages data combining all baselines.
TELFLM     procedure to TELL real-time FILLM a new APARM(1) value
TFILE      sorts and edits MkIII correlator UNIX-based A-file.
TI2HA      modifies times in UV data to hour angles
TIORD      checks data for time baseline ordering, displays failures
TLCAL      Converts JVLA telcal files to an SN file
TRUEP      determines true antenna polarization from special data sets
TVFLG      interactive flagging of UV data using the TV
U2CAT      list a user's UV and scratch files on disk IN2DISK
U3CAT      list a user's UV and scratch files on disk IN3DISK
U4CAT      list a user's UV and scratch files on disk IN4DISK
U5CAT      list a user's UV and scratch files on disk IN5DISK
UBAVG      Baseline dependent time averaging of uv data
UCAT       list a user's UV and scratch files on disk INDISK
UFLAG      Plots and edits data using a uv-plane grid and the TV
UJOIN      modifies UV data converting IFs to spectral channels
UOCAT      list a user's UV and scratch files on disk OUTDISK
USUBA      Assign subarrays within a uv-data file
UV1TYPE    Convolving function type 1, pillbox or square wave
UV2MS      Append single-source file to multi-source file.
UV2TB      Converts UV autocorrelation spectra to tables
UV2TYPE    Convolving function type 2, exponential function
UV3TYPE    Convolving function type 3, sinc function
UV4TYPE    Convolving function type 4, exponent times sinc function
UV5TYPE    Convolving function type 5, spheroidal function
UV6TYPE    Convolving function type 6, exponent times BessJ1(x) / x
UVADC      Fourier transforms and corrects a model and adds to uv data.
UVAVG      Average or merge a sorted (BT, TB) uv database
UVBAS      averages several channels and subtracts from uv data.
UVBOX      radius of the smoothing box used for uniform weighting
UVBXFN     type of function used when counting for uniform weighting
UVCMP      Convert a UV database to or from compressed format
UVCON      Generate sample UV coverage given a user defined array layout
UVCOP      Task to copy a subset of a UV data file
UVCOPPRM   Parameter adverb array for task UVCOP
UVCRS      Finds the crossing points of UV-ellipses.
UVDEC      Decrements the number of spectral channels, keeping every nth
UVDGP      Copy a UV data file, deleting a portion of it
UVDI1      Subtract UV data(averaged up to one time) from the other UV data
UVDIF      prints differences between two UV data sets
UVFIL      Create, fill a uv database from user supplied information
UVFIT      Fits source models to uv data.
UVFIX      Recomputes u,v,w for a uv database
UVFIXPRM   Parameter adverb array for task UVFIX
UVFLG      Flags UV-data
UVFND      prints selected data from UV data set to search for problems
UVFRE      Makes one data set have the spectral structure of another
UVGIT      Fits source models to uv data.
UVGLU      Glues UV data frequency blocks back together
UVHGM      Plots statistics of uv data files as histogram.
UVHIM      Makes image of the histogram on two user-chosen axes
UVHOL      prints holography data from a UV data base with calibration
UVIMG      Grid UV data into an "image"
UVLIN      Fits and removes continuum visibility spectrum, also can flag
UVLOD      Read export or FITS data from a tape or disk
UVLSD      least squares fit to channels and divides the uv data.
UVLSF      least squares fit to channels and subtracts from uv data.
UVMLN      edits data based on the rms of the spectrum
UVMOD      Modify UV database by adding a model incl spectral index
UVMTH       Averages one data set and applied it to another.
UVNOU      flags uv samples near the U,V  axes to reduce interference
UVPLT      plots data from a UV data base
UVPOL      modifies UV data to make complex image and beam
UVPRM      measures parameters from a UV data base
UVPRT      prints data from a UV data base with calibration
UVRANGE    Specify range of projected baselines
UVRFI      Mitigate RFI by Fourier transform or fitting the circle
UVSEN      Determine RMS sidelobe level and brightness sensitivity
UVSIM      Generate sample UV coverage given a user defined array layout
UVSIZE     specifies number of pixels on X and Y axes of a UV image
UVSRT      Sort a UV dataset into a specified order
UVSUB      Subtracts/divides a model from/into a uv data base
UVTAPER    Widths in U and V of gaussian weighting taper function
UVWAX      flags uv samples near the U,V  axes to reduce interference
UVWTFN     Specify weighting function, Uniform or Natural
VBGLU      Glues together data from multiple passes thru the VLBA corr.
VECTOR     selects method of averaging UV data
VLA        puts the list of VLA antennas in the current file on stack
VLBA       puts the list of VLBA antennas in the current file on stack
VLBIN      Task to read VLBI data from an NRAO/MPI MkII correlator
VPFLG      Resets flagging to all or all corss-hand whenever some are flagged
VPLOT      plots uv data and model from CC file
VTEST      Measures velocity discrepancy across fields
WEIGHTIT   Controls modification of weights before gain/fringe solutions
WETHR      Plots selected contents of WX tables, flags data based on WX
WIPER      plots and edits data from a UV data base using the TV
WRTDISK    writes images / uv data w extensions to tape in FITS format
WRTPROCS   Procedures to simplify the reduction of VLBA data
WTMOD      modifies weights in a UV data set
WTUV       Specifies the weight to use for UV data outside UVRANGE
XYDIF      find/apply X minus Y linear polarization phase difference
ZEROSP     Specify how to include zero spacing fluxes in FT of UV data
\end{verbatim}\eve

\vfill\eject
\sects{\hspace{0.5em}VERB}

\vskip 0.5pt\todx{ABOUT VERB}\iodx{verb}
\bbve\begin{verbatim}
VERB
Type:  General type of POPS symbol
Use:   Verbs are the magic words which cause FORTRAN code to
       execute some function.  They are compiled into AIPS
       by the programmers and their meaning remains fixed
       at least until the programmers change their minds.
Grammar:   Verbs may be given either in compile mode or in
       regular execute mode.  In the former, their pointers
       are stored with the procedure and they are executed
       when the procedure is invoked.  In the latter, they
       are compiled with the other statements and parameters
       on the input line and then executed before a new
       input line is read.
Execution:  Verbs are executed when the line in which they
       appear is executed and are simply referenced by their
       name.  The syntax "GO verb_name" is converted by AIPS to
       "TPUT verb_name ; verb_name" which saves the adverbs of
       "verb_name" for a later TGET and then executes
       "verb_name".  The syntax "TASK = 'verb_name' ; GO" will
       not work.
****************************************************************

ABOUT      displays lists and information on tasks, verbs, adverbs
ABS        returns absolute value of argument
ACOS       Returns arc cosine of argument (half-circle)
ACTNOISE   puts estimate of actual image uncertainty and zero in header
ADDBEAM    Inserts clean beam parameters in image header
ADDDISK    makes a computer's disks available to the current AIPS session
ALLDEST    Delete a group or all of a users data files
ALTDEF     Sets frequency vs velocity relationship into image header
ALTSWTCH   Switches between frequency and velocity in image header
APROPOS    displays all help 1-line summaries containing specified words
ASIN       Returns arc sine of argument (half-circle)
ATAN2      Returns arc tangent of two arguments (full circle)
ATAN       Returns arc tangent of argument (half-circle)
AVEOT      Advances tape to end-of-information point
AVFILE     Moves tape forward or back to end-of-file marks
AVMAP      Advance tape by one image (IBM-CV = obsolete tape file)
AXDEFINE   Define or modify an image axis description
BAMODIFY   edits characters in a line of a batch work file
BATCH      starts entry of commands into batch-job work file
BATCLEAR   removes all text from a batch work file
BATEDIT    starts an edit (replace, insert) session on a batch work file
BATLIST    lists the contents of a batch work file
BDFLIST    List contents of EVLA ASD data file
BY         gives increment to use in FOR loops in POPS language
CALDIR     lists calibrator source models available as AIPS FITS files
CATALOG    list one or more entries in the user's data directory
CEIL       returns smallest integer greater than or equal the argument
CELGAL     switches header between celestial and galactic coordinates
CHAR       converts number to character string
CHARMULT   Changes the multiplication factor for TV characters
CHKNAME    Checks for existence of the specified image name
CLR2NAME   clears adverbs specifying the second input image
CLR3NAME   clears adverbs specifying the third input image
CLR4NAME   clears adverbs specifying the fourth input image
CLR5NAME   clears adverbs specifying the fourth input image
CLRMSG     deletes messages from the user's message file
CLRNAME    clears adverbs specifying the first input image
CLRONAME   clears adverbs specifying the first output image
CLRSTAT    remove any read or write status flags on a directory entry
CLRTEMP    clears the temporary literal area during a procedure
CODECIML   Convert between decimal and sexagesimal coordinate values
COODEFIN   Define or modify an image axis coordinate description
COPIXEL    Convert between physical and pixel coordinate values
COS        returns cosine of the argument in degrees
COSTAR     Verb to plot a symbol at given position on top of a TV image
COWINDOW   Set a window based on coordinates
CPUTIME    displays curren tcpu and real time usage of the AIPS task
CURBLINK   switch TV cursor between steady and blinking displays
CURVALUE   displays image intensities selected via the TV cursor
DAYNUMBR   finds day nuumber of an image or uv data set
DEFAULT    Verb-like sets adverbs for a task or verb to initial values
DELAY      Verb to pause AIPS for DETIME seconds
DELBOX     Verb to delet boxes with TV cursor & graphics display.
DFILEBOX   Verb to delete Clean boxes with TV cursor & write to file
DISMOUNT   disables a magnetic tape and dismounts it from the tape drive
DRAWBOX    Verb to draw Clean boxes on the display
DUMP       displays portions of the POPS symbol table in all formats
EGETHEAD   returns parameter value from image header and error code
EGETNAME   fills in input name adverbs by catalog slot number, w error
EHEX       converts decimal to extended hex
END        marks end of block (FOR, WHILE, IF) of POPS code
EPOCONV    Convert between J2000 and B1950 coordinates
EPOSWTCH   Switches between B1950 and J2000 coordinates in header
EVLA       puts the list of eVLA antennas in the current file on stack
EXIT       ends an AIPS batch or interactive session
EXP        returns the exponential of the argument
EXPLAIN    displays help + extended information describing a task/symbol
EXTDEST    deletes one or more extension files
EXTLIST    lists detailed information about contents of extension files
FILEBOX    Verb to reset Clean boxes with TV cursor & write to file
FILEZAP    Delete an external file
FLOOR      returns largest integer <= argument
FOR        starts an iterative sequence of operations in POPS language
FREESPAC   displays available disk space for AIPS in local system
GAMMASET   changes the gamma-correction exponent used in the TV OFM
GET2NAME   fills 2nd input image name parameters by catalog slot number
GET3NAME   fills 3rd input image name parameters by catalog slot number
GET4NAME   fills 4th input image name parameters by catalog slot number
GET5NAME   fills 5th input image name parameters by catalog slot number
GETDATE    Convert the current date and time to a string
GETHEAD    returns parameter value from image header
GETNAME    fills 1st input image name parameters by catalog slot number
GETONAME   fills 1st output image name parameters by catalog slot number
GETPOPSN   Verb to return the pops number on the stack
GETTHEAD   returns keyword and other values value from a table header
GETVERS    finds maximum version number of an extension file
GO         starts a task, detaching it from AIPS or AIPSB
GRBLINK    Verb which blinks 2 TV graphics planes
GRCLEAR    clears the contents of the specified TV graphics channels
GRDROP     deletes the specified gripe entry
GREAD      reads the colors of the specified TV graphics channel
GRINDEX    lists users and time of all gripe entries
GRIPE      enter a suggestion or bug report for the AIPS programmers
GRLIST     lists contents of specified gripe entry
GROFF      turns off specified TV graphics channelS
GRON       turns on specified TV graphics channelS
GWRITE     reads the colors of the specified TV graphics channel
HELP       displays information on tasks, verbs, adverbs
HINOTE     adds user-generated lines to the history extension file
HITEXT     writes lines from history extension file to text file
HSA        puts the list of HSA antennas in the current file on stack
HUEWEDGE   Show a wedge on the TV suitable for TVHUEINT displays
IM2HEAD    displays the image 2 header contents to terminal, message file
IM2TV      Verb to convert pixel coordinates to TV pixels
IM3HEAD    displays the image 3 header contents to terminal, message file
IM4HEAD    displays the image 4 header contents to terminal, message file
IM5HEAD    displays the image 5 header contents to terminal, message file
IMCENTER   returns pixel position of sub-image centroid
IMDIST     determines spherical distance between two pixels
IMERASE    replaces an image portion of the TV screen with zeros
IMHEADER   displays the image header contents to terminal, message file
IMOHEAD    displays the output image header contents
IMPOS      displays celestial coordinates selected by the TV cursor
IMSTAT     returns statistics of a sub-image
IMVAL      returns image intensity and coordinate at specified pixel
IMWEDGE    load step wedge of full range of image values to TV
IMXY       returns pixel coordinates selected by the TV cursor
INP        displays adverb values for task, verb, or proc - quick form
INPUTS     displays adverb values for task, verb, or proc - to msg file
JOBLIST    lists contents of a submitted and pending batch job
KLEENEX    ends an AIPS interactive session wiping the slate klean
LENGTH     returns length of string to last non-blank character
LN         returns the natural logarithm of the argument
LOG        returns the base-10 logarithm of the argument
M2CAT      displays images in the user's catalog directory for IN2DISK
M3CAT      displays images in the user's catalog directory for IN3DISK
M4CAT      displays images in the user's catalog directory for IN4DISK
M5CAT      displays images in the user's catalog directory for IN5DISK
MAXFIT     returns pixel position and image intensity at a maximum
MAX        returns the maximum of its two arguments
MCAT       lists images in the user's catalog directory on disk INDISK
MFITSET    gets adverbs for running IMFIT and JMFIT
MIN        returns the minimum of its two arguments
MOCAT      displays images in the user's catalog directory for OUTDISK
MOD        returns remainder after division of 1st argument by 2nd
MODULUS    returns square root of sum of squares of its two arguments
MOUNT      makes a tape drive available to user's AIPS and tasks
NAMEGET    fills 1st input image name parameters by default matching
OBITIMAG   Access to OBIT task Imager without self-cal or peeling
OBITMAP    Simplified access to OBIT task Imager
OBITPEEL   Access to OBIT task Imager with self-cal and peeling
OBITSCAL   Access to OBIT task Imager with self-cal, NOT peeling
OFFHUINT   Proc which restores TV functions to normal after TVHUE
OFFPSEUD   Verb which deactivates all pseudo-color displays
OFFSCROL   Verb which deactivates scroll of an image
OFFTRAN    Verb which restores transfer function to normal
OFFZOOM    Verb which returns the hardware IIS zoom to normal
OFMADJUS   interactive linear adjustment of current TV OFM lookup tables
OFMCONT    creates/modifies TV color OFMs with level or wedged contours
OFMDIR     lists names of the user's and system's OFM files from OFMFIL
OFMGET     loads TV OFMS from an OFM save file
OFMLIST    lists the current TV OFM table(s) on the terminal or printer
OFMSAVE    saves the TV's current OFM lookup table in a text file
OFMTWEAK   interactive modification of current TV OFM lookup tables
OFMZAP     deletes an OFM lookup table save file
OUTPUTS    displays adverb values returned from task, verb, or proc
PARALLEL   Verb to set or show degree of parallelism
PASSWORD   Verb to change the current password for the login user
PCAT       Verb to list entries in the user's catalog (no log file).
PLGET      gets the adverbs used to make a particular plot file
PRINTER    Verb to set or show the printer(s) used
PRINT      Print the value of an expression
PROCEDUR   Define a POPS procedure using procedure editor
PROC       Define a POPS procedure using procedure editor.
PRTHI      prints selected contents of the history extension file
PRTMSG     prints selected contents of the user's message file
PSEUDOVB   Declares a name to be a symbol of type pseudoverb
PUTHEAD    Verb to modify image header parameters.
PUTTHEAD   inserts a given value into a table keyword/value pair
PUTVALUE   Verb to store a pixel value at specified position
Q2HEADER   Verb to summarize the image 2 header: positions at center
Q3HEADER   Verb to summarize the image 3 header: positions at center
Q4HEADER   Verb to summarize the image 4 header: positions at center
Q5HEADER   Verb to summarize the image 5 header: positions at center
QGETVERS   finds maximum version number of an extension file quietly
QHEADER    Verb to summarize the image header: positions at center
QIMVAL     returns image intensity and coordinate at specified pixel
QINP       displays adverb values for task, verb, or proc - restart form
QOHEADER   Verb to summarize the output image header: center positions
QUEUES     Verb to list all submitted jobs in the job queue
RANDOM     Compute a random number from 0 to 1
READ       Read a value from the users terminal
REBOX      Verb to reset boxes with TV cursor & graphics display.
RECAT      Verb to compress the entries in a catalog file
REHEX      converts extended hex string to decimal
REMDISK    removes a computer's disks from the current AIPS session
REMOVIE    Verb to rerun a previously loaded (TVMOVIE) movie
RENAME     Rename a file (UV or Image)
RENUMBER   Verb to change the catalog number of an image.
REROAM     Verb to use previous roam image mode, then does roam
RESCALE    Verb to modify image scale factor and offset
RESTART    Verb to trim the message log file and restart AIPS
RESTORE    Read POPS memory file from a common area.
RETURN     Exit a procedure allowing a higher level proc to continue.
REVERSN    checks disk for presence of extension files
REWIND     Verb to rewind a tape
ROAM       Roam around an image too large for the display.
ROAMOFF    Verb to recover image from roam display in simple display mode
RUN        Pseudoverb to read an external RUN files into AIPS.
SAVDEST    Verb to destroy all save files of a user.
SAVE       Pseudoverb to save full POPS environment in named file
SCALAR     Declares a variable to be a scalar in a procedure
SCAT       lists scratch files in the user's catalog directory on all disks
SCRATCH    delete a procedure from the symbol table.
SCRDEST    Verb to destroy scratch files left by bombed tasks.
SET1DG     Verb to set 1D gaussian fitting initial guesses.
SETDEBUG   Verb to set the debug print and execution level
SETMAXAP   Examines/alters system parameter limiting dynamic pseudo-AP
SETROAM    Verb use to set roam image mode, then do roam.  OBSOLETE
SETSLICE   Set slice endpoints on the TV interactively
SG2RUN     Verb copies the K area to a text file suitable for RUN
SGDESTR    Verb-like to destroy named POPS environment save file
SGINDEX    Verb lists SAVE areas by name and time of last SAVE.
SHOW       Verblike to display the TELL adverbs of a task.
SIN        Compute the sine of a value
SIZEFILE   return file size plus estimate of IMAGR work file size
SPY        Verb to determine the execution status of all AIPS tasks
SQRT       Square root function
STALIN     revises history by deleting lines from history extension file
STQUEUE    Verb to list pending TELL operations
STRING     Declare a symbol to be a string variable in POPS
SUBMIT     Verb which submits a batch work file to the job queue
SUBSTR     Function verb to specify a portion of a STRING variable
SYSTEM     Verb to send a command to the operating system
T1VERB     Temporary verb for testing (also T2VERB...T9VERB)
TABGET     returns table entry for specified row, column and subscript.
TABPUT     replaces table entry for specified row, column and subscript.
TAN        Tangent function
TAPES      Verb to show the TAPES(s) available
TGET       Verb-like gets adverbs from last GO of a task
TGINDEX    Verb lists those tasks for which TGET will work.
THEN       Specified the action if an IF test is true
TIMDEST    Verb to destroy all files which are too old
TK1SET     Verb to reset 1D gaussian fitting initial guess.
TKAGUESS   Verb to re-plot slice model guess directly on TEK
TKAMODEL   Verb to add slice model display directly on TEK
TKARESID   Verb to add slice model residuals directly on TEK
TKASLICE   Verb to add a slice display on TEK from slice file
TKERASE    Erase the graphics screen or window
TKGUESS    Verb to display slice model guess directly on TEK
TKMODEL    Verb to display slice model directly on TEK
TKPOS      Read a position from the graphics screen or window
TKRESID    Verb to display slice model residuals directly on TEK
TKSET      Verb to set 1D gaussian fitting initial guesses.
TKSLICE    Verb to display slice file directly on TEK
TKVAL      Verb to obtain value under cursor from a slice
TKXY       Verb to obtain pixel value under cursor
TO         Specifies upper limit of a FOR loop
TPHEAD     Verb to list image header from FITS or IBM-CV tape
TPUT       Verb-like puts adverbs from a task in file for TGETs
TV1SET     Verb to reset 1D gaussian fitting initial guess on TV plot.
TV3COLOR   Verb to initiate 3-color display using 3 TV channels
TVACOMPS   Verb to add slice model components directly on TV graphics
TVAGUESS   Verb to re-plot slice model guess directly on TV graphics
TVAMODEL   Verb to add slice model display directly on TV graphics
TVANOT     Verb to load anotation to the TV image or graphics
TVARESID   Verb to add slice model residuals directly on TV graphics
TVASLICE   Verb to add a slice display on TV graphics from slice file
TVBLINK    Verb which blinks 2 TV planes, can do enhancement also
TVBOX      Verb to set boxes with TV cursor & graphics display.
TVCLEAR    Verb to clear image from TV channel(s)
TVCOMPS    Verb to display slice model components directly on TV graphics
TVCUBE     Verb to load a cube into tv channel(s) & run a movie
TVFIDDLE   Verb enhances B/W or color TV image with zooms
TVGUESS    Verb to display slice model guess directly on TV graphics
TVHELIX    Verb to activate a helical hue-intensity TV pseudo-coloring
TVHUEINT   Verb to make hue/intensity display from 2 TV channels
TVILINE    Verb to draw a straight line on an image on the TV
TVINIT     Verb to return TV display to a virgin state
TVLABEL    Verb to label the (map) image on the TV
TVLAYOUT   Verb to label the holography image on the TV with panel layout
TVLINE     Verb to load a straight line to the TV image or graphics
TVLOD      Verb to load an image into a TV channel
TVLUT      Verb which modifies the transfer function of the image
TVMBLINK   Verb which blinks 2 TV planes either auto or manually
TVMLUT     Verb which modifies the transfer function of the image
TVMODEL    Verb to display slice model directly on TV graphics
TVMOVIE    Verb to load a cube into tv channel(s) & run a movie
TVNAME     Verb to fill image name of that under cursor
TVOFF      Verb which turns off TV channel(s).
TVON       Turns on one or all TV image planes
TVPHLAME   Verb to activate "flame-like" pseudo-color displays
TVPOS      Read a TV screen position using cursor
TVPSEUDO   Verb to activate three types of pseudo-color displays
TVRESID    Verb to display slice model residuals directly on TV graphics
TVROAM     Load up to 16 TV image planes and roam a subset thereof
TVSCROL    Shift position of image on the TV screen
TVSET      Verb to set slice Gaussian fitting initial guesses from TV plot
TVSLICE    Verb to display slice file directly on TV
TVSPLIT    Compare two TV image planes, showing halves
TVSTAR     Verb to plot star positions on top of a TV image
TVSTAT     Find the mean and RMS in a blotch region on the TV
TVTRANSF   Interactively alters the TV image plane transfer function
TVWEDGE    Show a linear wedge on the TV
TVWINDOW   Set a window on the TV with the cursor
TVWLABEL   Put a label on the wedge that you just put on the TV
TVZOOM     Activate the TV zoom
TYPE       Type the value of an expression
U2CAT      list a user's UV and scratch files on disk IN2DISK
U3CAT      list a user's UV and scratch files on disk IN3DISK
U4CAT      list a user's UV and scratch files on disk IN4DISK
U5CAT      list a user's UV and scratch files on disk IN5DISK
UCAT       list a user's UV and scratch files on disk INDISK
UNQUE      remove a given job from the job queue
UOCAT      list a user's UV and scratch files on disk OUTDISK
USAVE      Pseudoverb to save full POPS environment in named file
VALUE      Convert a string to a numeric value
VERB       Declares a name to be a symbol of type verb
VGET       Verb-like gets adverbs from version task parameter save area
VGINDEX    Verb lists those tasks for which VGET will work.
VLA        puts the list of VLA antennas in the current file on stack
VLBA       puts the list of VLBA antennas in the current file on stack
VPUT       Verb-like puts adverbs from a task in files for VGETs
WAITTASK   halt AIPS until specified task is finished
WEDERASE   Load a wedge portion of the TV with zeros
XHELP      Accesses hypertext help system
ZAP        Delete a catalog entry and its extension files
\end{verbatim}\eve

\vfill\eject
\sects{\hspace{0.5em}VLA}

\vskip 0.5pt\todx{ABOUT VLA}\iodx{VLA}
\bbve\begin{verbatim}
APGPS      Apply GPS-derived ionospheric corrections
ASDMFILE   Full path to EVLA ASDM/BDF directory
BDF2AIPS   Read EVLA ASDM/BDF data into AIPS
BDFLIST    List contents of EVLA ASD data file
BPWAY      Determines channel-dependent relative weights
BREAK      procedure to TELL FILLM to break all current uv files, start new
CALDIR     lists calibrator source models available as AIPS FITS files
CALIN      specifies name of input disk file usually with calibration data
CLCOR      applies user-selected corrections to the calibration CL table
CONFIG     Configuration ID number within an EVLA ASDM/BDF data set
CVEL       shifts spectral-line UV data to a given velocity
DAYFX      Fixes day number problems left by FILLM
DFCOR      applies user-selected corrections to the calibration CL table
DOOSRO     calibrating amplitude and phase, and imaging VLA data
DOVLAMP    Produces amp calibration file for phased-VLA VLBI data
FARAD      add ionospheric Faraday rotation to CL table
FEW        procedure to TELL FILLM to append incoming data to existing uv files
FILLM      reads VLA on-line/archive format uv data tapes (post Jan 88)
FIXAL      least squares fit aliasing function and remove
FLOPM      reverses the spectral order of UV data, can fix VLA error
FRMAP      Task to build a map using fringe rate spectra
FXALIAS    least squares fit aliasing function and remove
GCPLT      Plots gain curves from text files
GPSDL      Calculate ionospheric delay and Faraday rotation corrections
LDGPS      load GPS data from an ASCII file
MANY       procedure to TELL FILLM to start new uv files on each scan
MAPBM      Map VLA beam polarization
MORIF      Combines IFs or breaks spectral windows into multiple windows (IFs)
NCHAN      Number of spectral channels in each spectral window
NIF        Number of IFs (spectral windows) in a data set
PANEL      Convert HOLOG output to panel adjustment table
PCVEL      shifts spectral-line UV data to a given velocity: planet version
PEEK       fits pointing model function to output from the VLA
PIPEAIPS   calibrating amplitude and phase, and imaging VLA data
PRTSY      Task to print statistics from the SY table
QUIT       procedure to TELL FILLM to stop at the end of the current scan
REIFS      Breaks spectral windows into multiple spectral windows (IFs)
REWAY      computes weights based in rms in spectra
RFLAG      Flags data set based on time and freq rms in fringe visibilities
STOP       procedure to TELL FILLM to break all current uv files and stop
SYSOL      undoes and re-does nominal sensitivity application for Solar data
TECOR      Calculate ionospheric delay and Faraday rotation corrections
TELFLM     procedure to TELL real-time FILLM a new APARM(1) value
TLCAL      Converts JVLA telcal files to an SN file
TYAPL      undoes and re-does nominal sensitivity application
TYSMO      smooths and filters a calibration TY or SY table
USERLIST   Alphabetic and numeric list of VLA users, points to real list
VLABP      VLA antenna beam polarization correction for snapshot images
VLAMP      Makes ANTAB file for phased VLA used in VLBI observations
VLANT      applies VLA/EVLA antenna position corrections from OPs files
VLAPROCS   Procedures to simplify the reduction of VLBA data
VLARUN     calibrating amplitude and phase, and imaging VLA data
VLATECR    Calculate ionospheric delay and Faraday rotation corrections
\end{verbatim}\eve

\sects{\hspace{0.5em}VLBI}

\vskip 0.5pt\todx{ABOUT VLBI}\iodx{VLBI}
\bbve\begin{verbatim}
ACCOR      Corrects cross amplitudes using auto correlation measurements
ACFIT      Determine antenna gains from autocorrelations
ACSCL      Corrects cross amplitudes using auto correlation measurements
AFILE      sorts and edits MkIII correlator A-file.
ALIAS      adverb to alias antenna numbers to one another
ANBPL      plots and prints  uv data converted to antenna based values
ANCAL      Places antenna-based Tsys and gain corrections in CL table
ANTAB      Read amplitude calibration information into AIPS
ANTNAME    A list of antenna (station) names
APCAL      Apply TY and GC tables to generate an SN table
ASTROMET   Describes the process of astrometric/geodetic reduction in AIPS
BANDPOL    specifies polarizations of individual IFs
BLING      find residual rate and delay on individual baselines
BSCAN      seeks best scan to use for phase cal, fringe search, ..
BSPRT      print BS tables
CALIN      specifies name of input disk file usually with calibration data
CAPLT      plots closure amplitude and model from CC file
CL2HF      Convert CL table to HF table
CLCOR      applies user-selected corrections to the calibration CL table
CLPLT      plots closure phase and model from CC file
CLVLB      Corrects CL table gains for pointing offsets in VLBI data
CROSSPOL   Procedure to make complex poln. images and beam.
CRSFRING   Procedure to calibrate cross pol. delay and phase offsets
CVEL       shifts spectral-line UV data to a given velocity
CXPOLN     Procedure to make complex poln. images and beam.
DFCOR      applies user-selected corrections to the calibration CL table
DOVLAMP    Produces amp calibration file for phased-VLA VLBI data
DTSIM      Generate fake UV data
EDITR      Interactive baseline-oriented visibility editor using the TV
FRING      fringe fit data to determine antenna calibration, delay, rate
FRMAP      Task to build a map using fringe rate spectra
FRPLT      Task to plot fringe rate spectra
FXAVG      Procedure to enable VLBA delay de-correlation corrections
FXPOL      Corrects VLBA polarization assignments
HF2SV      convert HF tables from FRING/MBDLY to form used by Calc/Solve
HFPRT      write HF tables from CL2HF
HLPEDIPC   Interactive PC table editor PCEDT - run-time help
HLPEDIPD   Interactive PD table editor PDEDT - run-time help
HLPIBLED   Interactive Baseline based visibility Editor - run-time help
HLPSCIMG   Full-featured image plus self-cal loops, editing - run-time help
HLPSCMAP   Imaging plus self-cal and editing SCMAP - run-time help
IBLED      Interactive BaseLine based visibility EDitor
KRING      fringe fit data to determine antenna calibration, delay, rate
M3TAR      translate Haystack MKIII VLBI format "A" TAR's into AIPS
MATCH      changes antenna, source, FQ numbers to match a data set
MBDLY      Fits multiband delays from IF phases, updates SN table
MERGECAL   Procedure to merge calibration records after concatenation
MK3IN      translate Haystack MKIII VLBI format "A" tapes into AIPS
MK3TX      extract text files from a MKIII VLBI archive tape
OBPLT      Plot columns of an OB table.
OOFRING    fringe fit data to determine antenna calibration, delay, rate
PCASS      Finds amplitude bandpass shape from pulse-cal table data
PCAVG      Averages pulse-cal (PC) tables over time
PCCOR      Corrects phases using  PCAL tones data from PC table
PCEDT      Interactive TV task to edit pulse-cal (PC) tables
PCFIT      Finds delays and phases using a pulse-cal (PC) table
PCFLG      interactive flagging of Pulse-cal data in channel-TB using the TV
PCHIS      Generates a histogram plot file from text input, e.g. from PCRMS
PCLOD      Reads ascii file containing pulse-cal info to PC table.
PCPLT      Plots pulse-cal tables in 2 dimensions as function of time
PCRMS      Finds statistics of a pulse-cal table; flags bad times and channels
PCVEL      shifts spectral-line UV data to a given velocity: planet version
PHSLIMIT   gives a phase value in degrees
POLSN      Make a SN table from cross polarized fringe fit
RESEQ      Renumber antennas
RLDLY      fringe fit data to determine antenna R-L delay difference
SEARCH     Ordered list of antennas for fring searches
SNEDT      Interactive SN/CL/TY/SY table editor using the TV
TAU0       Opacities by antenna number
TECOR      Calculate ionospheric delay and Faraday rotation corrections
TFILE      sorts and edits MkIII correlator UNIX-based A-file.
TRECVR     Receiver temperatures by polarization and antenna
UVPOL      modifies UV data to make complex image and beam
VBCAL      Scale visibility amplitudes by antenna based constants
VBGLU      Glues together data from multiple passes thru the VLBA corr.
VBMRG      Merge VLBI data, eliminate duplicate correlations
VLAMP      Makes ANTAB file for phased VLA used in VLBI observations
VLASUMM    Plots selected contents of SN or CL files
VLBAAMP    applies a-priori amplitude corrections to VLBA data
VLBAARCH   Procedure to archive VLBA correlator data
VLBACALA   applies a-priori amplitude corrections to VLBA data
VLBACCOR   applies a-priori amplitude corrections to VLBA data
VLBACPOL   Procedure to calibrate cross-polarization delays
VLBACRPL   Plots crosscorrelations
VLBAEOPS   Corrects Earth orientation parameters
VLBAFIX    Procedure that fixes VLBA data, if necessary
VLBAFPOL   Checks and corrects polarization labels for VLBA data
VLBAFQS    Copies different FQIDS to separate files
VLBAFRGP   Fringe fit phase referenced data and apply calibration
VLBAFRNG   Fringe fit data and apply calibration
VLBAIT     Procedure to read and process VLBA data (Phil Diamond)
VLBAKRGP   Fringe fit phase referenced data and apply calibration
VLBAKRNG   Fringe fit data and apply calibration
VLBALOAD   Loads VLBA data
VLBAMCAL   Merges redundant calibration data
VLBAMPCL   Calculates and applies manual instrumental phase calibration
VLBAPANG   Corrects for parallactic angle
VLBAPCOR   Calculates and applies instrumental phase calibration
VLBAPIPE   applies amplitude and phase calibration procs to VLBA data
VLBARUN    applies amplitude and phase calibration procs to VLBA data
VLBASNPL   Plots selected contents of SN or CL files
VLBASRT    Sorts VLBA data, if necessary
VLBASUBS   looks for subarrays in VLBA data
VLBASUMM   Prints a summary of a VLBI experiment
VLBATECR   Calculate ionospheric delay and Faraday rotation corrections
VLBAUTIL   Procedures to simplify the reduction of VLBA data
VLBIN      Task to read VLBI data from an NRAO/MPI MkII correlator
VLBINPRM   Control parameters to read data from NRAO/MPI MkII correlators
VLOG       Pre-process external VLBA calibration files
VPLOT      plots uv data and model from CC file
WEIGHTIT   Controls modification of weights before gain/fringe solutions
\end{verbatim}\eve

\normalsize

\sects{Additional recipes}

% chapter 13 *************************************************
\recipe{Saut\'ed sole tobago with bananas, pecans and lime}

\bre
\Item {Preheat 1/2 cup {\bf vegetable oil} in a heavy sauce pan over
     medium-high heat.}
\Item {Dredge 8 filets of {\bf sole} or {\bf flounder} lightly in {\bf
     flour}.}
\Item {Saut\'ee until golden brown, about 3 minutes each side.  Remove
     to warm platter.}
\Item {Pour off excess oil and wipe down sauce pan. Place pan back on
     stove over high head; add 1/4 cup {\bf butter}.}
\Item {When foamy and just starting to brown, add 2 cups diagonally
     sliced {\bf bananas} (1/2'' slices) and 1 cup {\bf pecan} halves.
     Toss and cook for 1 minute.}
\Item {Add 1/2 cup  fresh {\bf lime juice} and 1 cup dry white {\bf
     wine} (or light stock) . Cook for another 2 minutes.}
\Item {Add 1/4 cup {\bf fresh herbs} (mint, parsley, coriander, basil
     or tarragon).}
\Item {Pour sauce and bananas over fish. Garnish with additional
     banana slices and lime wedges. }
\item[ ]{\hfill Thanks to Turbana Corporation ({\tt www.turbana.com}).}
\ere

% chapter *************************************************
\recipe{Banana sweet potato puff casserole}

\bre
\Item {In a large bowl, combine 2 cups mashed {\bf sweet potatoes}, 1
     cup mashed ripe {\bf bananas} (3 medium), 3/4 teaspoon {\bf curry
     powder}, 1/3 cup {\bf sour cream}, 1/2 teaspoon {\bf salt}, and 1
     {\bf egg}.}
\Item {Beat with electric mixer until light and very fluffy. Turn into
     1 quart casserole dish.}
\Item {Bake at \dgg{350} for 20 minutes or until puffed and lightly
     browned.}
\item[ ]{\hfill Thanks to Turbana Corporation ({\tt www.turbana.com}).}
\ere

% chapter *************************************************
\recipe{Cream of banana soup}

\bre
\Item {Cook 1 quart green {\bf banana} pulp, 1 1/2 quarts {\bf chicken
     stock}, 1 small {\bf celery stalk}, 1/2 {\bf onion}, 1 {\bf
     carrot}, 1 small {\bf bay leaf}, 5 {\bf peppercorns}, and {\bf
     salt} to taste together for about 30 minutes until the mixture
     thickens.}
\Item {Strain over 1/4 cup {\bf flour} and 1/4 cup {\bf butter} which
     have been combined as for a white sauce.  Cook until thickened.}
\Item {Just before serving, add 2 cups {\bf cream} or {\bf milk} and
     heat.}
\Item {Serve with a slice of lemon on each plate as a garnish.}
\ere
