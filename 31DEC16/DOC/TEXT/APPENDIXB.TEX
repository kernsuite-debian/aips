%-----------------------------------------------------------------------
%! Going AIPS Appendix B
%# Documentation LaTeX
%-----------------------------------------------------------------------
%;  Copyright (C) 1995
%;  Associated Universities, Inc. Washington DC, USA.
%;
%;  This program is free software; you can redistribute it and/or
%;  modify it under the terms of the GNU General Public License as
%;  published by the Free Software Foundation; either version 2 of
%;  the License, or (at your option) any later version.
%;
%;  This program is distributed in the hope that it will be useful,
%;  but WITHOUT ANY WARRANTY; without even the implied warranty of
%;  MERCHANTABILITY or FITNESS FOR A PARTICULAR PURPOSE.  See the
%;  GNU General Public License for more details.
%;
%;  You should have received a copy of the GNU General Public
%;  License along with this program; if not, write to the Free
%;  Software Foundation, Inc., 675 Massachusetts Ave, Cambridge,
%;  MA 02139, USA.
%;
%;  Correspondence concerning AIPS should be addressed as follows:
%;          Internet email: aipsmail@nrao.edu.
%;          Postal address: AIPS Project Office
%;                          National Radio Astronomy Observatory
%;                          520 Edgemont Road
%;                          Charlottesville, VA 22903-2475 USA
%-----------------------------------------------------------------------
%-----------------------------------------------------------------------
\setcounter{chapter}{1} % really appendix B
\chapter{Shopping lists}
\setcounter{page}{1}
\section{Introduction}
This appendix contains the one line descriptions of each of the AIPS
system subroutines that may be called from applications software
arrainged by category (a given routine may have several entries).
Not all of the subroutines described in the following lists may be
called for all applications software.  In particular, routines in
directory AIPSUB may only be called from program AIPS or other tasks
in the AIPPGM directory.  Z2 and Y3 routines may only be called from
other ``Z'' or ``Y'' routines.


This list should simplify finding the appropriate routine inside the
AIPS system.  Each routine name is prefixed with the logical name of
the directory in which it resides.  A summary of the categories is
given below.

\begin{itemize}  % list level 1
\item AP-APPL  These are routines that use ``Array Processor''
routines for a particular operation.
\item AP-FFT  These are routines that use ``Array Processor''
routines for FFT (Fast Fourier Transform) operation.
\item AP-UTIL  These are utility routines that use ``Array Processor''
routines.
\item BATCH  These routines are related to AIPS batch functions.
\item BINARY  These routines process external binary format data.
\item CALIBRATION  These routines are related to the calibration
package of routines.
\item CATALOG  These routines are related to the AIPS catalog.
\item CHARACTER  There routines are AIPS character manipulating
functions.
\item COORDINATES  These routines manipulate astronomical coordinate
systems.
\item EXT-APPL These are applications routines for extension files;
generally tables.
\item EXT-UTIL  These are utility routines for extension files.
\item FITS  These routine are for processing data in FITS files.
\item GRAPHICS These are the AIPS graphics routines.
\item HEADER  These routines process AIPS catalog header records.
\item HISTORY  These routines process AIPS history files or records.
\item IO-APPL  These are applications routines for the AIPS I/O system.
\item IO-BASIC  These are the basic routines for the AIPS I/O system.
\item IO-TV  These are the routines that communicate with the image
display.
\item IO-UTIL  These are utility routines for the AIPS I/O system.
\item IO-WAWA  These are the ``WAWA'' or ``Easy IO'' package of
routines.
\item MAP  These routines deal with images.
\item MAP-UTIL  These are utility routines dealing with images.
\item MATH  These are basic mathematical routines.
\item MESSAGES  These routines deal with sending messages to the user.
\item MODELING  These routine involve model fitting or calculation.
\item PARSING  These routine involve parsing information from
character strings.
\item PLOT-APPL  These are applications plotting routines.
\item PLOT-UTIL  These are utility plotting routines.
\item POPS-APPL  These are POPS applications routines (verbs).
\item POPS-LANG  These are parts of the POPS language processor.
\item POPS-UTIL  These are POPS utility routines.
\item PRINTER  These are routines related to printers.
\item SDISH  These are routines for processing single dish data.
\item SERVICE  These are various service routines.
\item SLICE  These are routines that deal with slices through images.
\item SORT  These are sorting routines.
\item SPECTRAL  These are routine related to spectroscopy.
\item SYSTEM  These are AIPS system functions.
\item TAPE  These are routines related to reading tape or other
external binary files.
\item TERMINAL  These are routines for I/O to user terminals
\item TEXT  These are routines related to text files.
\item TV  These are routines related to the image display.
\item TV-APPL  These are applications routines related to the image
display.
\item TV-BASIC  These are basic image display routines.
\item TV-IO  These are I/O applications routines related to the image
display.
\item TV-UTIL  These are utility routines related to the image
display.
\item UTILITY  These are general utility routines.
\item UV  These routines deal with uv (interferometer) data.
\item UV-UTIL  These utility routines deal with uv (interferometer)
data.
\item VLA  These are routines that are specific to the VLA (NRAO Very
Large Array)
\item Y0  These are the main top level TV (image display) routines.
\item Y1  These are the second level TV routines.
\item Y2  These are the IIS specific TV routines; these are unlikely
to be supported on othe displays.
\item Y3  These are ``Y'' routines that can only be called from other
``Y'' routines.
\item Z  These are routines which may contain system dependent
functions.
\item Z-2  These are routines which may contain system dependent
functions but may only be called from other ``Z'' routines.
\end{itemize}  % list level 1
 
 
\subsection{AP-APPL}
\begin{verbatim}
QNOT:ALGSUB.FOR        Interpolates model visibility grom a grid and subtracts from uv data.
QNOT:APCONV.FOR        Disk based 2-D convolution using FFTs.
QNOT:CONV.FOR          *TESS routine: Convolve a map with a beam.
QNOT:DISPTV.FOR        *TESS routine: Display an image on a TV
QPSAP:Q1FIN.FOR        Finish gridding a row of uv data.
QPSAP:Q1GRD.FOR        Grid a uv data.
QPSAP:QBAKSU.FOR       Back substitution.
QPSAP:QBOXSU.FOR       Boxcar sum of a vector.
QPSAP:QCLNSU.FOR       Low level Clark CLEAN routine.
QPSAP:QCRVMU.FOR       Complex-real vector multiply.
QPSAP:QCSQTR.FOR       inplace transpose of square, complex matrix.
QPSAP:QCTLUT.FOR       Initialize cosine lookup table etc.
QPSAP:QCVCMU.FOR       Scalar complex times conjugate of vector to real.
QPSAP:QCVCON.FOR       Complex conjugate of a vector.
QPSAP:QCVEXP.FOR       Vector complex exponential.
QPSAP:QCVJAD.FOR       Complex vector conjugate of vector add.
QPSAP:QCVMAG.FOR       Complex vector magnitude squared.
QPSAP:QCVMMA.FOR       Max. square of modulus of complex vector.
QPSAP:QCVMOV.FOR       Complex vector move.
QPSAP:QCVMUL.FOR       Complex vector multiply.
QPSAP:QCVSDI.FOR       Divide weighted complex vector by complex scalar.
QPSAP:QCVSMS.FOR       Subtract real vector*complex scalar from vector.
QPSAP:QDIRAD.FOR       Directed vector add.
QPSAP:QFINGR.FOR       Finish gridding row of uv data.
QPSAP:QGADIV.FOR       Divide Gaus. model vis. into uv data.
QPSAP:QGASUB.FOR       Subtract Gaus. model vis. from uv data.
QPSAP:QGET.FOR         Move data from pseudo-AP memory to "host".
QPSAP:QGRD1.FOR        Convolves visibility data onto a grid.
QPSAP:QGRD2.FOR        Convolves linear polarization data onto a grid.
QPSAP:QGRD3.FOR        Convolve visibility data onto a grid.
QPSAP:QGRD4.FOR        Convolves visibility data onto a grid.
QPSAP:QGRDCC.FOR       Grid and FT Clean components.
QPSAP:QGRDFI.FOR       Finish griding a row of uv data.
QPSAP:QGRDMI.FOR       Combined complex vector in gridding uv data.
QPSAP:QGRID.FOR        Grid uv data into row.
QPSAP:QGRIDA.FOR       Grid visibility data.
QPSAP:QHIST.FOR        Make histogram of a vector.
QPSAP:QINT.FOR         Interpolates model visibilities from a grid.
QPSAP:QINTP.FOR        Interpolates model visibilities from a grid.
QPSAP:QLVGT.FOR        Vector logical greater than.
QPSAP:QMAKMS.FOR       Make mask depending on vector, scalar comparison.
QPSAP:QMAXMI.FOR       Find maximum and minimum of a vector.
QPSAP:QMAXV.FOR        Find maximum value element of a vector.
QPSAP:QMCALC.FOR       Compute model visibility from point model.
QPSAP:QMENT.FOR        MEM routine
QPSAP:QMINV.FOR        Find minimum value element of a vector
QPSAP:QMTRAN.FOR       matrix transpose.
QPSAP:QMTYP.FOR        Chose DFT or gridded interpolation method.
QPSAP:QMULCL.FOR       High level Clark CLEAN routine
QPSAP:QPHSRO.FOR       Add phase gradient to a complex array.
QPSAP:QPOLAR.FOR       Vector rectangular-to-polar conversion.
QPSAP:QPTDIV.FOR       Divide point model visibility into uv data.
QPSAP:QPTFAZ.FOR       Compute phase in model visibilities.
QPSAP:QPTSUB.FOR       Subtract point model visibility from uv data.
QPSAP:QRECT.FOR        Vector polar-to-rectangular conversion.
QPSAP:QRFT.FOR         Does real, inverse FT with arbitrary spacing.
QPSAP:QSEARC.FOR       VLBI fringe search with FFT.
QPSAP:QSPDIV.FOR       Divide Gaussian model  visibility into uv data.
QPSAP:QSPSUB.FOR       Subtract Gaussian model visibility from uv data.
QPSAP:QSVE.FOR         Sum the elements of a vector.
QPSAP:QSVESQ.FOR       Sum the squares of the elements of a vector.
QPSAP:QUVIN.FOR        Interpolate visibility model from a grid.
QPSAP:QUVINT.FOR       Interpolate model visibility from grid.
QPSAP:QVABS.FOR        Vector absolute value.
QPSAP:QVADD.FOR        Vector add.
QPSAP:QVCLIP.FOR       Vector clip.
QPSAP:QVCLR.FOR        Vector zero.
QPSAP:QVCOS.FOR        Vector cosine.
QPSAP:QVDIV.FOR        Vector divide.
QPSAP:QVEXP.FOR        Vector exponentiate.
QPSAP:QVFILL.FOR       Vector fill.
QPSAP:QVFIX.FOR        Vector fix.
QPSAP:QVFLT.FOR        Vector float.
QPSAP:QVIDIV.FOR       Divide a vector by the product of two integers.
QPSAP:QVINDE.FOR       Vector index (gather)
QPSAP:QVLN.FOR         Vector natural logrithm
QPSAP:QVMA.FOR         Vector multiply and vector add.
QPSAP:QVMOV.FOR        Vector move.
QPSAP:QVMUL.FOR        Vector multiply.
QPSAP:QVNEG.FOR        Negate the elements of a vector.
QPSAP:QVRVRS.FOR       Reverse the elements of a vector.
QPSAP:QVSADD.FOR       Vector scalar add.
QPSAP:QVSIN.FOR        Vector sine.
QPSAP:QVSMA.FOR        Vector scalar multiply and vector add.
QPSAP:QVSMAF.FOR       Scalar multiply and and round.
QPSAP:QVSMSA.FOR       Vector scalar multiply and scalar add.
QPSAP:QVSMUL.FOR       Vector scalar multiply.
QPSAP:QVSQ.FOR         Square vector.
QPSAP:QVSQRT.FOR       Vector square root.
QPSAP:QVSUB.FOR        Vector subtract.
QPSAP:QVSWAP.FOR       Vector swap.
QPSAP:QVTRAN.FOR       Inplace transpose of a matrix of vectors.
QPSAP:QVTSMU.FOR       Vector table scalar multiply.
QPSAP:QXXPTS.FOR       Subtract point model visibility from uv data.
QNOT:VISDFT.FOR        Compute DFT of model and subtract/divide from/into uv data.
\end{verbatim}
 
\subsection{AP-FFT}
\begin{verbatim}
APLSUB:AP2SIZ.FOR      returns largest power of 2 not exceeding 1024 times first argument
QSUB:APXPOS.FOR        In place transpose of complex array.
QNOT:CONV1.FOR         First of four routines to convolve two real images.
QNOT:CONV2.FOR         Second of four routines to convolve two real images.
QNOT:CONV3.FOR         Third of four routines to convolve two real images.
QNOT:CONV4.FOR         Fourth of four routines to convolve two real images.
APLNOT:DSKFFT.FOR      2-D disk based FFT using AP.
APLNOT:EMPTY1.FOR      DSKFFT utility routine
APLNOT:EMPTY2.FOR      DSKFFT utility routine
QNOT:FFTIM.FOR         FFTs an image for uv interpolation.
APLNOT:FILL1.FOR       DSKFFT utility routine
APLNOT:FILL2.FOR       DSKFFT utility routine.
QNOT:MAKMAP.FOR        Makes image or beam from uv data set.
APLSUB:MINSK.FOR       Inits use of MSKIP to read noncontiguous, evenly spaced rows in a map
APLSUB:MSKIP.FOR       Reads noncontiguous, but evenly spaced rows in a map (see also MINSK)
QSUB:PASS1.FOR         First of two routines to FFT an image file.
QSUB:PASS2.FOR         Second of two routines to FFT an image file.
QPSAP:QCFFT.FOR        Complex 1-D FFT.
QPSAP:QRFFT.FOR        Real-half plane complex FFT
\end{verbatim}
 
\subsection{AP-UTIL}
\begin{verbatim}
QNOT:APIO.FOR          Copies image-like data between disk and "AP memory".
QSUB:APROLL.FOR        Copies AP "memory" to disk, gives up AP then reloads AP
QNOT:CCSGRD.FOR        Transforms CLEAN components to a grid.
QNOT:CONVFN.FOR        Computes convolving fn. kernels and stores them in "AP memory"
QNOT:GRDCOR.FOR        Normalizes and corrects image for gridding convolution fn.
QNOT:GRDCRM.FOR        Loads CLEAN components into AP for uv model computation.
QNOT:GRDSUB.FOR        Subtracts transform of CLEAN components from uv data.
QNOT:GRDTAB.FOR        Computes Fourier transform of gridding convolution function.
QNOT:INTPFN.FOR        Computes interpolation kernals and put them into "AP memory".
QNOT:MAKMAP.FOR        Makes image or beam from uv data set.
QPSAP:QGSP.FOR         Read "S-pad" register
QPSAP:QINIT.FOR        Initialize "AP".
QPSAP:QPUT.FOR         Move data from "host" to "AP" memory.
QPSAP:QRLSE.FOR        Release "AP".
QSUB:QROLL.FOR         Determines if time to roll AP, if so calls APROLL.
QPSAP:QWAIT.FOR        Suspend host until AP done.
QPSAP:QWD.FOR          Suspend host until AP data transfer done.
QPSAP:QWR.FOR          Suspend host until AP computations complete.
QNOT:UVGRID.FOR        Grids uv data to be FFTed.
QNOT:UVMDIV.FOR        Divides a uv data set by the Fourier transform of a model.
QNOT:UVMSUB.FOR        Subtracts the Fourier transform of a model from a uv data set.
QNOT:UVMTYP.FOR        Determines relative CPU times for DFT or gridded interpolation.
QNOT:UVTBGD.FOR        Grids uv data in arbitrary sort order to be FFTed.
QNOT:UVTBUN.FOR        Determines and applies uniform weighting to uv data in arb. order.
QNOT:UVUNIF.FOR        Determines and applies uniform weighting to a uv data set.
\end{verbatim}
 
\subsection{BATCH}
\begin{verbatim}
AIPSUB:AUA.FOR         verb to submit batch jobs to AIPSC and the QMNGR queues
AIPSUB:AUB.FOR         verbs to prepare, edit, and review batch jobs and queues
APLSUB:BATPRT.FOR      prints header/trailer messages for printer tasks when run in batch
APLSUB:BATQ.FOR        performs operations on batch queue control file such as OPEN RUN CLOS
AIPSUB:BBUILD.FOR      reads input lines and adds them to the text file for a batch job
\end{verbatim}
 
\subsection{BINARY}
\begin{verbatim}
APLGEN:ZBYMOV.FOR      move 8-bit bytes from in-buffer to out-buffer
APLGEN:ZBYTFL.FOR      interchange bytes in buffer if needed to go between local & standard
APLGEN:ZC8CL.FOR       convert packed ASCII buffer to local character string
APLGEN:ZCLC8.FOR       convert local character string to packed ASCII buffer
APLGEN:ZDHPRL.FOR      convert 64-bit HP floating buffer to local DOUBLE PRECISION values
APLGEN:ZGETCH.FOR      get a character from a REAL word
APLGEN:ZI16IL.FOR      convert FITS-standard 16-bit integers to local integers
APLGEN:ZI32IL.FOR      convert FITS-standard 32-bit integers from buffer into local integers
APLGEN:ZI8IL.FOR       convert 8-bit unsigned integers in buffer to local integers
APLGEN:ZILI16.FOR      convert local integers to 16-bit FITS integers in a buffer
APLGEN:ZILI32.FOR      convert local integer into FITS-standard 32-bit integers
APLGEN:ZPUTCH.FOR      inserts 8-bit "character" into a word
APLGEN:ZR32RL.FOR      convert 32-bit IEEE floating buffer to local REAL values
APLGEN:ZR64RL.FOR      convert 64-bit IEEE floating-point buffer to local "DOUBLE PRECISION"
APLGEN:ZR8P4.FOR       converts pseudo I*4 to double precision - for tape handling only
APLGEN:ZRDMF.FOR       convert DEC Magtape Format (36 bits data in 40 bits) to 2 integers
APLGEN:ZRHPRL.FOR      convert 32-bit HP floating buffer to local REAL values
APLGEN:ZRLR32.FOR      converts buffer of local REAL values to IEEE 32-bit floating-point
APLGEN:ZRLR64.FOR      convert buffer of local double precision values to IEEE 64-bit float.
APLGEN:ZRM2RL.FOR      convert Modcomp to local single precision floating point
APLGEN:ZUVPAK.FOR      Pack visibility data, 1 correlator per real with magic value blank.
APLGEN:ZUVXPN.FOR      Expands packed visibility data and adds weight
\end{verbatim}
 
\subsection{CALIBRATION}
\begin{verbatim}
APLNOT:BLGET.FOR       Sets up for interpolation in baseline (BL) table
APLNOT:BLINI.FOR       Create/open/init I/O to BL table
APLNOT:BLREFM.FOR      Checks existence of BL table, changes format if necessary
APLNOT:BLSET.FOR       Fills current baseline calibration table
APLNOT:BPASET.FOR      Sets up the bandpass table array for use by DATBND.
APLNOT:BPGET.FOR       Sets bandpass correction arrays in common
APLNOT:BPINI.FOR       Create/open/initialize bandpass (BP) table
APLNOT:BPREFM.FOR      Checks existence of BP table, changes format if necessary
APLNOT:CALADJ.FOR      Adjusts solution (SN) table phases to a common reference antenna.
APLNOT:CALCOP.FOR      Copies selected uv data with calibration and editing
APLNOT:CALINI.FOR      Creates/opens/initializes calibration (CL) table
APLNOT:CALREF.FOR      Adjusts the reference antenna in an SN table.
APLNOT:CGASET.FOR      Maintains calibration values in an array in common
APLNOT:CHNCOP.FOR      Copies selected portions of the IF table
APLNOT:CHNDAT.FOR      Creates/Opens/Reads/Writes/Closes an IF table.
APLNOT:CLREFM.FOR      Checks existence of CL table, changes format if necessary
APLNOT:CLUPDA.FOR      Concatenates, rereferences, smooths SN tables and applies it to CL.
APLNOT:CMPARM.FOR      Determines blocks of data in a vis. record to decompress
APLNOT:CSINI.FOR       Create/Open/Init Single dish calibration (CS) table
APLNOT:CSLGET.FOR      Reads CL (or SN) table and sets up for interpolation.
APLNOT:DATBND.FOR      Applies the bandpass correction to data.
APLNOT:DATCAL.FOR      Applies calibration to data
APLNOT:DATFLG.FOR      Flags data specified in flagging table
APLNOT:DATGET.FOR      Reads, selects, calibrates and edits data.
APLNOT:DATPOL.FOR      Apply polarization corrections to data.
APLNOT:DCALSD.FOR      Apply Single dish calibration to data.
APLNOT:DGETSD.FOR      Reads, selects single dish data, calibrates and edits.
APLNOT:DGGET.FOR       Selects uv data and changes Stokes
APLNOT:DGHEAD.FOR      Fills output CATBLK for UVGET
APLNOT:DGINIT.FOR      Sets arrays for selecting data and changing Stokes
APLNOT:FLAGUP.FOR      Updates the Flag (FG) table.
APLNOT:FLGINI.FOR      Create/Open/Init Flag (FG) table.
APLNOT:FLGSTK.FOR      Set Stokes flag for uv flagging.
APLNOT:FNDSOU.FOR      Find source numbers for a list of sources.
APLNOT:FQINI.FOR       Create/open/initialize frequency (FQ) table
APLNOT:FQMATC.FOR      Check if selection criteria match FQ table entries.
APLNOT:GACSIN.FOR      Initializes CS file, and prepares table to be applied.
APLNOT:GAINI.FOR       Creates and initializes gain (GA) extension tables.
APLNOT:GAININ.FOR      Initializes calibration table for application.
APLNOT:GETFQ.FOR       Find info on a given frequency id.
APLNOT:GETSOU.FOR      Find info on a given source id.
APLNOT:INDXIN.FOR      Initializes index (NX) file, finds first scan selected.
APLNOT:IOBSRC.FOR      Search for antennas in the current bandpass buffer.
APLNOT:LXYPOL.FOR      Fills polarization correction table for AT like linear polarization.
APLNOT:MULSDB.FOR      Determines if a uv file is multi- or single- source.
APLNOT:NDXINI.FOR      Create/open/init index (NX) table
APLNOT:NXTFLG.FOR      Manages flagging info in tables in common.
APLNOT:PARANG.FOR      Computes antenna parallactic angles
APLNOT:POLSET.FOR      Fills polarization correction table from info in AN table.
APLNOT:SCINTP.FOR      Interpolates bandpass tables in time.
APLNOT:SCLOAD.FOR      Copies part of one bandpass scratch file to another for efficiency.
APLNOT:SDCGET.FOR      Sets up to interpolate in Single dish calibration (CS) table.
APLNOT:SDCSET.FOR      Interpolates single dish calibration data for current time.
APLNOT:SDGET.FOR       Reads single dish data with optional calibration and flagging
APLNOT:SELINI.FOR      Initialize data selection and control in commons in DSEL.INC
APLNOT:SELSMG.FOR      Selects calibrator data, smooths solutions.
APLNOT:SET1VS.FOR      Sets up pointer and weights arrays for selecting uv data.
APLNOT:SETSM.FOR       Determines type of spectral smoothing and sets up look up table.
APLNOT:SETSTK.FOR      Sets STOKES parameters correctly for plotting routines
APLNOT:SMOSP.FOR       Convolves a spectrum with a tabulated function.
APLNOT:SN2CL.FOR       Apply an SN to a CL table.
APLNOT:SNAPP.FOR       Append SN tables and keep track of reference antennas.
APLNOT:SNINI.FOR       Create/open/initialize solution (SN) tables.
APLNOT:SNREFM.FOR      Checks existence of SN table, changes format if necessary
APLNOT:SNSMO.FOR       Smooths solution (SN) tables
APLNOT:SOUELV.FOR      Computes source hour angles and elevations
APLNOT:SOUFIL.FOR      Fills in arrays of source numbers to be included or excluded.
APLNOT:SOURNU.FOR      Look up source numbers for a list of names.
APLNOT:TABBL.FOR       Do IO to Baseline (BL) table after setup by BLINI.
APLNOT:TABBP.FOR       Does I/O to bandpass (BP) table opened by BPINI
APLNOT:TABCAL.FOR      Does I/O to Calibration (CL) table opened by CALINI
APLNOT:TABCS.FOR       Does I/O to single dish calibration (CS) table opened by CSINI
APLNOT:TABFLG.FOR      Does I/O to Flag (FG) table opened by FLGINI
APLNOT:TABFQ.FOR       Does I/O to frequency (FQ) table opened by FQINI
APLNOT:TABGA.FOR       Does I/O to GAIN (GA) table opened by GAINI
APLNOT:TABNDX.FOR      Does I/O to Index (NX) table opened by NDXINI
APLNOT:TABSN.FOR       Does I/O to Solution (SN) table opened by SNINI
APLNOT:TABSOU.FOR      Does I/O to Source (SU) table opened by SOUINI
APLNOT:TABTY.FOR       Does I/O to Tsys (TY) table opened by TYINI
APLNOT:TYINI.FOR       Create/open/initialize Tsys (TY) table
APLNOT:UVGET.FOR       Read UV data with optional calibration, editing, selection, etc.
APLNOT:VISCNT.FOR      Determines number of visibility records requested of UVGET
\end{verbatim}
 
\subsection{CATALOG}
\begin{verbatim}
AIPSUB:AU3.FOR         Verbs to display contents of catalogs and headers: CATA, IMHE ...
AIPSUB:AU7.FOR         Verbs to print history, rescale image, alter axis descriptions
AIPSUB:AU8.FOR         Verbs to get or clear name adverbs, destroy extension files
AIPSUB:CATCR.FOR       Create and initialize catalog (CA) files
APLSUB:CATDIR.FOR      Manipulates the catalog directory: OPEN, CLOS, various SRCHs, ...
APLSUB:CATIME.FOR      Stores current, or recovers previous, date and time in packed format
APLSUB:CATIO.FOR       Reads/writes header blocks in the catalog file
APLSUB:CATKEY.FOR      Reads/writes the Keyword section of an AIPS header file
AIPSUB:CATLST.FOR      List the contents of the catalog directory file
APLSUB:CATOPN.FOR      Opens the catalog directory file and returns its size
APLSUB:CHSTAT.FOR      Changes numeric code used to record the status of the catalog entry
APLSUB:CHWMAT.FOR      Matches a pattern string having wild-card chars with a test string
AIPSUB:DESCR.FOR       Destroys all scratch files for tasks which are no longer active
APLSUB:HDRBUF.FOR      Translates AIPS header to/from FITS-standard integer form
APLSUB:ICOPEN.FOR      Opens image catalog for the specified image plane (call from Y only)
APLSUB:IMA2MP.FOR      Converts pixel numbers in a TV-image into real image pixels
APLSUB:MADDEX.FOR      Adds extension file to catalog header
APLSUB:MAKOUT.FOR      Convert input and output names to actual output names in standard way
APLSUB:MAPCLR.FOR      Clears status flags in catalog and deletes lists of files
APLSUB:MAPCLS.FOR      Closes cataloged file, updating header and catalog status if needed
APLSUB:MAPOPN.FOR      Open file pointed to by catalog entry and mark the entry busy
APLSUB:MCREAT.FOR      Create and catalog a map file
APLSUB:MDESTR.FOR      Deletes a catalog entry and all files assocated with it
APLSUB:MP2IMA.FOR      Convert image pixel positions to TV pixel positions
APLSUB:NXTMAP.FOR      Opens next catalog entry matching the input parameters
APLSUB:PSFORM.FOR      Analyses a wild-card string, preparing an array for pattern matching
AIPSUB:RENUMB.FOR      Renumbers an entry in the catalog (CA) file
APLSUB:STXT.FOR        Translates catalog status code into a character string
APLSUB:TKCATL.FOR      Performs operations on the Graphics image catalog
APLSUB:UVCREA.FOR      Create and catalog a uv data base file
\end{verbatim}
 
\subsection{CHARACTER}
\begin{verbatim}
APLSUB:CH2NUM.FOR      converts string containing an integer in ASCII form into the integer
APLSUB:CHBLNK.FOR      returns position of first non-blank character in portion of string
APLSUB:CHCOMP.FOR      compares two HOLLERITH strings
APLSUB:CHCOPY.FOR      moves characters from one HELLERITH string to another
APLSUB:CHFILL.FOR      fills portion of HOLLERITH string with a specified character
APLSUB:CHLTOU.FOR      converts a CHARACTER string to all upper case letters
APLSUB:CHMATC.FOR      searches one HOLLERITH string for the occurrence of another
APLSUB:CHR2H.FOR       converts a Fortran CHARACTER variable to an AIPS HOLLERITH string
APLSUB:CHWMAT.FOR      matches a pattern string having wild-card chars with a test string
APLSUB:FILZCH.FOR      replaces blank characters with
APLSUB:H2CHR.FOR       convert AIPS Hollerith string to Fortran CHARACTER variable
APLSUB:IFPC.FOR        returns the number of HOLLERITH locations needed to hold N characters
APLSUB:ITRIM.FOR       returns length of CHARACTER variable to last non-blank
APLSUB:JTRIM.FOR       clears nulls, returns length of CHARACTER variable to last non-blank
APLSUB:NAMEST.FOR      packs image name in string with leading and trailing blanks removed
APLSUB:PSFORM.FOR      analyses a wild-card string, preparing an array for pattern matching
APLSUB:SPFIL.FOR       fills HOLLERITH string with blanks beginning at first null
APLSUB:STLTOU.FOR      converts any characters beween single quotes to upper case
APLSUB:TRIM.FOR        removes leading and trailing blanks, returns actual length of string
APLSUB:UNPACK.FOR      converts a packed character buffer into one with 1 character/integer
\end{verbatim}
 
\subsection{COORDINATES}
\begin{verbatim}
APLNOT:ATFPNT.FOR      Routine to calculate X-Y coords from galactic coords
AIPSUB:AU7.FOR         verbs to print history, rescale image, alter axis descriptions
APLSUB:AXSTRN.FOR      encodes axis type and value in a string
APLNOT:BDN.FOR         Computes Besselian day numbers of Julian date.
APLSUB:COORDD.FOR      converts angles between degrees and sexagesimal format
APLSUB:COORDT.FOR      translates between celestial, galactic, and eccliptic coordinates
APLNOT:DA13.FOR        Computes arguments A1, A2 and A3 of the mean motion of the sun.
APLNOT:DA46.FOR        Computes arguments A4, A5 and A6 of the mean motion of the moon
APLNOT:DAPM.FOR        Converts apparent to mean positions.
APLSUB:DIRCOS.FOR      determines direction cosines between ref position and test position
APLSUB:DIRDEC.FOR      finds longitude pixel and latitude given latitude pixel and longitude
APLSUB:DIRRA.FOR       finds latitude pixel and longitude given longitude pixel and latitude
APLNOT:DMAP.FOR        Compute apparent position from mean position
APLSUB:FNDX.FOR        returns X-axis coordinate value given X pixel and Y coordinate value
APLSUB:FNDY.FOR        returns Y-axis coordinate value given Y pixel and X coordinate value
APLNOT:GRD.FOR         Compute the general relativity displacements in RA and DEC.
APLSUB:JABER.FOR       Compute vectors needed for J2000 aberation and GR light bending.
APLSUB:JNUT.FOR        Computes nutation from IAU 1980 series
APLSUB:JPOLAR.FOR      Correct rectangular position for polar motion.
APLSUB:JPRECS.FOR      Precess between apparent and J2000 epoch positions.
APLSUB:JPRENU.FOR      Compute rotation matrix for precession and nutation IAU 1980 series.
APLSUB:LABINI.FOR      initializes commons for labeling of plots (calls SETLOC)
APLSUB:LMPIX.FOR       returns pixel location corresponding to specified coordinates
APLSUB:METSCA.FOR      scale a value to the range 1-999 and provide a metric prefix to match
APLSUB:MP2SKY.FOR      calls SETLOC, XYVAL to convert image pixel to physical coordinates
APLSUB:NEWPOS.FOR      returns astronomical coordinates given direction cosines, projection
APLNOT:NUT2.FOR        Computes nutation in longitude and obliquity for a Julian date.
APLNOT:NUT4.FOR        Computes nutation using a non ridgid earth model
APLNOT:PARANG.FOR      Computes antenna parallactic angles
APLNOT:PRECES.FOR      Convert between mean and apparent positions (B1950 only)
APLSUB:SETLOC.FOR      sets location common for coordinate computations and display
APLSUB:SKY2MP.FOR      calls SETLOC, XYPIX to convert sky coordinates to map pixel locations
APLSUB:SKYFRM.FOR      returns string with character representation of a corrdinate
APLSUB:SLAEVP.FOR      Earth position and motion ephemeris (J2000)
APLSUB:SLBINI.FOR      initializes labeling for slice plots
APLNOT:SOUELV.FOR      Computes source hour angles and elevations
APLSUB:XYPIX.FOR       returns pixel position corresponding to given coordinates
APLSUB:XYVAL.FOR       returns coordinate values corresponding to specified pixel position
\end{verbatim}
 
\subsection{EXT-APPL}
\begin{verbatim}
APLSUB:ANTDAT.FOR      Returns the reference date and frequency for each array in uv dataset
APLSUB:ANTINI.FOR      creates and intializes antenna tables
APLNOT:BLINI.FOR       Create/open/init I/O to BL table
APLNOT:BLREFM.FOR      Checks existence of BL table, changes format if necessary
APLNOT:BLSET.FOR       Fills current baseline calibration table
APLNOT:BPASET.FOR      Sets up the bandpass table array for use by DATBND.
APLNOT:BPREFM.FOR      Checks existence of BP table, changes format if necessary
APLNOT:CALADJ.FOR      Adjusts solution (SN) table phases to a common reference antenna.
APLNOT:CALINI.FOR      Creates/opens/initializes calibration (CL) table
APLSUB:CCINI.FOR       creates and/or opens a CC (components) extension table
APLNOT:CCMERG.FOR      Compresses a CLEAN component (CC) table
APLNOT:CLREFM.FOR      Checks existence of CL table, changes format if necessary
APLNOT:CLUPDA.FOR      Concatenates, rereferences, smooths SN tables and applies it to CL.
APLNOT:CSLGET.FOR      Reads CL (or SN) table and sets up for interpolation.
APLSUB:EXTHIS.FOR      adds to history file for contents of FITS extension file being read
APLSUB:EXTREQ.FOR      parse FITS tape record for required extension file FITS keywords
APLNOT:FLAGUP.FOR      Updates the Flag (FG) table.
APLNOT:FNDSOU.FOR      Find source numbers for a list of sources.
APLNOT:GACSIN.FOR      Initializes CS file, and prepares table to be applied.
APLNOT:GAININ.FOR      Initializes calibration table for application.
APLNOT:GETANT.FOR      Reads AN table and stores the info in common.
APLNOT:GETFQ.FOR       Find info on a given frequency id.
APLNOT:GETSOU.FOR      Find info on a given source id.
APLNOT:GNFSMO.FOR      Boxcar smooths and ASCAL solution (GA) file.
APLNOT:GNSMO.FOR       Optimized spline smoothing of amplitudes in ASCAL (GN) file.
APLNOT:GRDAT.FOR       Getn info about CLEAN components for GRDSUB.
QNOT:GRDCRM.FOR        Loads CLEAN components into AP for uv model computation.
APLNOT:INDXIN.FOR      Initializes index (NX) file, finds first scan selected.
APLNOT:ITBSRT.FOR      Read a table and write a scratch file to be sorted.
APLNOT:LXYPOL.FOR      Fills polarization correction table for AT like linear polarization.
APLNOT:MULSDB.FOR      Determines if a uv file is multi- or single- source.
APLNOT:NXTFLG.FOR      Manages flagging info in tables in common.
APLNOT:OTBSRT.FOR      Copies sorted table from scratch file to table form
APLNOT:POLSET.FOR      Fills polarization correction table from info in AN table.
APLNOT:SDCGET.FOR      Sets up to interpolate in Single dish calibration (CS) table.
APLNOT:SELSMG.FOR      Selects calibrator data, smooths solutions.
APLNOT:SETSTK.FOR      Sets STOKES parameters correctly for plotting routines
APLNOT:SN2CL.FOR       Apply an SN to a CL table.
APLNOT:SNAPP.FOR       Append SN tables and keep track of reference antennas.
APLNOT:SNREFM.FOR      Checks existence of SN table, changes format if necessary
APLNOT:SNSMO.FOR       Smooths solution (SN) tables
APLNOT:SOUFIL.FOR      Fills in arrays of source numbers to be included or excluded.
APLNOT:SOURNU.FOR      Look up source numbers for a list of names.
APLNOT:SUMARY.FOR      Accumulates and lists CLEAN components
APLSUB:TABAN.FOR       I/O to antenna tables (following initialization by ANTINI)
APLNOT:TABAXI.FOR      parse FITS tape record for required extension file FITS keywords
APLSUB:TABLIN.FOR      reads a line from the data portion of a FITS extension of type TABLE
APLNOT:TYINI.FOR       Create/open/initialize Tsys (TY) table
APLNOT:VISCNT.FOR      Determines number of visibility records requested of UVGET
\end{verbatim}
 
\subsection{EXT-UTIL}
\begin{verbatim}
APLSUB:ALLTAB.FOR      Copies all table extension files from one catalog slot to another
AIPSUB:AU8.FOR         verbs to get or clear name adverbs, destroy extension files
APLNOT:BPINI.FOR       Create/open/initialize bandpass (BP) table
APLNOT:CHNCOP.FOR      Copies selected portions of the IF table
APLNOT:CHNDAT.FOR      Creates/Opens/Reads/Writes/Closes an IF table.
APLNOT:CSINI.FOR       Create/Open/Init Single dish calibration (CS) table
APLSUB:DELEXT.FOR      removes an extension file from the header in the catalog file
APLSUB:EXTCOP.FOR      copies extension file of the EXTINI/EXTIO variety
APLSUB:EXTINI.FOR      creates and/or opens an extension file of the EXTINI/EXTIO type
APLSUB:EXTIO.FOR       does random access IO to extension files of the EXTINI/EXTIO type
APLNOT:FLGINI.FOR      Create/Open/Init Flag (FG) table.
APLSUB:FNDCOL.FOR      locvates logical column numbers for given titles in a Table
APLSUB:FNDEXT.FOR      returns latest version number of specified extension file type
APLNOT:FQINI.FOR       Create/open/initialize frequency (FQ) table
APLNOT:GAINI.FOR       Creates and initializes gain (GA) extension tables.
APLSUB:GETCOL.FOR      returns value and type found at specified column and row in a table
APLSUB:GETHUT.FOR      returns column titles, units, types, lengths in logical column order
APLNOT:GETNAN.FOR      Find number of antennas and subarrays from AN tables.
APLNOT:GTPAIR.FOR      Returns specified Keyword-value pair from an open AIPS table
APLSUB:ISTAB.FOR       finds if an extension file exists and whether it is a standard table
APLSUB:MADDEX.FOR      adds extension file to catalog header
APLNOT:MAKTAB.FOR      Create and initialize table from data in common /TABHDR/ (FITS)
APLNOT:NDXINI.FOR      Create/open/init index (NX) table
APLSUB:OPEXT.FOR       opens a specified extension file
APLSUB:PUTCOL.FOR      returns value and type found at specified column and row in a table
APLNOT:R3DTAB.FOR      Read data from FITS 3-D table and write AIPS table.
APLSUB:RESCSL.FOR      Rescale flux-like data in any SLice files.
APLNOT:RWTAB.FOR       Read FITS ASCII table data and write AIPS table file.
APLNOT:SDTCRD.FOR      Parse "SINGLDSH" FITS table headers, get some keywords.
APLSUB:SELSTR.FOR      builds string displaying the functions applied to columns of table
APLNOT:SNINI.FOR       Create/open/initialize solution (SN) tables.
APLNOT:SOUINI.FOR      Create/initialize/open source (SU) table
APLNOT:TABAPP.FOR      Appends one table to the end of a similar table.
APLNOT:TABBL.FOR       Do IO to Baseline (BL) table after setup by BLINI.
APLNOT:TABBP.FOR       Does I/O to bandpass (BP) table opened by BPINI
APLNOT:TABCAL.FOR      Does I/O to Calibration (CL) table opened by CALINI
APLSUB:TABCOP.FOR      copies one or all tables extension files of specified type
APLNOT:TABCS.FOR       Does I/O to single dish calibration (CS) table opened by CSINI
APLNOT:TABF3D.FOR      Determines repeat count and data type for FITS 3-D tables entries.
APLNOT:TABFLG.FOR      Does I/O to Flag (FG) table opened by FLGINI
APLNOT:TABFQ.FOR       Does I/O to frequency (FQ) table opened by FQINI
APLNOT:TABFRM.FOR      Parses format for FITS ASCI table entries.
APLNOT:TABGA.FOR       Does I/O to GAIN (GA) table opened by GAINI
APLNOT:TABHDK.FOR      Reads a FITS table header.
APLNOT:TABHDR.FOR      Reads a FITS table header.
APLSUB:TABINI.FOR      create/open a table extension file
APLSUB:TABIO.FOR       reads/writes tables extension files
APLSUB:TABKEY.FOR      reads/writes the Keyword section of an AIPS table file
APLSUB:TABMRG.FOR      merges rows of an an input table file
APLNOT:TABNDX.FOR      Does I/O to Index (NX) table opened by NDXINI
APLNOT:TABSN.FOR       Does I/O to Solution (SN) table opened by SNINI
APLNOT:TABSOU.FOR      Does I/O to Source (SU) table opened by SOUINI
APLNOT:TABSPC.FOR      Determines repeat count and data type for FITS 3-D tables entries.
APLNOT:TABSRT.FOR      Sorts the entries in an AIPS table.
APLNOT:TABTY.FOR       Does I/O to Tsys (TY) table opened by TYINI
\end{verbatim}
 
\subsection{FITS}
\begin{verbatim}
APLNOT:ATCONV.FOR      Fix AIPS FITS tables
APLSUB:CHAVRT.FOR      converts between local HOLL and local INT binary forms for transport
APLNOT:CHKTAB.FOR      Check fields of known FITS table types.
APLSUB:EXTHIS.FOR      adds to history file for contents of FITS extension file being read
APLSUB:EXTREQ.FOR      parse FITS tape record for required extension file FITS keywords
APLSUB:FPARSE.FOR      interprets card image from FITS header into AIPS header format
AIPSUB:FWRITE.FOR      converts FITS header to AIPS header and displays it with MSGWRT
APLSUB:GETCRD.FOR      parses card image from FITS header, returns recognized keyword
APLSUB:GETLOG.FOR      returns value of logical variable from character buffer
APLSUB:GETNUM.FOR      returns numeric field from character buffer
APLSUB:GETSTR.FOR      returns a string value (was enclosed by quotes) from character buffer
APLSUB:GETSYM.FOR      returns next symbol in character-form card image
APLSUB:GTWCRD.FOR      returns allowed keyword from FITS header card image
APLSUB:IDWCRD.FOR      returns allowed keyword from FITS header card image
APLSUB:JULDAY.FOR      converts a character-encoded calendar date to Julian day number
APLNOT:MAKTAB.FOR      Create and initialize table from data in common /TABHDR/ (FITS)
AIPSUB:MSGHDR.FOR      lists header contents for standard header plus random parameters
APLNOT:PTF3D.FOR       Copies 8-bit bytes to tape.
APLNOT:R3DTAB.FOR      Read data from FITS 3-D table and write AIPS table.
APLSUB:REAVRT.FOR      converts between local REAL and local INT binary forms for transport
APLNOT:RWTAB.FOR       Read FITS ASCII table data and write AIPS table file.
APLNOT:SDTCRD.FOR      Parse "SINGLDSH" FITS table headers, get some keywords.
APLSUB:SETBSC.FOR      determines scaling/offset parameters to convert image to integer
APLSUB:SETDEF.FOR      fills FITS reader area for table-file extensions with defaults
APLSUB:SKPBLK.FOR      find next non-blank card image in a FITS header, read tape if needed
APLSUB:SKPEXT.FOR      finishes reading FITS extension header, skips the extension data
APLNOT:TABAXI.FOR      parse FITS tape record for required extension file FITS keywords
APLNOT:TABF3D.FOR      Determines repeat count and data type for FITS 3-D tables entries.
APLNOT:TABFRM.FOR      Parses format for FITS ASCI table entries.
APLNOT:TABHDK.FOR      Reads a FITS table header.
APLNOT:TABHDR.FOR      Reads a FITS table header.
APLSUB:TABLIN.FOR      reads a line from the data portion of a FITS extension of type TABLE
APLNOT:TABSPC.FOR      Determines repeat count and data type for FITS 3-D tables entries.
APLNOT:TPIOHD.FOR      Reads tape header and tests if FITS, tape labels etc.
APLGEN:ZBYTF2.FOR      interchange bytes in buffer if needed to go between local & standard
APLGEN:ZBYTFL.FOR      interchange bytes in buffer if needed to go between local & standard
APLGEN:ZTPMID.FOR      pseudo-tape disk read/write for 2880-bytes records
APLGEN:ZTPOPD.FOR      open a pseudo-tape, sequential disk file for FITS
APLGEN:ZTPWAD.FOR      "wait" for IO operation to complete on pseudo-tape disk file (ZTPMID)
APLGEN:ZX8XL.FOR       convert FITS table bit array to AIPS bit array
APLGEN:ZXLX8.FOR       convert AIPS bit array to FITS binary table bit array
\end{verbatim}
 
\subsection{GRAPHICS}
\begin{verbatim}
AIPSUB:AU9A.FOR        verbs to read TEK cursor and display pixel, sky, image values
AIPSUB:AU9B.FOR        verbs to plot slices and models on graphics
AIPSUB:AU9C.FOR        verbs to set initial guesses for slice model fits using TEK graphics
AIPSUB:SET1DG.FOR      sets initial guess parameters with the TEK for fitting slices
APLNOT:SETSTK.FOR      Sets STOKES parameters correctly for plotting routines
AIPSUB:SLOCIN.FOR      initialize location common for slice (on the TEK) model fitting
APLSUB:TEKFLS.FOR      writes any remaining buffer to the TK graphics device, zeros buffer
APLSUB:TEKVEC.FOR      write bright or dark, scaled or unscaled vector to TK graphics device
APLSUB:TKCATL.FOR      performs operations on the Graphics image catalog
APLSUB:TKCHAR.FOR      writes characters to a TK graphics device
APLSUB:TKCLR.FOR       clears the TK graphics screen
APLSUB:TKCURS.FOR      turns on, reads, turns off the TK graphics cursor
APLSUB:TKDVEC.FOR      converts vector command to TK graphics commands
AIPSUB:TKGGPL.FOR      plots model slice on Graphics
AIPSUB:TKGMPL.FOR      plot model fit to slice on the graphics terminal
APLSUB:TKLAB.FOR       labels axes on plot directly to a TK graphics device, draw ticks
AIPSUB:TKRSPL.FOR      plots residuals between slice and its model on Graphics device
AIPSUB:TKSLAC.FOR      activates and reads TEK cursor, converts result to image coordinates
APLSUB:TKSLIN.FOR      initialize parameters for plotting a slice directly on a TK graphics
AIPSUB:TKSLPL.FOR      plot a slice on graphics device
APLSUB:TKTICS.FOR      writes tick marks and labels directly to TK graphics device
APLGEN:ZTKBUF.FOR      flush TK buffer if needed, then store 8-bit byte in buffer
APLGEN:ZTKCL2.FOR      close a Tektronix device
APLGEN:ZTKCLS.FOR      close the TK device
APLGEN:ZTKFI2.FOR      read/write from/to a Tektronix device
APLGEN:ZTKOP2.FOR      read/write from/to a Tektronix device
APLGEN:ZTKOPN.FOR      open a TK device
\end{verbatim}
 
 
\subsection{HEADER}
\begin{verbatim}
AIPSUB:AU3.FOR         verbs to display contents of catalogs and headers: CATA, IMHE ...
AIPSUB:AU7.FOR         verbs to print history, rescale image, alter axis descriptions
AIPSUB:AU7A.FOR        verbs to put/get header values, to put values into images
APLSUB:AXEFND.FOR      finds axis number for specified axis type
APLSUB:BLDSNM.FOR      builds a name for a scratch file
APLSUB:CATIO.FOR       reads/writes header blocks in the catalog file
APLSUB:CATKEY.FOR      reads/writes the Keyword section of an AIPS header file
APLNOT:COINC.FOR       Checks if two maps are exactly coincident.
APLNOT:DGHEAD.FOR      Fills output CATBLK for UVGET
APLNOT:FRQTAB.FOR      Fill Frequency table in common for IFs and channels
APLNOT:GETCTL.FOR      Determine Stokes'  type of Clean map and other modeling info.
APLNOT:IMCREA.FOR      Fills catalog header for an image and optionally creates and catalogs
APLSUB:JULDAY.FOR      converts a character-encoded calendar date to Julian day number
AIPSUB:KWIKHD.FOR      list header contents in abbreviated, image centered form
APLSUB:LMPIX.FOR       returns pixel location corresponding to specified coordinates
APLSUB:LSTHDR.FOR      lists header contents in standard form with MSGWRT
AIPSUB:MSGHDR.FOR      lists header contents for standard header plus random parameters
APLSUB:NAMEST.FOR      packs image name in string with leading and trailing blanks removed
APLSUB:ROTFND.FOR      find the coordinate rotation angle from the catalog header
APLSUB:SUBHDR.FOR      changes input to output header correcting for subimaging
APLSUB:SWAPAX.FOR      swaps the values for two axes
APLSUB:UVPGET.FOR      determines pointers to UV data from the header
APLSUB:VHDRIN.FOR      computes pointers (subscripts) to address components of the header
\end{verbatim}
 
\subsection{HISTORY}
\begin{verbatim}
AIPSUB:AU7.FOR         verbs to print history, rescale image, alter axis descriptions
APLSUB:EXTHIS.FOR      adds to history file for contents of FITS extension file being read
APLSUB:HENCO1.FOR      Adds INNAME, INCLASS, INSEQ, INDISK to an open history file
APLSUB:HENCO2.FOR      Adds IN2NAME, IN2CLASS, IN2SEQ, IN2DISK to an open history file
APLSUB:HENCO3.FOR      Adds IN3NAME, IN3CLASS, IN3SEQ, IN3DISK to an open history file
APLSUB:HENCOO.FOR      adds OUTNAME, OUTCLASS, OUTSEQ, OUTDISK to an open history file
APLSUB:HIAD80.FOR      puts an 80-character card image into a history file as required
APLSUB:HIADD.FOR       adds a history record ("card" = 72 characters) to a history file
APLSUB:HIADDN.FOR      Writes one history line to several history files
APLSUB:HICLOS.FOR      closes a history file, flushing the buffer if desired
APLSUB:HICOPY.FOR      copies one history file to the end of a second
APLSUB:HICREA.FOR      open a history file, creating one if needed
APLSUB:HIINIT.FOR      initializes the history common area - must be called before history
APLSUB:HIIO.FOR        does IO and file expansion (if needed) on HI files
APLSUB:HILOCT.FOR      manipulates the history table, opening, closing, located an entry
APLSUB:HIMERG.FOR      creates several new history files by merging several old ones.
APLSUB:HIOPEN.FOR      opens a history file, preparing common pointers and reading record 1
APLSUB:HIPLOT.FOR      places a record in the history file concerning a plot file creation
APLNOT:HIREAD.FOR      Reads next history card from a history file
APLSUB:HISCOP.FOR      creates new history file and copies an old one to it
\end{verbatim}
 
\subsection{IO-APPL}
\begin{verbatim}
APLNOT:FQMATC.FOR      Check if selection criteria match FQ table entries.
APLNOT:MAKGAU.FOR      *TESS routine: Make a Gaussian convolution function.
APLNOT:MLTMAP.FOR      *TESS routine: multiplies an image by a value, writes another.
APLNOT:PLNPUT.FOR      Copies a subregion of a scratch file image to a cataloged image.
APLNOT:RESID.FOR       *TESS routine: Computes residual image.
APLNOT:SDGET.FOR       Reads single dish data with optional calibration and flagging
APLNOT:SDTCRD.FOR      Parse "SINGLDSH" FITS table headers, get some keywords.
APLNOT:STEP.FOR        *TESS routine: adds a fraction of one image to another.
APLNOT:SUBMAP.FOR      *TESS Routine: Subtract two images.
APLNOT:TVFOAD.FOR      TVFLG routine to load and image with smoothing converting to display.
APLNOT:UVDOUT.FOR      Divides uv model in one half of a record into other, writes result.
APLNOT:UVDPAD.FOR      Reformat UV data record, doubling size and zero extra words.
APLNOT:UVGET.FOR       Read UV data with optional calibration, editing, selection, etc.
\end{verbatim}
 
\subsection{IO-BASIC}
\begin{verbatim}
APLSUB:FSERCH.FOR      determines type and number of entries in the common file table (FTAB)
APLSUB:IAMOK.FOR       decides if a disk file type is allowed for the user on a disk
APLSUB:LSERCH.FOR      opens, locates, closes entries in the common file table (FTAB)
APLSUB:MDISK.FOR       reads or writes a row from an image
APLSUB:MINIT.FOR       intializes IO and pointers for quick-return image IO via MDISK
APLSUB:UVDISK.FOR      reads/writes records of arbitrary length, esp UV data, see UVINIT
APLSUB:UVINIT.FOR      initializes IO for arbitrary length records via UVDISK, esp UV data
APLGEN:ZCLOSE.FOR      closes open devices: disk, line printer, terminal
APLGEN:ZCMPR2.FOR      truncate a disk file, returning blocks to the system
APLGEN:ZCMPRS.FOR      release space from the end of an open disk file
APLGEN:ZCREA2.FOR      create the specified disk file
APLGEN:ZCREAT.FOR      creates a disk file
APLGEN:ZDACLS.FOR      close a disk file
APLGEN:ZDAOPN.FOR      open the specified disk file
APLGEN:ZDEST2.FOR      destroy a closed disk file
APLGEN:ZDESTR.FOR      destroy a closed disk file
APLGEN:ZEXIS2.FOR      return size of disk file and if  it exists
APLGEN:ZEXIST.FOR      return file size and, consequently, whether file exists
APLGEN:ZEXPN2.FOR      expand an open disk file
APLGEN:ZEXPND.FOR      expand an open disk file --- either map or non-map now allowed
APLGEN:ZFI2.FOR        read/write one 256-integer record from/to a non-map disk file
APLGEN:ZFIO.FOR        reads and writes single 256-integer records to non-map disk files
APLGEN:ZFRE2.FOR       return AIPS data disk free space information
APLGEN:ZMI2.FOR        read/write large blocks of data from/to disk, quick return
APLGEN:ZMIO.FOR        random-access, quick return (double buffer) disk IO for large blocks
APLGEN:ZMKTMP.FOR      convert a "temporary" file name into a unique name
APLGEN:ZMSGCL.FOR      close Message file or terminal
APLGEN:ZMSGDK.FOR      disk IO to message file
APLGEN:ZMSGOP.FOR      open a message file or message terminal
APLGEN:ZMSGXP.FOR      expand the message file
APLGEN:ZOPEN.FOR       open binary disk files and line printer and TTY devices
APLGEN:ZPATH.FOR       convert a file name
APLGEN:ZPHFIL.FOR      construct a physical file or device name from AIPS logical parameters
APLGEN:ZPHOLV.FOR      construct a physical file - version for UPDAT
APLGEN:ZRENA2.FOR      rename a file
APLGEN:ZTFILL.FOR      zero-fill, initialize a file IO table (FTAB) entry
APLGEN:ZTPOP2.FOR      open a tape device for double-buffer, asymchronous IO
APLGEN:ZTPWA2.FOR      wait for read/write from/to a tape device
APLGEN:ZWAI2.FOR       wait for read/write large blocks of data from/to disk
APLGEN:ZWAIT.FOR       wait for asynchronous ("MAP") IO to finish
\end{verbatim}
 
\subsection{IO-TV}
\begin{verbatim}
APLGEN:ZARGC2.FOR      close an ARGS TV device
APLGEN:ZARGCL.FOR      close an ARGS TV device
APLGEN:ZARGMC.FOR      issues a master clear to an ARGS TV
APLGEN:ZARGO2.FOR      open ARGS TV device
APLGEN:ZARGOP.FOR      open ARGS TV device
APLGEN:ZIPACK.FOR      pack/unpack long integers into short integer buffer
APLGEN:ZM70CL.FOR      close an IIS Model 70 TV device, flushing any buffer
APLGEN:ZM70M2.FOR      issues a master clear to an IIS Model 70 TV
APLGEN:ZM70MC.FOR      issues a master clear to an IIS Model 70 TV
APLGEN:ZM70OP.FOR      open IIS Model 70 TV device
APLGEN:ZM70XF.FOR      read/write data to IIS Model 70 TV with buffering
APLGEN:ZV20CL.FOR      close a Comtal Vision 1/20 TV device
APLGEN:ZV20OP.FOR      open Comtal Vision 1/20 TV device
APLGEN:ZV20XF.FOR      read/write data to Comtal Vision 1/20 TV device
\end{verbatim}
 
\subsection{IO-UTIL}
\begin{verbatim}
APLNOT:AKCESS.FOR      *TESS routine to read or write files
APLNOT:AKCLOS.FOR      *TESS routine to close files
APLNOT:AKOPEN.FOR      *TESS I/O routine to open files.
QNOT:APIO.FOR          Copies image-like data between disk and "AP memory".
APLNOT:APPLPB.FOR      *TESS routine to apply a taper to an image
QSUB:APROLL.FOR        Copies AP "memory" to disk, gives up AP then reloads AP
APLNOT:BGTOSM.FOR      *TESS routine to copy a subset of a large image to a small one
APLSUB:COMOFF.FOR      determines start block number of a plane in an N-dimensional image
APLNOT:COPMAP.FOR      *TESS routine to copy an image.
APLSUB:DBINIT.FOR      checks map window and initializes for map double buffer IO
APLSUB:DIE.FOR         closes down tasks which use DFIL.INC to maintain status of files
APLNOT:DIVMAP.FOR      Tim Corwell routine: Divide one image by another.
APLNOT:FILSWP.FOR      *TESS routine: switch file info
APLNOT:FLAT.FOR        *TESS routine: initialize an image to a value.
APLSUB:FSWTCH.FOR      switches names and addresses of two files
APLNOT:GETROW.FOR      Read row of an image opened with INTMIO
APLNOT:GTBWRT.FOR      Routine used by GRIDTB to write buffers.
APLNOT:GTF3D.FOR       Copies real-world bytes from a tape buffer, reading if necessary.
APLNOT:HIREAD.FOR      Reads next history card from a history file
APLNOT:INTMIO.FOR      Open an image file for use with GETROW
APLNOT:LINIO.FOR       Reads/writes line to/from an image.
APLNOT:MAKCVM.FOR      *TESS routine: Make image with residuals added.
APLSUB:MAPCLS.FOR      closes cataloged file, updating header and catalog status if needed
APLSUB:MAPOPN.FOR      open file pointed to by catalog entry and mark the entry busy
APLSUB:MDESTR.FOR      deletes a catalog entry and all files assocated with it
APLNOT:REIMIO.FOR      Reinitialize for image I/O using INTMIO
APLNOT:SCINTP.FOR      Interpolates bandpass tables in time.
APLNOT:SCLOAD.FOR      Copies part of one bandpass scratch file to another for efficiency.
APLSUB:SCREAT.FOR      create an AIPS-standard scratch file w common DFIL.INC, ...
APLNOT:SMTOBG.FOR      *TESS routine: Copies small image to a large one.
APLSUB:SNDY.FOR        closes all files, then deletes all scratch files
APLNOT:TABSRT.FOR      Sorts the entries in an AIPS table.
APLNOT:TPIOHD.FOR      Reads tape header and tests if FITS, tape labels etc.
APLNOT:VECWIN.FOR      Interpretes BLC and TRC into useable values as a vector.
APLGEN:ZDIR.FOR        build a full path name to files in AIPS-standard areas (HE, RU, ...)
APLGEN:ZFULLN.FOR      convert file name to full pathname with no logicals
APLGEN:ZRENAM.FOR      rename a disk file
\end{verbatim}
 
\subsection{IO-WAWA}
\begin{verbatim}
APLSUB:A2WAWA.FOR      packs WaWa IO NameString from its components
APLSUB:CLENUP.FOR      closes all open files and deletes all scratch files for this task
APLSUB:FILCLS.FOR      close file opened by FILOPN, flushing write buffers, clearing catalog
APLSUB:FILCR.FOR       create associated or scratch non-map file
APLNOT:FILDEF.FOR      Fills in default values in WAWA namestring
APLSUB:FILDES.FOR      destroy the specified file or associated file
APLSUB:FILIO.FOR       reads/writes 256-integer record to non-map file opened by FILOPN
APLSUB:FILNUM.FOR      finds the FILTABle entry number for an open file
APLSUB:FILOPN.FOR      open image, associated, or scratch file (WaWa system)
APLSUB:GETHDR.FOR      get the catalog header for an open file (WaWa)
APLNOT:GETWIN.FOR      Get current window of file open in WAWA IO system.
APLNOT:GTNAME.FOR      WAWA IO routine to fill in a namestring for an open file
APLSUB:H2WAWA.FOR      packs AIPS adverb values into WaWa IO NameString
APLSUB:HDRINF.FOR      returns consecutivew items of specified type from header for WaWa
APLSUB:HDRWIN.FOR      sets image corners via WINDOW, revises header to that of output
APLNOT:IMOPEN.FOR      Open the TV under the system set up by IOSET
APLNOT:IMWIN.FOR       Set up window on TV device
APLSUB:IOSET.FOR       initialize tables and set buffer space for WaWa IO
APLNOT:MADD.FOR        Routine to add windows of open images.
APLNOT:MAKNAM.FOR      Constructs WAWA namestring (Now use H2WAWA or A2WAWA)
APLSUB:MAPCOP.FOR      Copy a map
APLSUB:MAPCR.FOR       create and catalog an image in the WaWa package
APLSUB:MAPIO.FOR       reads or writes a file opened by FILOPN (WaWa IO)
APLSUB:MAPMAX.FOR      determine extrema of image opened by FILOPN and update header
APLSUB:MAPWIN.FOR      set/reset the window parameters for an open file (in WaWa)
APLSUB:MAPXY.FOR       sets WaWa windows for a window in the top plane of an image
APLNOT:MCOPY.FOR       Copies a window in one image to another.
APLNOT:MFILL.FOR       Fill a window in an image with a given value.
APLSUB:OPENCF.FOR      opens a cataloged file (main file only), simplifies call to FILOPN
APLSUB:PRENAM.FOR      checks name-string for WaWa IO package - fills in some defaults
APLNOT:PRTERR.FOR      Prints standard WaWa error message and namestring of file.
APLSUB:PRTNAM.FOR      prints the contents of a WaWa-IO file Namestring
APLNOT:SAVHDR.FOR      Save catalog header for an open file.
APLSUB:SCRNAM.FOR      build scratch file name string in the WaWa form
APLSUB:TSKBEG.FOR      task start up operations (common inits, GTPARM, RELPOP) for WaWa
APLSUB:TSKEND.FOR      closes down a task and its files in the WaWa system
APLSUB:UNSCR.FOR       delete all scratch files belonging to this tasl
APLSUB:WAWA2A.FOR      unpacks WaWa IO NameString into its components
\end{verbatim}
 
\subsection{MAP}
\begin{verbatim}
QNOT:APCONV.FOR        Disk based 2-D convolution using FFTs.
QNOT:APIO.FOR          Copies image-like data between disk and "AP memory".
AIPSUB:AU9.FOR         verbs to fit or interpolate the image intensity (MAXFIT, IMVAL)
APLNOT:BMSHP.FOR       *TESS routine to fit an elliptical Gaussian to a dirty beam
QNOT:CCSGRD.FOR        Transforms CLEAN components to a grid.
APLNOT:COINC.FOR       Checks if two maps are exactly coincident.
APLNOT:COMCLR.FOR      Scale and map complex array into RBG space, amp=inten. phase=hue.
QNOT:CONV.FOR          *TESS routine: Convolve a map with a beam.
QNOT:CONV1.FOR         First of four routines to convolve two real images.
QNOT:CONV2.FOR         Second of four routines to convolve two real images.
QNOT:CONV3.FOR         Third of four routines to convolve two real images.
QNOT:CONV4.FOR         Fourth of four routines to convolve two real images.
APLNOT:COPMAP.FOR      *TESS routine to copy an image.
QNOT:DISPTV.FOR        *TESS routine: Display an image on a TV
APLNOT:DSKFFT.FOR      2-D disk based FFT using AP.
QNOT:FFTIM.FOR         FFTs an image for uv interpolation.
APLNOT:GETCTL.FOR      Determine Stokes' type of Clean map and other modeling info.
APLNOT:GETROW.FOR      Read row of an image opened with INTMIO
QNOT:GRDCOR.FOR        Normalizes and corrects image for gridding convolution fn.
QNOT:GRDCRM.FOR        Loads CLEAN components into AP for uv model computation.
APLNOT:GRDFLT.FOR      Sets default gridding convolution functions.
QNOT:GRDSUB.FOR        Subtracts transform of CLEAN components from uv data.
QNOT:GRDTAB.FOR        Computes Fourier transform of gridding convolution function.
APLNOT:GRIDTB.FOR      Makes a gridded image of the UV data in TB order.
APLNOT:IMCREA.FOR      Fills catalog header for an image and optionally creates and catalogs
APLNOT:INTMIO.FOR      Open an image file for use with GETROW
QNOT:INTPFN.FOR        Computes interpolation kernals and put them into "AP memory".
APLNOT:LINIO.FOR       Reads/writes line to/from an image.
APLNOT:MADD.FOR        Routine to add windows of open images.
APLNOT:MAKCVM.FOR      *TESS routine: Make image with residuals added.
APLNOT:MAKGAU.FOR      *TESS routine: Make a Gaussian convolution function.
QNOT:MAKMAP.FOR        Makes image or beam from uv data set.
APLNOT:MCOPY.FOR       Copies a window in one image to another.
APLNOT:MFILL.FOR       Fill a window in an image with a given value.
APLNOT:MLTMAP.FOR      *TESS routine: multiplies an image by a value, writes another.
QSUB:PASS1.FOR         First of two routines to FFT an image file.
QSUB:PASS2.FOR         Second of two routines to FFT an image file.
APLNOT:PLNPUT.FOR      Copies a subregion of a scratch file image to a cataloged image.
APLNOT:REIMIO.FOR      Reinitialize for image I/O using INTMIO
APLNOT:RESID.FOR       *TESS routine: Computes residual image.
APLNOT:SAVHDR.FOR      Save catalog header for an open file.
APLNOT:SMTOBG.FOR      *TESS routine: Copies small image to a large one.
APLNOT:SUBMAP.FOR      *TESS Routine: Subtract two images.
APLNOT:SUMARY.FOR      Accumulates and lists CLEAN components
QNOT:UVMDIV.FOR        Divides a uv data set by the Fourier transform of a model.
QNOT:UVMSUB.FOR        Subtracts the Fourier transform of a model from a uv data set.
APLNOT:VECWIN.FOR      Interpretes BLC and TRC into useable values as a vector.
QNOT:VISDFT.FOR        Compute DFT of model and subtract/divide from/into uv data.
APLNOT:VMBLKD.FOR      *TESS Routine: Initialize constants in common.
APLNOT:VTTELL.FOR      *TESS Routine: checks TELL file.
\end{verbatim}
 
\subsection{MAP-UTIL}
\begin{verbatim}
APLNOT:ADDMAP.FOR      *TESS routine to add images
APLNOT:APLPBI.FOR      *TESS routine to apply a taper to an image. VLA only!
APLSUB:BLTGLE.FOR      returns angle from A through a test position to B
APLSUB:BLTLIS.FOR      lists any segments of current row which fall inside blotch regions
APLSUB:COMOFF.FOR      determines start block number of a plane in an N-dimensional image
AIPSUB:CUBINT.FOR      does 2-dimensional cubic interpolation of array values to position
APLSUB:DBINIT.FOR      checks map window and initializes for map double buffer IO
APLSUB:HDRWIN.FOR      sets image corners via WINDOW, revises header to that of output
APLSUB:MAPSIZ.FOR      returns the file size needed to hold the specified image in AIPS
APLSUB:MAPSNC.FOR      creates a scratch image file of specified dimensionality
APLSUB:MCREAT.FOR      create and catalog a map file
APLSUB:MDISK.FOR       reads or writes a row from an image
APLSUB:MINIT.FOR       intializes IO and pointers for quick-return image IO via MDISK
APLSUB:MINSK.FOR       inits use of MSKIP to read noncontiguous, evenly spaced rows in a map
APLSUB:MSKIP.FOR       reads noncontiguous, but evenly spaced rows in a map (see also MINSK)
APLSUB:PEAKFN.FOR      returns location of maximum within 5 pixels of image plane center
APLSUB:PLNGET.FOR      reads subimage of a plane and writes it to scratch file with shifts
APLSUB:RESCAL.FOR      Scales and offsets a cataloged image, updates CATBLK
APLSUB:SETBSC.FOR      determines scaling/offset parameters to convert image to integer
APLSUB:SNRVAL.FOR      substitutes specified value for magic blank value in a buffer
APLSUB:SUBHDR.FOR      changes input to output header correcting for subimaging
APLSUB:WINDOW.FOR      translates user BLC, TRC parameters into usable window arrays
APLSUB:WRBLNK.FOR      write blanked pixels at all pixels corresponding to specified pixel
APLSUB:WRPLAN.FOR      copies an N dimensional plane to a N or N+1 dimensional image
\end{verbatim}
 
\subsection{MATH}
\begin{verbatim}
APLNOT:APPLPB.FOR      *TESS routine to apply a taper to an image
APLNOT:BOXBSM.FOR      Box car smoothing of an irregularly spaced array with blanking
APLNOT:BOXSMO.FOR      Does boxcar smoothing of an irregularly spaced array.
APLNOT:BSC.FOR         Computes Besselian star constants
APLNOT:CALRES.FOR      *TESS routine to calculate the residuals of an image.
APLNOT:CAXPY.FOR       Linpack routine: Complex constant times a vector plus a vector
APLNOT:CD.FOR          Computes Besselian day numbers C and D for aberration
APLNOT:CGEDI.FOR       Linpack routine: Determinant and inverse of a complex matrix
APLNOT:CGEFA.FOR       Linpack routine: Factors complex matrix by Gaussian Elimination
APLNOT:CLD.FOR         Converts Julian date to civil date
QNOT:CONV.FOR          *TESS routine: Convolve a map with a beam.
QNOT:CONV1.FOR         First of four routines to convolve two real images.
QNOT:CONV2.FOR         Second of four routines to convolve two real images.
QNOT:CONV3.FOR         Third of four routines to convolve two real images.
QNOT:CONV4.FOR         Fourth of four routines to convolve two real images.
QNOT:CONVFN.FOR        Computes convolving fn. kernels and stores them in "AP memory"
APLSUB:COVAR.FOR       Determines the covariance matrix of an M x N matrix
APLNOT:CSCAL.FOR       Linpack routine: Complex constant times vector
APLNOT:CSWAP.FOR       Linpack routine: Swaps two complex vectors
AIPSUB:CUBINT.FOR      does 2-dimensional cubic interpolation of array values to position
APLNOT:DA13.FOR        Computes arguments A1, A2 and A3 of the mean motion of the sun.
APLNOT:DA46.FOR        Computes arguments A4, A5 and A6 of the mean motion of the moon
APLNOT:DAPM.FOR        Converts apparent to mean positions.
APLNOT:DCUV.FOR        Computes unit vector for a given celestial position.
APLNOT:DDOT.FOR        Linpack routine: Form dot product of two vectors (DOUBLE)
APLNOT:DERF.FOR        Double precision erf function
APLNOT:DIVMAP.FOR      Tim Corwell routine: Divide one image by another.
APLNOT:DMACH.FOR       Linpack? routine: Sets machine precision parameters. (DOUBLE)
APLNOT:DMAP.FOR        Compute apparent position from mean position
APLNOT:DNRM2.FOR       Compute Euclidean norm of N-Vector
APLSUB:DPMPAR.FOR      returns machine precision or smallest or largest magnitude
APLNOT:DPRE.FOR        Compute General precession matrix
APLNOT:DTRC.FOR        Transforms spherical coordinates given transform matrix.
APLNOT:DUVC.FOR        Converts unit vector to celestial coordinates.
APLNOT:DVDMIN.FOR      Davidon
APLSUB:ENORM.FOR       computes the Euclidean norm of a N-vector
APLNOT:EPS.FOR         Computes mean obliquity of the Ecliptic for a Julian date.
APLNOT:ERF.FOR         error function.
APLNOT:FNDVAR.FOR      *TESS routine: Convert errors in Jy/beam to Jy per cell
APLNOT:FOURG.FOR       Cooley-Tukey fast fourier transform.
APLNOT:FOURYF.FOR      Fast Fourier transform by W. Newman - vectorizes.
APLNOT:GNFSMO.FOR      Boxcar smooths and ASCAL solution (GA) file.
APLNOT:GNSMO.FOR       Optimized spline smoothing of amplitudes in ASCAL (GN) file.
APLNOT:GRD.FOR         Compute the general relativity displacements in RA and DEC.
QNOT:GRDCOR.FOR        Normalizes and corrects image for gridding convolution fn.
APLNOT:GRDFLT.FOR      Sets default gridding convolution functions.
QNOT:GRDTAB.FOR        Computes Fourier transform of gridding convolution function.
APLNOT:GSTROT.FOR      Computes GST at UT=0 and earth rotation rate.
APLNOT:ICAMAX.FOR      Linpack routine: Index of complex element with max. abs. value
APLNOT:ICSORT.FOR      Two key in memory sort by one of several methods
APLSUB:JABER.FOR       Compute vectors needed for J2000 aberation and GR light bending.
APLSUB:JNUT.FOR        Computes nutation from IAU 1980 series
APLSUB:JPOLAR.FOR      Correct rectangular position for polar motion.
APLSUB:JPRECS.FOR      Precess between apparent and J2000 epoch positions.
APLSUB:JPRENU.FOR      Compute rotation matrix for precession and nutation IAU 1980 series.
APLNOT:L1.FOR          Compute L1 solution to an overdetermined system of linear equations
APLNOT:LG2BIT.FOR      Converts between bit arrays and logical arrays
APLSUB:LMDER.FOR       minimize the sum of squares of M nonlinear functions in N variables
APLSUB:LMDER1.FOR      minimize the sum of squares of M nonlinear functions in N variables
APLSUB:LMSTR.FOR       minimize sum of squares of M nonlinear functions in N variables
APLSUB:LMSTR1.FOR      minimize sum of squares of M nonlinear functions in N variables
APLNOT:MACHIN.FOR      Returns the smallest positive value that added to 1.0 is .gt. 1.0.
APLNOT:MAKGAU.FOR      *TESS routine: Make a Gaussian convolution function.
APLSUB:MATVMU.FOR      multiplies a matrix and a vector
APLNOT:MLTMAP.FOR      *TESS routine: multiplies an image by a value, writes another.
APLNOT:NULB.FOR        Finds a root of a function in an interval.
APLNOT:NUT2.FOR        Computes nutation in longitude and obliquity for a Julian date.
APLNOT:NUT4.FOR        Computes nutation using a non ridgid earth model
APLNOT:PARANG.FOR      Computes antenna parallactic angles
QSUB:PASS1.FOR         First of two routines to FFT an image file.
QSUB:PASS2.FOR         Second of two routines to FFT an image file.
APLSUB:PERMAT.FOR      permutes rows or columns of matrix according to permutation vector
APLNOT:PRECES.FOR      Convert between mean and apparent positions (B1950 only)
APLNOT:QKSORT.FOR      Two key "quick" sort routine to sort arrays.
APLSUB:QRFAC.FOR       computes a QR factorization of an MxN matrix
APLSUB:QRSOLV.FOR      completes the least squares matrix solution
APLSUB:RANDIN.FOR      initializes tables for random number routine RANDUM
APLSUB:RANDUM.FOR      generates random number between 0 and 1; initialized by RANDIN
APLNOT:RESID.FOR       *TESS routine: Computes residual image.
APLNOT:RFFTF.FOR       Vectorizable, table lookup Fast Fourier transform (non-AP)
APLSUB:RWUPDT.FOR      computes the QR decomposition of an upper triangular matrix + a row
APLSUB:SLAEVP.FOR      Earth position and motion ephemeris (J2000)
APLNOT:SOUELV.FOR      Computes source hour angles and elevations
APLNOT:SPHFN.FOR       Evaluate rational approx. to selected spheriodial functions.
APLNOT:STEP.FOR        *TESS routine: adds a fraction of one image to another.
APLNOT:SUBMAP.FOR      *TESS Routine: Subtract two images.
\end{verbatim}
 
\subsection{MESSAGES}
\begin{verbatim}
APLGEN:ZMSGCL.FOR      close Message file or terminal
\end{verbatim}
 
\subsection{MODELING}
\begin{verbatim}
QNOT:ALGSUB.FOR        Interpolates model visibility grom a grid and subtracts from uv data.
APLNOT:BMSHP.FOR       *TESS routine to fit an elliptical Gaussian to a dirty beam
QNOT:CCSGRD.FOR        Transforms CLEAN components to a grid.
APLSUB:COVAR.FOR       Determines the covariance matrix of an M x N matrix
APLSUB:DECONV.FOR      deconvolves two gaussians
QNOT:FFTIM.FOR         FFTs an image for uv interpolation.
APLNOT:FRQTAB.FOR      Fill Frequency table in common for IFs and channels
APLSUB:GETERR.FOR      calculates the errors on the fitted parameters
APLNOT:GRDAT.FOR       Getn info about CLEAN components for GRDSUB.
QNOT:GRDCRM.FOR        Loads CLEAN components into AP for uv model computation.
APLNOT:GRDSET.FOR      Creates scratch files and sets up for GRDSUB
QNOT:GRDSUB.FOR        Subtracts transform of CLEAN components from uv data.
QNOT:INTPFN.FOR        Computes interpolation kernals and put them into "AP memory".
APLSUB:LMDER.FOR       minimize the sum of squares of M nonlinear functions in N variables
APLSUB:LMDER1.FOR      minimize the sum of squares of M nonlinear functions in N variables
APLSUB:LMPAR.FOR       completes solution of the MxN matrix least squares problem
APLSUB:LMSTR.FOR       minimize sum of squares of M nonlinear functions in N variables
APLSUB:LMSTR1.FOR      minimize sum of squares of M nonlinear functions in N variables
APLSUB:MOM.FOR         calculates moments in a 16x16 data array
AIPSUB:PFIT.FOR        parabolic fit to 3x3 matrix
APLSUB:QRFAC.FOR       computes a QR factorization of an MxN matrix
APLSUB:QRSOLV.FOR      completes the least squares matrix solution
APLSUB:RWUPDT.FOR      computes the QR decomposition of an upper triangular matrix + a row
APLNOT:SETGDS.FOR      Sets up for UV model computation, fills common in DGDS.INC
APLNOT:UVDOUT.FOR      Divides uv model in one half of a record into other, writes result.
APLNOT:UVDPAD.FOR      Reformat UV data record, doubling size and zero extra words.
QNOT:UVMDIV.FOR        Divides a uv data set by the Fourier transform of a model.
QNOT:UVMSUB.FOR        Subtracts the Fourier transform of a model from a uv data set.
QNOT:VISDFT.FOR        Compute DFT of model and subtract/divide from/into uv data.
\end{verbatim}
 
\subsection{PARSING}
\begin{verbatim}
APLSUB:CH2NUM.FOR      converts string containing an integer in ASCII form into the integer
APLSUB:CHLTOU.FOR      converts a CHARACTER string to all upper case letters
APLNOT:CITC2D.FOR      KEYIN routine: parses Double precision value from a character string
APLNOT:CITC2I.FOR      KEYIN routine: parse an integer from a character string.
APLNOT:CITC2R.FOR      KEYIN routine: parses a floating value from a character string
APLNOT:CITCPR.FOR      KEYIN routine: character compare with wild cards.
APLNOT:CITEXP.FOR      KEYIN routine: evaluate an expression in a character string
APLNOT:CITSKP.FOR      KEYIN routine: Find next non blank character in a string.
APLNOT:DCODEF.FOR      Decodes data from a character string using a format.
APLSUB:FPARSE.FOR      interprets card image from FITS header into AIPS header format
APLSUB:GETCRD.FOR      parses card image from FITS header, returns recognized keyword
APLNOT:GETKEY.FOR      Parses symbol = value from a character string.
APLSUB:GETLOG.FOR      returns value of logical variable from character buffer
APLSUB:GETNUM.FOR      returns numeric field from character buffer
APLSUB:GETSTR.FOR      returns a string value (was enclosed by quotes) from character buffer
APLSUB:GETSYM.FOR      returns next symbol in character-form card image
APLSUB:GTWCRD.FOR      returns allowed keyword from FITS header card image
APLSUB:IDWCRD.FOR      returns allowed keyword from FITS header card image
APLNOT:KEYIN.FOR       AIPS version of CIT parsing routine
APLNOT:SDTCRD.FOR      Parse "SINGLDSH" FITS table headers, get some keywords.
APLNOT:TABF3D.FOR      Determines repeat count and data type for FITS 3-D tables entries.
APLNOT:TABFRM.FOR      Parses format for FITS ASCI table entries.
APLNOT:TABSPC.FOR      Determines repeat count and data type for FITS 3-D tables entries.
\end{verbatim}
 
\subsection{PLOT-APPL}
\begin{verbatim}
APLNOT:AITOFF.FOR      writes vectors for Aitoff projection grid to plot file
AIPSUB:AU8A.FOR        verb EXTLIST to list contents of plot files and other extension files
APLNOT:COMCLR.FOR      Scale and map complex array intp RBG space, amp=inten. phase=hue.
\end{verbatim}
 
\subsection{PLOT-UTIL}
\begin{verbatim}
APLSUB:AXSTRN.FOR      encodes axis type and value in a string
APLSUB:CHNTIC.FOR      counts characters to the left of a plot (for labeling vertical axis)
APLSUB:CLAB1.FOR       puts axis labels in plot file and calls CTICS to draw and label ticks
APLSUB:CLAB2.FOR       puts axis labels in plot file and calls CTICS to draw and label ticks
APLSUB:COMLAB.FOR      initializes line drawing and labels plot with text, contour levels
APLSUB:CONDRW.FOR      writes contour plot to a plot file
APLSUB:CTICS.FOR       writes tick marks and tick labels to a plot file
APLSUB:GCHAR.FOR       writes a draw character string command record into a plot file
APLSUB:GFINIS.FOR      writes the end of plot record into a plot file and closes it down
APLSUB:GINIT.FOR       creates, opens, initializes plot file (does not catalog it)
APLSUB:GINITG.FOR      writes an initialize-for-grey-scale record into a plot file
APLSUB:GINITL.FOR      writes an initialize-for-line-drawing command into a plot file
APLSUB:GMCAT.FOR       writes a copy-misc-image-catalog-info records into a plot file
APLSUB:GPHWRT.FOR      write plot buffer to file, prepares buffer for more commands
APLSUB:GPOS.FOR        write a position-"pen" command into a plot file
APLSUB:GRAYPX.FOR      writes an array of grey values into a plot file
APLSUB:GVEC.FOR        writes a move-pen-down (or write vector) command in a plot file
APLSUB:HIPLOT.FOR      places a record in the history file concerning a plot file creation
APLSUB:INTEDG.FOR      returns intersections of a line with the edges of a box
APLSUB:ISCALE.FOR      scale a buffer by various functions to an integer buffer (ie for TV)
APLSUB:LABINI.FOR      initializes commons for labeling of plots (calls SETLOC)
APLSUB:LABNO.FOR       write a tick mark numeric label in a plot file
APLSUB:LINLIM.FOR      clips X,Y values at edges of rectangular area with interpolation
APLNOT:PLEND.FOR       End-of-plot clean-up functions: Gary plot package.
APLSUB:PLGRY.FOR       draws grey scale commands in the plot file: Gary plot package
APLNOT:PLMAKE.FOR      creates & opens plot file, puts into map header, writes first record
APLSUB:PLPOS.FOR       puts a position vector command in a plot file: Gary plot package
APLSUB:PLVEC.FOR       puts a draw vector command in a plot file: Gary plot package
APLSUB:RNGSET.FOR      set plat intensity range from image header and user parameters
APLSUB:SETLOC.FOR      sets location common for coordinate computations and display
APLSUB:SCALMM.FOR      computes plot scaling factors and plot scale in arc sec per mm
APLSUB:SLBINI.FOR      initializes labeling for slice plots
APLSUB:STARPL.FOR      adds to plot plus signs at coordinates given in an ST (star) file
APLSUB:TICCOR.FOR      correct tick lengths from increments in dir cosines to coordinates
APLSUB:TICINC.FOR      determines tick mark lengths and increments for CTICS, ...
APLSUB:TKLAB.FOR       labels axes on plot directly to a TK graphics device, draw ticks
APLSUB:TKSLIN.FOR      initialize parameters for plotting a slice directly on a TK graphics
APLSUB:TKTICS.FOR      writes tick marks and labels directly to TK graphics device
APLGEN:ZDOPRT.FOR      reads bit file and causes it to be plotted on printer/plotter
APLGEN:ZLASC2.FOR      spool a closed laser printer print/plot file
APLGEN:ZLASCL.FOR      close and spool a laser printer print/plot file
APLGEN:ZLASIO.FOR      open, write to, close and spool a laser printer print/plot file
APLGEN:ZLASOP.FOR      open a laser printer print/plot file
APLGEN:ZLWIO.FOR       open, write to, close and spool a PostScript print/plot file
APLGEN:ZLWOP.FOR       open a PostScript (LaserWriter) print/plot file
\end{verbatim}
 
\subsection{POPS-APPL}
\begin{verbatim}
AIPSUB:AU1.FOR         prints and clears the message file, sets up for EXIT and RESTART
AIPSUB:AU1A.FOR        does parameter display: INPUTS, SHOW, HELP, EXPLAIN
AIPSUB:AU2.FOR         handles task-related activities: GO, TELL, WAIT, ABORT, SPY, TPUT
AIPSUB:AU2A.FOR        verb functions on task save and Save/Get files: TGET, SGdestr, index
AIPSUB:AU3.FOR         verbs to display contents of catalogs and headers: CATA, IMHE ...
AIPSUB:AU3A.FOR        verbs for disk management: FREE, ALLDEST, TIMDEST, etc.
AIPSUB:AU3B.FOR        verbs to rearrange the entries in the catalog file: RECAT, RENUMBER
AIPSUB:AU4.FOR         verbs to handle basic tape operations: TPHEAD, MOUNT, AVFILE, ...
AIPSUB:AU5.FOR         basic TV verbs to do on/off, read cursor position, init the TV, ...
AIPSUB:AU5A.FOR        verbs to load images to the TV including ROAM
AIPSUB:AU5B.FOR        verbs to anotate TV images
AIPSUB:AU5C.FOR        verbs to draw wedges on TV, erase images, set corners with TV cursor
AIPSUB:AU5D.FOR        verbs to load and run TV movie sequences
AIPSUB:AU6.FOR         verbs to manipulate TV scroll, zoom, color tables, and TVHUEINT
AIPSUB:AU6A.FOR        verbs to set the TV blank and white LUT linearly and to blink planes
AIPSUB:AU6B.FOR        verb to display image value at pixel indicated by TV cursor (CURVAL)
AIPSUB:AU6C.FOR        verb to alter zoom and enhance image in standard way: TVFIDDLE
AIPSUB:AU6D.FOR        verbs to do image statistics in blotch regions: TVSTAT, IMSTAT
AIPSUB:AU7.FOR         verbs to print history, rescale image, alter axis descriptions
AIPSUB:AU7A.FOR        verbs to put/get header values, to put values into images
AIPSUB:AU8.FOR         verbs to get or clear name adverbs, destroy extension files
AIPSUB:AU8A.FOR        verb EXTLIST to list contents of plot files and other extension files
AIPSUB:AU9.FOR         verbs to fit or interpolate the image intensity (MAXFIT, IMVAL)
AIPSUB:AU9A.FOR        verbs to read TEK cursor and display pixel, sky, image values
AIPSUB:AU9B.FOR        verbs to plot slices and models on graphics
AIPSUB:AU9C.FOR        verbs to set initial guesses for slice model fits using TEK graphics
AIPSUB:AUA.FOR         verb to submit batch jobs to AIPSC and the QMNGR queues
AIPSUB:AUB.FOR         verbs to prepare, edit, and review batch jobs and queues
AIPSUB:AUC.FOR         verbs to enter, list, drop gripes, enter password
AIPSUB:AUT.FOR         site-specific test verbs
AIPSUB:PRTALN.FOR      prints line on CRT orprinter, handles page full AND POPS type-ahead
AIPSUB:TASKWT.FOR      waits for tasks to begin, send resumption signal, and/or terminate
\end{verbatim}
 
\subsection{POPS-LANG}
\begin{verbatim}
AIPSUB:BCLEAN.FOR      Pops items from B-stack to A-stack until BPR-stack precedence < NEXTP
AIPSUB:CHUNT.FOR       searches symbol table for character string accepting min match
AIPSUB:COMPIL.FOR      parses line of input with GETFLD, builds stacks for execution
AIPSUB:CONCAT.FOR      creates temporary literal on stack = concatanation of 2 strings
AIPSUB:EDITOR.FOR      does operations needed at start and end of editing existing procedure
AIPSUB:EQUIV.FOR       checks whether two variables are logically equivalent
AIPSUB:GETFLD.FOR      finds the next symbol in KARBUF and determines its pointers
AIPSUB:GETNME.FOR      gets the next name in the input character buffer
AIPSUB:HELPS.FOR       executes "pseudoverb name" -> hidden verb w name on stack (INP, RUN)
AIPSUB:HUNT.FOR        searches a linked list for words to be matched
AIPSUB:INIT.FOR        initializes symbol, procedure text tables, and commons for POPS
AIPSUB:KWICK.FOR       verbs:  math, assignment, comparison, looping, branching, proc calls
AIPSUB:LLOCAT.FOR      allocates space in linked-list array and handles link pointers
AIPSUB:LTSTOR.FOR      allocate storage for literal if needed, return pointer in any case
AIPSUB:MASSGN.FOR      handles array = value(s) constructs
AIPSUB:OERROR.FOR      gives user error message, resets parameters to read next input line
AIPSUB:POLISH.FOR      parses the input text buffer, building stacks; executes pseudoverbs
AIPSUB:POP.FOR         pops item from stack
AIPSUB:PREAD.FOR       reads an input line from current input source (CRT, RUN file, batch)
AIPSUB:PSEUDO.FOR      compiles pseudoverbs: PROC, declarations, IF, THEN, WHILE, FINISH, ..
AIPSUB:PUSH.FOR        pushes item onto stack advancing the stack pointer
AIPSUB:RLOCAT.FOR      allocates space in linked-list array and handles link pointers
AIPSUB:SETTYP.FOR      replaces the symbol type code in the data description structure
APLSUB:STLTOU.FOR      converts any characters beween single quotes to upper case
AIPSUB:STORES.FOR      stores proc code; pseudoverbs: SAVE, GET, RESTORE, STORE, LIST, ...
AIPSUB:SUBS.FOR        converts variable with subscript to the appropriate scalar
AIPSUB:SYMBOL.FOR      obtains symbol identification from symbol table; creates new symbols
AIPSUB:VERBS.FOR       calls verbs subroutines (AUnc) by verb number - interactive version
AIPSUB:VERBSB.FOR      calls verbs subroutines (AUnc) by verb number - batch version
AIPSUB:VERBSC.FOR      calls verbs subroutines (AUnc) by verb number - Checker version
\end{verbatim}
 
\subsection{POPS-UTIL}
\begin{verbatim}
AIPSUB:ASSGN.FOR       performs the assignment functions of scalar/vector = scalar/vector
AIPSUB:CONFRM.FOR      asks user to respond yes or no to some question
APLSUB:GETNUM.FOR      returns numeric field from character buffer
APLSUB:GETSTR.FOR      returns a string value (was enclosed by quotes) from character buffer
AIPSUB:PRTMSG.FOR      prints and deletes messages from the MS file
AIPSUB:RDUSER.FOR      reads the user number from the terminal
AIPSUB:SCHOLD.FOR      wait for user input on screen full, allows type ahead, quit, continue
AIPSUB:SGLAST.FOR      does a SAVE or GET of the K array cataloged as LASTEXIT.
AIPSUB:SGLOCA.FOR      locates a Save/Get file by name in catalog of SG files
AIPSUB:UINIT.FOR       general, non-language initialization routine; only calls VHDRIN
\end{verbatim}
 
\subsection{PRINTER}
\begin{verbatim}
APLSUB:BATPRT.FOR      prints header/trailer messages for printer tasks when run in batch
APLSUB:DATDAT.FOR      converts "DD/MM/YY" form of date to "dd-mmm-yyyy" for printing
AIPSUB:PRTALN.FOR      prints line on CRT orprinter, handles page full AND POPS type-ahead
APLSUB:PRTLIN.FOR      prints line on printer or terminal with page-full handling, headers
APLGEN:ZDOPRT.FOR      reads bit file and causes it to be plotted on printer/plotter
APLGEN:ZENDPG.FOR      advance printer if needed to avoid electrostatic-printer "burn-out"
APLGEN:ZLASC2.FOR      spool a closed laser printer print/plot file
APLGEN:ZLASCL.FOR      close and spool a laser printer print/plot file
APLGEN:ZLASIO.FOR      open, write to, close and spool a laser printer print/plot file
APLGEN:ZLASOP.FOR      open a laser printer print/plot file
APLGEN:ZLPCL2.FOR      queue a file to the line printer and delete
APLGEN:ZLPCLS.FOR      close an open printer device
APLGEN:ZLPOP2.FOR      open a line-printer text file - actual OPEN call
APLGEN:ZLPOPN.FOR      open a line-printer text file
APLGEN:ZLWIO.FOR       open, write to, close and spool a PostScript print/plot file
APLGEN:ZLWOP.FOR       open a PostScript (LaserWriter) print/plot file
\end{verbatim}
 
\subsection{SDISH}
\begin{verbatim}
APLNOT:CSINI.FOR       Create/Open/Init Single dish calibration (CS) table
APLNOT:DCALSD.FOR      Apply Single dish calibration to data.
APLNOT:DGETSD.FOR      Reads, selects single dish data, calibrates and edits.
APLNOT:GACSIN.FOR      Initializes CS file, and prepares table to be applied.
APLNOT:SDCGET.FOR      Sets up to interpolate in Single dish calibration (CS) table.
APLNOT:SDCSET.FOR      Interpolates single dish calibration data for current time.
APLNOT:SDGET.FOR       Reads single dish data with optional calibration and flagging
APLNOT:SDTCRD.FOR      Parse "SINGLDSH" FITS table headers, get some keywords.
APLNOT:TABCS.FOR       Does I/O to single dish calibration (CS) table opened by CSINI
\end{verbatim}
 
\subsection{SERVICE}
\begin{verbatim}
AIPSUB:AIPINI.FOR      does all AIPS initializations for a stand-alone program
AIPSUB:DESCR.FOR       destroys all scratch files for tasks which are no longer active
APLSUB:INQFLT.FOR      inquire of the user for specified number of floating-point values
APLSUB:INQGEN.FOR      inquire of user for specified list of integer, float, & char values
APLSUB:INQINT.FOR      inquire of user for specified number of integer values
APLSUB:INQSTR.FOR      request character string from user (1st N characters of line)
APLGEN:ZADDR.FOR       determine if 2 addresses inside computer are the same
APLGEN:ZDELA2.FOR      delay current process a specified interval
APLGEN:ZDELAY.FOR      delay current process a specified interval
APLGEN:ZERRO2.FOR      return system error message for given system error code
APLGEN:ZGTBIT.FOR      get array of bits from a word
APLGEN:ZHEX.FOR        encode an integer into hexadecimal characters
APLGEN:ZKDUMP.FOR      display portions of an array in various Fortran formats
APLGEN:ZMSGWR.FOR      call MSGWRT based on call arguments - for C routines to call MSGWRT
APLGEN:ZMYVER.FOR      returns OLD, NEW, or TST based on translation of logical AIPS_VERSION
APLGEN:ZPTBIT.FOR      put array of bits into a word
APLGEN:ZTIME.FOR       return the local time of day
\end{verbatim}
 
\subsection{SLICE}
\begin{verbatim}
AIPSUB:AU9B.FOR        verbs to plot slices and models on graphics
AIPSUB:AU9C.FOR        verbs to set initial guesses for slice model fits using TEK graphics
AIPSUB:SET1DG.FOR      sets initial guess parameters with the TEK for fitting slices
AIPSUB:SLOCIN.FOR      initialize location common for slice (on the TEK) model fitting
AIPSUB:TKGGPL.FOR      plots model slice on Graphics
AIPSUB:TKGMPL.FOR      plot model fit to slice on the graphics terminal
AIPSUB:TKRSPL.FOR      plots residuals between slice and its model on Graphics device
APLSUB:TKSLIN.FOR      initialize parameters for plotting a slice directly on a TK graphics
AIPSUB:TKSLPL.FOR      plot a slice on graphics device
\end{verbatim}
 
\subsection{SORT}
\begin{verbatim}
APLSUB:LSORT.FOR       sort a data buffer minimizing number times records are switched
APLSUB:MERGE.FOR       sorts by merging previously sorted blocks of records
APLSUB:OSORT.FOR       does quick sort on array of vectors, then reorders by calling PERMAT
APLSUB:PERMAT.FOR      permutes rows or columns of matrix according to permutation vector
APLSUB:SHSORT.FOR      Shell sort of an array or records on two keys
\end{verbatim}
 
\subsection{SPECTRAL}
\begin{verbatim}
APLNOT:DATBND.FOR      Applies the bandpass correction to data.
APLNOT:FQMATC.FOR      Check if selection criteria match FQ table entries.
APLNOT:FRQTAB.FOR      Fill Frequency table in common for IFs and channels
APLNOT:IOBSRC.FOR      Search for antennas in the current bandpass buffer.
APLNOT:SCINTP.FOR      Interpolates bandpass tables in time.
APLNOT:SCLOAD.FOR      Copies part of one bandpass scratch file to another for efficiency.
APLNOT:SETSM.FOR       Determines type of spectral smoothing and sets up look up table.
APLNOT:SMOSP.FOR       Convolves a spectrum with a tabulated function.
APLNOT:TABBP.FOR       Does I/O to bandpass (BP) table opened by BPINI
APLNOT:UVGET.FOR       Read UV data with optional calibration, editing, selection, etc.
\end{verbatim}
 
\subsection{SYSTEM}
\begin{verbatim}
APLSUB:ACOUNT.FOR      Writes beginning and final entries in the AIPS accounting file
AIPSUB:AIPINI.FOR      does all AIPS initializations for a stand-alone program
APLSUB:BATQ.FOR        performs operations on batch queue control file such as OPEN RUN CLOS
APLSUB:DIE.FOR         closes down tasks which use DFIL.INC to maintain status of files
APLSUB:DIETSK.FOR      closes a task: restarting AIPS, settling the accounting, issuing msg
APLSUB:PASENC.FOR      encrypts a 12-character password into 3 Holleriths
APLSUB:PASWRD.FOR      prompts for and checks password if the user has a non-blank PW entry
APLSUB:RELPOP.FOR      places a return code in the task data file, thereby resuming AIPS
APLSUB:WHOAMI.FOR      given root task name, gets actual task name and finds NPOPS number
APLGEN:ZABOR2.FOR      establishes or carries out (when appropriate) abort handling
APLGEN:ZABORT.FOR      establishes or carries out (when appropriate) abort handling
APLGEN:ZACTV8.FOR      activate the requested program, returning process ID information
APLGEN:ZCPU.FOR        return current process CPU time and IO count
APLGEN:ZDATE.FOR       return the local date
APLGEN:ZDCHI2.FOR      initialize device and Z-routine characteristics commons - local vals
APLGEN:ZDCHIC.FOR      set more system parameters; make them available to C routines
APLGEN:ZDCHIN.FOR      initialize message, device and Z-routine characteristics commons
APLGEN:ZFREE.FOR       display available disk space
APLGEN:ZGNAME.FOR      get name of current process
APLGEN:ZPRI2.FOR       raise or lower the process priority
APLGEN:ZPRIO.FOR       raise or lower the process priority
APLGEN:ZPRPAS.FOR      prompt user and read 12-character password (invisible) from CRT
APLGEN:ZSETUP.FOR      performs system-level operations after VERNAM, TSKNAM, NPOPS known
APLGEN:ZSTAI2.FOR      does any system cleanup needed at the end of interactive AIPS session
APLGEN:ZSTAIP.FOR      does any system cleanup needed at the end of interactive AIPS session
APLGEN:ZTACT2.FOR      inquires if a task is currently active on the local computer
APLGEN:ZTACTQ.FOR      inquires if a task is currently active on the local computer
APLGEN:ZTKILL.FOR      deletes (or kills) the specified process
APLGEN:ZTQSP2.FOR      display AIPS account or all processes running on the system
APLGEN:ZTQSPY.FOR      display AIPS account or all processes running on the system
APLGEN:ZTRLOG.FOR      translate a logical name
APLGEN:ZWHOMI.FOR      determines AIPSxn task name; sets NPOPS, assigns TV and TK devices
\end{verbatim}
 
\subsection{TAPE}
\begin{verbatim}
APLSUB:DWRITE.FOR      translate "DEC" format map header and display parameters
APLSUB:EXTHIS.FOR      adds to history file for contents of FITS extension file being read
APLSUB:EXTREQ.FOR      parse FITS tape record for required extension file FITS keywords
APLSUB:FNDEOT.FOR      advances tape to logical end of information (2 consecutive EOFs)
AIPSUB:FWRITE.FOR      converts FITS header to AIPS header and displays it with MSGWRT
APLNOT:GTF3D.FOR       Copies real-world bytes from a tape buffer, reading if necessary.
APLSUB:MLREOF.FOR      advances tape to end of file and reports records read in TAPIO system
APLNOT:PTF3D.FOR       Copies 8-bit bytes to tape.
APLNOT:R3DTAB.FOR      Read data from FITS 3-D table and write AIPS table.
APLNOT:RWTAB.FOR       Read FITS ASCII table data and write AIPS table file.
APLNOT:SDTCRD.FOR      Parse "SINGLDSH" FITS table headers, get some keywords.
APLSUB:SKPBLK.FOR      find next non-blank card image in a FITS header, read tape if needed
APLSUB:SKPEXT.FOR      finishes reading FITS extension header, skips the extension data
APLNOT:TABAXI.FOR      parse FITS tape record for required extension file FITS keywords
APLNOT:TABF3D.FOR      Determines repeat count and data type for FITS 3-D tables entries.
APLNOT:TABFRM.FOR      Parses format for FITS ASCI table entries.
APLSUB:TABLIN.FOR      reads a line from the data portion of a FITS extension of type TABLE
APLNOT:TABSPC.FOR      Determines repeat count and data type for FITS 3-D tables entries.
APLSUB:TAPIO.FOR       read/writes tape and FITS disk files
APLSUB:TPHEAD.FOR      reads a tape record, advances over label file, decides if it
APLNOT:TPIOHD.FOR      Reads tape header and tests if FITS, tape labels etc.
AIPSUB:UWRITE.FOR      writes summary of UV Export-format tape
APLSUB:VBOUT.FOR       writes variable length, blocked records of 16-bit integers to tape
APLGEN:ZBKLD1.FOR      initialize environment for BAKLD
APLGEN:ZBKLD2.FOR      does BACKUP operation: load images from tape to directory
APLGEN:ZBKLD3.FOR      clean up system things for BAKLD ending
APLGEN:ZBKTP1.FOR      initialize BACKUP to tape operation for BAKTP
APLGEN:ZBKTP2.FOR      write a cataloged file plus extensions to BACKUP tape in BAKTP
APLGEN:ZBKTP3.FOR      clean up host environment at end of BAKTP
APLGEN:ZBYTF2.FOR      interchange bytes in buffer if needed to go between local & standard
APLGEN:ZBYTFL.FOR      interchange bytes in buffer if needed to go between local & standard
APLGEN:ZMCACL.FOR      convert Modcomp compressed ASCII to Hollerith characters (for FILLR)
APLGEN:ZMOUN2.FOR      mount or dismount magnetic tape device
APLGEN:ZMOUNT.FOR      mount or dismount magnetic tape device
APLGEN:ZR8P4.FOR       converts pseudo I*4 to double precision - for tape handling only
APLGEN:ZRDMF.FOR       convert DEC Magtape Format (36 bits data in 40 bits) to 2 integers
APLGEN:ZRM2RL.FOR      convert Modcomp to local single precision floating point
APLGEN:ZTAP2.FOR       position (forward/back record/file), write EOF, etc. for tapes
APLGEN:ZTAPE.FOR       mount, dismount, position, write EOF, etc. for tapes
APLGEN:ZTAPIO.FOR      tape operations for IMPFIT (compressed FITS transport tape)
APLGEN:ZTPCL2.FOR      close a tape device
APLGEN:ZTPCLD.FOR      close pseudo-tape disk file
APLGEN:ZTPCLS.FOR      closes a tape device (real or pseudo-tape disk)
APLGEN:ZTPMI2.FOR      tape read/write
APLGEN:ZTPMID.FOR      pseudo-tape disk read/write for 2880-bytes records
APLGEN:ZTPMIO.FOR      read/write tape devices with quick return IO methods
APLGEN:ZTPOP2.FOR      open a tape device for double-buffer, asymchronous IO
APLGEN:ZTPOPD.FOR      open a pseudo-tape, sequential disk file for FITS
APLGEN:ZTPOPN.FOR      open tape or pseudo-tape device
APLGEN:ZTPWA2.FOR      wait for read/write from/to a tape device
APLGEN:ZTPWAD.FOR      "wait" for IO operation to complete on pseudo-tape disk file (ZTPMID)
APLGEN:ZTPWAT.FOR      wait for asynchronous IO to finish on tape or pseudo-tape disk
\end{verbatim}
 
\subsection{TERMINAL}
\begin{verbatim}
APLGEN:ZPRMPT.FOR      prompt user and read 80-characters from CRT screen
APLGEN:ZPRPAS.FOR      prompt user and read 12-character password (invisible) from CRT
APLGEN:ZTTBUF.FOR      reads terminal input with no prompt or wait - simulates TV trackball
APLGEN:ZTTCLS.FOR      close a terminal device
APLGEN:ZTTOP2.FOR      open a message terminal
APLGEN:ZTTOPN.FOR      open a terminal device
APLGEN:ZTTYIO.FOR      read/write buffer to terminal
\end{verbatim}
 
\subsection{TEXT}
\begin{verbatim}
APLSUB:TXTMAT.FOR      min match handling for text files (calls ZTXMAT, does messages)
APLSUB:VERMAT.FOR      min match handling for text file names incl sequence of directories
APLGEN:ZDIR.FOR        build a full path name to files in AIPS-standard areas (HE, RU, ...)
APLGEN:ZTCLOS.FOR      close text file opened with ZTOPEN
APLGEN:ZTOPE2.FOR      open text file for ZTOPEN
APLGEN:ZTOPEN.FOR      open text file - logical area, version, member name as arguments
APLGEN:ZTREAD.FOR      read next 80-character record in sequential text file (ZTOPEN type)
APLGEN:ZTXCLS.FOR      clos text file opened via ZTXOPN
APLGEN:ZTXIO.FOR       read/write a line to a text file
APLGEN:ZTXMA2.FOR      find all file names matching a given wildcard specification
APLGEN:ZTXMAT.FOR      return list of files in specified area beginning with specified chars
APLGEN:ZTXOP2.FOR      translate the file name and open a text file
APLGEN:ZTXOPN.FOR      open a text file for read or write
\end{verbatim}
 
\subsection{TV}
\begin{verbatim}
APLNOT:IMWIN.FOR       Set up window on TV device
\end{verbatim}
 
\subsection{TV-APPL}
\begin{verbatim}
AIPSUB:AU5.FOR         basic TV verbs to do on/off, read cursor position, init the TV, ...
AIPSUB:AU5A.FOR        verbs to load images to the TV including ROAM
AIPSUB:AU5B.FOR        verbs to anotate TV images
AIPSUB:AU5C.FOR        verbs to draw wedges on TV, erase images, set corners with TV cursor
AIPSUB:AU5D.FOR        verbs to load and run TV movie sequences
AIPSUB:AU6.FOR         verbs to manipulate TV scroll, zoom, color tables, and TVHUEINT
AIPSUB:AU6A.FOR        verbs to set the TV blank and white LUT linearly and to blink planes
AIPSUB:AU6B.FOR        verb to display image value at pixel indicated by TV cursor (CURVAL)
AIPSUB:AU6C.FOR        verb to alter zoom and enhance image in standard way: TVFIDDLE
AIPSUB:AU6D.FOR        verbs to do image statistics in blotch regions: TVSTAT, IMSTAT
QNOT:DISPTV.FOR        *TESS routine: Display an image on a TV
YSUB:GRBOXS.FOR        sets rectangular boxes or diagonal line with TV cursor and graphics
AIPSUB:GRLUTS.FOR      interactive piecewise linear LUT using graphic plane and cursor
AIPSUB:HIENH.FOR       interactive linear enhancement of 2-image hue-intensity TV display
AIPSUB:HILUT.FOR       calculates new LUTs for hue-intensity display and sends them to TV
YSUB:IAXIS1.FOR        draws axis labels and tick marks (via ITICS) on TV
APLNOT:IMOPEN.FOR      Open the TV under the system set up by IOSET
YSUB:ITICS.FOR         draws tick marks and labels on TV
AIPSUB:TVBLNK.FOR      blinks two TV channels, cursor controls rate
APLNOT:TVFOAD.FOR      TVFLG routine to load and image with smoothing converting to display.
AIPSUB:TVMOVI.FOR      runs movie algorithm on pre-loaded images, with interactions
AIPSUB:TVROAM.FOR      does interactive multi-channel "ROAM" display on pre-loaded images
YGEN:YCUCOR.FOR        correct cursor position for scroll; return image coordinates, header
YGEN:YISDRM.FOR        read/write data memory of NRAO-ISU device
YGEN:YISDSC.FOR        read/write micro-processor memory of NRAO-ISU device
YGEN:YISJMP.FOR        cause microprocessor jump toaddress in NRAO-ISU device
YGEN:YISLOD.FOR        loads/unloads program memory of NRAO-ISU device
YGEN:YISMPM.FOR        reads/writes microprocessor memory of the NRAO-ISU device
YGEN:YMKCUR.FOR        selects the form of the cursor to be displayed
\end{verbatim}
 
\subsection{TV-BASIC}
\begin{verbatim}
YGEN:YALUCT.FOR        drives the TV arithmetic logic unit - not to be used much
YGEN:YCONST.FOR        controls the constant registers added to the TV picture - not used
YGEN:YCRCTL.FOR        controls the TV cursor visibility, position; reads trackball buttons
YGEN:YFDBCK.FOR        causes a feedback operation in the TV
YGEN:YGGRAM.FOR        controls the TV graphics color assignments
YGEN:YGRAFE.FOR        controls the graphics control register (IIS function)
YGEN:YGRAPH.FOR        turns TV graphics planes on and off
YGEN:YIFM.FOR          read/write TV Input look-up-table
YGEN:YIMGIO.FOR        read/write data to the TV grey and graphics memories
YGEN:YINIT.FOR         initialize everything about the TV
YGEN:YLUT.FOR          read/write channel-based look-up-table
YGEN:YMNMAX.FOR        read 3 min/max values from TV data paths (IIS only, not used)
YGEN:YOFM.FOR          read/write all-channel look-up-table ("output function memory")
YGEN:YRHIST.FOR        read the histogram of the selected TV output color
YGEN:YSCROL.FOR        write the scroll registers (shift location of 1 or more TV channels)
YGEN:YSHIFT.FOR        read/write the shift (bias) registers of the TV (IIS M70, not used)
YGEN:YSPLIT.FOR        set channel selection by split-screen quadrant
YGEN:YSTCUR.FOR        reads/writes the cursor pattern array
YGEN:YTVCIN.FOR        initialize TV characteristics common (not needed much - see TVOPEN)
YGEN:YTVCLS.FOR        close the TV, including TV device and TV control/parameter disk file
YGEN:YTVOPN.FOR        open the TV device and the TV disk control/parameter file.
YGEN:YZOOMC.FOR        set the TV zoom magnification and center
\end{verbatim}
 
\subsection{TV-IO}
\begin{verbatim}
YGEN:YTVCL2.FOR        close actual TV device (called by YTVCLS)
YGEN:YTVMC.FOR         issue a master clear to reinitialize IO to the TV
YGEN:YTVOP2.FOR        open actual TV device (called by YTVOPN)
APLGEN:ZARGS.FOR       sends command to/from the ARGS TV device
APLGEN:ZARGXF.FOR      translates IIS Model 70 commands into calls to ZARGS for ARGS TV
APLGEN:ZDEAC2.FOR      close DeAnza TV device
APLGEN:ZDEACL.FOR      close DeAnza TV device
APLGEN:ZDEAMC.FOR      issue a master clear to the TV - for DeAnzas this is a No-Op
APLGEN:ZDEAO2.FOR      opens DeAnza TV device
APLGEN:ZDEAOP.FOR      opens DeAnza TV device
APLGEN:ZDEAX2.FOR      do actual read/write from/to DeAnza device
APLGEN:ZDEAXF.FOR      do IO to DeAnza TV
APLGEN:ZIVSOP.FOR      opens IVAS TV device - using the IIS package
APLGEN:ZM70C2.FOR      close IIS Model 70/75 TV device
APLGEN:ZM70O2.FOR      opens IIS Model 70.75 TV device
APLGEN:ZM70X2.FOR      read/write from/to IIS Model 70/75 device
APLGEN:ZTTBUF.FOR      reads terminal input with no prompt or wait - simulates TV trackball
APLGEN:ZV20C2.FOR      close Comtal Vision 1/20 TV device
APLGEN:ZV20MC.FOR      issue a master clear to the TV - for Comtal this is a No-Op
APLGEN:ZV20O2.FOR      opens Comtal Vision 1/20 TV device
APLGEN:ZV20X2.FOR      does I/O to Comtal Vision 1/20 TV device
APLGEN:ZVTVC2.FOR      close virtual TV connection to remote, real-TV computer
APLGEN:ZVTVC3.FOR      close connection in real-TV computer to client, virtual-TV computer
APLGEN:ZVTVCL.FOR      close connection in client (virtual-TV) to server (remote, real-TV)
APLGEN:ZVTVGC.FOR      close & reopen connection in server (real-TV) to client (virtual-tv)
APLGEN:ZVTVO3.FOR      open connection in server (real-TV) to client (virtual-TV)
APLGEN:ZVTVOP.FOR      opens connection from client (virtual-TV) to server (real-TV)
APLGEN:ZVTVRC.FOR      closes channel in server (real-TV) to client (virtual-TV)
APLGEN:ZVTVRO.FOR      open socket in server (real-TV) to any client (virtual-TV)
APLGEN:ZVTVRX.FOR      does IO for server (real TV) to client (Virtual-TV) incl close/reopen
APLGEN:ZVTVX2.FOR      writes/reads to/from server for the client (virtual TV) machine
APLGEN:ZVTVX3.FOR      reads/writes from/to client (virtual TV) for the server (real TV)
APLGEN:ZVTVXF.FOR      sends data from the client (virtual TV) to server (real TV)
\end{verbatim}
 
\subsection{TV-UTIL}
\begin{verbatim}
YSUB:BLTFIL.FOR        fills in closed polygons on a tv "blotch" plane
APLSUB:CHAVRT.FOR      converts between local HOLL and local INT binary forms for transport
APLSUB:DECBIT.FOR      converts decimal coded number to bit coded (e.g. 13 -> 0000101)
YSUB:DLINTR.FOR        interactive delays, cursor tests, prevent wraparound
AIPSUB:GRPOLY.FOR      uses TV graphics to let user develop polygonal blotch regions
APLSUB:HDRBUF.FOR      translates AIPS header to/from FITS-standard integer form
YSUB:IENHNS.FOR        interactive linear enhancement of TV black & white LUTs
YSUB:ILNCLR.FOR        computes and loads a piecewise linear OFM to the TV
APLSUB:IMA2MP.FOR      converts pixel numbers in a TV-image into real image pixels
YSUB:IMANOT.FOR        draws a character string with black background to graphics
YSUB:IMCCLR.FOR        write color contour OFM to TV from standard sets
YSUB:IMCHAR.FOR        writes character string to TV
YSUB:IMLCLR.FOR        continuous colors fom blue thru green to red (or rotations thereof)
YSUB:IMPCLR.FOR        writes OFM with color contour helix in lightness-hue-saturation space
YSUB:IMVECT.FOR        draws connected line segments on TV
APLSUB:ISCALE.FOR      scale a buffer by various functions to an integer buffer (ie for TV)
APLSUB:MKYBUF.FOR      packages a command line into machine-independent form
APLSUB:MOVIST.FOR      sets/resets the movie status parameters in the TV common
APLSUB:MP2IMA.FOR      convert image pixel positions to TV pixel positions
APLSUB:REALOG.FOR      converts numbers between floating and an integer version of their log
APLSUB:REAVRT.FOR      converts between local REAL and local INT binary forms for transport
YSUB:TVCLOS.FOR        does error checks on device open, then closes the TV via YTVCLS
YSUB:TVFIDL.FOR        standard, simple interactive B&W LUT and color enhancements, zooming
AIPSUB:TVFIND.FOR      determines which of the visibile images on the TV the user desires
YSUB:TVLOAD.FOR        load image to a TV memory from open MA file
YSUB:TVOPEN.FOR        sets LUNs, calls YTVOPN to open the TV device, does error messages
YSUB:TVWHER.FOR        turns on cursor, waits for button, returns quadrant, position, button
APLSUB:TVWIND.FOR      determines image windows for TV, including for interpolation & Roam
APLSUB:UNYBUF.FOR      unpacks a machine-independent integer buffer into local command line
YGEN:YCHRW.FOR         writes characters into image and graphics planes
YGEN:YCINIT.FOR        initialize image catalog for specified TV memory plane
YGEN:YCNECT.FOR        write line segment between 2 points on TV
YGEN:YCOVER.FOR        checks for overlapped images on the TV by quadrant
YGEN:YCREAD.FOR        read the image catalog, return image header for TV only
YGEN:YCURSE.FOR        read and control TV cursor
YGEN:YCWRIT.FOR        write image header to image catalog, update image catalog directory
YGEN:YFILL.FOR         fill rectangle of TV memory with a constant value
YGEN:YFIND.FOR         determines the unique TV image of desired type, returns catalog block
YGEN:YGYHDR.FOR        builds basic TV IO header to write gray scale data
YGEN:YLOCAT.FOR        return TV positions for set of image positions
YGEN:YLOWON.FOR        select least on bit in a bit mask integer
YGEN:YMAGIC.FOR        initialize graphics, zoom, scroll units for IIS Model 75 (level 3)
YGEN:YMKHDR.FOR        builds standard TV-IO header, used for IIS Models 70 and 75
YGEN:YSLECT.FOR        turn gray and graphics planes on and off
YGEN:YTCOMP.FOR        decide if a parameter has changed
YGEN:YZERO.FOR         fill a TV memory plane with zeros
APLGEN:ZVTVO2.FOR      open connection in client (virtual-TV) to server (remote, real-TV)
\end{verbatim}
 
\subsection{UTILITY}
\begin{verbatim}
APLSUB:AP2SIZ.FOR      returns largest power of 2 not exceeding 1024 times first argument
APLSUB:BLDSNM.FOR      builds a name for a scratch file
APLSUB:BLDTNM.FOR      constructs full task name by appending NPOPS to task root name
APLSUB:BLTGLE.FOR      returns angle from A through a test position to B
APLSUB:BOUNDS.FOR      prints message if 1 or 2 values are outside a specified range
APLSUB:CATIME.FOR      stores current, or recovers previous, date and time in packed format
APLSUB:COMPAR.FOR      compares two integer arrays and returns .TRUE. if they are equal
APLSUB:COORDD.FOR      converts angles between degrees and sexagesimal format
APLSUB:COPY.FOR        copies integer words from one array to another
APLSUB:DAT2JD.FOR      converts date and time to a Julian date
APLSUB:DATDAT.FOR      converts "DD/MM/YY" form of date to "dd-mmm-yyyy" for printing
APLSUB:DTINIT.FOR      inits parameters for displaying elapsed CPU and real time w DTTIME
APLSUB:DTTIME.FOR      displays elapsed CPU and real times since last call to DTINIT
APLSUB:FILL.FOR        fills an integer array with an integer constant
APLSUB:FMATCH.FOR      returns pointer to location of small array in a bigger array
APLSUB:FRMT.FOR        encode floating number removing trailing zeros, alter accuracy if nec
APLSUB:GETRLS.FOR      returns the name of the current release (edited each quarter in CV)
APLSUB:GREG.FOR        converts Julian day number to date in character form
APLSUB:GTPARM.FOR      starts tasks, getting parameters and task ID number, does accounting
APLSUB:GTTELL.FOR      gets any parameters sent to task by AIPS verb TELL
APLSUB:H2CHR.FOR       convert AIPS Hollerith string to Fortran CHARACTER variable
APLSUB:IROUND.FOR      rounds a REAL to the nearest INTEGER
APLSUB:ITRIM.FOR       returns length of CHARACTER variable to last non-blank
APLSUB:JD2DAT.FOR      converts Julian day number to calendar date and time
APLSUB:JTRIM.FOR       clears nulls, returns length of CHARACTER variable to last non-blank
APLSUB:JULDAY.FOR      converts a character-encoded calendar date to Julian day number
APLSUB:LINTER.FOR      does linear interpolation of a 1-D INTEGER array
APLSUB:LSORT.FOR       sort a data buffer minimizing number times records are switched
APLSUB:METSCA.FOR      scale a value to the range 1-999 and provide a metric prefix to match
APLSUB:MSGWRT.FOR      writes messages to log file and/or terminal - a fundamental routine!
APLSUB:NAMEST.FOR      packs image name in string with leading and trailing blanks removed
APLSUB:NMATCC.FOR      returns next character in a string not matching a specified constant
APLSUB:NMATCH.FOR      returns next word in INTEGER array not matching a specified constant
APLSUB:RCOPY.FOR       copies one real array into another
APLSUB:REALOG.FOR      converts numbers between floating and an integer version of their log
APLSUB:RFILL.FOR       fills a real array with a constant
APLSUB:SETUP.FOR       does several task start up chores for non-interactive tasks
APLSUB:STRLIN.FOR      computes integer array as linear interpolation between two points
APLSUB:TIMDAT.FOR      convert integer time and date to character form for display
APLGEN:ZERROR.FOR      prints strings associated with system error codes for Z routines
APLGEN:ZMSGER.FOR      prints strings associated with system error codes for ZMSG routines
\end{verbatim}
 
\subsection{UV}
\begin{verbatim}
QNOT:ALGSUB.FOR        Interpolates model visibility grom a grid and subtracts from uv data.
APLNOT:CALREF.FOR      Adjusts the reference antenna in an SN table.
QNOT:CCSGRD.FOR        Transforms CLEAN components to a grid.
APLNOT:DGGET.FOR       Selects uv data and changes Stokes
APLNOT:DGHEAD.FOR      Fills output CATBLK for UVGET
APLNOT:DGINIT.FOR      Sets arrays for selecting data and changing Stokes
QNOT:FFTIM.FOR         FFTs an image for uv interpolation.
APLNOT:FQMATC.FOR      Check if selection criteria match FQ table entries.
APLNOT:GAININ.FOR      Initializes calibration table for application.
APLNOT:GET1VS.FOR      Extract desired uv data, 1 value per freq. channel.
APLNOT:GETANT.FOR      Reads AN table and stores the info in common.
APLNOT:GETCTL.FOR      Determine Stokes' type of Clean map and other modeling info.
APLNOT:GETFQ.FOR       Find info on a given frequency id.
APLNOT:GETSOU.FOR      Find info on a given source id.
APLNOT:GETSTN.FOR      Reads the VLB station list opened in VBLIN and VBCIT
APLNOT:GNFSMO.FOR      Boxcar smooths and ASCAL solution (GA) file.
APLNOT:GNSMO.FOR       Optimized spline smoothing of amplitudes in ASCAL (GN) file.
QNOT:GRDCRM.FOR        Loads CLEAN components into AP for uv model computation.
APLNOT:GRDFLT.FOR      Sets default gridding convolution functions.
APLNOT:GRDSET.FOR      Creates scratch files and sets up for GRDSUB
QNOT:GRDSUB.FOR        Subtracts transform of CLEAN components from uv data.
APLNOT:GRIDTB.FOR      Makes a gridded image of the UV data in TB order.
APLNOT:INDXIN.FOR      Initializes index (NX) file, finds first scan selected.
QNOT:INTPFN.FOR        Computes interpolation kernals and put them into "AP memory".
APLNOT:IOBSRC.FOR      Search for antennas in the current bandpass buffer.
APLNOT:LXYPOL.FOR      Fills polarization correction table for AT like linear polarization.
QNOT:MAKMAP.FOR        Makes image or beam from uv data set.
APLNOT:MULSDB.FOR      Determines if a uv file is multi- or single- source.
APLNOT:NDXINI.FOR      Create/open/init index (NX) table
APLNOT:NXTFLG.FOR      Manages flagging info in tables in common.
APLNOT:PARANG.FOR      Computes antenna parallactic angles
APLNOT:POLSET.FOR      Fills polarization correction table from info in AN table.
QPSAP:Q1FIN.FOR        Finish gridding a row of uv data.
QPSAP:Q1GRD.FOR        Grid a uv data.
QPSAP:QFINGR.FOR       Finish gridding row of uv data.
QPSAP:QGADIV.FOR       Divide Gaus. model vis. into uv data.
QPSAP:QGASUB.FOR       Subtract Gaus. model vis. from uv data.
QPSAP:QGRD1.FOR        Convolves visibility data onto a grid.
QPSAP:QGRD2.FOR        Convolves linear polarization data onto a grid.
QPSAP:QGRD3.FOR        Convolve visibility data onto a grid.
QPSAP:QGRD4.FOR        Convolves visibility data onto a grid.
QPSAP:QGRDFI.FOR       Finish griding a row of uv data.
QPSAP:QGRDMI.FOR       Combined complex vector in gridding uv data.
QPSAP:QGRID.FOR        Grid uv data into row.
QPSAP:QGRIDA.FOR       Grid visibility data.
QPSAP:QINT.FOR         Interpolates model visibilityes from a grid.
QPSAP:QINTP.FOR        Interpolates model visibilities from a grid.
QPSAP:QMCALC.FOR       Compute model visibility from point model.
QPSAP:QPTDIV.FOR       Divide point model visibility into uv data.
QPSAP:QPTFAZ.FOR       zCompute phase in model visibilities.
QPSAP:QPTSUB.FOR       Subtract point model visibility from uv data.
QPSAP:QSPDIV.FOR       Divide Gaussian model  visibility into uv data.
QPSAP:QSPSUB.FOR       Subtract Gaussian model visibility from uv data.
QPSAP:QUVIN.FOR        Interpolate visibility model from a grid.
QPSAP:QUVINT.FOR       Interpolate model visibility from grid.
QPSAP:QXXPTS.FOR       Subtract point model visibility from uv data.
APLNOT:SCINTP.FOR      Interpolates bandpass tables in time.
APLNOT:SCLOAD.FOR      Copies part of one bandpass scratch file to another for efficiency.
APLNOT:SELINI.FOR      Initialize data selection and control in commons in DSEL.INC
APLNOT:SELSMG.FOR      Selects calibrator data, smooths solutions.
APLNOT:SET1VS.FOR      Sets up pointer and weights arrays for selecting uv data.
APLNOT:SETGDS.FOR      Sets up for UV model computation, fills common in DGDS.INC
APLNOT:SETGRD.FOR      Sets up for gridding uv data.
APLNOT:SETSM.FOR       Determines type of spectral smoothing and sets up look up table.
APLNOT:SETSTK.FOR      Sets STOKES parameters correctly for plotting routines
APLNOT:SMOSP.FOR       Convolves a spectrum with a tabulated function.
APLNOT:SN2CL.FOR       Apply an SN to a CL table.
APLNOT:SNAPP.FOR       Append SN tables and keep track of reference antennas.
APLNOT:SNINI.FOR       Create/open/initialize solution (SN) tables.
APLNOT:SNSMO.FOR       Smooths solution (SN) tables
APLNOT:SOUELV.FOR      Computes source hour angles and elevations
APLNOT:SOUFIL.FOR      Fills in arrays of source numbers to be included or excluded.
APLNOT:SOURNU.FOR      Look up source numbers for a list of names.
APLNOT:TABBL.FOR       Do IO to Baseline (BL) table after setup by BLINI.
APLNOT:TABBP.FOR       Does I/O to bandpass (BP) table opened by BPINI
APLNOT:TABCAL.FOR      Does I/O to Calibration (CL) table opened by CALINI
APLNOT:TABCS.FOR       Does I/O to single dish calibration (CS) table opened by CSINI
APLNOT:TABFLG.FOR      Does I/O to Flag (FG) table opened by FLGINI
APLNOT:TABFQ.FOR       Does I/O to frequency (FQ) table opened by FQINI
APLNOT:TABGA.FOR       Does I/O to GAIN (GA) table opened by GAINI
APLNOT:TABNDX.FOR      Does I/O to Index (NX) table opened by NDXINI
APLNOT:TABSN.FOR       Does I/O to Solution (SN) table opened by SNINI
APLNOT:TABSOU.FOR      Does I/O to Source (SU) table opened by SOUINI
APLNOT:TABTY.FOR       Does I/O to Tsys (TY) table opened by TYINI
APLNOT:TYINI.FOR       Create/open/initialize Tsys (TY) table
APLNOT:UVDOUT.FOR      Divides uv model in one half of a record into other, writes result.
APLNOT:UVDPAD.FOR      Reformat UV data record, doubling size and zero extra words.
APLNOT:UVGET.FOR       Read UV data with optional calibration, editing, selection, etc.
QNOT:UVGRID.FOR        Grids uv data to be FFTed.
QNOT:UVMDIV.FOR        Divides a uv data set by the Fourier transform of a model.
QNOT:UVMSUB.FOR        Subtracts the Fourier transform of a model from a uv data set.
QNOT:UVTBGD.FOR        Grids uv data in arbitrary sort order to be FFTed.
QNOT:UVTBUN.FOR        Determines and applies uniform weighting to uv data in arb. order.
QNOT:UVUNIF.FOR        Determines and applies uniform weighting to a uv data set.
APLNOT:VISCNT.FOR      Determines number of visibility records requested of UVGET
QNOT:VISDFT.FOR        Compute DFT of model and subtract/divide from/into uv data.
\end{verbatim}
 
\subsection{UV-UTIL}
\begin{verbatim}
APLNOT:AN10RS.FOR      Determines a list of antenna pairs from adverbs ANTANNA, BASELINE
APLSUB:ANTDAT.FOR      Returns the reference date and frequency for each array in uv dataset
APLNOT:CALCOP.FOR      Copies selected uv data with calibration and editing
APLSUB:GETVIS.FOR      uses setup from SETVIS to get and reformat a visibility sample
APLSUB:MERGE.FOR       sorts by merging previously sorted blocks of records
APLSUB:OSORT.FOR       does quick sort on array of vectors, then reorders by calling PERMAT
APLSUB:REQBAS.FOR      Apply ANTENNA and BASELINE selection adverbs to a baseline
APLSUB:SETVIS.FOR      initializes pointers to select/convert uv data to desired form
APLSUB:SHSORT.FOR      Shell sort of an array or records on two keys
APLSUB:UVCREA.FOR      create and catalog a uv data base file
APLSUB:UVDISK.FOR      reads/writes records of arbitrary length, esp UV data, see UVINIT
APLSUB:UVINIT.FOR      initializes IO for arbitrary length records via UVDISK, esp UV data
APLSUB:UVPGET.FOR      determines pointers to UV data from the header
AIPSUB:UWRITE.FOR      writes summary of UV Export-format tape
APLSUB:VISCHK.FOR      checks if UV data sample is desired, returns it in RR, LL, RL, LR
\end{verbatim}
 
\subsection{VLA}
\begin{verbatim}
APLNOT:APLPBI.FOR      *TESS routine to apply a taper to an image. VLA only!
APLNOT:APPLPB.FOR      *TESS routine to apply a taper to an image
\end{verbatim}
 
\subsection{Y0}
\begin{verbatim}
YGEN:YCINIT.FOR        initialize image catalog for specified TV memory plane
YGEN:YCNECT.FOR        write line segment between 2 points on TV
YGEN:YCOVER.FOR        checks for overlapped images on the TV by quadrant
YGEN:YCREAD.FOR        read the image catalog, return image header for TV only
YGEN:YCUCOR.FOR        correct cursor position for scroll; return image coordinates, header
YGEN:YCURSE.FOR        read and control TV cursor
YGEN:YCWRIT.FOR        write image header to image catalog, update image catalog directory
YGEN:YFILL.FOR         fill rectangle of TV memory with a constant value
YGEN:YFIND.FOR         determines the unique TV image of desired type, returns catalog block
YGEN:YLOCAT.FOR        return TV positions for set of image positions
YGEN:YLOWON.FOR        select least on bit in a bit mask integer
YGEN:YSLECT.FOR        turn gray and graphics planes on and off
YGEN:YTCOMP.FOR        decide if a parameter has changed
\end{verbatim}
 
\subsection{Y1}
\begin{verbatim}
YGEN:YCRCTL.FOR        controls the TV cursor visibility, position; reads trackball buttons
YGEN:YGRAPH.FOR        turns TV graphics planes on and off
YGEN:YIMGIO.FOR        read/write data to the TV grey and graphics memories
YGEN:YINIT.FOR         initialize everything about the TV
YGEN:YLUT.FOR          read/write channel-based look-up-table
YGEN:YOFM.FOR          read/write all-channel look-up-table ("output function memory")
YGEN:YSCROL.FOR        write the scroll registers (shift location of 1 or more TV channels)
YGEN:YSPLIT.FOR        set channel selection by split-screen quadrant
YGEN:YTVCIN.FOR        initialize TV characteristics common (not needed much - see TVOPEN)
YGEN:YTVCLS.FOR        close the TV, including TV device and TV control/parameter disk file
YGEN:YTVMC.FOR         issue a master clear to reinitialize IO to the TV
YGEN:YTVOPN.FOR        open the TV device and the TV disk control/parameter file.
YGEN:YZERO.FOR         fill a TV memory plane with zeros
YGEN:YZOOMC.FOR        set the TV zoom magnification and center
\end{verbatim}
 
\subsection{Y2}
\begin{verbatim}
YGEN:YALUCT.FOR        drives the TV arithmetic logic unit - not to be used much
YGEN:YCONST.FOR        controls the constant registers added to the TV picture - not used
YGEN:YFDBCK.FOR        causes a feedback operation in the TV
YGEN:YIFM.FOR          read/write TV Input look-up-table
YGEN:YMNMAX.FOR        read 3 min/max values from TV data paths (IIS only, not used)
YGEN:YRHIST.FOR        read the histogram of the selected TV output color
YGEN:YSHIFT.FOR        read/write the shift (bias) registers of the TV (IIS M70, not used)
\end{verbatim}
 
\subsection{Y3}
\begin{verbatim}
YGEN:YGGRAM.FOR        controls the TV graphics color assignments
YGEN:YGRAFE.FOR        controls the graphics control register (IIS function)
YGEN:YGYHDR.FOR        builds basic TV IO header to write gray scale data
YGEN:YISDRM.FOR        read/write data memory of NRAO-ISU device
YGEN:YISDSC.FOR        read/write micro-processor memory of NRAO-ISU device
YGEN:YISJMP.FOR        cause microprocessor jump toaddress in NRAO-ISU device
YGEN:YISLOD.FOR        loads/unloads program memory of NRAO-ISU device
YGEN:YISMPM.FOR        reads/writes microprocessor memory of the NRAO-ISU device
YGEN:YMAGIC.FOR        initialize graphics, zoom, scroll units for IIS Model 75 (level 3)
YGEN:YMKCUR.FOR        selects the form of the cursor to be displayed
YGEN:YMKHDR.FOR        builds standard TV-IO header, used for IIS Models 70 and 75
YGEN:YSTCUR.FOR        reads/writes the cursor pattern array
YGEN:YTVCL2.FOR        close actual TV device (called by YTVCLS)
YGEN:YTVOP2.FOR        open actual TV device (called by YTVOPN)
\end{verbatim}
 
\subsection{Z}
\begin{verbatim}
APLGEN:ZABORT.FOR      establishes or carries out (when appropriate) abort handling
APLGEN:ZACTV8.FOR      activate the requested program, returning process ID information
APLGEN:ZADDR.FOR       determine if 2 addresses inside computer are the same
APLGEN:ZARGCL.FOR      close an ARGS TV device
APLGEN:ZARGMC.FOR      issues a master clear to an ARGS TV
APLGEN:ZARGOP.FOR      open ARGS TV device
APLGEN:ZARGXF.FOR      translates IIS Model 70 commands into calls to ZARGS for ARGS TV
APLGEN:ZBKLD1.FOR      initialize environment for BAKLD
APLGEN:ZBKLD2.FOR      does BACKUP operation: load images from tape to directory
APLGEN:ZBKLD3.FOR      clean up system things for BAKLD ending
APLGEN:ZBKTP1.FOR      initialize BACKUP to tape operation for BAKTP
APLGEN:ZBKTP2.FOR      write a cataloged file plus extensions to BACKUP tape in BAKTP
APLGEN:ZBKTP3.FOR      clean up host environment at end of BAKTP
APLGEN:ZBYMOV.FOR      move 8-bit bytes from in-buffer to out-buffer
APLGEN:ZBYTFL.FOR      interchange bytes in buffer if needed to go between local & standard
APLGEN:ZC8CL.FOR       convert packed ASCII buffer to local character string
APLGEN:ZCLC8.FOR       convert local character string to packed ASCII buffer
APLGEN:ZCLOSE.FOR      closes open devices: disk, line printer, terminal
APLGEN:ZCMPRS.FOR      release space from the end of an open disk file
APLGEN:ZCPU.FOR        return current process CPU time and IO count
APLGEN:ZCREAT.FOR      creates a disk file
APLGEN:ZDATE.FOR       return the local date
APLGEN:ZDCHIN.FOR      initialize message, device and Z-routine characteristics commons
APLGEN:ZDEACL.FOR      close DeAnza TV device
APLGEN:ZDEAOP.FOR      opens DeAnza TV device
APLGEN:ZDEAXF.FOR      do IO to DeAnza TV
APLGEN:ZDELAY.FOR      delay current process a specified interval
APLGEN:ZDESTR.FOR      destroy a closed disk file
APLGEN:ZDHPRL.FOR      convert 64-bit HP floating buffer to local DOUBLE PRECISION values
APLGEN:ZDM2DL.FOR      convert Modcomp REAL*6 and REAL*8 to local double precision
APLGEN:ZENDPG.FOR      advance printer if needed to avoid electrostatic-printer "burn-out"
APLGEN:ZERROR.FOR      prints strings associated with system error codes for Z routines
APLGEN:ZEXIST.FOR      return file size and, consequently, whether file exists
APLGEN:ZEXPND.FOR      expand an open disk file --- either map or non-map now allowed
APLGEN:ZFIO.FOR        reads and writes single 256-integer records to non-map disk files
APLGEN:ZFREE.FOR       display available disk space
APLGEN:ZGETCH.FOR      get a character from a REAL word
APLGEN:ZGNAME.FOR      get name of current process
APLGEN:ZGTBIT.FOR      get array of bits from a word
APLGEN:ZHEX.FOR        encode an integer into hexadecimal characters
APLGEN:ZIPACK.FOR      pack/unpack long integers into short integer buffer
APLGEN:ZKDUMP.FOR      display portions of an array in various Fortran formats
APLGEN:ZLASIO.FOR      open, write to, close and spool a laser printer print/plot file
APLGEN:ZLPCLS.FOR      close an open printer device
APLGEN:ZLPOPN.FOR      open a line-printer text file
APLGEN:ZLWIO.FOR       open, write to, close and spool a PostScript print/plot file
APLGEN:ZM70MC.FOR      issues a master clear to an IIS Model 70 TV
APLGEN:ZM70OP.FOR      open IIS Model 70 TV device
APLGEN:ZM70XF.FOR      read/write data to IIS Model 70 TV with buffering
APLGEN:ZMIO.FOR        random-access, quick return (double buffer) disk IO for large blocks
APLGEN:ZMKTMP.FOR      convert a "temporary" file name into a unique name
APLGEN:ZMOUNT.FOR      mount or dismount magnetic tape device
APLGEN:ZMSGCL.FOR      close Message file or terminal
APLGEN:ZMSGDK.FOR      disk IO to message file
APLGEN:ZMSGER.FOR      prints strings associated with system error codes for ZMSG routines
APLGEN:ZMSGOP.FOR      open a message file or message terminal
APLGEN:ZMSGXP.FOR      expand the message file
APLGEN:ZOPEN.FOR       open binary disk files and line printer and TTY devices
APLGEN:ZPHFIL.FOR      construct a physical file or device name from AIPS logical parameters
APLGEN:ZPHOLV.FOR      construct a physical file - version for UPDAT
APLGEN:ZPRIO.FOR       raise or lower the process priority
APLGEN:ZPRMPT.FOR      prompt user and read 80-characters from CRT screen
APLGEN:ZPRPAS.FOR      prompt user and read 12-character password (invisible) from CRT
APLGEN:ZPTBIT.FOR      put array of bits into a word
APLGEN:ZPUTCH.FOR      inserts 8-bit "character" into a word
APLGEN:ZRDMF.FOR       convert DEC Magtape Format (36 bits data in 40 bits) to 2 integers
APLGEN:ZRENAM.FOR      rename a disk file
APLGEN:ZRLR64.FOR      convert buffer of local double precision values to IEEE 64-bit float.
APLGEN:ZRM2RL.FOR      convert Modcomp to local single precision floating point
APLGEN:ZSTAIP.FOR      does any system cleanup needed at the end of interactive AIPS session
APLGEN:ZTACTQ.FOR      inquires if a task is currently active on the local computer
APLGEN:ZTAPE.FOR       mount, dismount, position, write EOF, etc. for tapes
APLGEN:ZTAPIO.FOR      tape operations for IMPFIT (compressed FITS transport tape)
APLGEN:ZTIME.FOR       return the local time of day
APLGEN:ZTOPEN.FOR      open text file - logical area, version, member name as arguments
APLGEN:ZTPWAT.FOR      wait for asynchronous IO to finish on tape or pseudo-tape disk
APLGEN:ZTQSPY.FOR      display AIPS account or all processes running on the system
APLGEN:ZTTBUF.FOR      reads terminal input with no prompt or wait - simulates TV trackball
APLGEN:ZTXCLS.FOR      clos text file opened via ZTXOPN
APLGEN:ZTXIO.FOR       read/write a line to a text file
APLGEN:ZTXMAT.FOR      return list of files in specified area beginning with specified chars
APLGEN:ZTXOPN.FOR      open a text file for read or write
APLGEN:ZV20CL.FOR      close a Comtal Vision 1/20 TV device
APLGEN:ZV20MC.FOR      issue a master clear to the TV - for Comtal this is a No-Op
APLGEN:ZV20OP.FOR      open Comtal Vision 1/20 TV device
APLGEN:ZV20XF.FOR      read/write data to Comtal Vision 1/20 TV device
APLGEN:ZWHOMI.FOR      determines AIPSxn task name; sets NPOPS, assigns TV and TK devices
\end{verbatim}
 
\subsection{Z2}
\begin{verbatim}
APLGEN:ZABOR2.FOR      establishes or carries out (when appropriate) abort handling
APLGEN:ZARGC2.FOR      close an ARGS TV device
APLGEN:ZARGO2.FOR      open ARGS TV device
APLGEN:ZARGS.FOR       sends command to/from the ARGS TV device
APLGEN:ZBYTF2.FOR      interchange bytes in buffer if needed to go between local & standard
APLGEN:ZCMPR2.FOR      truncate a disk file, returning blocks to the system
APLGEN:ZCREA2.FOR      create the specified disk file
APLGEN:ZDACLS.FOR      close a disk file
APLGEN:ZDAOPN.FOR      open the specified disk file
APLGEN:ZDCHI2.FOR      initialize device and Z-routine characteristics commons - local vals
APLGEN:ZDCHIC.FOR      set more system parameters; make them available to C routines
APLGEN:ZDEAC2.FOR      close DeAnza TV device
APLGEN:ZDEAO2.FOR      opens DeAnza TV device
APLGEN:ZDELA2.FOR      delay current process a specified interval
APLGEN:ZDEST2.FOR      destroy a closed disk file
APLGEN:ZDIR.FOR        build a full path name to files in AIPS-standard areas (HE, RU, ...)
APLGEN:ZERRO2.FOR      return system error message for given system error code
APLGEN:ZEXIS2.FOR      return size of disk file and if  it exists
APLGEN:ZEXPN2.FOR      expand an open disk file
APLGEN:ZFI2.FOR        read/write one 256-integer record from/to a non-map disk file
APLGEN:ZFRE2.FOR       return AIPS data disk free space information
APLGEN:ZLASC2.FOR      spool a closed laser printer print/plot file
APLGEN:ZLASCL.FOR      close and spool a laser printer print/plot file
APLGEN:ZLASOP.FOR      open a laser printer print/plot file
APLGEN:ZLPCL2.FOR      queue a file to the line printer and delete
APLGEN:ZLPOP2.FOR      open a line-printer text file - actual OPEN call
APLGEN:ZM70C2.FOR      close IIS Model 70/75 TV device
APLGEN:ZM70O2.FOR      opens IIS Model 70.75 TV device
APLGEN:ZMI2.FOR        read/write large blocks of data from/to disk, quick return
APLGEN:ZMOUN2.FOR      mount or dismount magnetic tape device
APLGEN:ZMSGWR.FOR      call MSGWRT based on call arguments - for C routines to call MSGWRT
APLGEN:ZPATH.FOR       convert a file name
APLGEN:ZPRI2.FOR       raise or lower the process priority
APLGEN:ZRENA2.FOR      rename a file
APLGEN:ZSTAI2.FOR      does any system cleanup needed at the end of interactive AIPS session
APLGEN:ZTACT2.FOR      inquires if a task is currently active on the local computer
APLGEN:ZTAP2.FOR       position (forward/back record/file), write EOF, etc. for tapes
APLGEN:ZTKCL2.FOR      close a Tektronix device
APLGEN:ZTKOP2.FOR      read/write from/to a Tektronix device
APLGEN:ZTOPE2.FOR      open text file for ZTOPEN
APLGEN:ZTPCL2.FOR      close a tape device
APLGEN:ZTPMI2.FOR      tape read/write
APLGEN:ZTPMID.FOR      pseudo-tape disk read/write for 2880-bytes records
APLGEN:ZTPOP2.FOR      open a tape device for double-buffer, asymchronous IO
APLGEN:ZTPOPD.FOR      open a pseudo-tape, sequential disk file for FITS
APLGEN:ZTPWA2.FOR      wait for read/write from/to a tape device
APLGEN:ZTQSP2.FOR      display AIPS account or all processes running on the system
APLGEN:ZTTCLS.FOR      close a terminal device
APLGEN:ZTTOP2.FOR      open a message terminal
APLGEN:ZTTOPN.FOR      open a terminal device
APLGEN:ZTXMA2.FOR      find all file names matching a given wildcard specification
APLGEN:ZTXOP2.FOR      translate the file name and open a text file
APLGEN:ZV20C2.FOR      close Comtal Vision 1/20 TV device
APLGEN:ZV20O2.FOR      opens Comtal Vision 1/20 TV device
APLGEN:ZV20X2.FOR      does I/O to Comtal Vision 1/20 TV device
APLGEN:ZWAI2.FOR       wait for read/write large blocks of data from/to disk
\end{verbatim}
