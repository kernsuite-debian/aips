%-----------------------------------------------------------------------
%;  Copyright (C) 2013-2014, 2016
%;  Associated Universities, Inc. Washington DC, USA.
%;
%;  This program is free software; you can redistribute it and/or
%;  modify it under the terms of the GNU General Public License as
%;  published by the Free Software Foundation; either version 2 of
%;  the License, or (at your option) any later version.
%;
%;  This program is distributed in the hope that it will be useful,
%;  but WITHOUT ANY WARRANTY; without even the implied warranty of
%;  MERCHANTABILITY or FITNESS FOR A PARTICULAR PURPOSE.  See the
%;  GNU General Public License for more details.
%;
%;  You should have received a copy of the GNU General Public
%;  License along with this program; if not, write to the Free
%;  Software Foundation, Inc., 675 Massachusetts Ave, Cambridge,
%;  MA 02139, USA.
%;
%;  Correspondence concerning AIPS should be addressed as follows:
%;          Internet email: aipsmail@nrao.edu.
%;          Postal address: AIPS Project Office
%;                          National Radio Astronomy Observatory
%;                          520 Edgemont Road
%;                          Charlottesville, VA 22903-2475 USA
%-----------------------------------------------------------------------
\setcounter{chapter}{11}
\APPEN{Handling EVLA P-band Data in \AIPS}{Special Considerations for EVLA
      P-band Data Processing in \AIPS}{EVLAlowband}
\renewcommand{\Chapt}{26}

\renewcommand{\titlea}{31-December-2015 (revised 3-February-2016)}
\renewcommand{\Rheading}{\AIPS\ \cookbook:~\titlea\hfill}
\renewcommand{\Lheading}{\hfill \AIPS\ \cookbook:~\titlea}
\markboth{\Lheading}{\Rheading}

The \Indx{EVLA} has been equipped with new, wide-band receivers in the
range 230 MHz to 470 MHz, known as P-band.  Most of the data reduction
and imaging at this band are similar to those at other bands, so the
following will assume some degree of familiarity with the rest of the
\Cookbook, particularly \Rappen{EVLAdata}\@.  This guide was initially
prepared by Minnie Mao from an earlier guide written by Susan Neff.
Enough has changed with the P-band system that a thorough revision of
the Appendix has been done.

{\large \bf Note that the choice of file names, disk assignments,
catalog numbers, and the like depends on your circumstances and will
almost certainly not be the same as those shown in the examples
below.  Note also that experienced (but still reasonable) EVLA users
disagree on the details of data reduction.  Some compromise in these
details has been attempted below.}

\sects{P-band calibration and editing in \AIPS}

\begin{enumerate}
\item\ Download your data set from the archive found at\\
\hphantom{MMMMMMM}{\tt https://archive.nrao.edu/archive/ArchiveQuery}\\
making sure that you select SDM-BDF data set (all files) as the
download data format.

\item\ Before starting {\tt AIPS} and while in the terminal window you
  intend to use for {\tt AIPS}, enter
\dispi{cd {\it my\_data\_directory} \CR}{to change to the data
             directory containing the top-level directory of your
             data set.}
\dispi{export MYAREA=`pwd` \CR}{for bash shells, or}
\dispi{setenv MYAREA `pwd` \CR}{for tcsh and c shells.}
\dispe{Note the back ticks around {\tt pwd} which cause the output of
{\tt pwd} to appear in the typed line.}

\item\ Start {\tt AIPS} with, for example,
\dispi{aips tv=local pr=3 \CR}{to use Unix sockets for the TV and
       printer 3}
\dispe{and enter your user number when prompted.}

\item\ Look at the SDM-BDF file contents with {\tt BDFLIST} --- make
  sure you do {\it not} have a quote at the end of the file name, or
  else the lower case characters get changed into upper case and the
  file won't be found.\todx{BDFLIST}
\dispi{default bdflist \CR}{initialize all adverbs}
\dispi{dowait 1; docrt 1 \CR}{to get output on terminal as well as log
       file.}
\dispi{asdmfile(1) = 'MYAREA: \CR}{two parts are concatenated.}
\dispil{asdmfile(2) = 'TSUB0001.sb23643640.eb23666175.56454.14034020834
       \CR}{}
\dispi{bdflist \CR}{to examine file contents.}
\dispe{Pay particular attention to the configuration numbers.  These
  refer to each unique correlator {\it and} antenna setup, not just
  the size of the current VLA\@.  Each configuration that you want
  will need to be loaded individually.}
\vfill\eject

\item\ Load data into \AIPS\ with {\tt BDF2AIPS} --- I'm doing
  everything on disk $d$ because that's where there's more room.
  Note, do not use {\tt DEFAULT} to retain the adverbs from the
  previous step.  Loading data may take a while depending on the size
  of your data.\todx{BDF2AIPS}\iodx{EVLA}
\dispi{outn 'my-name' \CR}{to select a meaningful file name.}
\dispi{config $c$ \CR}{to select configuration $c$.}
\dispi{outdi $d$ \CR}{to select the disk.}
\dispi{bdf2aips \CR}{to translate the data into \AIPS.}
\dispe{Do this as often as needed to load all desired configurations.}

\item\ Look at header of the \AIPS\ file written by {\tt BDF2AIPS}
\dispi{indi $d$; getn 1 \CR}{to select disk and catalog number,
       assuming you had no files on disk $d$ previously.}
\dispi{imh \CR}{to see the details of the $uv$-data set header.}

\item\ look at where the telescopes were during your observation with
     {\tt \tndx{PRTAN}}
\dispi{default prtan \CR}{ }
\dispi{indi $d$; getn 1 \CR}{ }
\dispi{docrt = -1 \CR}{to make a printer listing to keep at your
       terminal.}
\dispi{go prtan \CR}{ }

\item\ Look at the observation summary with {\tt \tndx{LISTR}}
\dispi{default listr \CR}{ }
\dispi{indi $d$; getn 1 \CR}{ }
\dispi{opty 'scan' \CR}{to list scans and sources.}
\dispi{docrt = -1 \CR}{to make a printer listing to keep at your
       terminal.}
\dispi{go \CR}{ }

At this stage, Minnie's summary recommends separating P-band from
4-band data.  At present, only P-band is recorded so that step is not
needed.  Then, Minnie recommends using {\tt POSSM} to find dead
antennas plus those with crossed polarization.  Then she recommends
{\tt UVFLG} to flag the dead ones and {\tt FIXRL} to uncross the
mis-wired ones.  These problems have been essentially eliminated.
In addition, P-band data are now delivered by default in a number of
relatively narrow spectral windows (called IFs in \AIPS) which helps
minimize data lost to RFI and eliminates the {\tt NOIFS} and {\tt
  MORIF} steps from Minnie's guide.

\item\ However, the requantizer gains are now allowed to vary and must
  be corrected.\todx{TYAPL}
\dispi{default tyapl \CR}{to select the needed task}
\dispi{indi $d$; getn 1 \CR}{to select the input file.}
\dispi{inext 'SY' ; optype 'PGN' \CR}{to select requantizer gain
       correction.}
\dispi{inext 'SY' ; in2ver = 1 \CR}{to apply the SysPower table, the
       version number is not defaulted.}
\dispi{outdis $d$ ; go \CR}{to stay on one disk.}

\item\ Narrow-band RFI will normally cause ringing in P-band data.
  Use {\tt POSSM} to examine some baselines from a calibration source
  to check this if you want to be careful.  The default setup delivers
  data from outer IFs for which the sensitivity of the P-band receiver
  is poor.  Use \todx{SPLAT} to drop the insensitive IFs and to do
  Hanning smoothing to reduce the ringing.
\vfill\eject
\dispi{default splat \CR}{to select the needed task}
\dispi{indi $d$; getn 2 \CR}{to select the (new) input file.}
\dispi{flagver 1 \CR}{to apply the on-line flags.}
\dispi{smooth = 1, 0 \CR}{to do Hanning smoothing once and for all.}
\dispi{bif = 3; eif = 15 \CR}{to select the more sensitive IFs.}
\dispi{outdis $d$ ; go \CR}{to put the output on the same disk.}

\item\ Edit UV data with {\tt \tndx{EDITA}} and the {\tt SY} table.
  {\it The {\tt SY} table $P_{\rm sum}$ parameter is sensitive to the
    occasional bursts of RFI which render the data useless.  Flag all
    seriously discrepant points but do not flag really tightly since
    some noise in this parameter is normal and ``correct.''}
\dispi{default edita \CR}{ }
\dispi{indi $d$; getn 3 \CR}{ }
\dispi{inext 'SY' \CR}{ }
\dispi{go \CR}{ }

\item\ At this time, it is best to identify those spectral channels
  affected by RFI throughout the run and to flag them once and for
  all.  {\tt \tndx{FTFLG}} will display the data combining all
  included baselines in a single display (per polarization), while the
  very similar {\tt SPFLG} allows more discrimination (and more work)
  by   displaying the baselines separately.  Be sure to flag all
  sources if you flag channels over all times.\iodx{EVLA}
\dispi{default ftflg \CR}{ }
\dispi{indi $d$; getn 3 \CR}{ }
\dispi{calcode '*' \CR}{to examine calibrators only}
\dispi{dparm(6) = $x$ \CR}{to specify the basic integration time in
       the data in seconds.}
\dispi{go \CR}{ }
\dispe{If you use {\tt SPFLG}, you should limit the times, sources,
       and/or baselines examined.  The flagging then is best done with
       {\tt \tndx{UVFLG}} from notes made while you study the data
       with {\tt SPFLG}\@.}
\dispi{default uvflg \CR}{}
\dispi{indi $d$; getn 3 \CR}{ }
\dispi{bif = $bi$ ; eif = $ei$ \CR}{to set range of spectral windows.}
\dispi{bchan = $bc$ ; echan = $ec$ \CR}{to set range of spectral
       channels.}
\dispi{antennas = $a1, a2, a3$ \CR}{to select antennas to flag.}
\dispi{baseline = $b1, b2, b3, b4$ \CR}{to limit flags to the
       baselines from the pairs $a1$ through $a3$ with $b1$ through
       $b4$.}
 \dispi{reason 'bad channels' \CR}{to set a reason in the flag
       table.}
\dispi{go \CR}{and repeat as needed with different IFs, channels,
       antennas, etc.}

\item\ Calibrate for the delay errors with {\tt \tndx{FRING}}\@.
  Choose a time range in the middle of the observation where you know
  the antennas have settled down.  A short time range is all you need
  --- look at {\tt LISTR} --- because you don't want time variations
  to come into play.  This is also why you should choose something
  bright --- like 3C286, which is used here as the name for the
  generic fringe- and baseline-fitting calibration source..
\dispi{default fring \CR}{ }
\dispi{indi $d$; getn 3 \CR}{ }
\dispi{calso '3C286' ' \CR}{to select the primary calibration source.}
\dispi{timerang 0 3 40 0 0 3 40 55 \CR}{to select $< 1$ minute of good
       data from the calibration source.}
\dispi{solint 1 \CR}{to average over all included times.}
\dispi{aparm(5) 1 ; aparm(6) 1 \CR}{to fit a single delay over all IFs
       and to print some extra detail in the fitting.}
\dispi{dparm(9) 1 \CR}{to avoid any fitting for rates --- important.}
\dispi{refant 22 \CR}{to set a reference antenna, if one is known to
       be better than others.}
\dispi{go \CR}{ }
\dispe{This print level will display the delays found, which should
       all be a few nano-seconds although larger (correct) delays
       sometimes arise.}

\item\ Apply these delay solutions to the calibration ({\tt CL}) table
with {\tt \tndx{CLCAL}}
\dispi{default clcal \CR}{ }
\dispi{indi $d$; getn 3 \CR}{ }
\dispi{calso '3C286','' \CR}{ }
\dispi{snver 1 \CR}{to specify the {\tt SN} table version written by
       {\tt FRING}\@.}
\dispi{go \CR}{to copy {\tt CL} table 1 to version 2 adding the delay
       correction, taken to be constant in time.}

\item\ You should flag out any bad data on your bandpass calibrator.
  Some users feel that a quick-and-dirty flagging is enough at this
  stage.  Use {\tt CLIP} with a carefully chosen, perhaps
  IF-dependent, cutoff level.  A more detailed editing is done by
  other users.  Here are options to use {\tt RFLAG} and {\tt SPFLG}.
  Maybe you'd like to {\tt \tndx{RFLAG}} first and then eyeball with
  {\tt SPFLG}\@.  The options are endless!  The first execution of
  {\tt RFLAG} determines flagging level parameters and makes plots on
  the TV for you to be sure that they are reasonable,  The second
  execution applies these levels and determines new levels in case you
  want to run {\tt RFLAG} even more
  times.\label{it:beginbpass}\iodx{EVLA}
\dispi{default rflag \CR}{ }
\dispi{indi $d$; getn 3 \CR}{ }
\dispi{sources(1) '3C286' \CR}{ }
\dispi{docal 1 \CR}{to correct delays and a-priori gains.}
\dispi{avgchan 11 \CR}{to set the width of the median-window filter
       run across the spectral channels in each IF\@}
\dispi{fparm = 3, $x$, -1, -1 \CR}{to do a small rolling time buffer,
       set the sample interval, and to have {\tt NOISE} and {\tt
         SCUTOFF} set IF-dependent cutoff levels.}
\dispi{fparm(13) = 1000 \CR}{to clip amplitudes at 1000 Jy.  There are
       many more {\tt FPARM} options you may choose also.}
\dispi{doplot 8 ; dotv 1 \CR}{to do the 2 most useful plots on the
       TV.}
\dispi{go \CR}{ }
\dispe{{\tt RFLAG} will return cutoff levels to the appropriate
adverbs.  Therefore, to apply the flags using the flagging levels
found above, do {\em not} do a {\tt tget rflag} which will reset the
levels to zero.  Instead, simply do}
\dispi{doplot = -8 \CR}{to apply the cutoff levels and determine new
       ones, plotting on the TV.}
\dispi{go \CR}{ }
\dispe{{\tt RFLAG} will make a new flag table each time it determines
flags ({\tt DOPLOT} $\leq 0$).  Check the results with}
\dispi{default spflg \CR}{ }
\dispi{indi $d$; getn 3 \CR}{ }
\dispi{sources(1)  '3C286' \CR}{to examine only the bandpass calibrator.}
\dispi{dparm(6) $x$ \CR}{to set sampling time in seconds}
\dispi{docal 1 \CR}{to correct delays and a-priori gains.}
\dispi{go \CR}{ }

It is at this point that experienced users differ significantly in
their approach.  If you have only one bandpass-calibration scan and the
ionosphere was reasonably behaved (phases nearly constant through that
scan), then you should run {\tt BPASS} on the calibration source in
place.  However, when one has multiple bandpass scans and/or a badly
behaved ionosphere, it may be best to split the calibration scans out
and self-calibrate them on a short time interval.  The addition of
multiple scans where the source has been phased up, but the remaining
RFI has not, will reduce the effect of that RFI substantially.  This
scheme is described next.

\item\ Split off your calibrator for bandpass calibration --- this is
  only necessary for self-calibrating the BP calibrator.
\dispi{default split \CR}{ }
\dispi{indi $d$; getn 3 \CR}{ }
\dispi{outdi indi ; source '3C286' ' \CR}{to select calibrator and stay
       on the same disk.}
\dispi{outcl 'bpspl' \CR}{to indicate the use for this file.}
\dispi{docalib 1 \CR}{to apply delay and a-priori calibration.}
\dispi{go \CR}{To make a single-source file.}

\item\ Self-cal on your bandpass cal with {\tt \tndx{CALIB}} to solve
  for wiggles in phase as a function of time.\iodx{EVLA}
\dispi{default calib \CR}{ }
\dispi{indi $d$; getn 4 \CR}{ }
\dispi{refant 22 \CR}{to stay with the chosen reference antenna.}
\dispi{solint $y$ \CR}{to average $n$ records in each solution
       interval, set $y = n x$. }
\dispi{solty 'L1R' ; solmode 'P' \CR}{to do phase-only solutions.}
\dispi{aparm(6) 1 \CR}{to get some diagnostic messages.}
\dispi{go \CR}{to write a new {\tt SN} table.}

Eyeball this new table with {\tt \tndx{SNPLT}} --- should be flat, not
all over the place.
\dispi{default snplt \CR}{ }
\dispi{indi $d$; getn 4 \CR}{ }
\dispi{inext 'sn' ; opty 'phas' \CR}{to plot the phases in the new table.}
\dispi{nplots 4 ; dotv 1 \CR}{to have 4 plots per TV page.}
\dispi{opco 'alif' ; do3col 1 \CR}{to combine all IFs in each colorful
       plot.}
\dispi{go \CR}{ }
\dispe{Or, if these plots are too crowded, turn off {\tt OPCODE} and
  {\tt DO3COL}.}

\item\ Use {\tt \tndx{BPASS}} to calibrate the bandpass shape ---
  corrects for wiggles in phase and amplitude as a function of
  frequency \label{it:endbpass}
\dispi{default bpass \CR}{ }
\dispi{indi $d$; getn 4 \CR}{ }
\dispi{docalib 1 \CR}{to apply the new {\tt SN} table phases.}
\dispi{refant 22 \CR}{ }
\dispi{solty 'l1r' ; solint = -1 \CR}{to average the entire time range.}
\dispi{bpassprm(5) = 1; bpassprm(10) = 3 \CR}{to normalize the
       solutions at the end rather than the data record-by-record.}
\dispi{go \CR}{To generate a new {\tt BP} table containing one record
per antenna.}

\item\ Eyeball the resulting bandpass ({\tt BP}) table --- {\tt
    APARM(8) = 2} plots the {\tt BP} table in {\tt \tndx{POSSM}}\@.
    If you see odd bandpass shapes or spikes in the bandpass, you may
    need to edit your data more extensively and repeat steps
    \ref{it:beginbpass} through \ref{it:endbpass}.
\dispi{tget possm \CR}{ }
\dispi{indi $d$; getn 4 \CR}{ }
\dispi{solint = -1 \CR}{ }
\dispi{aparm(8) 2 ; aparm(9) 1 \CR}{to plot all IFs together for each
       bandpass.}
\dispi{nplots 2 \CR}{2 per page is enough here.}
\dispi{go \CR}{ }

\item\ Copy the bandpass table back to your main data with {\tt
    \tndx{TACOP}}\@.
\dispi{default tacop \CR}{ }
\dispi{indi $d$; getn 4 \CR}{ }
\dispi{inext 'bp' \CR}{ }
\dispi{outdi indi ; geton 3 \CR}{to select output name parameters}
\dispi{go \CR}{ }

\item\ Since {\tt RFLAG} probably made a large number of flags, let us
  copy the data file with {\tt \tndx{UVCOP}} and apply the
  flags.\iodx{EVLA}
\dispi{default uvcop \CR}{ }
\dispi{indi $d$; getn 3 \CR}{ }
\dispi{outdi indi ; outn inna \CR}{to stay on $d$.}
\dispi{outcl 'fixcp' \CR}{with a meaningful class name}
\dispi{flagver $n$ \CR}{to apply the highest flag table ($n$); use
       {\tt IMHEAD} if unsure.  Version 0 means no flagging here.}
\dispi{go \CR}{ }

\item\ Use {\tt \tndx{SETJY}} to compute the flux for the primary flux
calibrator (3C286), storing it in the {\tt SU} table.
\dispi{default setjy \CR}{ }
\dispi{indi $d$; getn 5 \CR}{ }
\dispi{source '3C286' ' \CR}{specify all ``known'' calibrator sources.}
\dispi{opty 'calc' \CR}{to calculate and display fluxes from the
       frequencies plus tables of known sources.}
\dispi{go \CR}{}

\item\ Go through all of your calibration sources, one at a time, to
  check for RFI with {\tt \tndx{POSSM}} and, if present, to address it
  with {\tt SPFLG} and/or {\tt RFLAG}
\dispi{tget possm \CR}{ }
\dispi{source '${\rm calsrc}_i$' '  \CR}{to examine the $i^{\rm th}$
       calibration source.}
\dispi{indi $d$; getn 5 \CR}{ }
\dispi{docal 1 ; doband 1 \CR}{apply our hard-won calibration.}
\dispi{solint 0 \CR}{to look at the average over all time.}
\dispi{aparm(9) 1 \CR}{to plot all IFs together.}
\dispi{nplots 2 \CR}{2 per page is enough here.}
\dispi{go \CR}{ }
%\vfill\eject
\dispe{If RFI is present (it almost certainly will be), decide whether
  {\tt \tndx{SPFLG}} might be worth the work.  If so}
\dispi{default spflg \CR}{ }
\dispi{indi $d$; getn 5 \CR}{ }
\dispi{source '${\rm calsrc}_i$' '  \CR}{to examine the $i^{\rm th}$
       calibration source.}
\dispi{dparm(6) $x$ \CR}{ }
\dispi{docal 1 ; doband 1 \CR}{apply the calibration.}
\dispi{go \CR}{ }
\dispe{Two rounds or more of {\tt \tndx{RFLAG}} may also be needed.
  Check with {\tt POSSM} to be sure.}
\dispi{default rflag \CR}{ }
\dispi{indi $d$; getn 5 \CR}{ }
\dispi{source '${\rm calsrc}_i$' '  \CR}{to examine the $i^{\rm th}$
       calibration source.}
\dispi{docal 1 ; doband 1 \CR}{apply the calibration.}
\dispi{avgchan 11 \CR}{to set the width of the median-window filter
       run across the spectral channels in each IF\@}
\dispi{fparm = 3, $x$, -1, -1 \CR}{to do a small rolling time buffer,
       set the sample interval, and to have {\tt NOISE} and {\tt
         SCUTOFF} set IF-dependent cutoff levels.}
\dispi{fparm(13) = 1000 \CR}{to clip amplitudes at 1000 Jy.  There are
       many more {\tt FPARM} options you may choose also.}
\dispi{doplot 8 ; dotv 1 \CR}{to do the 2 most useful plots on the
       TV.}
\dispi{go \CR}{ }
\dispe{{\tt RFLAG} will return cutoff levels to the appropriate
adverbs.  Therefore, to apply the flags using the flagging levels
found above, do {\em not} do a {\tt tget rflag} which will reset the
levels to zero.  Instead, simply do}
\dispi{doplot = -8 \CR}{to apply the cutoff levels and determine new
       ones, plotting on the TV.}
\dispi{go \CR}{ }
\dispe{{\tt RFLAG} will make a new flag table each time it determines
flags ({\tt DOPLOT} $\leq 0$).}

\item\ Since {\tt RFLAG} probably made a large number of flags, let us
  copy the data file once again with {\tt \tndx{UVCOP}} and apply the
  flags. \iodx{EVLA}
\dispi{default uvcop \CR}{ }
\dispi{indi $d$; getn 5 \CR}{ }
\dispi{outdi indi ; outn inna \CR}{to stay on $d$.}
\dispi{outcl 'flaged' \CR}{with a meaningful class name}
\dispi{flagver $n$ \CR}{to apply the highest flag table ($n$); use
       {\tt IMHEAD} if unsure.  Version 0 means no flagging here.}
\dispi{go \CR}{ }

\item\ Since all flag tables $< n$ were already merged into table $n$
  which we have applied, delete all flag tables in the output file.
  \todx{EXTDEST}
\dispi{default extdest \CR}{}
\dispi{indi $d$; getn 6 \CR}{}
\dispi{inext 'fg' ; invers = -1 \CR}{to delete all remaining versions
       of flag table.}
\dispi{extdest \CR}{}

%\vfill\eject
\item\ Find the amplitude and phase calibration as a function of {\it
    time} using {\tt \tndx{CALIB}}
\dispi{default calib \CR}{ }
\dispi{indi $d$; getn 6 \CR}{ }
\dispi{calcode '*' \CR}{to select all calibration sources}
\dispi{docalib 1 ; doband 1 \CR}{to apply all existing calibration.}
\dispi{refant 22 \CR}{to keep our reference antenna.}
\dispi{solint 0 \CR}{One solution per calibrator scan is okay with
       good short-term phase stability.  Otherwise, set to the longest
       time in minutes over which phase is stable.}
\dispi{solty 'L1R' ; solmode 'a\&p' \CR}{to solve for amplitude and
       phase with an L1 method iterated robustly.}
\dispi{aparm(6) 1 \CR}{to print closure error statistics.}
\dispi{go \CR}{ }

\item\ The solution table ({\tt SN}) produced by the previous step
  will contain amplitude gains at multiple levels, correct ones for
  those sources with fluxes in the source table (3C286) and incorrect
  ones for those which {\tt CALIB} was forced to call 1 Jy.  To solve
  for the unknown fluxes of the secondary calibrators \todx{GETJY}
\dispi{default getjy \CR}{ }
\dispi{indi $d$; getn 6 \CR}{ }
\dispi{source '${\rm calsrc}_1$', '${\rm calsrc}_2$', '${\rm
    calsrc}_3$', $\ldots$ \CR}{to list the secondary calibrators.}
\dispi{calsour = '3c286' ' \CR}{to list the known calibrator(s).}
\dispi{snver = $n$ \CR}{to select only the {\tt SN} table just written
       by {\tt CCALIB}\@.  Use {\tt IMHEADER} to find the maximum
       version number $n$.}
\dispi{go \CR}{to solve for the secondary fluxes as functions of IF.}
\dispe{Check the values displayed carefully to make sure that they are
sensible.  {\tt SOUSP} offers tools to examine and adjust the fluxes
and {\tt SN} table $n$.}

\item\ Look at both amplitude and phase solutions with {\tt
    \tndx{SNPLT}}.\iodx{EVLA}
\dispi{tvini \CR}{ }
\dispi{default snplt \CR}{ }
\dispi{indi $d$; getn 6 \CR}{ }
\dispi{inext 'sn' ; opty 'amp' \CR}{to plot amplitudes.}
\dispi{nplots 4 ; dotv 1 \CR}{to plot 4 panels on each TV page.}
\dispi{opco 'alif' ; do3col 1 \CR}{to plot all IFs in each panel using
       color to distinguish them.}
\dispi{go \CR}{ }
\dispe{One can also plot IFs in separate panels or one at a time if
  these plots are too crowded.  Then}
\dispi{opty 'ph'; go \CR}{To check the phases.}

\item\ Apply the solutions to the calibration table with {\tt
    \tndx{CLCAL}}
\dispi{default clcal \CR}{ }
\dispi{indi $d$; getn 6 \CR}{ }
\dispi{calcode '*' \CR}{to select all calibration sources}
\dispi{sampty 'box' ; bparm 0.3, 0.3 \CR}{To smooth the {\tt SN} table
       if you used {\tt SOLINT} less than full scans.  Leave this out
       if you used scan averages.}
\dispi{refant 22 \CR}{ }
\dispi{go \CR}{ }

\item\ Use {\tt SNPLT} to check that the solutions have been
   interpolated between calibrators.
\dispi{tget snplt \CR}{ }
\dispi{inext 'cl' \CR}{ }
\dispi{go \CR}{ }

\item\ Make an image of 3C286 as a sanity check using {\tt \tndx{IMAGR}}
\dispi{default imagr \CR}{ }
\dispi{indi $d$; getn 6 ; outdi indi \CR}{ }
\dispi{docal 1 ; doband 1 \CR}{to apply the calibration.}
\dispi{source = '3c286' ' \CR}{to select only the primary calibrator.}
\dispi{bchan 5; echan = $n-4$ \CR}{to omit edge channels in each IF,
       where $n$ is the number of channels in an IF, now usually 128.}
\dispi{nchav = echan-bchan+1; chinc nchav \CR}{to average all spectral
       channels into the image.}
\dispi{cellsize 1.\CR}{to set the image cell spacing --- depends on
       configuration.}
\dispi{imsi  1024 \CR}{to make a largish image to see sources around
       3C286.}
\dispi{outn '3C286\_quick' \CR}{ }
\dispi{niter 100 ; dotv 1 \CR}{to do some Cleaning and guide the
       progress with the TV}
\dispi{go \CR}{ }

\item\ At this point, it may be simpler to {\tt \tndx{SPLIT}} the data
  set and apply calibration to the target sources which we then can
  work with one at a time.\iodx{EVLA}
\dispi{defsult split \CR}{ }
\dispi{indi $d$; getn 6 \CR}{ }
\dispi{outdi indi \CR}{ }
\dispi{docal 1 ; doband 1 \CR}{ }
\dispi{sources = '${\rm target}_1$', '${\rm target}_2$', $\ldots$
      \CR}{to select only target sources.}
\dispi{go \CR}{ }
\dispe{To create a number of separate, single-source $uv$ data sets
  each with separate target source.}

\item\ Flag RFI and bad data out of the target data set with {\tt
  \tndx{SPFLG}}, which can be tedious.
\dispi{default spflg \CR}{ }
\dispi{indi $d$; getn 9 \CR}{to select the first of the split target
       data sets.}
\dispi{dparm(6) $x$ \CR}{to set integration time.}
\dispi{go \CR}{ }
\dispe{Then do two or more rounds of {\tt \tndx{RFLAG}}.}
\dispi{default rflag \CR}{ }
\dispi{indi $d$; getn 9 \CR}{ }
\dispi{avgchan 11 \CR}{Median window width on spectra.}
\dispi{fparm = 3, $x$, -1, -1 \CR}{to do a small rolling time buffer,
       set the sample interval, and to have {\tt NOISE} and {\tt
         SCUTOFF} set IF-dependent cutoff levels.}
\dispi{doplot 8 ; dotv 1 \CR}{To examine the two important plots on
       the TV.}
\dispi{go \CR}{ }
\dispe{and then apply the flags levels and determine new ones}
\dispi{doplot -8  ; go \CR}{ }
\dispe{and perhaps repeat this line for more flagging.}

\item\ Because we used {\tt TYAPL} only to apply the re-quantizer
  gains, your data weights simply reflect integration time.  For
  better continuum imaging, weights should reflect the actual data
  uncertainties.  So re-weight your data with {\tt \tndx{REWAY}}
\dispi{default reway \CR}{ }
\dispi{indi $d$; getn 9 \CR}{ }
\dispi{outdi indi \CR}{ }
\dispi{aparm 31,0,500 \CR}{to find rmses from rolling buffers 31 time
       intervals long on a baseline basis and then smooth them over
       500 seconds.}
\dispi{go \CR}{ }
\dispe{Note that this applies the final {\tt SPFLG}-{\tt RFLAG} flag
       table.  Delete the other flag table versions with}
\dispi{default extdest \CR}{}
\dispi{indi $d$; getn 10 \CR}{}
\dispi{inext 'fg' ; invers = -1 \CR}{to delete all remaining versions
       of flag table.}
\dispi{extdest \CR}{}

\sects{P-band imaging in \AIPS}

\item\ The following steps relate to imaging your calibrated data. As
  the aim of this appendix is calibration, the imaging will be
  discussed only briefly here.  Make a dirty image of the center with
  {\tt \tndx{IMAGR}}\iodx{EVLA}
\dispi{default imagr \CR}{ }
\dispi{indi $d$; getn 10; outdi indi \CR}{ }
\dispi{bchan 5; echan = $n-4$ \CR}{to omit edge channels in each IF,
       where $n$ is the number of channels in an IF.}
\dispi{nchav = echan-bchan+1; chinc nchav \CR}{to average all spectral
       channels into the image.}
\dispi{cellsize 10 \CR}{to set a cell size which will depend on the
       VLA configuration.}
\dispi{imsi 1024,1024 \CR}{ }
\dispi{niter 0 \CR}{to do no Cleaning}
\dispi{go \CR}{ }

\item\ Clean image with faceting, auto-boxing, and TV display; we use
{\tt DEFAULT IMAGR} here to initialize all of {\tt IMAGR}'s many
parameters  \label{it:imagr}\todx{SETFC}
\dispi{default imagr \CR}{ }
\dispi{indi $d$; getn 10; outdi indi \CR}{ }
\dispi{task 'setfc' \CR}{ }
\dispi{cellsize  0 ; imsize  0 \CR}{to have {\tt SETFC} determine the
       correct cell and image size for these data.}
\dispi{bparm  2.2, 5, 2, 8, 0.3 \CR}{to image fully out to a radius of
       2 degrees and to include 0.3 Jy NVSS sources out to a radius
       of 8 degrees.}
\dispi{flux 1 \CR}{to include exterior sources above 1 Jy only.}
\dispi{boxfile 'MYAREA:Target1.box \CR}{to list the facets found.}
\dispi{go \CR}{ }
\dispe{The following does not use default since we need adverbs set
  for, and by, {\tt SETFC}, such as inname, imsize, and cellsize
  \todx{IMAGR}}
\dispi{task 'IMAGR' \CR}{ }
\dispi{flux  0 \CR}{ }
\dispi{bchan 5; echan = $n-4$ \CR}{to omit edge channels in each IF,
       where $n$ is the number of channels in an IF.}
\dispi{nchav = echan-bchan+1; chinc nchav \CR}{to average all spectral
       channels into the image.}
\dispi{niter 5000 \CR}{to Clean 5000 iterations.}
\dispi{do3dim true ; overlap 2 \CR}{to use true 3-D geometry, Cleaning
       one facet at a time in each cycle (recommended).}
\dispi{dotv true \CR}{to control the imaging with the TV.}
\dispi{oboxfile boxfile \CR}{to save all changes to Clean boxes.}
\dispi{im2parm=2,0 \CR}{to allow up to 2 Clean boxes to be created
       automatically at each cycle.}
\dispi{go \CR}{ }
\dispe{Note, the field may have enough spectral index variation to
  require imaging (and self-calibration) of each spectral window (IF)
  individually.}

\item\ There is normally enough flux in a P-band field to do
  phase self-calibration and we recommend that you now do that.  Use
  {\tt CALIB} with the output of step~\ref{it:imagr} as the model
  including all {\tt NFIELD} facets as the {\tt NMAPS} images.
  \label{it:selfcal}\todx{CALIB}\iodx{EVLA}
\dispi{default calib \CR}{}
\dispi{indi $d$; getn 10; outdi indi \CR}{ }
\dispi{nmaps = nfield \CR}{to include all facets}
\dispi{in2d = indisk; get2n $M$ \CR}{to set the $2^{\rm nd}$ file name
       to the output image of class {\tt ICL001} (catalog number
       $M$).}
\dispi{soltyp 'l1r' ; solmode 'p' \CR}{to start with phase-only
       solutions.}
\dispi{solint = $n x$ \CR}{to set a short integration time as $n$
       times the sampling interval.}
\dispi{go \CR}{}

\item\ Examine the output {\tt SN} table with {\tt EDITA} to see if
  there are times with bad solutions.  If not, repeat steps
  \ref{it:imagr} and \ref{it:selfcal} using
\dispi{tget imagr; docal = 1; go \CR}{to make new images.}
\dispi{tget calib; in2seq = 2 \CR}{to point at the second set of
       images.}
\dispi{solmode 'a\&p' \CR}{to do amplitude as well {\it if there is
       enough flux in the field}.}
\dispi{go \CR}{}
\dispe{Re-examine the new {\tt SN} table with {\tt EDITA}.  In
  particular, watch for periods of time when the amplitude gain of an
  antenna gets rather larger than normal.  It has been found that the
  ionosphere can defocus sources on scales as small as the antenna
  diameter.  It is best to delete such data.}

\item\ Make image(s) and final calibrated data sets into FITS files
  with {\tt \tndx{FITTP}}\@.  Exit {\tt AIPS} and
\dispi{setenv FIT /lustre/mmao/Pband \CR}{for tcsh and c shells}
\dispi{export FIT=/lustre/mmao/Pband \CR}{for bash shell}
\dispe{Then start {\tt AIPS} again}
\dispi{default fittp \CR}{ }
\dispi{indi $d$; getn $M$ \CR}{to select an image or data set}
\dispi{dataout 'FIT:VirA.FITS \CR}{to make an appropriate output name}
\dispi{go \CR}{ }
\dispe{And repeat with new $M$ and {\tt DATAOUT} values.}

\end{enumerate}
\vfill\eject
\sects{Additional Recipes}


% chapter  I  *************************************************
\recipe{Banana mandarin cheese pie}

\bre
\Item {In large mixer bowl, beat 8 ounces softened {\bf cream cheese}
     until fluffy.}
\Item {Gradually beat in 8 ounces {\bf sweetened condensed milk} until
     smooth.}
\Item {Stir in 1 teaspoon {\bf lemon juice} and 1 teaspoon {\bf
     vanilla extract}.}
\Item {Slice 2 medium {\bf bananas}, dip in lemon juice, and drain.}
\Item {Line 8(?)-inch {\bf graham cracker pie crust} with bananas and
     about 2/3 of an 11-ounce can (drained) {\bf mandarin oranges}.}
\Item {Pour filling over fruit and chill for 3 hours or until set.}
\Item {Garnish top with remaining orange segments and 1 medium {\bf
     banana} sliced and dipped in lemon juice.}
\ere


% chapter  I  *************************************************
\recipe{Banana poundcake}

\bre
\Item {Mix in large bowl until blended:}
\vskip 4pt
\hbox to \hsize{\hskip 75pt\baselineskip 14pt\vbox{\halign{\rt{{$#$}}\quad%
&\lft{#}\cr
1 {1\over3}&cups mashed {\bf bananas} (4 medium)\cr
1          &pkg.~($18{1\over2}$ oz.) {\bf yellow cake mix}\cr
1          &pkg.~($3{3\over4}$ oz.) instant {\bf vanilla pudding mix}\cr
1\over3    &cup {\bf salad oil}\cr
1\over2    &cup {\bf water}\cr
1\over2    &teaspoon {\bf cinnamon}\cr
1\over2    &teaspoon {\bf nutmeg}\cr
4          &{\bf eggs} at room temperature\cr}}\hfil}
\Item {Beat at medium speed for 4 minutes.}
\Item {Turn batter into greased and lightly floured 10-inch tube pan.}
\Item {Bake in \dgg{350} oven for 1 hour or until cake tester inserted
in cake comes out clean.}
\Item {Cool in pan 10 minutes, then turn out onto rack and cool
completely.}
\Item {If desired, dust with confectioners sugar before serving.}
\item[ ]{Thanks to the United Fresh Fruit and Vegetable Association.}
\ere

% chapter 7 *************************************************
\recipe{Frozen Push-Ups}

\bre
\Item {Peel 2 {\bf bananas} and slice into blender or
    food processor.}
\Item {Add 1 6-ounce can frozen {\bf orange juice} (thawed), 1/2
    cup instant non-fat {\bf dry milk}, 1/2 cup {\bf water}, and 1 cup
    plain low-fat {\bf yogurt}.}
\Item {Cover and blend until foamy.  Pour into small paper cups
    and freeze.}
\Item {To eat, squeeze bottom of cup.}
\item[ ]{\hfill Thanks to Ruthe Eshleman {\it The American Heart
     Association Cookbook}.}
\ere
