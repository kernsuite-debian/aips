%-----------------------------------------------------------------------
%;  Copyright (C) 2002-2007, 2010, 2013-2014, 2016
%;  Associated Universities, Inc. Washington DC, USA.
%;
%;  This program is free software; you can redistribute it and/or
%;  modify it under the terms of the GNU General Public License as
%;  published by the Free Software Foundation; either version 2 of
%;  the License, or (at your option) any later version.
%;
%;  This program is distributed in the hope that it will be useful,
%;  but WITHOUT ANY WARRANTY; without even the implied warranty of
%;  MERCHANTABILITY or FITNESS FOR A PARTICULAR PURPOSE.  See the
%;  GNU General Public License for more details.
%;
%;  You should have received a copy of the GNU General Public
%;  License along with this program; if not, write to the Free
%;  Software Foundation, Inc., 675 Massachusetts Ave, Cambridge,
%;  MA 02139, USA.
%;
%;  Correspondence concerning AIPS should be addressed as follows:
%;          Internet email: aipsmail@nrao.edu.
%;          Postal address: AIPS Project Office
%;                          National Radio Astronomy Observatory
%;                          520 Edgemont Road
%;                          Charlottesville, VA 22903-2475 USA
%-----------------------------------------------------------------------

\APPEN{High-Frequency VLA Data in \AIPS}{Hints for Reducing High-Frequency VLA Data in \AIPS}{highfreq}
\renewcommand{\Chapt}{24}

\renewcommand{\titlea}{31-December-2007 (revised 11-November-2016)}
\renewcommand{\Rheading}{\AIPS\ \cookbook:~\titlea\hfill}
\renewcommand{\Lheading}{\hfill \AIPS\ \cookbook:~\titlea}
\markboth{\Lheading}{\Rheading}

High-frequency data (22 or 43 GHz) from the VLA may be reduced
occasionally with the standard centimeter-wavelength recipe given in
this \cookbook, particularly in the smaller arrays.  However, quire
frequently,, the standard recipe will be inadequate for such data,
particularly in the larger (A and B) array configurations.
Nevertheless, VLA data taken at these high-frequencies in the largest
array configurations can be calibrated in almost all cases with only a
few minor adjustments to the centimeter wavelength recipe.
\Iodx{high-frequency VLA data}\iodx{calibration}

One reason for more complicated calibration is the high resolution,
which resolves the standard flux density calibrators, particularly
3C48. However, most of the problems are caused by the atmosphere,
where the troposphere introduces rapid phase fluctuations between the
antenna elements of the interferometer.  Both effects scale with
baseline length expressed in units of wavelength, but the latter also
heavily depends on the current weather; phases are sometimes observed
to wind on time scales of less than a minute. This causes
decorrelation during your calibrator and target source scans, and
requires you to determine phase-only calibration, before the flux
density (\ie\ gain) calibration should be attempted.

In this appendix, an approach to reducing high-frequency VLA data in
\AIPS\ is described which should help to overcome the most common
problems. It is assumed that the reader has some experience with
reducing data in \AIPS, and is familiar with the ``standard recipe''
(\eg\ \Rappen{cont}), tools to examine the data, to apply self-cal,
and if appropriate, to deal with spectral-line and polarization
calibration issues. If not, you should read the \AIPS\ \cookbook\
first (in particular \Rchap{cal}).

High-frequency calibration begins when loading the data,
requiring specific parameters in {\tt \tndx{FILLM}} to be set
(ideally one should use these {\tt FILLM} inputs for all frequencies).

Run {\tt FILLM} with:
\dispt{DOWEIGHT\qs 1 \CR}{to apply \Tsys\ weights for each individual
          {\tt IF} and polarization.}
\dispt{DOUVCOMP\qs $-$1 \CR}{to store the data without compression,
          which discards individual IF weights.}
\dispt{CPARM\qs 0 ; CPARM(8)\qs 0.05 \CR}{to use a short time interval
          in the {\tt CL} table entries (in min); 0.05min = 3s.}
\dispt{BPARM\qs 0 ; BPARM(10)\qs 0.75 \CR}{to apply opacity and gain
          curve corrections with zenith opacited weighted 75\%\ by the
          measured weather and  25\%\ by seasonal averages.}
\pd

This creates a {\tt CL} table that can be interpolated over very short
intervals, hopefully short enough to cover the atmospheric phase
fluctuations accurately. The default {\tt CL} table interval is 5
minutes, which may be fine for centimeter wavelengths, but is much too
long for proper interpolation of high-frequency phases.  Also, you have
``nominal sensitivity'' weights for individual {\tt IF}/Pol entries,
which reflect sensitivity differences between the receivers, IFs, etc.
To retain this ``nominal sensitivity'' weighting you are required to
set {\tt DOCALIB=1} (actually $0 < {\tt DOCALIB} \leq 99$ and a
non-negative value for {\tt GAINUSE}) in all the calibration tasks
during the remainder of the data calibration.

\begin{figure}
\centering
%\resizebox{\hsize}{!}{\gname{vplotP1}}
\resizebox{\hsize}{!}{\gbb{533,540}{vplotP1}}
\caption[{\tt VPLOT} of ``bad'' phases]{Uncalibrated VLA A-array 43
GHz phases to the reference antenna in the center of the array (22)
over 3 hours on a strong source (frequency switched every 6 minutes)
as a function of time plotted by {\tt \tndx{VPLOT}} look very
volatile.  Around 16:00 hours they even wrap 360 degrees within one
minute.  They cannot be calibrated using the default {\tt CL} table
interval of 5 minutes.\Iodx{high-frequency VLA
data}\iodx{calibration}}
\label{fig:badphase}
\end{figure}

\begin{figure}
\centering
%\resizebox{\hsize}{!}{\gname{vplotP2}}
\resizebox{\hsize}{!}{\gbb{536,539}{vplotP2}}
\caption[{\tt VPLOT} of ``okay'' phases]{A blow up from {\tt
\tndx{VPLOT}} of the time range 16:02 to 16:07 from
Figure~\ref{fig:badphase}.  The uncalibrated visibility phases are
seen to be well behaved, albeit on a short time scale.  They can be
calibrated if the {\tt SN} table solutions are found on these
typical short time scales and are interpolated with a {\tt CL} table
that has sufficiently short spacing between entries to allow for
interpolation of the rapid phase fluctuations.  An interval of 20
seconds would be okay here.}
\label{fig:okphase}
\end{figure}

The importance of the {\tt CL} table interval is illustrated in
Figure~\ref{fig:badphase} and Figure~\ref{fig:okphase}.  On the large
scale, the phases look beyond redemption.  But, on a relatively short
time scale, the phases are relatively well behaved and may be
calibrated easily.

After loading your data, check your {\tt CL} table entries, \eg\ {\tt
LISTR} with {\tt OPTYPE 'GAIN'}, {\tt PRTAB} with {\tt DOHMS 1}, or
{\tt SNPLT} on a short (few minutes) time range with {\tt OPTYPE
'AMP'}\@.  Make sure the entries are at the interval you expect (much
less than a minute) and that the opacity and gain curve corrections
have been applied (gains deviating from one by a few percent).
Inspect your continuum or ``channel 0'' data (gains, system
temperatures), and flag bad data. For example, you also may wish to
flag antenna 1, which is known to have bad optics at 43 GHz, and the
antennas without a 43 GHz receiver (currently in December 2002,
antennas 9 and 15, but for earlier observations you may want to check
the receiver status page ({\tt
http://www.vla.nrao.edu/astro/guides/highfreq/}),
or your observation log --- these antennas may have been left present
in your data when you first do a pointing scan in X-band).  Standard
tools for data inspection and flagging are described in \Rchap{cal} of
the \AIPS\ \cookbook\ ({\tt LISTR}, {\tt UVPRT}, {\tt VPLOT}, {\tt
SNPLT}, {\tt UVFLG}, {\tt TVFLG}, {\tt EDITR}, {\tt EDITA}, and many
more).  Make sure that at least your calibrators are ``clean.''  Run
{\tt VPLOT} on your calibrators with a reference antenna close to the
center of the array (determined by using {\tt PRTAN}) to get an
indication how rapidly your phases fluctuate; use {\tt ANTENNA} {\it
reference\_antenna} {\tt 0}; {\tt SOLINT 0}; {\tt BPARM 0 2 0} (for
phase only).  If your program source is too weak to allow
self-calibration and the phase change from one scan on your calibrator
to the next is of the order of $180^{\circ}$, you probably want to
flag the source data in between the calibrator scans.  Task {\tt
\tndx{SNFLG}} may be useful for doing this flagging.

Note that fast-switching, when used, will have changed the source
names you used in making the observe schedule file. Your sources will
have been renamed to their J2000 positions, making it difficult to
recognize the calibrator and target scans when you run {\tt LISTR}
({\tt OPTYPE 'SCAN'}).

Run {\tt \tndx{SETJY}} on your absolute flux density calibrator: 3C286
= J1331+305 = B1328+307, or 3C48 = J0137+331 = B0134+329. And maybe it
is a good idea to make a copy of your correct {\tt CL} table number
one (actually all tables) with {\tt TASAV} before continuing, so in
case of accidents, you have {\tt CL} table one with the opacity/gain
corrections applied.  ({\tt INDXR} may be used to re-create a {\tt CL}
table, including the opacity and gain corrections made by {\tt
FILLM}\@.\Iodx{high-frequency VLA data}\iodx{calibration}

Run {\tt \tndx{VLANT}} to correct phases for improved estimates of the
antenna positions.  Note that this task requires your computer to be
connected to the Internet if your data are recent (within past 18
months or so).  Otherwise, apply baseline corrections following advice
at {\tt www.vla.nrao.edu/astro/archive/baselines/}.

Run {\tt \tndx{CALIB}}, at this stage to correct for phase only, with
a small solution interval (depending on your signal to noise, \eg\ 20
seconds) on all your calibrator sources. You should use the
Clean-components models for 3C286 or 3C48 provided with \AIPS\@.  See
\Sec{vlacalmodels} for information on {\tt CALDIR} and {\tt CALRD}\@.
Then run {\tt CALIB} on these sources separately using the appropriate
model.  There are also models for 3C147 and 3C138\@.

Inputs to the first pass of {\tt \tndx{CALIB}}:
\dispt{CALSOUR '{\it cal1\/}', '{\it cal2\/}', $\ldots$ \CR}{to define
         your calibrators; {\bf all but those for which you plan to
         use a model, \eg\ 3C48}.}
\dispt{DOCALIB 1 \CR}{to apply nominal sensitivities, essential that
         $0 < {\tt DOCALIB} \leq 99$.}
\dispt{GAINUSE 0 \CR}{apply latest {\tt CL} table (is version 2 after
         {\tt VLANT}\@.}
\dispt{REFANT {\it reference\_antenna} \CR}{to pick a well behaved
         antenna in the array center.}
\dispt{SOLINT 20/60 \CR}{to solve every 20 seconds; may have to try
         some values.}
\dispt{SOLMODE 'P' \CR}{to do phase calibration only at this stage.}
\dispt{SNVER 1 \CR}{to collect all solutions in {\tt SN} table one.}
\pd

And, if you have 3C48, 3C138, 3C147, and/or 3C286 as absolute flux
density calibrator(s), you should re-run {\tt CALIB}, one calibrator
source at a time, with the previous/above values plus:
\dispt{CALSOUR '{\it ssss\/}', ' ' \CR}{to specify the name you have
         used for the calibrator source.}
\dispt{IN2DISK\qs {\it d2} \CR}{to specify the disk with the source
         model.}
\dispt{GET2NAME\qs {\it ctn2\/} \CR}{to specify the {\tt CC} model to
         be used by its catalog number.}
\dispt{INVERS\qs 0 \CR}{to use the model's latest {\tt CC}-version
         (\ie\ one).}
\dispt{NCOMP\qs 0 \CR}{to use all the {\tt CC}-components of the
         model.}
\pd

This may work, but there is no guarantee. Some tricks to apply, in no
particular order, in your data set or {\tt CALIB} to obtain a larger
relative portion of good versus bad solutions would
be:\iodx{calibration}\Iodx{high-frequency VLA data}
\begin{itemize}
\item\ Flag some more bad data points on your calibrator sources.
\item\ Discard antennas with uncertain baseline positions (see
       observing log file).
\item\ Choose a different reference antenna (the one you have might be
       misbehaving).
\item\ Decrease the {\tt UVRANGE} to weight short baselines (centrally
       located antennas) more in the solution.
\item\ Use {\tt SOLTYPE 'L1'} to be less sensitive to outlying points.
\item\ Use {\tt FRING} instead of {\tt CALIB} with a larger {\tt
       SOLINT} to solve for the phase rates, switching off the delay
       search with {\tt DPARM(2) = -1}.
\item\ Increase or decrease {\tt SOLINT}; increase for weak, decrease
       for strong sources.
\item\ Decrease the SNR cutoff {\tt APARM(7)} (default 5) to include
       more noisy but possibly valid solutions.
\item\ Decrease the number of antennas required for a solution ({\tt
       APARM(1)}, default 6) to require fewer antennas
\item\ Recreate 'CH 0' from 'LINE' to get up to 25\% more bandwidth on
       calibrators.
\end{itemize}

Note that at 43 GHz in A-array the unprojected \uv-distance between
the outer two antennas on one arm is 0.5 Mega-wavelengths, and the
outer 6 antennas --- the default for {\tt APARM(1)} --- require good
solutions out to 2 Mega-wavelengths for {\tt CALIB} to accept the
solution for your outermost antenna. Hence, it is a good idea to set
{\tt APARM(1)} to \eg\ four (or three, if you're willing to check
the output {\tt SN} table carefully).

Check the resulting {\tt SN} table number one with {\tt LISTR} ({\tt
OPTYPE 'GAIN'}, {\tt DPARM 1 0}) or {\tt SNPLT} ({\tt INEXT 'SN'},
{\tt OPTYPE 'PHAS'}), and judge whether you have enough solutions and
whether you believe the phases shown are likely to reflect the
variation caused by the troposphere. If not, fiddle around with your
data and/or parameters in {\tt CALIB} as suggested above and try
again. In case the majority of solutions are fine, you may want to
edit spurious points in your {\tt SN} table with \eg\ {\tt SNEDT},
{\tt SNCOR}, or {\tt SNSMO}\@.
\Iodx{high-frequency VLA data}\iodx{calibration}

Once you are satisfied with the phases in your {\tt SN} table, you
want to apply phase corrections to minimize decorrelation in your
calibrator scans before you determine the absolute flux density
scale. To insert the corrections, run {\tt \tndx{CLCAL}} with:
\dispt{SOURCES ' ' \CR}{to correct phases for all sources.}
\dispt{CALSOUR '{\it cal1\/}', '{\it cal2\/}', $\ldots$ \CR}{to
          include {\bf all} your calibrators.}
\dispt{INTERPOL '2PT' \CR}{to interpolate between solutions ({\tt
          'SIMP'} will average phases over a scan).}
\dispt{SNVER 1; GAINVER 1;GAINUSE 2 \CR}{to apply {\tt SN}\#1 to {\tt
          CL}\#1, creating {\tt CL}\#2.}
\dispt{REFANT {\it reference\_antenna} \CR}{to select the same antenna
          as used in {\tt CALIB}.}
\pd

In less straightforward observations you may not be able to run {\tt
CLCAL} only once, \eg\ when you are switching frequencies.  If in
doubt, consult \Rchap{cal} of the \AIPS\ \cookbook. It is however very
simple to run {\tt CLCAL} multiple times. Inspect your new {\tt CL}
table two (three after {\tt VLANT}) for unexpected dubious
interpolations and extrapolations ({\tt LISTR} with {\tt OPTYPE
'GAIN'}, {\tt DPARM 1 0}, or {\tt SNPLT} with {\tt INEXT 'CL'}, {\tt
OPTYPE 'PHAS'}) and backtrack possible problems.

Now re-run {\tt \tndx{CALIB}} with the corrected phases to obtain the
flux density scale.  Begin with those sources not requiring models:
\dispt{CALSOUR '{\it cal1\/}', '{\it cal2\/}', $\ldots$ \CR}{to
          identify your calibrators; all, except 3C48 and 3C286.}
\dispt{DOCALIB 1 \CR}{to apply antenna gain/opacity and antenna
          location corrections to the data and theirweights.}
\dispt{GAINUSE 0 \CR}{to apply latest {\tt CL} table (is 3 here).}
\dispt{REFANT {\it reference\_antenna} \CR}{to pick the same antenna
          as used before.}
\dispt{SOLINT\qs 0 \CR}{to average over the full scan; remember that
          phase variations are corrected by {\tt DOCALIB}.}
\dispt{SOLMODE 'A\&P' \CR}{to do full calibration to get flux
          densities and residual phases.}
\dispt{SNVER 2 \CR}{to collect solutions in a new {\tt SN} table
          (two).}
\pd

Then run {\tt CALIB} again for the absolute flux density calibrator
3C286, or 3C48, using a model and the previous/above values plus:
\dispt{CALSOUR '{\it sssss\/}', ' ' \CR}{to specify the name you have
         for the source.}
\dispt{UVRANGE 0 \CR}{to use the full uv range without restrictions.}
\dispt{IN2DISK\qs {\it d2} \CR}{to specify the disk with the source
         model.}
\dispt{GET2NAME\qs {\it ctn2\/} \CR}{to specify the model to be used
         by its catalog number.}
\dispt{INVERS\qs 0 \CR}{to use the model's latest {\tt CC}-version
         (\ie\ one).}
\dispt{NCOMP\qs 0 \CR}{to use all the {\tt CC}-components of the
         model.\todx{CALIB}}
\pd

The same tricks may be used as for phase-only to improve the ratio of
good to bad solutions.  Check your {\tt SN} table 2 thoroughly; the
phases {\bf must} be zero or very close to zero (therefore {\tt
INTERPOL '2PT'} is preferred over {\tt INTERPOL 'SIMP'} in {\tt
CLCAL}), and you want to make sure the gains of your reference antenna
do not scatter too much for individual sources.  Before {\tt GETJY}
the flux density scale is not fixed so the average gain will depend on
source.  {\tt GETJY} corrects this so that the gains for each antenna
should be similar for all sources.  If you can identify misbehaving
antennas, flag them, delete {\tt SN} table 2 (as it does not overwrite
values for which data have been deleted), and re-run {\tt CALIB} as
many times as needed to re-create {\tt SN} table two. Cautious users
will start from the beginning.\Iodx{high-frequency VLA data}
\iodx{calibration}

Run {\tt \tndx{GETJY}} to obtain the secondary calibrator flux
densities:
\dispt{SOURCES '{\it cal1\/}', '{\it cal2\/}', $\ldots$ \CR}{to
         specify the unknown sources; {\it not} the primary
         calibrator(s).}
\dispt{CALSOUR '3C48', ' ' \CR}{to specify the source name(s) you have
         used in {\tt SETJY}.}
\dispt{SNVER\qs 2 \CR}{to point to the flux density/gain solution
         table.}
\dispe{If you used a model for the flux density calibrator, your flux
scale is tied to the flux density of that calibrator in the {\tt SU}
table.  The model Clean components are scaled to match that flux.}

Carefully note the flux densities reported by {\tt GETJY} and do not
trust these values blindly.  {\tt LISTR} or {\tt SNPLT} may point out
problematic antenna solutions, requiring you to flag some more data and
start over.  If you flag data, it is best to delete {\tt SN} table
\#2.  Solutions at the times of deleted data will not be overwritten.
It is helpful to know what flux you expect for your secondary
calibrators.  See the full source list at {\tt
aips2.nrao.edu/vla/calflux.html}.  It is particularly helpful to use
one or more of the sources regularly monitored by NRAO staff; see {\tt
html://www.aoc.nrao.edu/$\sim$smyers/calibration/} for total flux as
well as polarization information.  You want to check the values,
because sometimes the flux densities deviate considerably from the
expected values and make no sense.  This could be the case if the
pointing solutions that were determined prior to your primary
calibrator scan are inappropriate for this particular primary
calibrator scan \eg\ when it is windy or when the cloud cover on your
single primary calibrator scan differs from the cloud cover on the
secondary calibrators, or, even worse, when these combine. In some
cases you may be forced to approximate the flux density scale by
entering a (recent) flux density for one of your secondary
calibrators, ignoring the primary calibrator scan and accepting an
introduced flux density uncertainty. If you decide you have to
restart, do not forget to delete your {\tt SN} and {\tt CL} tables
(except for {\tt CL} table 1 and 2 if you used {\tt VLANT}) and to
reset the flux densities of all your calibrators with {\tt SETJY}
({\tt OPTYPE 'REJY'}), before entering a {\tt ZEROSP} for a new flux
density calibrator source (also with {\tt SETJY})\@.  Re-iterate until
you are happy with the flux-density scale.

The final flux density calibration table is obtained by running {\tt
\tndx{CLCAL}} again:
\dispt{SOURCES ' ' \CR}{to calibrate flux densities for all sources.}
\dispt{CALSOUR '{\it cal1\/}', '{\it cal2\/}', $\ldots$  \CR}{to
         include calibrators to use for your targets.}
\dispt{INTERPOL '2PT', or 'SIMP' \CR}{(to specify the interpolation
           method: no real difference for per-scan solutions).}
\dispt{SNVER 2; GAINVER 2; GAINUSE 3  \CR}{to apply {\tt SN}\#2 to
           {\tt CL}\#2, creating {\tt CL}\#3.}
\dispt{REFANT {\it reference\_antenna} \CR}{to select the same
           antenna as used in {\tt CALIB}.}
\pd

From here you are almost ready to follow the usual ``standard
recipe,'' \ie\ polarization and bandpass calibration if appropriate,
and splitting into single-source data sets.  However, remember to set
{\tt DOCALIB = 1} in all these tasks as long as you are working on the
multi-source data set and haven't applied initial phase, flux density
(including polarization, bandpass) and ``nominal sensitivity''
calibration with {\tt SPLIT}. After {\tt SPLIT}, the individual
weights will have been entered in the data, properly scaled by the
latest {\tt CL} table you've made. Using your single source calibrated
data, set {\tt DOCALIB = -1} in your subsequent imaging and analysis
tasks, unless you do self-calibration.

If you anticipated checking your fast-switching calibration by
including a ``check source'' (a moderately strong source observed a
few times with the same fast-switching parameters at about the same
distance from your fast-switching source as your target source, but
not necessarily in the same direction), you can now assess a snapshot
of your calibration by imaging this source. If the fast-switching has
worked perfectly, your check source has the expected morphology,
flux density, and position.  Any position error on the check source
should indicate the accuracy of the astrometry on your target source.
If you did not include a check source, all but the astrometry and
spatial dependence of the calibration can be inferred from your
fast-switching source by imaging a scan (use a modified {\tt SN}/{\tt
CL} table by skipping the calibration on this scan) with the
calibration derived from the two neighboring
scans.\Iodx{high-frequency VLA data}\iodx{calibration}

\sects{Additional recipes}

% chapter D *************************************************
\recipe{Banana bran muffins}

\bre
\Item {Preheat oven to \dgg{400}.}
\Item {Grease 12 2.75-inch muffin cups.}
\Item {In bowl, combine 1/2 cup crushed {\bf cereal} (1.5 cups
   un-crushed Multi-Bran Chex recommended), 1.5 cups all-purpose {\bf
   flour}, 1/2 cup {\bf sugar}, 1/3 cup chopped {\bf nuts} (optional),
   2.5 teaspoons {\bf baking powder}, and 1/2 teaspoon {\bf baking
   soda}.}
\Item {In a separate bowl, combine 3 large mashed {\bf
   bananas} (1.5 cups), 1 {\bf egg} slightly beaten, 1/4 cup
   vegetable {\bf oil}, 2 tablespoons {\bf water}, and 1 teaspoon {\bf
   vanilla extract}.}
\Item {Add to cereal mixture and stir just until moistened.  Do
   not over-mix.}
\Item {Divide evenly among muffin cups.}
\Item {Bake 18--20 minutes, or until tester inserted in center
   comes out clean.}
\item[ ]{\hfill Thanks to Ralston Purina Company.}
\ere
\vfill\eject

% chapter D *************************************************
\recipe{Banana-pineapple rum bread}

\bre
\Item {Place 1/2 cup {\bf white rum} and 1/2 cup diced {\bf dried
     pineapple} in a bowl, cover, and let sit for at least one hour.}
\Item {In a mixing bowl, beat together 4 tablespoon {\bf butter} or
     margarine and 3/4 cup {\bf sugar}.  Add 1 extra large {\bf egg}
     and continue beating until light and fluffy.}
\Item {Add 2 large mashed ripe {\bf bananas} and mix well.  Beat in
     1/3 cup plain {\bf yogurt} --- curdling of the mixture is
     normal.}
\Item {In another mixing bowl, combine 2 cups {\bf all-purpose flour},
     1/2 tablespoon {\bf baking soda}, 1 teaspoon ground {\bf
     cinnamon}, 1 teaspoon ground {\bf nutmeg}, 1 teaspoon ground {\bf
     allspice}, and 1/2 teaspoon {\bf salt}.}
\Item {Add the wet ingredients and mix until well blended.  Drain the
     pineapple and add.  Fold in 1/2 cup coarsely chopped {\bf
     pecans}.}
\Item {Pour into liberally greased 9-inch loaf pan.  Bake at \dgg{350}
     for 45 to 55 minutes or until the bread passes the toothpick
     test.  Remove the pan from the oven and let it sit for 10
     minutes, before turning out on a rack to cool.}
\item[ ]{\hfill Thanks to Tim D. Culey, Baton Rouge, La. ({\tt
     tsculey@bigfoot.com}).}
\ere

% chapter D ***********************************************
\recipe{Dulce Zacateca\~no}

\bre
\Item {Peel 3 large not-too-ripe {\bf bananas} and slice
    lengthwise.  Saute in 5 tablespoons {\bf butter} until golden
    brown.  Drain on paper, place in a shallow baking dish, and
    sprinkle with a little {\bf sugar}.}
\Item {Whip 1/2 cup {\bf heavy sweet cream}.  Add 1/4 cup {\bf
    sugar}, 1/4 cup {\bf dry sherry wine}, and 1 teaspoon {\bf
    vanilla}.  Pour over bananas covering them completely.  Chill and
    serve very cold.}
\item[ ]{\hfill Thanks to Ruth Mulvey and Luisa Alvarez {\it Good
     Food from Mexico}.}
\ere

% chapter  D **********************************************
\recipe{Virginia's instant banana pie}

\bre
\Item {Mix 1 cup {\bf sour cream}, 1 cup {\bf milk}, and 1 small
     package {\bf instant vanilla pudding} until mixture thickens.}
\Item {Slice 3 medium {\bf bananas} into the bottom of a 9-inch {\bf
     graham cracker pie crust}.}
\Item {Pour the pudding over the bananas and refrigerate at least 2 hours.}
\ere

% chapter  D *************************************************
\recipe{Chocolate chip banana bread}

\bre
\Item {Blend 2 cups mashed {\bf bananas}, 1 tablespoon grated {\bf
     orange peel}, and 1/3 cup {\bf orange juice} in a bowl.  Beat in
     3 {\bf eggs}.  Stir in 1 cup packed {\bf brown sugar} and 1/3 cup
     {\bf vegetable oil}.}
\Item {Combine 2-1/2 cups {\bf all-purpose flour}, 1 cup {\bf
     chocolate chips} 2 teaspoons {\bf baking powder}, 1/2 teaspoon
     {\bf baking soda}, 1/2 teaspoon {\bf salt}, and 1/2 teaspoon {\bf
     nutmeg}.}
\Item {Stir dry ingredients into banana mixture just until blended.
     Pour into 4 greased 5-3/4 x 3-1/4-inch loaf pans.}
\Item {Bake in \dgg{350} oven for 45 to 55 minutes or until tester
     inserted comes out clean.  Let cool in pans on rack for 10
     minutes.  Remove from pan and let cool completely on rack.}
\item[ ]{\hfill Thanks to Tim D. Culey, Baton Rouge, La. ({\tt
     tsculey@bigfoot.com}).}
\ere
