%-----------------------------------------------------------------------
%;  Copyright (C) 1995, 1998, 2000-2001, 2004, 2007, 2009-2010,
%;  Copyright (C) 2013-2014
%;  Associated Universities, Inc. Washington DC, USA.
%;
%;  This program is free software; you can redistribute it and/or
%;  modify it under the terms of the GNU General Public License as
%;  published by the Free Software Foundation; either version 2 of
%;  the License, or (at your option) any later version.
%;
%;  This program is distributed in the hope that it will be useful,
%;  but WITHOUT ANY WARRANTY; without even the implied warranty of
%;  MERCHANTABILITY or FITNESS FOR A PARTICULAR PURPOSE.  See the
%;  GNU General Public License for more details.
%;
%;  You should have received a copy of the GNU General Public
%;  License along with this program; if not, write to the Free
%;  Software Foundation, Inc., 675 Massachusetts Ave, Cambridge,
%;  MA 02139, USA.
%;
%;  Correspondence concerning AIPS should be addressed as follows:
%;          Internet email: aipsmail@nrao.edu.
%;          Postal address: AIPS Project Office
%;                          National Radio Astronomy Observatory
%;                          520 Edgemont Road
%;                          Charlottesville, VA 22903-2475 USA
%-----------------------------------------------------------------------
% Summary of standard aips calibration process.
% last edited by Eric Greisen for CookBook inclusion

\appen{Easy Continuum UV-Data Calibration and Imaging}{cont}
\renewcommand{\Chapt}{21}

\renewcommand{\titlea}{31-December-2009 (revised 26-November-2009)}
\renewcommand{\Rheading}{\AIPS\ \cookbook:~\titlea\hfill}
\renewcommand{\Lheading}{\hfill \AIPS\ \cookbook:~\titlea}
\markboth{\Lheading}{\Rheading}

This appendix contains a step-by-step guide to calibrating simple
continuum data from interferometers using \AIPS\@.  This older guide
is now preceded with information about a new ``pipeline'' of \AIPS\
procedures which can {\em automagically} calibrate, image, and even
self-calibrate many multi-source continuum (and line) data sets.
Although this pipeline is unlikely to produce immediately publishable
results in most cases, it does produce a data set with preliminary
calibration and editing which can then be edited further ``by hand.''
Following the additional editing, the pipeline may be re-run to
produce improved images.  When good single-source data sets are
produced, self-calibration can also be performed using the methods
described in \Sec{selfcal}.

\Sects{VLARUN}{vlarun}

The user should read \AIPS\ Memo 112, ``Capabilities of the VLA
pipeline in AIPS,'' by Lorant O. Sjouwerman dated March 19, 2007.
This memo goes into details and advice that are beyond the scope of
this appendix.  The memo may be obtained from the \AIPS\ web set
{\tt http://www.aips.nrao.edu/}.

To use the pipeline, you must first load the VLA Archive or other
multi-source visibility data into \AIPS; see \Sec{fillm}.  To acquire
the pipeline, including all procedures and special adverbs, enter
\dispt{RUN\qs \Tndx{VLARUN} \CR}{to define the procedures and adverbs}
\dispe{Do this only once.  The procedures will be remembered in your
{\tt LASTEXIT} {\tt SAVE}/{\tt GET} file.}

The pipeline has three stages: (1) calibration and editing, (2) basic
imaging, and (3) self-calibration.  Study the recommendations and
other advice in {\tt EXPLAIN VLARUN \CR}\@.  Note that {\tt VLARUN}
can also handle spectral-line data.  Then, the inputs for the
calibration stage include
\dispt{TASK\qs 'VLARUN' ; INP \CR}{to review the inputs needed.}
\dispt{WORKDISK\qs $n$ \CR}{to specify the disk containing the $uv$
       data.}
\dispt{CATNUM\qs $catn$ \CR}{to specify the catalog number of the $uv$
       data; {\tt INNAME} et al.~may also be used.}
\dispt{FASTSW\qs 1 \CR}{to correct source name peculiarities induced
       by fast-switching observations.}
\dispt{AUTOFLAG\qs 2 \CR}{to use {\tt FLAGR} on all data and {\tt
       QUACK} on all but high frequency data sets}
\dispt{PHAINT\qs 1 ; AMPINT\qs 5 \CR}{to set the phase calibration
       interval to 1 minute and the amplitude calibration interval to
       5 minutes.}
\dispt{DOMODEL\qs 1 \CR}{to use standard flux calibrator models where
       available.}
\dispt{NOPAUSE\qs -1 \CR}{to have the pipeline pause after {\tt GETJY}
       to allow you to evaluate whether it is okay to proceed.}
\dispt{AUTOPLOT\qs -1 \CR}{to make no diagnostic plot files.}
\dispt{DOIMAGES\qs -1 \CR}{to do calibration and editing only.}
\dispt{INP ; TPUT \CR}{to double check the inputs and save them.}
\dispt{VLARUN \CR}{to calibrate and edit your data.}
\pd

The pipeline can do imaging and even self-calibration, although the
latter is not for the faint at heart.  To include this in your pipeline
run, enter
\dispt{DOIMAGES\qs 1 \CR}{to request imaging.}
\dispt{ARRYSIZE\qs 0 \CR}{to let the procedure find the array
       dimensions.}
\dispt{IMSIZE\qs -1 \CR}{to image the target source over the full
       primary beam, with smaller areas for calibrators.}
\dispt{NITER\qs $nit$ ; CUTOFF\qs $f$ \CR}{to set the number of
       iterations $nit$ and Clean flux limit $f$.  {\tt NITER} must be
       $> 0$ for the image to be Cleaned.  Use conservative values to
       begin with and higher limits in subsequent runs.}
\dispt{ALLIMG\qs 1 \CR}{to make images of calibrators as well as
       target sources.}
\dispt{SLFCAL\qs0 \CR}{to avoid self-calibration initially.}
\dispt{INP ; TPUT \CR}{to double check the inputs and save them.}
\dispt{VLARUN \CR}{to calibrate, edit, and image your data.}
\pd

In the first run of {\tt \tndx{VLARUN}}, it is recommended to avoid
self-calibration although the making of images will help you evaluate
the quality of the auto-editing.  If your data are not ``perfect''
(\ie\ good enough), then consider one or more of the interactive
editing tasks such as {\tt TVFLG} (\Sec{tvflg}), {\tt SPFLG}
(\Sec{spflg}), {\tt EDITA} (\Sec{edita}), and {\tt EDITR}
(\Sec{editr}) along with the advice given on editing in
\Sec{initedit}, \Sec{caledit}, \Sec{uvflg}, and \Sec{editing}.  Then
\dispt{TGET\qs VLARUN \CR}{to restore the inputs.}
\dispt{AUTOFLAG\qs -1 \CR}{to do no more automatic flagging.}
\dispe{Consider raising {\tt NITER} and/or lowering {\tt CUTOFF}
       in this second pass, plus}
\dispt{SLFCAL\qs $snit$ \CR}{to also do self-calibration with
       interactivity on the TV ($snit > 0$) or in a batch-like process
       ($snit < 0$) for $|snit|$ self-cal iterations.}
\dispt{INP ; TPUT \CR}{to double check the inputs and save them.}
\dispt{VLARUN \CR}{to calibrate, edit, and image your data.}
\dispe{Repeat as needed.  Note that {\tt VLARUN} has been used on many
of the data sets in the VLA archive, with the results archived at
{\tt http://archive.nrao.edu/}.  These results include both images and
calibrated $uv$ data sets.}


%\vfill\eject
%-----------------------------------------------------------------------
%      Langston memo re calibration
%-----------------------------------------------------------------------
\sects{Basic calibration}

\large

\newcommand{\IF}{{\normalsize \sc IF}~}
\newcommand{\SN}{{\normalsize \sc SN}~}
\newcommand{\SU}{{\normalsize \sc SU}~}
\newcommand{\TCTES}{{\normalsize \sc 3C48}}
\newcommand{\TCPCL}{{\normalsize \sc 3C286}}
\newcommand{\TCOTE}{{\normalsize \sc 3C138}}
\newcommand{\freqid}{{\normalsize \sc FREQID}}

\centerline{{\it From VLA Archive Tape to a UV FITS Tape}}
\normalsize
\centerline{{\it \AIPS\ Memo No.~76 Updated}}
\centerline{Glen Langston}

\normalstyle

The Gentle User enters the Computer room with a VLA archive tape
containing a scientific breakthrough.  The user's sources are named
\APEIN{source1} and \APEIN{source2}.  The interferometer phase is
calibrated by observations of \APEIN{cal1} and \APEIN{cal2}.  The flux
density scale is calibrated by observing \TCTES\ (=0137+331) and
polarization is calibrated with observations of \TCPCL\ and/or \TCOTE.
Mount the tape on drive number {\it n}, log in and start \AIPS. Example
input: \APEIN{AIPS NEW}.  Mount the tape:
\APEIN{INTAPE={\it n}; DENS=6250; MOUNT}.
\beddes
\myitem{\tndx{PRTTP}} Find out what is on the tape, get project number
   and bands.~
\APEIN{TASK='PRTTP'; PRTLEV=-2; NFILES=0; INP; GO; WAIT; REWIND}.
\myitem{\tndx{FILLM}} Load your data from tape.
Select only one band at a time to process.
\APEIN{TASK='FILLM'; VLAOBS='?'; BAND='\ '; NFILES=?; DOWEIGHT 1; INP;
   GO}
(Replace all \APEIN{?}'s with appropriate values.) \APEIN{FILLM} will
load your visibilities (\uvdata) with weights suitable for calibration
into a large file for each band.  It also creates 6 \AIPS\ tables
each; these tables have two letter names which are:
\iodx{extension files}
\tablestyle
\beddes
\mybitem{HI} Human readable history of things done to your data.
Use PRTHI to read it.
\mybitem{AN} Antenna location and polarization tables.  Antenna
polarization \indx{calibration} is placed here.
\mybitem{NX} Index into visibility file based source name and
observation time.  Not modified by calibration.
\mybitem{SU} Source table contains the list of sources observed
and indexes into the frequency table.  The flux densities of the
calibration sources are entered into this table.
\mybitem{FQ} Frequencies of observation and bandwidth with index
into visibility data. Not modified.
\mybitem{CL} Calibration table describing the antenna based gains.
Version 1 should never be modified.
The CL table contains entries at regular time intervals (\ie\ 2
minutes) for each antenna.
{\bf The ultimate goal of calibration is to create a good \CL version 2.}
Use PRTAB to read tables.
\eeddes
\normalstyle
\myitem{\tndx{PRTAN}} Print out the antenna locations.
\APEIN{TASK='PRTAN'; PRTLEV=0; INP; GO}.
Choose a good {\it Reference} antenna (called {\it R}) near the center
of the array (\APEIN{REFANT=R}).  Check the VLA operator log to make
sure the antenna was OK during the entire observation.
\myitem{\tndx{QUACK}} Flag the bad points at the beginning of each
scan, even the ones with good amplitudes could have bad phases.
Creates a Flag Table (\APEIN{FG}).  You want to use \APEIN{FG} table
version 1 for all tasks.
\APEIN{TASK='QUACK'; FLAGV=1; OPCOD='~'; APARM=0; SOUR='~'; INP; GO}
deletes the first six seconds of each scan, which may not be enough.
\tablestyle
\beddes
\mybitem{FG} A flag table marks bad data. FG tables contain an index
into the UV data based on time range, antenna number, frequency and
\IF number.
\eeddes
\normalstyle
\myitem{\tndx{LISTR}} Lists your UV data in a variety of ways.  Make a list
of your observations.
\APEIN{TASK='LISTR'; OPTYP='SCAN'; DOCRT=-1; SOUR='~'; CALC='*';}
\APEIN{TIMER=0; INP; GO}.
NOTE: IF you have observed in a way as to create more than one
\freqid,  you must run through the entire calibration once for EACH
\freqid\@.  For new users, it is better to use \APEIN{UVCOP} to copy
each \freqid\ into separate files and calibrate each file separately.
This is required if you are doing polarization calibration.
\myitem{\tndx{UVCOP}} Skip this step if your data consists of only one
\freqid\@. Copy different \freqid s into separate files.
\APEIN{TASK='UVCOP'; FREQID=?; CLRON; OUTDI=INDI; INP; GO}.
The result will be a \APEIN{??.UVCOP}~ file.
\myitem{\tndx{SETJY}} Sets the flux of your flux calibration source in
the \SU table.
\APEIN{TASK='SETJY'; SOUR='3C48','~'; OPTYP='CALC'; FREQID=1; INP; GO}.
Adjust flux density for partial resolution following the rules in the
VLA Calibration Source Manual or the \AIPS\ \COOKBOOK.
\myitem{\tndx{TASAV}} As insurance, make a copy of all your tables.
\APEIN{TASK='TASAV'; CLRON; OUTDI=INDI; INP; GO}.
%------------------------------------------------------------------
\begin{figure}[t]
\centering
%\resizebox{!}{3.3in}{\gname{uvpltA1}\hspace{0.1cm}\gname{uvpltA2}}
\resizebox{!}{3.3in}{\gbb{537,565}{uvpltA1}\hspace{0.1cm}\gbb{535,565}{uvpltA2}}
\caption[Visibility data before and after calibration]{{\it (left)}
Un-calibrated {\it uv}-data and  {\it (right)} calibrated {\it
uv}-data from a C-band snapshot of \TCTES.  Default VLA gains are a
tenth of the actual gains and can show significant scatter.  Only wild
{\it uv} points $\sim$50\% greater than the average can be detected
before calibration.}
\end{figure}
%------------------------------------------------------------------
\myitem{\tndx{CALIB}} \APEIN{CALIB} is the heart of the \AIPS\
   calibration package.  \APEIN{RUN \tndx{VLAPROCS}}, an \AIPS\ {\it
   runfile}, to create procedures \APEIN{\tndx{VLACALIB},
   \tndx{VLACLCAL}} and \APEIN{VLARESET}.  The procedure
   \APEIN{VLACALIB} runs \APEIN{CALIB}.  Set \APEIN{UVRANGE} and
   \APEIN{ANTENNA} to zero to allow use of models for \TCTES.  For L,
   C and X band 5\% and 5 degree errors are OK; for other bands the
   limits are higher.  \APEIN{CALIB} places  antenna amplitude and
   phase corrections into an \SN table for the time of observation of
   phase calibration sources.
\tablestyle
\beddes
\mybitem{SN} Solution table contains antenna based amplitude and phase
   corrections for the time of observations of the calibration sources.
   These \SN table results are latter interpolated for all times of
   observation and placed in a \CL table.  Only the \CL table
   corrections will be applied to the program sources.
\eeddes
\normalstyle
   \APEIN{TASK='VLACAL';}~ \APEIN{CALS='3C48','~';}~
   \APEIN{CALCODE='*';}~ \APEIN{REFANT={\it R};}~
   \APEIN{UVRA=0; SNVER=1; DOCALIB=-1; DOPRINT=1;}~
   \APEIN{MINAMP=10; MINPH=10; INP; VLACAL}.
   \APEIN{VLACALIB} will load and use a source model for \APEIN{3C48},
   making \APEIN{UVRANGE} unnecessary.  The task \APEIN{CALIB} lists
   antenna pairs which deviate significantly from the solution.  If
   you have lots of errors, then carefully examine your data using
   \APEIN{\tndx{TVFLG}} or \APEIN{LISTR}. (See \AIPS\ \COOKBOOK\
   \iodx{COOKBOOK} for a lengthy discussion on flagging.)~  If one
   antenna is bad over a limited time range, use \APEIN{UVFLG} to flag
   that antenna for the time from just after the previous good
   \APEIN{cal} observation to before the next good \APEIN{cal}
   observation.
\myitem{\tndx{UVFLG}} Flag bad UV-data.
   \APEIN{TASK='UVFLG'; ANTEN=?,0; BASELI=?,0; TIMER=?; OUTFGVER=1;
   SOUR='~'; OPCOD='~'; INP; GO}.~
   If in doubt about any data, \APEIN{FLAG THEM!}  If you have flagged
   the primary calibrator, return to \APEIN{CALIB} above and try
   again.
\myitem{\tndx{CALIB}} Now calibrate the antenna gain based on the rest
   of the cal sources.  Look in the Calibrator manual for UV limits;
   if there are limits, \APEIN{VLACAL} must be run separately for
   these sources.  \APEIN{TGET VLACAL; CALS='cal1','cal2','~';
   ANTEN=0; BASELI=0; UVRANGE=?,?; INP; VLACAL}\@.  Flag bad antennas
   listed.  Each execution of \APEIN{CALIB} replaces previous
   corrections in the \SN table or appends new corrections.  If
   unsatisfied with a \APEIN{VLACAL} execution, all effects of it are
   removed by running \APEIN{VLACAL} again for the same sources (but
   different \APEIN{ADVERBS} or after flagging bad data).
\myitem{\tndx{GETJY}} Sets the flux of phase \indx{calibration}
    sources in the \SU table.
   \APEIN{TASK 'GETJY'; SOUR='cal1,'cal2','~';}~
   \APEIN{CALS='3C48','~'; BIF=0; EIF=0; INP; GO}.~
   \APEIN{GETJY} over-writes existing \SU table entries, and is not
   affected by previous executions.
\myitem{\tndx{TASAV}} Good time to save your tables.
   \APEIN{TGET TASAV; INP; GO}.~
\myitem{\tndx{CLCAL}} Read the antenna amplitude and phase corrections
   from the \SN table and interpolate the corrections into a new \CL
   table.  \APEIN{CLCAL} applies calibration source corrections to the
   program sources.  Each execution of \APEIN{CLCAL} adds to output
   \CL table version 2.  \APEIN{CLCAL} is run using the procedure
   \APEIN{VLACLCAL}.~  \APEIN{TASK='VLACLC';}~
   \APEIN{SOUR='source1','cal1','~';}~
   \APEIN{CALS='cal1','~'; OPCODE='CALI';}~ \APEIN{TIMER=0;}~
   \APEIN{INTERP='2PT';}~ \APEIN{INP; VLACLC}.~ Run \APEIN{CLCAL} for
   the second source using the second calibrator.  \APEIN{TGET VLACLC;
   SOUR='source2','cal2','~'; CALS='cal2','~'; INP; VLACLC}.~
   Move the \SN table corrections for \TCTES\ into the \CL table.
   \APEIN{TGET VLACLC; SOUR='3C48','~';CALS='3C48','~'; INP; VLACLC}.
   (\TCTES\ could also be calibrated with \APEIN{cal1} or \APEIN{cal2}.)
\myitem{\tndx{LISTR}} Make a matrix listing of the Amplitude and RMS
   of \indx{calibration} sources with calibration applied.  Look for
   wild points.~
\APEIN{TASK='LISTR'; OPTYP='MATX'; SOUR='cal1','cal2','~'; DOCAL=2;}~
\APEIN{DOCRT=-1; DPARM=3,1,0; UVRA=0; ANTEN=0; BASELI=0; BIF=1; INP; GO}.~
   If only a few points are bad, flag them and continue.  If too many
   are bad, delete \CL table 2 and the \SN tables using
   \APEIN{VLARESET}.  Then return to the first \APEIN{CALIB} step.  If
   the data look good, run \APEIN{LISTR} again for \IF two.
   \APEIN{TGET LISTR; BIF=2; INP; GO}
\myitem{\tndx{UVPLT}} Plot the \uvdata\ in a variety of ways.  Make a
   Flux versus Time plot first.  Choose \APEIN{XINC} so the plot will
   have no more than 1000 points.
   \APEIN{TASK='UVPLT'; SOUR='source1','~'; XINC=10; BPARM(1)=11;
   DOCAL=2; BIF=1; INP; GO}.~
   Look at the plot with \APEIN{LWPLA, TKPL, TVPL} or \APEIN{TXPL}.~~
   Plot other \IF. Flag wild points. Plot Flux versus baseline.
   \APEIN{TGET UVPLT; BPARM=0; INP; GO}.
\eeddes

Calibration is now complete for continuum, un-polarized observations.
Write the calibrated data to tape with \APEIN{FITTP} if you don't want
to calibrate the polarization.  To create images from the \uvdata\ use
\APEIN{SPLIT} to calibrate the multi-source data and create a single
source \uvdata\ set.  (\APEIN{FITTP} and \APEIN{SPLIT} are described
at the end of the polarization calibration process.)

\sects{Polarization calibration}

For \indx{polarization} observations, the following steps are
required.  For 21cm or longer wavelength observations, ionospheric
Faraday rotation corrections may be needed.  See \APEIN{FARAD} in the
\AIPS\ \COOKBOOK, but don't expect much help anymore.
\beddes
\myitem{\tndx{TASAV}} As added insurance, save your tables again.
\APEIN{TGET TASAV; INP; GO}.
\myitem{\tndx{LISTR}} Print the parallactic angles of the calibration
sources.
\APEIN{TGET LISTR; SOUR='~'; CALC='*';
OPTYP='GAIN'; DPARM=9,0; INP; GO}~
\myitem{\tndx{PCAL}} Intrinsic antenna polarization calculation.
\APEIN{PCAL} will be successful
only if cal.~sources are observed at several parallactic angles.
\APEIN{PCAL} will modify the \APEIN{AN} and \APEIN{SU} tables.
\APEIN{TASK='PCAL'; CALS='cal1','cal2','~'; BIF=1; EIF=2;
DOCAL=2; REFANT={\it R}; INP; GO}
\myitem{\tndx{RLDIF}} Now determine the absolute linear polarization angle.
Make a matrix listing of the angle of \TCPCL.
\APEIN{TASK 'RLDIF'; SOUR='3C286','~'; DOCAL=2; BIF=1; EIF=0; DOPOL=1;}~
\APEIN{GAINUSE=2; DOCRT=-1; INP; GO}.~
The observed angles are different for each frequency and \IF.  This
task returns the average angles for all IFs in \APEIN{CLCORPRM}\@.
\myitem{\tndx{CLCOR}} Now apply the angle corrections to CL table 2.  The
relative phase of Left and Right circular polarization produces the
linear polarization angle and the phase correction is applied to L.
The phase difference (twice the angle of linear polarization) for
\TCPCL\ is $66^o$ and for \TCOTE, $\phi=-18^o$ at L band, perhaps
$-24^o$ at higher frequencies.  Change \APEIN{RLDIF}'s results to this
form with \APEIN{FOR I = 1:20; CLCORP(I)=66-CLCORP(I); END}\@.  Then
\APEIN{TASK='CLCOR'; STOKES='L'; SOUR='~'; OPCOD='POLR'; BIF=1; EIF=2;
GAINVER=2; GAINUSE=0; INP; GO}.~
Run \APEIN{RLDIF} again to check the phases.~
\APEIN{TGET RLDIF; INP; GO}\@.  Note that \APEIN{CLCOR} copies the
\APEIN{CL} table version 2 to 3 while applying the phase correction.
If the phases are wrong, delete version 3, return to \APEIN{PCAL} and
\APEIN{RLDIF} and then do another \APEIN{CLCOR}.
\eeddes

\sects{Backup and imaging}

\beddes
\myitem{\tndx{FITTP}} Writes the output {\it uv}-data to tape.
\APEIN{DISMOUNT} your archive; \APEIN{MOUNT} your output tape.
\APEIN{TASK='FITTP'; DOEOT=1; OUTTAP=INTAP; INP; GO}.~  Use
\APEIN{DOEOT=-1} when at the beginning of a new tape.
\myitem{\tndx{SPLIT}} The \AIPS\ calibration process only modifies the
tables associated with the multi-source \uvdata\ set.  \APEIN{SPLIT}
selects individual sources, reads the \CL table and multiplies the
visibilities by the corrections to produce a calibrated single-source
\uvdata\ set.
\APEIN{TASK='SPLIT'; SOUR='~'; CALC='~'; UVRA=0; TIMER=0; DOCAL=2;}
\APEIN{FLAGVER=1; GAINUSE=3; DOPOL=1; DOBAN=-1; BIF=0; EIF=0; STOKES='~';}
\APEIN{BLVER=-1; APARM=0; DOUVCOM=1; ICHANSEL=0; INP; GO}
\myitem{Mapping} Use your favorite Fourier Transform task
(\eg\ \APEIN{IMAGR}) to produce images from the calibrated data.
Procedure \APEIN{MAPPR} provides a simplified interface to
\APEIN{IMAGR}\@.
\eeddes

\sects{Additional recipes}

% chapter A *************************************************
\recipe{Banana colada}

\bre
\Item  {Peel and slice 1 ripe {\bf  banana}.}
\Item {Place sliced banana in blender along with 6 ounces {\bf
     pineapple juice} (or crushed tinned pineapple in its own juice)
     and 1 ounce {\bf rum} plus 1 ounce {\bf coconut rum} {\it or\/} 2
     ounce {\bf rum} plus 1 teaspoon {\bf Coco Lopez}.}
\Item {Optionally add 1 ounce {\bf banana liqueur}.}
\Item {Blend until smooth.}
\Item {Add crushed ice, if so desired.}
\Item {If the mixture is too thick, add more juice (or more rum if
     you prefer!); if too thin, add more banana. This is a really easy
     recipe to adjust to one's taste.}
\ere

% chapter A *************************************************
\recipe{Breaded chicken and bananas}

\bre
\Item {In food processor, blend 1 can {\bf condensed milk}, 1/3 cup
     {\bf milk}, 1/2 cup flaked {\bf coconut}, and 1/4 cup {\bf lemon
     juice} until smooth. Pour into a bowl.}
\Item {Prepare 3 cups {\bf corn flake crumbs} in another bowl or plate.}
\Item {Cut  6 very firm {\bf bananas} lengthwise, dip in milk mixture,
     roll in corn flakes, and set aside.}
\Item {Cut 2 {\bf chickens} into pieces, dip in milk mixture, roll in
     corn flakes, and place in greased baking pans (2 13x9 pans may be
     required).}
\Item {Sprinkle chicken with 1/2 cup melted {\bf butter} and bake as
     \dgg{350} for one hour.}
\Item { Arrange bananas over the chicken. Sprinkle with 1/4 cup melted
     {\bf butter}. Bake 15 minutes longer or until chicken juices run
     clear.}
\Item {Garnish with sliced star and/or kiwi fruits if desired.}
\item[ ]{\hfill Thanks to Turbana Corporation ({\tt www.turbana.com}).}
\ere

% chapter A *************************************************
\recipe{Banana-pineapple bread}

\bre
\Item {Mix together 1 cup chopped {\bf nuts}, 2-1/2 cups {\bf sugar},
     5 cups {\bf flour}, 1 teaspoon {\bf salt}, 1 teaspoon {\bf baking
     powder}, and 1 teaspoon {\bf cinnamon}.}
\Item {Mix together 1-1/2 cups {\bf vegetable oil}, 3 {\bf eggs}, 3
     mashed {\bf bananas}, 1 teaspoon {\bf lemon juice}, and 1 can
     {\bf crushed pineapple} (drained).}
\Item {Combine.  Bake at \dgg{350} for one hour.}
\item[ ]{\hfill Thanks to Tim D. Culey, Baton Rouge, La. ({\tt
     tsculey@bigfoot.com}).}
\ere
